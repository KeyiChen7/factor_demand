\documentclass{article}

% Language setting
% Replace `english' with e.g. `spanish' to change the document language
\usepackage[english]{babel}

% Set page size and margins
% Replace `letterpaper' with`a4paper' for UK/EU standard size
\usepackage[letterpaper,top=2cm,bottom=2cm,left=3cm,right=3cm,marginparwidth=1.75cm]{geometry}

% Useful packages
\usepackage{amsmath}
\usepackage{graphicx}
\usepackage[colorlinks=true, allcolors=blue]{hyperref}
\usepackage{indentfirst}

\title{Final Project}
\author{Jonathan Cai, Keyi Chen, Adam Aldad, and Jean-Sébastien Gaultier}

\begin{document}
\maketitle
 
\section{Introduction}

In our final project, we replicated the first two tables of "Factor Demand and Factor Returns" by Cameron Peng and Chen Wang 
found \href{https://papers.ssrn.com/sol3/papers.cfm?abstract_id=3327849}{here}. The paper covers the persistence of factor demand 
and reveals the prevalence of factor rebalancing; We focus on the paper's discussion of factor demands. Table 1 summarizes a sample
 of US domestic equity mutual funds from 1980 to 2019, and table 2 summarizes the distribution of factor betas for mutual funds. 

\section{Replicating Table 1}

\subsection{Retrieving the Data}


We pulled our data from WRDS' Monthly Total Net Assets, Returns, and Net Asset Values table
 found \href{https://wrds-www.wharton.upenn.edu/data-dictionary/crsp_q_mutualfunds/monthly_tna_ret_nav/}{here}. The paper 
 only uses US domestic equity mutual funds in their analysis. Accordingly, we pulled data
  from WRDS' \href{https://wrds-www.wharton.upenn.edu/data-dictionary/crsp_q_mutualfunds/fund_style/}{Style attributes for each fund} table in order to 
  filter the data. Our quarterly fund holdings data was pulled from 
  the \href{https://wrds-www.wharton.upenn.edu/data-dictionary/tr_mutualfunds/s12/}{Thomson-Reuters Mutual Fund Holdings (s12)} dataset

\subsection{Cleaning the Data}

We found this part of the replication process challenging - first, finding an optimal way to filter the data
 to only "US domestic equity" took various trials and errors. However, we discovered the \textbf{crsp\_obj\_cd} column 
 of the \textbf{crsp.fund\_style} table to be the best way to achieve this filter. Furthermore,
  when using the fund-level identifier \textbf{wficn}, we discovered a handful of occurrences where 
  one \textbf{crsp\_fundno} matches with multiple \textbf{wficn}. We suspected it could have to do 
  something with delisting / merging of funds, and ultimately decided to drop these samples 
  based on the descriptions in the paper. After obtaining the appropriate \textbf{wficn} values,
   we computed the yearly returns by first computing each fund's monthly returns. At this point, we 
   ran into another issue: in our attempt replicate the paper's use of \textbf{mtna}, we noticed that
    not all \textbf{mtna} values are available. To solve this issue, we decided to use simple
     average instead because we expected different share classes of a given mutual fund to have similar
      returns. We then merged the TNA and yearly return information and got the following table: 

\begin{table}[ht]
\centering
\begin{tabular}{lrrrr}
\toprule
 & wficn & year & $crsp_{TNA}$ & yret \\
\midrule
0 & 100001 & 1990 & 169.57 & 0.03 \\
1 & 100001 & 1991 & 330.03 & 0.30 \\
2 & 100001 & 1992 & 596.27 & 0.06 \\
3 & 100001 & 1993 & 857.67 & 0.06 \\
4 & 100001 & 1994 & 876.19 & -0.01 \\
\bottomrule
\end{tabular}

\end{table}

We then followed a similar process when preparing the S12 data. As such, we ran into tangentially similar issues: missing TNA values, minor discrepancies between \textbf{mflink1} and \textbf{mflink2}, and some troubles with filtering the data to Domestic Equity. After solving these issues through various methods, we formed a table describing the S12 TNA data: 




\end{document}