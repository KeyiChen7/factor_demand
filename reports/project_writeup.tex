\documentclass{article}


% Language setting

% Replace `english' with e.g. `spanish' to change the document language

\usepackage[english]{babel}

\usepackage{caption}
% Set page size and margins

% Replace `letterpaper' with`a4paper' for UK/EU standard size

\usepackage[letterpaper,top=2cm,bottom=2cm,left=3cm,right=3cm,marginparwidth=1.75cm]{geometry}


% Useful packages

\usepackage{amsmath}

\usepackage{graphicx}

\usepackage[colorlinks=true, allcolors=blue]{hyperref}

\usepackage{indentfirst}


\title{Final Project}

\author{Jonathan Cai, Keyi Chen, Adam Aldad, and Jean-Sébastien Gaultier}

% \doublespacing
\begin{document}

\maketitle
 
\section{Introduction}


In our final project, we replicated the first two tables of "Factor Demand and Factor Returns" by Cameron Peng and Chen Wang 
found \href{https://papers.ssrn.com/sol3/papers.cfm?abstract_id=3327849}{here}. The paper covers the persistence of factor demand 
and reveals the prevalence of factor rebalancing; We focus on the paper's discussion of factor demands. Table 1 summarizes a sample
of US domestic equity mutual funds from 1980 to 2019, and table 2 summarizes the distribution of factor betas for mutual funds. 


\section{Replicating Table 1}


\subsection{Retrieving the Data}



We pulled our data from WRDS' Monthly Total Net Assets, Returns, and Net Asset Values table
found \href{https://wrds-www.wharton.upenn.edu/data-dictionary/crsp_q_mutualfunds/monthly_tna_ret_nav/}{here}. The paper 
only uses US domestic equity mutual funds in their analysis. Accordingly, we pulled data
from WRDS' \href{https://wrds-www.wharton.upenn.edu/data-dictionary/crsp_q_mutualfunds/fund_style/}{Style attributes for each fund} table in order to 
filter the data. Our quarterly fund holdings data was pulled from 
the \href{https://wrds-www.wharton.upenn.edu/data-dictionary/tr_mutualfunds/s12/}{Thomson-Reuters Mutual Fund Holdings (s12)} dataset


\subsection{Cleaning the Data}


We found this part of the replication process challenging - first, finding an optimal way to filter the data
to only "US domestic equity" took various trials and errors. However, we discovered the \textbf{crsp\_obj\_cd} column 
of the \textbf{crsp.fund\_style} table to be the best way to achieve this filter. Furthermore,
when using the fund-level identifier \textbf{wficn}, we discovered a handful of occurrences where 
one \textbf{crsp\_fundno} matches with multiple \textbf{wficn}. We suspected it could have to do 
something with delisting / merging of funds, and ultimately decided to drop these samples 
based on the descriptions in the paper. After obtaining the appropriate \textbf{wficn} values,
we computed the yearly returns by first computing each fund's
monthly returns. At this point, we 
ran into another issue: in our attempt replicate the paper's use of \textbf{mtna}, we noticed that
not all \textbf{mtna} values are available. To solve this issue, we decided to use simple
average instead because we expected different share classes of a given mutual fund to have similar
returns. We then merged the TNA and yearly return information and got the following table: 


\begin{table}[ht]

\centering
\captionsetup{labelformat=empty, font=bf}
\caption{Yearly Returns and Year End TNA}
\begin{tabular}{lrrrr}
\toprule
 & wficn & year & $crsp_{TNA}$ & yret \\
\midrule
0 & 100001 & 1990 & 169.57 & 0.03 \\
1 & 100001 & 1991 & 330.03 & 0.30 \\
2 & 100001 & 1992 & 596.27 & 0.06 \\
3 & 100001 & 1993 & 857.67 & 0.06 \\
4 & 100001 & 1994 & 876.19 & -0.01 \\
\bottomrule
\end{tabular}

\vspace{5pt} % Adds some vertical space between the table and the ellipses, adjust as needed

\begin{tabular}{c} % Creates a single centered column for the ellipses

\multicolumn{1}{c}{\ldots} \\ % Ensures ellipses are centered below the table
\end{tabular}
\end{table}


We then followed a similar process when preparing the S12 data. As such, we ran into
tangentially similar issues: missing TNA values, minor discrepancies between
\textbf{mflink1} and \textbf{mflink2}, and some troubles with filtering the data 
to Domestic Equity. After solving these issues through various methods, we formed a
table describing the S12 TNA data: 


\begin{table}

\centering
\captionsetup{labelformat=empty, font=bf}
\caption{Dometic Equity}
\begin{tabular}{lrrrr}
\toprule
 & wficn & year & assets & useq_{tna} \\
\midrule
0 & 100001.00 & 1990 & 16957.00 & 161803.10 \\
1 & 100001.00 & 1991 & 33003.00 & 314952.40 \\
2 & 100001.00 & 1992 & 59627.00 & 578201.50 \\
3 & 100001.00 & 1993 & 84286.00 & 821482.00 \\
4 & 100001.00 & 1994 & 92961.00 & 896403.43 \\
\bottomrule
\end{tabular}


\vspace{5pt} % Adds some vertical space between the table and the ellipses, adjust as needed

\begin{tabular}{c} % Creates a single centered column for the ellipses

\multicolumn{1}{c}{\ldots} \\ % Ensures ellipses are centered below the table
\end{tabular}

\end{table} 

Finally, we merge the CRSP and S12 data, and ultimately create a 
close replication of table 1 of the paper. 

% \begin{table}[ht]

% \centering

% \begin{tabular}{lrrrrrrrrr}
\toprule
 & wficn & year & crsp_tna & yret & assets & useq_tna_k & tna_ratio & eq_ratio_1 & eq_ratio_2 \\
\midrule
20690 & 102784 & 1980 & 86.77 & 0.41 & 8677.00 & 82975.90 & 1.00 & 0.96 & 0.96 \\
13438 & 101740 & 1980 & 33.51 & 0.52 & 3351.00 & 27170.69 & 1.00 & 0.81 & 0.81 \\
13333 & 101729 & 1980 & 54.17 & 0.35 & 5416.00 & 49798.88 & 1.00 & 0.92 & 0.92 \\
13318 & 101724 & 1980 & 165.50 & 0.33 & 16548.00 & 148559.09 & 1.00 & 0.90 & 0.90 \\
13246 & 101714 & 1980 & 18.68 & 0.74 & 1870.00 & 18460.85 & 1.00 & 0.99 & 0.99 \\
20827 & 102796 & 1980 & 1.91 & 0.21 & 189.00 & 1525.02 & 1.01 & 0.80 & 0.81 \\
13106 & 101666 & 1980 & 48.30 & 0.27 & 4832.00 & 42108.97 & 1.00 & 0.87 & 0.87 \\
12980 & 101618 & 1980 & 122.20 & 0.38 & 12220.00 & 111227.97 & 1.00 & 0.91 & 0.91 \\
25261 & 103464 & 1980 & 68.86 & 0.44 & 6888.00 & 65455.38 & 1.00 & 0.95 & 0.95 \\
371 & 100046 & 1980 & 37.16 & 0.20 & 3716.00 & 36858.30 & 1.00 & 0.99 & 0.99 \\
12895 & 101606 & 1980 & 251.10 & 0.77 & 25109.00 & 211107.84 & 1.00 & 0.84 & 0.84 \\
20539 & 102767 & 1980 & 647.36 & 0.44 & 64736.00 & 613970.74 & 1.00 & 0.95 & 0.95 \\
13641 & 101754 & 1980 & 22.70 & 0.28 & 2242.00 & 20492.04 & 1.01 & 0.90 & 0.91 \\
3952 & 100535 & 1980 & 62.67 & 0.07 & 6267.00 & 59123.76 & 1.00 & 0.94 & 0.94 \\
3948 & 100531 & 1980 & 29.10 & 0.36 & 2912.00 & 26864.93 & 1.00 & 0.92 & 0.92 \\
19902 & 102655 & 1980 & 571.57 & 0.52 & 57157.00 & 466698.87 & 1.00 & 0.82 & 0.82 \\
14112 & 101805 & 1980 & 613.40 & 0.61 & 61340.00 & 583906.82 & 1.00 & 0.95 & 0.95 \\
3899 & 100514 & 1980 & 22.08 & 0.33 & 2213.00 & 20560.84 & 1.00 & 0.93 & 0.93 \\
7372 & 101038 & 1980 & 193.00 & 0.20 & 19349.00 & 173293.62 & 1.00 & 0.90 & 0.90 \\
19930 & 102659 & 1980 & 970.83 & 0.58 & 97083.00 & 868504.92 & 1.00 & 0.89 & 0.89 \\
1329 & 100190 & 1980 & 712.26 & 0.73 & 71307.00 & 690607.54 & 1.00 & 0.97 & 0.97 \\
13847 & 101781 & 1980 & 68.43 & 0.46 & 6843.00 & 58756.77 & 1.00 & 0.86 & 0.86 \\
20217 & 102689 & 1980 & 5.13 & 0.36 & 513.00 & 4728.42 & 1.00 & 0.92 & 0.92 \\
13820 & 101769 & 1980 & 178.66 & 0.44 & 17866.00 & 164693.69 & 1.00 & 0.92 & 0.92 \\
20260 & 102702 & 1980 & 23.38 & 0.32 & 2338.00 & 21257.33 & 1.00 & 0.91 & 0.91 \\
3885 & 100510 & 1980 & 3.28 & 0.48 & 328.00 & 2825.71 & 1.00 & 0.86 & 0.86 \\
25477 & 103494 & 1980 & 70.05 & 0.42 & 7004.00 & 72699.56 & 1.00 & 1.04 & 1.04 \\
3378 & 100430 & 1980 & 43.09 & 0.51 & 4819.00 & 47349.36 & 0.89 & 1.10 & 0.98 \\
12862 & 101603 & 1980 & 419.25 & 0.32 & 41925.00 & 363178.75 & 1.00 & 0.87 & 0.87 \\
25054 & 103429 & 1980 & 191.77 & 0.40 & 19177.00 & 164064.10 & 1.00 & 0.86 & 0.86 \\
11694 & 101457 & 1980 & 41.77 & 0.39 & 4177.00 & 38527.53 & 1.00 & 0.92 & 0.92 \\
22197 & 102996 & 1980 & 48.72 & 0.44 & 4872.00 & 48514.79 & 1.00 & 1.00 & 1.00 \\
11626 & 101448 & 1980 & 1.10 & 0.17 & 107.00 & 896.40 & 1.03 & 0.81 & 0.84 \\
22289 & 103003 & 1980 & 6.40 & 0.13 & 641.00 & 6302.24 & 1.00 & 0.98 & 0.98 \\
24973 & 103406 & 1980 & 424.87 & 0.35 & 42487.00 & 367302.90 & 1.00 & 0.86 & 0.86 \\
22357 & 103008 & 1980 & 20.58 & 0.42 & 2051.00 & 19068.86 & 1.00 & 0.93 & 0.93 \\
22393 & 103011 & 1980 & 24.30 & 0.42 & 2433.00 & 21393.00 & 1.00 & 0.88 & 0.88 \\
1651 & 100222 & 1980 & 205.49 & 0.47 & 20549.00 & 198123.91 & 1.00 & 0.96 & 0.96 \\
22417 & 103013 & 1980 & 94.04 & 0.13 & 9405.00 & 83906.04 & 1.00 & 0.89 & 0.89 \\
2934 & 100389 & 1980 & 14.04 & 0.36 & 1403.00 & 13695.51 & 1.00 & 0.98 & 0.98 \\
21229 & 102839 & 1980 & 5.60 & 0.29 & 559.00 & 5873.67 & 1.00 & 1.05 & 1.05 \\
3281 & 100420 & 1980 & 2.17 & 0.34 & 224.00 & 2040.85 & 0.97 & 0.94 & 0.91 \\
3165 & 100412 & 1980 & 4.10 & 0.18 & 413.00 & 3456.62 & 0.99 & 0.84 & 0.84 \\
21626 & 102909 & 1980 & 53.50 & 0.31 & 4865.00 & 48796.89 & 1.10 & 0.91 & 1.00 \\
1498 & 100209 & 1980 & 25.85 & 0.29 & 2556.00 & 23088.37 & 1.01 & 0.89 & 0.90 \\
21731 & 102948 & 1980 & 13.40 & 0.27 & 1339.00 & 12001.39 & 1.00 & 0.90 & 0.90 \\
1578 & 100217 & 1980 & 25.30 & 0.18 & 2530.00 & 22949.51 & 1.00 & 0.91 & 0.91 \\
7958 & 101068 & 1980 & 53.50 & 0.70 & 5351.00 & 49083.06 & 1.00 & 0.92 & 0.92 \\
49086 & 240222 & 1980 & 171.91 & 0.49 & 17190.00 & 162186.65 & 1.00 & 0.94 & 0.94 \\
21793 & 102952 & 1980 & 62.66 & 0.47 & 6265.00 & 56863.26 & 1.00 & 0.91 & 0.91 \\
12045 & 101494 & 1980 & 40.80 & 0.58 & 4077.00 & 41027.77 & 1.00 & 1.01 & 1.01 \\
2974 & 100395 & 1980 & 227.36 & 0.28 & 30240.00 & 213159.88 & 0.75 & 0.94 & 0.70 \\
21937 & 102981 & 1980 & 48.18 & 0.72 & 4818.00 & 47929.59 & 1.00 & 0.99 & 0.99 \\
21755 & 102951 & 1980 & 161.03 & 0.34 & 16091.00 & 155667.90 & 1.00 & 0.97 & 0.97 \\
551 & 100076 & 1980 & 149.35 & 0.34 & 14935.00 & 135452.72 & 1.00 & 0.91 & 0.91 \\
19820 & 102653 & 1980 & 1133.75 & 0.30 & 113374.00 & 972429.53 & 1.00 & 0.86 & 0.86 \\
14537 & 101876 & 1980 & 94.78 & 0.23 & 9479.00 & 84314.45 & 1.00 & 0.89 & 0.89 \\
17461 & 102307 & 1980 & 16.99 & 0.45 & 1699.00 & 15242.78 & 1.00 & 0.90 & 0.90 \\
17530 & 102318 & 1980 & 54.26 & 0.36 & 5426.00 & 48423.67 & 1.00 & 0.89 & 0.89 \\
16342 & 102130 & 1980 & 30.29 & 0.19 & 3028.00 & 30345.07 & 1.00 & 1.00 & 1.00 \\
5922 & 100812 & 1980 & 115.08 & 0.54 & 11531.00 & 99918.42 & 1.00 & 0.87 & 0.87 \\
5036 & 100705 & 1980 & 22.81 & 0.37 & 2281.00 & 18967.35 & 1.00 & 0.83 & 0.83 \\
16530 & 102153 & 1980 & 13.60 & 0.29 & 1355.00 & 13069.22 & 1.00 & 0.96 & 0.96 \\
46575 & 200257 & 1980 & 665.57 & 0.26 & 55751.00 & 486735.92 & 1.19 & 0.73 & 0.87 \\
4919 & 100695 & 1980 & 48.72 & 0.31 & 4872.00 & 49232.17 & 1.00 & 1.01 & 1.01 \\
18251 & 102419 & 1980 & 361.37 & 0.42 & 36137.00 & 372969.34 & 1.00 & 1.03 & 1.03 \\
16088 & 102080 & 1980 & 55.13 & 0.34 & 5513.00 & 48589.22 & 1.00 & 0.88 & 0.88 \\
15842 & 102048 & 1980 & 190.00 & 0.40 & 21236.00 & 204227.32 & 0.89 & 1.07 & 0.96 \\
57770 & 401028 & 1980 & 13.70 & 0.22 & 1370.00 & 11975.20 & 1.00 & 0.87 & 0.87 \\
5628 & 100777 & 1980 & 1.10 & -0.26 & 129.00 & 977.32 & 0.85 & 0.89 & 0.76 \\
17020 & 102231 & 1980 & 827.34 & 0.33 & 82734.00 & 798526.12 & 1.00 & 0.97 & 0.97 \\
17054 & 102235 & 1980 & 112.94 & 0.38 & 10494.00 & 110503.05 & 1.08 & 0.98 & 1.05 \\
16948 & 102224 & 1980 & 14.40 & 0.44 & 1444.00 & 12158.52 & 1.00 & 0.84 & 0.84 \\
5492 & 100767 & 1980 & 72.90 & 0.45 & 7291.00 & 64837.42 & 1.00 & 0.89 & 0.89 \\
26269 & 103551 & 1980 & 5.70 & 0.26 & 570.00 & 5820.92 & 1.00 & 1.02 & 1.02 \\
17210 & 102271 & 1980 & 37.99 & 0.49 & 3799.00 & 36412.69 & 1.00 & 0.96 & 0.96 \\
17417 & 102306 & 1980 & 64.75 & 0.44 & 6444.00 & 53721.81 & 1.00 & 0.83 & 0.83 \\
959 & 100113 & 1980 & 1057.09 & 0.31 & 105789.00 & 999525.00 & 1.00 & 0.95 & 0.94 \\
16829 & 102197 & 1980 & 22.46 & 0.20 & 2227.00 & 19864.51 & 1.01 & 0.88 & 0.89 \\
17276 & 102278 & 1980 & 52.01 & 0.26 & 5201.00 & 49796.45 & 1.00 & 0.96 & 0.96 \\
26198 & 103546 & 1980 & 827.73 & 0.23 & 82809.00 & 810027.45 & 1.00 & 0.98 & 0.98 \\
17364 & 102284 & 1980 & 16.95 & 0.44 & 1695.00 & 15337.69 & 1.00 & 0.90 & 0.90 \\
4685 & 100666 & 1980 & 1.80 & 0.25 & 175.00 & 1405.97 & 1.03 & 0.78 & 0.80 \\
15704 & 102016 & 1980 & 808.33 & 0.41 & 85912.00 & 824462.65 & 0.94 & 1.02 & 0.96 \\
18302 & 102425 & 1980 & 169.87 & 0.46 & 16987.00 & 141814.24 & 1.00 & 0.83 & 0.83 \\
15025 & 101953 & 1980 & 5.80 & 0.12 & 579.00 & 5486.39 & 1.00 & 0.95 & 0.95 \\
6617 & 100906 & 1980 & 401.92 & 0.41 & 40194.00 & 336799.53 & 1.00 & 0.84 & 0.84 \\
4201 & 100575 & 1980 & 227.50 & 0.54 & 22752.00 & 217201.23 & 1.00 & 0.95 & 0.95 \\
4200 & 100574 & 1980 & 13.80 & 0.49 & 1378.00 & 11961.19 & 1.00 & 0.87 & 0.87 \\
14855 & 101919 & 1980 & 65.85 & 0.36 & 6585.00 & 57666.62 & 1.00 & 0.88 & 0.88 \\
15155 & 101964 & 1980 & 14.50 & 0.16 & 1454.00 & 13703.40 & 1.00 & 0.95 & 0.94 \\
6728 & 100947 & 1980 & 83.00 & 0.47 & 8332.00 & 75412.89 & 1.00 & 0.91 & 0.91 \\
6815 & 100962 & 1980 & 27.97 & 0.55 & 2792.00 & 23860.09 & 1.00 & 0.85 & 0.85 \\
14593 & 101892 & 1980 & 58.70 & 0.01 & 5873.00 & 55842.08 & 1.00 & 0.95 & 0.95 \\
4156 & 100567 & 1980 & 40.71 & 0.30 & 2478.00 & 20206.57 & 1.64 & 0.50 & 0.82 \\
4155 & 100564 & 1980 & 15.40 & 0.50 & 1544.00 & 12612.53 & 1.00 & 0.82 & 0.82 \\
19547 & 102616 & 1980 & 21.30 & 0.33 & 2131.00 & 20369.14 & 1.00 & 0.96 & 0.96 \\
637 & 100087 & 1980 & 53.40 & 0.66 & 5340.00 & 53253.63 & 1.00 & 1.00 & 1.00 \\
1190 & 100160 & 1980 & 44.34 & 0.16 & 4434.00 & 39196.22 & 1.00 & 0.88 & 0.88 \\
18334 & 102441 & 1980 & 90.88 & 0.41 & 9087.00 & 80240.65 & 1.00 & 0.88 & 0.88 \\
6238 & 100833 & 1980 & 105.77 & 0.41 & 10728.00 & 85589.55 & 0.99 & 0.81 & 0.80 \\
15553 & 102005 & 1980 & 107.41 & 0.57 & 12169.00 & 103411.32 & 0.88 & 0.96 & 0.85 \\
15509 & 102000 & 1980 & 68.76 & 0.37 & 6872.00 & 64351.37 & 1.00 & 0.94 & 0.94 \\
6282 & 100847 & 1980 & 4.76 & 0.19 & 453.00 & 3963.00 & 1.05 & 0.83 & 0.87 \\
4606 & 100644 & 1980 & 36.23 & 0.29 & 3623.00 & 32004.07 & 1.00 & 0.88 & 0.88 \\
25668 & 103508 & 1980 & 78.22 & 0.36 & 6376.00 & 68170.20 & 1.23 & 0.87 & 1.07 \\
6359 & 100856 & 1980 & 57.28 & 0.33 & 5728.00 & 52602.52 & 1.00 & 0.92 & 0.92 \\
4458 & 100634 & 1980 & 36.82 & 0.40 & 3682.00 & 35138.76 & 1.00 & 0.95 & 0.95 \\
18784 & 102522 & 1980 & 161.57 & 0.26 & 16157.00 & 151396.45 & 1.00 & 0.94 & 0.94 \\
47232 & 200344 & 1980 & 144.00 & 0.33 & 14395.00 & 129932.38 & 1.00 & 0.90 & 0.90 \\
47360 & 210681 & 1980 & 62.00 & 0.30 & 6267.00 & 63751.30 & 0.99 & 1.03 & 1.02 \\
4362 & 100614 & 1980 & 74.70 & 0.41 & 7470.00 & 64445.62 & 1.00 & 0.86 & 0.86 \\
10591 & 101300 & 1980 & 9.38 & 0.06 & 938.00 & 8206.55 & 1.00 & 0.87 & 0.87 \\
23526 & 103184 & 1980 & 62.50 & 0.51 & 6253.00 & 57363.27 & 1.00 & 0.92 & 0.92 \\
23230 & 103152 & 1980 & 201.84 & 0.25 & 20184.00 & 208477.79 & 1.00 & 1.03 & 1.03 \\
9244 & 101123 & 1980 & 111.20 & 0.56 & 9597.00 & 105090.87 & 1.16 & 0.95 & 1.10 \\
2589 & 100346 & 1980 & 2.23 & 0.27 & 223.00 & 2105.24 & 1.00 & 0.95 & 0.94 \\
2564 & 100344 & 1980 & 9.50 & 0.31 & 1018.00 & 9485.13 & 0.93 & 1.00 & 0.93 \\
2180 & 100295 & 1980 & 8.20 & 0.45 & 716.00 & 7363.15 & 1.15 & 0.90 & 1.03 \\
10442 & 101286 & 1980 & 76.81 & 0.35 & 7685.00 & 63409.94 & 1.00 & 0.83 & 0.83 \\
8948 & 101107 & 1980 & 644.00 & 0.26 & 61809.00 & 646756.85 & 1.04 & 1.00 & 1.05 \\
1727 & 100227 & 1980 & 49.28 & 0.19 & 4928.00 & 39800.38 & 1.00 & 0.81 & 0.81 \\
9705 & 101170 & 1980 & 2.00 & 0.73 & 199.00 & 1917.77 & 1.01 & 0.96 & 0.96 \\
23324 & 103164 & 1980 & 18.20 & 0.32 & 1819.00 & 14744.38 & 1.00 & 0.81 & 0.81 \\
23500 & 103183 & 1980 & 223.70 & 0.63 & 22381.00 & 217421.12 & 1.00 & 0.97 & 0.97 \\
9711 & 101177 & 1980 & 4.71 & 0.10 & 471.00 & 4033.25 & 1.00 & 0.86 & 0.86 \\
10177 & 101268 & 1980 & 41.25 & 0.51 & 4125.00 & 35181.27 & 1.00 & 0.85 & 0.85 \\
9732 & 101181 & 1980 & 71.53 & 0.40 & 7152.00 & 71616.64 & 1.00 & 1.00 & 1.00 \\
22878 & 103077 & 1980 & 9.60 & 0.27 & 956.00 & 8628.62 & 1.00 & 0.90 & 0.90 \\
10053 & 101257 & 1980 & 193.49 & 0.36 & 19349.00 & 200010.49 & 1.00 & 1.03 & 1.03 \\
2294 & 100315 & 1980 & 290.90 & 0.28 & 29074.00 & 284770.59 & 1.00 & 0.98 & 0.98 \\
23458 & 103179 & 1980 & 125.00 & 0.79 & 12550.00 & 115753.73 & 1.00 & 0.93 & 0.92 \\
9260 & 101127 & 1980 & 30.31 & 0.52 & 3031.00 & 27026.89 & 1.00 & 0.89 & 0.89 \\
10547 & 101299 & 1980 & 8.13 & 0.15 & 813.00 & 7354.75 & 1.00 & 0.90 & 0.90 \\
2282 & 100313 & 1980 & 98.66 & 0.37 & 9866.00 & 100320.19 & 1.00 & 1.02 & 1.02 \\
10279 & 101277 & 1980 & 7.00 & 0.26 & 698.00 & 6010.73 & 1.00 & 0.86 & 0.86 \\
125 & 100010 & 1980 & 124.51 & 0.49 & 12450.00 & 122157.82 & 1.00 & 0.98 & 0.98 \\
10282 & 101278 & 1980 & 10.60 & 0.24 & 1086.00 & 9938.62 & 0.98 & 0.94 & 0.92 \\
22924 & 103089 & 1980 & 57.56 & 0.31 & 5756.00 & 57763.25 & 1.00 & 1.00 & 1.00 \\
24548 & 103352 & 1980 & 44.60 & 0.40 & 4463.00 & 36705.41 & 1.00 & 0.82 & 0.82 \\
23559 & 103192 & 1980 & 2.20 & 0.13 & 226.00 & 2067.58 & 0.97 & 0.94 & 0.91 \\
10398 & 101285 & 1980 & 11.00 & 0.38 & 1315.00 & 12299.22 & 0.84 & 1.12 & 0.94 \\
3900 & 100514 & 1981 & 20.01 & -0.00 & 1952.00 & 19523.21 & 1.02 & 0.98 & 1.00 \\
3379 & 100430 & 1981 & 45.40 & -0.12 & 3832.00 & 40012.23 & 1.18 & 0.88 & 1.04 \\
14113 & 101805 & 1981 & 430.10 & -0.23 & 38383.00 & 400684.89 & 1.12 & 0.93 & 1.04 \\
552 & 100076 & 1981 & 132.80 & -0.05 & 13282.00 & 120086.78 & 1.00 & 0.90 & 0.90 \\
3949 & 100531 & 1981 & 20.70 & -0.20 & 1943.00 & 18271.10 & 1.07 & 0.88 & 0.94 \\
14096 & 101804 & 1981 & 106.38 & -0.03 & 10377.00 & 105760.70 & 1.03 & 0.99 & 1.02 \\
3953 & 100535 & 1981 & 68.98 & 0.20 & 6898.00 & 65677.59 & 1.00 & 0.95 & 0.95 \\
6729 & 100947 & 1981 & 99.60 & -0.02 & 9681.00 & 86736.62 & 1.03 & 0.87 & 0.90 \\
19292 & 102583 & 1981 & 49.90 & 0.01 & 4992.00 & 41216.95 & 1.00 & 0.83 & 0.83 \\
19821 & 102653 & 1981 & 919.43 & -0.13 & 91943.00 & 834641.12 & 1.00 & 0.91 & 0.91 \\
12863 & 101603 & 1981 & 357.02 & 0.01 & 36748.00 & 286898.84 & 0.97 & 0.80 & 0.78 \\
2295 & 100315 & 1981 & 255.12 & -0.06 & 25540.00 & 240483.48 & 1.00 & 0.94 & 0.94 \\
12896 & 101606 & 1981 & 336.10 & 0.07 & 33620.00 & 287163.43 & 1.00 & 0.85 & 0.85 \\
10054 & 101257 & 1981 & 122.20 & -0.23 & 14076.00 & 141175.55 & 0.87 & 1.16 & 1.00 \\
19548 & 102616 & 1981 & 17.70 & -0.05 & 1758.00 & 17289.46 & 1.01 & 0.98 & 0.98 \\
14538 & 101876 & 1981 & 91.22 & 0.06 & 9134.00 & 86153.06 & 1.00 & 0.94 & 0.94 \\
3886 & 100510 & 1981 & 4.20 & 0.04 & 379.00 & 3500.70 & 1.11 & 0.83 & 0.92 \\
14003 & 101800 & 1981 & 116.18 & -0.11 & 11103.00 & 110137.41 & 1.05 & 0.95 & 0.99 \\
19931 & 102659 & 1981 & 875.77 & -0.08 & 87577.00 & 708583.17 & 1.00 & 0.81 & 0.81 \\
20691 & 102784 & 1981 & 89.93 & -0.04 & 8911.00 & 77473.12 & 1.01 & 0.86 & 0.87 \\
20828 & 102796 & 1981 & 1.75 & -0.03 & 171.00 & 1591.19 & 1.03 & 0.91 & 0.93 \\
20580 & 102774 & 1981 & 68.66 & 0.07 & 7196.00 & 55081.60 & 0.95 & 0.80 & 0.77 \\
20540 & 102767 & 1981 & 662.40 & -0.09 & 58799.00 & 609541.45 & 1.13 & 0.92 & 1.04 \\
22925 & 103089 & 1981 & 50.50 & -0.07 & 5046.00 & 49936.27 & 1.00 & 0.99 & 0.99 \\
1330 & 100190 & 1981 & 516.80 & -0.13 & 51680.00 & 469150.58 & 1.00 & 0.91 & 0.91 \\
13848 & 101781 & 1981 & 63.19 & -0.02 & 6410.00 & 60068.90 & 0.99 & 0.95 & 0.94 \\
23231 & 103152 & 1981 & 199.17 & -0.04 & 21828.00 & 198413.55 & 0.91 & 1.00 & 0.91 \\
13821 & 101769 & 1981 & 159.55 & -0.11 & 16000.00 & 139377.45 & 1.00 & 0.87 & 0.87 \\
25262 & 103464 & 1981 & 60.92 & -0.08 & 5606.00 & 53756.72 & 1.09 & 0.88 & 0.96 \\
13642 & 101754 & 1981 & 22.91 & 0.01 & 2290.00 & 19953.91 & 1.00 & 0.87 & 0.87 \\
8949 & 101107 & 1981 & 537.13 & -0.05 & 53710.00 & 517249.32 & 1.00 & 0.96 & 0.96 \\
7373 & 101038 & 1981 & 230.24 & 0.07 & 23020.00 & 211247.76 & 1.00 & 0.92 & 0.92 \\
10280 & 101277 & 1981 & 5.70 & -0.04 & 568.00 & 4931.99 & 1.00 & 0.87 & 0.87 \\
20261 & 102702 & 1981 & 22.70 & 0.02 & 2047.00 & 21748.78 & 1.11 & 0.96 & 1.06 \\
15026 & 101953 & 1981 & 5.90 & 0.20 & 589.00 & 4934.35 & 1.00 & 0.84 & 0.84 \\
14856 & 101919 & 1981 & 94.63 & 0.08 & 7228.00 & 74273.18 & 1.31 & 0.78 & 1.03 \\
17462 & 102307 & 1981 & 15.20 & -0.02 & 1525.00 & 14495.40 & 1.00 & 0.95 & 0.95 \\
16343 & 102130 & 1981 & 29.30 & 0.03 & 2774.00 & 28353.31 & 1.06 & 0.97 & 1.02 \\
9261 & 101127 & 1981 & 30.71 & -0.13 & 2654.00 & 24281.44 & 1.16 & 0.79 & 0.91 \\
5852 & 100809 & 1981 & 1635.87 & 0.05 & 151750.00 & 1314888.27 & 1.08 & 0.80 & 0.87 \\
5923 & 100812 & 1981 & 126.84 & -0.15 & 10890.00 & 99173.29 & 1.16 & 0.78 & 0.91 \\
23527 & 103184 & 1981 & 56.20 & -0.11 & 5621.00 & 46074.12 & 1.00 & 0.82 & 0.82 \\
4920 & 100695 & 1981 & 44.59 & -0.03 & 4459.00 & 40134.89 & 1.00 & 0.90 & 0.90 \\
17798 & 102361 & 1981 & 8.68 & -0.21 & 868.00 & 7474.83 & 1.00 & 0.86 & 0.86 \\
4789 & 100682 & 1981 & 1.80 & 0.04 & 149.00 & 1315.55 & 1.21 & 0.73 & 0.88 \\
46576 & 200257 & 1981 & 640.50 & -0.00 & 72404.00 & 517724.73 & 0.88 & 0.81 & 0.72 \\
46594 & 200259 & 1981 & 600.91 & 0.06 & 56099.00 & 453366.81 & 1.07 & 0.75 & 0.81 \\
17418 & 102306 & 1981 & 63.00 & 0.01 & 5366.00 & 49389.47 & 1.17 & 0.78 & 0.92 \\
23560 & 103192 & 1981 & 2.70 & 0.01 & 261.00 & 2745.15 & 1.03 & 1.02 & 1.05 \\
17021 & 102231 & 1981 & 645.75 & -0.13 & 64575.00 & 606584.56 & 1.00 & 0.94 & 0.94 \\
24549 & 103352 & 1981 & 40.20 & -0.08 & 4021.00 & 35648.63 & 1.00 & 0.89 & 0.89 \\
5493 & 100767 & 1981 & 62.10 & -0.12 & 4766.00 & 46004.78 & 1.30 & 0.74 & 0.97 \\
26270 & 103551 & 1981 & 5.50 & 0.02 & 561.00 & 5155.76 & 0.98 & 0.94 & 0.92 \\
17211 & 102271 & 1981 & 42.51 & 0.03 & 3902.00 & 32179.78 & 1.09 & 0.76 & 0.82 \\
960 & 100113 & 1981 & 997.40 & -0.04 & 99740.00 & 949799.90 & 1.00 & 0.95 & 0.95 \\
17277 & 102278 & 1981 & 55.03 & 0.09 & 5000.00 & 43570.10 & 1.10 & 0.79 & 0.87 \\
26199 & 103546 & 1981 & 960.83 & 0.17 & 89809.00 & 933072.14 & 1.07 & 0.97 & 1.04 \\
57809 & 401072 & 1981 & 24.97 & 0.02 & 2497.00 & 21700.31 & 1.00 & 0.87 & 0.87 \\
17365 & 102284 & 1981 & 13.40 & -0.13 & 947.00 & 9275.07 & 1.41 & 0.69 & 0.98 \\
18205 & 102416 & 1981 & 45.40 & -0.10 & 4541.00 & 43558.61 & 1.00 & 0.96 & 0.96 \\
16006 & 102065 & 1981 & 15.80 & -0.08 & 1585.00 & 15241.78 & 1.00 & 0.96 & 0.96 \\
18785 & 102522 & 1981 & 142.49 & 0.01 & 14250.00 & 130029.59 & 1.00 & 0.91 & 0.91 \\
47361 & 210681 & 1981 & 63.23 & 0.03 & 5961.00 & 62017.77 & 1.06 & 0.98 & 1.04 \\
4363 & 100614 & 1981 & 59.60 & -0.08 & 5956.00 & 51540.83 & 1.00 & 0.86 & 0.87 \\
9733 & 101181 & 1981 & 58.40 & -0.11 & 5523.00 & 57772.54 & 1.06 & 0.99 & 1.05 \\
1191 & 100160 & 1981 & 35.64 & -0.09 & 3564.00 & 34306.17 & 1.00 & 0.96 & 0.96 \\
24706 & 103376 & 1981 & 281.20 & -0.06 & 28120.00 & 279488.73 & 1.00 & 0.99 & 0.99 \\
638 & 100087 & 1981 & 38.05 & -0.12 & 3810.00 & 38766.01 & 1.00 & 1.02 & 1.02 \\
21230 & 102839 & 1981 & 4.00 & -0.06 & 396.00 & 3902.15 & 1.01 & 0.98 & 0.99 \\
23459 & 103179 & 1981 & 126.61 & -0.15 & 12661.00 & 119866.98 & 1.00 & 0.95 & 0.95 \\
15156 & 101964 & 1981 & 13.15 & -0.04 & 1315.00 & 12312.17 & 1.00 & 0.94 & 0.94 \\
4459 & 100634 & 1981 & 34.94 & -0.03 & 3494.00 & 32917.74 & 1.00 & 0.94 & 0.94 \\
6360 & 100856 & 1981 & 53.45 & 0.01 & 5344.00 & 49374.93 & 1.00 & 0.92 & 0.92 \\
25669 & 103508 & 1981 & 97.73 & 0.05 & 9773.00 & 82513.72 & 1.00 & 0.84 & 0.84 \\
18252 & 102419 & 1981 & 287.55 & -0.11 & 28754.00 & 274297.31 & 1.00 & 0.95 & 0.95 \\
15843 & 102048 & 1981 & 195.02 & -0.08 & 19502.00 & 190258.03 & 1.00 & 0.98 & 0.98 \\
15705 & 102016 & 1981 & 729.12 & -0.06 & 71326.00 & 686405.48 & 1.02 & 0.94 & 0.96 \\
18303 & 102425 & 1981 & 238.04 & -0.06 & 23804.00 & 194132.23 & 1.00 & 0.82 & 0.82 \\
1119 & 100155 & 1981 & 13.29 & 0.11 & 1248.00 & 12002.90 & 1.07 & 0.90 & 0.96 \\
6239 & 100833 & 1981 & 101.20 & -0.11 & 9195.00 & 77687.59 & 1.10 & 0.77 & 0.84 \\
23501 & 103183 & 1981 & 178.38 & -0.19 & 17838.00 & 155564.54 & 1.00 & 0.87 & 0.87 \\
4607 & 100644 & 1981 & 31.12 & -0.07 & 3112.00 & 25423.71 & 1.00 & 0.82 & 0.82 \\
15510 & 102000 & 1981 & 57.17 & -0.07 & 5744.00 & 48458.26 & 1.00 & 0.85 & 0.84 \\
2283 & 100313 & 1981 & 123.55 & -0.04 & 10769.00 & 119166.87 & 1.15 & 0.96 & 1.11 \\
3282 & 100420 & 1981 & 2.02 & 0.00 & 202.00 & 1998.62 & 1.00 & 0.99 & 0.99 \\
17055 & 102235 & 1981 & 80.85 & -0.09 & 8750.00 & 75419.35 & 0.92 & 0.93 & 0.86 \\
10443 & 101286 & 1981 & 56.70 & -0.10 & 4878.00 & 50395.62 & 1.16 & 0.89 & 1.03 \\
22879 & 103077 & 1981 & 9.24 & -0.01 & 924.00 & 8585.82 & 1.00 & 0.93 & 0.93 \\
22167 & 102994 & 1981 & 30.64 & -0.07 & 3064.00 & 25327.12 & 1.00 & 0.83 & 0.83 \\
22394 & 103011 & 1981 & 26.04 & 0.05 & 2587.00 & 22548.19 & 1.01 & 0.87 & 0.87 \\
10548 & 101299 & 1981 & 6.10 & -0.05 & 712.00 & 6915.35 & 0.86 & 1.13 & 0.97 \\
11695 & 101457 & 1981 & 43.04 & 0.06 & 4304.00 & 41455.77 & 1.00 & 0.96 & 0.96 \\
21938 & 102981 & 1981 & 47.48 & -0.25 & 4660.00 & 39876.83 & 1.02 & 0.84 & 0.86 \\
22769 & 103061 & 1981 & 6.90 & -0.00 & 689.00 & 5717.50 & 1.00 & 0.83 & 0.83 \\
22198 & 102996 & 1981 & 49.47 & -0.04 & 4947.00 & 46022.76 & 1.00 & 0.93 & 0.93 \\
3166 & 100412 & 1981 & 3.30 & -0.01 & 288.00 & 2887.14 & 1.15 & 0.87 & 1.00 \\
21732 & 102948 & 1981 & 12.00 & 0.01 & 1234.00 & 11253.97 & 0.97 & 0.94 & 0.91 \\
22358 & 103008 & 1981 & 25.19 & -0.05 & 2594.00 & 21603.75 & 0.97 & 0.86 & 0.83 \\
2975 & 100395 & 1981 & 198.09 & -0.04 & 19620.00 & 157631.79 & 1.01 & 0.80 & 0.80 \\
7959 & 101068 & 1981 & 107.25 & 0.16 & 10730.00 & 100264.13 & 1.00 & 0.93 & 0.93 \\
22662 & 103028 & 1981 & 12.67 & 0.02 & 1270.00 & 10818.63 & 1.00 & 0.85 & 0.85 \\
2777 & 100358 & 1981 & 11.44 & -0.01 & 1144.00 & 11203.84 & 1.00 & 0.98 & 0.98 \\
21756 & 102951 & 1981 & 147.75 & -0.06 & 14782.00 & 137075.67 & 1.00 & 0.93 & 0.93 \\
1728 & 100227 & 1981 & 53.48 & 0.04 & 5176.00 & 45396.04 & 1.03 & 0.85 & 0.88 \\
49087 & 240222 & 1981 & 198.27 & -0.15 & 19810.00 & 167711.89 & 1.00 & 0.85 & 0.85 \\
2935 & 100389 & 1981 & 14.05 & -0.03 & 1410.00 & 13551.64 & 1.00 & 0.96 & 0.96 \\
12046 & 101494 & 1981 & 31.20 & -0.17 & 3124.00 & 27907.26 & 1.00 & 0.89 & 0.89 \\
1579 & 100217 & 1981 & 29.30 & 0.04 & 2706.00 & 24013.52 & 1.08 & 0.82 & 0.89 \\
2590 & 100346 & 1981 & 2.13 & 0.02 & 210.00 & 1981.65 & 1.02 & 0.93 & 0.94 \\
1652 & 100222 & 1981 & 180.54 & -0.06 & 18005.00 & 177267.76 & 1.00 & 0.98 & 0.98 \\
24955 & 103397 & 1981 & 3.80 & -0.08 & 380.00 & 3513.05 & 1.00 & 0.92 & 0.92 \\
10592 & 101300 & 1981 & 11.56 & 0.20 & 1015.00 & 9497.80 & 1.14 & 0.82 & 0.94 \\
10399 & 101285 & 1981 & 12.05 & -0.08 & 895.00 & 7620.79 & 1.35 & 0.63 & 0.85 \\
24975 & 103406 & 1982 & 432.72 & 0.36 & 43272.00 & 361697.04 & 1.00 & 0.84 & 0.84 \\
15844 & 102048 & 1982 & 214.79 & 0.15 & 21479.00 & 185518.36 & 1.00 & 0.86 & 0.86 \\
2976 & 100395 & 1982 & 213.10 & 0.14 & 18336.00 & 199010.41 & 1.16 & 0.93 & 1.09 \\
3887 & 100510 & 1982 & 5.90 & 0.21 & 590.00 & 5998.95 & 1.00 & 1.02 & 1.02 \\
13822 & 101769 & 1982 & 189.01 & 0.29 & 18901.00 & 159062.58 & 1.00 & 0.84 & 0.84 \\
17463 & 102307 & 1982 & 20.80 & 0.36 & 2182.00 & 18707.76 & 0.95 & 0.90 & 0.86 \\
2936 & 100389 & 1982 & 15.34 & 0.11 & 1534.00 & 14501.86 & 1.00 & 0.95 & 0.95 \\
21795 & 102952 & 1982 & 167.64 & 0.16 & 12734.00 & 111676.82 & 1.32 & 0.67 & 0.88 \\
19204 & 102570 & 1982 & 34.14 & 0.43 & 3160.00 & 29294.93 & 1.08 & 0.86 & 0.93 \\
12047 & 101494 & 1982 & 35.77 & 0.21 & 3577.00 & 30372.63 & 1.00 & 0.85 & 0.85 \\
14097 & 101804 & 1982 & 124.32 & 0.25 & 9448.00 & 120594.21 & 1.32 & 0.97 & 1.28 \\
23561 & 103192 & 1982 & 3.80 & 0.37 & 278.00 & 2886.30 & 1.37 & 0.76 & 1.04 \\
3901 & 100514 & 1982 & 59.74 & 0.17 & 5974.00 & 61137.70 & 1.00 & 1.02 & 1.02 \\
47362 & 210681 & 1982 & 72.50 & 0.17 & 6367.00 & 69794.01 & 1.14 & 0.96 & 1.10 \\
6313 & 100849 & 1982 & 46.69 & 0.33 & 3682.00 & 38152.30 & 1.27 & 0.82 & 1.04 \\
19932 & 102659 & 1982 & 1198.53 & 0.23 & 119853.00 & 1065501.13 & 1.00 & 0.89 & 0.89 \\
23232 & 103152 & 1982 & 261.17 & 0.20 & 26921.00 & 261524.40 & 0.97 & 1.00 & 0.97 \\
17278 & 102278 & 1982 & 65.18 & 0.28 & 5286.00 & 52945.73 & 1.23 & 0.81 & 1.00 \\
24627 & 103373 & 1982 & 37.75 & 0.03 & 2591.00 & 23294.33 & 1.46 & 0.62 & 0.90 \\
13849 & 101781 & 1982 & 71.59 & 0.28 & 7159.00 & 66662.18 & 1.00 & 0.93 & 0.93 \\
2284 & 100313 & 1982 & 160.46 & 0.26 & 16050.00 & 158954.80 & 1.00 & 0.99 & 0.99 \\
11938 & 101473 & 1982 & 35.80 & 0.25 & 3580.00 & 30202.56 & 1.00 & 0.84 & 0.84 \\
16007 & 102065 & 1982 & 33.74 & 0.38 & 3374.00 & 30963.01 & 1.00 & 0.92 & 0.92 \\
17366 & 102284 & 1982 & 20.06 & 0.27 & 1630.00 & 18519.49 & 1.23 & 0.92 & 1.14 \\
26200 & 103546 & 1982 & 1250.96 & 0.22 & 117480.00 & 973101.41 & 1.06 & 0.78 & 0.83 \\
6361 & 100856 & 1982 & 71.59 & 0.24 & 5662.00 & 61205.36 & 1.26 & 0.85 & 1.08 \\
14004 & 101800 & 1982 & 131.09 & 0.27 & 11935.00 & 120297.80 & 1.10 & 0.92 & 1.01 \\
17419 & 102306 & 1982 & 79.93 & 0.25 & 5862.00 & 59620.36 & 1.36 & 0.75 & 1.02 \\
18253 & 102419 & 1982 & 286.30 & 0.08 & 25127.00 & 236844.66 & 1.14 & 0.83 & 0.94 \\
15706 & 102016 & 1982 & 873.47 & 0.28 & 86706.00 & 810150.30 & 1.01 & 0.93 & 0.93 \\
14114 & 101805 & 1982 & 594.98 & 0.20 & 35991.00 & 493238.40 & 1.65 & 0.83 & 1.37 \\
6240 & 100833 & 1982 & 116.78 & 0.05 & 9758.00 & 80644.66 & 1.20 & 0.69 & 0.83 \\
14494 & 101865 & 1982 & 39.73 & 0.02 & 3439.00 & 41590.01 & 1.16 & 1.05 & 1.21 \\
25670 & 103508 & 1982 & 119.01 & 0.16 & 8643.00 & 86733.85 & 1.38 & 0.73 & 1.00 \\
961 & 100113 & 1982 & 1130.92 & 0.22 & 113092.00 & 1058268.52 & 1.00 & 0.94 & 0.94 \\
2296 & 100315 & 1982 & 248.50 & 0.15 & 24853.00 & 237948.21 & 1.00 & 0.96 & 0.96 \\
14857 & 101919 & 1982 & 160.98 & 0.30 & 9931.00 & 140812.14 & 1.62 & 0.87 & 1.42 \\
553 & 100076 & 1982 & 164.24 & 0.21 & 12204.00 & 140970.30 & 1.35 & 0.86 & 1.16 \\
9707 & 101170 & 1982 & 5.10 & 0.07 & 446.00 & 4211.77 & 1.14 & 0.83 & 0.94 \\
22770 & 103061 & 1982 & 4.60 & -0.04 & 460.00 & 3715.47 & 1.00 & 0.81 & 0.81 \\
19822 & 102653 & 1982 & 1007.51 & 0.17 & 100751.00 & 944277.83 & 1.00 & 0.94 & 0.94 \\
12032 & 101491 & 1982 & 94.85 & 0.31 & 7790.00 & 68718.26 & 1.22 & 0.72 & 0.88 \\
1957 & 100263 & 1982 & 3.00 & 0.13 & 273.00 & 2457.58 & 1.10 & 0.82 & 0.90 \\
15511 & 102000 & 1982 & 65.84 & 0.29 & 6590.00 & 60575.38 & 1.00 & 0.92 & 0.92 \\
4608 & 100644 & 1982 & 33.91 & 0.19 & 3391.00 & 30901.93 & 1.00 & 0.91 & 0.91 \\
10055 & 101257 & 1982 & 200.55 & 0.09 & 12304.00 & 183969.54 & 1.63 & 0.92 & 1.50 \\
14539 & 101876 & 1982 & 93.45 & 0.12 & 9340.00 & 84540.67 & 1.00 & 0.90 & 0.91 \\
14430 & 101858 & 1982 & 62.70 & 0.39 & 4349.00 & 42654.14 & 1.44 & 0.68 & 0.98 \\
3997 & 100538 & 1982 & 38.48 & 0.25 & 3012.00 & 38487.59 & 1.28 & 1.00 & 1.28 \\
18304 & 102425 & 1982 & 373.10 & 0.21 & 30515.00 & 295431.35 & 1.22 & 0.79 & 0.97 \\
19904 & 102655 & 1982 & 411.51 & 0.02 & 41151.00 & 343616.35 & 1.00 & 0.84 & 0.84 \\
10179 & 101268 & 1982 & 41.12 & 0.22 & 3766.00 & 36860.82 & 1.09 & 0.90 & 0.98 \\
22880 & 103077 & 1982 & 13.11 & 0.12 & 1311.00 & 11821.32 & 1.00 & 0.90 & 0.90 \\
22199 & 102996 & 1982 & 79.01 & 0.51 & 7901.00 & 68515.14 & 1.00 & 0.87 & 0.87 \\
15555 & 102005 & 1982 & 293.16 & 0.41 & 24741.00 & 268829.88 & 1.18 & 0.92 & 1.09 \\
21939 & 102981 & 1982 & 70.61 & 0.32 & 7060.00 & 62753.96 & 1.00 & 0.89 & 0.89 \\
3950 & 100531 & 1982 & 22.70 & 0.25 & 2273.00 & 20730.39 & 1.00 & 0.91 & 0.91 \\
18336 & 102441 & 1982 & 106.40 & 0.39 & 8246.00 & 76510.42 & 1.29 & 0.72 & 0.93 \\
3954 & 100535 & 1982 & 67.98 & 0.11 & 6798.00 & 65643.54 & 1.00 & 0.97 & 0.97 \\
4203 & 100575 & 1982 & 122.30 & 0.09 & 12972.00 & 122801.20 & 0.94 & 1.00 & 0.95 \\
1120 & 100155 & 1982 & 17.80 & 0.41 & 1354.00 & 15415.54 & 1.31 & 0.87 & 1.14 \\
6730 & 100947 & 1982 & 147.74 & 0.20 & 14755.00 & 118674.86 & 1.00 & 0.80 & 0.80 \\
11696 & 101457 & 1982 & 54.05 & 0.28 & 5405.00 & 49598.74 & 1.00 & 0.92 & 0.92 \\
23460 & 103179 & 1982 & 197.91 & 0.25 & 19791.00 & 177441.95 & 1.00 & 0.90 & 0.90 \\
5924 & 100812 & 1982 & 189.47 & 0.03 & 10624.00 & 97545.97 & 1.78 & 0.51 & 0.92 \\
18800 & 102523 & 1982 & 47.87 & 0.33 & 2674.00 & 45015.16 & 1.79 & 0.94 & 1.68 \\
5038 & 100705 & 1982 & 67.00 & 0.40 & 4982.00 & 68143.92 & 1.34 & 1.02 & 1.37 \\
16950 & 102224 & 1982 & 21.10 & 0.46 & 2113.00 & 18270.03 & 1.00 & 0.87 & 0.86 \\
10593 & 101300 & 1982 & 17.96 & 0.30 & 1537.00 & 13940.20 & 1.17 & 0.78 & 0.91 \\
17659 & 102342 & 1982 & 2.80 & 0.12 & 275.00 & 2605.13 & 1.02 & 0.93 & 0.95 \\
22395 & 103011 & 1982 & 28.69 & 0.09 & 2869.00 & 23734.32 & 1.00 & 0.83 & 0.83 \\
46577 & 200257 & 1982 & 681.04 & 0.11 & 77433.00 & 559583.60 & 0.88 & 0.82 & 0.72 \\
18786 & 102522 & 1982 & 155.41 & 0.21 & 15541.00 & 138434.90 & 1.00 & 0.89 & 0.89 \\
20219 & 102689 & 1982 & 6.03 & 0.32 & 434.00 & 5009.76 & 1.39 & 0.83 & 1.15 \\
12932 & 101616 & 1982 & 282.00 & 0.34 & 28170.00 & 226129.08 & 1.00 & 0.80 & 0.80 \\
25263 & 103464 & 1982 & 77.94 & 0.21 & 6078.00 & 68338.08 & 1.28 & 0.88 & 1.12 \\
10400 & 101285 & 1982 & 14.11 & 0.38 & 1235.00 & 12288.54 & 1.14 & 0.87 & 1.00 \\
3283 & 100420 & 1982 & 1.95 & 0.23 & 194.00 & 1951.73 & 1.01 & 1.00 & 1.01 \\
4364 & 100614 & 1982 & 63.80 & 0.24 & 5646.00 & 53491.21 & 1.13 & 0.84 & 0.95 \\
9262 & 101127 & 1982 & 45.90 & 0.30 & 3482.00 & 35619.69 & 1.32 & 0.78 & 1.02 \\
22168 & 102994 & 1982 & 31.05 & 0.17 & 2700.00 & 27321.03 & 1.15 & 0.88 & 1.01 \\
16344 & 102130 & 1982 & 37.92 & 0.36 & 3656.00 & 34126.08 & 1.04 & 0.90 & 0.93 \\
24707 & 103376 & 1982 & 439.09 & 0.09 & 43910.00 & 436949.47 & 1.00 & 1.00 & 1.00 \\
17022 & 102231 & 1982 & 704.79 & 0.28 & 70479.00 & 665674.22 & 1.00 & 0.94 & 0.94 \\
3167 & 100412 & 1982 & 4.30 & 0.30 & 393.00 & 3972.74 & 1.09 & 0.92 & 1.01 \\
12897 & 101606 & 1982 & 457.60 & 0.33 & 45760.00 & 442286.93 & 1.00 & 0.97 & 0.97 \\
1192 & 100160 & 1982 & 40.37 & 0.21 & 4037.00 & 38723.22 & 1.00 & 0.96 & 0.96 \\
3380 & 100430 & 1982 & 72.82 & 0.12 & 5542.00 & 49756.81 & 1.31 & 0.68 & 0.90 \\
9305 & 101131 & 1982 & 69.53 & 0.37 & 7606.00 & 78818.97 & 0.91 & 1.13 & 1.04 \\
24550 & 103352 & 1982 & 45.60 & 0.13 & 3882.00 & 41112.84 & 1.17 & 0.90 & 1.06 \\
17056 & 102235 & 1982 & 94.28 & 0.23 & 9311.00 & 83624.68 & 1.01 & 0.89 & 0.90 \\
373 & 100046 & 1982 & 33.07 & 0.29 & 3347.00 & 31568.09 & 0.99 & 0.95 & 0.94 \\
1666 & 100223 & 1982 & 46.50 & 0.29 & 4090.00 & 39648.94 & 1.14 & 0.85 & 0.97 \\
1653 & 100222 & 1982 & 138.12 & -0.18 & 11393.00 & 123597.35 & 1.21 & 0.89 & 1.08 \\
1729 & 100227 & 1982 & 73.86 & 0.46 & 7386.00 & 71776.13 & 1.00 & 0.97 & 0.97 \\
7960 & 101068 & 1982 & 458.43 & 0.48 & 46054.00 & 397177.76 & 1.00 & 0.87 & 0.86 \\
443 & 100056 & 1982 & 33.12 & 0.28 & 3312.00 & 27718.42 & 1.00 & 0.84 & 0.84 \\
17799 & 102361 & 1982 & 7.95 & -0.04 & 868.00 & 8119.83 & 0.92 & 1.02 & 0.94 \\
20581 & 102774 & 1982 & 104.70 & 0.55 & 10003.00 & 81170.21 & 1.05 & 0.78 & 0.81 \\
20541 & 102767 & 1982 & 875.08 & 0.34 & 72176.00 & 871218.39 & 1.21 & 1.00 & 1.21 \\
13440 & 101740 & 1982 & 101.67 & 0.31 & 10167.00 & 81652.66 & 1.00 & 0.80 & 0.80 \\
21757 & 102951 & 1982 & 188.39 & 0.22 & 15897.00 & 176599.05 & 1.19 & 0.94 & 1.11 \\
15027 & 101953 & 1982 & 8.00 & 0.09 & 573.00 & 5497.68 & 1.40 & 0.69 & 0.96 \\
7374 & 101038 & 1982 & 360.70 & 0.37 & 36590.00 & 341729.86 & 0.99 & 0.95 & 0.93 \\
49088 & 240222 & 1982 & 223.81 & 0.25 & 22388.00 & 192061.29 & 1.00 & 0.86 & 0.86 \\
9713 & 101177 & 1982 & 7.25 & 0.44 & 503.00 & 5260.60 & 1.44 & 0.73 & 1.05 \\
13643 & 101754 & 1982 & 15.55 & 0.31 & 1555.00 & 14559.45 & 1.00 & 0.94 & 0.94 \\
10549 & 101299 & 1982 & 9.90 & 0.46 & 795.00 & 9111.85 & 1.25 & 0.92 & 1.15 \\
20262 & 102702 & 1982 & 27.87 & 0.34 & 2787.00 & 26288.48 & 1.00 & 0.94 & 0.94 \\
2591 & 100346 & 1982 & 2.32 & 0.13 & 232.00 & 2010.80 & 1.00 & 0.87 & 0.87 \\
8950 & 101107 & 1982 & 553.20 & 0.15 & 49000.00 & 512283.27 & 1.13 & 0.93 & 1.05 \\
46595 & 200259 & 1982 & 949.34 & 0.19 & 87884.00 & 787575.45 & 1.08 & 0.83 & 0.90 \\
26271 & 103551 & 1982 & 5.30 & 0.16 & 585.00 & 5025.94 & 0.91 & 0.95 & 0.86 \\
19293 & 102583 & 1982 & 46.30 & -0.08 & 4209.00 & 37510.21 & 1.10 & 0.81 & 0.89 \\
22663 & 103028 & 1982 & 25.71 & 0.23 & 1724.00 & 15614.36 & 1.49 & 0.61 & 0.91 \\
16090 & 102080 & 1982 & 58.55 & 0.13 & 5855.00 & 54324.64 & 1.00 & 0.93 & 0.93 \\
15157 & 101964 & 1982 & 16.78 & 0.39 & 1678.00 & 16048.97 & 1.00 & 0.96 & 0.96 \\
46585 & 200258 & 1982 & 180.21 & 0.36 & 16371.00 & 144751.65 & 1.10 & 0.80 & 0.88 \\
22359 & 103008 & 1982 & 32.21 & 0.35 & 3221.00 & 29063.80 & 1.00 & 0.90 & 0.90 \\
10444 & 101286 & 1982 & 59.44 & 0.30 & 5225.00 & 55658.78 & 1.14 & 0.94 & 1.07 \\
12737 & 101592 & 1982 & 29.60 & 0.41 & 2496.00 & 22217.22 & 1.19 & 0.75 & 0.89 \\
24823 & 103381 & 1982 & 113.00 & 0.30 & 11320.00 & 109215.84 & 1.00 & 0.97 & 0.96 \\
57810 & 401072 & 1982 & 33.90 & 0.25 & 3390.00 & 30516.63 & 1.00 & 0.90 & 0.90 \\
21733 & 102949 & 1982 & 170.57 & 0.29 & 14301.00 & 154740.93 & 1.19 & 0.91 & 1.08 \\
4921 & 100695 & 1982 & 49.97 & 0.21 & 4997.00 & 44428.98 & 1.00 & 0.89 & 0.89 \\
18787 & 102522 & 1983 & 182.23 & 0.29 & 18223.00 & 164761.58 & 1.00 & 0.90 & 0.90 \\
5589 & 100773 & 1983 & 393.00 & -0.01 & 39325.00 & 385515.28 & 1.00 & 0.98 & 0.98 \\
6362 & 100856 & 1983 & 63.91 & 0.17 & 6371.00 & 58388.03 & 1.00 & 0.91 & 0.92 \\
4461 & 100634 & 1983 & 145.49 & 0.21 & 14159.00 & 136933.36 & 1.03 & 0.94 & 0.97 \\
25671 & 103508 & 1983 & 130.04 & 0.20 & 13004.00 & 113276.99 & 1.00 & 0.87 & 0.87 \\
23480 & 103180 & 1983 & 37.20 & -0.05 & 3718.00 & 30355.88 & 1.00 & 0.82 & 0.82 \\
22169 & 102994 & 1983 & 36.63 & 0.28 & 3652.00 & 31461.19 & 1.00 & 0.86 & 0.86 \\
6314 & 100849 & 1983 & 47.72 & 0.10 & 4772.00 & 44413.71 & 1.00 & 0.93 & 0.93 \\
4365 & 100614 & 1983 & 67.50 & 0.16 & 6919.00 & 58669.21 & 0.98 & 0.87 & 0.85 \\
46596 & 200259 & 1983 & 1423.93 & 0.34 & 135154.00 & 1117240.11 & 1.05 & 0.78 & 0.83 \\
17800 & 102361 & 1983 & 8.05 & 0.10 & 986.00 & 7081.50 & 0.82 & 0.88 & 0.72 \\
22360 & 103008 & 1983 & 41.12 & 0.26 & 4111.00 & 35669.15 & 1.00 & 0.87 & 0.87 \\
5995 & 100818 & 1983 & 369.71 & 0.18 & 36770.00 & 346379.67 & 1.01 & 0.94 & 0.94 \\
4922 & 100695 & 1983 & 57.28 & 0.19 & 5728.00 & 51445.82 & 1.00 & 0.90 & 0.90 \\
9247 & 101123 & 1983 & 149.40 & 0.33 & 15316.00 & 128067.21 & 0.98 & 0.86 & 0.84 \\
16091 & 102080 & 1983 & 95.66 & 0.22 & 9425.00 & 87817.52 & 1.01 & 0.92 & 0.93 \\
17696 & 102346 & 1983 & 28.64 & 0.17 & 2863.00 & 24856.05 & 1.00 & 0.87 & 0.87 \\
17660 & 102342 & 1983 & 3.98 & 0.11 & 400.00 & 3923.62 & 0.99 & 0.99 & 0.98 \\
22396 & 103011 & 1983 & 30.28 & 0.05 & 3029.00 & 28122.50 & 1.00 & 0.93 & 0.93 \\
5039 & 100705 & 1983 & 73.83 & 0.19 & 8538.00 & 64869.86 & 0.86 & 0.88 & 0.76 \\
17533 & 102318 & 1983 & 163.69 & 0.24 & 16369.00 & 133201.11 & 1.00 & 0.81 & 0.81 \\
5854 & 100809 & 1983 & 1937.84 & 0.20 & 193800.00 & 1608305.67 & 1.00 & 0.83 & 0.83 \\
24957 & 103397 & 1983 & 11.10 & 0.12 & 698.00 & 6995.52 & 1.59 & 0.63 & 1.00 \\
16345 & 102130 & 1983 & 50.72 & 0.22 & 5071.00 & 47519.85 & 1.00 & 0.94 & 0.94 \\
11349 & 101398 & 1983 & 108.70 & 0.34 & 10878.00 & 103438.70 & 1.00 & 0.95 & 0.95 \\
5061 & 100714 & 1983 & 182.77 & 0.10 & 18622.00 & 179160.30 & 0.98 & 0.98 & 0.96 \\
17367 & 102284 & 1983 & 22.97 & 0.14 & 2517.00 & 20026.20 & 0.91 & 0.87 & 0.80 \\
8342 & 101084 & 1983 & 718.50 & 0.52 & 61249.00 & 628396.81 & 1.17 & 0.87 & 1.03 \\
22200 & 102996 & 1983 & 145.43 & 0.23 & 14169.00 & 123810.62 & 1.03 & 0.85 & 0.87 \\
15512 & 102000 & 1983 & 75.48 & 0.27 & 7550.00 & 70528.91 & 1.00 & 0.93 & 0.93 \\
4609 & 100644 & 1983 & 33.08 & 0.07 & 3308.00 & 30796.41 & 1.00 & 0.93 & 0.93 \\
22771 & 103061 & 1983 & 4.30 & -0.04 & 439.00 & 3535.56 & 0.98 & 0.82 & 0.81 \\
9708 & 101170 & 1983 & 5.20 & -0.01 & 516.00 & 4625.53 & 1.01 & 0.89 & 0.90 \\
6241 & 100833 & 1983 & 131.11 & 0.20 & 13110.00 & 120069.00 & 1.00 & 0.92 & 0.92 \\
15556 & 102005 & 1983 & 612.66 & 0.26 & 61455.00 & 549136.77 & 1.00 & 0.90 & 0.89 \\
18337 & 102441 & 1983 & 183.40 & 0.28 & 18080.00 & 146797.28 & 1.01 & 0.80 & 0.81 \\
1121 & 100155 & 1983 & 22.70 & 0.26 & 2270.00 & 21079.11 & 1.00 & 0.93 & 0.93 \\
5117 & 100726 & 1983 & 43.94 & 0.11 & 4601.00 & 41344.36 & 0.96 & 0.94 & 0.90 \\
2779 & 100358 & 1983 & 12.18 & 0.17 & 1442.00 & 11396.84 & 0.84 & 0.94 & 0.79 \\
23503 & 103183 & 1983 & 269.38 & 0.14 & 26939.00 & 228946.95 & 1.00 & 0.85 & 0.85 \\
15707 & 102016 & 1983 & 925.20 & 0.15 & 94505.00 & 842933.97 & 0.98 & 0.91 & 0.89 \\
1654 & 100222 & 1983 & 116.91 & 0.03 & 13713.00 & 113842.16 & 0.85 & 0.97 & 0.83 \\
2183 & 100295 & 1983 & 11.38 & 0.10 & 1138.00 & 9462.25 & 1.00 & 0.83 & 0.83 \\
17420 & 102306 & 1983 & 88.00 & 0.23 & 8610.00 & 82530.12 & 1.02 & 0.94 & 0.96 \\
18254 & 102419 & 1983 & 302.80 & 0.22 & 32182.00 & 252388.87 & 0.94 & 0.83 & 0.78 \\
24976 & 103406 & 1983 & 500.55 & 0.26 & 50055.00 & 425101.21 & 1.00 & 0.85 & 0.85 \\
16008 & 102065 & 1983 & 56.22 & -0.01 & 5622.00 & 55422.85 & 1.00 & 0.99 & 0.99 \\
213 & 100019 & 1983 & 174.10 & 0.25 & 17410.00 & 155842.68 & 1.00 & 0.90 & 0.90 \\
15845 & 102048 & 1983 & 217.22 & 0.16 & 21722.00 & 209356.96 & 1.00 & 0.96 & 0.96 \\
47363 & 210681 & 1983 & 86.20 & 0.23 & 8303.00 & 81722.98 & 1.04 & 0.95 & 0.98 \\
5495 & 100767 & 1983 & 62.00 & 0.21 & 6513.00 & 58247.51 & 0.95 & 0.94 & 0.89 \\
10401 & 101285 & 1983 & 43.87 & 0.30 & 2276.00 & 19997.56 & 1.93 & 0.46 & 0.88 \\
20542 & 102767 & 1983 & 975.67 & 0.11 & 107368.00 & 962822.08 & 0.91 & 0.99 & 0.90 \\
2592 & 100346 & 1983 & 2.56 & 0.17 & 300.00 & 2200.58 & 0.85 & 0.86 & 0.73 \\
7961 & 101068 & 1983 & 1606.00 & 0.39 & 160700.00 & 1440351.34 & 1.00 & 0.90 & 0.90 \\
20582 & 102774 & 1983 & 164.38 & 0.15 & 16497.00 & 164496.95 & 1.00 & 1.00 & 1.00 \\
102 & 100009 & 1983 & 4.27 & -0.06 & 427.00 & 3436.30 & 1.00 & 0.80 & 0.80 \\
444 & 100056 & 1983 & 34.58 & 0.13 & 3458.00 & 30250.75 & 1.00 & 0.87 & 0.87 \\
25182 & 103460 & 1983 & 80.97 & 0.24 & 8044.00 & 75577.62 & 1.01 & 0.93 & 0.94 \\
20693 & 102784 & 1983 & 218.40 & 0.15 & 21761.00 & 182241.98 & 1.00 & 0.83 & 0.84 \\
1392 & 100195 & 1983 & 30.70 & -0.01 & 3074.00 & 28704.76 & 1.00 & 0.94 & 0.93 \\
22664 & 103028 & 1983 & 116.73 & 0.23 & 11673.00 & 97957.78 & 1.00 & 0.84 & 0.84 \\
20263 & 102702 & 1983 & 37.31 & 0.29 & 3734.00 & 33936.42 & 1.00 & 0.91 & 0.91 \\
7375 & 101038 & 1983 & 540.40 & 0.38 & 54000.00 & 501759.39 & 1.00 & 0.93 & 0.93 \\
21758 & 102951 & 1983 & 258.00 & 0.13 & 25800.00 & 228513.99 & 1.00 & 0.89 & 0.89 \\
11697 & 101457 & 1983 & 63.49 & 0.13 & 6349.00 & 56805.59 & 1.00 & 0.89 & 0.89 \\
2977 & 100395 & 1983 & 238.10 & 0.24 & 23810.00 & 236061.98 & 1.00 & 0.99 & 0.99 \\
14005 & 101800 & 1983 & 141.90 & 0.26 & 11766.00 & 114376.39 & 1.21 & 0.81 & 0.97 \\
17213 & 102271 & 1983 & 176.37 & 0.11 & 17293.00 & 161694.71 & 1.02 & 0.92 & 0.94 \\
1332 & 100190 & 1983 & 613.10 & 0.21 & 61310.00 & 565494.01 & 1.00 & 0.92 & 0.92 \\
22620 & 103027 & 1983 & 137.90 & -0.04 & 13796.00 & 124420.02 & 1.00 & 0.90 & 0.90 \\
19933 & 102659 & 1983 & 1355.43 & 0.20 & 135543.00 & 1215048.91 & 1.00 & 0.90 & 0.90 \\
8951 & 101107 & 1983 & 637.49 & 0.27 & 64000.00 & 630630.63 & 1.00 & 0.99 & 0.99 \\
13644 & 101754 & 1983 & 18.30 & 0.16 & 1814.00 & 15551.22 & 1.01 & 0.85 & 0.86 \\
24824 & 103381 & 1983 & 536.61 & 0.27 & 54099.00 & 540881.59 & 0.99 & 1.01 & 1.00 \\
49046 & 240212 & 1983 & 294.97 & -0.02 & 35314.00 & 292767.76 & 0.84 & 0.99 & 0.83 \\
12933 & 101616 & 1983 & 355.32 & 0.20 & 35530.00 & 289791.59 & 1.00 & 0.82 & 0.82 \\
374 & 100046 & 1983 & 31.72 & 0.16 & 3264.00 & 29761.16 & 0.97 & 0.94 & 0.91 \\
3381 & 100430 & 1983 & 76.10 & 0.15 & 7310.00 & 70121.50 & 1.04 & 0.92 & 0.96 \\
12898 & 101606 & 1983 & 630.92 & 0.13 & 63090.00 & 595015.61 & 1.00 & 0.94 & 0.94 \\
17023 & 102231 & 1983 & 717.85 & 0.15 & 75161.00 & 697896.51 & 0.96 & 0.97 & 0.93 \\
25264 & 103464 & 1983 & 111.79 & 0.16 & 11175.00 & 103366.13 & 1.00 & 0.92 & 0.92 \\
12720 & 101591 & 1983 & 8.20 & 0.00 & 819.00 & 6848.20 & 1.00 & 0.84 & 0.84 \\
3168 & 100412 & 1983 & 4.90 & 0.14 & 497.00 & 4803.13 & 0.99 & 0.98 & 0.97 \\
1730 & 100227 & 1983 & 100.92 & 0.18 & 10092.00 & 100406.56 & 1.00 & 0.99 & 0.99 \\
1463 & 100207 & 1983 & 1151.75 & 0.19 & 115175.00 & 975582.65 & 1.00 & 0.85 & 0.85 \\
24551 & 103352 & 1983 & 49.30 & 0.16 & 5127.00 & 42362.41 & 0.96 & 0.86 & 0.83 \\
5429 & 100763 & 1983 & 73.55 & 0.38 & 7350.00 & 61061.82 & 1.00 & 0.83 & 0.83 \\
12865 & 101603 & 1983 & 381.75 & 0.14 & 38180.00 & 351865.13 & 1.00 & 0.92 & 0.92 \\
3888 & 100510 & 1983 & 7.96 & 0.21 & 796.00 & 7455.13 & 1.00 & 0.94 & 0.94 \\
12983 & 101618 & 1983 & 164.12 & 0.18 & 16410.00 & 150805.15 & 1.00 & 0.92 & 0.92 \\
20751 & 102789 & 1983 & 13.50 & 0.39 & 1153.00 & 9751.76 & 1.17 & 0.72 & 0.85 \\
26272 & 103551 & 1983 & 6.58 & 0.19 & 658.00 & 5777.66 & 1.00 & 0.88 & 0.88 \\
128 & 100010 & 1983 & 155.70 & 0.21 & 15570.00 & 140993.12 & 1.00 & 0.91 & 0.91 \\
22927 & 103089 & 1983 & 65.67 & 0.19 & 6795.00 & 61548.19 & 0.97 & 0.94 & 0.91 \\
17057 & 102235 & 1983 & 86.68 & 0.05 & 9669.00 & 88594.20 & 0.90 & 1.02 & 0.92 \\
10594 & 101300 & 1983 & 26.59 & 0.16 & 2022.00 & 17362.91 & 1.32 & 0.65 & 0.86 \\
1501 & 100209 & 1983 & 410.01 & 0.15 & 41001.00 & 329843.22 & 1.00 & 0.80 & 0.80 \\
17464 & 102307 & 1983 & 31.56 & 0.23 & 2698.00 & 26128.83 & 1.17 & 0.83 & 0.97 \\
14098 & 101804 & 1983 & 155.67 & 0.25 & 14487.00 & 144317.15 & 1.07 & 0.93 & 1.00 \\
8260 & 101081 & 1983 & 16.50 & 0.37 & 1273.00 & 12784.12 & 1.30 & 0.77 & 1.00 \\
24865 & 103383 & 1983 & 29.60 & 0.02 & 2137.00 & 21356.42 & 1.39 & 0.72 & 1.00 \\
6731 & 100947 & 1983 & 221.84 & 0.30 & 22184.00 & 209659.34 & 1.00 & 0.95 & 0.95 \\
10030 & 101256 & 1983 & 18.94 & 0.05 & 2581.00 & 25908.00 & 0.73 & 1.37 & 1.00 \\
11939 & 101473 & 1983 & 51.12 & 0.28 & 5112.00 & 42298.46 & 1.00 & 0.83 & 0.83 \\
2285 & 100313 & 1983 & 150.79 & 0.24 & 20330.00 & 170447.09 & 0.74 & 1.13 & 0.84 \\
12048 & 101494 & 1983 & 53.02 & 0.16 & 4989.00 & 43690.26 & 1.06 & 0.82 & 0.88 \\
2937 & 100389 & 1983 & 18.98 & 0.28 & 1898.00 & 18783.18 & 1.00 & 0.99 & 0.99 \\
16830 & 102197 & 1983 & 10.32 & -0.01 & 1032.00 & 9953.64 & 1.00 & 0.96 & 0.96 \\
14936 & 101925 & 1983 & 398.19 & 0.00 & 39776.00 & 341514.55 & 1.00 & 0.86 & 0.86 \\
1581 & 100217 & 1983 & 57.67 & 0.17 & 5225.00 & 49067.82 & 1.10 & 0.85 & 0.94 \\
17279 & 102278 & 1983 & 69.46 & 0.17 & 5286.00 & 58687.76 & 1.31 & 0.84 & 1.11 \\
10445 & 101286 & 1983 & 62.25 & 0.19 & 6782.00 & 55599.51 & 0.92 & 0.89 & 0.82 \\
9735 & 101181 & 1983 & 95.10 & 0.28 & 10851.00 & 87863.23 & 0.88 & 0.92 & 0.81 \\
1193 & 100160 & 1983 & 32.13 & 0.08 & 3213.00 & 26986.32 & 1.00 & 0.84 & 0.84 \\
640 & 100087 & 1983 & 117.30 & 0.29 & 11730.00 & 114088.55 & 1.00 & 0.97 & 0.97 \\
8301 & 101082 & 1983 & 81.80 & 0.14 & 10432.00 & 103654.18 & 0.78 & 1.27 & 0.99 \\
24708 & 103376 & 1983 & 673.83 & 0.24 & 67804.00 & 677749.13 & 0.99 & 1.01 & 1.00 \\
15158 & 101964 & 1983 & 18.97 & 0.16 & 1897.00 & 18381.43 & 1.00 & 0.97 & 0.97 \\
2933 & 100380 & 1983 & 20.60 & 0.37 & 2051.00 & 20394.43 & 1.00 & 0.99 & 0.99 \\
23461 & 103179 & 1983 & 292.50 & 0.12 & 29250.00 & 255844.22 & 1.00 & 0.87 & 0.87 \\
6620 & 100906 & 1983 & 358.97 & 0.22 & 35900.00 & 300469.07 & 1.00 & 0.84 & 0.84 \\
25612 & 103498 & 1983 & 337.95 & 0.19 & 33795.00 & 322095.53 & 1.00 & 0.95 & 0.95 \\
57811 & 401072 & 1983 & 55.04 & 0.21 & 5504.00 & 46157.44 & 1.00 & 0.84 & 0.84 \\
8192 & 101079 & 1983 & 80.25 & 0.20 & 9865.00 & 94202.56 & 0.81 & 1.17 & 0.95 \\
3998 & 100538 & 1983 & 68.09 & 0.15 & 6853.00 & 61325.37 & 0.99 & 0.90 & 0.89 \\
3955 & 100535 & 1983 & 74.43 & 0.21 & 7390.00 & 71291.40 & 1.01 & 0.96 & 0.96 \\
962 & 100113 & 1983 & 1123.45 & 0.10 & 112345.00 & 1051392.80 & 1.00 & 0.94 & 0.94 \\
23327 & 103164 & 1983 & 11.24 & 0.07 & 1377.00 & 10593.88 & 0.82 & 0.94 & 0.77 \\
3951 & 100531 & 1983 & 22.00 & 0.12 & 2195.00 & 19325.74 & 1.00 & 0.88 & 0.88 \\
14148 & 101819 & 1983 & 43.10 & 0.02 & 3865.00 & 35219.51 & 1.12 & 0.82 & 0.91 \\
26201 & 103546 & 1983 & 1719.46 & 0.30 & 172710.00 & 1423391.96 & 1.00 & 0.83 & 0.82 \\
14115 & 101805 & 1983 & 610.29 & 0.24 & 70666.00 & 567174.63 & 0.86 & 0.93 & 0.80 \\
10180 & 101268 & 1983 & 47.32 & 0.19 & 4879.00 & 39241.72 & 0.97 & 0.83 & 0.80 \\
3902 & 100514 & 1983 & 69.27 & 0.22 & 7247.00 & 68558.36 & 0.96 & 0.99 & 0.95 \\
22881 & 103077 & 1983 & 16.29 & 0.22 & 1628.00 & 15502.95 & 1.00 & 0.95 & 0.95 \\
2297 & 100315 & 1983 & 230.98 & 0.16 & 23098.00 & 219532.60 & 1.00 & 0.95 & 0.95 \\
10056 & 101257 & 1983 & 182.73 & 0.07 & 23683.00 & 237629.65 & 0.77 & 1.30 & 1.00 \\
19823 & 102653 & 1983 & 1013.08 & 0.12 & 101308.00 & 950786.15 & 1.00 & 0.94 & 0.94 \\
10550 & 101299 & 1983 & 18.75 & 0.18 & 1213.00 & 10020.02 & 1.55 & 0.53 & 0.83 \\
14540 & 101876 & 1983 & 111.67 & 0.29 & 13020.00 & 117599.92 & 0.86 & 1.05 & 0.90 \\
554 & 100076 & 1983 & 192.43 & 0.20 & 19822.00 & 170492.70 & 0.97 & 0.89 & 0.86 \\
25480 & 103494 & 1983 & 203.26 & -0.01 & 20326.00 & 202675.91 & 1.00 & 1.00 & 1.00 \\
14495 & 101865 & 1983 & 41.58 & 0.16 & 4757.00 & 47473.09 & 0.87 & 1.14 & 1.00 \\
23530 & 103184 & 1984 & 151.90 & -0.01 & 15207.00 & 125587.47 & 1.00 & 0.83 & 0.83 \\
17214 & 102271 & 1984 & 177.49 & -0.06 & 16474.00 & 171085.22 & 1.08 & 0.96 & 1.04 \\
24669 & 103375 & 1984 & 1.90 & 0.02 & 189.00 & 1894.31 & 1.01 & 1.00 & 1.00 \\
26202 & 103546 & 1984 & 2521.52 & 0.19 & 252150.00 & 2127687.14 & 1.00 & 0.84 & 0.84 \\
963 & 100113 & 1984 & 837.32 & -0.06 & 83732.00 & 791294.71 & 1.00 & 0.95 & 0.95 \\
5232 & 100741 & 1984 & 23.90 & -0.12 & 2600.00 & 20993.29 & 0.92 & 0.88 & 0.81 \\
1655 & 100222 & 1984 & 84.99 & -0.21 & 8176.00 & 80768.83 & 1.04 & 0.95 & 0.99 \\
5118 & 100726 & 1984 & 37.90 & -0.08 & 3940.00 & 34588.06 & 0.96 & 0.91 & 0.88 \\
17421 & 102306 & 1984 & 93.78 & 0.05 & 7121.00 & 79802.88 & 1.32 & 0.85 & 1.12 \\
22397 & 103011 & 1984 & 30.22 & 0.02 & 3024.00 & 28256.12 & 1.00 & 0.93 & 0.93 \\
17280 & 102278 & 1984 & 65.99 & -0.01 & 6178.00 & 66423.59 & 1.07 & 1.01 & 1.08 \\
1668 & 100223 & 1984 & 52.70 & 0.15 & 4394.00 & 41799.15 & 1.20 & 0.79 & 0.95 \\
5159 & 100730 & 1984 & 167.60 & -0.15 & 18000.00 & 135029.34 & 0.93 & 0.81 & 0.75 \\
22621 & 103027 & 1984 & 108.65 & -0.12 & 10864.00 & 106616.02 & 1.00 & 0.98 & 0.98 \\
17368 & 102284 & 1984 & 18.15 & -0.10 & 1977.00 & 15145.39 & 0.92 & 0.83 & 0.77 \\
17058 & 102235 & 1984 & 65.90 & -0.14 & 6586.00 & 52971.35 & 1.00 & 0.80 & 0.80 \\
17024 & 102231 & 1984 & 567.33 & -0.09 & 59625.00 & 572788.58 & 0.95 & 1.01 & 0.96 \\
5062 & 100714 & 1984 & 145.77 & -0.00 & 16209.00 & 136454.27 & 0.90 & 0.94 & 0.84 \\
20264 & 102702 & 1984 & 41.53 & 0.02 & 4153.00 & 35041.15 & 1.00 & 0.84 & 0.84 \\
25171 & 103451 & 1984 & 7.10 & -0.00 & 1268.00 & 10359.89 & 0.56 & 1.46 & 0.82 \\
20221 & 102689 & 1984 & 6.54 & -0.17 & 654.00 & 5653.28 & 1.00 & 0.86 & 0.86 \\
19934 & 102659 & 1984 & 1273.00 & -0.10 & 127303.00 & 1106451.94 & 1.00 & 0.87 & 0.87 \\
1333 & 100190 & 1984 & 558.81 & -0.05 & 55881.00 & 517546.86 & 1.00 & 0.93 & 0.93 \\
21797 & 102952 & 1984 & 210.28 & -0.10 & 21028.00 & 179053.48 & 1.00 & 0.85 & 0.85 \\
23234 & 103152 & 1984 & 175.20 & -0.02 & 17020.00 & 167376.69 & 1.03 & 0.96 & 0.98 \\
20543 & 102767 & 1984 & 1015.86 & 0.00 & 92659.00 & 921167.03 & 1.10 & 0.91 & 0.99 \\
2318 & 100319 & 1984 & 7.82 & -0.05 & 784.00 & 7709.99 & 1.00 & 0.99 & 0.98 \\
22882 & 103077 & 1984 & 15.31 & -0.04 & 1531.00 & 14827.19 & 1.00 & 0.97 & 0.97 \\
3903 & 100514 & 1984 & 65.64 & -0.06 & 6564.00 & 62936.47 & 1.00 & 0.96 & 0.96 \\
19906 & 102655 & 1984 & 472.00 & 0.03 & 47199.00 & 406556.17 & 1.00 & 0.86 & 0.86 \\
2298 & 100315 & 1984 & 211.47 & 0.00 & 21147.00 & 206737.00 & 1.00 & 0.98 & 0.98 \\
22665 & 103028 & 1984 & 114.00 & 0.02 & 11494.00 & 101916.24 & 0.99 & 0.89 & 0.89 \\
3956 & 100535 & 1984 & 76.92 & 0.15 & 6546.00 & 76000.80 & 1.18 & 0.99 & 1.16 \\
25481 & 103494 & 1984 & 167.31 & -0.15 & 16650.00 & 166301.50 & 1.00 & 0.99 & 1.00 \\
3889 & 100510 & 1984 & 8.47 & -0.16 & 847.00 & 8116.16 & 1.00 & 0.96 & 0.96 \\
21735 & 102949 & 1984 & 236.35 & 0.00 & 23630.00 & 191320.05 & 1.00 & 0.81 & 0.81 \\
20583 & 102774 & 1984 & 152.29 & -0.11 & 14541.00 & 151278.67 & 1.05 & 0.99 & 1.04 \\
22928 & 103089 & 1984 & 55.48 & -0.03 & 5487.00 & 52099.44 & 1.01 & 0.94 & 0.95 \\
3169 & 100412 & 1984 & 5.80 & 0.02 & 501.00 & 4435.23 & 1.16 & 0.76 & 0.89 \\
1464 & 100207 & 1984 & 1236.53 & -0.01 & 123653.00 & 1000651.37 & 1.00 & 0.81 & 0.81 \\
1731 & 100227 & 1984 & 86.23 & -0.16 & 8623.00 & 81870.20 & 1.00 & 0.95 & 0.95 \\
25183 & 103460 & 1984 & 94.61 & -0.18 & 9461.00 & 86477.62 & 1.00 & 0.91 & 0.91 \\
3382 & 100430 & 1984 & 62.67 & -0.05 & 6270.00 & 59216.05 & 1.00 & 0.94 & 0.94 \\
25265 & 103464 & 1984 & 129.73 & -0.07 & 12973.00 & 117812.85 & 1.00 & 0.91 & 0.91 \\
1502 & 100209 & 1984 & 376.62 & -0.06 & 37662.00 & 319125.50 & 1.00 & 0.85 & 0.85 \\
24825 & 103381 & 1984 & 412.01 & -0.19 & 41200.00 & 403140.27 & 1.00 & 0.98 & 0.98 \\
1393 & 100195 & 1984 & 24.10 & -0.15 & 2411.00 & 23686.58 & 1.00 & 0.98 & 0.98 \\
20694 & 102784 & 1984 & 223.56 & -0.06 & 19188.00 & 172378.48 & 1.17 & 0.77 & 0.90 \\
19824 & 102653 & 1984 & 965.50 & -0.01 & 95972.00 & 880454.34 & 1.01 & 0.91 & 0.92 \\
17465 & 102307 & 1984 & 29.92 & -0.08 & 2990.00 & 24460.99 & 1.00 & 0.82 & 0.82 \\
3999 & 100538 & 1984 & 64.69 & -0.10 & 6115.00 & 56740.71 & 1.06 & 0.88 & 0.93 \\
2938 & 100389 & 1984 & 17.01 & -0.08 & 1680.00 & 16796.06 & 1.01 & 0.99 & 1.00 \\
22201 & 102996 & 1984 & 145.89 & -0.15 & 15510.00 & 151219.30 & 0.94 & 1.04 & 0.97 \\
22229 & 102997 & 1984 & 44.22 & -0.11 & 4815.00 & 42962.91 & 0.92 & 0.97 & 0.89 \\
2780 & 100358 & 1984 & 9.03 & -0.20 & 902.00 & 8874.79 & 1.00 & 0.98 & 0.98 \\
23504 & 103183 & 1984 & 215.20 & -0.11 & 21655.00 & 185918.20 & 0.99 & 0.86 & 0.86 \\
4792 & 100682 & 1984 & 7.30 & 0.24 & 736.00 & 5918.49 & 0.99 & 0.81 & 0.80 \\
4923 & 100695 & 1984 & 57.49 & 0.01 & 5749.00 & 51171.72 & 1.00 & 0.89 & 0.89 \\
22361 & 103008 & 1984 & 38.94 & -0.06 & 3894.00 & 34722.43 & 1.00 & 0.89 & 0.89 \\
17661 & 102342 & 1984 & 4.06 & 0.04 & 405.00 & 4020.73 & 1.00 & 0.99 & 0.99 \\
4610 & 100644 & 1984 & 28.71 & -0.06 & 2871.00 & 24171.76 & 1.00 & 0.84 & 0.84 \\
2286 & 100313 & 1984 & 86.95 & -0.15 & 8695.00 & 86064.53 & 1.00 & 0.99 & 0.99 \\
25538 & 103496 & 1984 & 181.09 & -0.09 & 18150.00 & 176095.42 & 1.00 & 0.97 & 0.97 \\
19295 & 102583 & 1984 & 36.60 & 0.07 & 3572.00 & 31892.21 & 1.02 & 0.87 & 0.89 \\
19206 & 102570 & 1984 & 52.89 & 0.08 & 5246.00 & 43860.51 & 1.01 & 0.83 & 0.84 \\
25613 & 103498 & 1984 & 235.62 & -0.26 & 24020.00 & 217256.15 & 0.98 & 0.92 & 0.90 \\
4334 & 100609 & 1984 & 8.00 & 0.01 & 796.00 & 7588.92 & 1.01 & 0.95 & 0.95 \\
1207 & 100166 & 1984 & 1028.96 & -0.01 & 102900.00 & 916116.35 & 1.00 & 0.89 & 0.89 \\
24709 & 103376 & 1984 & 664.52 & -0.11 & 66308.00 & 663496.35 & 1.00 & 1.00 & 1.00 \\
1194 & 100160 & 1984 & 28.66 & 0.05 & 2866.00 & 25997.56 & 1.00 & 0.91 & 0.91 \\
4366 & 100614 & 1984 & 64.53 & 0.00 & 6453.00 & 59203.92 & 1.00 & 0.92 & 0.92 \\
18788 & 102522 & 1984 & 171.34 & 0.01 & 17133.00 & 146727.64 & 1.00 & 0.86 & 0.86 \\
24866 & 103383 & 1984 & 105.16 & -0.16 & 9824.00 & 98323.63 & 1.07 & 0.93 & 1.00 \\
22170 & 102994 & 1984 & 32.65 & -0.05 & 3329.00 & 29812.64 & 0.98 & 0.91 & 0.90 \\
22772 & 103061 & 1984 & 2.80 & -0.32 & 283.00 & 2256.23 & 0.99 & 0.81 & 0.80 \\
4462 & 100634 & 1984 & 155.14 & -0.06 & 15514.00 & 151203.92 & 1.00 & 0.97 & 0.97 \\
2593 & 100346 & 1984 & 2.53 & 0.04 & 253.00 & 2150.39 & 1.00 & 0.85 & 0.85 \\
18305 & 102425 & 1984 & 745.60 & 0.03 & 74601.00 & 653590.81 & 1.00 & 0.88 & 0.88 \\
74631 & 601143 & 1984 & 181.30 & -0.13 & 18881.00 & 166795.32 & 0.96 & 0.92 & 0.88 \\
15557 & 102005 & 1984 & 649.55 & -0.11 & 65868.00 & 593011.58 & 0.99 & 0.91 & 0.90 \\
11940 & 101473 & 1984 & 58.10 & 0.13 & 5810.00 & 49396.47 & 1.00 & 0.85 & 0.85 \\
7376 & 101038 & 1984 & 643.40 & 0.05 & 64300.00 & 537006.79 & 1.00 & 0.83 & 0.84 \\
11698 & 101457 & 1984 & 58.39 & -0.05 & 5839.00 & 52475.34 & 1.00 & 0.90 & 0.90 \\
8302 & 101082 & 1984 & 60.80 & -0.01 & 6100.00 & 58461.43 & 1.00 & 0.96 & 0.96 \\
16346 & 102130 & 1984 & 65.36 & 0.08 & 6544.00 & 53257.41 & 1.00 & 0.81 & 0.81 \\
13645 & 101754 & 1984 & 19.02 & 0.07 & 1902.00 & 15339.87 & 1.00 & 0.81 & 0.81 \\
11632 & 101455 & 1984 & 486.07 & -0.06 & 48610.00 & 417808.67 & 1.00 & 0.86 & 0.86 \\
8343 & 101084 & 1984 & 543.70 & -0.17 & 54400.00 & 517097.63 & 1.00 & 0.95 & 0.95 \\
15513 & 102000 & 1984 & 79.40 & 0.07 & 7941.00 & 72906.71 & 1.00 & 0.92 & 0.92 \\
11584 & 101431 & 1984 & 8.30 & 0.04 & 837.00 & 7958.14 & 0.99 & 0.96 & 0.95 \\
14116 & 101805 & 1984 & 558.32 & -0.21 & 46676.00 & 396468.85 & 1.20 & 0.71 & 0.85 \\
11350 & 101398 & 1984 & 105.70 & -0.03 & 10569.00 & 94417.12 & 1.00 & 0.89 & 0.89 \\
14099 & 101804 & 1984 & 176.96 & -0.09 & 19675.00 & 186967.49 & 0.90 & 1.06 & 0.95 \\
11490 & 101425 & 1984 & 16.45 & 0.03 & 1926.00 & 17942.45 & 0.85 & 1.09 & 0.93 \\
8381 & 101085 & 1984 & 42.10 & 0.21 & 4200.00 & 34080.62 & 1.00 & 0.81 & 0.81 \\
13851 & 101781 & 1984 & 180.47 & -0.08 & 18047.00 & 151719.80 & 1.00 & 0.84 & 0.84 \\
11551 & 101430 & 1984 & 3.70 & 0.04 & 365.00 & 3339.26 & 1.01 & 0.90 & 0.91 \\
10595 & 101300 & 1984 & 32.43 & 0.21 & 3243.00 & 30902.28 & 1.00 & 0.95 & 0.95 \\
8193 & 101079 & 1984 & 42.00 & 0.02 & 4300.00 & 37889.91 & 0.98 & 0.90 & 0.88 \\
445 & 100056 & 1984 & 24.95 & -0.20 & 2495.00 & 20245.68 & 1.00 & 0.81 & 0.81 \\
12984 & 101618 & 1984 & 152.56 & -0.04 & 15400.00 & 124719.54 & 0.99 & 0.82 & 0.81 \\
375 & 100046 & 1984 & 29.88 & 0.05 & 2971.00 & 27031.31 & 1.01 & 0.90 & 0.91 \\
12899 & 101606 & 1984 & 593.32 & -0.19 & 59300.00 & 539774.95 & 1.00 & 0.91 & 0.91 \\
16009 & 102065 & 1984 & 54.72 & -0.12 & 5472.00 & 53446.67 & 1.00 & 0.98 & 0.98 \\
12866 & 101603 & 1984 & 315.72 & -0.06 & 31940.00 & 275104.23 & 0.99 & 0.87 & 0.86 \\
15846 & 102048 & 1984 & 182.10 & -0.03 & 18210.00 & 176529.80 & 1.00 & 0.97 & 0.97 \\
15708 & 102016 & 1984 & 828.13 & -0.04 & 77392.00 & 740798.80 & 1.07 & 0.89 & 0.96 \\
46597 & 200259 & 1984 & 1727.84 & 0.05 & 127000.00 & 1197557.16 & 1.36 & 0.69 & 0.94 \\
12721 & 101591 & 1984 & 15.91 & -0.06 & 1655.00 & 14165.35 & 0.96 & 0.89 & 0.86 \\
12049 & 101494 & 1984 & 55.34 & -0.02 & 5226.00 & 45752.79 & 1.06 & 0.83 & 0.88 \\
16092 & 102080 & 1984 & 80.75 & -0.08 & 8766.00 & 79895.51 & 0.92 & 0.99 & 0.91 \\
12034 & 101491 & 1984 & 77.52 & -0.04 & 7752.00 & 63992.37 & 1.00 & 0.83 & 0.83 \\
15663 & 102013 & 1984 & 78.70 & -0.14 & 7715.00 & 73048.28 & 1.02 & 0.93 & 0.95 \\
7962 & 101068 & 1984 & 1954.40 & 0.02 & 195400.00 & 1578035.30 & 1.00 & 0.81 & 0.81 \\
15029 & 101953 & 1984 & 13.00 & 0.16 & 1295.00 & 10994.85 & 1.00 & 0.85 & 0.85 \\
49047 & 240212 & 1984 & 226.57 & -0.03 & 25917.00 & 230476.28 & 0.87 & 1.02 & 0.89 \\
13337 & 101729 & 1984 & 63.51 & 0.08 & 6353.00 & 50861.99 & 1.00 & 0.80 & 0.80 \\
49090 & 240222 & 1984 & 229.48 & -0.01 & 22954.00 & 192196.12 & 1.00 & 0.84 & 0.84 \\
6315 & 100849 & 1984 & 43.50 & -0.02 & 4350.00 & 39255.12 & 1.00 & 0.90 & 0.90 \\
214 & 100019 & 1984 & 210.98 & 0.04 & 21098.00 & 172508.69 & 1.00 & 0.82 & 0.82 \\
6819 & 100962 & 1984 & 248.82 & -0.19 & 24880.00 & 205795.81 & 1.00 & 0.83 & 0.83 \\
16952 & 102224 & 1984 & 32.29 & -0.25 & 3319.00 & 32089.29 & 0.97 & 0.99 & 0.97 \\
14149 & 101819 & 1984 & 52.95 & 0.11 & 4978.00 & 45638.40 & 1.06 & 0.86 & 0.92 \\
47236 & 200344 & 1984 & 637.66 & -0.03 & 63766.00 & 510301.97 & 1.00 & 0.80 & 0.80 \\
10242 & 101276 & 1984 & 37.16 & -0.08 & 3716.00 & 30258.42 & 1.00 & 0.81 & 0.81 \\
9736 & 101181 & 1984 & 68.29 & -0.27 & 7586.00 & 58353.99 & 0.90 & 0.85 & 0.77 \\
9248 & 101123 & 1984 & 121.60 & -0.12 & 12020.00 & 100747.04 & 1.01 & 0.83 & 0.84 \\
103 & 100009 & 1984 & 9.42 & -0.01 & 919.00 & 8017.34 & 1.03 & 0.85 & 0.87 \\
8952 & 101107 & 1984 & 601.80 & -0.02 & 60200.00 & 573024.17 & 1.00 & 0.95 & 0.95 \\
16831 & 102197 & 1984 & 10.05 & 0.02 & 1010.00 & 9118.92 & 1.00 & 0.91 & 0.90 \\
555 & 100076 & 1984 & 177.92 & -0.02 & 17792.00 & 158995.47 & 1.00 & 0.89 & 0.89 \\
14937 & 101925 & 1984 & 317.31 & -0.01 & 33267.00 & 294427.32 & 0.95 & 0.93 & 0.89 \\
6732 & 100947 & 1984 & 242.34 & -0.00 & 24232.00 & 219018.06 & 1.00 & 0.90 & 0.90 \\
14541 & 101876 & 1984 & 99.96 & -0.04 & 10058.00 & 92563.26 & 0.99 & 0.93 & 0.92 \\
9709 & 101170 & 1984 & 3.00 & -0.28 & 304.00 & 2494.71 & 0.99 & 0.83 & 0.82 \\
641 & 100087 & 1984 & 119.33 & -0.06 & 11933.00 & 111428.28 & 1.00 & 0.93 & 0.93 \\
15159 & 101964 & 1984 & 17.30 & -0.02 & 1730.00 & 16889.23 & 1.00 & 0.98 & 0.98 \\
47364 & 210681 & 1984 & 79.60 & -0.08 & 8035.00 & 76994.50 & 0.99 & 0.97 & 0.96 \\
10551 & 101299 & 1984 & 20.41 & 0.01 & 2040.00 & 18591.76 & 1.00 & 0.91 & 0.91 \\
16861 & 102213 & 1984 & 18.82 & -0.03 & 1884.00 & 16573.87 & 1.00 & 0.88 & 0.88 \\
5590 & 100773 & 1984 & 577.60 & -0.07 & 57760.00 & 521665.11 & 1.00 & 0.90 & 0.90 \\
5496 & 100767 & 1984 & 49.30 & -0.07 & 5193.00 & 48201.17 & 0.95 & 0.98 & 0.93 \\
7030 & 100992 & 1984 & 3.60 & 0.06 & 360.00 & 3428.71 & 1.00 & 0.95 & 0.95 \\
10446 & 101286 & 1984 & 70.14 & 0.11 & 7006.00 & 65497.05 & 1.00 & 0.93 & 0.93 \\
14432 & 101858 & 1984 & 95.28 & -0.04 & 10528.00 & 85715.71 & 0.90 & 0.90 & 0.81 \\
6363 & 100856 & 1984 & 60.76 & 0.02 & 6076.00 & 53883.02 & 1.00 & 0.89 & 0.89 \\
46209 & 200199 & 1984 & 363.18 & 0.04 & 36320.00 & 317220.80 & 1.00 & 0.87 & 0.87 \\
14859 & 101919 & 1984 & 294.22 & -0.22 & 31028.00 & 266838.91 & 0.95 & 0.91 & 0.86 \\
1394 & 100195 & 1985 & 25.23 & 0.19 & 2521.00 & 23126.98 & 1.00 & 0.92 & 0.92 \\
6733 & 100947 & 1985 & 440.93 & 0.35 & 44098.00 & 378651.22 & 1.00 & 0.86 & 0.86 \\
14150 & 101819 & 1985 & 86.51 & 0.33 & 7116.00 & 59616.15 & 1.22 & 0.69 & 0.84 \\
74632 & 601143 & 1985 & 151.80 & 0.18 & 17492.00 & 166040.29 & 0.87 & 1.09 & 0.95 \\
1208 & 100166 & 1985 & 1313.04 & 0.22 & 131304.00 & 1108276.10 & 1.00 & 0.84 & 0.84 \\
12985 & 101618 & 1985 & 183.75 & 0.32 & 18380.00 & 150702.77 & 1.00 & 0.82 & 0.82 \\
4335 & 100609 & 1985 & 8.93 & 0.26 & 893.00 & 7222.87 & 1.00 & 0.81 & 0.81 \\
25266 & 103464 & 1985 & 156.88 & 0.20 & 15688.00 & 144267.07 & 1.00 & 0.92 & 0.92 \\
20832 & 102796 & 1985 & 1.74 & 0.21 & 174.00 & 1439.13 & 1.00 & 0.83 & 0.83 \\
1404 & 100196 & 1985 & 150.97 & 0.16 & 15097.00 & 135729.35 & 1.00 & 0.90 & 0.90 \\
13206 & 101704 & 1985 & 124.22 & 0.23 & 12422.00 & 115339.96 & 1.00 & 0.93 & 0.93 \\
25614 & 103498 & 1985 & 243.05 & 0.21 & 24300.00 & 225255.56 & 1.00 & 0.93 & 0.93 \\
13251 & 101714 & 1985 & 48.40 & 0.19 & 4841.00 & 42029.34 & 1.00 & 0.87 & 0.87 \\
20753 & 102789 & 1985 & 41.40 & 0.27 & 3337.00 & 26998.42 & 1.24 & 0.65 & 0.81 \\
19207 & 102570 & 1985 & 74.41 & 0.27 & 7338.00 & 63037.79 & 1.01 & 0.85 & 0.86 \\
14860 & 101919 & 1985 & 299.29 & 0.15 & 26622.00 & 286306.82 & 1.12 & 0.96 & 1.08 \\
13020 & 101622 & 1985 & 51.91 & 0.47 & 6640.00 & 66735.99 & 0.78 & 1.29 & 1.01 \\
14938 & 101925 & 1985 & 283.70 & 0.35 & 27989.00 & 270075.47 & 1.01 & 0.95 & 0.96 \\
20695 & 102784 & 1985 & 307.38 & 0.37 & 28287.00 & 284846.76 & 1.09 & 0.93 & 1.01 \\
14433 & 101858 & 1985 & 115.63 & 0.24 & 12264.00 & 112260.60 & 0.94 & 0.97 & 0.92 \\
13825 & 101769 & 1985 & 275.13 & 0.27 & 27513.00 & 255982.79 & 1.00 & 0.93 & 0.93 \\
1334 & 100190 & 1985 & 656.84 & 0.30 & 65684.00 & 626801.14 & 1.00 & 0.95 & 0.95 \\
3957 & 100535 & 1985 & 123.23 & 0.43 & 12323.00 & 113065.99 & 1.00 & 0.92 & 0.92 \\
13852 & 101781 & 1985 & 244.42 & 0.27 & 24442.00 & 212325.28 & 1.00 & 0.87 & 0.87 \\
14007 & 101800 & 1985 & 243.48 & 0.33 & 19900.00 & 180012.22 & 1.22 & 0.74 & 0.90 \\
19935 & 102659 & 1985 & 1474.03 & 0.24 & 147474.00 & 1385996.96 & 1.00 & 0.94 & 0.94 \\
3890 & 100510 & 1985 & 14.43 & 0.16 & 1443.00 & 12987.70 & 1.00 & 0.90 & 0.90 \\
7031 & 100992 & 1985 & 15.82 & 0.35 & 1563.00 & 15622.39 & 1.01 & 0.99 & 1.00 \\
14100 & 101804 & 1985 & 222.95 & 0.23 & 20956.00 & 224186.00 & 1.06 & 1.01 & 1.07 \\
3904 & 100514 & 1985 & 77.36 & 0.23 & 7736.00 & 71370.25 & 1.00 & 0.92 & 0.92 \\
19907 & 102655 & 1985 & 529.47 & 0.23 & 52947.00 & 461585.49 & 1.00 & 0.87 & 0.87 \\
14117 & 101805 & 1985 & 668.81 & 0.28 & 64320.00 & 650307.11 & 1.04 & 0.97 & 1.01 \\
20222 & 102689 & 1985 & 7.66 & 0.31 & 766.00 & 6771.94 & 1.00 & 0.88 & 0.88 \\
6820 & 100962 & 1985 & 356.35 & 0.22 & 35683.00 & 311832.92 & 1.00 & 0.88 & 0.87 \\
4000 & 100538 & 1985 & 64.70 & 0.27 & 6467.00 & 62637.53 & 1.00 & 0.97 & 0.97 \\
446 & 100056 & 1985 & 24.54 & 0.22 & 2454.00 & 22898.37 & 1.00 & 0.93 & 0.93 \\
25482 & 103494 & 1985 & 205.90 & 0.35 & 20600.00 & 186780.37 & 1.00 & 0.91 & 0.91 \\
20544 & 102767 & 1985 & 1075.05 & 0.29 & 107783.00 & 1007442.04 & 1.00 & 0.94 & 0.93 \\
14542 & 101876 & 1985 & 114.25 & 0.26 & 11593.00 & 98749.29 & 0.99 & 0.86 & 0.85 \\
7377 & 101038 & 1985 & 917.80 & 0.29 & 91800.00 & 775946.70 & 1.00 & 0.85 & 0.85 \\
556 & 100076 & 1985 & 202.43 & 0.28 & 20242.00 & 186338.18 & 1.00 & 0.92 & 0.92 \\
14497 & 101865 & 1985 & 33.24 & 0.32 & 3324.00 & 33470.73 & 1.00 & 1.01 & 1.01 \\
13646 & 101754 & 1985 & 22.15 & 0.25 & 2215.00 & 18625.99 & 1.00 & 0.84 & 0.84 \\
5431 & 100763 & 1985 & 85.86 & 0.28 & 8586.00 & 76613.73 & 1.00 & 0.89 & 0.89 \\
376 & 100046 & 1985 & 32.43 & 0.25 & 3240.00 & 26090.03 & 1.00 & 0.80 & 0.81 \\
22929 & 103089 & 1985 & 69.01 & 0.32 & 6262.00 & 61755.77 & 1.10 & 0.89 & 0.99 \\
104 & 100009 & 1985 & 15.71 & 0.46 & 1571.00 & 14287.88 & 1.00 & 0.91 & 0.91 \\
24826 & 103381 & 1985 & 415.00 & 0.26 & 41540.00 & 415055.23 & 1.00 & 1.00 & 1.00 \\
10243 & 101276 & 1985 & 44.56 & 0.29 & 4456.00 & 38445.20 & 1.00 & 0.86 & 0.86 \\
1732 & 100227 & 1985 & 98.17 & 0.31 & 9824.00 & 97845.85 & 1.00 & 1.00 & 1.00 \\
10403 & 101285 & 1985 & 43.79 & 0.15 & 4379.00 & 38784.21 & 1.00 & 0.89 & 0.89 \\
22883 & 103077 & 1985 & 18.20 & 0.34 & 1820.00 & 17776.47 & 1.00 & 0.98 & 0.98 \\
24867 & 103383 & 1985 & 104.01 & 0.23 & 10490.00 & 104459.07 & 0.99 & 1.00 & 1.00 \\
22773 & 103061 & 1985 & 3.03 & 0.13 & 303.00 & 2457.83 & 1.00 & 0.81 & 0.81 \\
10447 & 101286 & 1985 & 114.62 & 0.33 & 11462.00 & 107475.29 & 1.00 & 0.94 & 0.94 \\
10552 & 101299 & 1985 & 29.66 & 0.26 & 2966.00 & 25722.06 & 1.00 & 0.87 & 0.87 \\
2594 & 100346 & 1985 & 3.31 & 0.34 & 331.00 & 3109.40 & 1.00 & 0.94 & 0.94 \\
22666 & 103028 & 1985 & 179.00 & 0.34 & 17900.00 & 149095.18 & 1.00 & 0.83 & 0.83 \\
22622 & 103027 & 1985 & 154.00 & 0.27 & 9298.00 & 92431.98 & 1.66 & 0.60 & 0.99 \\
1669 & 100223 & 1985 & 93.56 & 0.28 & 6801.00 & 66653.40 & 1.38 & 0.71 & 0.98 \\
22398 & 103011 & 1985 & 36.98 & 0.33 & 3265.00 & 32982.13 & 1.13 & 0.89 & 1.01 \\
1656 & 100222 & 1985 & 80.09 & 0.14 & 8010.00 & 75549.82 & 1.00 & 0.94 & 0.94 \\
11351 & 101398 & 1985 & 121.90 & 0.20 & 12189.00 & 99932.35 & 1.00 & 0.82 & 0.82 \\
8953 & 101107 & 1985 & 715.70 & 0.28 & 71600.00 & 684062.65 & 1.00 & 0.96 & 0.96 \\
23235 & 103152 & 1985 & 221.13 & 0.34 & 22034.00 & 194463.24 & 1.00 & 0.88 & 0.88 \\
2319 & 100319 & 1985 & 35.21 & 0.39 & 3530.00 & 32492.02 & 1.00 & 0.92 & 0.92 \\
10182 & 101268 & 1985 & 52.51 & 0.29 & 5251.00 & 50611.97 & 1.00 & 0.96 & 0.96 \\
1911 & 100250 & 1985 & 21.16 & 0.39 & 2116.00 & 18497.95 & 1.00 & 0.87 & 0.87 \\
9433 & 101135 & 1985 & 2.05 & 0.31 & 205.00 & 2004.83 & 1.00 & 0.98 & 0.98 \\
24630 & 103373 & 1985 & 155.90 & 0.31 & 12431.00 & 115927.83 & 1.25 & 0.74 & 0.93 \\
9308 & 101131 & 1985 & 82.63 & 0.29 & 8263.00 & 80955.95 & 1.00 & 0.98 & 0.98 \\
23531 & 103184 & 1985 & 278.10 & 0.29 & 23271.00 & 216491.88 & 1.20 & 0.78 & 0.93 \\
24670 & 103375 & 1985 & 4.10 & 0.55 & 410.00 & 4067.17 & 1.00 & 0.99 & 0.99 \\
9249 & 101123 & 1985 & 169.80 & 0.42 & 12163.00 & 141678.72 & 1.40 & 0.83 & 1.16 \\
23505 & 103183 & 1985 & 224.37 & 0.27 & 22437.00 & 196616.93 & 1.00 & 0.88 & 0.88 \\
11491 & 101425 & 1985 & 25.96 & 0.24 & 1936.00 & 19357.07 & 1.34 & 0.75 & 1.00 \\
9530 & 101149 & 1985 & 3.96 & 0.28 & 396.00 & 3855.77 & 1.00 & 0.97 & 0.97 \\
9710 & 101170 & 1985 & 2.90 & 0.24 & 288.00 & 2464.97 & 1.01 & 0.85 & 0.86 \\
23482 & 103180 & 1985 & 94.89 & 0.45 & 9489.00 & 83627.66 & 1.00 & 0.88 & 0.88 \\
9716 & 101177 & 1985 & 47.11 & 0.15 & 4812.00 & 44431.19 & 0.98 & 0.94 & 0.92 \\
9737 & 101181 & 1985 & 66.47 & 0.05 & 7098.00 & 71171.77 & 0.94 & 1.07 & 1.00 \\
24710 & 103376 & 1985 & 860.01 & 0.34 & 86060.00 & 853656.31 & 1.00 & 0.99 & 0.99 \\
23463 & 103179 & 1985 & 176.10 & 0.25 & 17610.00 & 143942.55 & 1.00 & 0.82 & 0.82 \\
2287 & 100313 & 1985 & 107.35 & 0.31 & 10734.00 & 104330.96 & 1.00 & 0.97 & 0.97 \\
2299 & 100315 & 1985 & 225.66 & 0.30 & 22743.00 & 223254.17 & 0.99 & 0.99 & 0.98 \\
24786 & 103380 & 1985 & 1335.01 & 0.04 & 133550.00 & 1322484.03 & 1.00 & 0.99 & 0.99 \\
2185 & 100295 & 1985 & 11.13 & 0.21 & 1014.00 & 8885.83 & 1.10 & 0.80 & 0.88 \\
22362 & 103008 & 1985 & 48.70 & 0.36 & 4871.00 & 45463.08 & 1.00 & 0.93 & 0.93 \\
21798 & 102952 & 1985 & 262.65 & 0.20 & 21829.00 & 232150.59 & 1.20 & 0.88 & 1.06 \\
25172 & 103451 & 1985 & 9.69 & 0.21 & 783.00 & 7834.68 & 1.24 & 0.81 & 1.00 \\
21736 & 102949 & 1985 & 365.84 & 0.37 & 30247.00 & 289278.10 & 1.21 & 0.79 & 0.96 \\
1503 & 100209 & 1985 & 404.02 & 0.13 & 40402.00 & 379182.37 & 1.00 & 0.94 & 0.94 \\
1493 & 100208 & 1985 & 91.58 & 0.24 & 9158.00 & 79228.05 & 1.00 & 0.87 & 0.87 \\
25184 & 103460 & 1985 & 125.24 & 0.23 & 12524.00 & 114217.44 & 1.00 & 0.91 & 0.91 \\
12722 & 101591 & 1985 & 22.66 & 0.13 & 2282.00 & 18193.05 & 0.99 & 0.80 & 0.80 \\
3170 & 100412 & 1985 & 6.20 & 0.25 & 615.00 & 5846.86 & 1.01 & 0.94 & 0.95 \\
21478 & 102873 & 1985 & 115.53 & 0.28 & 8775.00 & 74272.83 & 1.32 & 0.64 & 0.85 \\
1465 & 100207 & 1985 & 1759.75 & 0.22 & 175975.00 & 1515563.04 & 1.00 & 0.86 & 0.86 \\
12867 & 101603 & 1985 & 346.69 & 0.31 & 34670.00 & 289303.65 & 1.00 & 0.83 & 0.83 \\
21148 & 102835 & 1985 & 7.87 & 0.05 & 509.00 & 4854.08 & 1.55 & 0.62 & 0.95 \\
12900 & 101606 & 1985 & 726.70 & 0.36 & 72670.00 & 669901.38 & 1.00 & 0.92 & 0.92 \\
3383 & 100430 & 1985 & 91.90 & 0.28 & 8413.00 & 78594.11 & 1.09 & 0.86 & 0.93 \\
12050 & 101494 & 1985 & 72.05 & 0.28 & 6739.00 & 61740.72 & 1.07 & 0.86 & 0.92 \\
8165 & 101078 & 1985 & 12.00 & 0.26 & 1040.00 & 11809.50 & 1.15 & 0.98 & 1.14 \\
24978 & 103406 & 1985 & 604.60 & 0.26 & 60460.00 & 538804.60 & 1.00 & 0.89 & 0.89 \\
11552 & 101430 & 1985 & 6.10 & 0.28 & 491.00 & 5203.96 & 1.24 & 0.85 & 1.06 \\
2781 & 100358 & 1985 & 9.93 & 0.26 & 993.00 & 8163.43 & 1.00 & 0.82 & 0.82 \\
22230 & 102997 & 1985 & 48.33 & 0.34 & 4833.00 & 48357.89 & 1.00 & 1.00 & 1.00 \\
11585 & 101431 & 1985 & 19.35 & 0.29 & 1632.00 & 17085.07 & 1.19 & 0.88 & 1.05 \\
25021 & 103426 & 1985 & 34.35 & 0.26 & 3434.00 & 27589.59 & 1.00 & 0.80 & 0.80 \\
8344 & 101084 & 1985 & 476.30 & 0.08 & 47600.00 & 385342.12 & 1.00 & 0.81 & 0.81 \\
22171 & 102994 & 1985 & 35.87 & 0.24 & 3324.00 & 27451.03 & 1.08 & 0.77 & 0.83 \\
8303 & 101082 & 1985 & 194.90 & 0.59 & 19500.00 & 178722.31 & 1.00 & 0.92 & 0.92 \\
1583 & 100217 & 1985 & 65.89 & 0.17 & 6992.00 & 70021.43 & 0.94 & 1.06 & 1.00 \\
8262 & 101081 & 1985 & 143.20 & 0.41 & 14300.00 & 116361.04 & 1.00 & 0.81 & 0.81 \\
8233 & 101080 & 1985 & 83.10 & 0.56 & 7437.00 & 74375.66 & 1.12 & 0.90 & 1.00 \\
11941 & 101473 & 1985 & 87.81 & 0.33 & 8781.00 & 74178.39 & 1.00 & 0.84 & 0.84 \\
25059 & 103429 & 1985 & 174.15 & 0.23 & 17415.00 & 161504.39 & 1.00 & 0.93 & 0.93 \\
2939 & 100389 & 1985 & 21.67 & 0.32 & 2163.00 & 21325.95 & 1.00 & 0.98 & 0.99 \\
63163 & 500486 & 1985 & 31.20 & 0.19 & 3399.00 & 27941.41 & 0.92 & 0.90 & 0.82 \\
8194 & 101079 & 1985 & 32.30 & 0.18 & 3200.00 & 26610.03 & 1.01 & 0.82 & 0.83 \\
12035 & 101491 & 1985 & 82.88 & 0.21 & 8288.00 & 69627.76 & 1.00 & 0.84 & 0.84 \\
49048 & 240212 & 1985 & 238.31 & 0.42 & 23659.00 & 215584.60 & 1.01 & 0.90 & 0.91 \\
1872 & 100247 & 1985 & 107.84 & 0.35 & 10784.00 & 94016.35 & 1.00 & 0.87 & 0.87 \\
964 & 100113 & 1985 & 710.85 & 0.32 & 60972.00 & 716330.44 & 1.17 & 1.01 & 1.17 \\
16832 & 102197 & 1985 & 11.93 & 0.31 & 1193.00 & 10987.08 & 1.00 & 0.92 & 0.92 \\
15709 & 102016 & 1985 & 907.67 & 0.26 & 87854.00 & 867477.16 & 1.03 & 0.96 & 0.99 \\
5119 & 100726 & 1985 & 43.26 & 0.30 & 4326.00 & 38112.98 & 1.00 & 0.88 & 0.88 \\
15558 & 102005 & 1985 & 830.61 & 0.26 & 72288.00 & 737531.52 & 1.15 & 0.89 & 1.02 \\
47237 & 200344 & 1985 & 884.67 & 0.21 & 88467.00 & 725799.67 & 1.00 & 0.82 & 0.82 \\
18339 & 102441 & 1985 & 235.50 & 0.33 & 23550.00 & 200355.11 & 1.00 & 0.85 & 0.85 \\
18789 & 102522 & 1985 & 198.81 & 0.27 & 19881.00 & 178728.68 & 1.00 & 0.90 & 0.90 \\
5233 & 100741 & 1985 & 21.91 & 0.14 & 2191.00 & 18835.02 & 1.00 & 0.86 & 0.86 \\
4924 & 100695 & 1985 & 69.01 & 0.30 & 6901.00 & 61269.21 & 1.00 & 0.89 & 0.89 \\
17215 & 102271 & 1985 & 236.23 & 0.35 & 21584.00 & 245475.24 & 1.09 & 1.04 & 1.14 \\
5160 & 100730 & 1985 & 166.20 & 0.17 & 16616.00 & 145716.86 & 1.00 & 0.88 & 0.88 \\
5591 & 100773 & 1985 & 832.10 & 0.25 & 82814.00 & 780232.15 & 1.00 & 0.94 & 0.94 \\
17369 & 102284 & 1985 & 23.89 & 0.39 & 2094.00 & 20015.23 & 1.14 & 0.84 & 0.96 \\
6316 & 100849 & 1985 & 48.78 & 0.18 & 4878.00 & 43128.98 & 1.00 & 0.88 & 0.88 \\
18329 & 102435 & 1985 & 151.43 & 0.32 & 15143.00 & 122526.29 & 1.00 & 0.81 & 0.81 \\
15664 & 102013 & 1985 & 100.05 & 0.24 & 9204.00 & 92571.88 & 1.09 & 0.93 & 1.01 \\
17662 & 102342 & 1985 & 6.37 & 0.36 & 641.00 & 5978.10 & 0.99 & 0.94 & 0.93 \\
17698 & 102346 & 1985 & 27.72 & 0.27 & 2772.00 & 24346.72 & 1.00 & 0.88 & 0.88 \\
57813 & 401072 & 1985 & 72.25 & 0.35 & 7225.00 & 63479.49 & 1.00 & 0.88 & 0.88 \\
1123 & 100155 & 1985 & 23.08 & 0.25 & 2310.00 & 21076.00 & 1.00 & 0.91 & 0.91 \\
4463 & 100634 & 1985 & 250.57 & 0.32 & 25057.00 & 239771.30 & 1.00 & 0.96 & 0.96 \\
18306 & 102425 & 1985 & 788.10 & 0.17 & 78835.00 & 735695.96 & 1.00 & 0.93 & 0.93 \\
17281 & 102278 & 1985 & 79.38 & 0.36 & 7380.00 & 81790.95 & 1.08 & 1.03 & 1.11 \\
4513 & 100637 & 1985 & 3.06 & 0.09 & 306.00 & 2864.93 & 1.00 & 0.94 & 0.94 \\
16093 & 102080 & 1985 & 93.98 & 0.33 & 9398.00 & 90580.87 & 1.00 & 0.96 & 0.96 \\
15624 & 102010 & 1985 & 210.20 & 0.26 & 17908.00 & 177809.08 & 1.17 & 0.85 & 0.99 \\
4367 & 100614 & 1985 & 70.97 & 0.24 & 7097.00 & 68673.57 & 1.00 & 0.97 & 0.97 \\
16347 & 102130 & 1985 & 113.99 & 0.33 & 11399.00 & 95442.54 & 1.00 & 0.84 & 0.84 \\
16010 & 102065 & 1985 & 73.05 & 0.39 & 7305.00 & 71841.23 & 1.00 & 0.98 & 0.98 \\
15160 & 101964 & 1985 & 21.55 & 0.35 & 2155.00 & 21348.59 & 1.00 & 0.99 & 0.99 \\
17059 & 102235 & 1985 & 69.58 & 0.23 & 6800.00 & 64211.52 & 1.02 & 0.92 & 0.94 \\
1195 & 100160 & 1985 & 29.55 & 0.24 & 2955.00 & 25551.46 & 1.00 & 0.86 & 0.86 \\
16953 & 102224 & 1985 & 35.82 & 0.23 & 3595.00 & 32642.99 & 1.00 & 0.91 & 0.91 \\
1005 & 100122 & 1985 & 5.00 & 0.08 & 500.00 & 4689.29 & 1.00 & 0.94 & 0.94 \\
642 & 100087 & 1985 & 168.53 & 0.36 & 16850.00 & 161549.38 & 1.00 & 0.96 & 0.96 \\
18803 & 102523 & 1985 & 355.40 & 0.27 & 29162.00 & 240454.02 & 1.22 & 0.68 & 0.82 \\
17466 & 102307 & 1985 & 34.19 & 0.29 & 3419.00 & 28752.26 & 1.00 & 0.84 & 0.84 \\
17025 & 102231 & 1985 & 627.80 & 0.29 & 62780.00 & 601619.10 & 1.00 & 0.96 & 0.96 \\
1008 & 100127 & 1985 & 147.11 & 0.26 & 12748.00 & 138603.86 & 1.15 & 0.94 & 1.09 \\
4611 & 100644 & 1985 & 31.49 & 0.29 & 3149.00 & 28935.55 & 1.00 & 0.92 & 0.92 \\
18209 & 102416 & 1985 & 313.53 & 0.23 & 31353.00 & 287450.29 & 1.00 & 0.92 & 0.92 \\
17175 & 102268 & 1985 & 20.07 & 0.02 & 2007.00 & 16489.05 & 1.00 & 0.82 & 0.82 \\
15514 & 102000 & 1985 & 158.58 & 0.37 & 16040.00 & 138370.69 & 0.99 & 0.87 & 0.86 \\
17422 & 102306 & 1985 & 151.56 & 0.38 & 12295.00 & 115185.65 & 1.23 & 0.76 & 0.94 \\
16862 & 102213 & 1985 & 22.44 & 0.29 & 2244.00 & 19660.17 & 1.00 & 0.88 & 0.88 \\
5497 & 100767 & 1985 & 55.20 & 0.28 & 5478.00 & 55306.74 & 1.01 & 1.00 & 1.01 \\
47365 & 210681 & 1985 & 86.80 & 0.27 & 8488.00 & 86378.74 & 1.02 & 1.00 & 1.02 \\
26274 & 103551 & 1985 & 8.33 & 0.29 & 833.00 & 7358.24 & 1.00 & 0.88 & 0.88 \\
5063 & 100714 & 1985 & 116.08 & 0.17 & 14079.00 & 132762.25 & 0.82 & 1.14 & 0.94 \\
18256 & 102419 & 1985 & 266.12 & 0.34 & 26612.00 & 246763.59 & 1.00 & 0.93 & 0.93 \\
11942 & 101473 & 1986 & 131.16 & 0.18 & 13120.00 & 108912.33 & 1.00 & 0.83 & 0.83 \\
22231 & 102997 & 1986 & 40.35 & 0.17 & 4035.00 & 38008.17 & 1.00 & 0.94 & 0.94 \\
8304 & 101082 & 1986 & 229.30 & 0.22 & 22900.00 & 209911.52 & 1.00 & 0.92 & 0.92 \\
5064 & 100714 & 1986 & 70.25 & 0.14 & 7656.00 & 64951.19 & 0.92 & 0.92 & 0.85 \\
16348 & 102130 & 1986 & 177.92 & 0.18 & 17792.00 & 153446.30 & 1.00 & 0.86 & 0.86 \\
25060 & 103429 & 1986 & 180.88 & 0.18 & 18088.00 & 150399.90 & 1.00 & 0.83 & 0.83 \\
17467 & 102307 & 1986 & 33.52 & 0.10 & 2804.00 & 28216.94 & 1.20 & 0.84 & 1.01 \\
26001 & 103533 & 1986 & 43.43 & 0.21 & 4343.00 & 39350.69 & 1.00 & 0.91 & 0.91 \\
11700 & 101457 & 1986 & 68.56 & 0.10 & 6855.00 & 60162.93 & 1.00 & 0.88 & 0.88 \\
22203 & 102996 & 1986 & 186.73 & 0.18 & 18673.00 & 164052.03 & 1.00 & 0.88 & 0.88 \\
8263 & 101081 & 1986 & 92.60 & 0.15 & 9300.00 & 85703.83 & 1.00 & 0.93 & 0.92 \\
5042 & 100705 & 1986 & 62.13 & 0.10 & 7406.00 & 52365.44 & 0.84 & 0.84 & 0.71 \\
8234 & 101080 & 1986 & 94.00 & 0.16 & 9400.00 & 83617.55 & 1.00 & 0.89 & 0.89 \\
7932 & 101067 & 1986 & 39.40 & 0.27 & 3900.00 & 31654.28 & 1.01 & 0.80 & 0.81 \\
7895 & 101065 & 1986 & 39.60 & 0.04 & 3700.00 & 35406.27 & 1.07 & 0.89 & 0.96 \\
1494 & 100208 & 1986 & 83.54 & -0.10 & 8353.00 & 72356.56 & 1.00 & 0.87 & 0.87 \\
4794 & 100682 & 1986 & 32.03 & 0.18 & 3203.00 & 25792.39 & 1.00 & 0.81 & 0.81 \\
25185 & 103460 & 1986 & 116.61 & 0.06 & 11800.00 & 106960.14 & 0.99 & 0.92 & 0.91 \\
3171 & 100412 & 1986 & 6.00 & 0.06 & 648.00 & 5431.54 & 0.93 & 0.91 & 0.84 \\
12723 & 101591 & 1986 & 19.50 & 0.02 & 1950.00 & 17314.80 & 1.00 & 0.89 & 0.89 \\
18166 & 102414 & 1986 & 111.42 & 0.08 & 11161.00 & 107349.74 & 1.00 & 0.96 & 0.96 \\
1466 & 100207 & 1986 & 2221.16 & 0.11 & 222116.00 & 2133587.64 & 1.00 & 0.96 & 0.96 \\
16011 & 102065 & 1986 & 54.97 & 0.15 & 5497.00 & 53824.27 & 1.00 & 0.98 & 0.98 \\
2940 & 100389 & 1986 & 26.53 & 0.15 & 2653.00 & 26100.28 & 1.00 & 0.98 & 0.98 \\
25947 & 103531 & 1986 & 22.93 & 0.13 & 2293.00 & 18362.08 & 1.00 & 0.80 & 0.80 \\
8195 & 101079 & 1986 & 111.80 & 0.05 & 11200.00 & 97831.30 & 1.00 & 0.88 & 0.87 \\
17663 & 102342 & 1986 & 20.48 & 0.24 & 1755.00 & 17187.57 & 1.17 & 0.84 & 0.98 \\
12036 & 101491 & 1986 & 80.04 & 0.15 & 8000.00 & 66322.72 & 1.00 & 0.83 & 0.83 \\
8166 & 101078 & 1986 & 6.70 & 0.05 & 700.00 & 5764.52 & 0.96 & 0.86 & 0.82 \\
12051 & 101494 & 1986 & 87.47 & 0.14 & 8472.00 & 73633.88 & 1.03 & 0.84 & 0.87 \\
17699 & 102346 & 1986 & 24.89 & 0.08 & 2489.00 & 20586.54 & 1.00 & 0.83 & 0.83 \\
3010 & 100398 & 1986 & 35.86 & 0.08 & 3590.00 & 30444.23 & 1.00 & 0.85 & 0.85 \\
8054 & 101073 & 1986 & 630.70 & 0.11 & 63100.00 & 547973.77 & 1.00 & 0.87 & 0.87 \\
4925 & 100695 & 1986 & 68.63 & 0.08 & 6863.00 & 56910.34 & 1.00 & 0.83 & 0.83 \\
7964 & 101068 & 1986 & 7405.50 & 0.24 & 740500.00 & 6212854.51 & 1.00 & 0.84 & 0.84 \\
21737 & 102949 & 1986 & 442.66 & 0.17 & 44266.00 & 361283.24 & 1.00 & 0.82 & 0.82 \\
49049 & 240212 & 1986 & 237.19 & 0.16 & 28954.00 & 222609.24 & 0.82 & 0.94 & 0.77 \\
2782 & 100358 & 1986 & 10.06 & 0.20 & 1006.00 & 9712.46 & 1.00 & 0.97 & 0.97 \\
11337 & 101396 & 1986 & 21.84 & 0.02 & 2184.00 & 19440.85 & 1.00 & 0.89 & 0.89 \\
9738 & 101181 & 1986 & 51.80 & -0.04 & 6532.00 & 42919.27 & 0.79 & 0.83 & 0.66 \\
24711 & 103376 & 1986 & 928.00 & 0.19 & 92456.00 & 924810.28 & 1.00 & 1.00 & 1.00 \\
23464 & 103179 & 1986 & 191.42 & 0.17 & 19142.00 & 162251.55 & 1.00 & 0.85 & 0.85 \\
9947 & 101215 & 1986 & 10.27 & 0.12 & 915.00 & 7626.62 & 1.12 & 0.74 & 0.83 \\
9986 & 101233 & 1986 & 59.01 & 0.10 & 6292.00 & 58719.39 & 0.94 & 1.00 & 0.93 \\
17216 & 102271 & 1986 & 303.58 & 0.19 & 23632.00 & 301414.58 & 1.28 & 0.99 & 1.28 \\
2288 & 100313 & 1986 & 96.19 & 0.10 & 10777.00 & 96598.73 & 0.89 & 1.00 & 0.90 \\
5234 & 100741 & 1986 & 62.60 & 0.11 & 6264.00 & 56430.16 & 1.00 & 0.90 & 0.90 \\
16863 & 102213 & 1986 & 32.95 & 0.16 & 3295.00 & 29342.76 & 1.00 & 0.89 & 0.89 \\
2300 & 100315 & 1986 & 234.83 & 0.19 & 23483.00 & 228780.84 & 1.00 & 0.97 & 0.97 \\
16833 & 102197 & 1986 & 15.55 & 0.23 & 1555.00 & 13795.70 & 1.00 & 0.89 & 0.89 \\
10183 & 101268 & 1986 & 60.49 & 0.20 & 6049.00 & 55169.30 & 1.00 & 0.91 & 0.91 \\
2320 & 100319 & 1986 & 46.92 & 0.09 & 4692.00 & 42302.36 & 1.00 & 0.90 & 0.90 \\
23236 & 103152 & 1986 & 248.35 & 0.13 & 26238.00 & 209457.46 & 0.95 & 0.84 & 0.80 \\
8954 & 101107 & 1986 & 673.40 & 0.14 & 67300.00 & 647400.99 & 1.00 & 0.96 & 0.96 \\
965 & 100113 & 1986 & 652.01 & 0.13 & 65200.00 & 643777.01 & 1.00 & 0.99 & 0.99 \\
22930 & 103089 & 1986 & 70.65 & 0.15 & 7987.00 & 60359.24 & 0.88 & 0.85 & 0.76 \\
105 & 100009 & 1986 & 37.30 & 0.13 & 3716.00 & 35988.77 & 1.00 & 0.96 & 0.97 \\
24787 & 103380 & 1986 & 2049.00 & 0.21 & 206115.00 & 2059184.83 & 0.99 & 1.00 & 1.00 \\
5498 & 100767 & 1986 & 48.80 & 0.08 & 5269.00 & 51390.97 & 0.93 & 1.05 & 0.98 \\
23483 & 103180 & 1986 & 63.99 & -0.05 & 6399.00 & 56035.25 & 1.00 & 0.88 & 0.88 \\
9336 & 101132 & 1986 & 2.56 & 0.07 & 226.00 & 2046.25 & 1.13 & 0.80 & 0.91 \\
5432 & 100763 & 1986 & 73.65 & 0.05 & 7353.00 & 68859.33 & 1.00 & 0.93 & 0.94 \\
1912 & 100250 & 1986 & 23.50 & 0.20 & 2350.00 & 21018.51 & 1.00 & 0.89 & 0.89 \\
9434 & 101135 & 1986 & 3.38 & 0.29 & 338.00 & 3041.08 & 1.00 & 0.90 & 0.90 \\
24631 & 103373 & 1986 & 163.80 & 0.12 & 15880.00 & 144962.32 & 1.03 & 0.88 & 0.91 \\
9309 & 101131 & 1986 & 66.89 & 0.06 & 6688.00 & 63742.65 & 1.00 & 0.95 & 0.95 \\
17026 & 102231 & 1986 & 596.45 & 0.16 & 59645.00 & 567020.12 & 1.00 & 0.95 & 0.95 \\
23532 & 103184 & 1986 & 253.69 & 0.15 & 25369.00 & 214448.37 & 1.00 & 0.85 & 0.85 \\
16954 & 102224 & 1986 & 29.41 & 0.00 & 2941.00 & 27455.01 & 1.00 & 0.93 & 0.93 \\
9266 & 101127 & 1986 & 70.27 & 0.19 & 7027.00 & 56802.14 & 1.00 & 0.81 & 0.81 \\
1873 & 100247 & 1986 & 160.97 & 0.20 & 16097.00 & 136330.66 & 1.00 & 0.85 & 0.85 \\
24671 & 103375 & 1986 & 7.45 & 0.28 & 736.00 & 7371.88 & 1.01 & 0.99 & 1.00 \\
9250 & 101123 & 1986 & 163.60 & 0.15 & 12163.00 & 168766.74 & 1.35 & 1.03 & 1.39 \\
17060 & 102235 & 1986 & 66.66 & 0.15 & 7695.00 & 58315.46 & 0.87 & 0.87 & 0.76 \\
23506 & 103183 & 1986 & 226.59 & 0.17 & 22660.00 & 220461.53 & 1.00 & 0.97 & 0.97 \\
9531 & 101149 & 1986 & 4.02 & 0.22 & 402.00 & 3561.69 & 1.00 & 0.89 & 0.89 \\
2186 & 100295 & 1986 & 11.21 & 0.12 & 1121.00 & 10084.90 & 1.00 & 0.90 & 0.90 \\
9569 & 101151 & 1986 & 11.46 & -0.02 & 1146.00 & 10547.02 & 1.00 & 0.92 & 0.92 \\
24827 & 103381 & 1986 & 287.00 & 0.10 & 28558.00 & 285779.39 & 1.00 & 1.00 & 1.00 \\
57814 & 401072 & 1986 & 67.90 & 0.05 & 9268.00 & 69595.99 & 0.73 & 1.02 & 0.75 \\
8665 & 101094 & 1986 & 23.50 & 0.14 & 2300.00 & 19986.89 & 1.02 & 0.85 & 0.87 \\
22399 & 103011 & 1986 & 41.08 & 0.15 & 4160.00 & 37881.48 & 0.99 & 0.92 & 0.91 \\
1657 & 100222 & 1986 & 64.32 & -0.02 & 6240.00 & 57818.90 & 1.03 & 0.90 & 0.93 \\
8518 & 101089 & 1986 & 9.60 & 0.08 & 1000.00 & 9024.77 & 0.96 & 0.94 & 0.90 \\
10754 & 101330 & 1986 & 47.01 & -0.02 & 4860.00 & 38009.33 & 0.97 & 0.81 & 0.78 \\
11352 & 101398 & 1986 & 114.99 & 0.21 & 11499.00 & 97121.42 & 1.00 & 0.84 & 0.84 \\
8448 & 101087 & 1986 & 17.70 & 0.10 & 1800.00 & 14235.19 & 0.98 & 0.80 & 0.79 \\
17423 & 102306 & 1986 & 202.84 & 0.22 & 14677.00 & 138917.85 & 1.38 & 0.68 & 0.95 \\
11492 & 101425 & 1986 & 22.19 & 0.11 & 2219.00 & 21554.12 & 1.00 & 0.97 & 0.97 \\
22363 & 103008 & 1986 & 51.86 & 0.13 & 5188.00 & 48778.46 & 1.00 & 0.94 & 0.94 \\
26090 & 103538 & 1986 & 14.29 & 0.06 & 1429.00 & 13282.57 & 1.00 & 0.93 & 0.93 \\
8383 & 101085 & 1986 & 210.70 & 0.24 & 21100.00 & 206423.17 & 1.00 & 0.98 & 0.98 \\
1009 & 100127 & 1986 & 143.59 & 0.12 & 14300.00 & 131655.39 & 1.00 & 0.92 & 0.92 \\
24960 & 103397 & 1986 & 23.00 & 0.13 & 2389.00 & 19711.66 & 0.96 & 0.86 & 0.83 \\
5120 & 100726 & 1986 & 78.80 & 0.16 & 7880.00 & 64600.55 & 1.00 & 0.82 & 0.82 \\
22623 & 103027 & 1986 & 93.72 & 0.26 & 9370.00 & 90568.23 & 1.00 & 0.97 & 0.97 \\
8747 & 101097 & 1986 & 9.50 & 0.20 & 1000.00 & 8937.44 & 0.95 & 0.94 & 0.89 \\
17282 & 102278 & 1986 & 103.92 & 0.24 & 7942.00 & 99550.04 & 1.31 & 0.96 & 1.25 \\
1733 & 100227 & 1986 & 93.66 & 0.11 & 9366.00 & 91857.10 & 1.00 & 0.98 & 0.98 \\
26508 & 103585 & 1986 & 18.50 & -0.03 & 1692.00 & 15581.66 & 1.09 & 0.84 & 0.92 \\
22884 & 103077 & 1986 & 18.20 & 0.16 & 2110.00 & 17678.06 & 0.86 & 0.97 & 0.84 \\
10404 & 101285 & 1986 & 30.86 & -0.06 & 3092.00 & 30354.99 & 1.00 & 0.98 & 0.98 \\
24868 & 103383 & 1986 & 153.00 & 0.26 & 15108.00 & 151191.31 & 1.01 & 0.99 & 1.00 \\
22774 & 103061 & 1986 & 2.04 & -0.05 & 204.00 & 1846.91 & 1.00 & 0.90 & 0.91 \\
5161 & 100730 & 1986 & 137.90 & 0.01 & 13787.00 & 118052.55 & 1.00 & 0.86 & 0.86 \\
10448 & 101286 & 1986 & 167.88 & 0.20 & 16789.00 & 155458.66 & 1.00 & 0.93 & 0.93 \\
10553 & 101299 & 1986 & 48.20 & 0.15 & 4818.00 & 40798.68 & 1.00 & 0.85 & 0.85 \\
17370 & 102284 & 1986 & 23.34 & 0.09 & 2637.00 & 18870.91 & 0.89 & 0.81 & 0.72 \\
2595 & 100346 & 1986 & 3.08 & 0.06 & 310.00 & 2908.20 & 0.99 & 0.95 & 0.94 \\
10244 & 101276 & 1986 & 47.85 & 0.13 & 4785.00 & 46015.82 & 1.00 & 0.96 & 0.96 \\
12811 & 101601 & 1986 & 250.02 & 0.06 & 25000.00 & 212830.87 & 1.00 & 0.85 & 0.85 \\
21799 & 102952 & 1986 & 296.15 & 0.08 & 29599.00 & 274350.70 & 1.00 & 0.93 & 0.93 \\
12868 & 101603 & 1986 & 364.47 & 0.18 & 36450.00 & 327154.27 & 1.00 & 0.90 & 0.90 \\
20223 & 102689 & 1986 & 7.52 & 0.12 & 752.00 & 6069.14 & 1.00 & 0.81 & 0.81 \\
15515 & 102000 & 1986 & 293.97 & 0.17 & 29397.00 & 273431.80 & 1.00 & 0.93 & 0.93 \\
20096 & 102667 & 1986 & 132.60 & -0.03 & 13260.00 & 114680.39 & 1.00 & 0.86 & 0.86 \\
19936 & 102659 & 1986 & 1034.08 & -0.00 & 103408.00 & 968552.64 & 1.00 & 0.94 & 0.94 \\
6288 & 100847 & 1986 & 3.50 & -0.09 & 350.00 & 2882.09 & 1.00 & 0.82 & 0.82 \\
13826 & 101769 & 1986 & 272.94 & 0.13 & 27294.00 & 252220.53 & 1.00 & 0.92 & 0.92 \\
1335 & 100190 & 1986 & 577.57 & 0.10 & 57757.00 & 554154.30 & 1.00 & 0.96 & 0.96 \\
13647 & 101754 & 1986 & 23.61 & 0.14 & 2374.00 & 20097.46 & 0.99 & 0.85 & 0.85 \\
4514 & 100637 & 1986 & 20.39 & 0.16 & 2039.00 & 19601.46 & 1.00 & 0.96 & 0.96 \\
3891 & 100510 & 1986 & 14.41 & -0.02 & 1441.00 & 13785.44 & 1.00 & 0.96 & 0.96 \\
14101 & 101804 & 1986 & 241.32 & 0.16 & 25589.00 & 223196.56 & 0.94 & 0.92 & 0.87 \\
19908 & 102655 & 1986 & 496.24 & 0.16 & 49624.00 & 405786.83 & 1.00 & 0.82 & 0.82 \\
14118 & 101805 & 1986 & 552.66 & 0.06 & 58076.00 & 464490.58 & 0.95 & 0.84 & 0.80 \\
4612 & 100644 & 1986 & 30.10 & 0.14 & 3010.00 & 28109.15 & 1.00 & 0.93 & 0.93 \\
15625 & 102010 & 1986 & 238.81 & 0.13 & 24520.00 & 225536.32 & 0.97 & 0.94 & 0.92 \\
20696 & 102784 & 1986 & 389.68 & 0.20 & 37853.00 & 363165.69 & 1.03 & 0.93 & 0.96 \\
1124 & 100155 & 1986 & 24.91 & 0.13 & 2492.00 & 23988.27 & 1.00 & 0.96 & 0.96 \\
15559 & 102005 & 1986 & 909.61 & 0.06 & 86364.00 & 840812.68 & 1.05 & 0.92 & 0.97 \\
20585 & 102774 & 1986 & 199.69 & 0.19 & 19791.00 & 191592.22 & 1.01 & 0.96 & 0.97 \\
20545 & 102767 & 1986 & 969.07 & 0.16 & 115866.00 & 1003893.54 & 0.84 & 1.04 & 0.87 \\
18340 & 102441 & 1986 & 243.15 & 0.20 & 24318.00 & 215147.45 & 1.00 & 0.88 & 0.88 \\
7378 & 101038 & 1986 & 1125.60 & 0.19 & 112600.00 & 1039403.51 & 1.00 & 0.92 & 0.92 \\
3705 & 100471 & 1986 & 7.40 & 0.12 & 594.00 & 6756.80 & 1.25 & 0.91 & 1.14 \\
447 & 100056 & 1986 & 16.20 & -0.06 & 2493.00 & 14777.71 & 0.65 & 0.91 & 0.59 \\
7070 & 100994 & 1986 & 11.30 & -0.03 & 910.00 & 9087.23 & 1.24 & 0.80 & 1.00 \\
6317 & 100849 & 1986 & 37.40 & -0.02 & 3739.00 & 35086.32 & 1.00 & 0.94 & 0.94 \\
19864 & 102654 & 1986 & 83.49 & 0.14 & 8335.00 & 77145.71 & 1.00 & 0.92 & 0.93 \\
6821 & 100962 & 1986 & 256.76 & -0.08 & 25676.00 & 228184.66 & 1.00 & 0.89 & 0.89 \\
582 & 100079 & 1986 & 72.45 & -0.03 & 7245.00 & 68108.71 & 1.00 & 0.94 & 0.94 \\
47366 & 210681 & 1986 & 84.40 & 0.13 & 9300.00 & 74624.28 & 0.91 & 0.88 & 0.80 \\
25540 & 103496 & 1986 & 291.71 & 0.23 & 29004.00 & 296629.94 & 1.01 & 1.02 & 1.02 \\
6734 & 100947 & 1986 & 643.87 & 0.13 & 64381.00 & 586460.28 & 1.00 & 0.91 & 0.91 \\
15161 & 101964 & 1986 & 22.23 & 0.12 & 2223.00 & 21935.34 & 1.00 & 0.99 & 0.99 \\
14861 & 101919 & 1986 & 231.90 & 0.00 & 26319.00 & 255041.50 & 0.88 & 1.10 & 0.97 \\
19208 & 102570 & 1986 & 89.95 & 0.17 & 9030.00 & 82306.21 & 1.00 & 0.92 & 0.91 \\
4162 & 100567 & 1986 & 292.58 & 0.19 & 29136.00 & 251935.10 & 1.00 & 0.86 & 0.86 \\
643 & 100087 & 1986 & 171.13 & 0.25 & 17113.00 & 171232.22 & 1.00 & 1.00 & 1.00 \\
14939 & 101925 & 1986 & 281.61 & 0.16 & 29330.00 & 261527.91 & 0.96 & 0.93 & 0.89 \\
6623 & 100906 & 1986 & 378.79 & 0.10 & 37879.00 & 334886.15 & 1.00 & 0.88 & 0.88 \\
25615 & 103498 & 1986 & 194.38 & 0.05 & 19363.00 & 169830.95 & 1.00 & 0.87 & 0.88 \\
74633 & 601143 & 1986 & 116.38 & 0.11 & 15912.00 & 143176.14 & 0.73 & 1.23 & 0.90 \\
1209 & 100166 & 1986 & 1425.50 & 0.15 & 142550.00 & 1157638.88 & 1.00 & 0.81 & 0.81 \\
1196 & 100160 & 1986 & 37.61 & 0.11 & 3761.00 & 34767.88 & 1.00 & 0.92 & 0.92 \\
6430 & 100870 & 1986 & 14.00 & 0.16 & 1795.00 & 16447.34 & 0.78 & 1.17 & 0.92 \\
4368 & 100614 & 1986 & 75.00 & 0.19 & 7568.00 & 69441.87 & 0.99 & 0.93 & 0.92 \\
14151 & 101819 & 1986 & 94.99 & 0.06 & 10198.00 & 85341.75 & 0.93 & 0.90 & 0.84 \\
7032 & 100992 & 1986 & 71.80 & 0.24 & 7180.00 & 68034.16 & 1.00 & 0.95 & 0.95 \\
4464 & 100634 & 1986 & 200.93 & 0.07 & 20092.00 & 172432.99 & 1.00 & 0.86 & 0.86 \\
47173 & 200342 & 1986 & 121.40 & 0.15 & 12200.00 & 107704.78 & 1.00 & 0.89 & 0.88 \\
3958 & 100535 & 1986 & 141.49 & 0.10 & 14100.00 & 135995.29 & 1.00 & 0.96 & 0.96 \\
6365 & 100856 & 1986 & 75.67 & 0.09 & 7567.00 & 68816.16 & 1.00 & 0.91 & 0.91 \\
14434 & 101858 & 1986 & 105.22 & 0.14 & 10521.00 & 92268.95 & 1.00 & 0.88 & 0.88 \\
25483 & 103494 & 1986 & 212.47 & 0.17 & 21250.00 & 185548.80 & 1.00 & 0.87 & 0.87 \\
47238 & 200344 & 1986 & 912.97 & 0.13 & 91297.00 & 824471.26 & 1.00 & 0.90 & 0.90 \\
557 & 100076 & 1986 & 213.31 & 0.13 & 19385.00 & 193593.50 & 1.10 & 0.91 & 1.00 \\
14543 & 101876 & 1986 & 124.95 & 0.21 & 12311.00 & 100317.63 & 1.01 & 0.80 & 0.81 \\
4001 & 100538 & 1986 & 71.53 & 0.11 & 8784.00 & 76677.50 & 0.81 & 1.07 & 0.87 \\
6859 & 100963 & 1986 & 45.97 & -0.03 & 4597.00 & 37550.89 & 1.00 & 0.82 & 0.82 \\
13853 & 101781 & 1986 & 276.35 & 0.15 & 27635.00 & 264961.25 & 1.00 & 0.96 & 0.96 \\
15968 & 102062 & 1986 & 16.39 & -0.04 & 1611.00 & 13861.23 & 1.02 & 0.85 & 0.86 \\
7689 & 101055 & 1986 & 172.50 & 0.13 & 17300.00 & 155998.14 & 1.00 & 0.90 & 0.90 \\
12901 & 101606 & 1986 & 710.35 & 0.20 & 71030.00 & 626444.68 & 1.00 & 0.88 & 0.88 \\
3448 & 100439 & 1986 & 2.29 & -0.05 & 228.00 & 2068.65 & 1.00 & 0.90 & 0.91 \\
18257 & 102419 & 1986 & 230.23 & 0.02 & 23032.00 & 212519.64 & 1.00 & 0.92 & 0.92 \\
18210 & 102416 & 1986 & 200.26 & 0.04 & 20026.00 & 191479.90 & 1.00 & 0.96 & 0.96 \\
25267 & 103464 & 1986 & 179.23 & 0.10 & 17367.00 & 158391.19 & 1.03 & 0.88 & 0.91 \\
15848 & 102048 & 1986 & 134.79 & 0.14 & 13479.00 & 126196.73 & 1.00 & 0.94 & 0.94 \\
15665 & 102013 & 1986 & 117.25 & 0.17 & 12052.00 & 110987.81 & 0.97 & 0.95 & 0.92 \\
20833 & 102796 & 1986 & 1.97 & 0.16 & 199.00 & 1858.06 & 0.99 & 0.94 & 0.93 \\
3384 & 100430 & 1986 & 59.88 & 0.04 & 6010.00 & 58093.10 & 1.00 & 0.97 & 0.97 \\
15710 & 102016 & 1986 & 859.63 & 0.12 & 89920.00 & 811589.54 & 0.96 & 0.94 & 0.90 \\
21149 & 102835 & 1986 & 11.12 & 0.13 & 1166.00 & 10420.96 & 0.95 & 0.94 & 0.89 \\
377 & 100046 & 1986 & 32.20 & 0.11 & 3220.00 & 26112.07 & 1.00 & 0.81 & 0.81 \\
13021 & 101622 & 1986 & 176.19 & 0.23 & 21250.00 & 211509.80 & 0.83 & 1.20 & 1.00 \\
1405 & 100196 & 1986 & 142.95 & 0.06 & 14295.00 & 132519.05 & 1.00 & 0.93 & 0.93 \\
3385 & 100430 & 1987 & 64.64 & -0.05 & 5922.00 & 56985.80 & 1.09 & 0.88 & 0.96 \\
4424 & 100624 & 1987 & 35.10 & 0.03 & 3505.00 & 34470.83 & 1.00 & 0.98 & 0.98 \\
5235 & 100741 & 1987 & 167.87 & 0.12 & 16732.00 & 167232.94 & 1.00 & 1.00 & 1.00 \\
14435 & 101858 & 1987 & 91.66 & 0.02 & 9135.00 & 81440.38 & 1.00 & 0.89 & 0.89 \\
8955 & 101107 & 1987 & 601.30 & -0.04 & 60130.00 & 568323.52 & 1.00 & 0.95 & 0.95 \\
57830 & 401082 & 1987 & 1.46 & 0.03 & 142.00 & 1349.43 & 1.03 & 0.93 & 0.95 \\
6366 & 100856 & 1987 & 83.78 & 0.11 & 8378.00 & 71492.95 & 1.00 & 0.85 & 0.85 \\
23237 & 103152 & 1987 & 271.96 & 0.09 & 25358.00 & 218971.95 & 1.07 & 0.81 & 0.86 \\
15849 & 102048 & 1987 & 151.53 & 0.27 & 15153.00 & 152190.34 & 1.00 & 1.00 & 1.00 \\
2321 & 100319 & 1987 & 35.73 & -0.09 & 3573.00 & 35242.88 & 1.00 & 0.99 & 0.99 \\
4369 & 100614 & 1987 & 81.00 & 0.02 & 8101.00 & 76759.55 & 1.00 & 0.95 & 0.95 \\
2301 & 100315 & 1987 & 219.56 & 0.03 & 21956.00 & 208082.95 & 1.00 & 0.95 & 0.95 \\
4387 & 100622 & 1987 & 31.30 & -0.11 & 3126.00 & 28619.43 & 1.00 & 0.91 & 0.92 \\
5162 & 100730 & 1987 & 122.62 & -0.02 & 12242.00 & 115884.96 & 1.00 & 0.95 & 0.95 \\
8836 & 101101 & 1987 & 1.30 & -0.22 & 130.00 & 1107.51 & 1.00 & 0.85 & 0.85 \\
19865 & 102654 & 1987 & 62.44 & -0.09 & 6244.00 & 57433.50 & 1.00 & 0.92 & 0.92 \\
10405 & 101285 & 1987 & 38.02 & 0.14 & 3802.00 & 32969.96 & 1.00 & 0.87 & 0.87 \\
14152 & 101819 & 1987 & 86.79 & 0.04 & 8679.00 & 72874.77 & 1.00 & 0.84 & 0.84 \\
22885 & 103077 & 1987 & 16.70 & 0.06 & 1670.00 & 16413.01 & 1.00 & 0.98 & 0.98 \\
7033 & 100992 & 1987 & 100.81 & -0.03 & 10204.00 & 101958.15 & 0.99 & 1.01 & 1.00 \\
6318 & 100849 & 1987 & 33.34 & 0.02 & 3334.00 & 30729.77 & 1.00 & 0.92 & 0.92 \\
4692 & 100666 & 1987 & 1.30 & -0.06 & 126.00 & 1132.54 & 1.03 & 0.87 & 0.90 \\
8865 & 101103 & 1987 & 1.40 & 0.07 & 140.00 & 1234.11 & 1.00 & 0.88 & 0.88 \\
19827 & 102653 & 1987 & 1267.97 & 0.04 & 107802.00 & 1058968.01 & 1.18 & 0.84 & 0.98 \\
8941 & 101106 & 1987 & 13.30 & -0.17 & 1340.00 & 10805.95 & 0.99 & 0.81 & 0.81 \\
3959 & 100535 & 1987 & 109.40 & -0.08 & 10857.00 & 107394.61 & 1.01 & 0.98 & 0.99 \\
57815 & 401072 & 1987 & 49.43 & -0.14 & 4943.00 & 44286.87 & 1.00 & 0.90 & 0.90 \\
47174 & 200342 & 1987 & 85.20 & -0.09 & 8510.00 & 79078.19 & 1.00 & 0.93 & 0.93 \\
14391 & 101854 & 1987 & 606.51 & -0.07 & 89769.00 & 521940.99 & 0.68 & 0.86 & 0.58 \\
4795 & 100682 & 1987 & 25.27 & -0.08 & 2700.00 & 22138.29 & 0.94 & 0.88 & 0.82 \\
18258 & 102419 & 1987 & 198.70 & -0.03 & 19869.00 & 168311.47 & 1.00 & 0.85 & 0.85 \\
17283 & 102278 & 1987 & 141.78 & 0.12 & 10167.00 & 122332.45 & 1.39 & 0.86 & 1.20 \\
10245 & 101276 & 1987 & 51.10 & 0.11 & 5110.00 & 50135.66 & 1.00 & 0.98 & 0.98 \\
16095 & 102080 & 1987 & 100.28 & -0.19 & 10028.00 & 86451.64 & 1.00 & 0.86 & 0.86 \\
4002 & 100538 & 1987 & 120.98 & -0.09 & 12090.00 & 118746.36 & 1.00 & 0.98 & 0.98 \\
23533 & 103184 & 1987 & 187.99 & 0.04 & 18800.00 & 166736.58 & 1.00 & 0.89 & 0.89 \\
18167 & 102414 & 1987 & 61.10 & -0.18 & 6107.00 & 59223.48 & 1.00 & 0.97 & 0.97 \\
14940 & 101925 & 1987 & 223.29 & -0.04 & 30870.00 & 185048.67 & 0.72 & 0.83 & 0.60 \\
9470 & 101138 & 1987 & 22.09 & 0.08 & 2252.00 & 20922.39 & 0.98 & 0.95 & 0.93 \\
12869 & 101603 & 1987 & 386.31 & 0.09 & 38600.00 & 319830.50 & 1.00 & 0.83 & 0.83 \\
18211 & 102416 & 1987 & 139.10 & -0.04 & 13915.00 & 117364.35 & 1.00 & 0.84 & 0.84 \\
18214 & 102417 & 1987 & 30.00 & 0.13 & 2997.00 & 24087.43 & 1.00 & 0.80 & 0.80 \\
19209 & 102570 & 1987 & 116.83 & 0.11 & 11821.00 & 109517.65 & 0.99 & 0.94 & 0.93 \\
23507 & 103183 & 1987 & 232.61 & 0.06 & 23261.00 & 219402.49 & 1.00 & 0.94 & 0.94 \\
17061 & 102235 & 1987 & 52.03 & -0.13 & 5140.00 & 49958.01 & 1.01 & 0.96 & 0.97 \\
14862 & 101919 & 1987 & 187.66 & 0.01 & 25578.00 & 167299.99 & 0.73 & 0.89 & 0.65 \\
17027 & 102231 & 1987 & 530.84 & 0.03 & 53084.00 & 491745.33 & 1.00 & 0.93 & 0.93 \\
2123 & 100292 & 1987 & 6.50 & 0.13 & 647.00 & 6264.97 & 1.00 & 0.96 & 0.97 \\
1874 & 100247 & 1987 & 196.77 & -0.00 & 19677.00 & 158733.26 & 1.00 & 0.81 & 0.81 \\
9337 & 101132 & 1987 & 6.01 & 0.05 & 601.00 & 6268.96 & 1.00 & 1.04 & 1.04 \\
5433 & 100763 & 1987 & 52.22 & -0.08 & 7927.00 & 51275.81 & 0.66 & 0.98 & 0.65 \\
1913 & 100250 & 1987 & 20.73 & -0.05 & 2073.00 & 18664.70 & 1.00 & 0.90 & 0.90 \\
16955 & 102224 & 1987 & 24.46 & -0.07 & 2446.00 & 21062.23 & 1.00 & 0.86 & 0.86 \\
16012 & 102065 & 1987 & 38.52 & -0.01 & 3852.00 & 36915.09 & 1.00 & 0.96 & 0.96 \\
9435 & 101135 & 1987 & 10.56 & 0.07 & 1056.00 & 9574.53 & 1.00 & 0.91 & 0.91 \\
9310 & 101131 & 1987 & 76.71 & 0.04 & 7671.00 & 71048.55 & 1.00 & 0.93 & 0.93 \\
9516 & 101142 & 1987 & 2.37 & 0.01 & 237.00 & 2200.82 & 1.00 & 0.93 & 0.93 \\
47367 & 210681 & 1987 & 87.70 & -0.02 & 8405.00 & 68323.13 & 1.04 & 0.78 & 0.81 \\
16834 & 102197 & 1987 & 14.29 & -0.05 & 1429.00 & 12261.74 & 1.00 & 0.86 & 0.86 \\
9739 & 101181 & 1987 & 39.22 & -0.09 & 5962.00 & 40088.45 & 0.66 & 1.02 & 0.67 \\
18156 & 102412 & 1987 & 16.60 & 0.09 & 1659.00 & 14559.15 & 1.00 & 0.88 & 0.88 \\
9071 & 101111 & 1987 & 4.80 & -0.03 & 480.00 & 4360.45 & 1.00 & 0.91 & 0.91 \\
47732 & 221051 & 1987 & 1111.10 & 0.10 & 111110.00 & 940334.35 & 1.00 & 0.85 & 0.85 \\
9987 & 101233 & 1987 & 89.59 & 0.10 & 10202.00 & 97351.87 & 0.88 & 1.09 & 0.95 \\
7071 & 100994 & 1987 & 31.39 & 0.02 & 3067.00 & 30634.92 & 1.02 & 0.98 & 1.00 \\
6860 & 100963 & 1987 & 47.95 & -0.04 & 4795.00 & 41294.31 & 1.00 & 0.86 & 0.86 \\
14544 & 101876 & 1987 & 112.82 & 0.00 & 11321.00 & 92214.65 & 1.00 & 0.82 & 0.81 \\
17217 & 102271 & 1987 & 440.85 & 0.18 & 47171.00 & 369441.64 & 0.93 & 0.84 & 0.78 \\
9060 & 101110 & 1987 & 26.90 & 0.04 & 2690.00 & 23095.80 & 1.00 & 0.86 & 0.86 \\
4163 & 100567 & 1987 & 523.41 & 0.11 & 53208.00 & 493798.87 & 0.98 & 0.94 & 0.93 \\
9718 & 101177 & 1987 & 20.44 & -0.19 & 3517.00 & 19595.51 & 0.58 & 0.96 & 0.56 \\
9532 & 101149 & 1987 & 14.35 & -0.05 & 1435.00 & 12536.99 & 1.00 & 0.87 & 0.87 \\
12902 & 101606 & 1987 & 642.59 & -0.01 & 64300.00 & 516234.99 & 1.00 & 0.80 & 0.80 \\
6735 & 100947 & 1987 & 588.05 & -0.03 & 58805.00 & 486025.30 & 1.00 & 0.83 & 0.83 \\
9570 & 101151 & 1987 & 17.86 & -0.05 & 1786.00 & 16993.50 & 1.00 & 0.95 & 0.95 \\
6778 & 100954 & 1987 & 21.47 & 0.09 & 2147.00 & 17767.72 & 1.00 & 0.83 & 0.83 \\
7690 & 101055 & 1987 & 145.30 & -0.02 & 14520.00 & 129280.22 & 1.00 & 0.89 & 0.89 \\
23484 & 103180 & 1987 & 37.13 & -0.02 & 3713.00 & 36552.07 & 1.00 & 0.98 & 0.98 \\
6431 & 100870 & 1987 & 18.42 & 0.09 & 1197.00 & 10901.50 & 1.54 & 0.59 & 0.91 \\
16864 & 102213 & 1987 & 36.58 & -0.03 & 3660.00 & 31249.09 & 1.00 & 0.85 & 0.85 \\
23465 & 103179 & 1987 & 171.97 & 0.09 & 17198.00 & 163074.58 & 1.00 & 0.95 & 0.95 \\
15969 & 102062 & 1987 & 11.94 & -0.08 & 1193.00 & 11587.75 & 1.00 & 0.97 & 0.97 \\
15162 & 101964 & 1987 & 19.77 & -0.03 & 1977.00 & 19629.11 & 1.00 & 0.99 & 0.99 \\
19909 & 102655 & 1987 & 756.55 & 0.18 & 75655.00 & 628494.62 & 1.00 & 0.83 & 0.83 \\
4683 & 100663 & 1987 & 4.75 & -0.21 & 513.00 & 5114.36 & 0.93 & 1.08 & 1.00 \\
17468 & 102307 & 1987 & 32.05 & -0.04 & 3205.00 & 26190.02 & 1.00 & 0.82 & 0.82 \\
11587 & 101431 & 1987 & 54.60 & 0.12 & 4973.00 & 41119.80 & 1.10 & 0.75 & 0.83 \\
22204 & 102996 & 1987 & 151.97 & -0.03 & 15196.00 & 140139.03 & 1.00 & 0.92 & 0.92 \\
12202 & 101519 & 1987 & 23.30 & 0.04 & 2327.00 & 22174.70 & 1.00 & 0.95 & 0.95 \\
22173 & 102994 & 1987 & 35.91 & 0.01 & 4525.00 & 29514.55 & 0.79 & 0.82 & 0.65 \\
21895 & 102976 & 1987 & 110.00 & -0.05 & 11003.00 & 100702.98 & 1.00 & 0.92 & 0.92 \\
11635 & 101455 & 1987 & 957.92 & 0.07 & 95792.00 & 836303.24 & 1.00 & 0.87 & 0.87 \\
3011 & 100398 & 1987 & 29.81 & -0.04 & 2703.00 & 25675.52 & 1.10 & 0.86 & 0.95 \\
8055 & 101073 & 1987 & 764.50 & 0.02 & 76450.00 & 643989.91 & 1.00 & 0.84 & 0.84 \\
4613 & 100644 & 1987 & 28.46 & 0.06 & 2846.00 & 26110.03 & 1.00 & 0.92 & 0.92 \\
17424 & 102306 & 1987 & 159.19 & -0.01 & 23209.00 & 164510.17 & 0.69 & 1.03 & 0.71 \\
17700 & 102346 & 1987 & 14.76 & 0.06 & 1476.00 & 13555.38 & 1.00 & 0.92 & 0.92 \\
7379 & 101038 & 1987 & 1220.70 & 0.06 & 122060.00 & 1140571.51 & 1.00 & 0.93 & 0.93 \\
13208 & 101704 & 1987 & 273.16 & 0.06 & 27318.00 & 251427.49 & 1.00 & 0.92 & 0.92 \\
20267 & 102702 & 1987 & 80.29 & -0.02 & 8029.00 & 75505.34 & 1.00 & 0.94 & 0.94 \\
3706 & 100471 & 1987 & 6.40 & -0.08 & 639.00 & 5970.04 & 1.00 & 0.93 & 0.93 \\
2783 & 100358 & 1987 & 10.42 & 0.16 & 1042.00 & 10197.67 & 1.00 & 0.98 & 0.98 \\
16349 & 102130 & 1987 & 200.50 & 0.02 & 20050.00 & 163717.47 & 1.00 & 0.82 & 0.82 \\
22232 & 102997 & 1987 & 42.58 & 0.15 & 4258.00 & 34455.97 & 1.00 & 0.81 & 0.81 \\
21800 & 102952 & 1987 & 270.89 & -0.01 & 37745.00 & 256474.91 & 0.72 & 0.95 & 0.68 \\
8305 & 101082 & 1987 & 231.60 & -0.01 & 23160.00 & 196202.99 & 1.00 & 0.85 & 0.85 \\
20546 & 102767 & 1987 & 766.80 & 0.04 & 102000.00 & 759534.68 & 0.75 & 0.99 & 0.74 \\
7417 & 101039 & 1987 & 45.30 & 0.07 & 4530.00 & 40339.72 & 1.00 & 0.89 & 0.89 \\
21934 & 102980 & 1987 & 14.40 & -0.26 & 1447.00 & 13697.07 & 1.00 & 0.95 & 0.95 \\
2941 & 100389 & 1987 & 30.47 & 0.03 & 3028.00 & 29480.65 & 1.01 & 0.97 & 0.97 \\
21944 & 102981 & 1987 & 66.40 & -0.18 & 6751.00 & 67506.63 & 0.98 & 1.02 & 1.00 \\
17607 & 102321 & 1987 & 19.33 & 0.07 & 1933.00 & 15614.87 & 1.00 & 0.81 & 0.81 \\
15626 & 102010 & 1987 & 210.52 & -0.10 & 31741.00 & 210491.70 & 0.66 & 1.00 & 0.66 \\
20697 & 102784 & 1987 & 529.60 & 0.11 & 61096.00 & 455997.89 & 0.87 & 0.86 & 0.75 \\
13303 & 101720 & 1987 & 10.43 & 0.01 & 1014.00 & 9486.61 & 1.03 & 0.91 & 0.94 \\
17664 & 102342 & 1987 & 21.00 & -0.04 & 2085.00 & 19962.70 & 1.01 & 0.95 & 0.96 \\
12037 & 101491 & 1987 & 63.87 & -0.12 & 5593.00 & 50033.69 & 1.14 & 0.78 & 0.89 \\
15666 & 102013 & 1987 & 107.87 & 0.02 & 14823.00 & 109071.70 & 0.73 & 1.01 & 0.74 \\
20755 & 102789 & 1987 & 87.30 & -0.02 & 9125.00 & 77277.86 & 0.96 & 0.89 & 0.85 \\
2981 & 100395 & 1987 & 432.62 & 0.00 & 43466.00 & 405026.42 & 1.00 & 0.94 & 0.93 \\
2902 & 100379 & 1987 & 81.78 & 0.09 & 8174.00 & 69633.90 & 1.00 & 0.85 & 0.85 \\
15686 & 102015 & 1987 & 161.93 & 0.06 & 23492.00 & 167278.20 & 0.69 & 1.03 & 0.71 \\
11701 & 101457 & 1987 & 66.28 & 0.08 & 6628.00 & 58053.06 & 1.00 & 0.88 & 0.88 \\
8264 & 101081 & 1987 & 30.20 & -0.17 & 3020.00 & 28150.17 & 1.00 & 0.93 & 0.93 \\
5043 & 100705 & 1987 & 60.55 & 0.10 & 7840.00 & 59147.81 & 0.77 & 0.98 & 0.75 \\
20586 & 102774 & 1987 & 251.94 & 0.05 & 28266.00 & 209320.75 & 0.89 & 0.83 & 0.74 \\
14119 & 101805 & 1987 & 439.53 & -0.06 & 67928.00 & 444636.36 & 0.65 & 1.01 & 0.65 \\
20224 & 102689 & 1987 & 6.94 & 0.01 & 694.00 & 6019.67 & 1.00 & 0.87 & 0.87 \\
17371 & 102284 & 1987 & 21.08 & -0.10 & 3035.00 & 19078.18 & 0.69 & 0.90 & 0.63 \\
10554 & 101299 & 1987 & 100.47 & 0.20 & 10047.00 & 86353.22 & 1.00 & 0.86 & 0.86 \\
5121 & 100726 & 1987 & 109.23 & 0.03 & 10693.00 & 87031.85 & 1.02 & 0.80 & 0.81 \\
8728 & 101096 & 1987 & 5.90 & -0.09 & 590.00 & 5180.09 & 1.00 & 0.88 & 0.88 \\
22668 & 103028 & 1987 & 391.15 & 0.07 & 39115.00 & 331643.60 & 1.00 & 0.85 & 0.85 \\
3449 & 100439 & 1987 & 2.34 & -0.06 & 219.00 & 2128.65 & 1.07 & 0.91 & 0.97 \\
4515 & 100637 & 1987 & 20.57 & 0.03 & 2057.00 & 19813.31 & 1.00 & 0.96 & 0.96 \\
8703 & 101095 & 1987 & 111.20 & 0.16 & 11120.00 & 97803.76 & 1.00 & 0.88 & 0.88 \\
3892 & 100510 & 1987 & 13.55 & -0.04 & 1355.00 & 12745.76 & 1.00 & 0.94 & 0.94 \\
22727 & 103049 & 1987 & 125.89 & 0.01 & 12589.00 & 101702.88 & 1.00 & 0.81 & 0.81 \\
6289 & 100847 & 1987 & 2.24 & -0.07 & 224.00 & 1983.98 & 1.00 & 0.89 & 0.89 \\
10449 & 101286 & 1987 & 294.76 & -0.01 & 29476.00 & 270629.63 & 1.00 & 0.92 & 0.92 \\
8803 & 101099 & 1987 & 5.10 & -0.09 & 510.00 & 4623.88 & 1.00 & 0.91 & 0.91 \\
49050 & 240212 & 1987 & 250.80 & 0.05 & 30066.00 & 229714.08 & 0.83 & 0.92 & 0.76 \\
14102 & 101804 & 1987 & 223.92 & 0.00 & 32510.00 & 233391.73 & 0.69 & 1.04 & 0.72 \\
2596 & 100346 & 1987 & 3.16 & 0.05 & 320.00 & 3073.07 & 0.99 & 0.97 & 0.96 \\
3905 & 100514 & 1987 & 90.07 & 0.07 & 9006.00 & 75271.93 & 1.00 & 0.84 & 0.84 \\
15711 & 102016 & 1987 & 821.65 & 0.06 & 103607.00 & 849903.54 & 0.79 & 1.03 & 0.82 \\
19937 & 102659 & 1987 & 855.46 & -0.07 & 85546.00 & 735838.29 & 1.00 & 0.86 & 0.86 \\
20097 & 102667 & 1987 & 164.84 & -0.02 & 16484.00 & 164061.09 & 1.00 & 1.00 & 1.00 \\
49093 & 240222 & 1987 & 216.70 & 0.04 & 21676.00 & 183852.75 & 1.00 & 0.85 & 0.85 \\
13648 & 101754 & 1987 & 23.01 & 0.01 & 2301.00 & 19814.74 & 1.00 & 0.86 & 0.86 \\
11493 & 101425 & 1987 & 20.16 & 0.01 & 1999.00 & 20266.23 & 1.01 & 1.01 & 1.01 \\
15516 & 102000 & 1987 & 327.06 & 0.00 & 36576.00 & 334241.72 & 0.89 & 1.02 & 0.91 \\
4926 & 100695 & 1987 & 66.12 & 0.05 & 6612.00 & 54411.11 & 1.00 & 0.82 & 0.82 \\
22364 & 103008 & 1987 & 52.21 & 0.06 & 5226.00 & 46385.47 & 1.00 & 0.89 & 0.89 \\
7965 & 101068 & 1987 & 7800.10 & 0.01 & 780010.00 & 6907792.26 & 1.00 & 0.89 & 0.89 \\
18527 & 102471 & 1987 & 97.09 & -0.02 & 9709.00 & 86257.55 & 1.00 & 0.89 & 0.89 \\
18442 & 102461 & 1987 & 27.00 & 0.12 & 2698.00 & 24420.80 & 1.00 & 0.90 & 0.91 \\
13854 & 101781 & 1987 & 246.08 & 0.00 & 24602.00 & 214298.38 & 1.00 & 0.87 & 0.87 \\
11183 & 101370 & 1987 & 4.35 & 0.01 & 435.00 & 3806.81 & 1.00 & 0.87 & 0.88 \\
22400 & 103011 & 1987 & 38.16 & -0.05 & 3838.00 & 36644.99 & 0.99 & 0.96 & 0.95 \\
8519 & 101089 & 1987 & 4.40 & -0.12 & 440.00 & 3823.67 & 1.00 & 0.87 & 0.87 \\
20834 & 102796 & 1987 & 1.81 & -0.04 & 94.00 & 944.67 & 1.92 & 0.52 & 1.00 \\
13827 & 101769 & 1987 & 274.96 & 0.06 & 27493.00 & 261238.25 & 1.00 & 0.95 & 0.95 \\
8384 & 101085 & 1987 & 77.10 & -0.09 & 7700.00 & 72500.24 & 1.00 & 0.94 & 0.94 \\
9374 & 101133 & 1987 & 1.15 & -0.11 & 115.00 & 1039.80 & 1.00 & 0.90 & 0.90 \\
25174 & 103451 & 1987 & 5.14 & -0.11 & 452.00 & 4241.75 & 1.14 & 0.82 & 0.94 \\
25948 & 103531 & 1987 & 32.05 & 0.06 & 3205.00 & 26223.71 & 1.00 & 0.82 & 0.82 \\
25397 & 103483 & 1987 & 11.11 & -0.05 & 1111.00 & 10662.34 & 1.00 & 0.96 & 0.96 \\
583 & 100079 & 1987 & 101.54 & -0.05 & 10170.00 & 101135.69 & 1.00 & 1.00 & 0.99 \\
1671 & 100223 & 1987 & 160.94 & 0.05 & 14696.00 & 138806.60 & 1.10 & 0.86 & 0.94 \\
1658 & 100222 & 1987 & 53.06 & -0.02 & 4928.00 & 48153.93 & 1.08 & 0.91 & 0.98 \\
24961 & 103397 & 1987 & 16.11 & -0.12 & 1611.00 & 14797.19 & 1.00 & 0.92 & 0.92 \\
26276 & 103551 & 1987 & 7.62 & -0.02 & 762.00 & 6512.09 & 1.00 & 0.85 & 0.85 \\
1406 & 100196 & 1987 & 164.23 & 0.09 & 16423.00 & 142283.59 & 1.00 & 0.87 & 0.87 \\
24632 & 103373 & 1987 & 149.80 & 0.01 & 22534.00 & 152888.64 & 0.66 & 1.02 & 0.68 \\
26002 & 103533 & 1987 & 47.44 & -0.01 & 4744.00 & 39357.45 & 1.00 & 0.83 & 0.83 \\
25268 & 103464 & 1987 & 211.18 & 0.05 & 27226.00 & 184527.35 & 0.78 & 0.87 & 0.68 \\
106 & 100009 & 1987 & 25.80 & -0.17 & 2580.00 & 24240.79 & 1.00 & 0.94 & 0.94 \\
1467 & 100207 & 1987 & 2240.15 & 0.01 & 224015.00 & 2114291.06 & 1.00 & 0.94 & 0.94 \\
1396 & 100195 & 1987 & 17.55 & -0.17 & 1755.00 & 14783.24 & 1.00 & 0.84 & 0.84 \\
24869 & 103383 & 1987 & 216.13 & 0.06 & 21406.00 & 214370.88 & 1.01 & 0.99 & 1.00 \\
26509 & 103585 & 1987 & 17.20 & -0.08 & 1718.00 & 14730.21 & 1.00 & 0.86 & 0.86 \\
24712 & 103376 & 1987 & 1231.87 & 0.13 & 122198.00 & 1221603.80 & 1.01 & 0.99 & 1.00 \\
1125 & 100155 & 1987 & 29.92 & 0.12 & 2366.00 & 25759.22 & 1.26 & 0.86 & 1.09 \\
24672 & 103375 & 1987 & 11.10 & 0.09 & 1113.00 & 11044.39 & 1.00 & 1.00 & 0.99 \\
24749 & 103377 & 1987 & 7.43 & 0.11 & 628.00 & 6186.46 & 1.18 & 0.83 & 0.99 \\
24788 & 103380 & 1987 & 2393.30 & 0.06 & 237899.00 & 2376099.07 & 1.01 & 0.99 & 1.00 \\
25616 & 103498 & 1987 & 123.84 & -0.09 & 12375.00 & 110040.46 & 1.00 & 0.89 & 0.89 \\
1197 & 100160 & 1987 & 30.16 & -0.02 & 3016.00 & 29212.09 & 1.00 & 0.97 & 0.97 \\
1210 & 100166 & 1987 & 1612.87 & 0.12 & 161287.00 & 1320520.85 & 1.00 & 0.82 & 0.82 \\
25541 & 103496 & 1987 & 287.89 & 0.03 & 28291.00 & 285862.24 & 1.02 & 0.99 & 1.01 \\
1734 & 100227 & 1987 & 98.02 & 0.12 & 9882.00 & 92686.95 & 0.99 & 0.95 & 0.94 \\
24828 & 103381 & 1987 & 247.84 & 0.07 & 24523.00 & 257233.33 & 1.01 & 1.04 & 1.05 \\
558 & 100076 & 1987 & 203.33 & 0.04 & 20333.00 & 188912.09 & 1.00 & 0.93 & 0.93 \\
917 & 100111 & 1987 & 42.37 & -0.01 & 4250.00 & 40532.35 & 1.00 & 0.96 & 0.95 \\
25484 & 103494 & 1987 & 204.81 & 0.05 & 20475.00 & 185458.65 & 1.00 & 0.91 & 0.91 \\
1585 & 100217 & 1987 & 64.52 & 0.04 & 8132.00 & 58258.82 & 0.79 & 0.90 & 0.72 \\
1010 & 100127 & 1987 & 167.61 & 0.19 & 17201.00 & 140452.92 & 0.97 & 0.84 & 0.82 \\
966 & 100113 & 1987 & 695.68 & 0.05 & 69761.00 & 674425.72 & 1.00 & 0.97 & 0.97 \\
26091 & 103538 & 1987 & 19.65 & -0.12 & 1965.00 & 19141.15 & 1.00 & 0.97 & 0.97 \\
74634 & 601143 & 1987 & 110.60 & 0.12 & 15685.00 & 110811.45 & 0.71 & 1.00 & 0.71 \\
1495 & 100208 & 1987 & 45.08 & -0.26 & 4508.00 & 37002.12 & 1.00 & 0.82 & 0.82 \\
644 & 100087 & 1987 & 286.45 & 0.10 & 25691.00 & 276232.28 & 1.11 & 0.96 & 1.08 \\
74635 & 601143 & 1988 & 92.30 & 0.06 & 9315.00 & 88203.06 & 0.99 & 0.96 & 0.95 \\
8520 & 101089 & 1988 & 5.30 & 0.17 & 684.00 & 5624.78 & 0.77 & 1.06 & 0.82 \\
16223 & 102102 & 1988 & 27.71 & 0.04 & 2771.00 & 26325.15 & 1.00 & 0.95 & 0.95 \\
7934 & 101067 & 1988 & 79.40 & 0.21 & 7763.00 & 73113.88 & 1.02 & 0.92 & 0.94 \\
14392 & 101854 & 1988 & 678.16 & 0.26 & 68505.00 & 624753.84 & 0.99 & 0.92 & 0.91 \\
49051 & 240212 & 1988 & 230.50 & 0.11 & 24417.00 & 211551.06 & 0.94 & 0.92 & 0.87 \\
5807 & 100799 & 1988 & 25.87 & 0.16 & 3281.00 & 24223.98 & 0.79 & 0.94 & 0.74 \\
5809 & 100801 & 1988 & 15.19 & 0.02 & 1895.00 & 18905.20 & 0.80 & 1.24 & 1.00 \\
8484 & 101088 & 1988 & 10.90 & -0.08 & 613.00 & 5060.29 & 1.78 & 0.46 & 0.83 \\
16350 & 102130 & 1988 & 230.36 & 0.23 & 23036.00 & 209068.83 & 1.00 & 0.91 & 0.91 \\
7966 & 101068 & 1988 & 8971.10 & 0.23 & 901600.00 & 8240663.81 & 1.00 & 0.92 & 0.91 \\
11339 & 101396 & 1988 & 27.47 & 0.36 & 2757.00 & 28454.85 & 1.00 & 1.04 & 1.03 \\
14342 & 101851 & 1988 & 40.50 & 0.20 & 4455.00 & 34877.92 & 0.91 & 0.86 & 0.78 \\
49094 & 240222 & 1988 & 228.50 & 0.19 & 22850.00 & 194232.16 & 1.00 & 0.85 & 0.85 \\
8450 & 101087 & 1988 & 3.50 & 0.19 & 424.00 & 3440.90 & 0.83 & 0.98 & 0.81 \\
14436 & 101858 & 1988 & 64.82 & -0.05 & 6482.00 & 62613.01 & 1.00 & 0.97 & 0.97 \\
3012 & 100398 & 1988 & 34.59 & 0.21 & 3459.00 & 31033.83 & 1.00 & 0.90 & 0.90 \\
25187 & 103460 & 1988 & 137.20 & 0.14 & 13926.00 & 128808.13 & 0.99 & 0.94 & 0.92 \\
7897 & 101065 & 1988 & 39.90 & 0.04 & 3957.00 & 34040.34 & 1.01 & 0.85 & 0.86 \\
14500 & 101865 & 1988 & 18.58 & 0.09 & 1858.00 & 17649.43 & 1.00 & 0.95 & 0.95 \\
3960 & 100535 & 1988 & 110.36 & 0.16 & 11573.00 & 104822.93 & 0.95 & 0.95 & 0.91 \\
6625 & 100906 & 1988 & 375.70 & 0.16 & 37671.00 & 310372.83 & 1.00 & 0.83 & 0.82 \\
22776 & 103061 & 1988 & 2.35 & 0.26 & 235.00 & 1954.99 & 1.00 & 0.83 & 0.83 \\
10246 & 101276 & 1988 & 62.41 & 0.18 & 6240.00 & 57985.94 & 1.00 & 0.93 & 0.93 \\
24870 & 103383 & 1988 & 201.41 & 0.02 & 20140.00 & 200471.11 & 1.00 & 1.00 & 1.00 \\
19866 & 102654 & 1988 & 66.50 & 0.18 & 6638.00 & 61663.13 & 1.00 & 0.93 & 0.93 \\
22932 & 103089 & 1988 & 73.59 & 0.01 & 8294.00 & 63025.02 & 0.89 & 0.86 & 0.76 \\
26902 & 103621 & 1988 & 47.30 & 0.02 & 5236.00 & 48409.61 & 0.90 & 1.02 & 0.92 \\
11184 & 101370 & 1988 & 3.59 & -0.02 & 359.00 & 3296.28 & 1.00 & 0.92 & 0.92 \\
1735 & 100227 & 1988 & 97.80 & 0.06 & 9780.00 & 81479.25 & 1.00 & 0.83 & 0.83 \\
1496 & 100208 & 1988 & 38.90 & 0.07 & 3890.00 & 36087.20 & 1.00 & 0.93 & 0.93 \\
24829 & 103381 & 1988 & 255.64 & 0.13 & 25560.00 & 252861.94 & 1.00 & 0.99 & 0.99 \\
559 & 100076 & 1988 & 180.79 & 0.10 & 18079.00 & 173234.86 & 1.00 & 0.96 & 0.96 \\
2494 & 100336 & 1988 & 11.19 & 0.14 & 833.00 & 6973.84 & 1.34 & 0.62 & 0.84 \\
16096 & 102080 & 1988 & 90.60 & 0.17 & 9015.00 & 82200.45 & 1.01 & 0.91 & 0.91 \\
22886 & 103077 & 1988 & 18.80 & 0.12 & 1887.00 & 16188.11 & 1.00 & 0.86 & 0.86 \\
4003 & 100538 & 1988 & 138.72 & 0.26 & 13620.00 & 126582.98 & 1.02 & 0.91 & 0.93 \\
9072 & 101111 & 1988 & 8.80 & 0.26 & 1143.00 & 9735.64 & 0.77 & 1.11 & 0.85 \\
59021 & 410401 & 1988 & 569.88 & -0.08 & 62114.00 & 469779.49 & 0.92 & 0.82 & 0.76 \\
21801 & 102952 & 1988 & 292.75 & 0.11 & 32213.00 & 303160.51 & 0.91 & 1.04 & 0.94 \\
14103 & 101804 & 1988 & 213.61 & 0.14 & 22362.00 & 201379.90 & 0.96 & 0.94 & 0.90 \\
4516 & 100637 & 1988 & 30.47 & 0.43 & 3047.00 & 29585.29 & 1.00 & 0.97 & 0.97 \\
8385 & 101085 & 1988 & 82.10 & 0.16 & 9319.00 & 76857.33 & 0.88 & 0.94 & 0.82 \\
21739 & 102949 & 1988 & 499.25 & 0.30 & 49054.00 & 426997.66 & 1.02 & 0.86 & 0.87 \\
18528 & 102471 & 1988 & 73.54 & 0.18 & 7354.00 & 66593.83 & 1.00 & 0.91 & 0.91 \\
5592 & 100773 & 1988 & 934.14 & 0.04 & 93410.00 & 885631.48 & 1.00 & 0.95 & 0.95 \\
24713 & 103376 & 1988 & 1193.83 & 0.03 & 119380.00 & 1119418.43 & 1.00 & 0.94 & 0.94 \\
218 & 100019 & 1988 & 562.70 & 0.25 & 56272.00 & 470648.34 & 1.00 & 0.84 & 0.84 \\
23466 & 103179 & 1988 & 151.34 & -0.04 & 15130.00 & 144235.05 & 1.00 & 0.95 & 0.95 \\
26510 & 103585 & 1988 & 18.78 & 0.22 & 1872.00 & 17208.40 & 1.00 & 0.92 & 0.92 \\
18443 & 102461 & 1988 & 24.27 & 0.07 & 2427.00 & 22507.87 & 1.00 & 0.93 & 0.93 \\
15033 & 101953 & 1988 & 200.21 & 0.15 & 21106.00 & 202658.62 & 0.95 & 1.01 & 0.96 \\
8891 & 101104 & 1988 & 11.30 & 0.27 & 1387.00 & 11374.18 & 0.81 & 1.01 & 0.82 \\
23485 & 103180 & 1988 & 33.80 & 0.12 & 3380.00 & 32260.53 & 1.00 & 0.95 & 0.95 \\
8056 & 101073 & 1988 & 720.10 & 0.23 & 90840.00 & 758193.50 & 0.79 & 1.05 & 0.83 \\
23487 & 103182 & 1988 & 23.70 & 0.09 & 2370.00 & 20656.45 & 1.00 & 0.87 & 0.87 \\
18387 & 102447 & 1988 & 292.58 & 0.27 & 27421.00 & 228338.49 & 1.07 & 0.78 & 0.83 \\
4614 & 100644 & 1988 & 28.36 & 0.10 & 2836.00 & 25720.45 & 1.00 & 0.91 & 0.91 \\
7691 & 101055 & 1988 & 138.40 & 0.16 & 12881.00 & 121525.36 & 1.07 & 0.88 & 0.94 \\
7034 & 100992 & 1988 & 97.96 & 0.29 & 9433.00 & 92378.43 & 1.04 & 0.94 & 0.98 \\
14120 & 101805 & 1988 & 474.32 & 0.12 & 45651.00 & 413108.09 & 1.04 & 0.87 & 0.90 \\
1211 & 100166 & 1988 & 1642.52 & 0.09 & 164252.00 & 1333886.36 & 1.00 & 0.81 & 0.81 \\
9061 & 101110 & 1988 & 14.20 & 0.07 & 1349.00 & 14121.48 & 1.05 & 0.99 & 1.05 \\
23238 & 103152 & 1988 & 301.09 & 0.12 & 29549.00 & 268653.02 & 1.02 & 0.89 & 0.91 \\
13304 & 101720 & 1988 & 10.20 & 0.02 & 1003.00 & 9153.40 & 1.02 & 0.90 & 0.91 \\
1198 & 100160 & 1988 & 33.59 & 0.10 & 3616.00 & 29204.61 & 0.93 & 0.87 & 0.81 \\
4370 & 100614 & 1988 & 86.90 & 0.26 & 8686.00 & 81695.08 & 1.00 & 0.94 & 0.94 \\
9740 & 101181 & 1988 & 33.48 & 0.11 & 3858.00 & 36508.11 & 0.87 & 1.09 & 0.95 \\
4388 & 100622 & 1988 & 38.80 & 0.27 & 3882.00 & 36554.42 & 1.00 & 0.94 & 0.94 \\
96651 & 611002 & 1988 & 514.99 & 0.12 & 51499.00 & 460067.03 & 1.00 & 0.89 & 0.89 \\
2322 & 100319 & 1988 & 53.39 & 0.32 & 5339.00 & 50000.97 & 1.00 & 0.94 & 0.94 \\
24789 & 103380 & 1988 & 2265.97 & 0.06 & 226590.00 & 2261649.66 & 1.00 & 1.00 & 1.00 \\
8866 & 101103 & 1988 & 2.40 & 0.20 & 295.00 & 2330.33 & 0.81 & 0.97 & 0.79 \\
7072 & 100994 & 1988 & 23.78 & 0.12 & 2326.00 & 22454.63 & 1.02 & 0.94 & 0.97 \\
4425 & 100624 & 1988 & 32.90 & 0.15 & 3293.00 & 32605.90 & 1.00 & 0.99 & 0.99 \\
2302 & 100315 & 1988 & 240.60 & 0.16 & 24060.00 & 232206.45 & 1.00 & 0.97 & 0.97 \\
12903 & 101606 & 1988 & 599.42 & 0.07 & 59940.00 & 527129.65 & 1.00 & 0.88 & 0.88 \\
24750 & 103377 & 1988 & 58.64 & 0.16 & 5870.00 & 56559.29 & 1.00 & 0.96 & 0.96 \\
2290 & 100313 & 1988 & 57.92 & -0.00 & 6632.00 & 50708.81 & 0.87 & 0.88 & 0.76 \\
1506 & 100209 & 1988 & 198.16 & 0.04 & 19816.00 & 180751.85 & 1.00 & 0.91 & 0.91 \\
4466 & 100634 & 1988 & 204.36 & 0.11 & 20435.00 & 169128.33 & 1.00 & 0.83 & 0.83 \\
14153 & 101819 & 1988 & 96.17 & 0.23 & 9617.00 & 84243.83 & 1.00 & 0.88 & 0.88 \\
3906 & 100514 & 1988 & 87.19 & 0.12 & 8717.00 & 79512.70 & 1.00 & 0.91 & 0.91 \\
19910 & 102655 & 1988 & 727.44 & 0.10 & 72648.00 & 642318.96 & 1.00 & 0.88 & 0.88 \\
63166 & 500486 & 1988 & 106.30 & 0.20 & 10793.00 & 100084.22 & 0.98 & 0.94 & 0.93 \\
25269 & 103464 & 1988 & 211.15 & 0.07 & 20813.00 & 197984.66 & 1.01 & 0.94 & 0.95 \\
3450 & 100439 & 1988 & 2.98 & 0.23 & 302.00 & 2715.60 & 0.99 & 0.91 & 0.90 \\
22205 & 102996 & 1988 & 114.56 & 0.02 & 12767.00 & 125267.75 & 0.90 & 1.09 & 0.98 \\
6736 & 100947 & 1988 & 712.14 & 0.23 & 71214.00 & 655526.25 & 1.00 & 0.92 & 0.92 \\
20835 & 102796 & 1988 & 2.06 & 0.18 & 129.00 & 1293.26 & 1.59 & 0.63 & 1.00 \\
6319 & 100849 & 1988 & 34.23 & 0.11 & 3423.00 & 29176.88 & 1.00 & 0.85 & 0.85 \\
8729 & 101096 & 1988 & 3.40 & 0.05 & 349.00 & 3540.39 & 0.97 & 1.04 & 1.01 \\
10185 & 101268 & 1988 & 53.07 & 0.05 & 5307.00 & 46205.88 & 1.00 & 0.87 & 0.87 \\
22233 & 102997 & 1988 & 38.67 & 0.07 & 3948.00 & 36759.70 & 0.98 & 0.95 & 0.93 \\
20756 & 102789 & 1988 & 79.10 & 0.18 & 8310.00 & 63951.10 & 0.95 & 0.81 & 0.77 \\
1407 & 100196 & 1988 & 169.53 & 0.16 & 16953.00 & 156669.26 & 1.00 & 0.92 & 0.92 \\
1397 & 100195 & 1988 & 11.70 & 0.12 & 1170.00 & 9964.30 & 1.00 & 0.85 & 0.85 \\
20698 & 102784 & 1988 & 545.10 & 0.12 & 54980.00 & 533506.93 & 0.99 & 0.98 & 0.97 \\
2784 & 100358 & 1988 & 10.07 & 0.02 & 1007.00 & 9098.05 & 1.00 & 0.90 & 0.90 \\
15517 & 102000 & 1988 & 341.67 & 0.18 & 33408.00 & 324270.47 & 1.02 & 0.95 & 0.97 \\
20643 & 102777 & 1988 & 146.90 & 0.16 & 14350.00 & 128713.05 & 1.02 & 0.88 & 0.90 \\
8704 & 101095 & 1988 & 26.60 & 0.11 & 2142.00 & 21582.18 & 1.24 & 0.81 & 1.01 \\
15163 & 101964 & 1988 & 22.21 & 0.10 & 2063.00 & 20126.65 & 1.08 & 0.91 & 0.98 \\
6290 & 100847 & 1988 & 1.90 & 0.28 & 190.00 & 1795.41 & 1.00 & 0.95 & 0.94 \\
22174 & 102994 & 1988 & 35.04 & 0.20 & 3490.00 & 34368.91 & 1.00 & 0.98 & 0.98 \\
14941 & 101925 & 1988 & 188.38 & 0.16 & 23483.00 & 190351.12 & 0.80 & 1.01 & 0.81 \\
12870 & 101603 & 1988 & 357.97 & 0.09 & 35800.00 & 298546.35 & 1.00 & 0.83 & 0.83 \\
2942 & 100389 & 1988 & 30.99 & 0.11 & 3115.00 & 28420.44 & 0.99 & 0.92 & 0.91 \\
25062 & 103429 & 1988 & 214.92 & 0.09 & 21500.00 & 175143.15 & 1.00 & 0.81 & 0.81 \\
10450 & 101286 & 1988 & 306.74 & 0.25 & 30674.00 & 296593.61 & 1.00 & 0.97 & 0.97 \\
6658 & 100908 & 1988 & 2.83 & 0.06 & 282.00 & 2355.37 & 1.00 & 0.83 & 0.84 \\
8808 & 101100 & 1988 & 29.60 & -0.00 & 2031.00 & 18034.65 & 1.46 & 0.61 & 0.89 \\
8804 & 101099 & 1988 & 2.70 & 0.09 & 258.00 & 2357.70 & 1.05 & 0.87 & 0.91 \\
1468 & 100207 & 1988 & 2273.55 & 0.12 & 227355.00 & 2107721.45 & 1.00 & 0.93 & 0.93 \\
21219 & 102838 & 1988 & 71.00 & 0.00 & 7148.00 & 64777.15 & 0.99 & 0.91 & 0.91 \\
6432 & 100870 & 1988 & 21.21 & 0.16 & 1372.00 & 11576.60 & 1.55 & 0.55 & 0.84 \\
2903 & 100379 & 1988 & 99.10 & 0.14 & 9798.00 & 83197.76 & 1.01 & 0.84 & 0.85 \\
21151 & 102835 & 1988 & 20.04 & 0.11 & 2204.00 & 18538.81 & 0.91 & 0.93 & 0.84 \\
47175 & 200342 & 1988 & 124.80 & 0.29 & 11650.00 & 112451.39 & 1.07 & 0.90 & 0.97 \\
8956 & 101107 & 1988 & 702.80 & 0.24 & 70162.00 & 641368.67 & 1.00 & 0.91 & 0.91 \\
6367 & 100856 & 1988 & 80.41 & 0.09 & 8041.00 & 68219.75 & 1.00 & 0.85 & 0.85 \\
20587 & 102774 & 1988 & 237.90 & 0.15 & 24945.00 & 240392.13 & 0.95 & 1.01 & 0.96 \\
3386 & 100430 & 1988 & 57.44 & 0.14 & 5738.00 & 56974.95 & 1.00 & 0.99 & 0.99 \\
14863 & 101919 & 1988 & 157.62 & 0.03 & 17796.00 & 146675.28 & 0.89 & 0.93 & 0.82 \\
10555 & 101299 & 1988 & 105.92 & 0.09 & 10592.00 & 103756.22 & 1.00 & 0.98 & 0.98 \\
21047 & 102825 & 1988 & 102.39 & 0.02 & 11116.00 & 92934.40 & 0.92 & 0.91 & 0.84 \\
12813 & 101601 & 1988 & 155.31 & 0.13 & 15530.00 & 124969.49 & 1.00 & 0.80 & 0.80 \\
47240 & 200344 & 1988 & 839.79 & 0.15 & 83979.00 & 706401.85 & 1.00 & 0.84 & 0.84 \\
22365 & 103008 & 1988 & 47.80 & 0.04 & 4775.00 & 44835.14 & 1.00 & 0.94 & 0.94 \\
25398 & 103483 & 1988 & 15.30 & 0.19 & 1530.00 & 14444.33 & 1.00 & 0.94 & 0.94 \\
57816 & 401072 & 1988 & 43.30 & 0.04 & 5962.00 & 43452.76 & 0.73 & 1.00 & 0.73 \\
22669 & 103028 & 1988 & 470.69 & 0.13 & 47069.00 & 401257.20 & 1.00 & 0.85 & 0.85 \\
15850 & 102048 & 1988 & 200.69 & 0.14 & 20103.00 & 197317.50 & 1.00 & 0.98 & 0.98 \\
6861 & 100963 & 1988 & 72.67 & 0.22 & 7267.00 & 60822.99 & 1.00 & 0.84 & 0.84 \\
20098 & 102667 & 1988 & 185.72 & 0.15 & 18572.00 & 179300.06 & 1.00 & 0.97 & 0.97 \\
15970 & 102062 & 1988 & 8.84 & 0.18 & 884.00 & 7415.56 & 1.00 & 0.84 & 0.84 \\
10035 & 101256 & 1988 & 3.85 & 0.32 & 385.00 & 3231.30 & 1.00 & 0.84 & 0.84 \\
1337 & 100190 & 1988 & 504.99 & 0.12 & 50499.00 & 443552.70 & 1.00 & 0.88 & 0.88 \\
10406 & 101285 & 1988 & 31.87 & 0.06 & 3187.00 & 30159.11 & 1.00 & 0.95 & 0.95 \\
21896 & 102976 & 1988 & 162.12 & 0.15 & 16212.00 & 162220.85 & 1.00 & 1.00 & 1.00 \\
2597 & 100346 & 1988 & 3.21 & 0.04 & 320.00 & 3205.86 & 1.00 & 1.00 & 1.00 \\
3880 & 100509 & 1988 & 9.44 & 0.20 & 944.00 & 7819.94 & 1.00 & 0.83 & 0.83 \\
16013 & 102065 & 1988 & 33.31 & 0.05 & 3667.00 & 31859.59 & 0.91 & 0.96 & 0.87 \\
3893 & 100510 & 1988 & 13.78 & 0.19 & 1378.00 & 13520.31 & 1.00 & 0.98 & 0.98 \\
22728 & 103049 & 1988 & 146.55 & 0.19 & 14655.00 & 120274.64 & 1.00 & 0.82 & 0.82 \\
107 & 100009 & 1988 & 19.56 & 0.13 & 1985.00 & 20045.38 & 0.99 & 1.02 & 1.01 \\
19938 & 102659 & 1988 & 915.35 & 0.14 & 91457.00 & 814835.98 & 1.00 & 0.89 & 0.89 \\
20225 & 102689 & 1988 & 6.90 & 0.10 & 690.00 & 6290.73 & 1.00 & 0.91 & 0.91 \\
22625 & 103027 & 1988 & 75.54 & 0.09 & 7554.00 & 78950.48 & 1.00 & 1.05 & 1.05 \\
20547 & 102767 & 1988 & 610.60 & 0.07 & 72316.00 & 656600.23 & 0.84 & 1.08 & 0.91 \\
15561 & 102005 & 1988 & 722.12 & 0.09 & 84677.00 & 716356.09 & 0.85 & 0.99 & 0.85 \\
6779 & 100954 & 1988 & 24.40 & 0.25 & 2440.00 & 20177.98 & 1.00 & 0.83 & 0.83 \\
24962 & 103397 & 1988 & 17.30 & 0.18 & 1730.00 & 15736.05 & 1.00 & 0.91 & 0.91 \\
2982 & 100395 & 1988 & 520.01 & 0.20 & 52001.00 & 437798.45 & 1.00 & 0.84 & 0.84 \\
1659 & 100222 & 1988 & 55.94 & 0.19 & 5594.00 & 55599.10 & 1.00 & 0.99 & 0.99 \\
10756 & 101330 & 1988 & 49.00 & 0.30 & 5096.00 & 40186.00 & 0.96 & 0.82 & 0.79 \\
22401 & 103011 & 1988 & 41.94 & 0.17 & 4190.00 & 37357.68 & 1.00 & 0.89 & 0.89 \\
15627 & 102010 & 1988 & 282.32 & 0.14 & 24280.00 & 217126.30 & 1.16 & 0.77 & 0.89 \\
645 & 100087 & 1988 & 291.03 & 0.11 & 28816.00 & 275662.07 & 1.01 & 0.95 & 0.96 \\
1672 & 100223 & 1988 & 148.90 & 0.12 & 10455.00 & 101028.21 & 1.42 & 0.68 & 0.97 \\
12725 & 101591 & 1988 & 25.35 & 0.24 & 2535.00 & 21759.43 & 1.00 & 0.86 & 0.86 \\
15667 & 102013 & 1988 & 113.36 & 0.27 & 11622.00 & 94217.05 & 0.98 & 0.83 & 0.81 \\
8667 & 101094 & 1988 & 26.80 & 0.09 & 1819.00 & 17897.08 & 1.47 & 0.67 & 0.98 \\
9024 & 101109 & 1988 & 3.50 & 0.16 & 471.00 & 3379.09 & 0.74 & 0.97 & 0.72 \\
15712 & 102016 & 1988 & 687.40 & 0.04 & 77852.00 & 755522.20 & 0.88 & 1.10 & 0.97 \\
12743 & 101592 & 1988 & 92.43 & 0.19 & 9243.00 & 77168.05 & 1.00 & 0.83 & 0.83 \\
22557 & 103021 & 1988 & 17.80 & 0.22 & 1780.00 & 16359.95 & 1.00 & 0.92 & 0.92 \\
365 & 100044 & 1988 & 3.32 & 0.01 & 230.00 & 2232.26 & 1.44 & 0.67 & 0.97 \\
584 & 100079 & 1988 & 165.00 & 0.18 & 15638.00 & 162771.33 & 1.06 & 0.99 & 1.04 \\
20278 & 102709 & 1988 & 9.20 & 0.02 & 980.00 & 8851.31 & 0.94 & 0.96 & 0.90 \\
3707 & 100471 & 1988 & 7.19 & 0.20 & 738.00 & 6401.08 & 0.97 & 0.89 & 0.87 \\
15687 & 102015 & 1988 & 255.70 & 0.05 & 15021.00 & 124112.49 & 1.70 & 0.49 & 0.83 \\
20268 & 102702 & 1988 & 82.10 & 0.11 & 9228.00 & 74990.74 & 0.89 & 0.91 & 0.81 \\
24673 & 103375 & 1988 & 14.18 & 0.11 & 1420.00 & 14124.25 & 1.00 & 1.00 & 0.99 \\
18215 & 102417 & 1988 & 32.70 & 0.20 & 3271.00 & 26402.36 & 1.00 & 0.81 & 0.81 \\
7266 & 101018 & 1988 & 38.20 & 0.03 & 3950.00 & 36558.87 & 0.97 & 0.96 & 0.93 \\
16865 & 102213 & 1988 & 39.23 & 0.23 & 3923.00 & 35875.58 & 1.00 & 0.91 & 0.91 \\
5236 & 100741 & 1988 & 138.85 & 0.14 & 13779.00 & 136655.46 & 1.01 & 0.98 & 0.99 \\
12203 & 101519 & 1988 & 20.42 & 0.19 & 2042.00 & 19816.92 & 1.00 & 0.97 & 0.97 \\
16956 & 102224 & 1988 & 25.61 & 0.25 & 2561.00 & 22358.48 & 1.00 & 0.87 & 0.87 \\
11702 & 101457 & 1988 & 59.80 & 0.07 & 5977.00 & 54624.56 & 1.00 & 0.91 & 0.91 \\
18157 & 102412 & 1988 & 16.40 & 0.12 & 1636.00 & 13310.80 & 1.00 & 0.81 & 0.81 \\
13541 & 101743 & 1988 & 7.14 & 0.19 & 714.00 & 5958.30 & 1.00 & 0.83 & 0.83 \\
17469 & 102307 & 1988 & 32.50 & 0.13 & 2635.00 & 26376.67 & 1.23 & 0.81 & 1.00 \\
17062 & 102235 & 1988 & 51.69 & 0.08 & 5618.00 & 55625.05 & 0.92 & 1.08 & 0.99 \\
13446 & 101740 & 1988 & 377.69 & 0.17 & 37768.00 & 335985.30 & 1.00 & 0.89 & 0.89 \\
57831 & 401082 & 1988 & 3.93 & 0.21 & 345.00 & 2783.24 & 1.14 & 0.71 & 0.81 \\
18309 & 102425 & 1988 & 514.50 & 0.18 & 51447.00 & 428502.32 & 1.00 & 0.83 & 0.83 \\
4796 & 100682 & 1988 & 23.00 & 0.20 & 2300.00 & 21590.20 & 1.00 & 0.94 & 0.94 \\
26092 & 103538 & 1988 & 16.46 & 0.09 & 1652.00 & 13656.28 & 1.00 & 0.83 & 0.83 \\
7189 & 101013 & 1988 & 9.30 & 0.04 & 928.00 & 8490.45 & 1.00 & 0.91 & 0.91 \\
9311 & 101131 & 1988 & 91.66 & 0.09 & 9166.00 & 73969.21 & 1.00 & 0.81 & 0.81 \\
9571 & 101151 & 1988 & 18.06 & 0.14 & 1806.00 & 16553.04 & 1.00 & 0.92 & 0.92 \\
1914 & 100250 & 1988 & 20.00 & 0.09 & 2000.00 & 18048.72 & 1.00 & 0.90 & 0.90 \\
8197 & 101079 & 1988 & 75.10 & 0.16 & 6747.00 & 54644.35 & 1.11 & 0.73 & 0.81 \\
4684 & 100663 & 1988 & 4.30 & 0.17 & 465.00 & 4603.21 & 0.93 & 1.07 & 0.99 \\
4927 & 100695 & 1988 & 68.73 & 0.12 & 6922.00 & 57513.07 & 0.99 & 0.84 & 0.83 \\
17665 & 102342 & 1988 & 19.23 & 0.13 & 1812.00 & 18117.20 & 1.06 & 0.94 & 1.00 \\
8306 & 101082 & 1988 & 181.10 & 0.09 & 18565.00 & 169274.46 & 0.98 & 0.93 & 0.91 \\
1011 & 100127 & 1988 & 169.86 & 0.01 & 17077.00 & 153079.66 & 0.99 & 0.90 & 0.90 \\
12053 & 101494 & 1988 & 101.50 & 0.11 & 10998.00 & 84485.64 & 0.92 & 0.83 & 0.77 \\
5279 & 100746 & 1988 & 49.15 & 0.02 & 4974.00 & 41668.38 & 0.99 & 0.85 & 0.84 \\
9338 & 101132 & 1988 & 5.99 & 0.15 & 599.00 & 5844.08 & 1.00 & 0.98 & 0.98 \\
2124 & 100292 & 1988 & 36.22 & 0.40 & 3478.00 & 33437.96 & 1.04 & 0.92 & 0.96 \\
17425 & 102306 & 1988 & 239.20 & 0.21 & 17746.00 & 143989.73 & 1.35 & 0.60 & 0.81 \\
23534 & 103184 & 1988 & 229.73 & 0.20 & 22970.00 & 193378.54 & 1.00 & 0.84 & 0.84 \\
16835 & 102197 & 1988 & 19.97 & 0.30 & 1997.00 & 18999.91 & 1.00 & 0.95 & 0.95 \\
9375 & 101133 & 1988 & 1.04 & 0.17 & 104.00 & 882.87 & 1.00 & 0.85 & 0.85 \\
25949 & 103531 & 1988 & 38.66 & 0.21 & 3866.00 & 33103.68 & 1.00 & 0.86 & 0.86 \\
8347 & 101084 & 1988 & 148.10 & -0.03 & 9742.00 & 90053.70 & 1.52 & 0.61 & 0.92 \\
9436 & 101135 & 1988 & 9.78 & 0.16 & 978.00 & 8502.06 & 1.00 & 0.87 & 0.87 \\
7302 & 101020 & 1988 & 1573.90 & 0.03 & 140120.00 & 1225327.11 & 1.12 & 0.78 & 0.87 \\
17701 & 102346 & 1988 & 7.36 & -0.04 & 736.00 & 6824.05 & 1.00 & 0.93 & 0.93 \\
17608 & 102321 & 1988 & 33.03 & 0.27 & 3303.00 & 29038.87 & 1.00 & 0.88 & 0.88 \\
18342 & 102441 & 1988 & 268.50 & 0.14 & 26852.00 & 219102.31 & 1.00 & 0.82 & 0.82 \\
12038 & 101491 & 1988 & 70.29 & 0.25 & 6456.00 & 59986.05 & 1.09 & 0.85 & 0.93 \\
5033 & 100704 & 1988 & 23.28 & 0.01 & 3380.00 & 24157.88 & 0.69 & 1.04 & 0.71 \\
476 & 100071 & 1988 & 63.20 & 0.04 & 8460.00 & 61044.81 & 0.75 & 0.97 & 0.72 \\
17537 & 102318 & 1988 & 1107.57 & 0.18 & 110757.00 & 983890.71 & 1.00 & 0.89 & 0.89 \\
17372 & 102284 & 1988 & 21.82 & 0.32 & 2571.00 & 20449.66 & 0.85 & 0.94 & 0.80 \\
5434 & 100763 & 1988 & 54.26 & 0.27 & 5426.00 & 46201.88 & 1.00 & 0.85 & 0.85 \\
17028 & 102231 & 1988 & 498.04 & 0.07 & 51959.00 & 467269.83 & 0.96 & 0.94 & 0.90 \\
918 & 100111 & 1988 & 44.37 & 0.20 & 4444.00 & 43143.76 & 1.00 & 0.97 & 0.97 \\
7418 & 101039 & 1988 & 100.50 & 0.23 & 10050.00 & 93257.83 & 1.00 & 0.93 & 0.93 \\
17178 & 102268 & 1988 & 12.88 & 0.07 & 1288.00 & 10415.80 & 1.00 & 0.81 & 0.81 \\
9517 & 101142 & 1988 & 4.06 & 0.29 & 406.00 & 3657.43 & 1.00 & 0.90 & 0.90 \\
18259 & 102419 & 1988 & 194.00 & 0.08 & 19397.00 & 163927.96 & 1.00 & 0.84 & 0.85 \\
9268 & 101127 & 1988 & 63.12 & 0.13 & 6311.00 & 58111.50 & 1.00 & 0.92 & 0.92 \\
7228 & 101015 & 1988 & 198.10 & 0.23 & 19806.00 & 181002.93 & 1.00 & 0.91 & 0.91 \\
967 & 100113 & 1988 & 760.62 & 0.17 & 73310.00 & 725994.45 & 1.04 & 0.95 & 0.99 \\
1126 & 100155 & 1988 & 40.37 & 0.12 & 3787.00 & 35405.39 & 1.07 & 0.88 & 0.93 \\
18212 & 102416 & 1988 & 138.30 & 0.21 & 13828.00 & 115543.94 & 1.00 & 0.84 & 0.84 \\
23508 & 103183 & 1988 & 174.51 & 0.01 & 17450.00 & 169967.63 & 1.00 & 0.97 & 0.97 \\
8265 & 101081 & 1988 & 26.20 & 0.12 & 3173.00 & 28974.65 & 0.83 & 1.11 & 0.91 \\
26003 & 103533 & 1988 & 54.52 & 0.28 & 5452.00 & 49219.47 & 1.00 & 0.90 & 0.90 \\
9533 & 101149 & 1988 & 14.53 & 0.14 & 1453.00 & 11873.99 & 1.00 & 0.82 & 0.82 \\
7380 & 101038 & 1988 & 1432.30 & 0.19 & 143234.00 & 1340736.02 & 1.00 & 0.94 & 0.94 \\
8236 & 101080 & 1988 & 66.60 & 0.26 & 6248.00 & 53371.95 & 1.07 & 0.80 & 0.85 \\
24633 & 103373 & 1988 & 163.51 & 0.15 & 16615.00 & 149861.06 & 0.98 & 0.92 & 0.90 \\
26277 & 103551 & 1988 & 7.95 & 0.11 & 795.00 & 7022.72 & 1.00 & 0.88 & 0.88 \\
13649 & 101754 & 1988 & 23.65 & 0.07 & 2365.00 & 19360.12 & 1.00 & 0.82 & 0.82 \\
14010 & 101800 & 1988 & 285.78 & 0.11 & 32820.00 & 254074.46 & 0.87 & 0.89 & 0.77 \\
11636 & 101455 & 1988 & 1121.37 & 0.18 & 112136.00 & 915981.51 & 1.00 & 0.82 & 0.82 \\
1875 & 100247 & 1988 & 189.80 & 0.08 & 18979.00 & 165684.17 & 1.00 & 0.87 & 0.87 \\
20291 & 102715 & 1989 & 628.80 & 0.32 & 54149.00 & 569016.90 & 1.16 & 0.90 & 1.05 \\
17666 & 102342 & 1989 & 27.22 & 0.33 & 2709.00 & 26534.79 & 1.00 & 0.97 & 0.98 \\
8668 & 101094 & 1989 & 11.00 & 0.12 & 1095.00 & 8931.63 & 1.00 & 0.81 & 0.82 \\
20371 & 102732 & 1989 & 359.67 & -0.00 & 38448.00 & 332798.99 & 0.94 & 0.93 & 0.87 \\
17179 & 102268 & 1989 & 12.90 & 0.14 & 1285.00 & 12602.66 & 1.00 & 0.98 & 0.98 \\
8705 & 101095 & 1989 & 4.90 & 0.04 & 524.00 & 4702.89 & 0.94 & 0.96 & 0.90 \\
20588 & 102774 & 1989 & 265.20 & 0.26 & 26939.00 & 257129.20 & 0.98 & 0.97 & 0.95 \\
22366 & 103008 & 1989 & 52.30 & 0.27 & 5230.00 & 50027.52 & 1.00 & 0.96 & 0.96 \\
20548 & 102767 & 1989 & 691.70 & 0.34 & 69918.00 & 702269.11 & 0.99 & 1.02 & 1.00 \\
17609 & 102321 & 1989 & 54.21 & 0.17 & 5421.00 & 44944.70 & 1.00 & 0.83 & 0.83 \\
15562 & 102005 & 1989 & 650.93 & 0.28 & 80856.00 & 712473.87 & 0.81 & 1.09 & 0.88 \\
6780 & 100954 & 1989 & 31.48 & 0.25 & 3148.00 & 25958.47 & 1.00 & 0.82 & 0.82 \\
20368 & 102730 & 1989 & 65.69 & 0.02 & 6658.00 & 53696.78 & 0.99 & 0.82 & 0.81 \\
24963 & 103397 & 1989 & 14.88 & 0.18 & 1488.00 & 13682.14 & 1.00 & 0.92 & 0.92 \\
13542 & 101743 & 1989 & 67.84 & 0.51 & 6784.00 & 57325.17 & 1.00 & 0.85 & 0.85 \\
1660 & 100222 & 1989 & 68.00 & 0.34 & 6843.00 & 63213.66 & 0.99 & 0.93 & 0.92 \\
22402 & 103011 & 1989 & 47.10 & 0.24 & 4710.00 & 37925.30 & 1.00 & 0.81 & 0.81 \\
15628 & 102010 & 1989 & 301.10 & 0.26 & 21482.00 & 197468.29 & 1.40 & 0.66 & 0.92 \\
1673 & 100223 & 1989 & 148.21 & 0.12 & 10816.00 & 105460.46 & 1.37 & 0.71 & 0.98 \\
24223 & 103299 & 1989 & 45.94 & 0.31 & 5168.00 & 44415.64 & 0.89 & 0.97 & 0.86 \\
1996 & 100280 & 1989 & 106.90 & 0.01 & 10690.00 & 95226.73 & 1.00 & 0.89 & 0.89 \\
26278 & 103551 & 1989 & 7.74 & 0.34 & 985.00 & 8966.70 & 0.79 & 1.16 & 0.91 \\
10066 & 101258 & 1989 & 74.86 & 0.01 & 7500.00 & 70517.09 & 1.00 & 0.94 & 0.94 \\
19939 & 102659 & 1989 & 1043.12 & 0.26 & 104312.00 & 920014.58 & 1.00 & 0.88 & 0.88 \\
1338 & 100190 & 1989 & 601.62 & 0.31 & 60162.00 & 525831.63 & 1.00 & 0.87 & 0.87 \\
2598 & 100346 & 1989 & 3.79 & 0.21 & 380.00 & 3573.33 & 1.00 & 0.94 & 0.94 \\
8266 & 101081 & 1989 & 26.20 & 0.19 & 2984.00 & 21171.01 & 0.88 & 0.81 & 0.71 \\
3881 & 100509 & 1989 & 13.21 & 0.14 & 1323.00 & 11867.57 & 1.00 & 0.90 & 0.90 \\
17029 & 102231 & 1989 & 555.00 & 0.34 & 55436.00 & 525651.24 & 1.00 & 0.95 & 0.95 \\
450 & 100056 & 1989 & 8.30 & 0.14 & 828.00 & 7207.78 & 1.00 & 0.87 & 0.87 \\
3894 & 100510 & 1989 & 15.11 & 0.33 & 1511.00 & 14665.49 & 1.00 & 0.97 & 0.97 \\
22777 & 103061 & 1989 & 3.90 & 0.47 & 386.00 & 3782.91 & 1.01 & 0.97 & 0.98 \\
24871 & 103383 & 1989 & 291.28 & 0.52 & 29100.00 & 285386.93 & 1.00 & 0.98 & 0.98 \\
19867 & 102654 & 1989 & 134.06 & 0.38 & 13406.00 & 121370.65 & 1.00 & 0.91 & 0.91 \\
8593 & 101091 & 1989 & 23.00 & 0.39 & 1922.00 & 15815.01 & 1.20 & 0.69 & 0.82 \\
13856 & 101781 & 1989 & 263.36 & 0.26 & 26336.00 & 230939.78 & 1.00 & 0.88 & 0.88 \\
9269 & 101127 & 1989 & 91.29 & 0.39 & 9129.00 & 81423.43 & 1.00 & 0.89 & 0.89 \\
560 & 100076 & 1989 & 187.81 & 0.29 & 18780.00 & 180392.97 & 1.00 & 0.96 & 0.96 \\
19911 & 102655 & 1989 & 826.58 & 0.24 & 82658.00 & 702172.34 & 1.00 & 0.85 & 0.85 \\
3907 & 100514 & 1989 & 87.43 & 0.31 & 8743.00 & 81549.17 & 1.00 & 0.93 & 0.93 \\
20004 & 102663 & 1989 & 23.80 & 0.41 & 2380.00 & 19765.03 & 1.00 & 0.83 & 0.83 \\
13829 & 101769 & 1989 & 327.29 & 0.31 & 32729.00 & 282451.13 & 1.00 & 0.86 & 0.86 \\
9025 & 101109 & 1989 & 37.60 & 0.58 & 3827.00 & 32695.60 & 0.98 & 0.87 & 0.85 \\
10062 & 101257 & 1989 & 5.72 & 0.05 & 664.00 & 5672.08 & 0.86 & 0.99 & 0.85 \\
24188 & 103296 & 1989 & 347.89 & 0.37 & 37349.00 & 335385.77 & 0.93 & 0.96 & 0.90 \\
17075 & 102253 & 1989 & 2.38 & 0.17 & 238.00 & 1902.44 & 1.00 & 0.80 & 0.80 \\
15713 & 102016 & 1989 & 900.73 & 0.36 & 90658.00 & 817400.73 & 0.99 & 0.91 & 0.90 \\
22558 & 103021 & 1989 & 27.70 & 0.11 & 2776.00 & 25857.86 & 1.00 & 0.93 & 0.93 \\
8237 & 101080 & 1989 & 59.60 & 0.31 & 6893.00 & 55710.38 & 0.86 & 0.93 & 0.81 \\
20269 & 102702 & 1989 & 84.20 & 0.12 & 8987.00 & 70081.80 & 0.94 & 0.83 & 0.78 \\
6824 & 100962 & 1989 & 258.19 & 0.09 & 25819.00 & 213284.90 & 1.00 & 0.83 & 0.83 \\
585 & 100079 & 1989 & 262.67 & 0.31 & 20415.00 & 262438.11 & 1.29 & 1.00 & 1.29 \\
8656 & 101093 & 1989 & 5.10 & 0.26 & 679.00 & 5114.30 & 0.75 & 1.00 & 0.75 \\
13819 & 101768 & 1989 & 6.87 & 0.16 & 589.00 & 5245.87 & 1.17 & 0.76 & 0.89 \\
20226 & 102689 & 1989 & 8.07 & 0.28 & 807.00 & 7115.64 & 1.00 & 0.88 & 0.88 \\
4928 & 100695 & 1989 & 79.03 & 0.26 & 7521.00 & 62314.73 & 1.05 & 0.79 & 0.83 \\
20099 & 102667 & 1989 & 279.10 & 0.22 & 27910.00 & 241912.80 & 1.00 & 0.87 & 0.87 \\
22705 & 103029 & 1989 & 279.57 & 0.23 & 27957.00 & 290534.14 & 1.00 & 1.04 & 1.04 \\
7860 & 101064 & 1989 & 259.80 & 0.22 & 19125.00 & 155402.51 & 1.36 & 0.60 & 0.81 \\
22670 & 103028 & 1989 & 954.46 & 0.30 & 95450.00 & 826384.20 & 1.00 & 0.87 & 0.87 \\
15834 & 102043 & 1989 & 37.71 & 0.03 & 3879.00 & 32832.97 & 0.97 & 0.87 & 0.85 \\
15833 & 102040 & 1989 & 15.80 & 0.17 & 1674.00 & 14405.11 & 0.94 & 0.91 & 0.86 \\
25399 & 103483 & 1989 & 23.01 & 0.28 & 2301.00 & 22252.48 & 1.00 & 0.97 & 0.97 \\
7267 & 101018 & 1989 & 64.30 & 0.36 & 5375.00 & 50820.65 & 1.20 & 0.79 & 0.95 \\
20644 & 102777 & 1989 & 183.70 & 0.29 & 15836.00 & 167419.89 & 1.16 & 0.91 & 1.06 \\
25270 & 103464 & 1989 & 235.74 & 0.27 & 23093.00 & 216635.75 & 1.02 & 0.92 & 0.94 \\
65074 & 500660 & 1989 & 356.98 & 0.65 & 37869.00 & 292138.64 & 0.94 & 0.82 & 0.77 \\
5237 & 100741 & 1989 & 152.55 & 0.32 & 15261.00 & 147187.98 & 1.00 & 0.96 & 0.96 \\
25101 & 103432 & 1989 & 758.36 & 0.19 & 75801.00 & 611963.63 & 1.00 & 0.81 & 0.81 \\
15164 & 101964 & 1989 & 22.63 & 0.28 & 2263.00 & 22446.99 & 1.00 & 0.99 & 0.99 \\
14942 & 101925 & 1989 & 190.19 & 0.20 & 20036.00 & 183515.80 & 0.95 & 0.96 & 0.92 \\
5123 & 100726 & 1989 & 91.25 & 0.25 & 9874.00 & 81963.58 & 0.92 & 0.90 & 0.83 \\
2983 & 100395 & 1989 & 648.66 & 0.25 & 64870.00 & 535843.76 & 1.00 & 0.83 & 0.83 \\
646 & 100087 & 1989 & 435.73 & 0.36 & 39193.00 & 416728.90 & 1.11 & 0.96 & 1.06 \\
17373 & 102284 & 1989 & 25.00 & 0.14 & 2500.00 & 21676.25 & 1.00 & 0.87 & 0.87 \\
8918 & 101105 & 1989 & 1.70 & 0.28 & 203.00 & 1857.16 & 0.84 & 1.09 & 0.91 \\
10407 & 101285 & 1989 & 37.66 & 0.30 & 3771.00 & 30409.76 & 1.00 & 0.81 & 0.81 \\
8842 & 101102 & 1989 & 6.10 & 0.30 & 755.00 & 6681.95 & 0.81 & 1.10 & 0.89 \\
47369 & 210681 & 1989 & 296.60 & 0.43 & 18765.00 & 165460.02 & 1.58 & 0.56 & 0.88 \\
63167 & 500486 & 1989 & 125.60 & 0.35 & 11604.00 & 106194.52 & 1.08 & 0.85 & 0.92 \\
1012 & 100127 & 1989 & 141.73 & 0.06 & 14317.00 & 126741.41 & 0.99 & 0.89 & 0.89 \\
1586 & 100217 & 1989 & 65.16 & 0.03 & 6393.00 & 55881.10 & 1.02 & 0.86 & 0.87 \\
25063 & 103429 & 1989 & 247.64 & 0.27 & 24752.00 & 212039.94 & 1.00 & 0.86 & 0.86 \\
17426 & 102306 & 1989 & 308.17 & 0.35 & 32041.00 & 272533.96 & 0.96 & 0.88 & 0.85 \\
2943 & 100389 & 1989 & 32.97 & 0.18 & 3290.00 & 28653.73 & 1.00 & 0.87 & 0.87 \\
12871 & 101603 & 1989 & 398.86 & 0.29 & 39880.00 & 339877.35 & 1.00 & 0.85 & 0.85 \\
10451 & 101286 & 1989 & 412.46 & 0.17 & 41116.00 & 380716.30 & 1.00 & 0.92 & 0.93 \\
9339 & 101132 & 1989 & 16.39 & 0.43 & 1639.00 & 15118.75 & 1.00 & 0.92 & 0.92 \\
24519 & 103338 & 1989 & 6.76 & 0.22 & 740.00 & 6952.71 & 0.91 & 1.03 & 0.94 \\
26093 & 103538 & 1989 & 11.40 & 0.15 & 1140.00 & 9283.44 & 1.00 & 0.81 & 0.81 \\
108 & 100009 & 1989 & 19.90 & 0.34 & 2124.00 & 20727.01 & 0.94 & 1.04 & 0.98 \\
21897 & 102976 & 1989 & 215.19 & 0.19 & 19629.00 & 215020.57 & 1.10 & 1.00 & 1.10 \\
1915 & 100250 & 1989 & 28.04 & 0.41 & 2804.00 & 22752.58 & 1.00 & 0.81 & 0.81 \\
1497 & 100208 & 1989 & 44.93 & 0.36 & 4493.00 & 42977.13 & 1.00 & 0.96 & 0.96 \\
1507 & 100209 & 1989 & 197.34 & 0.29 & 19734.00 & 186108.61 & 1.00 & 0.94 & 0.94 \\
8169 & 101078 & 1989 & 1.50 & 0.09 & 157.00 & 1374.44 & 0.96 & 0.92 & 0.88 \\
21802 & 102952 & 1989 & 282.62 & 0.23 & 28262.00 & 262834.84 & 1.00 & 0.93 & 0.93 \\
12039 & 101491 & 1989 & 83.63 & 0.40 & 8363.00 & 67427.08 & 1.00 & 0.81 & 0.81 \\
15034 & 101953 & 1989 & 219.45 & 0.22 & 20445.00 & 204660.02 & 1.07 & 0.93 & 1.00 \\
8892 & 101104 & 1989 & 13.50 & 0.33 & 1749.00 & 15467.59 & 0.77 & 1.15 & 0.88 \\
9376 & 101133 & 1989 & 1.84 & 0.37 & 184.00 & 1676.40 & 1.00 & 0.91 & 0.91 \\
12904 & 101606 & 1989 & 694.28 & 0.37 & 69430.00 & 576999.10 & 1.00 & 0.83 & 0.83 \\
3130 & 100411 & 1989 & 165.80 & -0.00 & 16580.00 & 148701.70 & 1.00 & 0.90 & 0.90 \\
25188 & 103460 & 1989 & 149.06 & 0.17 & 15747.00 & 142998.46 & 0.95 & 0.96 & 0.91 \\
6626 & 100906 & 1989 & 437.66 & 0.30 & 42794.00 & 388661.16 & 1.02 & 0.89 & 0.91 \\
13650 & 101754 & 1989 & 26.74 & 0.31 & 2674.00 & 23436.33 & 1.00 & 0.88 & 0.88 \\
10247 & 101276 & 1989 & 76.04 & 0.25 & 7604.00 & 73837.07 & 1.00 & 0.97 & 0.97 \\
3013 & 100398 & 1989 & 40.12 & 0.19 & 4010.00 & 35876.15 & 1.00 & 0.89 & 0.89 \\
1876 & 100247 & 1989 & 232.00 & 0.43 & 23200.00 & 207396.68 & 1.00 & 0.89 & 0.89 \\
21516 & 102878 & 1989 & 13.08 & 0.15 & 1763.00 & 15176.62 & 0.74 & 1.16 & 0.86 \\
74636 & 601143 & 1989 & 99.10 & 0.38 & 10341.00 & 93533.54 & 0.96 & 0.94 & 0.90 \\
620 & 100085 & 1989 & 76.44 & 0.02 & 7644.00 & 71407.19 & 1.00 & 0.93 & 0.93 \\
366 & 100044 & 1989 & 3.94 & 0.12 & 330.00 & 3269.32 & 1.19 & 0.83 & 0.99 \\
21521 & 102880 & 1989 & 23.40 & 0.23 & 2259.00 & 20696.08 & 1.04 & 0.88 & 0.92 \\
7381 & 101038 & 1989 & 1753.30 & 0.26 & 143234.00 & 1614884.79 & 1.22 & 0.92 & 1.13 \\
5164 & 100730 & 1989 & 82.94 & 0.15 & 8222.00 & 80823.95 & 1.01 & 0.97 & 0.98 \\
968 & 100113 & 1989 & 837.43 & 0.23 & 79761.00 & 776608.90 & 1.05 & 0.93 & 0.97 \\
15518 & 102000 & 1989 & 404.69 & 0.29 & 35950.00 & 380099.34 & 1.13 & 0.94 & 1.06 \\
6659 & 100908 & 1989 & 8.40 & 0.34 & 845.00 & 6960.72 & 0.99 & 0.83 & 0.82 \\
20836 & 102796 & 1989 & 2.11 & 0.09 & 212.00 & 1988.58 & 1.00 & 0.94 & 0.94 \\
6737 & 100947 & 1989 & 792.37 & 0.15 & 79237.00 & 711688.70 & 1.00 & 0.90 & 0.90 \\
6320 & 100849 & 1989 & 53.50 & 0.24 & 5350.00 & 50480.94 & 1.00 & 0.94 & 0.94 \\
22206 & 102996 & 1989 & 125.00 & 0.32 & 12462.00 & 120668.55 & 1.00 & 0.97 & 0.97 \\
7303 & 101020 & 1989 & 2252.50 & 0.27 & 215549.00 & 1921634.00 & 1.05 & 0.85 & 0.89 \\
9062 & 101110 & 1989 & 7.10 & 0.04 & 980.00 & 7930.25 & 0.72 & 1.12 & 0.81 \\
12772 & 101595 & 1989 & 83.98 & 0.45 & 8949.00 & 72143.39 & 0.94 & 0.86 & 0.81 \\
3451 & 100439 & 1989 & 6.32 & 0.42 & 780.00 & 6824.57 & 0.81 & 1.08 & 0.87 \\
5280 & 100746 & 1989 & 135.31 & 0.23 & 14056.00 & 130231.22 & 0.96 & 0.96 & 0.93 \\
9437 & 101135 & 1989 & 36.17 & 0.59 & 3617.00 & 30730.70 & 1.00 & 0.85 & 0.85 \\
10630 & 101302 & 1989 & 40.55 & 0.20 & 4053.00 & 33645.92 & 1.00 & 0.83 & 0.83 \\
10186 & 101268 & 1989 & 109.49 & 0.42 & 10949.00 & 102515.67 & 1.00 & 0.94 & 0.94 \\
8730 & 101096 & 1989 & 1.60 & 0.18 & 153.00 & 1514.90 & 1.05 & 0.95 & 0.99 \\
24982 & 103406 & 1989 & 877.55 & 0.28 & 87777.00 & 713751.62 & 1.00 & 0.81 & 0.81 \\
9312 & 101131 & 1989 & 93.96 & 0.22 & 9396.00 & 77974.41 & 1.00 & 0.83 & 0.83 \\
2785 & 100358 & 1989 & 14.58 & 0.48 & 1458.00 & 14152.38 & 1.00 & 0.97 & 0.97 \\
17538 & 102318 & 1989 & 1385.96 & 0.25 & 138596.00 & 1226569.68 & 1.00 & 0.88 & 0.88 \\
20699 & 102784 & 1989 & 731.40 & 0.35 & 69725.00 & 659279.48 & 1.05 & 0.90 & 0.95 \\
10742 & 101322 & 1989 & 23.89 & -0.00 & 2381.00 & 22253.96 & 1.00 & 0.93 & 0.93 \\
1408 & 100196 & 1989 & 171.55 & 0.16 & 17154.00 & 160780.40 & 1.00 & 0.94 & 0.94 \\
5034 & 100704 & 1989 & 19.07 & 0.18 & 2275.00 & 21861.97 & 0.84 & 1.15 & 0.96 \\
20757 & 102789 & 1989 & 77.75 & 0.20 & 7650.00 & 66070.20 & 1.02 & 0.85 & 0.86 \\
12765 & 101594 & 1989 & 84.95 & 0.43 & 8468.00 & 69025.45 & 1.00 & 0.81 & 0.82 \\
22234 & 102997 & 1989 & 42.00 & 0.30 & 4220.00 & 39640.20 & 1.00 & 0.94 & 0.94 \\
17241 & 102274 & 1989 & 20.57 & 0.17 & 2009.00 & 19391.02 & 1.02 & 0.94 & 0.97 \\
12054 & 101494 & 1989 & 105.00 & 0.31 & 10501.00 & 104759.22 & 1.00 & 1.00 & 1.00 \\
47176 & 200342 & 1989 & 152.00 & 0.23 & 13887.00 & 142644.28 & 1.09 & 0.94 & 1.03 \\
17470 & 102307 & 1989 & 34.78 & 0.19 & 3494.00 & 29068.53 & 1.00 & 0.84 & 0.83 \\
21220 & 102838 & 1989 & 96.00 & 0.20 & 10198.00 & 97826.85 & 0.94 & 1.02 & 0.96 \\
1469 & 100207 & 1989 & 2483.80 & 0.29 & 248380.00 & 2338275.00 & 1.00 & 0.94 & 0.94 \\
8805 & 101099 & 1989 & 1.10 & 0.14 & 109.00 & 1019.33 & 1.01 & 0.93 & 0.94 \\
14899 & 101921 & 1989 & 21.14 & 0.05 & 2114.00 & 18956.16 & 1.00 & 0.90 & 0.90 \\
21048 & 102825 & 1989 & 112.00 & 0.18 & 11883.00 & 97658.76 & 0.94 & 0.87 & 0.82 \\
9402 & 101134 & 1989 & 54.36 & 0.21 & 5436.00 & 44914.25 & 1.00 & 0.83 & 0.83 \\
8957 & 101107 & 1989 & 883.60 & 0.32 & 82491.00 & 824236.56 & 1.07 & 0.93 & 1.00 \\
13793 & 101764 & 1989 & 27.79 & -0.00 & 2779.00 & 23593.61 & 1.00 & 0.85 & 0.85 \\
6710 & 100916 & 1989 & 35.86 & 0.15 & 3596.00 & 30750.91 & 1.00 & 0.86 & 0.86 \\
8750 & 101097 & 1989 & 114.70 & 0.51 & 11345.00 & 96380.37 & 1.01 & 0.84 & 0.85 \\
3387 & 100430 & 1989 & 65.36 & 0.30 & 6968.00 & 69671.49 & 0.94 & 1.07 & 1.00 \\
47241 & 200344 & 1989 & 952.38 & 0.31 & 95237.00 & 883383.45 & 1.00 & 0.93 & 0.93 \\
10556 & 101299 & 1989 & 144.15 & 0.24 & 14399.00 & 124749.58 & 1.00 & 0.87 & 0.87 \\
14864 & 101919 & 1989 & 135.80 & 0.14 & 14112.00 & 134658.96 & 0.96 & 0.99 & 0.95 \\
17219 & 102271 & 1989 & 555.94 & 0.22 & 50515.00 & 556175.52 & 1.10 & 1.00 & 1.10 \\
6718 & 100921 & 1989 & 53.34 & 0.06 & 4821.00 & 50811.84 & 1.11 & 0.95 & 1.05 \\
3961 & 100535 & 1989 & 150.40 & 0.42 & 12741.00 & 144564.17 & 1.18 & 0.96 & 1.13 \\
1398 & 100195 & 1989 & 12.69 & 0.26 & 1268.00 & 11777.85 & 1.00 & 0.93 & 0.93 \\
25543 & 103496 & 1989 & 254.54 & 0.32 & 25454.00 & 240964.58 & 1.00 & 0.95 & 0.95 \\
49052 & 240212 & 1989 & 293.80 & 0.42 & 27071.00 & 281562.49 & 1.09 & 0.96 & 1.04 \\
11273 & 101390 & 1989 & 1062.30 & 0.02 & 98679.00 & 989343.28 & 1.08 & 0.93 & 1.00 \\
14121 & 101805 & 1989 & 516.43 & 0.24 & 52103.00 & 500926.37 & 0.99 & 0.97 & 0.96 \\
16836 & 102197 & 1989 & 34.71 & 0.28 & 3594.00 & 29915.08 & 0.97 & 0.86 & 0.83 \\
13342 & 101729 & 1989 & 197.82 & 0.27 & 19779.00 & 185054.27 & 1.00 & 0.94 & 0.94 \\
19211 & 102570 & 1989 & 131.70 & 0.22 & 13207.00 & 116386.30 & 1.00 & 0.88 & 0.88 \\
13967 & 101794 & 1989 & 39.73 & 0.33 & 3599.00 & 35894.13 & 1.10 & 0.90 & 1.00 \\
5808 & 100799 & 1989 & 17.82 & 0.20 & 2054.00 & 17134.41 & 0.87 & 0.96 & 0.83 \\
9518 & 101142 & 1989 & 8.57 & 0.38 & 857.00 & 7790.19 & 1.00 & 0.91 & 0.91 \\
2291 & 100313 & 1989 & 66.37 & 0.30 & 6140.00 & 64600.60 & 1.08 & 0.97 & 1.05 \\
8485 & 101088 & 1989 & 5.00 & 0.16 & 484.00 & 4447.35 & 1.03 & 0.89 & 0.92 \\
11495 & 101425 & 1989 & 12.24 & 0.17 & 1268.00 & 11600.68 & 0.97 & 0.95 & 0.91 \\
4467 & 100634 & 1989 & 266.93 & 0.29 & 26692.00 & 231678.08 & 1.00 & 0.87 & 0.87 \\
24751 & 103377 & 1989 & 144.16 & 0.35 & 14400.00 & 137490.59 & 1.00 & 0.95 & 0.95 \\
6967 & 100977 & 1989 & 5.52 & -0.00 & 523.00 & 4783.45 & 1.06 & 0.87 & 0.91 \\
25677 & 103508 & 1989 & 57.20 & 0.22 & 5717.00 & 47105.13 & 1.00 & 0.82 & 0.82 \\
8307 & 101082 & 1989 & 222.50 & 0.42 & 22582.00 & 194690.45 & 0.99 & 0.88 & 0.86 \\
18529 & 102471 & 1989 & 98.17 & 0.30 & 8585.00 & 84306.66 & 1.14 & 0.86 & 0.98 \\
22933 & 103089 & 1989 & 81.03 & 0.31 & 7888.00 & 73814.77 & 1.03 & 0.91 & 0.94 \\
16224 & 102102 & 1989 & 42.38 & 0.32 & 4238.00 & 40738.61 & 1.00 & 0.96 & 0.96 \\
7935 & 101067 & 1989 & 29.30 & 0.17 & 2824.00 & 28955.02 & 1.04 & 0.99 & 1.03 \\
13361 & 101730 & 1989 & 21.26 & 0.18 & 2126.00 & 18669.60 & 1.00 & 0.88 & 0.88 \\
8386 & 101085 & 1989 & 152.30 & 0.39 & 11364.00 & 101598.65 & 1.34 & 0.67 & 0.89 \\
8521 & 101089 & 1989 & 5.30 & 0.38 & 593.00 & 4272.55 & 0.89 & 0.81 & 0.72 \\
4517 & 100637 & 1989 & 95.94 & 0.32 & 9594.00 & 91483.57 & 1.00 & 0.95 & 0.95 \\
8348 & 101084 & 1989 & 71.00 & 0.17 & 8200.00 & 67471.73 & 0.87 & 0.95 & 0.82 \\
5457 & 100764 & 1989 & 11.28 & 0.32 & 1127.00 & 9467.00 & 1.00 & 0.84 & 0.84 \\
18310 & 102425 & 1989 & 550.13 & 0.21 & 55021.00 & 440141.76 & 1.00 & 0.80 & 0.80 \\
24674 & 103375 & 1989 & 24.64 & 0.50 & 2500.00 & 23982.69 & 0.99 & 0.97 & 0.96 \\
57832 & 401082 & 1989 & 6.00 & 0.07 & 627.00 & 5401.81 & 0.96 & 0.90 & 0.86 \\
2407 & 100323 & 1989 & 22.61 & 0.27 & 2260.00 & 19572.24 & 1.00 & 0.87 & 0.87 \\
38977 & 107387 & 1989 & 5.92 & 0.28 & 624.00 & 6439.71 & 0.95 & 1.09 & 1.03 \\
2323 & 100319 & 1989 & 86.90 & 0.22 & 8687.00 & 79844.82 & 1.00 & 0.92 & 0.92 \\
11355 & 101398 & 1989 & 94.57 & 0.24 & 9457.00 & 82093.51 & 1.00 & 0.87 & 0.87 \\
49095 & 240222 & 1989 & 267.56 & 0.31 & 26756.00 & 223026.90 & 1.00 & 0.83 & 0.83 \\
7035 & 100992 & 1989 & 128.62 & 0.29 & 12615.00 & 112997.73 & 1.02 & 0.88 & 0.90 \\
4389 & 100622 & 1989 & 43.90 & 0.12 & 4392.00 & 43157.78 & 1.00 & 0.98 & 0.98 \\
1212 & 100166 & 1989 & 2038.38 & 0.27 & 203838.00 & 1650593.58 & 1.00 & 0.81 & 0.81 \\
16866 & 102213 & 1989 & 48.90 & 0.34 & 4890.00 & 40953.18 & 1.00 & 0.84 & 0.84 \\
8451 & 101087 & 1989 & 2.90 & 0.14 & 335.00 & 2543.16 & 0.87 & 0.88 & 0.76 \\
18790 & 102522 & 1989 & 212.66 & 0.02 & 21275.00 & 178718.53 & 1.00 & 0.84 & 0.84 \\
23239 & 103152 & 1989 & 396.56 & 0.32 & 34799.00 & 370399.09 & 1.14 & 0.93 & 1.06 \\
24634 & 103373 & 1989 & 185.09 & 0.25 & 20005.00 & 166612.70 & 0.93 & 0.90 & 0.83 \\
1201 & 100162 & 1989 & 8.49 & 0.01 & 859.00 & 7216.03 & 0.99 & 0.85 & 0.84 \\
4371 & 100614 & 1989 & 124.80 & 0.29 & 12477.00 & 120910.92 & 1.00 & 0.97 & 0.97 \\
13305 & 101720 & 1989 & 9.90 & 0.12 & 1013.00 & 9279.13 & 0.98 & 0.94 & 0.92 \\
1199 & 100160 & 1989 & 26.58 & 0.20 & 2658.00 & 24998.85 & 1.00 & 0.94 & 0.94 \\
4751 & 100671 & 1989 & 37.32 & 0.35 & 3734.00 & 34756.31 & 1.00 & 0.93 & 0.93 \\
14154 & 101819 & 1989 & 114.66 & 0.21 & 11450.00 & 98956.33 & 1.00 & 0.86 & 0.86 \\
18169 & 102414 & 1989 & 66.05 & 0.18 & 6596.00 & 52840.03 & 1.00 & 0.80 & 0.80 \\
2303 & 100315 & 1989 & 273.40 & 0.22 & 27339.00 & 260316.71 & 1.00 & 0.95 & 0.95 \\
4426 & 100624 & 1989 & 34.40 & 0.29 & 3443.00 & 33980.96 & 1.00 & 0.99 & 0.99 \\
7967 & 101068 & 1989 & 12699.60 & 0.35 & 1243711.00 & 10416222.32 & 1.02 & 0.82 & 0.84 \\
57764 & 401018 & 1989 & 109.60 & 0.14 & 10930.00 & 100380.73 & 1.00 & 0.92 & 0.92 \\
16386 & 102132 & 1989 & 11.59 & 0.02 & 1198.00 & 11627.44 & 0.97 & 1.00 & 0.97 \\
96652 & 611002 & 1989 & 774.39 & 0.28 & 77439.00 & 691067.32 & 1.00 & 0.89 & 0.89 \\
7073 & 100994 & 1989 & 19.89 & 0.14 & 1781.00 & 17374.35 & 1.12 & 0.87 & 0.98 \\
11340 & 101396 & 1989 & 33.69 & 0.25 & 3370.00 & 30109.65 & 1.00 & 0.89 & 0.89 \\
24790 & 103380 & 1989 & 2858.68 & 0.40 & 285900.00 & 2825318.57 & 1.00 & 0.99 & 0.99 \\
5593 & 100773 & 1989 & 923.00 & 0.10 & 92280.00 & 872119.94 & 1.00 & 0.94 & 0.95 \\
14104 & 101804 & 1989 & 252.97 & 0.25 & 28008.00 & 254928.80 & 0.90 & 1.01 & 0.91 \\
25618 & 103498 & 1989 & 116.78 & 0.22 & 11678.00 & 106635.05 & 1.00 & 0.91 & 0.91 \\
16097 & 102080 & 1989 & 67.94 & -0.00 & 7767.00 & 64928.06 & 0.87 & 0.96 & 0.84 \\
1803 & 100238 & 1989 & 36.98 & 0.03 & 3700.00 & 38197.81 & 1.00 & 1.03 & 1.03 \\
26903 & 103621 & 1989 & 53.93 & 0.27 & 5314.00 & 54004.69 & 1.01 & 1.00 & 1.02 \\
18444 & 102461 & 1989 & 18.88 & 0.29 & 1888.00 & 17716.62 & 1.00 & 0.94 & 0.94 \\
4615 & 100644 & 1989 & 37.04 & 0.42 & 3704.00 & 34485.24 & 1.00 & 0.93 & 0.93 \\
9207 & 101119 & 1989 & 33.58 & -0.01 & 3358.00 & 27330.27 & 1.00 & 0.81 & 0.81 \\
18032 & 102387 & 1989 & 8.89 & 0.20 & 786.00 & 8511.22 & 1.13 & 0.96 & 1.08 \\
12204 & 101519 & 1989 & 24.20 & 0.39 & 2417.00 & 20155.23 & 1.00 & 0.83 & 0.83 \\
1736 & 100227 & 1989 & 108.06 & 0.24 & 10806.00 & 97716.62 & 1.00 & 0.90 & 0.90 \\
8556 & 101090 & 1989 & 9.40 & 0.07 & 919.00 & 7789.45 & 1.02 & 0.83 & 0.85 \\
1127 & 100155 & 1989 & 54.80 & 0.23 & 5428.00 & 53286.57 & 1.01 & 0.97 & 0.98 \\
7898 & 101065 & 1989 & 69.70 & 0.44 & 8632.00 & 64353.11 & 0.81 & 0.92 & 0.75 \\
4004 & 100538 & 1989 & 150.64 & 0.19 & 16099.00 & 133790.30 & 0.94 & 0.89 & 0.83 \\
23488 & 103182 & 1989 & 33.25 & 0.31 & 3316.00 & 27919.91 & 1.00 & 0.84 & 0.84 \\
9073 & 101111 & 1989 & 7.80 & 0.27 & 1312.00 & 10640.73 & 0.59 & 1.36 & 0.81 \\
24714 & 103376 & 1989 & 1642.03 & 0.43 & 164200.00 & 1565353.10 & 1.00 & 0.95 & 0.95 \\
24830 & 103381 & 1989 & 345.46 & 0.37 & 34500.00 & 336247.65 & 1.00 & 0.97 & 0.97 \\
11703 & 101457 & 1989 & 67.40 & 0.30 & 6742.00 & 59708.44 & 1.00 & 0.89 & 0.89 \\
9572 & 101151 & 1989 & 35.55 & 0.31 & 3555.00 & 31635.12 & 1.00 & 0.89 & 0.89 \\
22887 & 103077 & 1989 & 20.50 & 0.17 & 2317.00 & 20020.19 & 0.88 & 0.98 & 0.86 \\
25486 & 103494 & 1989 & 194.78 & 0.31 & 19478.00 & 178607.35 & 1.00 & 0.92 & 0.92 \\
14011 & 101800 & 1989 & 336.56 & 0.24 & 31903.00 & 305336.65 & 1.05 & 0.91 & 0.96 \\
4797 & 100682 & 1989 & 28.20 & 0.18 & 2800.00 & 28605.21 & 1.01 & 1.01 & 1.02 \\
13210 & 101704 & 1989 & 298.11 & 0.31 & 29811.00 & 261396.38 & 1.00 & 0.88 & 0.88 \\
26511 & 103585 & 1989 & 18.60 & 0.07 & 1857.00 & 16451.45 & 1.00 & 0.88 & 0.89 \\
23467 & 103179 & 1989 & 246.49 & 0.37 & 24577.00 & 209910.52 & 1.00 & 0.85 & 0.85 \\
57823 & 401081 & 1989 & 4.93 & 0.31 & 365.00 & 3634.09 & 1.35 & 0.74 & 1.00 \\
11185 & 101370 & 1989 & 9.12 & 0.36 & 912.00 & 8713.74 & 1.00 & 0.96 & 0.96 \\
14437 & 101858 & 1989 & 71.64 & 0.40 & 7300.00 & 61358.49 & 0.98 & 0.86 & 0.84 \\
57817 & 401072 & 1989 & 45.60 & 0.21 & 4487.00 & 41133.61 & 1.02 & 0.90 & 0.92 \\
26473 & 103582 & 1989 & 13.50 & 0.02 & 1303.00 & 11031.39 & 1.04 & 0.82 & 0.85 \\
23509 & 103183 & 1989 & 204.65 & 0.35 & 20484.00 & 172745.28 & 1.00 & 0.84 & 0.84 \\
53394 & 346113 & 1989 & 76.20 & 0.34 & 7168.00 & 65759.94 & 1.06 & 0.86 & 0.92 \\
16957 & 102224 & 1989 & 33.08 & 0.11 & 3308.00 & 32472.87 & 1.00 & 0.98 & 0.98 \\
6399 & 100864 & 1990 & 487.45 & 0.01 & 48745.00 & 479338.22 & 1.00 & 0.98 & 0.98 \\
57957 & 410010 & 1990 & 80.65 & 0.02 & 8065.00 & 71105.78 & 1.00 & 0.88 & 0.88 \\
4632 & 100659 & 1990 & 52.42 & 0.06 & 4265.00 & 49775.03 & 1.23 & 0.95 & 1.17 \\
25851 & 103525 & 1990 & 511.57 & 0.03 & 51157.00 & 491861.91 & 1.00 & 0.96 & 0.96 \\
17220 & 102271 & 1990 & 614.02 & 0.05 & 63985.00 & 592076.53 & 0.96 & 0.96 & 0.93 \\
17471 & 102307 & 1990 & 31.63 & -0.05 & 3156.00 & 28389.68 & 1.00 & 0.90 & 0.90 \\
25619 & 103498 & 1990 & 103.85 & -0.04 & 10385.00 & 91245.00 & 1.00 & 0.88 & 0.88 \\
21153 & 102835 & 1990 & 24.18 & -0.05 & 2419.00 & 21662.48 & 1.00 & 0.90 & 0.90 \\
18747 & 102507 & 1990 & 80.64 & 0.06 & 8064.00 & 80568.54 & 1.00 & 1.00 & 1.00 \\
25685 & 103514 & 1990 & 398.90 & 0.02 & 39890.00 & 349648.30 & 1.00 & 0.88 & 0.88 \\
1508 & 100209 & 1990 & 212.55 & 0.02 & 21255.00 & 195151.98 & 1.00 & 0.92 & 0.92 \\
16867 & 102213 & 1990 & 52.32 & -0.08 & 5232.00 & 42811.44 & 1.00 & 0.82 & 0.82 \\
5071 & 100715 & 1990 & 56.79 & -0.06 & 5679.00 & 51298.73 & 1.00 & 0.90 & 0.90 \\
4427 & 100624 & 1990 & 34.80 & -0.04 & 3475.00 & 29765.87 & 1.00 & 0.86 & 0.86 \\
21741 & 102949 & 1990 & 806.23 & -0.17 & 71196.00 & 695551.54 & 1.13 & 0.86 & 0.98 \\
5068 & 100714 & 1990 & 22.50 & -0.17 & 2250.00 & 21346.13 & 1.00 & 0.95 & 0.95 \\
15035 & 101953 & 1990 & 216.00 & -0.04 & 22406.00 & 182575.63 & 0.96 & 0.85 & 0.81 \\
3388 & 100430 & 1990 & 46.69 & -0.26 & 4677.00 & 45946.10 & 1.00 & 0.98 & 0.98 \\
4390 & 100622 & 1990 & 27.80 & -0.24 & 2778.00 & 27429.04 & 1.00 & 0.99 & 0.99 \\
26474 & 103582 & 1990 & 8.82 & -0.20 & 1110.00 & 7451.18 & 0.80 & 0.84 & 0.67 \\
26608 & 103593 & 1990 & 5606.85 & 0.04 & 560685.00 & 5289056.47 & 1.00 & 0.94 & 0.94 \\
21049 & 102825 & 1990 & 96.54 & -0.06 & 10846.00 & 83782.51 & 0.89 & 0.87 & 0.77 \\
10557 & 101299 & 1990 & 195.93 & 0.02 & 19572.00 & 164543.59 & 1.00 & 0.84 & 0.84 \\
7693 & 101055 & 1990 & 601.70 & 0.04 & 53463.00 & 466322.40 & 1.13 & 0.78 & 0.87 \\
6369 & 100856 & 1990 & 78.54 & -0.05 & 7854.00 & 66056.20 & 1.00 & 0.84 & 0.84 \\
8199 & 101079 & 1990 & 96.40 & -0.04 & 11513.00 & 93099.52 & 0.84 & 0.97 & 0.81 \\
4372 & 100614 & 1990 & 112.70 & -0.11 & 11270.00 & 97052.89 & 1.00 & 0.86 & 0.86 \\
18791 & 102522 & 1990 & 189.71 & -0.05 & 18971.00 & 166498.49 & 1.00 & 0.88 & 0.88 \\
5187 & 100731 & 1990 & 2785.90 & 0.03 & 276546.00 & 2301447.35 & 1.01 & 0.83 & 0.83 \\
4616 & 100644 & 1990 & 36.55 & 0.01 & 3655.00 & 31558.28 & 1.00 & 0.86 & 0.86 \\
47242 & 200344 & 1990 & 812.90 & -0.03 & 81290.00 & 748273.59 & 1.00 & 0.92 & 0.92 \\
47326 & 210203 & 1990 & 62.13 & 0.03 & 6210.00 & 60534.05 & 1.00 & 0.97 & 0.97 \\
15165 & 101964 & 1990 & 22.50 & 0.04 & 2250.00 & 21944.99 & 1.00 & 0.98 & 0.98 \\
4518 & 100637 & 1990 & 121.59 & -0.12 & 12200.00 & 119136.04 & 1.00 & 0.98 & 0.98 \\
25189 & 103460 & 1990 & 135.27 & -0.12 & 11467.00 & 127563.23 & 1.18 & 0.94 & 1.11 \\
5630 & 100780 & 1990 & 22.59 & 0.03 & 2259.00 & 18910.25 & 1.00 & 0.84 & 0.84 \\
5238 & 100741 & 1990 & 147.00 & -0.09 & 14673.00 & 144009.37 & 1.00 & 0.98 & 0.98 \\
280 & 100033 & 1990 & 3.03 & 0.02 & 437.00 & 2786.25 & 0.69 & 0.92 & 0.64 \\
6495 & 100878 & 1990 & 621.70 & 0.04 & 69400.00 & 569079.96 & 0.90 & 0.92 & 0.82 \\
26164 & 103545 & 1990 & 2334.46 & 0.03 & 233446.00 & 2201690.19 & 1.00 & 0.94 & 0.94 \\
47370 & 210681 & 1990 & 332.10 & 0.04 & 54817.00 & 270745.75 & 0.61 & 0.82 & 0.49 \\
26094 & 103538 & 1990 & 12.43 & -0.07 & 1243.00 & 11148.44 & 1.00 & 0.90 & 0.90 \\
969 & 100113 & 1990 & 620.44 & -0.04 & 37807.00 & 324842.95 & 1.64 & 0.52 & 0.86 \\
647 & 100087 & 1990 & 803.09 & 0.06 & 80309.00 & 769376.72 & 1.00 & 0.96 & 0.96 \\
4552 & 100639 & 1990 & 63.65 & 0.02 & 6365.00 & 53090.73 & 1.00 & 0.83 & 0.83 \\
4666 & 100660 & 1990 & 60.59 & 0.03 & 5577.00 & 53879.65 & 1.09 & 0.89 & 0.97 \\
18311 & 102425 & 1990 & 484.20 & -0.03 & 48415.00 & 400331.92 & 1.00 & 0.83 & 0.83 \\
21483 & 102873 & 1990 & 148.41 & -0.14 & 19116.00 & 123467.32 & 0.78 & 0.83 & 0.65 \\
5594 & 100773 & 1990 & 566.00 & -0.22 & 56590.00 & 516508.42 & 1.00 & 0.91 & 0.91 \\
21512 & 102877 & 1990 & 40.30 & -0.10 & 2917.00 & 26653.66 & 1.38 & 0.66 & 0.91 \\
45784 & 200140 & 1990 & 1101.53 & 0.03 & 110153.00 & 951690.20 & 1.00 & 0.86 & 0.86 \\
17374 & 102284 & 1990 & 22.84 & -0.13 & 2204.00 & 22141.84 & 1.04 & 0.97 & 1.00 \\
18530 & 102471 & 1990 & 90.78 & -0.12 & 9663.00 & 88965.31 & 0.94 & 0.98 & 0.92 \\
5228 & 100740 & 1990 & 148.70 & 0.05 & 14976.00 & 146446.34 & 0.99 & 0.98 & 0.98 \\
26204 & 103546 & 1990 & 6523.84 & 0.05 & 652384.00 & 5399159.20 & 1.00 & 0.83 & 0.83 \\
5124 & 100726 & 1990 & 89.19 & -0.01 & 8889.00 & 76644.07 & 1.00 & 0.86 & 0.86 \\
15136 & 101963 & 1990 & 32.04 & -0.06 & 3083.00 & 25094.62 & 1.04 & 0.78 & 0.81 \\
1013 & 100127 & 1990 & 131.84 & -0.03 & 13273.00 & 107058.07 & 0.99 & 0.81 & 0.81 \\
1470 & 100207 & 1990 & 2154.45 & -0.06 & 215445.00 & 1976917.93 & 1.00 & 0.92 & 0.92 \\
367 & 100044 & 1990 & 7.58 & -0.01 & 758.00 & 7436.80 & 1.00 & 0.98 & 0.98 \\
25817 & 103524 & 1990 & 2173.03 & 0.03 & 217303.00 & 2136538.00 & 1.00 & 0.98 & 0.98 \\
21240 & 102842 & 1990 & 5.50 & 0.03 & 550.00 & 5179.90 & 1.00 & 0.94 & 0.94 \\
21231 & 102840 & 1990 & 39.60 & 0.02 & 3960.00 & 37909.74 & 1.00 & 0.96 & 0.96 \\
21586 & 102895 & 1990 & 55.59 & 0.04 & 5536.00 & 55318.19 & 1.00 & 1.00 & 1.00 \\
5165 & 100730 & 1990 & 77.60 & -0.04 & 7787.00 & 72553.31 & 1.00 & 0.93 & 0.93 \\
18674 & 102502 & 1990 & 87.58 & 0.06 & 9452.00 & 75195.41 & 0.93 & 0.86 & 0.80 \\
21221 & 102838 & 1990 & 139.57 & 0.01 & 13957.00 & 131053.38 & 1.00 & 0.94 & 0.94 \\
18685 & 102503 & 1990 & 20.74 & -0.21 & 2074.00 & 16959.87 & 1.00 & 0.82 & 0.82 \\
4586 & 100642 & 1990 & 400.58 & 0.02 & 40058.00 & 354458.19 & 1.00 & 0.88 & 0.88 \\
1128 & 100155 & 1990 & 47.10 & -0.02 & 5568.00 & 49772.02 & 0.85 & 1.06 & 0.89 \\
18445 & 102461 & 1990 & 12.08 & -0.10 & 1208.00 & 9886.14 & 1.00 & 0.82 & 0.82 \\
21517 & 102878 & 1990 & 7.10 & -0.18 & 457.00 & 4599.72 & 1.55 & 0.65 & 1.01 \\
5638 & 100784 & 1990 & 172.97 & 0.04 & 14603.00 & 149238.02 & 1.18 & 0.86 & 1.02 \\
10452 & 101286 & 1990 & 352.08 & -0.09 & 35247.00 & 351280.45 & 1.00 & 1.00 & 1.00 \\
4468 & 100634 & 1990 & 270.67 & -0.03 & 27070.00 & 230482.13 & 1.00 & 0.85 & 0.85 \\
17427 & 102306 & 1990 & 341.87 & -0.03 & 34187.00 & 315431.31 & 1.00 & 0.92 & 0.92 \\
17242 & 102274 & 1990 & 17.47 & -0.16 & 1740.00 & 16336.36 & 1.00 & 0.94 & 0.94 \\
25783 & 103523 & 1990 & 178.84 & 0.04 & 17884.00 & 172777.89 & 1.00 & 0.97 & 0.97 \\
8810 & 101100 & 1990 & 82.10 & 0.02 & 9480.00 & 85612.73 & 0.87 & 1.04 & 0.90 \\
17984 & 102383 & 1990 & 5.96 & 0.02 & 596.00 & 5679.41 & 1.00 & 0.95 & 0.95 \\
12055 & 101494 & 1990 & 102.43 & -0.09 & 10243.00 & 92523.22 & 1.00 & 0.90 & 0.90 \\
12115 & 101511 & 1990 & 142.78 & -0.07 & 14270.00 & 141320.89 & 1.00 & 0.99 & 0.99 \\
4929 & 100695 & 1990 & 72.88 & -0.04 & 7250.00 & 64483.11 & 1.01 & 0.88 & 0.89 \\
19482 & 102607 & 1990 & 581.00 & 0.04 & 58596.00 & 576454.06 & 0.99 & 0.99 & 0.98 \\
25913 & 103529 & 1990 & 211.26 & 0.03 & 21126.00 & 204498.57 & 1.00 & 0.97 & 0.97 \\
25400 & 103483 & 1990 & 24.98 & -0.12 & 2469.00 & 22336.06 & 1.01 & 0.89 & 0.90 \\
25951 & 103531 & 1990 & 119.36 & -0.01 & 11936.00 & 114446.17 & 1.00 & 0.96 & 0.96 \\
16160 & 102089 & 1990 & 327.53 & -0.00 & 32026.00 & 258299.17 & 1.02 & 0.79 & 0.81 \\
15835 & 102043 & 1990 & 40.15 & -0.02 & 4015.00 & 35657.78 & 1.00 & 0.89 & 0.89 \\
12205 & 101519 & 1990 & 25.18 & 0.03 & 2185.00 & 20745.77 & 1.15 & 0.82 & 0.95 \\
11186 & 101370 & 1990 & 11.32 & -0.04 & 1132.00 & 10894.93 & 1.00 & 0.96 & 0.96 \\
17065 & 102241 & 1990 & 194.47 & 0.03 & 18154.00 & 157301.18 & 1.07 & 0.81 & 0.87 \\
7861 & 101064 & 1990 & 206.60 & -0.17 & 18439.00 & 166225.61 & 1.12 & 0.80 & 0.90 \\
16132 & 102085 & 1990 & 518.07 & 0.06 & 45799.00 & 480699.61 & 1.13 & 0.93 & 1.05 \\
57824 & 401081 & 1990 & 6.97 & -0.15 & 697.00 & 6571.11 & 1.00 & 0.94 & 0.94 \\
20100 & 102667 & 1990 & 304.61 & -0.03 & 30460.00 & 263586.62 & 1.00 & 0.87 & 0.87 \\
4798 & 100682 & 1990 & 27.52 & -0.02 & 2752.00 & 27805.70 & 1.00 & 1.01 & 1.01 \\
15904 & 102059 & 1990 & 83.85 & 0.07 & 8600.00 & 77178.88 & 0.98 & 0.92 & 0.90 \\
19459 & 102606 & 1990 & 3759.00 & 0.02 & 375058.00 & 3141414.79 & 1.00 & 0.84 & 0.84 \\
20227 & 102689 & 1990 & 7.67 & -0.02 & 767.00 & 7120.89 & 1.00 & 0.93 & 0.93 \\
733 & 100093 & 1990 & 29.50 & 0.09 & 2363.00 & 22147.94 & 1.25 & 0.75 & 0.94 \\
5315 & 100753 & 1990 & 1575.44 & 0.02 & 157540.00 & 1402419.74 & 1.00 & 0.89 & 0.89 \\
20273 & 102703 & 1990 & 177.38 & 0.05 & 22735.00 & 177833.71 & 0.78 & 1.00 & 0.78 \\
16247 & 102105 & 1990 & 32.11 & 0.03 & 3211.00 & 29772.84 & 1.00 & 0.93 & 0.93 \\
20270 & 102702 & 1990 & 115.80 & -0.09 & 15270.00 & 118131.57 & 0.76 & 1.02 & 0.77 \\
19357 & 102599 & 1990 & 3.00 & 0.05 & 338.00 & 3191.80 & 0.89 & 1.06 & 0.94 \\
20005 & 102663 & 1990 & 61.53 & -0.01 & 6153.00 & 52235.66 & 1.00 & 0.85 & 0.85 \\
15714 & 102016 & 1990 & 780.81 & -0.05 & 74780.00 & 747374.99 & 1.04 & 0.96 & 1.00 \\
11209 & 101376 & 1990 & 38.00 & 0.03 & 3802.00 & 35994.98 & 1.00 & 0.95 & 0.95 \\
19425 & 102601 & 1990 & 1390.00 & 0.03 & 139552.00 & 1310305.74 & 1.00 & 0.94 & 0.94 \\
10887 & 101345 & 1990 & 65.61 & -0.07 & 6760.00 & 56694.16 & 0.97 & 0.86 & 0.84 \\
15748 & 102017 & 1990 & 1265.71 & 0.02 & 140776.00 & 1157713.39 & 0.90 & 0.91 & 0.82 \\
8308 & 101082 & 1990 & 373.20 & 0.24 & 32474.00 & 328200.83 & 1.15 & 0.88 & 1.01 \\
5336 & 100754 & 1990 & 361.37 & 0.02 & 36140.00 & 322938.50 & 1.00 & 0.89 & 0.89 \\
16225 & 102102 & 1990 & 49.27 & -0.14 & 4927.00 & 46934.48 & 1.00 & 0.95 & 0.95 \\
4166 & 100567 & 1990 & 743.46 & 0.06 & 73794.00 & 622117.14 & 1.01 & 0.84 & 0.84 \\
4039 & 100543 & 1990 & 175.72 & 0.01 & 15760.00 & 179209.68 & 1.11 & 1.02 & 1.14 \\
4879 & 100691 & 1990 & 83.03 & -0.05 & 8303.00 & 66470.73 & 1.00 & 0.80 & 0.80 \\
19912 & 102655 & 1990 & 707.55 & -0.09 & 70755.00 & 576592.83 & 1.00 & 0.81 & 0.81 \\
26904 & 103621 & 1990 & 45.96 & -0.07 & 4294.00 & 40257.69 & 1.07 & 0.88 & 0.94 \\
18033 & 102387 & 1990 & 5.86 & -0.09 & 589.00 & 5781.42 & 0.99 & 0.99 & 0.98 \\
19868 & 102654 & 1990 & 95.70 & -0.12 & 9570.00 & 87760.00 & 1.00 & 0.92 & 0.92 \\
12149 & 101513 & 1990 & 26.36 & 0.03 & 2630.00 & 24309.62 & 1.00 & 0.92 & 0.92 \\
6002 & 100818 & 1990 & 422.55 & -0.01 & 42260.00 & 359949.27 & 1.00 & 0.85 & 0.85 \\
17998 & 102386 & 1990 & 51.46 & 0.04 & 5149.00 & 49904.31 & 1.00 & 0.97 & 0.97 \\
3943 & 100517 & 1990 & 64.81 & -0.08 & 6497.00 & 58771.77 & 1.00 & 0.91 & 0.90 \\
925 & 100112 & 1990 & 330.50 & 0.03 & 33025.00 & 287799.26 & 1.00 & 0.87 & 0.87 \\
53395 & 346113 & 1990 & 71.34 & -0.08 & 7280.00 & 63427.90 & 0.98 & 0.89 & 0.87 \\
919 & 100111 & 1990 & 53.34 & 0.04 & 5337.00 & 52237.55 & 1.00 & 0.98 & 0.98 \\
17958 & 102377 & 1990 & 11.60 & 0.08 & 1528.00 & 10114.00 & 0.76 & 0.87 & 0.66 \\
16958 & 102224 & 1990 & 46.06 & -0.18 & 4606.00 & 44926.99 & 1.00 & 0.98 & 0.98 \\
11738 & 101460 & 1990 & 519.88 & 0.06 & 51990.00 & 448525.81 & 1.00 & 0.86 & 0.86 \\
3908 & 100514 & 1990 & 82.28 & -0.03 & 8227.00 & 68389.05 & 1.00 & 0.83 & 0.83 \\
7899 & 101065 & 1990 & 223.90 & 0.44 & 16663.00 & 151530.29 & 1.34 & 0.68 & 0.91 \\
25487 & 103494 & 1990 & 202.06 & -0.01 & 20206.00 & 169688.75 & 1.00 & 0.84 & 0.84 \\
16098 & 102080 & 1990 & 36.99 & -0.28 & 3699.00 & 32344.40 & 1.00 & 0.87 & 0.87 \\
18071 & 102391 & 1990 & 21.95 & 0.01 & 2465.00 & 21987.86 & 0.89 & 1.00 & 0.89 \\
26279 & 103551 & 1990 & 7.26 & -0.03 & 725.00 & 6979.98 & 1.00 & 0.96 & 0.96 \\
19940 & 102659 & 1990 & 855.68 & -0.10 & 85549.00 & 735943.62 & 1.00 & 0.86 & 0.86 \\
1064 & 100150 & 1990 & 5.05 & -0.00 & 424.00 & 3760.66 & 1.19 & 0.74 & 0.89 \\
17051 & 102234 & 1990 & 233.95 & 0.02 & 21590.00 & 181375.20 & 1.08 & 0.78 & 0.84 \\
1339 & 100190 & 1990 & 542.55 & -0.03 & 54255.00 & 476230.36 & 1.00 & 0.88 & 0.88 \\
4899 & 100692 & 1990 & 23.86 & -0.10 & 2386.00 & 19614.04 & 1.00 & 0.82 & 0.82 \\
8620 & 101092 & 1990 & 5.00 & -0.15 & 340.00 & 2825.75 & 1.47 & 0.57 & 0.83 \\
3882 & 100509 & 1990 & 8.45 & -0.18 & 845.00 & 8272.26 & 1.00 & 0.98 & 0.98 \\
17030 & 102231 & 1990 & 478.06 & -0.05 & 47806.00 & 445895.33 & 1.00 & 0.93 & 0.93 \\
11704 & 101457 & 1990 & 62.90 & -0.02 & 6290.00 & 54008.98 & 1.00 & 0.86 & 0.86 \\
65125 & 500664 & 1990 & 92.01 & 0.02 & 8085.00 & 83298.91 & 1.14 & 0.91 & 1.03 \\
4005 & 100538 & 1990 & 87.30 & -0.12 & 10366.00 & 95998.10 & 0.84 & 1.10 & 0.93 \\
3895 & 100510 & 1990 & 13.27 & -0.09 & 1327.00 & 12418.80 & 1.00 & 0.94 & 0.94 \\
16016 & 102073 & 1990 & 2050.76 & -0.13 & 205075.00 & 1872716.83 & 1.00 & 0.91 & 0.91 \\
8267 & 101081 & 1990 & 22.80 & -0.24 & 1278.00 & 12506.09 & 1.78 & 0.55 & 0.98 \\
18217 & 102417 & 1990 & 54.20 & -0.15 & 5416.00 & 46450.12 & 1.00 & 0.86 & 0.86 \\
5035 & 100704 & 1990 & 14.07 & -0.07 & 2275.00 & 21408.62 & 0.62 & 1.52 & 0.94 \\
16454 & 102150 & 1990 & 6.05 & 0.09 & 605.00 & 5221.78 & 1.00 & 0.86 & 0.86 \\
4752 & 100671 & 1990 & 35.20 & -0.08 & 3520.00 & 33569.01 & 1.00 & 0.95 & 0.95 \\
20700 & 102784 & 1990 & 751.66 & -0.03 & 75555.00 & 700876.80 & 0.99 & 0.93 & 0.93 \\
7968 & 101068 & 1990 & 12325.70 & -0.05 & 1131468.00 & 10399211.34 & 1.09 & 0.84 & 0.92 \\
17539 & 102318 & 1990 & 1379.65 & -0.05 & 137965.00 & 1256592.03 & 1.00 & 0.91 & 0.91 \\
96653 & 611002 & 1990 & 954.53 & -0.03 & 95453.00 & 779719.61 & 1.00 & 0.82 & 0.82 \\
16387 & 102132 & 1990 & 12.89 & 0.00 & 1180.00 & 11791.91 & 1.09 & 0.92 & 1.00 \\
10743 & 101322 & 1990 & 18.36 & -0.09 & 1771.00 & 17621.31 & 1.04 & 0.96 & 0.99 \\
17573 & 102319 & 1990 & 355.42 & 0.06 & 35541.00 & 335655.42 & 1.00 & 0.94 & 0.94 \\
15519 & 102000 & 1990 & 368.49 & -0.08 & 35886.00 & 348680.58 & 1.03 & 0.95 & 0.97 \\
20645 & 102777 & 1990 & 183.88 & -0.10 & 19580.00 & 166687.11 & 0.94 & 0.91 & 0.85 \\
8452 & 101087 & 1990 & 2.00 & -0.16 & 173.00 & 1634.26 & 1.16 & 0.82 & 0.94 \\
3962 & 100535 & 1990 & 130.10 & -0.08 & 13010.00 & 120588.11 & 1.00 & 0.93 & 0.93 \\
38978 & 107387 & 1990 & 5.73 & -0.03 & 504.00 & 4769.67 & 1.14 & 0.83 & 0.95 \\
10631 & 101302 & 1990 & 39.91 & 0.00 & 3987.00 & 32945.01 & 1.00 & 0.83 & 0.83 \\
1200 & 100160 & 1990 & 24.16 & -0.12 & 2285.00 & 20469.23 & 1.06 & 0.85 & 0.90 \\
1202 & 100162 & 1990 & 7.61 & -0.11 & 759.00 & 6564.15 & 1.00 & 0.86 & 0.86 \\
5281 & 100746 & 1990 & 115.13 & -0.12 & 11511.00 & 114804.66 & 1.00 & 1.00 & 1.00 \\
3452 & 100439 & 1990 & 8.47 & -0.36 & 832.00 & 7699.68 & 1.02 & 0.91 & 0.93 \\
6321 & 100849 & 1990 & 50.08 & 0.03 & 5008.00 & 45221.74 & 1.00 & 0.90 & 0.90 \\
18170 & 102414 & 1990 & 52.70 & -0.02 & 5266.00 & 44443.49 & 1.00 & 0.84 & 0.84 \\
26005 & 103533 & 1990 & 163.57 & 0.17 & 16357.00 & 131151.07 & 1.00 & 0.80 & 0.80 \\
1213 & 100166 & 1990 & 1897.32 & -0.04 & 189732.00 & 1701980.69 & 1.00 & 0.90 & 0.90 \\
25271 & 103464 & 1990 & 243.34 & -0.00 & 22305.00 & 218849.96 & 1.09 & 0.90 & 0.98 \\
49096 & 240222 & 1990 & 247.01 & -0.05 & 24700.00 & 225041.93 & 1.00 & 0.91 & 0.91 \\
3486 & 100441 & 1990 & 422.10 & 0.04 & 42210.00 & 361682.00 & 1.00 & 0.86 & 0.86 \\
20609 & 102775 & 1990 & 567.93 & 0.04 & 58501.00 & 535384.52 & 0.97 & 0.94 & 0.92 \\
20589 & 102774 & 1990 & 237.20 & -0.07 & 25690.00 & 205698.62 & 0.92 & 0.87 & 0.80 \\
19212 & 102570 & 1990 & 109.92 & -0.06 & 11047.00 & 97019.12 & 1.00 & 0.88 & 0.88 \\
16332 & 102127 & 1990 & 5.94 & 0.03 & 594.00 & 5194.34 & 1.00 & 0.87 & 0.87 \\
11274 & 101390 & 1990 & 1053.52 & -0.03 & 97190.00 & 933735.58 & 1.08 & 0.89 & 0.96 \\
20377 & 102733 & 1990 & 2394.85 & 0.01 & 249264.00 & 2246411.96 & 0.96 & 0.94 & 0.90 \\
20372 & 102732 & 1990 & 260.08 & -0.10 & 28278.00 & 257118.97 & 0.92 & 0.99 & 0.91 \\
16298 & 102124 & 1990 & 493.14 & 0.05 & 49314.00 & 448124.52 & 1.00 & 0.91 & 0.91 \\
20369 & 102730 & 1990 & 53.65 & -0.03 & 5427.00 & 43727.04 & 0.99 & 0.81 & 0.81 \\
49053 & 240212 & 1990 & 376.13 & 0.16 & 33508.00 & 335452.63 & 1.12 & 0.89 & 1.00 \\
8669 & 101094 & 1990 & 11.30 & 0.01 & 840.00 & 9270.26 & 1.35 & 0.82 & 1.10 \\
25544 & 103496 & 1990 & 236.09 & -0.02 & 23609.00 & 201612.74 & 1.00 & 0.85 & 0.85 \\
17667 & 102342 & 1990 & 27.11 & 0.02 & 2746.00 & 26388.07 & 0.99 & 0.97 & 0.96 \\
20326 & 102716 & 1990 & 127.22 & 0.02 & 12113.00 & 118235.16 & 1.05 & 0.93 & 0.98 \\
20292 & 102715 & 1990 & 575.14 & -0.04 & 65719.00 & 511199.41 & 0.88 & 0.89 & 0.78 \\
20280 & 102709 & 1990 & 4.17 & -0.38 & 356.00 & 3787.80 & 1.17 & 0.91 & 1.06 \\
15629 & 102010 & 1990 & 245.56 & -0.07 & 14678.00 & 127996.27 & 1.67 & 0.52 & 0.87 \\
264 & 100030 & 1990 & 24.97 & 0.04 & 2989.00 & 23265.78 & 0.84 & 0.93 & 0.78 \\
11946 & 101473 & 1990 & 216.45 & -0.12 & 21200.00 & 171333.81 & 1.02 & 0.79 & 0.81 \\
16352 & 102130 & 1990 & 219.41 & -0.08 & 21941.00 & 193593.36 & 1.00 & 0.88 & 0.88 \\
20577 & 102768 & 1990 & 737.40 & 0.02 & 72487.00 & 581221.47 & 1.02 & 0.79 & 0.80 \\
20549 & 102767 & 1990 & 643.19 & -0.03 & 70419.00 & 594557.93 & 0.91 & 0.92 & 0.84 \\
17610 & 102321 & 1990 & 70.94 & -0.02 & 7094.00 & 63473.97 & 1.00 & 0.89 & 0.89 \\
57923 & 410003 & 1990 & 263.90 & -0.23 & 26711.00 & 237991.89 & 0.99 & 0.90 & 0.89 \\
5811 & 100801 & 1990 & 39.35 & -0.02 & 3931.00 & 39703.28 & 1.00 & 1.01 & 1.01 \\
312 & 100036 & 1990 & 29.84 & 0.04 & 2874.00 & 26329.38 & 1.04 & 0.88 & 0.92 \\
25884 & 103526 & 1990 & 696.88 & -0.02 & 69688.00 & 606871.34 & 1.00 & 0.87 & 0.87 \\
20758 & 102789 & 1990 & 55.37 & -0.07 & 4974.00 & 44185.88 & 1.11 & 0.80 & 0.89 \\
0 & 100001 & 1990 & 169.57 & 0.03 & 16957.00 & 161803.10 & 1.00 & 0.95 & 0.95 \\
24964 & 103397 & 1990 & 11.99 & -0.05 & 1199.00 & 11483.40 & 1.00 & 0.96 & 0.96 \\
14697 & 101911 & 1990 & 47.89 & -0.30 & 6671.00 & 43859.15 & 0.72 & 0.92 & 0.66 \\
22367 & 103008 & 1990 & 51.47 & 0.01 & 5146.00 & 49247.21 & 1.00 & 0.96 & 0.96 \\
9535 & 101149 & 1990 & 26.98 & 0.09 & 2698.00 & 23364.80 & 1.00 & 0.87 & 0.87 \\
2126 & 100292 & 1990 & 207.48 & -0.16 & 20505.00 & 195243.99 & 1.01 & 0.94 & 0.95 \\
24966 & 103399 & 1990 & 13.50 & 0.02 & 1349.00 & 12647.19 & 1.00 & 0.94 & 0.94 \\
22934 & 103089 & 1990 & 72.66 & -0.03 & 8297.00 & 67768.80 & 0.88 & 0.93 & 0.82 \\
7191 & 101013 & 1990 & 69.50 & -0.02 & 6949.00 & 63588.10 & 1.00 & 0.91 & 0.92 \\
23562 & 103193 & 1990 & 22.33 & 0.02 & 2230.00 & 18373.38 & 1.00 & 0.82 & 0.82 \\
24635 & 103373 & 1990 & 162.20 & -0.05 & 17086.00 & 147445.19 & 0.95 & 0.91 & 0.86 \\
6677 & 100910 & 1990 & 28.80 & 0.02 & 2925.00 & 28516.74 & 0.98 & 0.99 & 0.97 \\
13968 & 101794 & 1990 & 38.54 & -0.03 & 3853.00 & 36203.78 & 1.00 & 0.94 & 0.94 \\
14731 & 101912 & 1990 & 129.64 & -0.16 & 15147.00 & 122023.15 & 0.86 & 0.94 & 0.81 \\
2905 & 100379 & 1990 & 119.58 & -0.11 & 11958.00 & 103814.44 & 1.00 & 0.87 & 0.87 \\
24675 & 103375 & 1990 & 28.99 & -0.17 & 2400.00 & 23563.78 & 1.21 & 0.81 & 0.98 \\
9208 & 101119 & 1990 & 22.21 & -0.12 & 2221.00 & 19311.69 & 1.00 & 0.87 & 0.87 \\
14012 & 101800 & 1990 & 281.62 & -0.07 & 23406.00 & 205808.88 & 1.20 & 0.73 & 0.88 \\
14900 & 101921 & 1990 & 21.72 & -0.00 & 2172.00 & 19217.12 & 1.00 & 0.88 & 0.88 \\
10136 & 101263 & 1990 & 233.59 & 0.02 & 23359.00 & 201675.00 & 1.00 & 0.86 & 0.86 \\
13830 & 101769 & 1990 & 348.43 & 0.04 & 34843.00 & 333831.42 & 1.00 & 0.96 & 0.96 \\
9063 & 101110 & 1990 & 4.60 & -0.15 & 357.00 & 3442.57 & 1.29 & 0.75 & 0.96 \\
1674 & 100223 & 1990 & 132.69 & -0.02 & 13269.00 & 118872.18 & 1.00 & 0.90 & 0.90 \\
14943 & 101925 & 1990 & 155.02 & -0.05 & 17601.00 & 140222.11 & 0.88 & 0.90 & 0.80 \\
13857 & 101781 & 1990 & 206.32 & -0.05 & 20632.00 & 185175.28 & 1.00 & 0.90 & 0.90 \\
13874 & 101785 & 1990 & 502.79 & 0.03 & 50279.00 & 429333.78 & 1.00 & 0.85 & 0.85 \\
109 & 100009 & 1990 & 16.30 & -0.05 & 1432.00 & 14055.55 & 1.14 & 0.86 & 0.98 \\
21947 & 102981 & 1990 & 37.49 & -0.27 & 3750.00 & 38392.89 & 1.00 & 1.02 & 1.02 \\
7420 & 101039 & 1990 & 236.00 & -0.03 & 23608.00 & 227308.71 & 1.00 & 0.96 & 0.96 \\
13211 & 101704 & 1990 & 301.39 & -0.02 & 30139.00 & 262960.26 & 1.00 & 0.87 & 0.87 \\
63168 & 500486 & 1990 & 118.21 & -0.05 & 13767.00 & 113718.72 & 0.86 & 0.96 & 0.83 \\
23287 & 103158 & 1990 & 492.64 & 0.03 & 49700.00 & 480547.85 & 0.99 & 0.98 & 0.97 \\
1661 & 100222 & 1990 & 55.00 & -0.17 & 5494.00 & 51902.62 & 1.00 & 0.94 & 0.94 \\
1737 & 100227 & 1990 & 97.28 & -0.03 & 9730.00 & 90282.65 & 1.00 & 0.93 & 0.93 \\
2944 & 100389 & 1990 & 32.32 & -0.01 & 3232.00 & 26787.80 & 1.00 & 0.83 & 0.83 \\
14459 & 101861 & 1990 & 330.85 & 0.03 & 33085.00 & 298710.01 & 1.00 & 0.90 & 0.90 \\
24831 & 103381 & 1990 & 458.29 & 0.09 & 33300.00 & 384323.19 & 1.38 & 0.84 & 1.15 \\
2496 & 100336 & 1990 & 11.16 & -0.08 & 780.00 & 6518.97 & 1.43 & 0.58 & 0.84 \\
9486 & 101140 & 1990 & 340.31 & 0.03 & 34031.00 & 326910.57 & 1.00 & 0.96 & 0.96 \\
22323 & 103007 & 1990 & 550.25 & 0.03 & 55023.00 & 528200.26 & 1.00 & 0.96 & 0.96 \\
13217 & 101708 & 1990 & 55.73 & 0.03 & 6220.00 & 54304.15 & 0.90 & 0.97 & 0.87 \\
14122 & 101805 & 1990 & 492.52 & -0.06 & 47927.00 & 449160.90 & 1.03 & 0.91 & 0.94 \\
2388 & 100322 & 1990 & 22.02 & 0.02 & 2202.00 & 19825.11 & 1.00 & 0.90 & 0.90 \\
9173 & 101116 & 1990 & 215.00 & 0.01 & 17967.00 & 151250.08 & 1.20 & 0.70 & 0.84 \\
8958 & 101107 & 1990 & 699.70 & -0.13 & 83852.00 & 630502.74 & 0.83 & 0.90 & 0.75 \\
22176 & 102994 & 1990 & 50.47 & -0.07 & 5047.00 & 48420.26 & 1.00 & 0.96 & 0.96 \\
24752 & 103377 & 1990 & 210.63 & -0.09 & 19600.00 & 172554.37 & 1.07 & 0.82 & 0.88 \\
2304 & 100315 & 1990 & 221.29 & -0.09 & 22129.00 & 207625.98 & 1.00 & 0.94 & 0.94 \\
7074 & 100994 & 1990 & 15.11 & -0.04 & 1511.00 & 14041.96 & 1.00 & 0.93 & 0.93 \\
56 & 100004 & 1990 & 173.87 & -0.16 & 16009.00 & 167453.64 & 1.09 & 0.96 & 1.05 \\
22235 & 102997 & 1990 & 35.29 & -0.11 & 3529.00 & 32349.58 & 1.00 & 0.92 & 0.92 \\
2324 & 100319 & 1990 & 79.51 & -0.16 & 7951.00 & 75622.62 & 1.00 & 0.95 & 0.95 \\
7036 & 100992 & 1990 & 151.25 & -0.05 & 15125.00 & 129976.03 & 1.00 & 0.86 & 0.86 \\
22207 & 102996 & 1990 & 120.76 & 0.01 & 12076.00 & 108791.90 & 1.00 & 0.90 & 0.90 \\
14155 & 101819 & 1990 & 87.79 & -0.16 & 8779.00 & 73199.73 & 1.00 & 0.83 & 0.83 \\
23240 & 103152 & 1990 & 428.48 & 0.04 & 41431.00 & 397055.76 & 1.03 & 0.93 & 0.96 \\
6738 & 100947 & 1990 & 580.08 & -0.12 & 58008.00 & 482817.32 & 1.00 & 0.83 & 0.83 \\
9741 & 101186 & 1990 & 55.75 & 0.02 & 4336.00 & 37045.47 & 1.29 & 0.66 & 0.85 \\
24791 & 103380 & 1990 & 3196.66 & -0.00 & 291000.00 & 3078330.40 & 1.10 & 0.96 & 1.06 \\
6713 & 100918 & 1990 & 1.59 & 0.04 & 301.00 & 2523.24 & 0.53 & 1.58 & 0.84 \\
514 & 100072 & 1990 & 108.64 & -0.04 & 10864.00 & 107763.19 & 1.00 & 0.99 & 0.99 \\
9573 & 101151 & 1990 & 32.93 & -0.10 & 3293.00 & 32343.17 & 1.00 & 0.98 & 0.98 \\
9913 & 101212 & 1990 & 177.94 & 0.02 & 17800.00 & 143214.19 & 1.00 & 0.80 & 0.80 \\
7155 & 101000 & 1990 & 351.21 & 0.05 & 35120.00 & 342346.71 & 1.00 & 0.97 & 0.97 \\
6968 & 100977 & 1990 & 7.45 & -0.06 & 745.00 & 7054.62 & 1.00 & 0.95 & 0.95 \\
1804 & 100238 & 1990 & 33.13 & -0.13 & 3313.00 & 27535.72 & 1.00 & 0.83 & 0.83 \\
14865 & 101919 & 1990 & 106.15 & -0.06 & 13366.00 & 100546.63 & 0.79 & 0.95 & 0.75 \\
407 & 100055 & 1990 & 3210.47 & 0.03 & 303295.00 & 2772491.12 & 1.06 & 0.86 & 0.91 \\
14053 & 101801 & 1990 & 152.83 & -0.26 & 17260.00 & 124888.50 & 0.89 & 0.82 & 0.72 \\
2425 & 100324 & 1990 & 19.57 & -0.08 & 1957.00 & 17184.23 & 1.00 & 0.88 & 0.88 \\
13343 & 101729 & 1990 & 185.49 & -0.04 & 18551.00 & 153044.35 & 1.00 & 0.83 & 0.82 \\
14105 & 101804 & 1990 & 214.00 & -0.09 & 19988.00 & 185727.23 & 1.07 & 0.87 & 0.93 \\
2408 & 100323 & 1990 & 35.67 & 0.03 & 3567.00 & 28899.02 & 1.00 & 0.81 & 0.81 \\
2190 & 100295 & 1990 & 8.65 & -0.07 & 433.00 & 4073.16 & 2.00 & 0.47 & 0.94 \\
22132 & 102993 & 1990 & 400.60 & 0.07 & 40060.00 & 361373.23 & 1.00 & 0.90 & 0.90 \\
14087 & 101802 & 1990 & 162.30 & 0.02 & 15412.00 & 142484.70 & 1.05 & 0.88 & 0.92 \\
24715 & 103376 & 1990 & 1914.71 & -0.04 & 170600.00 & 1720013.49 & 1.12 & 0.90 & 1.01 \\
13362 & 101730 & 1990 & 9.99 & -0.00 & 999.00 & 9405.01 & 1.00 & 0.94 & 0.94 \\
9895 & 101211 & 1990 & 41.27 & 0.04 & 3701.00 & 33179.24 & 1.12 & 0.80 & 0.90 \\
7268 & 101018 & 1990 & 131.40 & 0.04 & 13106.00 & 119647.78 & 1.00 & 0.91 & 0.91 \\
22719 & 103042 & 1990 & 23.24 & 0.03 & 2323.00 & 22747.49 & 1.00 & 0.98 & 0.98 \\
9403 & 101134 & 1990 & 42.78 & -0.23 & 4278.00 & 36592.97 & 1.00 & 0.86 & 0.86 \\
2984 & 100395 & 1990 & 472.59 & -0.13 & 47150.00 & 399785.12 & 1.00 & 0.85 & 0.85 \\
15015 & 101943 & 1990 & 38.23 & -0.15 & 5509.00 & 33717.33 & 0.69 & 0.88 & 0.61 \\
7304 & 101020 & 1990 & 1414.60 & -0.16 & 135380.00 & 1185551.36 & 1.04 & 0.84 & 0.88 \\
2599 & 100346 & 1990 & 3.90 & 0.06 & 390.00 & 3612.27 & 1.00 & 0.93 & 0.93 \\
621 & 100085 & 1990 & 86.56 & 0.02 & 8656.00 & 75725.53 & 1.00 & 0.87 & 0.87 \\
74637 & 601143 & 1990 & 85.61 & -0.06 & 8154.00 & 74933.13 & 1.05 & 0.88 & 0.92 \\
24307 & 103308 & 1990 & 1578.05 & 0.04 & 157805.00 & 1281116.24 & 1.00 & 0.81 & 0.81 \\
3014 & 100398 & 1990 & 43.57 & -0.05 & 4357.00 & 40476.89 & 1.00 & 0.93 & 0.93 \\
10248 & 101276 & 1990 & 66.14 & -0.14 & 6614.00 & 63432.14 & 1.00 & 0.96 & 0.96 \\
6627 & 100906 & 1990 & 371.70 & -0.06 & 37170.00 & 329256.16 & 1.00 & 0.89 & 0.89 \\
22733 & 103050 & 1990 & 153.50 & -0.07 & 15350.00 & 134900.26 & 1.00 & 0.88 & 0.88 \\
9438 & 101135 & 1990 & 160.32 & 0.26 & 16032.00 & 129314.90 & 1.00 & 0.81 & 0.81 \\
586 & 100079 & 1990 & 316.08 & 0.01 & 31608.00 & 315442.82 & 1.00 & 1.00 & 1.00 \\
9377 & 101133 & 1990 & 3.86 & -0.07 & 386.00 & 3946.30 & 1.00 & 1.02 & 1.02 \\
1526 & 100213 & 1990 & 115.90 & -0.03 & 10746.00 & 105866.37 & 1.08 & 0.91 & 0.99 \\
24562 & 103368 & 1990 & 99.60 & 0.01 & 9960.00 & 86921.50 & 1.00 & 0.87 & 0.87 \\
22627 & 103027 & 1990 & 91.06 & -0.06 & 9106.00 & 81479.50 & 1.00 & 0.89 & 0.89 \\
7382 & 101038 & 1990 & 1690.40 & -0.03 & 168983.00 & 1532340.89 & 1.00 & 0.91 & 0.91 \\
13651 & 101754 & 1990 & 25.83 & -0.01 & 2583.00 & 21658.45 & 1.00 & 0.84 & 0.84 \\
6825 & 100962 & 1990 & 237.96 & -0.11 & 23795.00 & 211721.09 & 1.00 & 0.89 & 0.89 \\
22559 & 103021 & 1990 & 18.71 & -0.19 & 1871.00 & 17414.29 & 1.00 & 0.93 & 0.93 \\
22671 & 103028 & 1990 & 1103.39 & -0.04 & 110338.00 & 912218.89 & 1.00 & 0.83 & 0.83 \\
21803 & 102952 & 1990 & 310.41 & 0.01 & 31040.00 & 283945.23 & 1.00 & 0.91 & 0.91 \\
24525 & 103339 & 1990 & 18.16 & -0.01 & 1816.00 & 17886.23 & 1.00 & 0.98 & 0.98 \\
22706 & 103029 & 1990 & 246.93 & -0.07 & 24692.00 & 222048.49 & 1.00 & 0.90 & 0.90 \\
24520 & 103338 & 1990 & 12.50 & 0.00 & 1250.00 & 12344.99 & 1.00 & 0.99 & 0.99 \\
13794 & 101764 & 1990 & 35.51 & 0.02 & 3551.00 & 31058.67 & 1.00 & 0.87 & 0.87 \\
22736 & 103057 & 1990 & 444.67 & 0.05 & 44467.00 & 379786.53 & 1.00 & 0.85 & 0.85 \\
24395 & 103319 & 1990 & 64.53 & 0.02 & 6453.00 & 56021.64 & 1.00 & 0.87 & 0.87 \\
21898 & 102976 & 1990 & 222.15 & -0.10 & 22210.00 & 220209.93 & 1.00 & 0.99 & 0.99 \\
1997 & 100280 & 1990 & 107.30 & 0.02 & 10730.00 & 92677.11 & 1.00 & 0.86 & 0.86 \\
24189 & 103296 & 1990 & 343.24 & 0.00 & 31407.00 & 280996.27 & 1.09 & 0.82 & 0.89 \\
561 & 100076 & 1990 & 153.25 & -0.05 & 15324.00 & 145937.76 & 1.00 & 0.95 & 0.95 \\
6802 & 100956 & 1990 & 1.86 & -0.24 & 268.00 & 1817.11 & 0.70 & 0.97 & 0.68 \\
14988 & 101930 & 1990 & 289.81 & 0.02 & 25279.00 & 244874.47 & 1.15 & 0.84 & 0.97 \\
24872 & 103383 & 1990 & 389.10 & -0.16 & 34100.00 & 338856.73 & 1.14 & 0.87 & 0.99 \\
24224 & 103299 & 1990 & 38.86 & -0.07 & 3347.00 & 32937.03 & 1.16 & 0.85 & 0.98 \\
96654 & 611002 & 1991 & 1371.52 & 0.39 & 137152.00 & 1321659.10 & 1.00 & 0.96 & 0.96 \\
20134 & 102669 & 1991 & 17.30 & 0.38 & 1820.00 & 17181.00 & 0.95 & 0.99 & 0.94 \\
20101 & 102667 & 1991 & 486.02 & 0.33 & 48602.00 & 400410.23 & 1.00 & 0.82 & 0.82 \\
3944 & 100517 & 1991 & 91.08 & 0.38 & 9130.00 & 89103.81 & 1.00 & 0.98 & 0.98 \\
16388 & 102132 & 1991 & 19.95 & 0.38 & 1996.00 & 18760.88 & 1.00 & 0.94 & 0.94 \\
20145 & 102671 & 1991 & 26.77 & 0.28 & 2742.00 & 26558.49 & 0.98 & 0.99 & 0.97 \\
10964 & 101357 & 1991 & 7.28 & 0.22 & 728.00 & 7009.55 & 1.00 & 0.96 & 0.96 \\
22672 & 103028 & 1991 & 1750.07 & 0.26 & 175288.00 & 1519754.56 & 1.00 & 0.87 & 0.87 \\
2389 & 100322 & 1991 & 27.74 & 0.29 & 2772.00 & 24894.61 & 1.00 & 0.90 & 0.90 \\
7862 & 101064 & 1991 & 153.90 & -0.06 & 17475.00 & 152506.67 & 0.88 & 0.99 & 0.87 \\
9896 & 101211 & 1991 & 65.45 & 0.28 & 5697.00 & 51944.83 & 1.15 & 0.79 & 0.91 \\
41261 & 108710 & 1991 & 31.58 & 0.33 & 3160.00 & 27489.52 & 1.00 & 0.87 & 0.87 \\
25488 & 103494 & 1991 & 320.63 & 0.49 & 32284.00 & 268803.93 & 0.99 & 0.84 & 0.83 \\
8621 & 101092 & 1991 & 9.90 & 0.65 & 879.00 & 8007.48 & 1.13 & 0.81 & 0.91 \\
16338 & 102129 & 1991 & 11.26 & 0.17 & 1126.00 & 10639.82 & 1.00 & 0.95 & 0.94 \\
5812 & 100801 & 1991 & 59.72 & 0.35 & 5972.00 & 49724.00 & 1.00 & 0.83 & 0.83 \\
22707 & 103029 & 1991 & 228.74 & 0.22 & 24055.00 & 218464.63 & 0.95 & 0.96 & 0.91 \\
408 & 100055 & 1991 & 3605.22 & 0.22 & 356523.00 & 2911370.24 & 1.01 & 0.81 & 0.82 \\
2409 & 100323 & 1991 & 55.67 & 0.30 & 5567.00 & 50326.16 & 1.00 & 0.90 & 0.90 \\
3942 & 100515 & 1991 & 16.46 & 0.10 & 1464.00 & 14287.51 & 1.12 & 0.87 & 0.98 \\
53396 & 346113 & 1991 & 108.46 & 0.48 & 9860.00 & 95392.67 & 1.10 & 0.88 & 0.97 \\
26905 & 103621 & 1991 & 52.71 & 0.28 & 4909.00 & 51749.30 & 1.07 & 0.98 & 1.05 \\
22871 & 103074 & 1991 & 46.30 & 0.10 & 4247.00 & 41901.44 & 1.09 & 0.90 & 0.99 \\
7969 & 101068 & 1991 & 19257.10 & 0.41 & 1730635.00 & 17042453.89 & 1.11 & 0.88 & 0.98 \\
15836 & 102043 & 1991 & 49.30 & 0.25 & 4929.00 & 45123.25 & 1.00 & 0.92 & 0.92 \\
7601 & 101048 & 1991 & 86.70 & 0.65 & 6877.00 & 56412.81 & 1.26 & 0.65 & 0.82 \\
25401 & 103483 & 1991 & 61.81 & 0.34 & 6181.00 & 60024.11 & 1.00 & 0.97 & 0.97 \\
22628 & 103027 & 1991 & 183.87 & 0.42 & 15551.00 & 148222.13 & 1.18 & 0.81 & 0.95 \\
1758 & 100228 & 1991 & 92.98 & 0.29 & 9935.00 & 84494.55 & 0.94 & 0.91 & 0.85 \\
24873 & 103383 & 1991 & 766.87 & 0.74 & 62214.00 & 720708.44 & 1.23 & 0.94 & 1.16 \\
1214 & 100166 & 1991 & 2734.25 & 0.37 & 273425.00 & 2300718.91 & 1.00 & 0.84 & 0.84 \\
20228 & 102689 & 1991 & 10.39 & 0.42 & 1039.00 & 9222.46 & 1.00 & 0.89 & 0.89 \\
7582 & 101047 & 1991 & 187.90 & 0.12 & 16476.00 & 186697.38 & 1.14 & 0.99 & 1.13 \\
4340 & 100610 & 1991 & 377.90 & 0.26 & 37793.00 & 315176.62 & 1.00 & 0.83 & 0.83 \\
23241 & 103152 & 1991 & 633.15 & 0.42 & 55838.00 & 601933.00 & 1.13 & 0.95 & 1.08 \\
65196 & 500670 & 1991 & 568.86 & 0.03 & 56465.00 & 452291.72 & 1.01 & 0.80 & 0.80 \\
9140 & 101115 & 1991 & 1050.50 & 0.30 & 96044.00 & 980746.87 & 1.09 & 0.93 & 1.02 \\
22785 & 103065 & 1991 & 208.96 & 0.65 & 20760.00 & 173981.70 & 1.01 & 0.83 & 0.84 \\
3963 & 100535 & 1991 & 157.90 & 0.32 & 14719.00 & 155295.91 & 1.07 & 0.98 & 1.06 \\
19831 & 102653 & 1991 & 1846.02 & 0.34 & 184602.00 & 1477572.51 & 1.00 & 0.80 & 0.80 \\
13257 & 101714 & 1991 & 75.02 & 0.42 & 7503.00 & 65030.39 & 1.00 & 0.87 & 0.87 \\
11342 & 101396 & 1991 & 35.24 & 0.18 & 3533.00 & 28743.15 & 1.00 & 0.82 & 0.81 \\
9064 & 101110 & 1991 & 11.20 & 0.35 & 1185.00 & 10028.93 & 0.95 & 0.90 & 0.85 \\
16508 & 102152 & 1991 & 73.17 & 0.11 & 6893.00 & 65337.62 & 1.06 & 0.89 & 0.95 \\
16531 & 102154 & 1991 & 80.17 & 0.09 & 7438.00 & 70827.95 & 1.08 & 0.88 & 0.95 \\
57724 & 401001 & 1991 & 15.99 & 0.55 & 1599.00 & 15022.52 & 1.00 & 0.94 & 0.94 \\
10917 & 101354 & 1991 & 108.08 & 0.30 & 10814.00 & 97186.81 & 1.00 & 0.90 & 0.90 \\
8453 & 101087 & 1991 & 29.70 & 0.82 & 1580.00 & 16455.17 & 1.88 & 0.55 & 1.04 \\
14267 & 101842 & 1991 & 31.59 & 0.34 & 3123.00 & 29530.80 & 1.01 & 0.93 & 0.95 \\
18809 & 102523 & 1991 & 789.14 & 0.32 & 78910.00 & 650212.03 & 1.00 & 0.82 & 0.82 \\
24792 & 103380 & 1991 & 4634.05 & 0.32 & 416310.00 & 4486064.37 & 1.11 & 0.97 & 1.08 \\
49097 & 240222 & 1991 & 328.57 & 0.35 & 32834.00 & 286752.91 & 1.00 & 0.87 & 0.87 \\
16333 & 102127 & 1991 & 7.78 & 0.27 & 778.00 & 7805.73 & 1.00 & 1.00 & 1.00 \\
11275 & 101390 & 1991 & 1367.09 & 0.29 & 128290.00 & 1230465.61 & 1.07 & 0.90 & 0.96 \\
4204 & 100575 & 1991 & 19.75 & 0.13 & 1980.00 & 20303.65 & 1.00 & 1.03 & 1.03 \\
14546 & 101876 & 1991 & 121.91 & 0.25 & 12191.00 & 119453.47 & 1.00 & 0.98 & 0.98 \\
13212 & 101704 & 1991 & 546.58 & 0.57 & 54658.00 & 498843.27 & 1.00 & 0.91 & 0.91 \\
12562 & 101554 & 1991 & 59.20 & 0.09 & 4905.00 & 47456.97 & 1.21 & 0.80 & 0.97 \\
4099 & 100550 & 1991 & 12.56 & 0.46 & 1256.00 & 12165.77 & 1.00 & 0.97 & 0.97 \\
2468 & 100333 & 1991 & 90.44 & 0.26 & 9043.00 & 74905.67 & 1.00 & 0.83 & 0.83 \\
16217 & 102097 & 1991 & 21.80 & 0.19 & 1986.00 & 16211.44 & 1.10 & 0.74 & 0.82 \\
19460 & 102606 & 1991 & 4176.00 & 0.26 & 417642.00 & 3405765.89 & 1.00 & 0.82 & 0.82 \\
65126 & 500664 & 1991 & 243.39 & 0.27 & 17934.00 & 172800.07 & 1.36 & 0.71 & 0.96 \\
19483 & 102607 & 1991 & 727.00 & 0.36 & 72727.00 & 700666.59 & 1.00 & 0.96 & 0.96 \\
22889 & 103084 & 1991 & 37.40 & 0.47 & 3764.00 & 39059.48 & 0.99 & 1.04 & 1.04 \\
74783 & 601171 & 1991 & 54.60 & 0.30 & 5148.00 & 54454.18 & 1.06 & 1.00 & 1.06 \\
26812 & 103608 & 1991 & 32.20 & 0.34 & 2842.00 & 31564.56 & 1.13 & 0.98 & 1.11 \\
19426 & 102601 & 1991 & 1615.00 & 0.23 & 161457.00 & 1464402.86 & 1.00 & 0.91 & 0.91 \\
14558 & 101881 & 1991 & 12.38 & 0.32 & 1238.00 & 11954.44 & 1.00 & 0.97 & 0.97 \\
5966 & 100813 & 1991 & 172.47 & 0.30 & 8637.00 & 89363.10 & 2.00 & 0.52 & 1.03 \\
11210 & 101376 & 1991 & 81.37 & 0.18 & 8137.00 & 83691.65 & 1.00 & 1.03 & 1.03 \\
22939 & 103090 & 1991 & 583.69 & 0.26 & 55538.00 & 483642.86 & 1.05 & 0.83 & 0.87 \\
4121 & 100559 & 1991 & 4.45 & 0.33 & 441.00 & 4044.51 & 1.01 & 0.91 & 0.92 \\
22935 & 103089 & 1991 & 87.84 & 0.27 & 7265.00 & 83752.76 & 1.21 & 0.95 & 1.15 \\
16161 & 102089 & 1991 & 295.27 & 0.15 & 30211.00 & 268262.80 & 0.98 & 0.91 & 0.89 \\
562 & 100076 & 1991 & 185.51 & 0.37 & 18551.00 & 182183.46 & 1.00 & 0.98 & 0.98 \\
1738 & 100227 & 1991 & 118.55 & 0.37 & 12600.00 & 125592.40 & 0.94 & 1.06 & 1.00 \\
3909 & 100514 & 1991 & 103.55 & 0.35 & 10355.00 & 88282.43 & 1.00 & 0.85 & 0.85 \\
22737 & 103057 & 1991 & 498.90 & 0.29 & 45540.00 & 494720.90 & 1.10 & 0.99 & 1.09 \\
16133 & 102085 & 1991 & 916.07 & 0.25 & 91650.00 & 833402.30 & 1.00 & 0.91 & 0.91 \\
19913 & 102655 & 1991 & 756.82 & 0.15 & 75682.00 & 675775.89 & 1.00 & 0.89 & 0.89 \\
26785 & 103607 & 1991 & 178.20 & 0.67 & 15545.00 & 171317.89 & 1.15 & 0.96 & 1.10 \\
11149 & 101369 & 1991 & 420.93 & 0.34 & 42259.00 & 388282.84 & 1.00 & 0.92 & 0.92 \\
16017 & 102073 & 1991 & 2499.83 & 0.27 & 249983.00 & 2295171.11 & 1.00 & 0.92 & 0.92 \\
5933 & 100812 & 1991 & 704.58 & 0.51 & 70460.00 & 568360.54 & 1.00 & 0.81 & 0.81 \\
14439 & 101858 & 1991 & 89.83 & 0.33 & 8175.00 & 80014.80 & 1.10 & 0.89 & 0.98 \\
19540 & 102614 & 1991 & 136.96 & 0.48 & 10729.00 & 116265.68 & 1.28 & 0.85 & 1.08 \\
3896 & 100510 & 1991 & 14.42 & 0.48 & 1442.00 & 14102.76 & 1.00 & 0.98 & 0.98 \\
8558 & 101090 & 1991 & 20.30 & 0.31 & 1898.00 & 17085.12 & 1.07 & 0.84 & 0.90 \\
11187 & 101370 & 1991 & 17.44 & 0.34 & 1744.00 & 16546.44 & 1.00 & 0.95 & 0.95 \\
26753 & 103606 & 1991 & 21.60 & 0.31 & 2102.00 & 21336.48 & 1.03 & 0.99 & 1.02 \\
4040 & 100543 & 1991 & 325.96 & 0.99 & 29557.00 & 321270.10 & 1.10 & 0.99 & 1.09 \\
16226 & 102102 & 1991 & 100.59 & 0.74 & 10058.00 & 94056.45 & 1.00 & 0.94 & 0.94 \\
19392 & 102600 & 1991 & 12.00 & 0.26 & 1247.00 & 9799.58 & 0.96 & 0.82 & 0.79 \\
22720 & 103042 & 1991 & 55.82 & 0.58 & 4492.00 & 42715.93 & 1.24 & 0.77 & 0.95 \\
9075 & 101111 & 1991 & 36.30 & 0.66 & 3512.00 & 33617.25 & 1.03 & 0.93 & 0.96 \\
10975 & 101358 & 1991 & 88.12 & 0.23 & 8812.00 & 85642.00 & 1.00 & 0.97 & 0.97 \\
13218 & 101708 & 1991 & 70.26 & 0.31 & 7026.00 & 68726.60 & 1.00 & 0.98 & 0.98 \\
26749 & 103605 & 1991 & 110.90 & 0.31 & 10126.00 & 107730.85 & 1.10 & 0.97 & 1.06 \\
25545 & 103496 & 1991 & 347.62 & 0.46 & 36957.00 & 320412.39 & 0.94 & 0.92 & 0.87 \\
313 & 100036 & 1991 & 42.08 & 0.41 & 3836.00 & 35105.24 & 1.10 & 0.83 & 0.92 \\
2432 & 100330 & 1991 & 46.30 & 0.34 & 4630.00 & 42251.45 & 1.00 & 0.91 & 0.91 \\
22717 & 103039 & 1991 & 54.39 & 0.38 & 4951.00 & 51414.76 & 1.10 & 0.95 & 1.04 \\
47736 & 221051 & 1991 & 1835.20 & 0.46 & 158935.00 & 1621496.89 & 1.15 & 0.88 & 1.02 \\
16299 & 102124 & 1991 & 678.15 & 0.30 & 67815.00 & 636521.51 & 1.00 & 0.94 & 0.94 \\
20006 & 102663 & 1991 & 166.05 & 0.60 & 16605.00 & 147751.43 & 1.00 & 0.89 & 0.89 \\
15905 & 102059 & 1991 & 108.30 & 0.50 & 10835.00 & 101607.19 & 1.00 & 0.94 & 0.94 \\
11015 & 101360 & 1991 & 485.45 & 0.18 & 48386.00 & 401828.77 & 1.00 & 0.83 & 0.83 \\
19869 & 102654 & 1991 & 231.73 & 0.62 & 23173.00 & 195478.35 & 1.00 & 0.84 & 0.84 \\
49054 & 240212 & 1991 & 906.42 & 0.49 & 67608.00 & 682499.57 & 1.34 & 0.75 & 1.01 \\
19941 & 102659 & 1991 & 1470.44 & 0.52 & 147044.00 & 1295355.29 & 1.00 & 0.88 & 0.88 \\
4006 & 100538 & 1991 & 142.13 & 0.39 & 14382.00 & 140767.43 & 0.99 & 0.99 & 0.98 \\
9914 & 101212 & 1991 & 238.57 & 0.23 & 23860.00 & 202530.50 & 1.00 & 0.85 & 0.85 \\
19358 & 102599 & 1991 & 25.00 & 0.37 & 2549.00 & 22639.53 & 0.98 & 0.91 & 0.89 \\
2600 & 100346 & 1991 & 5.01 & 0.26 & 501.00 & 4446.50 & 1.00 & 0.89 & 0.89 \\
1340 & 100190 & 1991 & 713.28 & 0.39 & 71328.00 & 664604.05 & 1.00 & 0.93 & 0.93 \\
11113 & 101368 & 1991 & 576.30 & 0.15 & 57467.00 & 529141.22 & 1.00 & 0.92 & 0.92 \\
14460 & 101861 & 1991 & 378.62 & 0.29 & 36127.00 & 352249.45 & 1.05 & 0.93 & 0.98 \\
16248 & 102105 & 1991 & 47.03 & 0.35 & 4702.00 & 45825.03 & 1.00 & 0.97 & 0.97 \\
7900 & 101065 & 1991 & 1146.50 & 0.99 & 98895.00 & 836479.19 & 1.16 & 0.73 & 0.85 \\
16264 & 102113 & 1991 & 273.20 & 0.13 & 27320.00 & 273430.92 & 1.00 & 1.00 & 1.00 \\
3883 & 100509 & 1991 & 6.44 & 0.27 & 644.00 & 6369.31 & 1.00 & 0.99 & 0.99 \\
5631 & 100780 & 1991 & 31.32 & 0.31 & 3132.00 & 27850.47 & 1.00 & 0.89 & 0.89 \\
25620 & 103498 & 1991 & 129.04 & 0.37 & 13198.00 & 116717.61 & 0.98 & 0.90 & 0.88 \\
8268 & 101081 & 1991 & 53.10 & 0.62 & 4119.00 & 40948.57 & 1.29 & 0.77 & 0.99 \\
17031 & 102231 & 1991 & 598.42 & 0.38 & 59842.00 & 530591.22 & 1.00 & 0.89 & 0.89 \\
254 & 100022 & 1991 & 32.25 & 0.31 & 3225.00 & 31986.91 & 1.00 & 0.99 & 0.99 \\
57893 & 401372 & 1991 & 9.74 & 0.14 & 970.00 & 8652.33 & 1.00 & 0.89 & 0.89 \\
4900 & 100692 & 1991 & 75.38 & 0.56 & 7953.00 & 65539.04 & 0.95 & 0.87 & 0.82 \\
9487 & 101140 & 1991 & 455.03 & 0.42 & 45503.00 & 413850.57 & 1.00 & 0.91 & 0.91 \\
17052 & 102234 & 1991 & 271.74 & 0.29 & 26187.00 & 219443.92 & 1.04 & 0.81 & 0.84 \\
13831 & 101769 & 1991 & 984.56 & 0.67 & 98456.00 & 882503.06 & 1.00 & 0.90 & 0.90 \\
12116 & 101511 & 1991 & 216.57 & 0.50 & 21660.00 & 208365.94 & 1.00 & 0.96 & 0.96 \\
452 & 100056 & 1991 & 5.74 & 0.11 & 735.00 & 5499.22 & 0.78 & 0.96 & 0.75 \\
17066 & 102241 & 1991 & 208.55 & 0.23 & 20637.00 & 193779.91 & 1.01 & 0.93 & 0.94 \\
25952 & 103531 & 1991 & 116.32 & 0.00 & 11632.00 & 107419.77 & 1.00 & 0.92 & 0.92 \\
4930 & 100695 & 1991 & 85.44 & 0.28 & 8398.00 & 78612.76 & 1.02 & 0.92 & 0.94 \\
4956 & 100697 & 1991 & 31.92 & 0.26 & 3210.00 & 31368.87 & 0.99 & 0.98 & 0.98 \\
24190 & 103296 & 1991 & 700.84 & 0.41 & 58146.00 & 605265.21 & 1.21 & 0.86 & 1.04 \\
5316 & 100753 & 1991 & 1657.73 & 0.22 & 166307.00 & 1427985.58 & 1.00 & 0.86 & 0.86 \\
17668 & 102342 & 1991 & 40.77 & 0.37 & 4077.00 & 39292.11 & 1.00 & 0.96 & 0.96 \\
1998 & 100280 & 1991 & 102.10 & 0.13 & 10210.00 & 88511.91 & 1.00 & 0.87 & 0.87 \\
265 & 100030 & 1991 & 40.86 & 0.38 & 3196.00 & 31870.58 & 1.28 & 0.78 & 1.00 \\
4989 & 100698 & 1991 & 13.18 & 0.46 & 1167.00 & 10335.30 & 1.13 & 0.78 & 0.89 \\
5337 & 100754 & 1991 & 415.21 & 0.20 & 41577.00 & 357361.47 & 1.00 & 0.86 & 0.86 \\
4880 & 100691 & 1991 & 122.51 & 0.26 & 12241.00 & 115312.18 & 1.00 & 0.94 & 0.94 \\
13858 & 101781 & 1991 & 352.65 & 0.69 & 35265.00 & 322080.59 & 1.00 & 0.91 & 0.91 \\
12206 & 101519 & 1991 & 107.74 & 0.72 & 10774.00 & 102071.58 & 1.00 & 0.95 & 0.95 \\
7421 & 101039 & 1991 & 413.20 & 0.41 & 32624.00 & 361803.68 & 1.27 & 0.88 & 1.11 \\
4799 & 100682 & 1991 & 33.00 & 0.17 & 3250.00 & 30654.83 & 1.02 & 0.93 & 0.94 \\
26301 & 103567 & 1991 & 45.42 & 0.59 & 4326.00 & 40650.46 & 1.05 & 0.90 & 0.94 \\
13875 & 101785 & 1991 & 646.75 & 0.44 & 64675.00 & 591760.95 & 1.00 & 0.91 & 0.91 \\
5437 & 100763 & 1991 & 103.96 & 0.74 & 10279.00 & 93816.92 & 1.01 & 0.90 & 0.91 \\
18034 & 102387 & 1991 & 8.40 & 0.48 & 844.00 & 7741.02 & 1.00 & 0.92 & 0.92 \\
11705 & 101457 & 1991 & 91.43 & 0.44 & 9143.00 & 82649.96 & 1.00 & 0.90 & 0.90 \\
4832 & 100685 & 1991 & 1.16 & 0.35 & 121.00 & 1205.04 & 0.96 & 1.04 & 1.00 \\
26283 & 103564 & 1991 & 11.43 & 0.71 & 933.00 & 8521.75 & 1.22 & 0.75 & 0.91 \\
17999 & 102386 & 1991 & 78.10 & 0.35 & 7817.00 & 63493.16 & 1.00 & 0.81 & 0.81 \\
23886 & 103228 & 1991 & 32.73 & 0.43 & 3172.00 & 27853.67 & 1.03 & 0.85 & 0.88 \\
16959 & 102224 & 1991 & 131.08 & 0.45 & 13108.00 & 128071.35 & 1.00 & 0.98 & 0.98 \\
17985 & 102383 & 1991 & 6.36 & 0.23 & 636.00 & 6267.84 & 1.00 & 0.99 & 0.99 \\
11739 & 101460 & 1991 & 682.56 & 0.34 & 68515.00 & 602715.92 & 1.00 & 0.88 & 0.88 \\
17959 & 102377 & 1991 & 19.46 & 0.87 & 1477.00 & 16615.87 & 1.32 & 0.85 & 1.12 \\
11772 & 101461 & 1991 & 482.17 & 0.11 & 48056.00 & 430549.42 & 1.00 & 0.89 & 0.90 \\
920 & 100111 & 1991 & 63.47 & 0.33 & 6800.00 & 65660.21 & 0.93 & 1.03 & 0.97 \\
12150 & 101513 & 1991 & 42.44 & 0.21 & 4200.00 & 40250.31 & 1.01 & 0.95 & 0.96 \\
926 & 100112 & 1991 & 426.59 & 0.27 & 33025.00 & 333296.61 & 1.29 & 0.78 & 1.01 \\
24249 & 103300 & 1991 & 8.91 & -0.05 & 751.00 & 6375.09 & 1.19 & 0.72 & 0.85 \\
57924 & 410003 & 1991 & 339.70 & 0.34 & 33900.00 & 335437.62 & 1.00 & 0.99 & 0.99 \\
1 & 100001 & 1991 & 330.03 & 0.30 & 33003.00 & 314952.40 & 1.00 & 0.95 & 0.95 \\
5239 & 100741 & 1991 & 116.40 & 0.06 & 14514.00 & 143875.89 & 0.80 & 1.24 & 0.99 \\
24521 & 103338 & 1991 & 30.98 & 0.32 & 3097.00 & 25779.56 & 1.00 & 0.83 & 0.83 \\
26095 & 103538 & 1991 & 24.74 & 0.47 & 2474.00 & 20806.77 & 1.00 & 0.84 & 0.84 \\
24526 & 103339 & 1991 & 107.71 & 0.59 & 10771.00 & 93189.83 & 1.00 & 0.87 & 0.87 \\
9341 & 101132 & 1991 & 10.98 & -0.03 & 1098.00 & 9852.42 & 1.00 & 0.90 & 0.90 \\
5229 & 100740 & 1991 & 153.00 & 0.22 & 14658.00 & 151390.36 & 1.04 & 0.99 & 1.03 \\
5227 & 100738 & 1991 & 223.30 & 0.26 & 21418.00 & 207811.35 & 1.04 & 0.93 & 0.97 \\
45785 & 200140 & 1991 & 1122.12 & 0.23 & 112212.00 & 988505.95 & 1.00 & 0.88 & 0.88 \\
17375 & 102284 & 1991 & 32.17 & 0.51 & 2922.00 & 31682.30 & 1.10 & 0.98 & 1.08 \\
17428 & 102306 & 1991 & 458.96 & 0.41 & 45896.00 & 429055.84 & 1.00 & 0.93 & 0.93 \\
7383 & 101038 & 1991 & 2293.00 & 0.39 & 202349.00 & 2148997.39 & 1.13 & 0.94 & 1.06 \\
24552 & 103366 & 1991 & 106.03 & 0.12 & 9544.00 & 106001.67 & 1.11 & 1.00 & 1.11 \\
13652 & 101754 & 1991 & 33.84 & 0.32 & 3383.00 & 30263.75 & 1.00 & 0.89 & 0.89 \\
26165 & 103545 & 1991 & 3626.59 & 0.29 & 362659.00 & 3281173.10 & 1.00 & 0.90 & 0.90 \\
5166 & 100730 & 1991 & 136.90 & 0.48 & 10488.00 & 110399.30 & 1.31 & 0.81 & 1.05 \\
9378 & 101133 & 1991 & 90.97 & 0.74 & 9097.00 & 89652.82 & 1.00 & 0.99 & 0.99 \\
12041 & 101491 & 1991 & 73.34 & 0.31 & 7330.00 & 68702.74 & 1.00 & 0.94 & 0.94 \\
24563 & 103368 & 1991 & 114.95 & 0.27 & 11495.00 & 109178.46 & 1.00 & 0.95 & 0.95 \\
57958 & 410010 & 1991 & 91.86 & 0.21 & 9186.00 & 86357.46 & 1.00 & 0.94 & 0.94 \\
5188 & 100731 & 1991 & 3911.00 & 0.31 & 305372.00 & 3021688.02 & 1.28 & 0.77 & 0.99 \\
17243 & 102274 & 1991 & 42.75 & 0.42 & 4275.00 & 41442.09 & 1.00 & 0.97 & 0.97 \\
17611 & 102321 & 1991 & 175.31 & 0.43 & 17530.00 & 152957.88 & 1.00 & 0.87 & 0.87 \\
13795 & 101764 & 1991 & 76.01 & 0.44 & 7601.00 & 66382.89 & 1.00 & 0.87 & 0.87 \\
11947 & 101473 & 1991 & 270.10 & 0.35 & 25757.00 & 228662.86 & 1.05 & 0.85 & 0.89 \\
17574 & 102319 & 1991 & 557.40 & 0.40 & 55740.00 & 487966.51 & 1.00 & 0.88 & 0.88 \\
17181 & 102268 & 1991 & 13.21 & 0.57 & 1321.00 & 11301.56 & 1.00 & 0.86 & 0.86 \\
17540 & 102318 & 1991 & 2103.08 & 0.42 & 210307.00 & 1903526.29 & 1.00 & 0.91 & 0.91 \\
24308 & 103308 & 1991 & 2151.66 & 0.30 & 215166.00 & 1872801.40 & 1.00 & 0.87 & 0.87 \\
9314 & 101131 & 1991 & 141.25 & 0.67 & 14125.00 & 118760.73 & 1.00 & 0.84 & 0.84 \\
5282 & 100746 & 1991 & 172.30 & 0.32 & 13025.00 & 141095.86 & 1.32 & 0.82 & 1.08 \\
1014 & 100127 & 1991 & 191.69 & 0.54 & 19160.00 & 159936.15 & 1.00 & 0.83 & 0.83 \\
17472 & 102307 & 1991 & 39.59 & 0.26 & 3954.00 & 37383.78 & 1.00 & 0.94 & 0.95 \\
17221 & 102271 & 1991 & 996.83 & 0.57 & 99681.00 & 983340.54 & 1.00 & 0.99 & 0.99 \\
8200 & 101079 & 1991 & 72.10 & 0.00 & 8400.00 & 63153.36 & 0.86 & 0.88 & 0.75 \\
25914 & 103529 & 1991 & 334.79 & 0.30 & 33479.00 & 322115.73 & 1.00 & 0.96 & 0.96 \\
5259 & 100743 & 1991 & 306.00 & 0.32 & 19740.00 & 182950.05 & 1.55 & 0.60 & 0.93 \\
5072 & 100715 & 1991 & 89.01 & 0.38 & 8901.00 & 77962.24 & 1.00 & 0.88 & 0.88 \\
24396 & 103319 & 1991 & 88.87 & 0.31 & 7553.00 & 75549.19 & 1.18 & 0.85 & 1.00 \\
12056 & 101494 & 1991 & 145.27 & 0.42 & 14530.00 & 135380.39 & 1.00 & 0.93 & 0.93 \\
9404 & 101134 & 1991 & 35.45 & -0.07 & 3545.00 & 30493.10 & 1.00 & 0.86 & 0.86 \\
8309 & 101082 & 1991 & 1169.80 & 0.84 & 95040.00 & 985966.34 & 1.23 & 0.84 & 1.04 \\
5639 & 100784 & 1991 & 281.30 & 0.21 & 28130.00 & 274089.56 & 1.00 & 0.97 & 0.97 \\
515 & 100072 & 1991 & 486.80 & 0.70 & 48680.00 & 460189.21 & 1.00 & 0.95 & 0.95 \\
11534 & 101427 & 1991 & 96.38 & -0.06 & 9551.00 & 77514.84 & 1.01 & 0.80 & 0.81 \\
14106 & 101804 & 1991 & 278.69 & 0.42 & 27868.00 & 254618.04 & 1.00 & 0.91 & 0.91 \\
13344 & 101729 & 1991 & 231.79 & 0.31 & 23171.00 & 216825.49 & 1.00 & 0.94 & 0.94 \\
12350 & 101537 & 1991 & 102.92 & 0.24 & 7355.00 & 70335.20 & 1.40 & 0.68 & 0.96 \\
8388 & 101085 & 1991 & 264.50 & 0.21 & 23624.00 & 206103.06 & 1.12 & 0.78 & 0.87 \\
4519 & 100637 & 1991 & 264.36 & 0.50 & 26436.00 & 248659.36 & 1.00 & 0.94 & 0.94 \\
18646 & 102500 & 1991 & 360.48 & 0.28 & 29865.00 & 276253.92 & 1.21 & 0.77 & 0.93 \\
5595 & 100773 & 1991 & 730.00 & 0.45 & 73050.00 & 662334.04 & 1.00 & 0.91 & 0.91 \\
18531 & 102471 & 1991 & 114.77 & 0.33 & 10853.00 & 102024.50 & 1.06 & 0.89 & 0.94 \\
24716 & 103376 & 1991 & 3879.12 & 0.69 & 319338.00 & 3571972.16 & 1.21 & 0.92 & 1.12 \\
13363 & 101730 & 1991 & 16.96 & 0.36 & 1706.00 & 16485.71 & 0.99 & 0.97 & 0.97 \\
18517 & 102470 & 1991 & 302.28 & 0.35 & 29009.00 & 292846.64 & 1.04 & 0.97 & 1.01 \\
2238 & 100299 & 1991 & 8.63 & 0.39 & 785.00 & 8255.86 & 1.10 & 0.96 & 1.05 \\
9622 & 101160 & 1991 & 3.26 & 0.19 & 342.00 & 3293.00 & 0.95 & 1.01 & 0.96 \\
4553 & 100639 & 1991 & 92.61 & 0.32 & 9260.00 & 85263.85 & 1.00 & 0.92 & 0.92 \\
18504 & 102469 & 1991 & 77.95 & 0.21 & 5498.00 & 53558.24 & 1.42 & 0.69 & 0.97 \\
2224 & 100298 & 1991 & 34.94 & 0.19 & 3241.00 & 30685.48 & 1.08 & 0.88 & 0.95 \\
14088 & 101802 & 1991 & 191.06 & 0.29 & 19106.00 & 182497.26 & 1.00 & 0.96 & 0.96 \\
4469 & 100634 & 1991 & 431.46 & 0.34 & 43146.00 & 425561.57 & 1.00 & 0.99 & 0.99 \\
9174 & 101116 & 1991 & 620.40 & 0.21 & 34411.00 & 315438.67 & 1.80 & 0.51 & 0.92 \\
14123 & 101805 & 1991 & 783.21 & 0.73 & 78321.00 & 757782.75 & 1.00 & 0.97 & 0.97 \\
16596 & 102159 & 1991 & 13.00 & 0.09 & 1267.00 & 13179.95 & 1.03 & 1.01 & 1.04 \\
14188 & 101820 & 1991 & 9.20 & 0.11 & 737.00 & 8064.69 & 1.25 & 0.88 & 1.09 \\
4373 & 100614 & 1991 & 147.90 & 0.34 & 14786.00 & 130138.27 & 1.00 & 0.88 & 0.88 \\
18792 & 102522 & 1991 & 228.17 & 0.27 & 22817.00 & 207262.06 & 1.00 & 0.91 & 0.91 \\
14156 & 101819 & 1991 & 104.94 & 0.31 & 10242.00 & 89256.16 & 1.02 & 0.85 & 0.87 \\
4391 & 100622 & 1991 & 25.00 & 0.19 & 2503.00 & 24959.98 & 1.00 & 1.00 & 1.00 \\
26609 & 103593 & 1991 & 8215.39 & 0.24 & 821539.00 & 7575785.42 & 1.00 & 0.92 & 0.92 \\
2325 & 100319 & 1991 & 133.25 & 0.43 & 13325.00 & 118311.61 & 1.00 & 0.89 & 0.89 \\
18766 & 102508 & 1991 & 64.15 & 0.13 & 6421.00 & 60180.75 & 1.00 & 0.94 & 0.94 \\
57 & 100004 & 1991 & 284.59 & 0.41 & 24222.00 & 249043.72 & 1.17 & 0.88 & 1.03 \\
18748 & 102507 & 1991 & 323.48 & 0.68 & 32388.00 & 319247.39 & 1.00 & 0.99 & 0.99 \\
12401 & 101539 & 1991 & 51.05 & 0.10 & 3624.00 & 31813.89 & 1.41 & 0.62 & 0.88 \\
18686 & 102503 & 1991 & 31.83 & 0.53 & 3183.00 & 30410.45 & 1.00 & 0.96 & 0.96 \\
46258 & 200205 & 1991 & 24.90 & 0.02 & 3442.00 & 23316.62 & 0.72 & 0.94 & 0.68 \\
11497 & 101425 & 1991 & 12.75 & 0.27 & 1270.00 & 10750.04 & 1.00 & 0.84 & 0.85 \\
2293 & 100313 & 1991 & 62.06 & 0.39 & 7824.00 & 63625.28 & 0.79 & 1.03 & 0.81 \\
14054 & 101801 & 1991 & 131.59 & 0.08 & 13158.00 & 115064.68 & 1.00 & 0.87 & 0.87 \\
1777 & 100237 & 1991 & 87.35 & 0.10 & 7336.00 & 77090.05 & 1.19 & 0.88 & 1.05 \\
24753 & 103377 & 1991 & 292.24 & 0.36 & 26889.00 & 264135.64 & 1.09 & 0.90 & 0.98 \\
18675 & 102502 & 1991 & 158.42 & 0.44 & 10750.00 & 135809.94 & 1.47 & 0.86 & 1.26 \\
4428 & 100624 & 1991 & 47.80 & 0.28 & 4780.00 & 47976.20 & 1.00 & 1.00 & 1.00 \\
2305 & 100315 & 1991 & 256.80 & 0.26 & 25680.00 & 240800.64 & 1.00 & 0.94 & 0.94 \\
23288 & 103158 & 1991 & 639.42 & 0.28 & 64007.00 & 654082.26 & 1.00 & 1.02 & 1.02 \\
12299 & 101534 & 1991 & 99.19 & 0.27 & 7448.00 & 74517.03 & 1.33 & 0.75 & 1.00 \\
24636 & 103373 & 1991 & 264.21 & 0.46 & 24569.00 & 254152.09 & 1.08 & 0.96 & 1.03 \\
18262 & 102419 & 1991 & 230.20 & 0.29 & 23019.00 & 189304.28 & 1.00 & 0.82 & 0.82 \\
23563 & 103193 & 1991 & 26.98 & 0.22 & 2698.00 & 25010.22 & 1.00 & 0.93 & 0.93 \\
7192 & 101013 & 1991 & 264.40 & 0.43 & 26437.00 & 251205.94 & 1.00 & 0.95 & 0.95 \\
16868 & 102213 & 1991 & 91.46 & 0.62 & 9146.00 & 82205.06 & 1.00 & 0.90 & 0.90 \\
12212 & 101523 & 1991 & 13.20 & 0.09 & 1556.00 & 11440.35 & 0.85 & 0.87 & 0.74 \\
18218 & 102417 & 1991 & 166.70 & 0.72 & 16668.00 & 140666.34 & 1.00 & 0.84 & 0.84 \\
4732 & 100670 & 1991 & 68.70 & 0.10 & 6084.00 & 67218.10 & 1.13 & 0.98 & 1.10 \\
2127 & 100292 & 1991 & 262.28 & 0.33 & 26230.00 & 256065.38 & 1.00 & 0.98 & 0.98 \\
25818 & 103524 & 1991 & 4345.34 & 0.30 & 434533.00 & 4338298.15 & 1.00 & 1.00 & 1.00 \\
11672 & 101456 & 1991 & 35.27 & 0.26 & 3883.00 & 39106.80 & 0.91 & 1.11 & 1.01 \\
38979 & 107387 & 1991 & 6.62 & 0.49 & 662.00 & 6165.21 & 1.00 & 0.93 & 0.93 \\
4753 & 100671 & 1991 & 40.72 & 0.37 & 4072.00 & 38028.69 & 1.00 & 0.93 & 0.93 \\
26366 & 103572 & 1991 & 16.57 & 0.58 & 1655.00 & 14722.52 & 1.00 & 0.89 & 0.89 \\
18171 & 102414 & 1991 & 369.40 & 0.41 & 36941.00 & 343020.05 & 1.00 & 0.93 & 0.93 \\
25885 & 103526 & 1991 & 956.81 & 0.29 & 95680.00 & 860191.11 & 1.00 & 0.90 & 0.90 \\
18160 & 102412 & 1991 & 21.10 & 0.18 & 2108.00 & 18219.30 & 1.00 & 0.86 & 0.86 \\
23751 & 103212 & 1991 & 159.67 & 0.32 & 14630.00 & 139807.39 & 1.09 & 0.88 & 0.96 \\
9536 & 101149 & 1991 & 106.65 & 0.77 & 10665.00 & 90423.67 & 1.00 & 0.85 & 0.85 \\
25784 & 103523 & 1991 & 372.37 & 0.42 & 37237.00 & 354599.79 & 1.00 & 0.95 & 0.95 \\
48171 & 240040 & 1991 & 5.90 & 0.14 & 587.00 & 5353.75 & 1.01 & 0.91 & 0.91 \\
16838 & 102197 & 1991 & 45.62 & 0.27 & 4562.00 & 40048.34 & 1.00 & 0.88 & 0.88 \\
16766 & 102191 & 1991 & 41.30 & 0.07 & 4130.00 & 35737.91 & 1.00 & 0.87 & 0.87 \\
4587 & 100642 & 1991 & 524.12 & 0.31 & 52412.00 & 477723.58 & 1.00 & 0.91 & 0.91 \\
1805 & 100238 & 1991 & 34.77 & 0.21 & 3413.00 & 33173.48 & 1.02 & 0.95 & 0.97 \\
7454 & 101040 & 1991 & 175.00 & 0.36 & 15364.00 & 158313.57 & 1.14 & 0.90 & 1.03 \\
11591 & 101431 & 1991 & 170.03 & 0.22 & 16106.00 & 129300.78 & 1.06 & 0.76 & 0.80 \\
25686 & 103514 & 1991 & 569.83 & 0.25 & 56982.00 & 534741.07 & 1.00 & 0.94 & 0.94 \\
8059 & 101073 & 1991 & 1070.20 & 0.49 & 86421.00 & 795814.63 & 1.24 & 0.74 & 0.92 \\
7156 & 101000 & 1991 & 375.80 & 0.29 & 37580.00 & 329865.63 & 1.00 & 0.88 & 0.88 \\
4617 & 100644 & 1991 & 46.88 & 0.33 & 4688.00 & 43091.41 & 1.00 & 0.92 & 0.92 \\
26475 & 103582 & 1991 & 11.03 & 0.54 & 1043.00 & 11259.80 & 1.06 & 1.02 & 1.08 \\
9574 & 101151 & 1991 & 85.31 & 0.28 & 8531.00 & 81826.61 & 1.00 & 0.96 & 0.96 \\
4624 & 100648 & 1991 & 8.91 & 0.26 & 891.00 & 8563.77 & 1.00 & 0.96 & 0.96 \\
4633 & 100659 & 1991 & 116.07 & 0.44 & 9875.00 & 102838.84 & 1.18 & 0.89 & 1.04 \\
23537 & 103184 & 1991 & 590.43 & 0.34 & 59043.00 & 514667.81 & 1.00 & 0.87 & 0.87 \\
13969 & 101794 & 1991 & 58.68 & 0.54 & 5868.00 & 54916.33 & 1.00 & 0.94 & 0.94 \\
4667 & 100660 & 1991 & 73.46 & 0.27 & 6936.00 & 63169.32 & 1.06 & 0.86 & 0.91 \\
18312 & 102425 & 1991 & 665.70 & 0.44 & 66568.00 & 625082.30 & 1.00 & 0.94 & 0.94 \\
8350 & 101084 & 1991 & 124.80 & 0.59 & 10231.00 & 103917.59 & 1.22 & 0.83 & 1.02 \\
23511 & 103183 & 1991 & 344.97 & 0.46 & 34465.00 & 304524.26 & 1.00 & 0.88 & 0.88 \\
1129 & 100155 & 1991 & 77.76 & 0.42 & 7654.00 & 71137.89 & 1.02 & 0.91 & 0.93 \\
14013 & 101800 & 1991 & 364.95 & 0.42 & 36495.00 & 324696.04 & 1.00 & 0.89 & 0.89 \\
18345 & 102441 & 1991 & 403.10 & 0.39 & 40308.00 & 324446.60 & 1.00 & 0.80 & 0.80 \\
57825 & 401081 & 1991 & 8.06 & 0.14 & 810.00 & 7281.91 & 0.99 & 0.90 & 0.90 \\
24676 & 103375 & 1991 & 64.05 & 0.85 & 5496.00 & 60099.11 & 1.17 & 0.94 & 1.09 \\
15749 & 102017 & 1991 & 1530.29 & 0.28 & 153047.00 & 1291281.22 & 1.00 & 0.84 & 0.84 \\
17301 & 102280 & 1991 & 889.16 & 0.22 & 88916.00 & 765384.99 & 1.00 & 0.86 & 0.86 \\
12767 & 101594 & 1991 & 257.63 & 0.62 & 21586.00 & 207106.81 & 1.19 & 0.80 & 0.96 \\
74580 & 601139 & 1991 & 176.70 & 0.30 & 16859.00 & 138223.48 & 1.05 & 0.78 & 0.82 \\
20378 & 102733 & 1991 & 2819.08 & 0.19 & 251390.00 & 2218193.49 & 1.12 & 0.79 & 0.88 \\
8670 & 101094 & 1991 & 20.70 & 0.46 & 1698.00 & 16126.60 & 1.22 & 0.78 & 0.95 \\
1675 & 100223 & 1991 & 161.10 & 0.31 & 16160.00 & 140164.37 & 1.00 & 0.87 & 0.87 \\
676 & 100090 & 1991 & 12.57 & 0.43 & 1021.00 & 10380.19 & 1.23 & 0.83 & 1.02 \\
2985 & 100395 & 1991 & 508.03 & 0.23 & 48584.00 & 457992.55 & 1.05 & 0.90 & 0.94 \\
21154 & 102835 & 1991 & 43.06 & 0.41 & 4307.00 & 39905.54 & 1.00 & 0.93 & 0.93 \\
20918 & 102802 & 1991 & 23.49 & 0.13 & 2472.00 & 21998.47 & 0.95 & 0.94 & 0.89 \\
15016 & 101943 & 1991 & 44.94 & 0.09 & 4495.00 & 38514.88 & 1.00 & 0.86 & 0.86 \\
10744 & 101322 & 1991 & 23.34 & 0.31 & 2330.00 & 22129.18 & 1.00 & 0.95 & 0.95 \\
47178 & 200342 & 1991 & 123.40 & 0.26 & 12397.00 & 111714.63 & 1.00 & 0.91 & 0.90 \\
2906 & 100379 & 1991 & 166.52 & 0.37 & 16652.00 & 149844.50 & 1.00 & 0.90 & 0.90 \\
6678 & 100910 & 1991 & 33.65 & 0.24 & 3361.00 & 31169.66 & 1.00 & 0.93 & 0.93 \\
20457 & 102749 & 1991 & 21.48 & 0.11 & 1823.00 & 18889.76 & 1.18 & 0.88 & 1.04 \\
698 & 100091 & 1991 & 3.27 & 0.23 & 304.00 & 3028.72 & 1.08 & 0.93 & 1.00 \\
1662 & 100222 & 1991 & 61.16 & 0.31 & 6116.00 & 59423.52 & 1.00 & 0.97 & 0.97 \\
12791 & 101596 & 1991 & 222.29 & 0.62 & 19750.00 & 205150.93 & 1.13 & 0.92 & 1.04 \\
20373 & 102732 & 1991 & 277.90 & 0.24 & 27430.00 & 251572.50 & 1.01 & 0.91 & 0.92 \\
47327 & 210203 & 1991 & 90.92 & 0.55 & 9090.00 & 87838.66 & 1.00 & 0.97 & 0.97 \\
20370 & 102730 & 1991 & 47.08 & 0.10 & 4837.00 & 45500.88 & 0.97 & 0.97 & 0.94 \\
22177 & 102994 & 1991 & 60.20 & 0.26 & 5931.00 & 60230.15 & 1.02 & 1.00 & 1.02 \\
14944 & 101925 & 1991 & 166.06 & 0.27 & 15498.00 & 160369.85 & 1.07 & 0.97 & 1.03 \\
10853 & 101340 & 1991 & 1155.61 & 0.30 & 115561.00 & 962284.58 & 1.00 & 0.83 & 0.83 \\
6803 & 100956 & 1991 & 1.59 & 0.15 & 171.00 & 1469.51 & 0.93 & 0.92 & 0.86 \\
20327 & 102716 & 1991 & 161.06 & 0.26 & 15555.00 & 134704.97 & 1.04 & 0.84 & 0.87 \\
10101 & 101259 & 1991 & 16.07 & 0.15 & 1627.00 & 15664.65 & 0.99 & 0.97 & 0.96 \\
15036 & 101953 & 1991 & 205.78 & 0.25 & 19989.00 & 197086.52 & 1.03 & 0.96 & 0.99 \\
6447 & 100875 & 1991 & 148.51 & 0.31 & 14851.00 & 140109.77 & 1.00 & 0.94 & 0.94 \\
21513 & 102877 & 1991 & 51.70 & 0.16 & 8272.00 & 74675.51 & 0.62 & 1.44 & 0.90 \\
21232 & 102840 & 1991 & 42.76 & 0.26 & 4280.00 & 40993.32 & 1.00 & 0.96 & 0.96 \\
12991 & 101618 & 1991 & 130.23 & 0.25 & 13020.00 & 108767.94 & 1.00 & 0.84 & 0.84 \\
20361 & 102728 & 1991 & 176.81 & 0.12 & 15862.00 & 177524.02 & 1.11 & 1.00 & 1.12 \\
15670 & 102013 & 1991 & 111.28 & 0.24 & 10460.00 & 92840.07 & 1.06 & 0.83 & 0.89 \\
12774 & 101595 & 1991 & 186.29 & 0.59 & 12643.00 & 124477.97 & 1.47 & 0.67 & 0.98 \\
14901 & 101921 & 1991 & 25.97 & 0.19 & 2420.00 & 24354.26 & 1.07 & 0.94 & 1.01 \\
3672 & 100468 & 1991 & 90.08 & 0.09 & 9010.00 & 81001.10 & 1.00 & 0.90 & 0.90 \\
21222 & 102838 & 1991 & 200.68 & 0.23 & 20060.00 & 189627.09 & 1.00 & 0.94 & 0.95 \\
20701 & 102784 & 1991 & 1319.99 & 0.50 & 101654.00 & 1117157.86 & 1.30 & 0.85 & 1.10 \\
21993 & 102984 & 1991 & 110.51 & 0.30 & 9355.00 & 87373.37 & 1.18 & 0.79 & 0.93 \\
20930 & 102812 & 1991 & 42.16 & 0.24 & 4156.00 & 39460.87 & 1.01 & 0.94 & 0.95 \\
1509 & 100209 & 1991 & 345.68 & 0.60 & 34567.00 & 324379.30 & 1.00 & 0.94 & 0.94 \\
12906 & 101606 & 1991 & 923.22 & 0.47 & 92320.00 & 776927.45 & 1.00 & 0.84 & 0.84 \\
14866 & 101919 & 1991 & 159.08 & 0.56 & 13405.00 & 143743.31 & 1.19 & 0.90 & 1.07 \\
15564 & 102005 & 1991 & 741.08 & 0.22 & 73979.00 & 644544.51 & 1.00 & 0.87 & 0.87 \\
8945 & 101106 & 1991 & 43.10 & 0.26 & 3315.00 & 32412.75 & 1.30 & 0.75 & 0.98 \\
22071 & 102989 & 1991 & 441.00 & 0.30 & 35033.00 & 348940.18 & 1.26 & 0.79 & 1.00 \\
22133 & 102993 & 1991 & 711.90 & 0.46 & 55646.00 & 594523.95 & 1.28 & 0.84 & 1.07 \\
20590 & 102774 & 1991 & 314.17 & 0.37 & 28771.00 & 298570.43 & 1.09 & 0.95 & 1.04 \\
6370 & 100856 & 1991 & 155.86 & 0.40 & 15586.00 & 143927.18 & 1.00 & 0.92 & 0.92 \\
15087 & 101956 & 1991 & 92.65 & 0.68 & 6566.00 & 81423.94 & 1.41 & 0.88 & 1.24 \\
21050 & 102825 & 1991 & 83.15 & 0.23 & 7783.00 & 76861.41 & 1.07 & 0.92 & 0.99 \\
22368 & 103008 & 1991 & 62.54 & 0.26 & 6254.00 & 59515.94 & 1.00 & 0.95 & 0.95 \\
20550 & 102767 & 1991 & 756.54 & 0.29 & 73977.00 & 731630.99 & 1.02 & 0.97 & 0.99 \\
25228 & 103463 & 1991 & 118.66 & -0.04 & 12120.00 & 96465.92 & 0.98 & 0.81 & 0.80 \\
21484 & 102873 & 1991 & 167.50 & 0.31 & 16750.00 & 139651.29 & 1.00 & 0.83 & 0.83 \\
10137 & 101263 & 1991 & 290.26 & 0.28 & 29026.00 & 236707.71 & 1.00 & 0.82 & 0.82 \\
3389 & 100430 & 1991 & 47.22 & 0.21 & 4658.00 & 46225.79 & 1.01 & 0.98 & 0.99 \\
7656 & 101054 & 1991 & 1309.70 & 0.24 & 125842.00 & 1023240.43 & 1.04 & 0.78 & 0.81 \\
14698 & 101911 & 1991 & 60.35 & 0.26 & 6035.00 & 57411.26 & 1.00 & 0.95 & 0.95 \\
65076 & 500660 & 1991 & 208.41 & -0.16 & 20840.00 & 208495.25 & 1.00 & 1.00 & 1.00 \\
22099 & 102990 & 1991 & 208.86 & 0.27 & 18788.00 & 184765.07 & 1.11 & 0.88 & 0.98 \\
14989 & 101930 & 1991 & 382.95 & 0.33 & 38295.00 & 317609.88 & 1.00 & 0.83 & 0.83 \\
24967 & 103399 & 1991 & 15.42 & 0.29 & 1541.00 & 15220.31 & 1.00 & 0.99 & 0.99 \\
24965 & 103397 & 1991 & 11.46 & 0.25 & 1146.00 & 11353.06 & 1.00 & 0.99 & 0.99 \\
6782 & 100954 & 1991 & 45.93 & 0.26 & 4776.00 & 43873.29 & 0.96 & 0.96 & 0.92 \\
22404 & 103011 & 1991 & 64.40 & 0.29 & 3988.00 & 39954.38 & 1.61 & 0.62 & 1.00 \\
10514 & 101298 & 1991 & 263.53 & 0.06 & 25719.00 & 227432.68 & 1.02 & 0.86 & 0.88 \\
6712 & 100916 & 1991 & 26.02 & 0.44 & 2427.00 & 20918.17 & 1.07 & 0.80 & 0.86 \\
6355 & 100855 & 1991 & 56.31 & 0.12 & 5631.00 & 53145.45 & 1.00 & 0.94 & 0.94 \\
21899 & 102976 & 1991 & 312.96 & 0.34 & 31277.00 & 300961.10 & 1.00 & 0.96 & 0.96 \\
21100 & 102832 & 1991 & 34.47 & 0.36 & 2773.00 & 25876.89 & 1.24 & 0.75 & 0.93 \\
15520 & 102000 & 1991 & 463.01 & 0.31 & 46301.00 & 443695.67 & 1.00 & 0.96 & 0.96 \\
6529 & 100888 & 1991 & 39.72 & 0.13 & 3611.00 & 35351.49 & 1.10 & 0.89 & 0.98 \\
15021 & 101946 & 1991 & 9.17 & 0.23 & 926.00 & 9185.62 & 0.99 & 1.00 & 0.99 \\
15137 & 101963 & 1991 & 50.35 & 0.41 & 4455.00 & 38044.37 & 1.13 & 0.76 & 0.85 \\
20646 & 102777 & 1991 & 274.08 & 0.41 & 23405.00 & 258858.62 & 1.17 & 0.94 & 1.11 \\
12728 & 101591 & 1991 & 22.68 & 0.20 & 2268.00 & 18570.59 & 1.00 & 0.82 & 0.82 \\
10759 & 101330 & 1991 & 50.55 & 0.64 & 5210.00 & 40522.22 & 0.97 & 0.80 & 0.78 \\
368 & 100044 & 1991 & 13.25 & 0.28 & 1179.00 & 10841.05 & 1.12 & 0.82 & 0.92 \\
14732 & 101912 & 1991 & 177.88 & 0.39 & 17788.00 & 164339.26 & 1.00 & 0.92 & 0.92 \\
22324 & 103007 & 1991 & 683.35 & 0.31 & 68335.00 & 674422.41 & 1.00 & 0.99 & 0.99 \\
47243 & 200344 & 1991 & 986.74 & 0.32 & 98674.00 & 940224.32 & 1.00 & 0.95 & 0.95 \\
20610 & 102775 & 1991 & 421.16 & 0.25 & 39724.00 & 371389.34 & 1.06 & 0.88 & 0.93 \\
6541 & 100890 & 1991 & 188.82 & 0.12 & 17747.00 & 183139.71 & 1.06 & 0.97 & 1.03 \\
21587 & 102895 & 1991 & 75.85 & 0.28 & 7388.00 & 73954.82 & 1.03 & 0.98 & 1.00 \\
8959 & 101107 & 1991 & 900.20 & 0.36 & 77730.00 & 840365.18 & 1.16 & 0.93 & 1.08 \\
10249 & 101276 & 1991 & 98.90 & 0.65 & 9889.00 & 94682.60 & 1.00 & 0.96 & 0.96 \\
20271 & 102702 & 1991 & 160.49 & 0.50 & 12090.00 & 145244.76 & 1.33 & 0.91 & 1.20 \\
20759 & 102789 & 1991 & 111.86 & 0.33 & 6231.00 & 60314.77 & 1.80 & 0.54 & 0.97 \\
25065 & 103429 & 1991 & 405.24 & 0.59 & 40524.00 & 330100.63 & 1.00 & 0.81 & 0.81 \\
21742 & 102949 & 1991 & 1152.16 & 0.43 & 105840.00 & 1029416.31 & 1.09 & 0.89 & 0.97 \\
20274 & 102703 & 1991 & 182.65 & 0.16 & 18822.00 & 187712.22 & 0.97 & 1.03 & 1.00 \\
21403 & 102862 & 1991 & 15.43 & 0.26 & 1885.00 & 16733.42 & 0.82 & 1.08 & 0.89 \\
20293 & 102715 & 1991 & 905.16 & 0.24 & 82042.00 & 716346.74 & 1.10 & 0.79 & 0.87 \\
20838 & 102796 & 1991 & 2.56 & 0.38 & 256.00 & 2255.28 & 1.00 & 0.88 & 0.88 \\
22208 & 102996 & 1991 & 172.68 & 0.55 & 17268.00 & 149337.22 & 1.00 & 0.86 & 0.86 \\
1587 & 100217 & 1991 & 57.23 & 0.13 & 5571.00 & 54189.31 & 1.03 & 0.95 & 0.97 \\
6322 & 100849 & 1991 & 77.33 & 0.57 & 7733.00 & 70249.72 & 1.00 & 0.91 & 0.91 \\
7728 & 101056 & 1991 & 3355.50 & 0.42 & 268649.00 & 2607859.47 & 1.25 & 0.78 & 0.97 \\
622 & 100085 & 1991 & 152.15 & 0.43 & 15215.00 & 122426.97 & 1.00 & 0.80 & 0.80 \\
21523 & 102880 & 1991 & 20.30 & 0.27 & 2824.00 & 29145.29 & 0.72 & 1.44 & 1.03 \\
8811 & 101100 & 1991 & 28.60 & -0.23 & 3771.00 & 28968.97 & 0.76 & 1.01 & 0.77 \\
3715 & 100475 & 1991 & 51.20 & 0.22 & 5120.00 & 49333.84 & 1.00 & 0.96 & 0.96 \\
10068 & 101258 & 1991 & 60.71 & 0.31 & 5926.00 & 47533.79 & 1.02 & 0.78 & 0.80 \\
3710 & 100471 & 1991 & 7.71 & 0.25 & 770.00 & 6882.25 & 1.00 & 0.89 & 0.89 \\
10453 & 101286 & 1991 & 421.82 & 0.27 & 42193.00 & 387440.30 & 1.00 & 0.92 & 0.92 \\
10064 & 101257 & 1991 & 3.49 & 0.20 & 385.00 & 3070.50 & 0.91 & 0.88 & 0.80 \\
734 & 100093 & 1991 & 80.89 & 0.55 & 6127.00 & 67682.91 & 1.32 & 0.84 & 1.10 \\
6739 & 100947 & 1991 & 826.00 & 0.40 & 82510.00 & 744506.79 & 1.00 & 0.90 & 0.90 \\
12873 & 101603 & 1991 & 449.28 & 0.32 & 44930.00 & 372194.15 & 1.00 & 0.83 & 0.83 \\
3015 & 100398 & 1991 & 41.23 & 0.29 & 3876.00 & 37291.81 & 1.06 & 0.90 & 0.96 \\
6628 & 100906 & 1991 & 421.26 & 0.25 & 42126.00 & 385808.49 & 1.00 & 0.92 & 0.92 \\
20797 & 102793 & 1991 & 10.06 & 0.51 & 648.00 & 6159.27 & 1.55 & 0.61 & 0.95 \\
21385 & 102861 & 1991 & 56.65 & 0.42 & 4746.00 & 49443.40 & 1.19 & 0.87 & 1.04 \\
3487 & 100441 & 1991 & 494.87 & 0.30 & 49486.00 & 403493.60 & 1.00 & 0.82 & 0.82 \\
25190 & 103460 & 1991 & 271.44 & 0.72 & 20808.00 & 236368.17 & 1.30 & 0.87 & 1.14 \\
6532 & 100889 & 1991 & 29.31 & 0.09 & 2709.00 & 23904.47 & 1.08 & 0.82 & 0.88 \\
110 & 100009 & 1991 & 27.48 & 0.77 & 1967.00 & 26890.70 & 1.40 & 0.98 & 1.37 \\
6496 & 100878 & 1991 & 789.88 & 0.28 & 74150.00 & 732236.27 & 1.07 & 0.93 & 0.99 \\
22236 & 102997 & 1991 & 50.17 & 0.55 & 5017.00 & 48508.32 & 1.00 & 0.97 & 0.97 \\
648 & 100087 & 1991 & 3293.23 & 0.47 & 330293.00 & 3074402.23 & 1.00 & 0.93 & 0.93 \\
21948 & 102981 & 1991 & 69.16 & 0.61 & 6916.00 & 58893.13 & 1.00 & 0.85 & 0.85 \\
15715 & 102016 & 1991 & 1075.34 & 0.48 & 94974.00 & 989641.67 & 1.13 & 0.92 & 1.04 \\
382 & 100046 & 1991 & 44.23 & 0.37 & 4446.00 & 41076.58 & 0.99 & 0.93 & 0.92 \\
6826 & 100962 & 1991 & 429.71 & 0.56 & 42971.00 & 384793.16 & 1.00 & 0.90 & 0.90 \\
25272 & 103464 & 1991 & 365.82 & 0.28 & 31885.00 & 334108.78 & 1.15 & 0.91 & 1.05 \\
21321 & 102852 & 1991 & 141.80 & 0.59 & 14145.00 & 129938.79 & 1.00 & 0.92 & 0.92 \\
22560 & 103021 & 1991 & 24.26 & 0.40 & 2426.00 & 22199.54 & 1.00 & 0.91 & 0.92 \\
21241 & 102842 & 1991 & 6.11 & 0.23 & 611.00 & 5537.15 & 1.00 & 0.91 & 0.91 \\
1527 & 100213 & 1991 & 145.80 & 0.03 & 13082.00 & 128352.87 & 1.11 & 0.88 & 0.98 \\
15166 & 101964 & 1991 & 31.44 & 0.42 & 3144.00 & 31380.83 & 1.00 & 1.00 & 1.00 \\
6661 & 100908 & 1991 & 33.22 & 0.59 & 3318.00 & 26955.46 & 1.00 & 0.81 & 0.81 \\
21981 & 102983 & 1991 & 331.01 & 0.35 & 24844.00 & 207827.74 & 1.33 & 0.63 & 0.84 \\
7694 & 101055 & 1991 & 1376.40 & 0.48 & 114236.00 & 1106659.18 & 1.20 & 0.80 & 0.97 \\
63169 & 500486 & 1991 & 110.89 & 0.07 & 12862.00 & 113542.02 & 0.86 & 1.02 & 0.88 \\
1471 & 100207 & 1991 & 2591.68 & 0.32 & 259168.00 & 2489224.12 & 1.00 & 0.96 & 0.96 \\
587 & 100079 & 1991 & 517.82 & 0.44 & 51782.00 & 506292.31 & 1.00 & 0.98 & 0.98 \\
14189 & 101820 & 1992 & 10.98 & 0.08 & 918.00 & 8508.09 & 1.20 & 0.77 & 0.93 \\
6740 & 100947 & 1992 & 782.25 & 0.08 & 78224.00 & 714548.90 & 1.00 & 0.91 & 0.91 \\
4374 & 100614 & 1992 & 191.50 & 0.11 & 19145.00 & 176644.29 & 1.00 & 0.92 & 0.92 \\
25621 & 103498 & 1992 & 101.09 & -0.03 & 10428.00 & 93526.84 & 0.97 & 0.93 & 0.90 \\
9141 & 101115 & 1992 & 1564.40 & 0.07 & 145468.00 & 1440342.76 & 1.08 & 0.92 & 0.99 \\
677 & 100090 & 1992 & 23.23 & 0.11 & 1938.00 & 20369.01 & 1.20 & 0.88 & 1.05 \\
12318 & 101536 & 1992 & 171.74 & 0.09 & 13162.00 & 135165.14 & 1.30 & 0.79 & 1.03 \\
2821 & 100362 & 1992 & 31.31 & 0.21 & 3371.00 & 31841.46 & 0.93 & 1.02 & 0.94 \\
6629 & 100906 & 1992 & 497.06 & 0.21 & 49454.00 & 457394.51 & 1.01 & 0.92 & 0.92 \\
4520 & 100637 & 1992 & 470.66 & 0.14 & 47066.00 & 433302.76 & 1.00 & 0.92 & 0.92 \\
8389 & 101085 & 1992 & 247.40 & 0.11 & 24904.00 & 227858.00 & 0.99 & 0.92 & 0.91 \\
3454 & 100439 & 1992 & 69.18 & 0.28 & 7975.00 & 79784.50 & 0.87 & 1.15 & 1.00 \\
22134 & 102993 & 1992 & 587.89 & 0.06 & 58189.00 & 518361.29 & 1.01 & 0.88 & 0.89 \\
18749 & 102507 & 1992 & 625.62 & 0.02 & 62644.00 & 605276.11 & 1.00 & 0.97 & 0.97 \\
8751 & 101097 & 1992 & 111.40 & 0.04 & 8679.00 & 86980.60 & 1.28 & 0.78 & 1.00 \\
13345 & 101729 & 1992 & 192.41 & 0.05 & 19240.00 & 199054.37 & 1.00 & 1.03 & 1.03 \\
21900 & 102976 & 1992 & 336.47 & 0.05 & 33647.00 & 311754.11 & 1.00 & 0.93 & 0.93 \\
25678 & 103510 & 1992 & 29.09 & 0.07 & 2838.00 & 28279.62 & 1.02 & 0.97 & 1.00 \\
25273 & 103464 & 1992 & 503.38 & 0.10 & 43212.00 & 441612.31 & 1.16 & 0.88 & 1.02 \\
21485 & 102873 & 1992 & 178.13 & 0.16 & 17720.00 & 149611.17 & 1.01 & 0.84 & 0.84 \\
12796 & 101600 & 1992 & 121.28 & 0.07 & 10558.00 & 108094.11 & 1.15 & 0.89 & 1.02 \\
18532 & 102471 & 1992 & 111.22 & 0.06 & 9889.00 & 105659.65 & 1.12 & 0.95 & 1.07 \\
12768 & 101594 & 1992 & 325.40 & 0.00 & 32540.00 & 261704.37 & 1.00 & 0.80 & 0.80 \\
24717 & 103376 & 1992 & 4853.43 & -0.04 & 447188.00 & 4469647.48 & 1.09 & 0.92 & 1.00 \\
9743 & 101186 & 1992 & 98.53 & 0.07 & 8740.00 & 90776.84 & 1.13 & 0.92 & 1.04 \\
5317 & 100753 & 1992 & 1509.32 & 0.09 & 150930.00 & 1281233.10 & 1.00 & 0.85 & 0.85 \\
8960 & 101107 & 1992 & 1115.00 & 0.16 & 111526.00 & 1033610.05 & 1.00 & 0.93 & 0.93 \\
23274 & 103154 & 1992 & 13.71 & 0.05 & 1371.00 & 12736.57 & 1.00 & 0.93 & 0.93 \\
2306 & 100315 & 1992 & 244.57 & 0.09 & 24457.00 & 236330.15 & 1.00 & 0.97 & 0.97 \\
2239 & 100299 & 1992 & 20.48 & 0.21 & 1571.00 & 16762.80 & 1.30 & 0.82 & 1.07 \\
26610 & 103593 & 1992 & 10100.37 & 0.09 & 1010037.00 & 9376809.87 & 1.00 & 0.93 & 0.93 \\
18687 & 102503 & 1992 & 56.24 & 0.37 & 5624.00 & 51526.10 & 1.00 & 0.92 & 0.92 \\
7075 & 100996 & 1992 & 320.85 & 0.01 & 27503.00 & 274528.99 & 1.17 & 0.86 & 1.00 \\
18518 & 102470 & 1992 & 932.27 & 0.04 & 76850.00 & 782787.42 & 1.21 & 0.84 & 1.02 \\
23242 & 103152 & 1992 & 699.84 & 0.01 & 69613.00 & 653521.02 & 1.01 & 0.93 & 0.94 \\
22100 & 102990 & 1992 & 232.56 & 0.07 & 23260.00 & 215714.58 & 1.00 & 0.93 & 0.93 \\
5596 & 100773 & 1992 & 672.00 & 0.23 & 67170.00 & 575114.58 & 1.00 & 0.86 & 0.86 \\
3005 & 100397 & 1992 & 2.90 & 0.11 & 267.00 & 2677.61 & 1.09 & 0.92 & 1.00 \\
4430 & 100625 & 1992 & 71.10 & 0.03 & 7104.00 & 68977.83 & 1.00 & 0.97 & 0.97 \\
17669 & 102342 & 1992 & 42.60 & -0.01 & 4249.00 & 42060.03 & 1.00 & 0.99 & 0.99 \\
23289 & 103158 & 1992 & 700.47 & 0.06 & 70046.00 & 705148.68 & 1.00 & 1.01 & 1.01 \\
4429 & 100624 & 1992 & 51.90 & 0.05 & 5192.00 & 50985.09 & 1.00 & 0.98 & 0.98 \\
12402 & 101539 & 1992 & 77.36 & 0.09 & 6209.00 & 61280.11 & 1.25 & 0.79 & 0.99 \\
6371 & 100856 & 1992 & 154.19 & 0.05 & 15419.00 & 145328.65 & 1.00 & 0.94 & 0.94 \\
10633 & 101302 & 1992 & 197.81 & 0.10 & 19669.00 & 160941.72 & 1.01 & 0.81 & 0.82 \\
17575 & 102319 & 1992 & 742.25 & 0.09 & 74225.00 & 676772.53 & 1.00 & 0.91 & 0.91 \\
13796 & 101764 & 1992 & 199.55 & -0.01 & 19955.00 & 174360.95 & 1.00 & 0.87 & 0.87 \\
18546 & 102482 & 1992 & 58.44 & 0.39 & 4948.00 & 51103.62 & 1.18 & 0.87 & 1.03 \\
12792 & 101596 & 1992 & 213.83 & 0.00 & 21353.00 & 183746.82 & 1.00 & 0.86 & 0.86 \\
4470 & 100634 & 1992 & 518.37 & 0.12 & 51836.00 & 479877.77 & 1.00 & 0.93 & 0.93 \\
1177 & 100159 & 1992 & 46.24 & 0.18 & 3880.00 & 41033.12 & 1.19 & 0.89 & 1.06 \\
18767 & 102508 & 1992 & 195.07 & 0.04 & 19536.00 & 180676.60 & 1.00 & 0.93 & 0.92 \\
49180 & 240243 & 1992 & 161.43 & -0.09 & 16143.00 & 158525.09 & 1.00 & 0.98 & 0.98 \\
6530 & 100888 & 1992 & 37.10 & -0.06 & 4107.00 & 42516.02 & 0.90 & 1.15 & 1.04 \\
2326 & 100319 & 1992 & 184.34 & 0.25 & 18434.00 & 165815.17 & 1.00 & 0.90 & 0.90 \\
8731 & 101096 & 1992 & 5.70 & 0.02 & 503.00 & 4380.77 & 1.13 & 0.77 & 0.87 \\
4392 & 100622 & 1992 & 22.81 & 0.21 & 2290.00 & 22772.67 & 1.00 & 1.00 & 0.99 \\
12436 & 101541 & 1992 & 71.06 & 0.10 & 6479.00 & 59936.47 & 1.10 & 0.84 & 0.93 \\
9453 & 101137 & 1992 & 3.45 & 0.02 & 345.00 & 2826.34 & 1.00 & 0.82 & 0.82 \\
14124 & 101805 & 1992 & 930.36 & 0.10 & 93036.00 & 864985.05 & 1.00 & 0.93 & 0.93 \\
16619 & 102166 & 1992 & 207.00 & 0.10 & 17659.00 & 173227.39 & 1.17 & 0.84 & 0.98 \\
1778 & 100237 & 1992 & 87.11 & -0.03 & 9089.00 & 94115.99 & 0.96 & 1.08 & 1.04 \\
4990 & 100698 & 1992 & 13.53 & 0.05 & 1355.00 & 11957.40 & 1.00 & 0.88 & 0.88 \\
58 & 100004 & 1992 & 491.40 & 0.05 & 42450.00 & 385092.11 & 1.16 & 0.78 & 0.91 \\
65054 & 500659 & 1992 & 23.46 & 0.01 & 2350.00 & 20361.37 & 1.00 & 0.87 & 0.87 \\
6533 & 100889 & 1992 & 33.88 & 0.11 & 3611.00 & 30335.50 & 0.94 & 0.90 & 0.84 \\
13816 & 101767 & 1992 & 51.57 & -0.01 & 5157.00 & 42876.55 & 1.00 & 0.83 & 0.83 \\
11498 & 101425 & 1992 & 12.93 & 0.06 & 1290.00 & 11994.16 & 1.00 & 0.93 & 0.93 \\
17612 & 102321 & 1992 & 190.23 & 0.17 & 19023.00 & 172454.68 & 1.00 & 0.91 & 0.91 \\
11426 & 101402 & 1992 & 44.70 & 0.07 & 4360.00 & 43978.20 & 1.03 & 0.98 & 1.01 \\
24754 & 103377 & 1992 & 424.92 & 0.10 & 36865.00 & 369955.86 & 1.15 & 0.87 & 1.00 \\
17648 & 102334 & 1992 & 13.95 & 0.02 & 1000.00 & 8110.63 & 1.39 & 0.58 & 0.81 \\
59025 & 410401 & 1992 & 223.43 & -0.06 & 24276.00 & 190747.05 & 0.92 & 0.85 & 0.79 \\
14107 & 101804 & 1992 & 288.70 & 0.05 & 28870.00 & 269930.56 & 1.00 & 0.93 & 0.93 \\
22178 & 102994 & 1992 & 57.51 & 0.05 & 5751.00 & 52084.08 & 1.00 & 0.91 & 0.91 \\
20940 & 102813 & 1992 & 66.30 & 0.05 & 5203.00 & 54721.53 & 1.27 & 0.83 & 1.05 \\
8002 & 101069 & 1992 & 283.50 & 0.01 & 25572.00 & 253229.42 & 1.11 & 0.89 & 0.99 \\
26545 & 103591 & 1992 & 16.46 & 0.05 & 1646.00 & 15336.83 & 1.00 & 0.93 & 0.93 \\
6356 & 100855 & 1992 & 47.71 & 0.01 & 4771.00 & 43049.73 & 1.00 & 0.90 & 0.90 \\
11396 & 101400 & 1992 & 45.10 & 0.12 & 3973.00 & 43103.03 & 1.14 & 0.96 & 1.08 \\
57772 & 401035 & 1992 & 3.68 & 0.00 & 400.00 & 3232.19 & 0.92 & 0.88 & 0.81 \\
8421 & 101086 & 1992 & 70.90 & 0.04 & 4470.00 & 43042.17 & 1.59 & 0.61 & 0.96 \\
46259 & 200205 & 1992 & 22.55 & 0.03 & 2197.00 & 17616.58 & 1.03 & 0.78 & 0.80 \\
12351 & 101537 & 1992 & 148.93 & 0.07 & 12517.00 & 113321.48 & 1.19 & 0.76 & 0.91 \\
16597 & 102159 & 1992 & 5.16 & -0.08 & 615.00 & 6006.14 & 0.84 & 1.16 & 0.98 \\
11368 & 101399 & 1992 & 23.40 & 0.08 & 1805.00 & 17778.50 & 1.30 & 0.76 & 0.98 \\
57925 & 410003 & 1992 & 280.14 & 0.03 & 28014.00 & 277742.60 & 1.00 & 0.99 & 0.99 \\
111 & 100009 & 1992 & 35.21 & 0.14 & 3521.00 & 32085.75 & 1.00 & 0.91 & 0.91 \\
20919 & 102802 & 1992 & 30.70 & 0.09 & 2898.00 & 29198.84 & 1.06 & 0.95 & 1.01 \\
14157 & 101819 & 1992 & 108.20 & 0.14 & 10168.00 & 96093.06 & 1.06 & 0.89 & 0.95 \\
5640 & 100784 & 1992 & 335.92 & 0.11 & 33592.00 & 330864.27 & 1.00 & 0.98 & 0.98 \\
516 & 100072 & 1992 & 1298.65 & 0.15 & 130242.00 & 1179705.40 & 1.00 & 0.91 & 0.91 \\
18793 & 102522 & 1992 & 233.87 & 0.09 & 23391.00 & 215185.50 & 1.00 & 0.92 & 0.92 \\
17154 & 102261 & 1992 & 308.39 & 0.01 & 30796.00 & 286586.70 & 1.00 & 0.93 & 0.93 \\
5632 & 100780 & 1992 & 36.19 & 0.01 & 3619.00 & 32037.69 & 1.00 & 0.89 & 0.89 \\
20931 & 102812 & 1992 & 37.05 & 0.10 & 3720.00 & 35751.96 & 1.00 & 0.97 & 0.96 \\
22209 & 102996 & 1992 & 198.06 & 0.12 & 17630.00 & 182402.88 & 1.12 & 0.92 & 1.03 \\
11535 & 101427 & 1992 & 71.83 & -0.21 & 7183.00 & 59514.26 & 1.00 & 0.83 & 0.83 \\
8946 & 101106 & 1992 & 28.10 & 0.10 & 3090.00 & 31232.83 & 0.91 & 1.11 & 1.01 \\
57894 & 401372 & 1992 & 8.19 & 0.08 & 820.00 & 7845.30 & 1.00 & 0.96 & 0.96 \\
4754 & 100671 & 1992 & 45.60 & 0.12 & 4560.00 & 39898.42 & 1.00 & 0.87 & 0.87 \\
4901 & 100692 & 1992 & 117.54 & 0.12 & 10044.00 & 106742.46 & 1.17 & 0.91 & 1.06 \\
12207 & 101519 & 1992 & 194.95 & 0.01 & 19495.00 & 176752.26 & 1.00 & 0.91 & 0.91 \\
18172 & 102414 & 1992 & 401.24 & 0.10 & 40206.00 & 388220.02 & 1.00 & 0.97 & 0.97 \\
6662 & 100908 & 1992 & 72.50 & 0.06 & 7209.00 & 62986.86 & 1.01 & 0.87 & 0.87 \\
23716 & 103209 & 1992 & 193.28 & 0.02 & 17275.00 & 169601.86 & 1.12 & 0.88 & 0.98 \\
9488 & 101140 & 1992 & 449.03 & 0.03 & 44903.00 & 395630.63 & 1.00 & 0.88 & 0.88 \\
25886 & 103526 & 1992 & 1116.33 & 0.10 & 111633.00 & 1031333.03 & 1.00 & 0.92 & 0.92 \\
18161 & 102412 & 1992 & 20.23 & 0.06 & 2031.00 & 17463.53 & 1.00 & 0.86 & 0.86 \\
1472 & 100207 & 1992 & 2461.66 & 0.04 & 243172.00 & 2260272.77 & 1.01 & 0.92 & 0.93 \\
15228 & 101970 & 1992 & 51.70 & 0.02 & 5170.00 & 43977.72 & 1.00 & 0.85 & 0.85 \\
63170 & 500486 & 1992 & 102.19 & 0.07 & 10970.00 & 95064.07 & 0.93 & 0.93 & 0.87 \\
23752 & 103212 & 1992 & 193.21 & 0.17 & 15415.00 & 151126.60 & 1.25 & 0.78 & 0.98 \\
255 & 100022 & 1992 & 34.39 & 0.07 & 3124.00 & 31549.50 & 1.10 & 0.92 & 1.01 \\
8310 & 101082 & 1992 & 755.30 & -0.17 & 77103.00 & 676181.64 & 0.98 & 0.90 & 0.88 \\
21242 & 102842 & 1992 & 7.33 & 0.13 & 733.00 & 6595.94 & 1.00 & 0.90 & 0.90 \\
17032 & 102231 & 1992 & 614.86 & 0.11 & 56514.00 & 583477.52 & 1.09 & 0.95 & 1.03 \\
21982 & 102983 & 1992 & 642.12 & 0.08 & 64210.00 & 537959.36 & 1.00 & 0.84 & 0.84 \\
26367 & 103572 & 1992 & 25.10 & 0.06 & 2509.00 & 23193.53 & 1.00 & 0.92 & 0.92 \\
23564 & 103193 & 1992 & 25.07 & 0.07 & 2487.00 & 24598.63 & 1.01 & 0.98 & 0.99 \\
25819 & 103524 & 1992 & 6547.29 & 0.07 & 654729.00 & 6526821.46 & 1.00 & 1.00 & 1.00 \\
1065 & 100150 & 1992 & 8.04 & 0.02 & 786.00 & 7103.76 & 1.02 & 0.88 & 0.90 \\
12117 & 101511 & 1992 & 202.60 & -0.06 & 20262.00 & 172859.72 & 1.00 & 0.85 & 0.85 \\
15244 & 101972 & 1992 & 52.90 & 0.03 & 5290.00 & 44538.77 & 1.00 & 0.84 & 0.84 \\
9537 & 101149 & 1992 & 256.48 & 0.19 & 25647.00 & 243669.29 & 1.00 & 0.95 & 0.95 \\
14902 & 101921 & 1992 & 28.41 & 0.12 & 2885.00 & 25573.18 & 0.98 & 0.90 & 0.89 \\
16869 & 102213 & 1992 & 120.85 & 0.01 & 12085.00 & 114477.79 & 1.00 & 0.95 & 0.95 \\
927 & 100112 & 1992 & 460.17 & 0.04 & 42650.00 & 408075.65 & 1.08 & 0.89 & 0.96 \\
21994 & 102984 & 1992 & 209.45 & 0.10 & 20940.00 & 194936.85 & 1.00 & 0.93 & 0.93 \\
18219 & 102417 & 1992 & 427.06 & 0.17 & 41925.00 & 343491.35 & 1.02 & 0.80 & 0.82 \\
2128 & 100292 & 1992 & 254.70 & 0.12 & 23615.00 & 260664.51 & 1.08 & 1.02 & 1.10 \\
4733 & 100670 & 1992 & 63.90 & -0.01 & 6084.00 & 63681.83 & 1.05 & 1.00 & 1.05 \\
21233 & 102840 & 1992 & 43.57 & 0.08 & 4357.00 & 40250.70 & 1.00 & 0.92 & 0.92 \\
47328 & 210203 & 1992 & 105.00 & 0.10 & 10500.00 & 97828.33 & 1.00 & 0.93 & 0.93 \\
21235 & 102841 & 1992 & 37.76 & 0.03 & 3776.00 & 31915.88 & 1.00 & 0.85 & 0.85 \\
25852 & 103525 & 1992 & 1524.67 & 0.01 & 152467.00 & 1521881.99 & 1.00 & 1.00 & 1.00 \\
6448 & 100875 & 1992 & 163.17 & 0.19 & 16317.00 & 158541.90 & 1.00 & 0.97 & 0.97 \\
17053 & 102234 & 1992 & 271.38 & 0.10 & 26029.00 & 226709.08 & 1.04 & 0.84 & 0.87 \\
11836 & 101463 & 1992 & 115.46 & 0.01 & 13289.00 & 135582.37 & 0.87 & 1.17 & 1.02 \\
11673 & 101456 & 1992 & 37.16 & 0.18 & 3345.00 & 34717.61 & 1.11 & 0.93 & 1.04 \\
7422 & 101039 & 1992 & 567.90 & 0.15 & 47940.00 & 492309.85 & 1.18 & 0.87 & 1.03 \\
2088 & 100290 & 1992 & 243.80 & 0.11 & 23901.00 & 225900.60 & 1.02 & 0.93 & 0.95 \\
25915 & 103529 & 1992 & 415.53 & 0.07 & 41553.00 & 405265.64 & 1.00 & 0.98 & 0.98 \\
26284 & 103564 & 1992 & 55.54 & 0.13 & 3983.00 & 34591.64 & 1.39 & 0.62 & 0.87 \\
18000 & 102386 & 1992 & 65.00 & 0.14 & 6496.00 & 63233.78 & 1.00 & 0.97 & 0.97 \\
6480 & 100876 & 1992 & 6.16 & 0.05 & 591.00 & 5287.84 & 1.04 & 0.86 & 0.89 \\
4881 & 100691 & 1992 & 122.13 & 0.08 & 11923.00 & 120701.02 & 1.02 & 0.99 & 1.01 \\
21322 & 102852 & 1992 & 277.54 & -0.03 & 39980.00 & 338567.75 & 0.69 & 1.22 & 0.85 \\
23887 & 103228 & 1992 & 27.68 & 0.12 & 2652.00 & 25113.99 & 1.04 & 0.91 & 0.95 \\
6497 & 100878 & 1992 & 872.78 & 0.09 & 80400.00 & 801482.57 & 1.09 & 0.92 & 1.00 \\
17986 & 102383 & 1992 & 7.00 & 0.08 & 699.00 & 6042.53 & 1.00 & 0.86 & 0.86 \\
21949 & 102981 & 1992 & 68.45 & 0.08 & 6845.00 & 62271.63 & 1.00 & 0.91 & 0.91 \\
25066 & 103429 & 1992 & 428.98 & -0.04 & 42897.00 & 346101.22 & 1.00 & 0.81 & 0.81 \\
17960 & 102377 & 1992 & 53.70 & 0.14 & 3351.00 & 33372.85 & 1.60 & 0.62 & 1.00 \\
11740 & 101460 & 1992 & 1038.51 & 0.19 & 99228.00 & 884349.71 & 1.05 & 0.85 & 0.89 \\
8269 & 101081 & 1992 & 132.70 & 0.43 & 10032.00 & 102882.46 & 1.32 & 0.78 & 1.03 \\
16960 & 102224 & 1992 & 264.21 & 0.18 & 26421.00 & 251987.14 & 1.00 & 0.95 & 0.95 \\
15167 & 101964 & 1992 & 34.36 & 0.08 & 3436.00 & 33376.49 & 1.00 & 0.97 & 0.97 \\
5394 & 100760 & 1992 & 1173.69 & 0.02 & 117370.00 & 1028247.09 & 1.00 & 0.88 & 0.88 \\
11773 & 101461 & 1992 & 600.36 & 0.23 & 54248.00 & 559229.76 & 1.11 & 0.93 & 1.03 \\
21386 & 102861 & 1992 & 60.85 & 0.06 & 5092.00 & 50677.77 & 1.20 & 0.83 & 1.00 \\
12151 & 101513 & 1992 & 66.90 & 0.07 & 6695.00 & 64987.89 & 1.00 & 0.97 & 0.97 \\
23923 & 103242 & 1992 & 7.66 & 0.02 & 766.00 & 7402.80 & 1.00 & 0.97 & 0.97 \\
921 & 100111 & 1992 & 78.81 & 0.05 & 7200.00 & 71651.58 & 1.09 & 0.91 & 1.00 \\
4833 & 100685 & 1992 & 1.59 & -0.04 & 109.00 & 1039.71 & 1.46 & 0.65 & 0.95 \\
11706 & 101457 & 1992 & 111.08 & 0.08 & 11103.00 & 98823.05 & 1.00 & 0.89 & 0.89 \\
13845 & 101776 & 1992 & 36.81 & 0.01 & 3681.00 & 35258.84 & 1.00 & 0.96 & 0.96 \\
1588 & 100217 & 1992 & 57.49 & 0.13 & 5550.00 & 53913.36 & 1.04 & 0.94 & 0.97 \\
23788 & 103213 & 1992 & 587.87 & 0.01 & 60745.00 & 597300.09 & 0.97 & 1.02 & 0.98 \\
15209 & 101968 & 1992 & 56.00 & 0.03 & 5600.00 & 48319.85 & 1.00 & 0.86 & 0.86 \\
4800 & 100682 & 1992 & 36.00 & 0.18 & 3690.00 & 32398.51 & 0.98 & 0.90 & 0.88 \\
26334 & 103570 & 1992 & 9.94 & 0.11 & 994.00 & 9432.18 & 1.00 & 0.95 & 0.95 \\
21404 & 102862 & 1992 & 10.15 & 0.06 & 1287.00 & 10080.23 & 0.79 & 0.99 & 0.78 \\
13876 & 101785 & 1992 & 594.55 & -0.01 & 59455.00 & 534360.21 & 1.00 & 0.90 & 0.90 \\
26302 & 103567 & 1992 & 190.15 & 0.15 & 14951.00 & 125774.19 & 1.27 & 0.66 & 0.84 \\
45311 & 200022 & 1992 & 111.78 & 0.08 & 11180.00 & 102858.10 & 1.00 & 0.92 & 0.92 \\
21223 & 102838 & 1992 & 220.82 & 0.04 & 22082.00 & 199713.93 & 1.00 & 0.90 & 0.90 \\
13475 & 101741 & 1992 & 310.98 & 0.03 & 31095.00 & 261047.21 & 1.00 & 0.84 & 0.84 \\
18073 & 102392 & 1992 & 48.22 & 0.03 & 4988.00 & 44287.74 & 0.97 & 0.92 & 0.89 \\
18070 & 102390 & 1992 & 5.41 & 0.18 & 542.00 & 5214.76 & 1.00 & 0.96 & 0.96 \\
23851 & 103224 & 1992 & 30.62 & 0.20 & 2844.00 & 24705.91 & 1.08 & 0.81 & 0.87 \\
10454 & 101286 & 1992 & 358.72 & 0.04 & 35984.00 & 349370.80 & 1.00 & 0.97 & 0.97 \\
21315 & 102848 & 1992 & 34.70 & 0.11 & 2546.00 & 24413.14 & 1.36 & 0.70 & 0.96 \\
18035 & 102387 & 1992 & 10.20 & 0.14 & 1021.00 & 9075.07 & 1.00 & 0.89 & 0.89 \\
23869 & 103226 & 1992 & 83.24 & 0.08 & 8309.00 & 70328.50 & 1.00 & 0.84 & 0.85 \\
649 & 100087 & 1992 & 5608.77 & -0.01 & 562829.00 & 5337707.70 & 1.00 & 0.95 & 0.95 \\
21294 & 102846 & 1992 & 32.87 & 0.01 & 3287.00 & 29753.66 & 1.00 & 0.91 & 0.91 \\
22005 & 102985 & 1992 & 183.79 & 0.17 & 18380.00 & 173807.31 & 1.00 & 0.95 & 0.95 \\
13832 & 101769 & 1992 & 1660.38 & -0.02 & 166038.00 & 1607366.41 & 1.00 & 0.97 & 0.97 \\
1806 & 100238 & 1992 & 35.50 & 0.19 & 3274.00 & 34882.22 & 1.08 & 0.98 & 1.07 \\
18391 & 102447 & 1992 & 186.84 & 0.14 & 22016.00 & 190473.48 & 0.85 & 1.02 & 0.87 \\
21051 & 102825 & 1992 & 92.65 & 0.16 & 9270.00 & 86805.96 & 1.00 & 0.94 & 0.94 \\
14867 & 101919 & 1992 & 149.96 & -0.03 & 15232.00 & 123052.61 & 0.98 & 0.82 & 0.81 \\
47244 & 200344 & 1992 & 960.27 & 0.07 & 95041.00 & 870243.15 & 1.01 & 0.91 & 0.92 \\
57905 & 402003 & 1992 & 24.50 & 0.05 & 2450.00 & 20015.69 & 1.00 & 0.82 & 0.82 \\
5720 & 100790 & 1992 & 14.88 & 0.11 & 1656.00 & 17161.09 & 0.90 & 1.15 & 1.04 \\
1145 & 100157 & 1992 & 53.03 & 0.07 & 4901.00 & 51016.19 & 1.08 & 0.96 & 1.04 \\
25687 & 103514 & 1992 & 835.53 & 0.09 & 83553.00 & 797318.09 & 1.00 & 0.95 & 0.95 \\
13609 & 101748 & 1992 & 8.28 & -0.02 & 784.00 & 7618.08 & 1.06 & 0.92 & 0.97 \\
26476 & 103582 & 1992 & 11.20 & 0.08 & 994.00 & 9266.78 & 1.13 & 0.83 & 0.93 \\
9575 & 101151 & 1992 & 108.10 & 0.11 & 10810.00 & 92956.36 & 1.00 & 0.86 & 0.86 \\
4618 & 100644 & 1992 & 46.33 & 0.01 & 4633.00 & 42158.49 & 1.00 & 0.91 & 0.91 \\
10189 & 101268 & 1992 & 145.00 & 0.04 & 14502.00 & 132232.71 & 1.00 & 0.91 & 0.91 \\
8783 & 101098 & 1992 & 22.50 & 0.03 & 1850.00 & 18615.36 & 1.22 & 0.83 & 1.01 \\
2896 & 100373 & 1992 & 20.26 & -0.18 & 2090.00 & 17827.88 & 0.97 & 0.88 & 0.85 \\
7157 & 101000 & 1992 & 397.20 & 0.12 & 39720.00 & 345177.90 & 1.00 & 0.87 & 0.87 \\
4625 & 100648 & 1992 & 9.32 & 0.07 & 932.00 & 8551.23 & 1.00 & 0.92 & 0.92 \\
25719 & 103520 & 1992 & 21.10 & 0.02 & 2110.00 & 21380.60 & 1.00 & 1.01 & 1.01 \\
2213 & 100296 & 1992 & 3.60 & -0.00 & 658.00 & 6403.74 & 0.55 & 1.78 & 0.97 \\
5338 & 100754 & 1992 & 411.92 & 0.08 & 41190.00 & 396129.60 & 1.00 & 0.96 & 0.96 \\
45873 & 200153 & 1992 & 11.78 & -0.00 & 1360.00 & 12691.55 & 0.87 & 1.08 & 0.93 \\
4588 & 100642 & 1992 & 614.43 & 0.07 & 61328.00 & 525780.85 & 1.00 & 0.86 & 0.86 \\
26513 & 103590 & 1992 & 15.67 & 0.05 & 1567.00 & 14227.99 & 1.00 & 0.91 & 0.91 \\
23470 & 103179 & 1992 & 129.92 & 0.02 & 12992.00 & 109298.76 & 1.00 & 0.84 & 0.84 \\
18505 & 102469 & 1992 & 102.43 & 0.11 & 9291.00 & 88358.34 & 1.10 & 0.86 & 0.95 \\
24191 & 103296 & 1992 & 1023.26 & 0.02 & 92433.00 & 955754.80 & 1.11 & 0.93 & 1.03 \\
2225 & 100298 & 1992 & 112.08 & 0.09 & 10793.00 & 102569.87 & 1.04 & 0.92 & 0.95 \\
22072 & 102989 & 1992 & 572.90 & 0.07 & 57300.00 & 531867.23 & 1.00 & 0.93 & 0.93 \\
15329 & 101987 & 1992 & 242.60 & 0.04 & 24256.00 & 230050.01 & 1.00 & 0.95 & 0.95 \\
7138 & 100998 & 1992 & 29.92 & 0.02 & 1993.00 & 19900.50 & 1.50 & 0.67 & 1.00 \\
4957 & 100697 & 1992 & 32.65 & 0.12 & 3296.00 & 31193.16 & 0.99 & 0.96 & 0.95 \\
12300 & 101534 & 1992 & 146.61 & 0.13 & 11527.00 & 116196.89 & 1.27 & 0.79 & 1.01 \\
25191 & 103460 & 1992 & 289.98 & -0.09 & 23497.00 & 256501.85 & 1.23 & 0.88 & 1.09 \\
8238 & 101080 & 1992 & 41.50 & 0.03 & 3695.00 & 35257.40 & 1.12 & 0.85 & 0.95 \\
18436 & 102455 & 1992 & 11.38 & 0.02 & 755.00 & 7409.26 & 1.51 & 0.65 & 0.98 \\
17130 & 102258 & 1992 & 738.73 & 0.01 & 73026.00 & 672061.42 & 1.01 & 0.91 & 0.92 \\
7455 & 101040 & 1992 & 452.10 & 0.13 & 34064.00 & 307414.82 & 1.33 & 0.68 & 0.90 \\
16767 & 102191 & 1992 & 40.80 & 0.11 & 4081.00 & 36381.41 & 1.00 & 0.89 & 0.89 \\
4634 & 100659 & 1992 & 179.66 & 0.04 & 14177.00 & 141007.09 & 1.27 & 0.78 & 0.99 \\
8121 & 101076 & 1992 & 68.50 & 0.00 & 6208.00 & 56471.71 & 1.10 & 0.82 & 0.91 \\
25751 & 103521 & 1992 & 23.54 & 0.02 & 2354.00 & 22967.76 & 1.00 & 0.98 & 0.98 \\
21155 & 102835 & 1992 & 46.98 & -0.03 & 4698.00 & 38974.28 & 1.00 & 0.83 & 0.83 \\
7270 & 101018 & 1992 & 565.70 & 0.06 & 56570.00 & 524908.32 & 1.00 & 0.93 & 0.93 \\
2907 & 100379 & 1992 & 200.18 & 0.09 & 20018.00 & 196358.12 & 1.00 & 0.98 & 0.98 \\
16839 & 102197 & 1992 & 48.57 & 0.11 & 4857.00 & 39082.06 & 1.00 & 0.80 & 0.80 \\
23538 & 103184 & 1992 & 718.30 & 0.14 & 72411.00 & 602541.00 & 0.99 & 0.84 & 0.83 \\
25785 & 103523 & 1992 & 584.86 & 0.12 & 58484.00 & 555391.27 & 1.00 & 0.95 & 0.95 \\
8843 & 101102 & 1992 & 93.50 & -0.00 & 7439.00 & 68713.45 & 1.26 & 0.73 & 0.92 \\
14977 & 101926 & 1992 & 38.65 & 0.03 & 3754.00 & 37568.97 & 1.03 & 0.97 & 1.00 \\
48172 & 240040 & 1992 & 7.20 & 0.13 & 721.00 & 6715.28 & 1.00 & 0.93 & 0.93 \\
21187 & 102837 & 1992 & 31.61 & 0.04 & 3161.00 & 29142.67 & 1.00 & 0.92 & 0.92 \\
2027 & 100281 & 1992 & 15.70 & 0.00 & 1425.00 & 14747.50 & 1.10 & 0.94 & 1.03 \\
4695 & 100667 & 1992 & 13.56 & 0.15 & 1365.00 & 11011.54 & 0.99 & 0.81 & 0.81 \\
26425 & 103580 & 1992 & 6.09 & 0.01 & 609.00 & 4981.05 & 1.00 & 0.82 & 0.82 \\
14945 & 101925 & 1992 & 173.38 & 0.13 & 17371.00 & 155042.67 & 1.00 & 0.89 & 0.89 \\
24637 & 103373 & 1992 & 273.39 & 0.05 & 24825.00 & 269808.94 & 1.10 & 0.99 & 1.09 \\
47500 & 212408 & 1992 & 31.32 & -0.01 & 2502.00 & 25142.00 & 1.25 & 0.80 & 1.00 \\
13970 & 101794 & 1992 & 73.15 & 0.04 & 7315.00 & 65812.94 & 1.00 & 0.90 & 0.90 \\
2986 & 100395 & 1992 & 431.30 & 0.04 & 44120.00 & 392443.13 & 0.98 & 0.91 & 0.89 \\
21101 & 102832 & 1992 & 44.20 & 0.00 & 4058.00 & 41471.93 & 1.09 & 0.94 & 1.02 \\
15138 & 101963 & 1992 & 102.34 & 0.20 & 7788.00 & 71332.03 & 1.31 & 0.70 & 0.92 \\
15280 & 101978 & 1992 & 4.51 & 0.08 & 535.00 & 5410.04 & 0.84 & 1.20 & 1.01 \\
1816 & 100244 & 1992 & 52.77 & 0.06 & 5232.00 & 41900.80 & 1.01 & 0.79 & 0.80 \\
14014 & 101800 & 1992 & 359.20 & 0.08 & 35920.00 & 347343.82 & 1.00 & 0.97 & 0.97 \\
10515 & 101298 & 1992 & 212.56 & -0.20 & 21238.00 & 177328.96 & 1.00 & 0.83 & 0.83 \\
1130 & 100155 & 1992 & 84.50 & 0.06 & 8456.00 & 79739.12 & 1.00 & 0.94 & 0.94 \\
21133 & 102833 & 1992 & 2.56 & 0.02 & 184.00 & 1575.67 & 1.39 & 0.62 & 0.86 \\
24677 & 103375 & 1992 & 97.35 & 0.18 & 7752.00 & 88702.92 & 1.26 & 0.91 & 1.14 \\
18263 & 102419 & 1992 & 216.11 & 0.08 & 21565.00 & 188529.93 & 1.00 & 0.87 & 0.87 \\
6422 & 100868 & 1992 & 91.15 & 0.01 & 9114.00 & 83918.77 & 1.00 & 0.92 & 0.92 \\
17067 & 102241 & 1992 & 218.81 & 0.17 & 20788.00 & 215521.09 & 1.05 & 0.98 & 1.04 \\
4931 & 100695 & 1992 & 70.76 & 0.02 & 7077.00 & 67535.07 & 1.00 & 0.95 & 0.95 \\
8351 & 101084 & 1992 & 130.70 & 0.09 & 11810.00 & 95861.69 & 1.11 & 0.73 & 0.81 \\
13406 & 101738 & 1992 & 98.94 & 0.05 & 9893.00 & 81701.23 & 1.00 & 0.83 & 0.83 \\
18313 & 102425 & 1992 & 751.25 & 0.13 & 75324.00 & 718293.66 & 1.00 & 0.96 & 0.95 \\
6679 & 100910 & 1992 & 40.90 & 0.08 & 4131.00 & 37883.69 & 0.99 & 0.93 & 0.92 \\
4668 & 100660 & 1992 & 87.33 & 0.11 & 8110.00 & 80300.28 & 1.08 & 0.92 & 0.99 \\
57833 & 401082 & 1992 & 23.11 & 0.04 & 2310.00 & 21852.16 & 1.00 & 0.95 & 0.95 \\
23512 & 103183 & 1992 & 426.02 & 0.08 & 42822.00 & 375506.73 & 0.99 & 0.88 & 0.88 \\
25953 & 103531 & 1992 & 155.67 & 0.06 & 15567.00 & 150190.94 & 1.00 & 0.96 & 0.96 \\
20760 & 102789 & 1992 & 168.27 & 0.18 & 14294.00 & 129929.62 & 1.18 & 0.77 & 0.91 \\
20799 & 102795 & 1992 & 54.18 & 0.21 & 3969.00 & 38218.53 & 1.37 & 0.71 & 0.96 \\
1676 & 100223 & 1992 & 188.52 & 0.12 & 17735.00 & 164908.00 & 1.06 & 0.87 & 0.93 \\
7384 & 101038 & 1992 & 2571.70 & 0.15 & 237284.00 & 2247778.57 & 1.08 & 0.87 & 0.95 \\
22784 & 103064 & 1992 & 21.10 & 0.01 & 2110.00 & 20089.27 & 1.00 & 0.95 & 0.95 \\
8671 & 101094 & 1992 & 143.70 & 0.36 & 14749.00 & 126530.18 & 0.97 & 0.88 & 0.86 \\
3964 & 100535 & 1992 & 259.60 & 0.27 & 26026.00 & 213560.18 & 1.00 & 0.82 & 0.82 \\
26906 & 103621 & 1992 & 49.00 & 0.02 & 4868.00 & 46109.13 & 1.01 & 0.94 & 0.95 \\
22786 & 103065 & 1992 & 334.96 & -0.02 & 33486.00 & 283633.75 & 1.00 & 0.85 & 0.85 \\
13118 & 101668 & 1992 & 39.20 & 0.01 & 3920.00 & 39259.62 & 1.00 & 1.00 & 1.00 \\
7695 & 101055 & 1992 & 1815.50 & 0.08 & 181550.00 & 1623496.96 & 1.00 & 0.89 & 0.89 \\
19832 & 102653 & 1992 & 1946.25 & 0.06 & 179182.00 & 1507109.16 & 1.09 & 0.77 & 0.84 \\
6189 & 100829 & 1992 & 232.86 & 0.09 & 23360.00 & 208441.43 & 1.00 & 0.90 & 0.89 \\
9065 & 101110 & 1992 & 15.10 & 0.12 & 1468.00 & 13077.69 & 1.03 & 0.87 & 0.89 \\
6903 & 100968 & 1992 & 13.00 & 0.08 & 1056.00 & 10105.36 & 1.23 & 0.78 & 0.96 \\
12025 & 101488 & 1992 & 46.58 & 0.05 & 4658.00 & 44165.57 & 1.00 & 0.95 & 0.95 \\
53397 & 346113 & 1992 & 170.72 & 0.04 & 15236.00 & 147146.41 & 1.12 & 0.86 & 0.97 \\
22821 & 103067 & 1992 & 37.19 & 0.05 & 3719.00 & 30667.14 & 1.00 & 0.82 & 0.82 \\
19791 & 102652 & 1992 & 839.59 & 0.15 & 72504.00 & 596821.92 & 1.16 & 0.71 & 0.82 \\
74581 & 601139 & 1992 & 145.40 & 0.15 & 13193.00 & 109354.04 & 1.10 & 0.75 & 0.83 \\
20379 & 102733 & 1992 & 3636.90 & 0.09 & 310804.00 & 2572602.55 & 1.17 & 0.71 & 0.83 \\
11016 & 101360 & 1992 & 631.82 & 0.15 & 63257.00 & 530999.27 & 1.00 & 0.84 & 0.84 \\
74553 & 601136 & 1992 & 16.54 & -0.01 & 1654.00 & 16100.71 & 1.00 & 0.97 & 0.97 \\
17376 & 102284 & 1992 & 44.50 & 0.18 & 3968.00 & 41359.07 & 1.12 & 0.93 & 1.04 \\
19914 & 102655 & 1992 & 699.60 & 0.02 & 71724.00 & 654880.94 & 0.98 & 0.94 & 0.91 \\
20362 & 102728 & 1992 & 142.01 & 0.02 & 16090.00 & 153940.10 & 0.88 & 1.08 & 0.96 \\
3673 & 100468 & 1992 & 179.30 & 0.13 & 17926.00 & 161074.57 & 1.00 & 0.90 & 0.90 \\
74784 & 601171 & 1992 & 85.95 & 0.07 & 8107.00 & 84118.05 & 1.06 & 0.98 & 1.04 \\
26920 & 103628 & 1992 & 160.24 & 0.25 & 12097.00 & 132429.74 & 1.32 & 0.83 & 1.09 \\
22853 & 103073 & 1992 & 345.15 & 0.04 & 34515.00 & 313463.84 & 1.00 & 0.91 & 0.91 \\
8594 & 101091 & 1992 & 117.40 & 0.02 & 11110.00 & 103007.91 & 1.06 & 0.88 & 0.93 \\
7583 & 101047 & 1992 & 186.90 & 0.05 & 16476.00 & 185315.65 & 1.13 & 0.99 & 1.12 \\
2713 & 100355 & 1992 & 140.57 & 0.08 & 14057.00 & 138584.88 & 1.00 & 0.99 & 0.99 \\
24874 & 103383 & 1992 & 935.59 & -0.02 & 82968.00 & 780431.51 & 1.13 & 0.83 & 0.94 \\
17397 & 102286 & 1992 & 28.90 & 0.02 & 2890.00 & 25935.65 & 1.00 & 0.90 & 0.90 \\
20374 & 102732 & 1992 & 247.70 & 0.03 & 24017.00 & 192560.93 & 1.03 & 0.78 & 0.80 \\
65197 & 500670 & 1992 & 360.18 & -0.29 & 35934.00 & 298999.75 & 1.00 & 0.83 & 0.83 \\
1850 & 100245 & 1992 & 250.15 & 0.07 & 25708.00 & 222142.26 & 0.97 & 0.89 & 0.86 \\
20458 & 102749 & 1992 & 40.56 & 0.05 & 3439.00 & 35661.65 & 1.18 & 0.88 & 1.04 \\
9342 & 101132 & 1992 & 12.98 & -0.13 & 1298.00 & 12514.39 & 1.00 & 0.96 & 0.96 \\
22888 & 103083 & 1992 & 10.66 & 0.03 & 937.00 & 9633.74 & 1.14 & 0.90 & 1.03 \\
15054 & 101955 & 1992 & 38.93 & 0.31 & 3792.00 & 31799.65 & 1.03 & 0.82 & 0.84 \\
2498 & 100336 & 1992 & 15.46 & 0.11 & 1030.00 & 8443.56 & 1.50 & 0.55 & 0.82 \\
6783 & 100954 & 1992 & 63.49 & 0.09 & 6384.00 & 55993.26 & 0.99 & 0.88 & 0.88 \\
4041 & 100543 & 1992 & 394.53 & 0.17 & 39453.00 & 393275.01 & 1.00 & 1.00 & 1.00 \\
4348 & 100611 & 1992 & 471.80 & 0.21 & 47187.00 & 390149.03 & 1.00 & 0.83 & 0.83 \\
19628 & 102639 & 1992 & 71.25 & 0.05 & 6652.00 & 61138.15 & 1.07 & 0.86 & 0.92 \\
12586 & 101556 & 1992 & 5.62 & -0.04 & 528.00 & 5346.87 & 1.06 & 0.95 & 1.01 \\
19596 & 102636 & 1992 & 238.00 & 0.04 & 20521.00 & 192917.77 & 1.16 & 0.81 & 0.94 \\
8559 & 101090 & 1992 & 56.50 & 0.22 & 3908.00 & 37430.93 & 1.45 & 0.66 & 0.96 \\
26786 & 103607 & 1992 & 215.04 & 0.06 & 20338.00 & 195304.79 & 1.06 & 0.91 & 0.96 \\
19586 & 102635 & 1992 & 144.67 & 0.01 & 13571.00 & 112318.52 & 1.07 & 0.78 & 0.83 \\
12907 & 101606 & 1992 & 962.76 & 0.08 & 96280.00 & 835412.62 & 1.00 & 0.87 & 0.87 \\
3614 & 100463 & 1992 & 2.06 & 0.00 & 195.00 & 1724.95 & 1.05 & 0.84 & 0.88 \\
48371 & 240067 & 1992 & 18.09 & 0.04 & 1808.00 & 16246.81 & 1.00 & 0.90 & 0.90 \\
16134 & 102085 & 1992 & 928.00 & 0.09 & 89652.00 & 872427.28 & 1.04 & 0.94 & 0.97 \\
11150 & 101369 & 1992 & 624.36 & 0.04 & 54527.00 & 465890.91 & 1.15 & 0.75 & 0.85 \\
26813 & 103608 & 1992 & 32.40 & 0.02 & 3120.00 & 30763.73 & 1.04 & 0.95 & 0.99 \\
5967 & 100813 & 1992 & 205.64 & 0.05 & 11468.00 & 110911.76 & 1.79 & 0.54 & 0.97 \\
24522 & 103338 & 1992 & 43.78 & 0.03 & 4380.00 & 36127.92 & 1.00 & 0.83 & 0.82 \\
11114 & 101368 & 1992 & 423.62 & 0.13 & 43797.00 & 427179.49 & 0.97 & 1.01 & 0.98 \\
22872 & 103074 & 1992 & 59.39 & 0.06 & 5366.00 & 53716.18 & 1.11 & 0.90 & 1.00 \\
25489 & 103494 & 1992 & 327.43 & 0.05 & 33473.00 & 299586.92 & 0.98 & 0.91 & 0.90 \\
14665 & 101908 & 1992 & 7.40 & 0.09 & 774.00 & 7935.47 & 0.96 & 1.07 & 1.03 \\
27005 & 103644 & 1992 & 153.31 & 0.05 & 15331.00 & 145506.40 & 1.00 & 0.95 & 0.95 \\
4007 & 100538 & 1992 & 286.72 & 0.20 & 21686.00 & 232477.65 & 1.32 & 0.81 & 1.07 \\
24527 & 103339 & 1992 & 165.36 & 0.13 & 16540.00 & 147716.24 & 1.00 & 0.89 & 0.89 \\
11080 & 101367 & 1992 & 123.92 & 0.02 & 9486.00 & 87780.82 & 1.31 & 0.71 & 0.93 \\
10792 & 101331 & 1992 & 34.89 & 0.02 & 3474.00 & 31239.42 & 1.00 & 0.90 & 0.90 \\
7901 & 101065 & 1992 & 799.30 & -0.10 & 73543.00 & 639074.05 & 1.09 & 0.80 & 0.87 \\
14461 & 101861 & 1992 & 370.70 & 0.07 & 36064.00 & 351205.50 & 1.03 & 0.95 & 0.97 \\
24833 & 103381 & 1992 & 5299.28 & 0.01 & 427520.00 & 4144861.55 & 1.24 & 0.78 & 0.97 \\
26857 & 103614 & 1992 & 13.10 & 0.06 & 1316.00 & 11213.19 & 1.00 & 0.86 & 0.85 \\
21805 & 102952 & 1992 & 923.49 & -0.02 & 69274.00 & 773393.61 & 1.33 & 0.84 & 1.12 \\
26096 & 103538 & 1992 & 56.76 & 0.13 & 5676.00 & 49261.85 & 1.00 & 0.87 & 0.87 \\
15580 & 102007 & 1992 & 255.29 & 0.03 & 24036.00 & 212727.36 & 1.06 & 0.83 & 0.89 \\
26884 & 103620 & 1992 & 47.30 & 0.02 & 4646.00 & 40929.48 & 1.02 & 0.87 & 0.88 \\
15037 & 101953 & 1992 & 204.30 & 0.08 & 19686.00 & 209801.12 & 1.04 & 1.03 & 1.07 \\
3910 & 100514 & 1992 & 101.23 & 0.01 & 10123.00 & 92742.97 & 1.00 & 0.92 & 0.92 \\
10102 & 101259 & 1992 & 16.91 & 0.04 & 1727.00 & 15186.84 & 0.98 & 0.90 & 0.88 \\
10950 & 101356 & 1992 & 29.90 & 0.04 & 2990.00 & 25909.74 & 1.00 & 0.87 & 0.87 \\
22561 & 103021 & 1992 & 31.75 & 0.17 & 3175.00 & 26465.20 & 1.00 & 0.83 & 0.83 \\
8657 & 101093 & 1992 & 10.40 & 0.06 & 1279.00 & 11317.99 & 0.81 & 1.09 & 0.88 \\
8893 & 101104 & 1992 & 16.40 & 0.04 & 1268.00 & 12944.43 & 1.29 & 0.79 & 1.02 \\
2681 & 100352 & 1992 & 73.40 & 0.01 & 7340.00 & 72611.95 & 1.00 & 0.99 & 0.99 \\
15837 & 102043 & 1992 & 59.33 & 0.10 & 5947.00 & 53902.06 & 1.00 & 0.91 & 0.91 \\
22673 & 103028 & 1992 & 2834.20 & 0.04 & 175965.00 & 1630342.48 & 1.61 & 0.58 & 0.93 \\
369 & 100044 & 1992 & 18.02 & 0.11 & 1547.00 & 13925.40 & 1.16 & 0.77 & 0.90 \\
20146 & 102671 & 1992 & 28.70 & 0.05 & 3002.00 & 29245.62 & 0.96 & 1.02 & 0.97 \\
20272 & 102702 & 1992 & 143.60 & 0.02 & 13381.00 & 150261.80 & 1.07 & 1.05 & 1.12 \\
3808 & 100485 & 1992 & 59.80 & 0.01 & 4268.00 & 37683.01 & 1.40 & 0.63 & 0.88 \\
21743 & 102949 & 1992 & 1244.65 & 0.07 & 105910.00 & 1182732.03 & 1.18 & 0.95 & 1.12 \\
10965 & 101357 & 1992 & 29.54 & 0.07 & 2953.00 & 24073.49 & 1.00 & 0.82 & 0.82 \\
735 & 100093 & 1992 & 230.22 & 0.04 & 18243.00 & 183780.61 & 1.26 & 0.80 & 1.01 \\
7863 & 101064 & 1992 & 155.20 & -0.03 & 16198.00 & 133254.02 & 0.96 & 0.86 & 0.82 \\
20135 & 102669 & 1992 & 25.46 & 0.03 & 2702.00 & 25346.93 & 0.94 & 1.00 & 0.94 \\
26972 & 103642 & 1992 & 64.81 & 0.03 & 6481.00 & 62546.24 & 1.00 & 0.97 & 0.97 \\
3819 & 100489 & 1992 & 33.35 & 0.02 & 2935.00 & 29206.34 & 1.14 & 0.88 & 1.00 \\
1528 & 100213 & 1992 & 162.80 & 0.11 & 17542.00 & 172143.64 & 0.93 & 1.06 & 0.98 \\
20102 & 102667 & 1992 & 646.09 & 0.09 & 64609.00 & 618651.80 & 1.00 & 0.96 & 0.96 \\
20275 & 102703 & 1992 & 157.10 & 0.02 & 15717.00 & 146943.72 & 1.00 & 0.94 & 0.93 \\
20180 & 102676 & 1992 & 21.14 & 0.01 & 2114.00 & 20362.01 & 1.00 & 0.96 & 0.96 \\
57959 & 410010 & 1992 & 91.21 & -0.02 & 9121.00 & 82946.77 & 1.00 & 0.91 & 0.91 \\
8170 & 101078 & 1992 & 1.60 & 0.05 & 113.00 & 1042.41 & 1.42 & 0.65 & 0.92 \\
15750 & 102017 & 1992 & 1546.37 & 0.07 & 149064.00 & 1337688.51 & 1.04 & 0.87 & 0.90 \\
21761 & 102951 & 1992 & 1168.87 & 0.10 & 86750.00 & 694982.40 & 1.35 & 0.59 & 0.80 \\
13041 & 101623 & 1992 & 659.40 & 0.13 & 58212.00 & 497929.15 & 1.13 & 0.76 & 0.86 \\
12697 & 101588 & 1992 & 100.07 & 0.05 & 10007.00 & 90068.48 & 1.00 & 0.90 & 0.90 \\
20229 & 102689 & 1992 & 10.32 & 0.03 & 1032.00 & 9105.96 & 1.00 & 0.88 & 0.88 \\
6827 & 100962 & 1992 & 620.42 & 0.13 & 62042.00 & 532104.22 & 1.00 & 0.86 & 0.86 \\
9028 & 101109 & 1992 & 197.20 & -0.13 & 11652.00 & 114713.03 & 1.69 & 0.58 & 0.98 \\
41629 & 108793 & 1992 & 8.36 & -0.12 & 836.00 & 7305.46 & 1.00 & 0.87 & 0.87 \\
7602 & 101048 & 1992 & 179.00 & 0.10 & 12277.00 & 108876.67 & 1.46 & 0.61 & 0.89 \\
25402 & 103483 & 1992 & 96.46 & 0.17 & 9646.00 & 95122.40 & 1.00 & 0.99 & 0.99 \\
15716 & 102016 & 1992 & 1080.19 & 0.06 & 107039.00 & 965854.30 & 1.01 & 0.89 & 0.90 \\
10918 & 101354 & 1992 & 254.04 & 0.06 & 25404.00 & 231749.75 & 1.00 & 0.91 & 0.91 \\
15796 & 102026 & 1992 & 9.56 & 0.01 & 643.00 & 5440.16 & 1.49 & 0.57 & 0.85 \\
17284 & 102278 & 1992 & 156.26 & 0.17 & 15624.00 & 154457.71 & 1.00 & 0.99 & 0.99 \\
20195 & 102688 & 1992 & 10.32 & 0.00 & 1032.00 & 10305.36 & 1.00 & 1.00 & 1.00 \\
24933 & 103395 & 1992 & 85.70 & 0.07 & 8570.00 & 68725.91 & 1.00 & 0.80 & 0.80 \\
24564 & 103368 & 1992 & 68.38 & 0.06 & 6838.00 & 66713.13 & 1.00 & 0.98 & 0.98 \\
22708 & 103029 & 1992 & 233.38 & 0.09 & 19793.00 & 198634.70 & 1.18 & 0.85 & 1.00 \\
1341 & 100190 & 1992 & 759.58 & 0.08 & 74239.00 & 708751.81 & 1.02 & 0.93 & 0.95 \\
1510 & 100209 & 1992 & 406.62 & 0.10 & 40148.00 & 373493.66 & 1.01 & 0.92 & 0.93 \\
2601 & 100346 & 1992 & 5.71 & -0.02 & 571.00 & 4664.89 & 1.00 & 0.82 & 0.82 \\
623 & 100085 & 1992 & 239.66 & 0.16 & 23966.00 & 209799.51 & 1.00 & 0.88 & 0.88 \\
26132 & 103544 & 1992 & 275.40 & 0.02 & 27540.00 & 267326.86 & 1.00 & 0.97 & 0.97 \\
24553 & 103366 & 1992 & 112.90 & 0.05 & 9226.00 & 97473.68 & 1.22 & 0.86 & 1.06 \\
20294 & 102715 & 1992 & 1340.14 & 0.13 & 117063.00 & 1057169.20 & 1.14 & 0.79 & 0.90 \\
12042 & 101491 & 1992 & 73.34 & 0.03 & 7111.00 & 68123.89 & 1.03 & 0.93 & 0.96 \\
6574 & 100891 & 1992 & 154.86 & 0.04 & 15486.00 & 151018.22 & 1.00 & 0.98 & 0.98 \\
65127 & 500664 & 1992 & 592.43 & 0.08 & 48073.00 & 436901.24 & 1.23 & 0.74 & 0.91 \\
14547 & 101876 & 1992 & 125.93 & 0.12 & 12593.00 & 120141.34 & 1.00 & 0.95 & 0.95 \\
6804 & 100956 & 1992 & 1.27 & -0.04 & 130.00 & 1069.34 & 0.98 & 0.84 & 0.82 \\
3897 & 100510 & 1992 & 16.18 & 0.09 & 1336.00 & 14866.71 & 1.21 & 0.92 & 1.11 \\
22734 & 103050 & 1992 & 187.70 & 0.01 & 19489.00 & 174079.71 & 0.96 & 0.93 & 0.89 \\
20328 & 102716 & 1992 & 258.55 & 0.08 & 24201.00 & 221711.98 & 1.07 & 0.86 & 0.92 \\
16018 & 102073 & 1992 & 2819.46 & 0.10 & 281946.00 & 2556737.35 & 1.00 & 0.91 & 0.91 \\
22738 & 103057 & 1992 & 781.30 & 0.13 & 58505.00 & 543087.81 & 1.34 & 0.70 & 0.93 \\
3098 & 100409 & 1992 & 4.97 & 0.05 & 497.00 & 4963.35 & 1.00 & 1.00 & 1.00 \\
563 & 100076 & 1992 & 168.40 & 0.00 & 16839.00 & 167489.73 & 1.00 & 0.99 & 0.99 \\
13653 & 101754 & 1992 & 34.90 & -0.03 & 3490.00 & 31609.87 & 1.00 & 0.91 & 0.91 \\
20074 & 102665 & 1992 & 27.13 & 0.00 & 3168.00 & 23245.70 & 0.86 & 0.86 & 0.73 \\
26979 & 103643 & 1992 & 80.31 & 0.08 & 8031.00 & 79655.56 & 1.00 & 0.99 & 0.99 \\
971 & 100113 & 1992 & 799.54 & 0.15 & 79900.00 & 774612.04 & 1.00 & 0.97 & 0.97 \\
331 & 100039 & 1992 & 144.30 & 0.11 & 13708.00 & 110435.55 & 1.05 & 0.77 & 0.81 \\
6863 & 100966 & 1992 & 16.79 & 0.14 & 1719.00 & 17200.05 & 0.98 & 1.02 & 1.00 \\
3716 & 100475 & 1992 & 74.50 & 0.08 & 7450.00 & 70166.00 & 1.00 & 0.94 & 0.94 \\
22718 & 103039 & 1992 & 44.60 & -0.08 & 4425.00 & 44378.91 & 1.01 & 1.00 & 1.00 \\
588 & 100079 & 1992 & 604.31 & 0.05 & 60431.00 & 590428.15 & 1.00 & 0.98 & 0.98 \\
10976 & 101358 & 1992 & 139.33 & 0.08 & 13933.00 & 137465.03 & 1.00 & 0.99 & 0.99 \\
9379 & 101133 & 1992 & 246.63 & 0.27 & 24663.00 & 219871.31 & 1.00 & 0.89 & 0.89 \\
25179 & 103451 & 1992 & 4.28 & -0.11 & 430.00 & 4010.74 & 1.00 & 0.94 & 0.93 \\
22721 & 103042 & 1992 & 63.90 & -0.08 & 5881.00 & 51460.96 & 1.09 & 0.81 & 0.88 \\
19942 & 102659 & 1992 & 1547.26 & 0.11 & 118789.00 & 1329904.51 & 1.30 & 0.86 & 1.12 \\
1360 & 100192 & 1992 & 38.69 & 0.01 & 3867.00 & 37497.80 & 1.00 & 0.97 & 0.97 \\
26166 & 103545 & 1992 & 5416.73 & 0.12 & 541673.00 & 4945127.33 & 1.00 & 0.91 & 0.91 \\
15906 & 102059 & 1992 & 136.50 & 0.16 & 13571.00 & 119769.12 & 1.01 & 0.88 & 0.88 \\
17429 & 102306 & 1992 & 547.91 & 0.12 & 54791.00 & 528225.89 & 1.00 & 0.96 & 0.96 \\
8486 & 101088 & 1992 & 53.80 & 0.02 & 7281.00 & 63987.08 & 0.74 & 1.19 & 0.88 \\
47912 & 222809 & 1992 & 126.38 & -0.00 & 8703.00 & 82218.52 & 1.45 & 0.65 & 0.94 \\
409 & 100055 & 1992 & 3805.76 & 0.12 & 382532.00 & 3162187.01 & 0.99 & 0.83 & 0.83 \\
2410 & 100323 & 1992 & 65.65 & 0.03 & 6565.00 & 60739.63 & 1.00 & 0.93 & 0.93 \\
8143 & 101077 & 1992 & 8.50 & 0.01 & 727.00 & 7075.06 & 1.17 & 0.83 & 0.97 \\
17302 & 102280 & 1992 & 974.61 & 0.18 & 86970.00 & 850107.97 & 1.12 & 0.87 & 0.98 \\
16354 & 102130 & 1992 & 354.31 & 0.06 & 35431.00 & 295240.21 & 1.00 & 0.83 & 0.83 \\
26007 & 103533 & 1992 & 607.41 & -0.02 & 60741.00 & 517803.97 & 1.00 & 0.85 & 0.85 \\
23134 & 103134 & 1992 & 230.44 & 0.07 & 21642.00 & 230643.65 & 1.06 & 1.00 & 1.07 \\
16389 & 102132 & 1992 & 26.57 & 0.05 & 2692.00 & 25749.89 & 0.99 & 0.97 & 0.96 \\
2390 & 100322 & 1992 & 35.72 & 0.15 & 3572.00 & 32785.84 & 1.00 & 0.92 & 0.92 \\
20702 & 102784 & 1992 & 2074.10 & 0.10 & 159162.00 & 1614352.47 & 1.30 & 0.78 & 1.01 \\
21588 & 102895 & 1992 & 84.52 & 0.09 & 8419.00 & 81722.15 & 1.00 & 0.97 & 0.97 \\
8201 & 101079 & 1992 & 68.80 & -0.02 & 7343.00 & 64985.54 & 0.94 & 0.94 & 0.88 \\
16300 & 102124 & 1992 & 744.48 & 0.03 & 74447.00 & 668268.83 & 1.00 & 0.90 & 0.90 \\
2427 & 100324 & 1992 & 41.41 & 0.14 & 4141.00 & 33462.28 & 1.00 & 0.81 & 0.81 \\
25546 & 103496 & 1992 & 290.55 & -0.02 & 29454.00 & 273607.62 & 0.99 & 0.94 & 0.93 \\
49055 & 240212 & 1992 & 975.00 & -0.11 & 97041.00 & 919413.11 & 1.00 & 0.94 & 0.95 \\
11276 & 101390 & 1992 & 1568.89 & 0.08 & 148400.00 & 1362081.47 & 1.06 & 0.87 & 0.92 \\
4205 & 100575 & 1992 & 19.19 & 0.13 & 1981.00 & 19532.53 & 0.97 & 1.02 & 0.99 \\
5934 & 100812 & 1992 & 635.30 & -0.04 & 64180.00 & 570291.10 & 0.99 & 0.90 & 0.89 \\
16334 & 102127 & 1992 & 9.60 & -0.02 & 961.00 & 9585.66 & 1.00 & 1.00 & 1.00 \\
15490 & 101999 & 1992 & 547.10 & 0.03 & 54709.00 & 518299.97 & 1.00 & 0.95 & 0.95 \\
41262 & 108710 & 1992 & 32.09 & 0.06 & 3209.00 & 27891.48 & 1.00 & 0.87 & 0.87 \\
12757 & 101593 & 1992 & 135.72 & 0.04 & 13572.00 & 111668.69 & 1.00 & 0.82 & 0.82 \\
12008 & 101477 & 1992 & 48.10 & 0.15 & 2765.00 & 24835.96 & 1.74 & 0.52 & 0.90 \\
6542 & 100890 & 1992 & 222.81 & 0.09 & 22784.00 & 229438.63 & 0.98 & 1.03 & 1.01 \\
23066 & 103110 & 1992 & 35.98 & 0.21 & 3407.00 & 33200.15 & 1.06 & 0.92 & 0.97 \\
24397 & 103319 & 1992 & 126.06 & 0.00 & 12295.00 & 122686.10 & 1.03 & 0.97 & 1.00 \\
1411 & 100196 & 1992 & 179.81 & 0.10 & 17762.00 & 146700.22 & 1.01 & 0.82 & 0.83 \\
12874 & 101603 & 1992 & 496.24 & 0.11 & 49620.00 & 428496.79 & 1.00 & 0.86 & 0.86 \\
14268 & 101842 & 1992 & 48.77 & 0.08 & 4925.00 & 45902.18 & 0.99 & 0.94 & 0.93 \\
17541 & 102318 & 1992 & 2772.67 & 0.13 & 277267.00 & 2453069.41 & 1.00 & 0.88 & 0.88 \\
15471 & 101998 & 1992 & 115.80 & 0.04 & 11580.00 & 109040.54 & 1.00 & 0.94 & 0.94 \\
7519 & 101043 & 1992 & 642.70 & 0.08 & 61467.00 & 538393.35 & 1.05 & 0.84 & 0.88 \\
15469 & 101997 & 1992 & 146.90 & 0.03 & 14693.00 & 132321.22 & 1.00 & 0.90 & 0.90 \\
24793 & 103380 & 1992 & 4691.89 & -0.04 & 453426.00 & 4376564.44 & 1.03 & 0.93 & 0.97 \\
24309 & 103308 & 1992 & 2538.76 & 0.12 & 253876.00 & 2076190.84 & 1.00 & 0.82 & 0.82 \\
15465 & 101996 & 1992 & 227.90 & 0.01 & 22791.00 & 212172.36 & 1.00 & 0.93 & 0.93 \\
5019 & 100701 & 1992 & 7.17 & 0.33 & 455.00 & 4399.69 & 1.58 & 0.61 & 0.97 \\
15451 & 101992 & 1992 & 1190.80 & 0.01 & 101985.00 & 987645.61 & 1.17 & 0.83 & 0.97 \\
3488 & 100441 & 1992 & 514.07 & 0.11 & 49668.00 & 415410.89 & 1.04 & 0.81 & 0.84 \\
1759 & 100228 & 1992 & 108.08 & 0.03 & 10808.00 & 90034.21 & 1.00 & 0.83 & 0.83 \\
3016 & 100398 & 1992 & 64.10 & 0.26 & 4556.00 & 51883.07 & 1.41 & 0.81 & 1.14 \\
2358 & 100320 & 1992 & 12.27 & 0.04 & 1227.00 & 11882.02 & 1.00 & 0.97 & 0.97 \\
4341 & 100610 & 1992 & 464.30 & 0.13 & 46426.00 & 414642.61 & 1.00 & 0.89 & 0.89 \\
49098 & 240222 & 1992 & 345.02 & 0.11 & 34501.00 & 313894.02 & 1.00 & 0.91 & 0.91 \\
11948 & 101473 & 1992 & 335.48 & 0.20 & 33566.00 & 299493.75 & 1.00 & 0.89 & 0.89 \\
11358 & 101398 & 1992 & 164.62 & 0.25 & 16462.00 & 140229.15 & 1.00 & 0.85 & 0.85 \\
1215 & 100166 & 1992 & 2939.89 & 0.07 & 293988.00 & 2467297.10 & 1.00 & 0.84 & 0.84 \\
9897 & 101211 & 1992 & 74.94 & 0.06 & 7285.00 & 67812.39 & 1.03 & 0.90 & 0.93 \\
44390 & 109300 & 1992 & 88.41 & 0.09 & 9120.00 & 81217.13 & 0.97 & 0.92 & 0.89 \\
8454 & 101087 & 1992 & 20.40 & 0.05 & 1416.00 & 13209.56 & 1.44 & 0.65 & 0.93 \\
5776 & 100792 & 1992 & 2.38 & 0.03 & 169.00 & 1961.62 & 1.41 & 0.82 & 1.16 \\
14990 & 101930 & 1992 & 449.61 & 0.06 & 44961.00 & 412162.32 & 1.00 & 0.92 & 0.92 \\
23150 & 103136 & 1992 & 44.88 & 0.13 & 2557.00 & 25198.02 & 1.76 & 0.56 & 0.99 \\
6294 & 100847 & 1992 & 3.00 & 0.09 & 299.00 & 2848.53 & 1.00 & 0.95 & 0.95 \\
11345 & 101397 & 1992 & 56.38 & 0.14 & 5638.00 & 49629.60 & 1.00 & 0.88 & 0.88 \\
26642 & 103595 & 1992 & 79.18 & 0.10 & 7007.00 & 57772.41 & 1.13 & 0.73 & 0.82 \\
24335 & 103315 & 1992 & 20.81 & 0.17 & 1874.00 & 15667.65 & 1.11 & 0.75 & 0.84 \\
9315 & 101131 & 1992 & 273.59 & 0.13 & 27359.00 & 221126.46 & 1.00 & 0.81 & 0.81 \\
12057 & 101494 & 1992 & 153.06 & 0.06 & 15306.00 & 155048.42 & 1.00 & 1.01 & 1.01 \\
21524 & 102880 & 1992 & 9.70 & 0.02 & 1278.00 & 12800.95 & 0.76 & 1.32 & 1.00 \\
16509 & 102152 & 1992 & 94.17 & 0.08 & 8967.00 & 84507.42 & 1.05 & 0.90 & 0.94 \\
18951 & 102531 & 1992 & 3.49 & 0.03 & 349.00 & 3425.71 & 1.00 & 0.98 & 0.98 \\
1962 & 100263 & 1992 & 15.10 & 0.07 & 1669.00 & 13549.07 & 0.90 & 0.90 & 0.81 \\
10250 & 101276 & 1992 & 125.11 & 0.22 & 12511.00 & 118583.14 & 1.00 & 0.95 & 0.95 \\
5744 & 100791 & 1992 & 14.43 & 0.20 & 1178.00 & 12984.64 & 1.22 & 0.90 & 1.10 \\
22237 & 102997 & 1992 & 57.00 & 0.17 & 4635.00 & 53667.21 & 1.23 & 0.94 & 1.16 \\
16532 & 102154 & 1992 & 95.14 & 0.07 & 7943.00 & 74882.83 & 1.20 & 0.79 & 0.94 \\
18983 & 102540 & 1992 & 7.93 & 0.04 & 730.00 & 6653.61 & 1.09 & 0.84 & 0.91 \\
13859 & 101781 & 1992 & 410.22 & 0.00 & 41022.00 & 407394.02 & 1.00 & 0.99 & 0.99 \\
13213 & 101704 & 1992 & 850.33 & -0.03 & 85033.00 & 813178.16 & 1.00 & 0.96 & 0.96 \\
21614 & 102901 & 1992 & 40.40 & 0.03 & 3945.00 & 34367.19 & 1.02 & 0.85 & 0.87 \\
15017 & 101943 & 1992 & 37.69 & 0.08 & 3769.00 & 37040.12 & 1.00 & 0.98 & 0.98 \\
19359 & 102599 & 1992 & 96.00 & 0.29 & 9600.00 & 82415.71 & 1.00 & 0.86 & 0.86 \\
20591 & 102774 & 1992 & 368.38 & 0.18 & 33636.00 & 330148.62 & 1.10 & 0.90 & 0.98 \\
17479 & 102313 & 1992 & 206.97 & 0.04 & 15541.00 & 183709.37 & 1.33 & 0.89 & 1.18 \\
2 & 100001 & 1992 & 596.27 & 0.06 & 59627.00 & 578201.50 & 1.00 & 0.97 & 0.97 \\
52028 & 301299 & 1992 & 47.75 & 0.01 & 4775.00 & 38691.30 & 1.00 & 0.81 & 0.81 \\
20551 & 102767 & 1992 & 735.38 & 0.08 & 71448.00 & 728950.23 & 1.03 & 0.99 & 1.02 \\
19484 & 102607 & 1992 & 873.00 & 0.20 & 87357.00 & 828041.78 & 1.00 & 0.95 & 0.95 \\
22325 & 103007 & 1992 & 687.81 & 0.06 & 68780.00 & 678289.25 & 1.00 & 0.99 & 0.99 \\
16162 & 102089 & 1992 & 254.28 & 0.08 & 25433.00 & 238644.31 & 1.00 & 0.94 & 0.94 \\
8522 & 101089 & 1992 & 22.30 & 0.03 & 1574.00 & 14345.73 & 1.42 & 0.64 & 0.91 \\
15106 & 101958 & 1992 & 50.18 & 0.01 & 4174.00 & 43302.37 & 1.20 & 0.86 & 1.04 \\
14396 & 101854 & 1992 & 842.05 & 0.11 & 84205.00 & 690456.75 & 1.00 & 0.82 & 0.82 \\
3133 & 100411 & 1992 & 839.74 & 0.16 & 83974.00 & 798116.40 & 1.00 & 0.95 & 0.95 \\
12563 & 101554 & 1992 & 85.55 & 0.10 & 7448.00 & 67870.75 & 1.15 & 0.79 & 0.91 \\
7936 & 101067 & 1992 & 34.70 & 0.02 & 3440.00 & 28750.90 & 1.01 & 0.83 & 0.84 \\
4100 & 100550 & 1992 & 11.88 & -0.06 & 1188.00 & 11126.54 & 1.00 & 0.94 & 0.94 \\
19427 & 102601 & 1992 & 1786.00 & 0.14 & 178603.00 & 1646767.03 & 1.00 & 0.92 & 0.92 \\
14699 & 101911 & 1992 & 62.18 & 0.07 & 6218.00 & 54049.56 & 1.00 & 0.87 & 0.87 \\
24968 & 103402 & 1992 & 117.21 & 0.08 & 11720.00 & 111858.87 & 1.00 & 0.95 & 0.95 \\
1918 & 100250 & 1992 & 67.23 & 0.03 & 6759.00 & 56548.58 & 0.99 & 0.84 & 0.84 \\
17222 & 102271 & 1992 & 1173.98 & -0.07 & 117395.00 & 1167581.32 & 1.00 & 0.99 & 0.99 \\
65077 & 500660 & 1992 & 128.67 & -0.51 & 12870.00 & 106901.09 & 1.00 & 0.83 & 0.83 \\
10138 & 101263 & 1992 & 290.37 & -0.00 & 29037.00 & 253266.34 & 1.00 & 0.87 & 0.87 \\
2469 & 100333 & 1992 & 96.87 & 0.10 & 9687.00 & 80253.82 & 1.00 & 0.83 & 0.83 \\
11211 & 101376 & 1992 & 212.95 & 0.05 & 21295.00 & 207483.37 & 1.00 & 0.97 & 0.97 \\
1922 & 100251 & 1992 & 42.47 & 0.03 & 4385.00 & 39387.55 & 0.97 & 0.93 & 0.90 \\
16227 & 102102 & 1992 & 166.00 & 0.09 & 16570.00 & 150806.74 & 1.00 & 0.91 & 0.91 \\
19393 & 102600 & 1992 & 52.00 & 0.21 & 5271.00 & 45637.77 & 0.99 & 0.88 & 0.87 \\
2433 & 100330 & 1992 & 47.96 & 0.14 & 4795.00 & 39424.81 & 1.00 & 0.82 & 0.82 \\
19321 & 102594 & 1992 & 24.20 & 0.04 & 2262.00 & 24226.47 & 1.07 & 1.00 & 1.07 \\
26754 & 103606 & 1992 & 33.44 & 0.05 & 3232.00 & 31096.69 & 1.03 & 0.93 & 0.96 \\
19534 & 102612 & 1992 & 95.98 & 0.01 & 5159.00 & 48081.67 & 1.86 & 0.50 & 0.93 \\
8706 & 101095 & 1992 & 22.50 & 0.02 & 2275.00 & 19379.88 & 0.99 & 0.86 & 0.85 \\
22369 & 103008 & 1992 & 64.22 & 0.06 & 6422.00 & 59727.68 & 1.00 & 0.93 & 0.93 \\
26750 & 103605 & 1992 & 92.02 & 0.10 & 8969.00 & 89664.43 & 1.03 & 0.97 & 1.00 \\
20647 & 102777 & 1992 & 322.02 & 0.13 & 26734.00 & 297723.42 & 1.20 & 0.92 & 1.11 \\
9405 & 101134 & 1992 & 55.42 & -0.08 & 5542.00 & 52469.53 & 1.00 & 0.95 & 0.95 \\
6969 & 100977 & 1992 & 29.15 & 0.03 & 2657.00 & 21277.65 & 1.10 & 0.73 & 0.80 \\
13740 & 101762 & 1992 & 313.15 & 0.11 & 30517.00 & 250685.16 & 1.03 & 0.80 & 0.82 \\
13219 & 101708 & 1992 & 90.94 & 0.09 & 9094.00 & 87574.61 & 1.00 & 0.96 & 0.96 \\
17244 & 102274 & 1992 & 87.42 & 0.16 & 8751.00 & 82424.93 & 1.00 & 0.94 & 0.94 \\
19541 & 102614 & 1992 & 222.42 & 0.03 & 19961.00 & 210897.83 & 1.11 & 0.95 & 1.06 \\
15565 & 102005 & 1992 & 696.50 & 0.08 & 67304.00 & 585015.03 & 1.03 & 0.84 & 0.87 \\
14440 & 101858 & 1992 & 103.50 & 0.05 & 9872.00 & 103709.20 & 1.05 & 1.00 & 1.05 \\
699 & 100091 & 1992 & 5.49 & 0.07 & 483.00 & 5167.69 & 1.14 & 0.94 & 1.07 \\
22936 & 103089 & 1992 & 85.73 & 0.00 & 8299.00 & 75056.15 & 1.03 & 0.88 & 0.90 \\
16249 & 102105 & 1992 & 43.00 & 0.10 & 4286.00 & 38863.88 & 1.00 & 0.90 & 0.91 \\
14733 & 101912 & 1992 & 243.68 & 0.20 & 24368.00 & 241838.98 & 1.00 & 0.99 & 0.99 \\
15088 & 101956 & 1992 & 185.62 & 0.11 & 12871.00 & 150981.86 & 1.44 & 0.81 & 1.17 \\
96655 & 611002 & 1992 & 1586.96 & 0.07 & 158300.00 & 1618000.14 & 1.00 & 1.02 & 1.02 \\
8868 & 101103 & 1992 & 78.40 & 0.06 & 7034.00 & 68459.64 & 1.11 & 0.87 & 0.97 \\
537 & 100075 & 1992 & 38.23 & 0.21 & 3823.00 & 30768.45 & 1.00 & 0.80 & 0.80 \\
15521 & 102000 & 1992 & 514.12 & 0.18 & 51320.00 & 501192.75 & 1.00 & 0.97 & 0.98 \\
16265 & 102113 & 1992 & 262.98 & 0.06 & 26295.00 & 264257.88 & 1.00 & 1.00 & 1.00 \\
11188 & 101370 & 1992 & 18.95 & 0.05 & 1895.00 & 17116.77 & 1.00 & 0.90 & 0.90 \\
9915 & 101212 & 1992 & 307.64 & 0.12 & 30760.00 & 247927.79 & 1.00 & 0.81 & 0.81 \\
17184 & 102270 & 1993 & 97.79 & -0.00 & 9296.00 & 88336.71 & 1.05 & 0.90 & 0.95 \\
13610 & 101748 & 1993 & 9.11 & -0.07 & 904.00 & 8923.04 & 1.01 & 0.98 & 0.99 \\
48340 & 240065 & 1993 & 168.00 & 0.10 & 14358.00 & 143275.79 & 1.17 & 0.85 & 1.00 \\
6586 & 100900 & 1993 & 51.27 & 0.10 & 5127.00 & 44292.00 & 1.00 & 0.86 & 0.86 \\
12109 & 101507 & 1993 & 60.37 & 0.04 & 6294.00 & 59087.48 & 0.96 & 0.98 & 0.94 \\
5395 & 100760 & 1993 & 1016.84 & 0.12 & 111010.00 & 941397.88 & 0.92 & 0.93 & 0.85 \\
454 & 100056 & 1993 & 3.76 & -0.00 & 421.00 & 4166.31 & 0.89 & 1.11 & 0.99 \\
17285 & 102278 & 1993 & 191.98 & 0.17 & 17081.00 & 183642.22 & 1.12 & 0.96 & 1.08 \\
7385 & 101038 & 1993 & 3153.80 & 0.26 & 297322.00 & 2693752.43 & 1.06 & 0.85 & 0.91 \\
47404 & 210770 & 1993 & 97.00 & 0.02 & 9700.00 & 95283.07 & 1.00 & 0.98 & 0.98 \\
5241 & 100741 & 1993 & 131.51 & 0.17 & 13755.00 & 129618.54 & 0.96 & 0.99 & 0.94 \\
6498 & 100878 & 1993 & 937.56 & 0.09 & 81741.00 & 856296.39 & 1.15 & 0.91 & 1.05 \\
283 & 100033 & 1993 & 28.84 & 0.20 & 2882.00 & 24692.49 & 1.00 & 0.86 & 0.86 \\
15089 & 101956 & 1993 & 343.53 & 0.14 & 29768.00 & 261141.25 & 1.15 & 0.76 & 0.88 \\
45874 & 200153 & 1993 & 3.10 & -0.02 & 441.00 & 4285.77 & 0.70 & 1.38 & 0.97 \\
5318 & 100753 & 1993 & 1508.18 & 0.15 & 150820.00 & 1262958.88 & 1.00 & 0.84 & 0.84 \\
5284 & 100746 & 1993 & 344.97 & 0.13 & 31073.00 & 318108.62 & 1.11 & 0.92 & 1.02 \\
17131 & 102258 & 1993 & 682.70 & 0.07 & 67027.00 & 666536.88 & 1.02 & 0.98 & 0.99 \\
6534 & 100889 & 1993 & 34.73 & 0.17 & 3492.00 & 32892.67 & 0.99 & 0.95 & 0.94 \\
12058 & 101494 & 1993 & 164.43 & 0.13 & 16443.00 & 162278.83 & 1.00 & 0.99 & 0.99 \\
455 & 100068 & 1993 & 23.34 & 0.11 & 2109.00 & 19442.06 & 1.11 & 0.83 & 0.92 \\
5339 & 100754 & 1993 & 436.23 & 0.15 & 43620.00 & 394703.69 & 1.00 & 0.90 & 0.90 \\
15107 & 101958 & 1993 & 62.42 & 0.09 & 6292.00 & 59457.50 & 0.99 & 0.95 & 0.94 \\
5231 & 100740 & 1993 & 139.72 & -0.04 & 15322.00 & 147311.47 & 0.91 & 1.05 & 0.96 \\
57895 & 401372 & 1993 & 13.44 & 0.18 & 1148.00 & 10610.23 & 1.17 & 0.79 & 0.92 \\
13817 & 101767 & 1993 & 35.37 & -0.03 & 3536.00 & 29524.94 & 1.00 & 0.83 & 0.83 \\
12118 & 101511 & 1993 & 192.83 & 0.18 & 20832.00 & 201037.33 & 0.93 & 1.04 & 0.97 \\
6543 & 100890 & 1993 & 249.25 & 0.10 & 25772.00 & 254727.65 & 0.97 & 1.02 & 0.99 \\
13654 & 101754 & 1993 & 38.05 & -0.02 & 3362.00 & 32050.16 & 1.13 & 0.84 & 0.95 \\
14978 & 101926 & 1993 & 52.00 & 0.14 & 3950.00 & 44237.00 & 1.32 & 0.85 & 1.12 \\
26238 & 103547 & 1993 & 1847.21 & -0.01 & 184721.00 & 1730241.75 & 1.00 & 0.94 & 0.94 \\
928 & 100112 & 1993 & 536.98 & 0.10 & 53698.00 & 440982.24 & 1.00 & 0.82 & 0.82 \\
14946 & 101925 & 1993 & 202.52 & 0.14 & 19055.00 & 179773.97 & 1.06 & 0.89 & 0.94 \\
15038 & 101953 & 1993 & 199.22 & 0.09 & 19725.00 & 194485.32 & 1.01 & 0.98 & 0.99 \\
26280 & 103558 & 1993 & 122.71 & 0.03 & 10396.00 & 98108.13 & 1.18 & 0.80 & 0.94 \\
17155 & 102261 & 1993 & 769.90 & 0.16 & 75324.00 & 730734.80 & 1.02 & 0.95 & 0.97 \\
48250 & 240057 & 1993 & 4.89 & 0.06 & 488.00 & 4241.04 & 1.00 & 0.87 & 0.87 \\
12043 & 101491 & 1993 & 77.60 & 0.21 & 7756.00 & 73414.30 & 1.00 & 0.95 & 0.95 \\
13546 & 101743 & 1993 & 3575.88 & 0.03 & 374880.00 & 2918130.38 & 0.95 & 0.82 & 0.78 \\
5208 & 100736 & 1993 & 281.93 & 0.08 & 20898.00 & 214994.75 & 1.35 & 0.76 & 1.03 \\
7696 & 101055 & 1993 & 2542.70 & 0.16 & 242210.00 & 2165518.65 & 1.05 & 0.85 & 0.89 \\
57926 & 410003 & 1993 & 216.15 & 0.16 & 21615.00 & 208619.98 & 1.00 & 0.97 & 0.97 \\
8122 & 101076 & 1993 & 50.70 & -0.01 & 6591.00 & 56682.63 & 0.77 & 1.12 & 0.86 \\
972 & 100113 & 1993 & 848.19 & 0.14 & 86832.00 & 805809.77 & 0.98 & 0.95 & 0.93 \\
6630 & 100906 & 1993 & 591.54 & 0.16 & 58695.00 & 566366.28 & 1.01 & 0.96 & 0.96 \\
14991 & 101930 & 1993 & 529.20 & 0.09 & 52920.00 & 471363.99 & 1.00 & 0.89 & 0.89 \\
15002 & 101933 & 1993 & 49.93 & 0.03 & 4992.00 & 43723.31 & 1.00 & 0.88 & 0.88 \\
370 & 100044 & 1993 & 33.20 & 0.11 & 2954.00 & 28608.81 & 1.12 & 0.86 & 0.97 \\
12875 & 101603 & 1993 & 641.53 & 0.15 & 64150.00 & 522035.82 & 1.00 & 0.81 & 0.81 \\
8144 & 101077 & 1993 & 9.40 & 0.25 & 700.00 & 7027.49 & 1.34 & 0.75 & 1.00 \\
15018 & 101943 & 1993 & 37.90 & 0.23 & 4899.00 & 37210.10 & 0.77 & 0.98 & 0.76 \\
15139 & 101963 & 1993 & 310.17 & 0.14 & 24357.00 & 222065.73 & 1.27 & 0.72 & 0.91 \\
13797 & 101764 & 1993 & 189.03 & 0.04 & 18902.00 & 163746.95 & 1.00 & 0.87 & 0.87 \\
13833 & 101769 & 1993 & 1662.24 & 0.02 & 166224.00 & 1565175.48 & 1.00 & 0.94 & 0.94 \\
47373 & 210681 & 1993 & 6193.30 & 0.21 & 541300.00 & 4659567.14 & 1.14 & 0.75 & 0.86 \\
12908 & 101606 & 1993 & 984.91 & 0.09 & 98160.00 & 814038.41 & 1.00 & 0.83 & 0.83 \\
6575 & 100891 & 1993 & 143.80 & 0.04 & 14377.00 & 140510.05 & 1.00 & 0.98 & 0.98 \\
17033 & 102231 & 1993 & 593.79 & 0.06 & 61883.00 & 597254.41 & 0.96 & 1.01 & 0.97 \\
17245 & 102274 & 1993 & 124.98 & 0.14 & 12583.00 & 119675.18 & 0.99 & 0.96 & 0.95 \\
26577 & 103592 & 1993 & 1.57 & -0.03 & 157.00 & 1563.94 & 1.00 & 0.99 & 1.00 \\
922 & 100111 & 1993 & 59.70 & 0.07 & 5904.00 & 58673.06 & 1.01 & 0.98 & 0.99 \\
6251 & 100833 & 1993 & 463.00 & 0.05 & 52633.00 & 417727.46 & 0.88 & 0.90 & 0.79 \\
14700 & 101911 & 1993 & 85.10 & 0.20 & 8509.00 & 75908.05 & 1.00 & 0.89 & 0.89 \\
13214 & 101704 & 1993 & 997.30 & 0.05 & 99730.00 & 950548.07 & 1.00 & 0.95 & 0.95 \\
16219 & 102097 & 1993 & 14.50 & 0.03 & 1678.00 & 12812.85 & 0.86 & 0.88 & 0.76 \\
16228 & 102102 & 1993 & 316.70 & 0.23 & 31669.00 & 289616.09 & 1.00 & 0.91 & 0.91 \\
702 & 100092 & 1993 & 6.90 & 0.07 & 573.00 & 5730.38 & 1.20 & 0.83 & 1.00 \\
15522 & 102000 & 1993 & 525.36 & 0.10 & 52530.00 & 523473.41 & 1.00 & 1.00 & 1.00 \\
16250 & 102105 & 1993 & 61.23 & 0.13 & 6130.00 & 54794.68 & 1.00 & 0.89 & 0.89 \\
7937 & 101067 & 1993 & 26.20 & 0.13 & 2880.00 & 29649.19 & 0.91 & 1.13 & 1.03 \\
14734 & 101912 & 1993 & 397.30 & 0.22 & 39728.00 & 372902.80 & 1.00 & 0.94 & 0.94 \\
16194 & 102090 & 1993 & 53.95 & 0.06 & 5290.00 & 53251.35 & 1.02 & 0.99 & 1.01 \\
27085 & 103649 & 1993 & 2.48 & 0.11 & 137.00 & 1347.77 & 1.81 & 0.54 & 0.98 \\
16101 & 102080 & 1993 & 157.05 & 0.14 & 15705.00 & 127631.42 & 1.00 & 0.81 & 0.81 \\
12589 & 101557 & 1993 & 23.57 & 0.04 & 2394.00 & 19776.48 & 0.98 & 0.84 & 0.83 \\
12587 & 101556 & 1993 & 4.63 & 0.14 & 464.00 & 4291.71 & 1.00 & 0.93 & 0.92 \\
26787 & 103607 & 1993 & 320.66 & 0.18 & 29536.00 & 297882.62 & 1.09 & 0.93 & 1.01 \\
15566 & 102005 & 1993 & 710.62 & 0.16 & 70764.00 & 677319.68 & 1.00 & 0.95 & 0.96 \\
14441 & 101858 & 1993 & 113.81 & 0.17 & 11645.00 & 110045.85 & 0.98 & 0.97 & 0.95 \\
16163 & 102089 & 1993 & 241.11 & 0.08 & 25542.00 & 238517.23 & 0.94 & 0.99 & 0.93 \\
6224 & 100831 & 1993 & 312.55 & 0.07 & 30185.00 & 256446.16 & 1.04 & 0.82 & 0.85 \\
12730 & 101591 & 1993 & 32.24 & 0.22 & 6300.00 & 58778.35 & 0.51 & 1.82 & 0.93 \\
5935 & 100812 & 1993 & 485.12 & 0.02 & 52232.00 & 465006.70 & 0.93 & 0.96 & 0.89 \\
589 & 100079 & 1993 & 712.03 & 0.08 & 71203.00 & 671862.46 & 1.00 & 0.94 & 0.94 \\
26755 & 103606 & 1993 & 45.29 & 0.11 & 4260.00 & 40442.60 & 1.06 & 0.89 & 0.95 \\
27032 & 103645 & 1993 & 143.02 & -0.07 & 12980.00 & 127595.78 & 1.10 & 0.89 & 0.98 \\
5864 & 100809 & 1993 & 2850.47 & 0.06 & 290963.00 & 2464862.59 & 0.98 & 0.86 & 0.85 \\
26751 & 103605 & 1993 & 95.08 & 0.05 & 9607.00 & 97053.36 & 0.99 & 1.02 & 1.01 \\
14766 & 101913 & 1993 & 32.40 & 0.09 & 3375.00 & 34481.81 & 0.96 & 1.06 & 1.02 \\
16355 & 102130 & 1993 & 429.35 & 0.11 & 42934.00 & 356938.41 & 1.00 & 0.83 & 0.83 \\
14798 & 101914 & 1993 & 20.80 & 0.12 & 1658.00 & 16380.33 & 1.25 & 0.79 & 0.99 \\
7971 & 101068 & 1993 & 31705.10 & 0.25 & 3050980.00 & 27571048.57 & 1.04 & 0.87 & 0.90 \\
26643 & 103595 & 1993 & 98.95 & 0.04 & 9895.00 & 87396.98 & 1.00 & 0.88 & 0.88 \\
12511 & 101545 & 1993 & 37.65 & 0.06 & 4039.00 & 33540.09 & 0.93 & 0.89 & 0.83 \\
47913 & 222809 & 1993 & 202.98 & 0.04 & 20915.00 & 183751.74 & 0.97 & 0.91 & 0.88 \\
16457 & 102150 & 1993 & 524.31 & 0.00 & 52428.00 & 471971.27 & 1.00 & 0.90 & 0.90 \\
16488 & 102151 & 1993 & 9.84 & 0.05 & 788.00 & 7577.83 & 1.25 & 0.77 & 0.96 \\
5777 & 100792 & 1993 & 4.88 & 0.03 & 566.00 & 5842.23 & 0.86 & 1.20 & 1.03 \\
6964 & 100974 & 1993 & 4.77 & 0.07 & 478.00 & 4539.65 & 1.00 & 0.95 & 0.95 \\
26747 & 103604 & 1993 & 1.86 & 0.05 & 165.00 & 1568.14 & 1.13 & 0.84 & 0.95 \\
6970 & 100977 & 1993 & 22.00 & 0.08 & 2624.00 & 20574.27 & 0.84 & 0.94 & 0.78 \\
700 & 100091 & 1993 & 8.10 & 0.08 & 1116.00 & 7412.80 & 0.73 & 0.92 & 0.66 \\
5848 & 100808 & 1993 & 103.39 & 0.10 & 7933.00 & 77364.99 & 1.30 & 0.75 & 0.98 \\
15472 & 101998 & 1993 & 203.26 & 0.21 & 20326.00 & 195601.25 & 1.00 & 0.96 & 0.96 \\
6762 & 100953 & 1993 & 4.50 & 0.06 & 461.00 & 4587.51 & 0.98 & 1.02 & 1.00 \\
49056 & 240212 & 1993 & 820.78 & -0.00 & 78701.00 & 751799.86 & 1.04 & 0.92 & 0.96 \\
16335 & 102127 & 1993 & 9.27 & 0.02 & 901.00 & 8751.78 & 1.03 & 0.94 & 0.97 \\
15491 & 101999 & 1993 & 986.98 & 0.14 & 98698.00 & 867076.61 & 1.00 & 0.88 & 0.88 \\
16340 & 102129 & 1993 & 7.62 & -0.01 & 828.00 & 8533.18 & 0.92 & 1.12 & 1.03 \\
5817 & 100804 & 1993 & 237.02 & 0.01 & 24105.00 & 224028.13 & 0.98 & 0.95 & 0.93 \\
16301 & 102124 & 1993 & 728.06 & 0.07 & 72806.00 & 715863.82 & 1.00 & 0.98 & 0.98 \\
26814 & 103608 & 1993 & 33.54 & 0.12 & 3195.00 & 31254.11 & 1.05 & 0.93 & 0.98 \\
7902 & 101065 & 1993 & 557.20 & 0.01 & 50799.00 & 526194.66 & 1.10 & 0.94 & 1.04 \\
6864 & 100966 & 1993 & 18.88 & 0.16 & 1652.00 & 16023.30 & 1.14 & 0.85 & 0.97 \\
15907 & 102059 & 1993 & 186.72 & 0.16 & 18698.00 & 171134.04 & 1.00 & 0.92 & 0.92 \\
6813 & 100961 & 1993 & 14.45 & 0.02 & 1781.00 & 17764.19 & 0.81 & 1.23 & 1.00 \\
26973 & 103642 & 1993 & 68.19 & 0.13 & 6819.00 & 65293.00 & 1.00 & 0.96 & 0.96 \\
6077 & 100822 & 1993 & 8.08 & 0.16 & 753.00 & 8059.87 & 1.07 & 1.00 & 1.07 \\
26980 & 103643 & 1993 & 88.36 & 0.01 & 8836.00 & 91194.76 & 1.00 & 1.03 & 1.03 \\
13028 & 101622 & 1993 & 672.27 & 0.08 & 123700.00 & 1229050.32 & 0.54 & 1.83 & 0.99 \\
12706 & 101589 & 1993 & 43.93 & 0.23 & 4390.00 & 39839.12 & 1.00 & 0.91 & 0.91 \\
15692 & 102015 & 1993 & 376.91 & 0.04 & 36916.00 & 338278.75 & 1.02 & 0.90 & 0.92 \\
12664 & 101562 & 1993 & 52.80 & 0.19 & 3936.00 & 33918.24 & 1.34 & 0.64 & 0.86 \\
736 & 100093 & 1993 & 293.90 & 0.13 & 26822.00 & 268158.32 & 1.10 & 0.91 & 1.00 \\
6108 & 100823 & 1993 & 19.82 & 0.11 & 1515.00 & 14854.78 & 1.31 & 0.75 & 0.98 \\
12695 & 101567 & 1993 & 5.45 & 0.01 & 543.00 & 5154.69 & 1.00 & 0.95 & 0.95 \\
12696 & 101568 & 1993 & 5.29 & 0.01 & 530.00 & 4990.05 & 1.00 & 0.94 & 0.94 \\
6828 & 100962 & 1993 & 847.68 & 0.15 & 77162.00 & 636349.41 & 1.10 & 0.75 & 0.82 \\
15751 & 102017 & 1993 & 1640.43 & 0.10 & 162310.00 & 1340845.72 & 1.01 & 0.82 & 0.83 \\
7603 & 101048 & 1993 & 732.80 & 0.16 & 38881.00 & 349591.28 & 1.88 & 0.48 & 0.90 \\
12698 & 101588 & 1993 & 227.28 & 0.15 & 22730.00 & 203023.76 & 1.00 & 0.89 & 0.89 \\
15717 & 102016 & 1993 & 1135.12 & 0.14 & 113426.00 & 994291.29 & 1.00 & 0.88 & 0.88 \\
15797 & 102026 & 1993 & 30.02 & 0.49 & 2598.00 & 23825.69 & 1.16 & 0.79 & 0.92 \\
332 & 100039 & 1993 & 126.43 & 0.05 & 13953.00 & 115297.63 & 0.91 & 0.91 & 0.83 \\
15838 & 102043 & 1993 & 95.50 & 0.09 & 9416.00 & 85539.91 & 1.01 & 0.90 & 0.91 \\
15782 & 102018 & 1993 & 456.38 & 0.05 & 45428.00 & 441838.60 & 1.00 & 0.97 & 0.97 \\
26955 & 103638 & 1993 & 46.36 & 0.02 & 3465.00 & 32882.44 & 1.34 & 0.71 & 0.95 \\
6904 & 100968 & 1993 & 17.40 & 0.08 & 1546.00 & 12695.97 & 1.13 & 0.73 & 0.82 \\
15632 & 102010 & 1993 & 1089.12 & 0.26 & 97247.00 & 1001640.38 & 1.12 & 0.92 & 1.03 \\
14501 & 101867 & 1993 & 151.77 & 0.07 & 14281.00 & 127251.23 & 1.06 & 0.84 & 0.89 \\
13162 & 101698 & 1993 & 108.75 & 0.10 & 10875.00 & 104645.55 & 1.00 & 0.96 & 0.96 \\
47791 & 221485 & 1993 & 28.08 & 0.02 & 2808.00 & 27539.84 & 1.00 & 0.98 & 0.98 \\
14506 & 101871 & 1993 & 147.11 & 0.16 & 13480.00 & 117226.59 & 1.09 & 0.80 & 0.87 \\
26885 & 103620 & 1993 & 53.88 & 0.16 & 4762.00 & 45199.37 & 1.13 & 0.84 & 0.95 \\
14462 & 101861 & 1993 & 386.42 & 0.15 & 39286.00 & 367641.70 & 0.98 & 0.95 & 0.94 \\
27006 & 103644 & 1993 & 175.34 & 0.02 & 17534.00 & 165939.47 & 1.00 & 0.95 & 0.95 \\
14666 & 101908 & 1993 & 6.50 & 0.07 & 689.00 & 6872.70 & 0.94 & 1.06 & 1.00 \\
6784 & 100954 & 1993 & 77.99 & 0.14 & 7706.00 & 72243.96 & 1.01 & 0.93 & 0.94 \\
15581 & 102007 & 1993 & 298.35 & 0.22 & 29446.00 & 257264.90 & 1.01 & 0.86 & 0.87 \\
6190 & 100829 & 1993 & 338.97 & 0.17 & 27320.00 & 230272.06 & 1.24 & 0.68 & 0.84 \\
13119 & 101668 & 1993 & 38.30 & 0.20 & 3846.00 & 32826.44 & 1.00 & 0.86 & 0.85 \\
14548 & 101876 & 1993 & 134.49 & 0.13 & 13449.00 & 129855.35 & 1.00 & 0.97 & 0.97 \\
74554 & 601136 & 1993 & 19.75 & 0.10 & 1975.00 & 19608.01 & 1.00 & 0.99 & 0.99 \\
6046 & 100821 & 1993 & 13.04 & 0.13 & 812.00 & 7645.78 & 1.61 & 0.59 & 0.94 \\
12713 & 101590 & 1993 & 11.05 & 0.02 & 1100.00 & 11112.96 & 1.00 & 1.01 & 1.01 \\
564 & 100076 & 1993 & 157.76 & 0.04 & 15776.00 & 153726.71 & 1.00 & 0.97 & 0.97 \\
49004 & 240199 & 1993 & 15.55 & 0.06 & 2102.00 & 20269.68 & 0.74 & 1.30 & 0.96 \\
26921 & 103628 & 1993 & 346.07 & 0.24 & 25302.00 & 263275.76 & 1.37 & 0.76 & 1.04 \\
74785 & 601171 & 1993 & 76.47 & 0.09 & 8382.00 & 83606.08 & 0.91 & 1.09 & 1.00 \\
6015 & 100820 & 1993 & 8.37 & -0.01 & 806.00 & 8270.56 & 1.04 & 0.99 & 1.03 \\
26907 & 103621 & 1993 & 50.50 & 0.14 & 4944.00 & 49594.84 & 1.02 & 0.98 & 1.00 \\
16510 & 102152 & 1993 & 113.60 & 0.03 & 10869.00 & 98969.67 & 1.05 & 0.87 & 0.91 \\
14824 & 101916 & 1993 & 67.80 & 0.25 & 4892.00 & 47479.61 & 1.39 & 0.70 & 0.97 \\
5745 & 100791 & 1993 & 28.41 & 0.09 & 2669.00 & 27987.36 & 1.06 & 0.99 & 1.05 \\
6680 & 100910 & 1993 & 49.90 & 0.13 & 2825.00 & 24463.44 & 1.77 & 0.49 & 0.87 \\
13407 & 101738 & 1993 & 258.57 & 0.16 & 23933.00 & 195553.81 & 1.08 & 0.76 & 0.82 \\
13971 & 101794 & 1993 & 101.17 & 0.19 & 9401.00 & 87655.29 & 1.08 & 0.87 & 0.93 \\
57834 & 401082 & 1993 & 18.04 & 0.00 & 2509.00 & 17661.47 & 0.72 & 0.98 & 0.70 \\
5530 & 100771 & 1993 & 24.00 & 0.04 & 2200.00 & 20504.77 & 1.09 & 0.85 & 0.93 \\
15276 & 101977 & 1993 & 32.39 & 0.01 & 3140.00 & 28541.94 & 1.03 & 0.88 & 0.91 \\
26426 & 103580 & 1993 & 47.58 & 0.14 & 4472.00 & 44662.34 & 1.06 & 0.94 & 1.00 \\
48173 & 240040 & 1993 & 56.80 & 0.38 & 3208.00 & 27953.80 & 1.77 & 0.49 & 0.87 \\
26394 & 103579 & 1993 & 33.18 & 0.24 & 1732.00 & 16668.27 & 1.92 & 0.50 & 0.96 \\
5499 & 100769 & 1993 & 101.00 & 0.02 & 10100.00 & 100138.02 & 1.00 & 0.99 & 0.99 \\
47501 & 212408 & 1993 & 46.66 & 0.14 & 4128.00 & 43438.75 & 1.13 & 0.93 & 1.05 \\
13375 & 101736 & 1993 & 2.87 & 0.02 & 234.00 & 2296.89 & 1.23 & 0.80 & 0.98 \\
6423 & 100868 & 1993 & 87.40 & 0.09 & 9266.00 & 97093.36 & 0.94 & 1.11 & 1.05 \\
13372 & 101733 & 1993 & 32.26 & 0.09 & 3166.00 & 31523.83 & 1.02 & 0.98 & 1.00 \\
14015 & 101800 & 1993 & 393.75 & 0.13 & 40797.00 & 347912.91 & 0.97 & 0.88 & 0.85 \\
14868 & 101919 & 1993 & 137.63 & 0.13 & 15232.00 & 135068.25 & 0.90 & 0.98 & 0.89 \\
26477 & 103582 & 1993 & 11.60 & 0.13 & 994.00 & 8757.48 & 1.17 & 0.75 & 0.88 \\
6402 & 100864 & 1993 & 644.12 & 0.10 & 63657.00 & 640988.00 & 1.01 & 1.00 & 1.01 \\
47245 & 200344 & 1993 & 994.72 & 0.09 & 99526.00 & 881191.19 & 1.00 & 0.89 & 0.89 \\
47532 & 212658 & 1993 & 65.43 & 0.26 & 6543.00 & 59889.88 & 1.00 & 0.92 & 0.92 \\
7158 & 101000 & 1993 & 565.90 & 0.12 & 56442.00 & 487814.15 & 1.00 & 0.86 & 0.86 \\
15294 & 101982 & 1993 & 265.20 & 0.03 & 26505.00 & 250784.39 & 1.00 & 0.95 & 0.95 \\
12818 & 101601 & 1993 & 525.64 & 0.10 & 50110.00 & 403068.30 & 1.05 & 0.77 & 0.80 \\
15290 & 101980 & 1993 & 17.66 & 0.05 & 1329.00 & 13655.66 & 1.33 & 0.77 & 1.03 \\
12297 & 101532 & 1993 & 114.66 & -0.02 & 12437.00 & 112423.86 & 0.92 & 0.98 & 0.90 \\
15281 & 101978 & 1993 & 77.72 & 0.11 & 7283.00 & 66340.83 & 1.07 & 0.85 & 0.91 \\
26379 & 103573 & 1993 & 37.60 & 0.03 & 3422.00 & 34766.05 & 1.10 & 0.92 & 1.02 \\
15245 & 101972 & 1993 & 98.90 & 0.17 & 8682.00 & 88159.58 & 1.14 & 0.89 & 1.02 \\
13877 & 101785 & 1993 & 622.84 & 0.12 & 62284.00 & 532737.25 & 1.00 & 0.86 & 0.86 \\
15208 & 101967 & 1993 & 4.00 & 0.07 & 470.00 & 3818.93 & 0.85 & 0.95 & 0.81 \\
5439 & 100763 & 1993 & 283.46 & 0.22 & 28350.00 & 241661.59 & 1.00 & 0.85 & 0.85 \\
650 & 100087 & 1993 & 4712.36 & 0.02 & 475100.00 & 4151454.34 & 0.99 & 0.88 & 0.87 \\
26335 & 103570 & 1993 & 11.77 & 0.02 & 1176.00 & 9568.07 & 1.00 & 0.81 & 0.81 \\
383 & 100048 & 1993 & 60.13 & 0.07 & 4384.00 & 37078.58 & 1.37 & 0.62 & 0.85 \\
12183 & 101518 & 1993 & 47.24 & 0.18 & 3887.00 & 40014.15 & 1.22 & 0.85 & 1.03 \\
16961 & 102224 & 1993 & 488.91 & 0.19 & 48891.00 & 480539.49 & 1.00 & 0.98 & 0.98 \\
13860 & 101781 & 1993 & 487.52 & 0.17 & 48752.00 & 442488.85 & 1.00 & 0.91 & 0.91 \\
12152 & 101513 & 1993 & 59.97 & 0.08 & 5988.00 & 57961.06 & 1.00 & 0.97 & 0.97 \\
15210 & 101968 & 1993 & 69.10 & 0.17 & 5807.00 & 56640.29 & 1.19 & 0.82 & 0.98 \\
7423 & 101039 & 1993 & 1260.10 & 0.27 & 114301.00 & 1109635.36 & 1.10 & 0.88 & 0.97 \\
7194 & 101013 & 1993 & 2340.60 & 0.22 & 186157.00 & 1531987.75 & 1.26 & 0.65 & 0.82 \\
16870 & 102213 & 1993 & 134.70 & 0.09 & 13455.00 & 118968.09 & 1.00 & 0.88 & 0.88 \\
13948 & 101789 & 1993 & 183.36 & 0.01 & 15185.00 & 141097.64 & 1.21 & 0.77 & 0.93 \\
6449 & 100875 & 1993 & 197.11 & 0.17 & 18933.00 & 189342.68 & 1.04 & 0.96 & 1.00 \\
14903 & 101921 & 1993 & 33.89 & 0.14 & 3040.00 & 28466.86 & 1.11 & 0.84 & 0.94 \\
12208 & 101519 & 1993 & 175.15 & 0.04 & 19690.00 & 186049.22 & 0.89 & 1.06 & 0.94 \\
26368 & 103572 & 1993 & 32.47 & -0.05 & 2932.00 & 29320.70 & 1.11 & 0.90 & 1.00 \\
47329 & 210203 & 1993 & 149.94 & 0.12 & 14533.00 & 142826.83 & 1.03 & 0.95 & 0.98 \\
13929 & 101788 & 1993 & 235.10 & 0.01 & 21414.00 & 207888.55 & 1.10 & 0.88 & 0.97 \\
6663 & 100908 & 1993 & 103.20 & 0.06 & 10017.00 & 93543.00 & 1.03 & 0.91 & 0.93 \\
15229 & 101970 & 1993 & 40.90 & 0.12 & 3810.00 & 38177.62 & 1.07 & 0.93 & 1.00 \\
13910 & 101787 & 1993 & 257.22 & 0.03 & 19217.00 & 176974.54 & 1.34 & 0.69 & 0.92 \\
48205 & 240051 & 1993 & 108.31 & 0.24 & 10734.00 & 104747.78 & 1.01 & 0.97 & 0.98 \\
16768 & 102191 & 1993 & 45.20 & 0.21 & 4535.00 & 37796.33 & 1.00 & 0.84 & 0.83 \\
15168 & 101964 & 1993 & 39.08 & 0.13 & 3908.00 & 37887.56 & 1.00 & 0.97 & 0.97 \\
57804 & 401068 & 1993 & 13.77 & 0.06 & 1334.00 & 10792.25 & 1.03 & 0.78 & 0.81 \\
49099 & 240222 & 1993 & 368.70 & 0.14 & 37600.00 & 351541.61 & 0.98 & 0.95 & 0.93 \\
14190 & 101820 & 1993 & 200.90 & 0.12 & 19533.00 & 183801.60 & 1.03 & 0.91 & 0.94 \\
8003 & 101069 & 1993 & 299.90 & 0.10 & 30490.00 & 316760.15 & 0.98 & 1.06 & 1.04 \\
14158 & 101819 & 1993 & 116.36 & 0.13 & 11661.00 & 104547.18 & 1.00 & 0.90 & 0.90 \\
14221 & 101834 & 1993 & 16.17 & 0.04 & 1620.00 & 15816.96 & 1.00 & 0.98 & 0.98 \\
6324 & 100849 & 1993 & 79.92 & 0.01 & 7992.00 & 76650.83 & 1.00 & 0.96 & 0.96 \\
12437 & 101541 & 1993 & 77.39 & 0.18 & 7520.00 & 64103.29 & 1.03 & 0.83 & 0.85 \\
12403 & 101539 & 1993 & 134.03 & 0.05 & 13460.00 & 124272.24 & 1.00 & 0.93 & 0.92 \\
7076 & 100996 & 1993 & 415.10 & 0.09 & 41480.00 & 352596.58 & 1.00 & 0.85 & 0.85 \\
14144 & 101807 & 1993 & 113.84 & 0.02 & 13035.00 & 95885.13 & 0.87 & 0.84 & 0.74 \\
6721 & 100921 & 1993 & 81.66 & 0.18 & 9875.00 & 102021.20 & 0.83 & 1.25 & 1.03 \\
678 & 100090 & 1993 & 41.10 & 0.20 & 3561.00 & 35614.15 & 1.15 & 0.87 & 1.00 \\
14236 & 101835 & 1993 & 118.65 & 0.01 & 11870.00 & 117873.47 & 1.00 & 0.99 & 0.99 \\
16533 & 102154 & 1993 & 135.40 & 0.11 & 12657.00 & 102444.29 & 1.07 & 0.76 & 0.81 \\
7520 & 101043 & 1993 & 652.60 & 0.20 & 64390.00 & 525294.69 & 1.01 & 0.80 & 0.82 \\
14269 & 101842 & 1993 & 201.92 & 0.12 & 20004.00 & 197049.37 & 1.01 & 0.98 & 0.99 \\
26611 & 103593 & 1993 & 12638.51 & 0.13 & 1263851.00 & 11794063.17 & 1.00 & 0.93 & 0.93 \\
12769 & 101594 & 1993 & 344.14 & -0.01 & 35460.00 & 305589.88 & 0.97 & 0.89 & 0.86 \\
410 & 100055 & 1993 & 4156.59 & 0.13 & 382532.00 & 3414821.42 & 1.09 & 0.82 & 0.89 \\
7794 & 101061 & 1993 & 276.00 & 0.05 & 20527.00 & 205349.01 & 1.34 & 0.74 & 1.00 \\
15466 & 101996 & 1993 & 244.43 & 0.06 & 24443.00 & 224044.43 & 1.00 & 0.92 & 0.92 \\
15452 & 101992 & 1993 & 1162.37 & 0.07 & 116359.00 & 1077031.53 & 1.00 & 0.93 & 0.93 \\
6741 & 100947 & 1993 & 629.74 & 0.06 & 59127.00 & 597912.45 & 1.07 & 0.95 & 1.01 \\
5721 & 100790 & 1993 & 17.16 & 0.09 & 1622.00 & 16563.62 & 1.06 & 0.97 & 1.02 \\
7039 & 100992 & 1993 & 461.30 & 0.07 & 45042.00 & 372808.98 & 1.02 & 0.81 & 0.83 \\
15470 & 101997 & 1993 & 11.41 & 0.12 & 1142.00 & 11019.19 & 1.00 & 0.97 & 0.96 \\
12319 & 101536 & 1993 & 329.52 & 0.13 & 33874.00 & 305235.15 & 0.97 & 0.93 & 0.90 \\
14090 & 101802 & 1993 & 221.16 & 0.10 & 23238.00 & 208322.95 & 0.95 & 0.94 & 0.90 \\
5597 & 100773 & 1993 & 643.00 & 0.21 & 64300.00 & 627347.66 & 1.00 & 0.98 & 0.98 \\
6372 & 100856 & 1993 & 138.81 & -0.03 & 13881.00 & 123145.45 & 1.00 & 0.89 & 0.89 \\
12797 & 101600 & 1993 & 142.63 & 0.12 & 13357.00 & 139150.61 & 1.07 & 0.98 & 1.04 \\
12793 & 101596 & 1993 & 194.52 & 0.05 & 20148.00 & 174787.27 & 0.97 & 0.90 & 0.87 \\
26514 & 103590 & 1993 & 29.04 & 0.22 & 2904.00 & 27391.97 & 1.00 & 0.94 & 0.94 \\
15330 & 101987 & 1993 & 331.40 & 0.18 & 33140.00 & 322001.94 & 1.00 & 0.97 & 0.97 \\
7139 & 100998 & 1993 & 79.10 & 0.15 & 8899.00 & 73714.49 & 0.89 & 0.93 & 0.83 \\
7456 & 101040 & 1993 & 795.80 & 0.14 & 66420.00 & 646489.10 & 1.20 & 0.81 & 0.97 \\
12301 & 101534 & 1993 & 327.08 & 0.13 & 34740.00 & 315173.67 & 0.94 & 0.96 & 0.91 \\
13346 & 101729 & 1993 & 243.46 & 0.12 & 25969.00 & 232499.34 & 0.94 & 0.95 & 0.90 \\
57789 & 401061 & 1993 & 8.73 & -0.18 & 1550.00 & 8416.72 & 0.56 & 0.96 & 0.54 \\
74621 & 601142 & 1993 & 209.57 & 0.19 & 23631.00 & 203308.17 & 0.89 & 0.97 & 0.86 \\
12776 & 101595 & 1993 & 528.84 & 0.04 & 54856.00 & 436717.42 & 0.96 & 0.83 & 0.80 \\
14125 & 101805 & 1993 & 1076.96 & 0.25 & 112400.00 & 1091311.27 & 0.96 & 1.01 & 0.97 \\
49181 & 240243 & 1993 & 618.91 & 0.81 & 59966.00 & 555599.25 & 1.03 & 0.90 & 0.93 \\
5641 & 100784 & 1993 & 435.89 & 0.18 & 43589.00 & 411544.38 & 1.00 & 0.94 & 0.94 \\
47180 & 200342 & 1993 & 1716.10 & 0.23 & 153520.00 & 1356512.71 & 1.12 & 0.79 & 0.88 \\
517 & 100072 & 1993 & 2924.82 & 0.17 & 293815.00 & 2606219.76 & 1.00 & 0.89 & 0.89 \\
14108 & 101804 & 1993 & 273.94 & 0.09 & 29375.00 & 288431.11 & 0.93 & 1.05 & 0.98 \\
5633 & 100780 & 1993 & 38.21 & 0.10 & 3880.00 & 35898.42 & 0.98 & 0.94 & 0.93 \\
7107 & 100997 & 1993 & 20.40 & 0.11 & 1785.00 & 17852.83 & 1.14 & 0.88 & 1.00 \\
26546 & 103591 & 1993 & 16.17 & 0.11 & 1617.00 & 14700.73 & 1.00 & 0.91 & 0.91 \\
12352 & 101537 & 1993 & 214.16 & 0.11 & 23042.00 & 179616.50 & 0.93 & 0.84 & 0.78 \\
5616 & 100775 & 1993 & 288.98 & 0.02 & 25392.00 & 239561.24 & 1.14 & 0.83 & 0.94 \\
57785 & 401058 & 1993 & 33.74 & 0.15 & 3646.00 & 29338.60 & 0.93 & 0.87 & 0.80 \\
57805 & 401070 & 1993 & 25.10 & 0.05 & 2409.00 & 24987.54 & 1.04 & 1.00 & 1.04 \\
538 & 100075 & 1993 & 273.51 & 0.32 & 27351.00 & 220143.85 & 1.00 & 0.80 & 0.80 \\
5190 & 100731 & 1993 & 6549.64 & 0.14 & 621118.00 & 5128310.60 & 1.05 & 0.78 & 0.83 \\
2714 & 100355 & 1993 & 229.83 & 0.11 & 22728.00 & 227371.32 & 1.01 & 0.99 & 1.00 \\
53453 & 350572 & 1993 & 45.13 & 0.12 & 4744.00 & 47546.28 & 0.95 & 1.05 & 1.00 \\
19316 & 102591 & 1993 & 41.94 & 0.08 & 4006.00 & 36237.36 & 1.05 & 0.86 & 0.90 \\
1247 & 100167 & 1993 & 119.00 & 0.05 & 10536.00 & 104543.51 & 1.13 & 0.88 & 0.99 \\
2745 & 100357 & 1993 & 194.01 & 0.01 & 19484.00 & 175671.87 & 1.00 & 0.91 & 0.90 \\
11212 & 101376 & 1993 & 207.18 & 0.07 & 21517.00 & 219068.46 & 0.96 & 1.06 & 1.02 \\
8992 & 101108 & 1993 & 666.30 & 0.04 & 67424.00 & 546954.50 & 0.99 & 0.82 & 0.81 \\
22370 & 103008 & 1993 & 59.58 & 0.03 & 6138.00 & 62425.29 & 0.97 & 1.05 & 1.02 \\
24969 & 103402 & 1993 & 135.32 & 0.10 & 15610.00 & 148845.63 & 0.87 & 1.10 & 0.95 \\
96656 & 611002 & 1993 & 1962.35 & 0.09 & 196234.00 & 1901580.21 & 1.00 & 0.97 & 0.97 \\
19322 & 102594 & 1993 & 24.00 & 0.06 & 2301.00 & 22134.41 & 1.04 & 0.92 & 0.96 \\
22326 & 103007 & 1993 & 904.56 & 0.09 & 87664.00 & 837533.51 & 1.03 & 0.93 & 0.96 \\
8523 & 101089 & 1993 & 18.90 & 0.08 & 2637.00 & 24840.88 & 0.72 & 1.31 & 0.94 \\
19360 & 102599 & 1993 & 203.60 & 0.17 & 20364.00 & 201337.33 & 1.00 & 0.99 & 0.99 \\
19394 & 102600 & 1993 & 150.00 & 0.13 & 15005.00 & 142301.39 & 1.00 & 0.95 & 0.95 \\
19296 & 102588 & 1993 & 167.84 & 0.02 & 17737.00 & 146212.52 & 0.95 & 0.87 & 0.82 \\
10122 & 101262 & 1993 & 4.48 & 0.19 & 237.00 & 2398.77 & 1.89 & 0.54 & 1.01 \\
19196 & 102563 & 1993 & 449.07 & 0.10 & 45020.00 & 424578.22 & 1.00 & 0.95 & 0.94 \\
2662 & 100351 & 1993 & 202.78 & 0.03 & 19553.00 & 195502.83 & 1.04 & 0.96 & 1.00 \\
1677 & 100223 & 1993 & 248.30 & 0.12 & 24822.00 & 220705.51 & 1.00 & 0.89 & 0.89 \\
22593 & 103024 & 1993 & 117.70 & 0.08 & 11000.00 & 104956.83 & 1.07 & 0.89 & 0.95 \\
22562 & 103021 & 1993 & 37.63 & 0.15 & 3605.00 & 29616.79 & 1.04 & 0.79 & 0.82 \\
2682 & 100352 & 1993 & 96.29 & 0.12 & 8962.00 & 87784.53 & 1.07 & 0.91 & 0.98 \\
4123 & 100559 & 1993 & 14.89 & 0.15 & 1488.00 & 13262.09 & 1.00 & 0.89 & 0.89 \\
25578 & 103497 & 1993 & 11.91 & 0.08 & 1117.00 & 11175.40 & 1.07 & 0.94 & 1.00 \\
22521 & 103017 & 1993 & 309.94 & 0.10 & 23867.00 & 234001.95 & 1.30 & 0.75 & 0.98 \\
11277 & 101390 & 1993 & 1805.27 & 0.11 & 148400.00 & 1498496.74 & 1.22 & 0.83 & 1.01 \\
4206 & 100575 & 1993 & 20.70 & 0.12 & 2098.00 & 20990.65 & 0.99 & 1.01 & 1.00 \\
25547 & 103496 & 1993 & 302.35 & 0.16 & 27915.00 & 279172.82 & 1.08 & 0.92 & 1.00 \\
19263 & 102578 & 1993 & 85.55 & 0.01 & 16101.00 & 138517.40 & 0.53 & 1.62 & 0.86 \\
22494 & 103016 & 1993 & 37.47 & 0.13 & 3481.00 & 35257.28 & 1.08 & 0.94 & 1.01 \\
19275 & 102579 & 1993 & 36.35 & 0.06 & 3452.00 & 29554.26 & 1.05 & 0.81 & 0.86 \\
22461 & 103014 & 1993 & 89.34 & 0.16 & 6255.00 & 61574.16 & 1.43 & 0.69 & 0.98 \\
19428 & 102601 & 1993 & 1960.20 & 0.14 & 202998.00 & 1883129.89 & 0.97 & 0.96 & 0.93 \\
19597 & 102636 & 1993 & 356.24 & 0.08 & 34713.00 & 302979.53 & 1.03 & 0.85 & 0.87 \\
8560 & 101090 & 1993 & 61.50 & 0.29 & 4759.00 & 55329.29 & 1.29 & 0.90 & 1.16 \\
22135 & 102993 & 1993 & 451.85 & 0.05 & 44885.00 & 427193.03 & 1.01 & 0.95 & 0.95 \\
8947 & 101106 & 1993 & 23.30 & 0.13 & 3062.00 & 29274.14 & 0.76 & 1.26 & 0.96 \\
4042 & 100543 & 1993 & 523.77 & 0.29 & 52377.00 & 494646.43 & 1.00 & 0.94 & 0.94 \\
19629 & 102639 & 1993 & 110.03 & 0.20 & 13762.00 & 111136.33 & 0.80 & 1.01 & 0.81 \\
22101 & 102990 & 1993 & 192.66 & 0.02 & 20550.00 & 196805.85 & 0.94 & 1.02 & 0.96 \\
19650 & 102641 & 1993 & 143.49 & 0.16 & 11771.00 & 127522.06 & 1.22 & 0.89 & 1.08 \\
2874 & 100368 & 1993 & 59.20 & 0.02 & 6340.00 & 55149.51 & 0.93 & 0.93 & 0.87 \\
19663 & 102645 & 1993 & 123.43 & 0.09 & 12151.00 & 111633.04 & 1.02 & 0.90 & 0.92 \\
2883 & 100369 & 1993 & 152.30 & 0.01 & 15120.00 & 151381.11 & 1.01 & 0.99 & 1.00 \\
22040 & 102988 & 1993 & 76.42 & 0.04 & 6392.00 & 56847.45 & 1.20 & 0.74 & 0.89 \\
22009 & 102987 & 1993 & 224.19 & 0.13 & 20777.00 & 190346.95 & 1.08 & 0.85 & 0.92 \\
2892 & 100371 & 1993 & 19.90 & -0.01 & 2235.00 & 20441.38 & 0.89 & 1.03 & 0.91 \\
11081 & 101367 & 1993 & 214.97 & 0.23 & 20462.00 & 189642.45 & 1.05 & 0.88 & 0.93 \\
25490 & 103494 & 1993 & 331.10 & 0.07 & 28994.00 & 289947.72 & 1.14 & 0.88 & 1.00 \\
19587 & 102635 & 1993 & 165.27 & 0.02 & 16070.00 & 131465.33 & 1.03 & 0.80 & 0.82 \\
22269 & 102999 & 1993 & 49.11 & 0.26 & 4415.00 & 42363.46 & 1.11 & 0.86 & 0.96 \\
4101 & 100550 & 1993 & 9.46 & 0.02 & 1083.00 & 9739.74 & 0.87 & 1.03 & 0.90 \\
19462 & 102606 & 1993 & 4582.90 & 0.19 & 457811.00 & 3799389.56 & 1.00 & 0.83 & 0.83 \\
11197 & 101374 & 1993 & 10.46 & 0.02 & 1111.00 & 10703.20 & 0.94 & 1.02 & 0.96 \\
22238 & 102997 & 1993 & 100.82 & 0.35 & 8455.00 & 75461.76 & 1.19 & 0.75 & 0.89 \\
19485 & 102607 & 1993 & 1068.00 & 0.16 & 106791.00 & 1045729.92 & 1.00 & 0.98 & 0.98 \\
11189 & 101370 & 1993 & 19.06 & 0.09 & 1854.00 & 19386.82 & 1.03 & 1.02 & 1.05 \\
19532 & 102611 & 1993 & 19.00 & 0.01 & 1900.00 & 18533.89 & 1.00 & 0.98 & 0.98 \\
2822 & 100362 & 1993 & 47.30 & 0.23 & 4750.00 & 45485.93 & 1.00 & 0.96 & 0.96 \\
22210 & 102996 & 1993 & 198.96 & 0.05 & 21189.00 & 206787.10 & 0.94 & 1.04 & 0.98 \\
19535 & 102612 & 1993 & 91.48 & 0.06 & 9148.00 & 87243.42 & 1.00 & 0.95 & 0.95 \\
19542 & 102614 & 1993 & 245.80 & 0.05 & 24580.00 & 231656.65 & 1.00 & 0.94 & 0.94 \\
8961 & 101107 & 1993 & 1393.20 & 0.18 & 142740.00 & 1212666.83 & 0.98 & 0.87 & 0.85 \\
22179 & 102994 & 1993 & 53.25 & -0.02 & 5288.00 & 52482.85 & 1.01 & 0.99 & 0.99 \\
25029 & 103426 & 1993 & 218.14 & 0.11 & 15360.00 & 150745.80 & 1.42 & 0.69 & 0.98 \\
11151 & 101369 & 1993 & 695.92 & 0.11 & 71142.00 & 628585.52 & 0.98 & 0.90 & 0.88 \\
24986 & 103406 & 1993 & 1034.36 & 0.09 & 92382.00 & 851880.25 & 1.12 & 0.82 & 0.92 \\
4247 & 100598 & 1993 & 271.86 & 0.11 & 27571.00 & 272208.86 & 0.99 & 1.00 & 0.99 \\
19138 & 102551 & 1993 & 39.13 & 0.14 & 4632.00 & 45275.81 & 0.84 & 1.16 & 0.98 \\
18677 & 102502 & 1993 & 393.53 & 0.22 & 32056.00 & 366748.77 & 1.23 & 0.93 & 1.14 \\
9916 & 101212 & 1993 & 469.16 & 0.18 & 46900.00 & 385335.56 & 1.00 & 0.82 & 0.82 \\
52029 & 301299 & 1993 & 201.63 & 0.10 & 18770.00 & 162108.25 & 1.07 & 0.80 & 0.86 \\
4431 & 100625 & 1993 & 232.40 & 0.14 & 23227.00 & 227457.10 & 1.00 & 0.98 & 0.98 \\
22941 & 103090 & 1993 & 644.94 & 0.16 & 62176.00 & 523421.45 & 1.04 & 0.81 & 0.84 \\
22937 & 103089 & 1993 & 91.00 & 0.17 & 8754.00 & 82840.90 & 1.04 & 0.91 & 0.95 \\
2470 & 100333 & 1993 & 100.48 & 0.12 & 10048.00 & 86839.90 & 1.00 & 0.86 & 0.86 \\
9981 & 101216 & 1993 & 45.76 & 0.01 & 4461.00 & 44287.69 & 1.03 & 0.97 & 0.99 \\
18688 & 102503 & 1993 & 98.77 & 0.17 & 9877.00 & 96864.01 & 1.00 & 0.98 & 0.98 \\
1740 & 100227 & 1993 & 113.14 & 0.08 & 11298.00 & 106440.91 & 1.00 & 0.94 & 0.94 \\
22916 & 103085 & 1993 & 58.57 & 0.06 & 5097.00 & 50559.94 & 1.15 & 0.86 & 0.99 \\
18750 & 102507 & 1993 & 533.97 & -0.06 & 59577.00 & 611068.98 & 0.90 & 1.14 & 1.03 \\
2499 & 100336 & 1993 & 15.27 & 0.04 & 993.00 & 8470.43 & 1.54 & 0.55 & 0.85 \\
11450 & 101414 & 1993 & 15.20 & 0.01 & 1397.00 & 13466.61 & 1.09 & 0.89 & 0.96 \\
18768 & 102508 & 1993 & 134.86 & -0.01 & 16944.00 & 153242.88 & 0.80 & 1.14 & 0.90 \\
11427 & 101402 & 1993 & 36.20 & 0.20 & 3366.00 & 35572.88 & 1.08 & 0.98 & 1.06 \\
22873 & 103074 & 1993 & 85.70 & 0.01 & 6964.00 & 63018.85 & 1.23 & 0.74 & 0.90 \\
4393 & 100622 & 1993 & 25.10 & 0.19 & 2899.00 & 26993.58 & 0.87 & 1.08 & 0.93 \\
4446 & 100633 & 1993 & 100.72 & 0.16 & 10071.00 & 94693.23 & 1.00 & 0.94 & 0.94 \\
9853 & 101198 & 1993 & 56.41 & 0.21 & 3913.00 & 32944.11 & 1.44 & 0.58 & 0.84 \\
25679 & 103510 & 1993 & 38.50 & 0.14 & 3727.00 & 38848.08 & 1.03 & 1.01 & 1.04 \\
8390 & 101085 & 1993 & 275.50 & 0.13 & 29072.00 & 250700.85 & 0.95 & 0.91 & 0.86 \\
9881 & 101200 & 1993 & 25.06 & 0.03 & 2506.00 & 24766.16 & 1.00 & 0.99 & 0.99 \\
18571 & 102489 & 1993 & 10.90 & -0.00 & 1029.00 & 10876.41 & 1.06 & 1.00 & 1.06 \\
2411 & 100323 & 1993 & 63.48 & 0.05 & 6348.00 & 62060.84 & 1.00 & 0.98 & 0.98 \\
1178 & 100159 & 1993 & 90.67 & 0.13 & 5361.00 & 54437.76 & 1.69 & 0.60 & 1.02 \\
18583 & 102490 & 1993 & 39.50 & 0.02 & 3778.00 & 38508.92 & 1.05 & 0.97 & 1.02 \\
18599 & 102491 & 1993 & 78.94 & 0.08 & 7576.00 & 69716.44 & 1.04 & 0.88 & 0.92 \\
2428 & 100324 & 1993 & 55.41 & 0.12 & 5541.00 & 45071.51 & 1.00 & 0.81 & 0.81 \\
18630 & 102492 & 1993 & 105.29 & 0.17 & 10245.00 & 83913.42 & 1.03 & 0.80 & 0.82 \\
18633 & 102493 & 1993 & 175.15 & 0.05 & 17091.00 & 160513.60 & 1.02 & 0.92 & 0.94 \\
2434 & 100330 & 1993 & 64.07 & 0.23 & 5992.00 & 51698.23 & 1.07 & 0.81 & 0.86 \\
18648 & 102500 & 1993 & 675.32 & 0.13 & 62350.00 & 661549.05 & 1.08 & 0.98 & 1.06 \\
9898 & 101211 & 1993 & 50.34 & -0.06 & 5308.00 & 51885.55 & 0.95 & 1.03 & 0.98 \\
4471 & 100634 & 1993 & 605.40 & 0.13 & 60540.00 & 579332.09 & 1.00 & 0.96 & 0.96 \\
11499 & 101425 & 1993 & 11.67 & 0.10 & 1186.00 & 11520.94 & 0.98 & 0.99 & 0.97 \\
23067 & 103110 & 1993 & 58.17 & 0.12 & 5102.00 & 42586.04 & 1.14 & 0.73 & 0.83 \\
18794 & 102522 & 1993 & 251.97 & 0.13 & 25203.00 & 238105.05 & 1.00 & 0.94 & 0.94 \\
22822 & 103067 & 1993 & 37.70 & 0.03 & 3810.00 & 33493.35 & 0.99 & 0.89 & 0.88 \\
18984 & 102540 & 1993 & 7.60 & 0.05 & 721.00 & 6898.01 & 1.05 & 0.91 & 0.96 \\
2602 & 100346 & 1993 & 5.00 & -0.05 & 490.00 & 4013.01 & 1.02 & 0.80 & 0.82 \\
2605 & 100347 & 1993 & 243.30 & 0.01 & 21947.00 & 220981.76 & 1.11 & 0.91 & 1.01 \\
19016 & 102544 & 1993 & 125.37 & 0.18 & 8498.00 & 70990.53 & 1.48 & 0.57 & 0.84 \\
22710 & 103034 & 1993 & 68.59 & 0.08 & 5970.00 & 48018.57 & 1.15 & 0.70 & 0.80 \\
22709 & 103029 & 1993 & 135.00 & 0.10 & 17601.00 & 174294.62 & 0.77 & 1.29 & 0.99 \\
2624 & 100348 & 1993 & 58.97 & 0.03 & 5522.00 & 51817.90 & 1.07 & 0.88 & 0.94 \\
188 & 100018 & 1993 & 9.38 & 0.14 & 599.00 & 4796.85 & 1.57 & 0.51 & 0.80 \\
19040 & 102545 & 1993 & 20.47 & 0.16 & 2284.00 & 21154.74 & 0.90 & 1.03 & 0.93 \\
19064 & 102547 & 1993 & 75.29 & 0.08 & 6791.00 & 63085.51 & 1.11 & 0.84 & 0.93 \\
21009 & 102823 & 1993 & 17.72 & 0.03 & 1598.00 & 15004.32 & 1.11 & 0.85 & 0.94 \\
22674 & 103028 & 1993 & 2749.41 & 0.04 & 305945.00 & 2811650.81 & 0.90 & 1.02 & 0.92 \\
19100 & 102549 & 1993 & 12.51 & 0.07 & 1034.00 & 10468.94 & 1.21 & 0.84 & 1.01 \\
2643 & 100350 & 1993 & 34.50 & 0.03 & 3279.00 & 31821.35 & 1.05 & 0.92 & 0.97 \\
19131 & 102550 & 1993 & 44.20 & 0.20 & 4298.00 & 39247.07 & 1.03 & 0.89 & 0.91 \\
19056 & 102546 & 1993 & 60.61 & 0.04 & 4670.00 & 45777.29 & 1.30 & 0.76 & 0.98 \\
10190 & 101268 & 1993 & 343.01 & 0.26 & 34301.00 & 290469.07 & 1.00 & 0.85 & 0.85 \\
22735 & 103050 & 1993 & 133.55 & 0.05 & 15657.00 & 142184.30 & 0.85 & 1.06 & 0.91 \\
2530 & 100337 & 1993 & 111.70 & 0.01 & 8560.00 & 87550.45 & 1.30 & 0.78 & 1.02 \\
4375 & 100614 & 1993 & 211.40 & 0.10 & 21133.00 & 210468.03 & 1.00 & 1.00 & 1.00 \\
25622 & 103498 & 1993 & 91.39 & 0.13 & 8406.00 & 84061.18 & 1.09 & 0.92 & 1.00 \\
2533 & 100343 & 1993 & 96.08 & 0.10 & 8905.00 & 85149.53 & 1.08 & 0.89 & 0.96 \\
4349 & 100611 & 1993 & 1521.30 & 0.09 & 152134.00 & 1258470.03 & 1.00 & 0.83 & 0.83 \\
4342 & 100610 & 1993 & 678.50 & 0.14 & 67852.00 & 576774.85 & 1.00 & 0.85 & 0.85 \\
9066 & 101110 & 1993 & 48.30 & 0.19 & 2510.00 & 23241.88 & 1.92 & 0.48 & 0.93 \\
18811 & 102523 & 1993 & 1022.16 & 0.11 & 102216.00 & 903244.43 & 1.00 & 0.88 & 0.88 \\
18952 & 102531 & 1993 & 6.20 & 0.22 & 619.00 & 5350.73 & 1.00 & 0.86 & 0.86 \\
18842 & 102524 & 1993 & 276.20 & 0.10 & 28140.00 & 272860.93 & 0.98 & 0.99 & 0.97 \\
18873 & 102525 & 1993 & 74.67 & 0.14 & 6569.00 & 66464.80 & 1.14 & 0.89 & 1.01 \\
24875 & 103383 & 1993 & 795.75 & 0.05 & 85778.00 & 820570.47 & 0.93 & 1.03 & 0.96 \\
44391 & 109300 & 1993 & 134.19 & 0.13 & 12450.00 & 115517.88 & 1.08 & 0.86 & 0.93 \\
18904 & 102527 & 1993 & 128.75 & 0.12 & 9110.00 & 92487.15 & 1.41 & 0.72 & 1.02 \\
18939 & 102529 & 1993 & 2.43 & 0.04 & 178.00 & 1601.20 & 1.37 & 0.66 & 0.90 \\
22739 & 103057 & 1993 & 1509.80 & 0.11 & 127077.00 & 1055141.50 & 1.19 & 0.70 & 0.83 \\
22787 & 103065 & 1993 & 332.30 & 0.09 & 33872.00 & 313992.44 & 0.98 & 0.94 & 0.93 \\
22006 & 102985 & 1993 & 164.25 & 0.13 & 20300.00 & 188722.81 & 0.81 & 1.15 & 0.93 \\
19758 & 102651 & 1993 & 166.99 & 0.09 & 16192.00 & 155055.30 & 1.03 & 0.93 & 0.96 \\
1589 & 100217 & 1993 & 68.31 & 0.24 & 6218.00 & 68405.47 & 1.10 & 1.00 & 1.10 \\
25346 & 103478 & 1993 & 7.36 & 0.08 & 450.00 & 4430.93 & 1.64 & 0.60 & 0.98 \\
25342 & 103477 & 1993 & 7.81 & 0.02 & 496.00 & 4919.16 & 1.57 & 0.63 & 0.99 \\
21354 & 102854 & 1993 & 23.10 & 0.22 & 2471.00 & 19803.07 & 0.93 & 0.86 & 0.80 \\
21323 & 102852 & 1993 & 169.65 & 0.07 & 16965.00 & 163296.58 & 1.00 & 0.96 & 0.96 \\
20552 & 102767 & 1993 & 837.98 & 0.18 & 76882.00 & 807300.47 & 1.09 & 0.96 & 1.05 \\
3250 & 100419 & 1993 & 1.72 & 0.03 & 159.00 & 1279.53 & 1.08 & 0.74 & 0.80 \\
25340 & 103476 & 1993 & 2.74 & 0.10 & 245.00 & 2274.30 & 1.12 & 0.83 & 0.93 \\
21316 & 102848 & 1993 & 79.96 & 0.17 & 5311.00 & 54519.80 & 1.51 & 0.68 & 1.03 \\
65078 & 500660 & 1993 & 337.97 & 1.24 & 33967.00 & 325937.09 & 0.99 & 0.96 & 0.96 \\
10455 & 101286 & 1993 & 318.44 & -0.03 & 31831.00 & 306353.94 & 1.00 & 0.96 & 0.96 \\
20592 & 102774 & 1993 & 584.50 & 0.17 & 44226.00 & 429327.89 & 1.32 & 0.73 & 0.97 \\
10746 & 101322 & 1993 & 123.70 & 0.09 & 12364.00 & 119990.05 & 1.00 & 0.97 & 0.97 \\
21295 & 102846 & 1993 & 56.20 & 0.19 & 4902.00 & 45893.40 & 1.15 & 0.82 & 0.94 \\
3558 & 100456 & 1993 & 1.60 & 0.02 & 156.00 & 1262.06 & 1.03 & 0.79 & 0.81 \\
20648 & 102777 & 1993 & 443.25 & 0.32 & 36485.00 & 412206.05 & 1.21 & 0.93 & 1.13 \\
21387 & 102861 & 1993 & 66.09 & 0.15 & 5914.00 & 58512.98 & 1.12 & 0.89 & 0.99 \\
25369 & 103479 & 1993 & 16.89 & 0.02 & 1443.00 & 13003.73 & 1.17 & 0.77 & 0.90 \\
3674 & 100468 & 1993 & 219.30 & 0.08 & 19661.00 & 211841.40 & 1.12 & 0.97 & 1.08 \\
20375 & 102732 & 1993 & 212.36 & 0.09 & 22336.00 & 219839.88 & 0.95 & 1.04 & 0.98 \\
20380 & 102733 & 1993 & 5092.24 & 0.16 & 363887.00 & 3371257.99 & 1.40 & 0.66 & 0.93 \\
8672 & 101094 & 1993 & 164.80 & 0.33 & 15121.00 & 141906.32 & 1.09 & 0.86 & 0.94 \\
21486 & 102873 & 1993 & 185.78 & 0.11 & 18577.00 & 149852.15 & 1.00 & 0.81 & 0.81 \\
25192 & 103460 & 1993 & 288.49 & 0.08 & 27919.00 & 281869.00 & 1.03 & 0.98 & 1.01 \\
8707 & 101095 & 1993 & 36.90 & 0.21 & 2504.00 & 25729.98 & 1.47 & 0.70 & 1.03 \\
21446 & 102872 & 1993 & 47.13 & 0.19 & 4114.00 & 36359.57 & 1.15 & 0.77 & 0.88 \\
10793 & 101331 & 1993 & 42.56 & 0.06 & 4256.00 & 37720.68 & 1.00 & 0.89 & 0.89 \\
21415 & 102871 & 1993 & 10.20 & 0.24 & 1026.00 & 8967.02 & 0.99 & 0.88 & 0.87 \\
10761 & 101330 & 1993 & 272.08 & 0.21 & 27208.00 & 230509.00 & 1.00 & 0.85 & 0.85 \\
21405 & 102862 & 1993 & 6.51 & 0.14 & 555.00 & 5191.33 & 1.17 & 0.80 & 0.94 \\
20459 & 102749 & 1993 & 56.17 & 0.07 & 4882.00 & 47651.83 & 1.15 & 0.85 & 0.98 \\
3549 & 100455 & 1993 & 3.90 & 0.11 & 344.00 & 2877.55 & 1.13 & 0.74 & 0.84 \\
21243 & 102842 & 1993 & 9.78 & -0.02 & 1060.00 & 10218.86 & 0.92 & 1.05 & 0.96 \\
20850 & 102797 & 1993 & 40.70 & 0.04 & 3768.00 & 34688.58 & 1.08 & 0.85 & 0.92 \\
10516 & 101298 & 1993 & 407.13 & 0.74 & 40709.00 & 340694.60 & 1.00 & 0.84 & 0.84 \\
3455 & 100439 & 1993 & 73.96 & 0.16 & 8130.00 & 79206.50 & 0.91 & 1.07 & 0.97 \\
10634 & 101302 & 1993 & 342.42 & -0.03 & 34191.00 & 289212.15 & 1.00 & 0.84 & 0.85 \\
21082 & 102828 & 1993 & 98.38 & 0.03 & 9164.00 & 80060.41 & 1.07 & 0.81 & 0.87 \\
65055 & 500659 & 1993 & 23.93 & 0.19 & 2160.00 & 20855.68 & 1.11 & 0.87 & 0.97 \\
20840 & 102796 & 1993 & 1.79 & -0.05 & 127.00 & 1271.72 & 1.41 & 0.71 & 1.00 \\
21059 & 102827 & 1993 & 54.70 & 0.01 & 3070.00 & 29075.55 & 1.78 & 0.53 & 0.95 \\
21017 & 102824 & 1993 & 33.30 & 0.02 & 3095.00 & 28781.79 & 1.08 & 0.86 & 0.93 \\
20920 & 102802 & 1993 & 34.46 & 0.36 & 6069.00 & 49756.29 & 0.57 & 1.44 & 0.82 \\
25230 & 103463 & 1993 & 181.18 & 0.58 & 15498.00 & 146761.48 & 1.17 & 0.81 & 0.95 \\
20932 & 102812 & 1993 & 42.47 & 0.17 & 4217.00 & 38139.13 & 1.01 & 0.90 & 0.90 \\
10560 & 101299 & 1993 & 570.75 & 0.07 & 57076.00 & 548823.10 & 1.00 & 0.96 & 0.96 \\
20971 & 102814 & 1993 & 18.62 & 0.06 & 1886.00 & 18792.34 & 0.99 & 1.01 & 1.00 \\
21052 & 102825 & 1993 & 105.20 & 0.14 & 10518.00 & 101271.42 & 1.00 & 0.96 & 0.96 \\
25274 & 103464 & 1993 & 617.40 & 0.07 & 60090.00 & 596456.33 & 1.03 & 0.97 & 0.99 \\
21134 & 102833 & 1993 & 4.24 & 0.00 & 423.00 & 4141.34 & 1.00 & 0.98 & 0.98 \\
1473 & 100207 & 1993 & 2471.49 & 0.10 & 247198.00 & 2052136.24 & 1.00 & 0.83 & 0.83 \\
25338 & 103475 & 1993 & 2.90 & 0.21 & 260.00 & 2230.28 & 1.12 & 0.77 & 0.86 \\
25336 & 103474 & 1993 & 3.69 & 0.15 & 351.00 & 3335.76 & 1.05 & 0.90 & 0.95 \\
20703 & 102784 & 1993 & 3519.57 & 0.18 & 281169.00 & 2885396.33 & 1.25 & 0.82 & 1.03 \\
21102 & 102832 & 1993 & 37.83 & -0.05 & 3783.00 & 36749.31 & 1.00 & 0.97 & 0.97 \\
21234 & 102840 & 1993 & 32.00 & -0.05 & 3739.00 & 37676.40 & 0.86 & 1.18 & 1.01 \\
25305 & 103466 & 1993 & 95.09 & 0.02 & 7100.00 & 65220.78 & 1.34 & 0.69 & 0.92 \\
10696 & 101312 & 1993 & 14.95 & 0.22 & 1492.00 & 12937.55 & 1.00 & 0.87 & 0.87 \\
21188 & 102837 & 1993 & 50.90 & 0.15 & 5016.00 & 45811.77 & 1.01 & 0.90 & 0.91 \\
152 & 100016 & 1993 & 8.73 & 0.14 & 556.00 & 5689.17 & 1.57 & 0.65 & 1.02 \\
3489 & 100441 & 1993 & 558.89 & 0.15 & 55127.00 & 445214.02 & 1.01 & 0.80 & 0.81 \\
21156 & 102835 & 1993 & 51.81 & 0.11 & 4854.00 & 46787.74 & 1.07 & 0.90 & 0.96 \\
20800 & 102795 & 1993 & 119.00 & 0.18 & 10689.00 & 90263.50 & 1.11 & 0.76 & 0.84 \\
2391 & 100322 & 1993 & 45.70 & 0.23 & 3928.00 & 40040.94 & 1.16 & 0.88 & 1.02 \\
21525 & 102880 & 1993 & 9.10 & 0.10 & 983.00 & 7443.36 & 0.93 & 0.82 & 0.76 \\
3911 & 100514 & 1993 & 102.59 & 0.06 & 10259.00 & 83114.88 & 1.00 & 0.81 & 0.81 \\
25130 & 103436 & 1993 & 4.96 & 0.08 & 496.00 & 4740.36 & 1.00 & 0.96 & 0.96 \\
8920 & 101105 & 1993 & 10.30 & 0.29 & 1078.00 & 10822.70 & 0.96 & 1.05 & 1.00 \\
25423 & 103485 & 1993 & 8.62 & 0.04 & 613.00 & 6192.19 & 1.41 & 0.72 & 1.01 \\
2987 & 100395 & 1993 & 441.40 & 0.17 & 44302.00 & 410726.04 & 1.00 & 0.93 & 0.93 \\
25419 & 103484 & 1993 & 7.64 & 0.03 & 475.00 & 4455.39 & 1.61 & 0.58 & 0.94 \\
3898 & 100510 & 1993 & 8.40 & 0.08 & 840.00 & 8271.38 & 1.00 & 0.98 & 0.98 \\
21902 & 102979 & 1993 & 77.75 & 0.08 & 8200.00 & 65536.15 & 0.95 & 0.84 & 0.80 \\
112 & 100009 & 1993 & 67.02 & 0.18 & 4884.00 & 44492.49 & 1.37 & 0.66 & 0.91 \\
3006 & 100397 & 1993 & 3.80 & 0.21 & 362.00 & 3256.40 & 1.05 & 0.86 & 0.90 \\
25403 & 103483 & 1993 & 120.82 & 0.16 & 11928.00 & 120465.61 & 1.01 & 1.00 & 1.01 \\
25140 & 103439 & 1993 & 53.20 & 0.29 & 5170.00 & 44192.22 & 1.03 & 0.83 & 0.85 \\
3871 & 100508 & 1993 & 11.45 & 0.06 & 840.00 & 6998.25 & 1.36 & 0.61 & 0.83 \\
3017 & 100398 & 1993 & 98.70 & 0.20 & 9300.00 & 93467.03 & 1.06 & 0.95 & 1.01 \\
3863 & 100507 & 1993 & 4.71 & 0.14 & 320.00 & 2767.94 & 1.47 & 0.59 & 0.86 \\
19943 & 102659 & 1993 & 1627.72 & 0.22 & 162770.00 & 1556803.47 & 1.00 & 0.96 & 0.96 \\
21885 & 102969 & 1993 & 125.44 & 0.13 & 11413.00 & 118127.68 & 1.10 & 0.94 & 1.04 \\
10251 & 101276 & 1993 & 153.68 & 0.17 & 15368.00 & 143367.92 & 1.00 & 0.93 & 0.93 \\
19915 & 102655 & 1993 & 752.53 & 0.15 & 75253.00 & 670363.45 & 1.00 & 0.89 & 0.89 \\
25105 & 103432 & 1993 & 957.29 & 0.14 & 90131.00 & 751153.61 & 1.06 & 0.78 & 0.83 \\
25427 & 103487 & 1993 & 4.76 & 0.01 & 381.00 & 3163.23 & 1.25 & 0.66 & 0.83 \\
2908 & 100379 & 1993 & 253.08 & 0.08 & 25308.00 & 243041.43 & 1.00 & 0.96 & 0.96 \\
53398 & 346113 & 1993 & 250.80 & 0.19 & 22781.00 & 200120.11 & 1.10 & 0.80 & 0.88 \\
1265 & 100171 & 1993 & 492.02 & 0.16 & 47709.00 & 439530.57 & 1.03 & 0.89 & 0.92 \\
10204 & 101274 & 1993 & 27.20 & 0.04 & 2550.00 & 23503.82 & 1.07 & 0.86 & 0.92 \\
21995 & 102984 & 1993 & 352.87 & 0.11 & 33770.00 & 302190.18 & 1.04 & 0.86 & 0.89 \\
3965 & 100535 & 1993 & 233.80 & -0.00 & 24338.00 & 229086.42 & 0.96 & 0.98 & 0.94 \\
10977 & 101358 & 1993 & 181.88 & 0.15 & 18188.00 & 149967.60 & 1.00 & 0.82 & 0.82 \\
21983 & 102983 & 1993 & 804.63 & 0.06 & 77670.00 & 669686.74 & 1.04 & 0.83 & 0.86 \\
63171 & 500486 & 1993 & 115.74 & 0.13 & 13358.00 & 114947.38 & 0.87 & 0.99 & 0.86 \\
2947 & 100389 & 1993 & 86.04 & 0.08 & 8604.00 & 70879.73 & 1.00 & 0.82 & 0.82 \\
19871 & 102654 & 1993 & 619.12 & 0.17 & 61912.00 & 588504.97 & 1.00 & 0.95 & 0.95 \\
21950 & 102981 & 1993 & 63.26 & 0.10 & 7100.00 & 59108.77 & 0.89 & 0.93 & 0.83 \\
25067 & 103429 & 1993 & 446.70 & 0.09 & 41566.00 & 385657.17 & 1.07 & 0.86 & 0.93 \\
8595 & 101091 & 1993 & 173.10 & 0.09 & 10838.00 & 98284.89 & 1.60 & 0.57 & 0.91 \\
10223 & 101275 & 1993 & 73.70 & 0.03 & 7160.00 & 70979.42 & 1.03 & 0.96 & 0.99 \\
21847 & 102957 & 1993 & 123.90 & -0.00 & 12619.00 & 121147.82 & 0.98 & 0.98 & 0.96 \\
20008 & 102663 & 1993 & 501.45 & 0.24 & 50145.00 & 446378.60 & 1.00 & 0.89 & 0.89 \\
20230 & 102689 & 1993 & 9.49 & 0.02 & 970.00 & 9191.62 & 0.98 & 0.97 & 0.95 \\
10890 & 101345 & 1993 & 120.96 & 0.08 & 12838.00 & 102285.25 & 0.94 & 0.85 & 0.80 \\
21670 & 102939 & 1993 & 529.14 & 0.10 & 51527.00 & 509740.98 & 1.03 & 0.96 & 0.99 \\
20248 & 102696 & 1993 & 26.33 & 0.23 & 2140.00 & 21801.65 & 1.23 & 0.83 & 1.02 \\
25393 & 103482 & 1993 & 3.47 & 0.04 & 235.00 & 2388.58 & 1.47 & 0.69 & 1.02 \\
25373 & 103481 & 1993 & 15.40 & 0.04 & 1222.00 & 11759.63 & 1.26 & 0.76 & 0.96 \\
3099 & 100409 & 1993 & 8.33 & 0.27 & 833.00 & 6815.75 & 1.00 & 0.82 & 0.82 \\
21615 & 102901 & 1993 & 39.18 & 0.01 & 4180.00 & 32173.50 & 0.94 & 0.82 & 0.77 \\
3134 & 100411 & 1993 & 1527.54 & 0.23 & 152754.00 & 1380964.43 & 1.00 & 0.90 & 0.90 \\
3717 & 100475 & 1993 & 86.90 & 0.02 & 8073.00 & 83144.55 & 1.08 & 0.96 & 1.03 \\
1361 & 100192 & 1993 & 38.53 & 0.08 & 3853.00 & 37594.09 & 1.00 & 0.98 & 0.98 \\
20283 & 102709 & 1993 & 3.60 & 0.03 & 365.00 & 3698.13 & 0.99 & 1.03 & 1.01 \\
21589 & 102895 & 1993 & 193.97 & 0.31 & 19733.00 & 182086.84 & 0.98 & 0.94 & 0.92 \\
8658 & 101093 & 1993 & 15.90 & 0.31 & 1187.00 & 14031.58 & 1.34 & 0.88 & 1.18 \\
21520 & 102878 & 1993 & 6.04 & 0.13 & 629.00 & 5876.73 & 0.96 & 0.97 & 0.93 \\
1511 & 100209 & 1993 & 720.62 & 0.24 & 72062.00 & 679059.73 & 1.00 & 0.94 & 0.94 \\
25180 & 103451 & 1993 & 3.42 & 0.13 & 399.00 & 3885.70 & 0.86 & 1.14 & 0.97 \\
20042 & 102664 & 1993 & 452.12 & 0.23 & 45490.00 & 362109.09 & 0.99 & 0.80 & 0.80 \\
3038 & 100400 & 1993 & 31.20 & -0.02 & 3091.00 & 29369.59 & 1.01 & 0.94 & 0.95 \\
1342 & 100190 & 1993 & 854.87 & 0.10 & 85487.00 & 757196.75 & 1.00 & 0.89 & 0.89 \\
20103 & 102667 & 1993 & 790.90 & 0.18 & 79090.00 & 756994.74 & 1.00 & 0.96 & 0.96 \\
21806 & 102952 & 1993 & 765.84 & 0.09 & 83070.00 & 775846.25 & 0.92 & 1.01 & 0.93 \\
3820 & 100489 & 1993 & 56.74 & 0.04 & 4680.00 & 47111.80 & 1.21 & 0.83 & 1.01 \\
20136 & 102669 & 1993 & 24.48 & -0.05 & 2487.00 & 23870.00 & 0.98 & 0.98 & 0.96 \\
10286 & 101278 & 1993 & 335.14 & 0.21 & 33514.00 & 322895.39 & 1.00 & 0.96 & 0.96 \\
20196 & 102688 & 1993 & 57.92 & 0.08 & 5792.00 & 54321.15 & 1.00 & 0.94 & 0.94 \\
10966 & 101357 & 1993 & 126.91 & 0.08 & 12690.00 & 107713.89 & 1.00 & 0.85 & 0.85 \\
1529 & 100213 & 1993 & 230.26 & 0.17 & 27280.00 & 244071.10 & 0.84 & 1.06 & 0.89 \\
20147 & 102671 & 1993 & 38.44 & 0.10 & 3868.00 & 36246.44 & 0.99 & 0.94 & 0.94 \\
20168 & 102673 & 1993 & 54.71 & 0.12 & 5471.00 & 57407.75 & 1.00 & 1.05 & 1.05 \\
62997 & 500466 & 1993 & 96.09 & 0.09 & 8447.00 & 76974.53 & 1.14 & 0.80 & 0.91 \\
10951 & 101356 & 1993 & 61.94 & 0.23 & 6194.00 & 53581.26 & 1.00 & 0.87 & 0.87 \\
20181 & 102676 & 1993 & 24.66 & 0.03 & 2458.00 & 23309.84 & 1.00 & 0.95 & 0.95 \\
21762 & 102951 & 1993 & 1631.26 & 0.16 & 143014.00 & 1193941.35 & 1.14 & 0.73 & 0.83 \\
10919 & 101354 & 1993 & 432.94 & 0.05 & 43294.00 & 370591.88 & 1.00 & 0.86 & 0.86 \\
21744 & 102949 & 1993 & 1427.90 & 0.20 & 136113.00 & 1295199.03 & 1.05 & 0.91 & 0.95 \\
4521 & 100637 & 1993 & 772.74 & 0.22 & 77274.00 & 716606.17 & 1.00 & 0.93 & 0.93 \\
19072 & 102548 & 1993 & 74.35 & 0.16 & 6660.00 & 57190.70 & 1.12 & 0.77 & 0.86 \\
4869 & 100688 & 1993 & 82.69 & 0.05 & 7701.00 & 64471.89 & 1.07 & 0.78 & 0.84 \\
23135 & 103134 & 1993 & 261.00 & 0.09 & 24467.00 & 251694.33 & 1.07 & 0.96 & 1.03 \\
17897 & 102372 & 1993 & 1214.11 & 0.31 & 110700.00 & 1001433.97 & 1.10 & 0.82 & 0.90 \\
9538 & 101149 & 1993 & 253.57 & 0.15 & 25157.00 & 211687.02 & 1.01 & 0.83 & 0.84 \\
4864 & 100687 & 1993 & 76.56 & 0.09 & 7519.00 & 73166.17 & 1.02 & 0.96 & 0.97 \\
23565 & 103193 & 1993 & 24.55 & 0.06 & 2468.00 & 22664.77 & 0.99 & 0.92 & 0.92 \\
11774 & 101461 & 1993 & 814.41 & 0.14 & 72362.00 & 668196.94 & 1.13 & 0.82 & 0.92 \\
24638 & 103373 & 1993 & 242.07 & 0.13 & 26664.00 & 258458.13 & 0.91 & 1.07 & 0.97 \\
11741 & 101460 & 1993 & 1977.53 & 0.14 & 290430.00 & 1743565.36 & 0.68 & 0.88 & 0.60 \\
9211 & 101119 & 1993 & 48.69 & 0.15 & 4869.00 & 40473.74 & 1.00 & 0.83 & 0.83 \\
2160 & 100293 & 1993 & 54.34 & 0.03 & 5260.00 & 42697.67 & 1.03 & 0.79 & 0.81 \\
11707 & 101457 & 1993 & 149.70 & 0.16 & 13666.00 & 136689.43 & 1.10 & 0.91 & 1.00 \\
4834 & 100685 & 1993 & 3.95 & 0.06 & 361.00 & 3429.67 & 1.09 & 0.87 & 0.95 \\
23650 & 103205 & 1993 & 7.83 & 0.02 & 783.00 & 7226.05 & 1.00 & 0.92 & 0.92 \\
4874 & 100690 & 1993 & 23.53 & 0.12 & 1500.00 & 15512.44 & 1.57 & 0.66 & 1.03 \\
8270 & 101081 & 1993 & 126.10 & 0.18 & 21461.00 & 208907.04 & 0.59 & 1.66 & 0.97 \\
23852 & 103224 & 1993 & 84.12 & 0.14 & 6958.00 & 64562.30 & 1.21 & 0.77 & 0.93 \\
1066 & 100150 & 1993 & 10.80 & 0.12 & 786.00 & 7701.66 & 1.37 & 0.71 & 0.98 \\
2089 & 100290 & 1993 & 252.47 & 0.10 & 24957.00 & 228641.02 & 1.01 & 0.91 & 0.92 \\
11837 & 101463 & 1993 & 88.86 & 0.13 & 8851.00 & 90651.08 & 1.00 & 1.02 & 1.02 \\
18001 & 102386 & 1993 & 82.93 & 0.16 & 8293.00 & 72964.87 & 1.00 & 0.88 & 0.88 \\
17805 & 102364 & 1993 & 212.15 & 0.09 & 16486.00 & 172810.58 & 1.29 & 0.81 & 1.05 \\
256 & 100022 & 1993 & 39.35 & 0.13 & 3762.00 & 38110.52 & 1.05 & 0.97 & 1.01 \\
23753 & 103212 & 1993 & 443.50 & 0.19 & 44350.00 & 355609.35 & 1.00 & 0.80 & 0.80 \\
17836 & 102365 & 1993 & 242.87 & 0.10 & 21262.00 & 216252.45 & 1.14 & 0.89 & 1.02 \\
9520 & 101142 & 1993 & 304.46 & 0.36 & 30447.00 & 270847.77 & 1.00 & 0.89 & 0.89 \\
45312 & 200022 & 1993 & 222.78 & 0.12 & 22278.00 & 191646.24 & 1.00 & 0.86 & 0.86 \\
17864 & 102367 & 1993 & 274.53 & 0.15 & 23084.00 & 230865.93 & 1.19 & 0.84 & 1.00 \\
4882 & 100691 & 1993 & 169.37 & 0.16 & 17419.00 & 162830.04 & 0.97 & 0.96 & 0.93 \\
4902 & 100692 & 1993 & 212.88 & 0.18 & 21453.00 & 206554.97 & 0.99 & 0.97 & 0.96 \\
23513 & 103183 & 1993 & 369.10 & 0.03 & 37136.00 & 345823.13 & 0.99 & 0.94 & 0.93 \\
18036 & 102387 & 1993 & 11.57 & 0.13 & 1151.00 & 11348.97 & 1.00 & 0.98 & 0.99 \\
2247 & 100302 & 1993 & 8.50 & 0.02 & 912.00 & 7775.32 & 0.93 & 0.92 & 0.85 \\
11674 & 101456 & 1993 & 39.99 & 0.12 & 4013.00 & 36194.99 & 1.00 & 0.91 & 0.90 \\
25853 & 103525 & 1993 & 3102.94 & 0.10 & 310294.00 & 3101874.19 & 1.00 & 1.00 & 1.00 \\
9624 & 101160 & 1993 & 93.15 & 0.19 & 8190.00 & 67052.18 & 1.14 & 0.72 & 0.82 \\
18220 & 102417 & 1993 & 622.20 & 0.18 & 62965.00 & 540576.48 & 0.99 & 0.87 & 0.86 \\
4721 & 100668 & 1993 & 48.22 & 0.08 & 4568.00 & 45061.05 & 1.06 & 0.93 & 0.99 \\
9655 & 101161 & 1993 & 123.33 & 0.15 & 13419.00 & 125712.68 & 0.92 & 1.02 & 0.94 \\
25820 & 103524 & 1993 & 8272.70 & 0.10 & 827270.00 & 8235818.73 & 1.00 & 1.00 & 1.00 \\
9674 & 101165 & 1993 & 153.30 & 0.10 & 14541.00 & 144165.62 & 1.05 & 0.94 & 0.99 \\
9698 & 101167 & 1993 & 68.00 & 0.21 & 6064.00 & 51650.99 & 1.12 & 0.76 & 0.85 \\
23376 & 103174 & 1993 & 934.89 & 0.10 & 86720.00 & 731392.73 & 1.08 & 0.78 & 0.84 \\
24718 & 103376 & 1993 & 4552.73 & 0.04 & 467986.00 & 4226183.90 & 0.97 & 0.93 & 0.90 \\
2240 & 100299 & 1993 & 48.38 & 0.09 & 4125.00 & 39222.23 & 1.17 & 0.81 & 0.95 \\
2226 & 100298 & 1993 & 167.61 & 0.18 & 15329.00 & 149747.98 & 1.09 & 0.89 & 0.98 \\
4755 & 100671 & 1993 & 64.50 & 0.21 & 6401.00 & 59789.63 & 1.01 & 0.93 & 0.93 \\
18074 & 102396 & 1993 & 16.31 & 0.05 & 1579.00 & 12868.56 & 1.03 & 0.79 & 0.81 \\
1817 & 100244 & 1993 & 58.79 & 0.16 & 5879.00 & 51075.93 & 1.00 & 0.87 & 0.87 \\
2193 & 100295 & 1993 & 10.05 & 0.15 & 710.00 & 8085.27 & 1.42 & 0.80 & 1.14 \\
18114 & 102399 & 1993 & 4.38 & 0.05 & 470.00 & 3858.27 & 0.93 & 0.88 & 0.82 \\
9576 & 101151 & 1993 & 174.45 & 0.21 & 17445.00 & 154512.35 & 1.00 & 0.89 & 0.89 \\
24678 & 103375 & 1993 & 164.31 & 0.31 & 15491.00 & 141188.15 & 1.06 & 0.86 & 0.91 \\
25916 & 103529 & 1993 & 530.74 & 0.14 & 53073.00 & 520564.79 & 1.00 & 0.98 & 0.98 \\
8311 & 101082 & 1993 & 573.00 & 0.02 & 53637.00 & 549657.73 & 1.07 & 0.96 & 1.02 \\
25887 & 103526 & 1993 & 1135.15 & 0.07 & 113515.00 & 1048568.87 & 1.00 & 0.92 & 0.92 \\
18173 & 102414 & 1993 & 369.10 & 0.04 & 36602.00 & 348597.41 & 1.01 & 0.94 & 0.95 \\
4932 & 100695 & 1993 & 64.08 & 0.06 & 6404.00 & 61791.79 & 1.00 & 0.96 & 0.96 \\
23870 & 103226 & 1993 & 77.30 & 0.19 & 7676.00 & 74802.81 & 1.01 & 0.97 & 0.97 \\
26097 & 103538 & 1993 & 75.15 & 0.11 & 7515.00 & 63990.11 & 1.00 & 0.85 & 0.85 \\
1919 & 100250 & 1993 & 76.74 & 0.10 & 7674.00 & 67197.43 & 1.00 & 0.88 & 0.88 \\
5107 & 100724 & 1993 & 11.73 & 0.05 & 1184.00 & 11744.79 & 0.99 & 1.00 & 0.99 \\
3 & 100001 & 1993 & 857.67 & 0.06 & 84286.00 & 821482.00 & 1.02 & 0.96 & 0.97 \\
1923 & 100251 & 1993 & 133.53 & 0.13 & 12547.00 & 125477.10 & 1.06 & 0.94 & 1.00 \\
17430 & 102306 & 1993 & 937.67 & 0.16 & 90732.00 & 899689.47 & 1.03 & 0.96 & 0.99 \\
1016 & 100127 & 1993 & 177.39 & 0.22 & 17739.00 & 164366.62 & 1.00 & 0.93 & 0.93 \\
17480 & 102313 & 1993 & 350.40 & 0.20 & 26424.00 & 273237.60 & 1.33 & 0.78 & 1.03 \\
24413 & 103321 & 1993 & 126.70 & 0.02 & 9940.00 & 93224.97 & 1.27 & 0.74 & 0.94 \\
5076 & 100723 & 1993 & 8.16 & 0.05 & 825.00 & 8152.05 & 0.99 & 1.00 & 0.99 \\
9406 & 101134 & 1993 & 315.99 & 0.73 & 31596.00 & 267310.07 & 1.00 & 0.85 & 0.85 \\
5074 & 100715 & 1993 & 85.99 & 0.06 & 8417.00 & 73774.28 & 1.02 & 0.86 & 0.88 \\
24398 & 103319 & 1993 & 143.70 & 0.14 & 14580.00 & 147761.22 & 0.99 & 1.03 & 1.01 \\
12009 & 101477 & 1993 & 113.17 & 0.22 & 9664.00 & 94498.53 & 1.17 & 0.83 & 0.98 \\
24336 & 103315 & 1993 & 28.65 & 0.23 & 2774.00 & 23467.39 & 1.03 & 0.82 & 0.85 \\
12026 & 101488 & 1993 & 40.20 & 0.07 & 4022.00 & 39618.19 & 1.00 & 0.99 & 0.99 \\
17413 & 102305 & 1993 & 275.23 & 0.09 & 27523.00 & 225296.38 & 1.00 & 0.82 & 0.82 \\
24523 & 103338 & 1993 & 34.11 & 0.02 & 4422.00 & 39544.02 & 0.77 & 1.16 & 0.89 \\
57960 & 410010 & 1993 & 97.49 & 0.04 & 9749.00 & 83562.63 & 1.00 & 0.86 & 0.86 \\
24566 & 103369 & 1993 & 71.64 & 0.01 & 5333.00 & 51398.25 & 1.34 & 0.72 & 0.96 \\
24565 & 103368 & 1993 & 118.74 & 0.17 & 7903.00 & 76499.15 & 1.50 & 0.64 & 0.97 \\
24554 & 103366 & 1993 & 108.90 & 0.09 & 11759.00 & 118182.66 & 0.93 & 1.09 & 1.01 \\
9380 & 101133 & 1993 & 339.24 & 0.19 & 33924.00 & 331450.30 & 1.00 & 0.98 & 0.98 \\
9316 & 101131 & 1993 & 319.22 & 0.19 & 31922.00 & 261428.85 & 1.00 & 0.82 & 0.82 \\
1880 & 100247 & 1993 & 594.96 & 0.10 & 59496.00 & 514119.43 & 1.00 & 0.86 & 0.86 \\
26167 & 103545 & 1993 & 7616.33 & 0.14 & 761632.00 & 7031729.05 & 1.00 & 0.92 & 0.92 \\
26133 & 103544 & 1993 & 512.28 & 0.11 & 51228.00 & 498590.81 & 1.00 & 0.97 & 0.97 \\
5150 & 100727 & 1993 & 652.76 & -0.09 & 68316.00 & 671702.16 & 0.96 & 1.03 & 0.98 \\
17377 & 102284 & 1993 & 81.62 & 0.22 & 7790.00 & 66123.18 & 1.05 & 0.81 & 0.85 \\
1851 & 100245 & 1993 & 246.07 & 0.03 & 24607.00 & 215795.24 & 1.00 & 0.88 & 0.88 \\
17398 & 102286 & 1993 & 31.60 & 0.02 & 2882.00 & 28735.42 & 1.10 & 0.91 & 1.00 \\
24528 & 103339 & 1993 & 331.10 & 0.12 & 25906.00 & 242726.86 & 1.28 & 0.73 & 0.94 \\
5168 & 100730 & 1993 & 286.82 & 0.31 & 24039.00 & 221649.74 & 1.19 & 0.77 & 0.92 \\
24310 & 103308 & 1993 & 3061.37 & 0.16 & 298085.00 & 2675025.20 & 1.03 & 0.87 & 0.90 \\
17670 & 102342 & 1993 & 38.70 & 0.02 & 3876.00 & 38173.33 & 1.00 & 0.99 & 0.98 \\
2028 & 100281 & 1993 & 14.96 & 0.04 & 1538.00 & 15350.48 & 0.97 & 1.03 & 1.00 \\
24079 & 103264 & 1993 & 665.77 & 0.03 & 60845.00 & 581206.51 & 1.09 & 0.87 & 0.96 \\
4958 & 100697 & 1993 & 36.28 & 0.09 & 3493.00 & 35019.60 & 1.04 & 0.97 & 1.00 \\
4953 & 100696 & 1993 & 5.07 & 0.04 & 500.00 & 4284.29 & 1.01 & 0.85 & 0.86 \\
9489 & 101140 & 1993 & 510.39 & 0.18 & 51020.00 & 484269.20 & 1.00 & 0.95 & 0.95 \\
9273 & 101127 & 1993 & 477.02 & 0.10 & 43288.00 & 369541.83 & 1.10 & 0.77 & 0.85 \\
17745 & 102356 & 1993 & 1.43 & 0.10 & 146.00 & 1170.79 & 0.98 & 0.82 & 0.80 \\
24 & 100003 & 1993 & 79.68 & 0.05 & 6017.00 & 49909.12 & 1.32 & 0.63 & 0.83 \\
23924 & 103242 & 1993 & 5.47 & 0.02 & 737.00 & 7446.19 & 0.74 & 1.36 & 1.01 \\
17776 & 102357 & 1993 & 21.54 & 0.01 & 1363.00 & 11110.65 & 1.58 & 0.52 & 0.82 \\
23888 & 103228 & 1993 & 34.34 & 0.11 & 3221.00 & 31213.30 & 1.07 & 0.91 & 0.97 \\
25954 & 103531 & 1993 & 269.58 & 0.26 & 26958.00 & 230664.03 & 1.00 & 0.86 & 0.86 \\
9176 & 101116 & 1993 & 1456.00 & 0.16 & 166420.00 & 1258755.50 & 0.87 & 0.86 & 0.76 \\
17649 & 102334 & 1993 & 41.40 & 0.09 & 3340.00 & 27734.86 & 1.24 & 0.67 & 0.83 \\
4991 & 100698 & 1993 & 19.39 & 0.32 & 1915.00 & 17464.55 & 1.01 & 0.90 & 0.91 \\
26008 & 103533 & 1993 & 609.05 & 0.12 & 60905.00 & 499305.65 & 1.00 & 0.82 & 0.82 \\
9441 & 101135 & 1993 & 566.21 & -0.08 & 56621.00 & 508400.58 & 1.00 & 0.90 & 0.90 \\
17542 & 102318 & 1993 & 3179.09 & 0.06 & 299595.00 & 2995121.14 & 1.06 & 0.94 & 1.00 \\
5020 & 100701 & 1993 & 29.02 & 0.35 & 2382.00 & 24664.59 & 1.22 & 0.85 & 1.04 \\
17576 & 102319 & 1993 & 703.74 & 0.06 & 65693.00 & 657097.07 & 1.07 & 0.93 & 1.00 \\
24227 & 103299 & 1993 & 724.18 & 0.37 & 51088.00 & 612091.22 & 1.42 & 0.85 & 1.20 \\
17613 & 102321 & 1993 & 180.76 & 0.09 & 17048.00 & 170657.46 & 1.06 & 0.94 & 1.00 \\
2000 & 100280 & 1993 & 77.20 & 0.07 & 5747.00 & 53886.62 & 1.34 & 0.70 & 0.94 \\
267 & 100030 & 1993 & 80.76 & 0.10 & 8068.00 & 77422.64 & 1.00 & 0.96 & 0.96 \\
24192 & 103296 & 1993 & 1165.98 & 0.10 & 110097.00 & 1103188.04 & 1.06 & 0.95 & 1.00 \\
11949 & 101473 & 1993 & 560.01 & 0.20 & 56000.00 & 479940.98 & 1.00 & 0.86 & 0.86 \\
25786 & 103523 & 1993 & 927.95 & 0.14 & 92794.00 & 888674.92 & 1.00 & 0.96 & 0.96 \\
20998 & 102821 & 1993 & 507.64 & 0.05 & 52808.00 & 434819.44 & 0.96 & 0.86 & 0.82 \\
59 & 100004 & 1993 & 682.51 & 0.16 & 60714.00 & 504144.86 & 1.12 & 0.74 & 0.83 \\
23275 & 103154 & 1993 & 82.70 & 0.32 & 8269.00 & 74674.22 & 1.00 & 0.90 & 0.90 \\
2359 & 100320 & 1993 & 30.96 & 0.20 & 2370.00 & 25436.46 & 1.31 & 0.82 & 1.07 \\
23328 & 103164 & 1993 & 2.49 & 0.09 & 235.00 & 2489.80 & 1.06 & 1.00 & 1.06 \\
9762 & 101192 & 1993 & 13.00 & -0.01 & 1559.00 & 14459.41 & 0.83 & 1.11 & 0.93 \\
25720 & 103520 & 1993 & 50.62 & 0.01 & 5062.00 & 50633.48 & 1.00 & 1.00 & 1.00 \\
1779 & 100237 & 1993 & 56.03 & 0.10 & 5636.00 & 51617.09 & 0.99 & 0.92 & 0.92 \\
25752 & 103521 & 1993 & 190.11 & 0.18 & 19011.00 & 189664.69 & 1.00 & 1.00 & 1.00 \\
23290 & 103158 & 1993 & 838.00 & 0.09 & 83869.00 & 876543.95 & 1.00 & 1.05 & 1.05 \\
9744 & 101186 & 1993 & 112.10 & 0.08 & 11024.00 & 105208.54 & 1.02 & 0.94 & 0.95 \\
1131 & 100155 & 1993 & 90.90 & 0.11 & 9234.00 & 90494.88 & 0.98 & 1.00 & 0.98 \\
18437 & 102455 & 1993 & 15.95 & 0.09 & 1388.00 & 12698.24 & 1.15 & 0.80 & 0.91 \\
2307 & 100315 & 1993 & 245.41 & 0.10 & 24413.00 & 246375.54 & 1.01 & 1.00 & 1.01 \\
4669 & 100660 & 1993 & 96.45 & 0.07 & 9787.00 & 94939.88 & 0.99 & 0.98 & 0.97 \\
4619 & 100644 & 1993 & 44.49 & 0.02 & 4500.00 & 42251.94 & 0.99 & 0.95 & 0.94 \\
18314 & 102425 & 1993 & 746.20 & 0.03 & 74346.00 & 702559.53 & 1.00 & 0.94 & 0.94 \\
4589 & 100642 & 1993 & 666.65 & 0.09 & 66665.00 & 616715.67 & 1.00 & 0.93 & 0.93 \\
18533 & 102471 & 1993 & 111.37 & 0.13 & 10870.00 & 106637.07 & 1.02 & 0.96 & 0.98 \\
1116 & 100154 & 1993 & 128.48 & -0.02 & 20795.00 & 200461.85 & 0.62 & 1.56 & 0.96 \\
23302 & 103160 & 1993 & 11.10 & 0.04 & 1109.00 & 10346.45 & 1.00 & 0.93 & 0.93 \\
9142 & 101115 & 1993 & 1678.90 & 0.10 & 171730.00 & 1676122.57 & 0.98 & 1.00 & 0.98 \\
24794 & 103380 & 1993 & 4939.37 & 0.15 & 510983.00 & 4311042.50 & 0.97 & 0.87 & 0.84 \\
24755 & 103377 & 1993 & 724.14 & 0.20 & 69591.00 & 626421.76 & 1.04 & 0.87 & 0.90 \\
4635 & 100659 & 1993 & 166.78 & 0.02 & 16693.00 & 165759.99 & 1.00 & 0.99 & 0.99 \\
18519 & 102470 & 1993 & 767.36 & -0.03 & 89933.00 & 871682.59 & 0.85 & 1.14 & 0.97 \\
18506 & 102469 & 1993 & 155.69 & 0.26 & 14055.00 & 123181.64 & 1.11 & 0.79 & 0.88 \\
2327 & 100319 & 1993 & 220.00 & 0.16 & 20941.00 & 190216.86 & 1.05 & 0.86 & 0.91 \\
1760 & 100228 & 1993 & 119.96 & 0.11 & 11694.00 & 97122.01 & 1.03 & 0.81 & 0.83 \\
18347 & 102441 & 1993 & 414.30 & 0.20 & 41436.00 & 361336.77 & 1.00 & 0.87 & 0.87 \\
23151 & 103136 & 1993 & 112.65 & 0.08 & 9571.00 & 88275.83 & 1.18 & 0.78 & 0.92 \\
23243 & 103152 & 1993 & 736.03 & 0.01 & 72396.00 & 676358.01 & 1.02 & 0.92 & 0.93 \\
1146 & 100157 & 1993 & 73.71 & 0.10 & 6647.00 & 65182.69 & 1.11 & 0.88 & 0.98 \\
25688 & 103514 & 1993 & 1067.92 & 0.15 & 106792.00 & 1032280.84 & 1.00 & 0.97 & 0.97 \\
18392 & 102447 & 1993 & 204.61 & 0.18 & 20461.00 & 178834.43 & 1.00 & 0.87 & 0.87 \\
3135 & 100411 & 1994 & 2299.32 & 0.00 & 229932.00 & 1988053.65 & 1.00 & 0.86 & 0.86 \\
14398 & 101854 & 1994 & 967.32 & 0.01 & 96808.00 & 842635.99 & 1.00 & 0.87 & 0.87 \\
6965 & 100974 & 1994 & 6.20 & -0.04 & 548.00 & 5007.47 & 1.13 & 0.81 & 0.91 \\
52030 & 301299 & 1994 & 178.64 & -0.06 & 17864.00 & 163254.56 & 1.00 & 0.91 & 0.91 \\
22942 & 103090 & 1994 & 577.95 & -0.05 & 63589.00 & 464566.50 & 0.91 & 0.80 & 0.73 \\
2001 & 100280 & 1994 & 48.25 & 0.02 & 4856.00 & 47739.79 & 0.99 & 0.99 & 0.98 \\
10377 & 101284 & 1994 & 8.16 & 0.17 & 815.00 & 6941.50 & 1.00 & 0.85 & 0.85 \\
21590 & 102895 & 1994 & 449.08 & 0.10 & 44704.00 & 432745.60 & 1.00 & 0.96 & 0.97 \\
24259 & 103301 & 1994 & 669.28 & 0.03 & 64657.00 & 602608.02 & 1.04 & 0.90 & 0.93 \\
24193 & 103296 & 1994 & 1160.15 & -0.00 & 119269.00 & 1080746.18 & 0.97 & 0.93 & 0.91 \\
21556 & 102894 & 1994 & 51.70 & -0.01 & 6368.00 & 42410.83 & 0.81 & 0.82 & 0.67 \\
6535 & 100889 & 1994 & 32.30 & 0.00 & 3338.00 & 31049.92 & 0.97 & 0.96 & 0.93 \\
9763 & 101192 & 1994 & 7.14 & 0.00 & 765.00 & 7187.31 & 0.93 & 1.01 & 0.94 \\
9455 & 101137 & 1994 & 6.47 & 0.05 & 591.00 & 5677.80 & 1.10 & 0.88 & 0.96 \\
6544 & 100890 & 1994 & 258.88 & 0.04 & 27296.00 & 239163.71 & 0.95 & 0.92 & 0.88 \\
21416 & 102871 & 1994 & 26.80 & 0.04 & 2560.00 & 24864.24 & 1.05 & 0.93 & 0.97 \\
22976 & 103100 & 1994 & 186.32 & 0.01 & 19086.00 & 166738.77 & 0.98 & 0.89 & 0.87 \\
22957 & 103099 & 1994 & 93.63 & 0.01 & 8966.00 & 74021.94 & 1.04 & 0.79 & 0.83 \\
74786 & 601171 & 1994 & 212.95 & 0.01 & 10756.00 & 98945.77 & 1.98 & 0.46 & 0.92 \\
9442 & 101135 & 1994 & 488.47 & 0.01 & 48847.00 & 420416.56 & 1.00 & 0.86 & 0.86 \\
363 & 100041 & 1994 & 39.63 & -0.03 & 4059.00 & 38313.89 & 0.98 & 0.97 & 0.94 \\
15140 & 101963 & 1994 & 472.14 & -0.00 & 47247.00 & 422699.86 & 1.00 & 0.90 & 0.89 \\
23025 & 103103 & 1994 & 60.33 & -0.00 & 6687.00 & 60567.29 & 0.90 & 1.00 & 0.91 \\
21447 & 102872 & 1994 & 202.40 & 0.03 & 16465.00 & 163160.65 & 1.23 & 0.81 & 0.99 \\
25193 & 103460 & 1994 & 283.78 & -0.01 & 25238.00 & 259061.62 & 1.12 & 0.91 & 1.03 \\
23211 & 103145 & 1994 & 84.94 & -0.04 & 8125.00 & 76230.00 & 1.05 & 0.90 & 0.94 \\
21487 & 102873 & 1994 & 166.28 & -0.02 & 14887.00 & 146732.69 & 1.12 & 0.88 & 0.99 \\
21491 & 102875 & 1994 & 17.39 & 0.02 & 1243.00 & 10125.10 & 1.40 & 0.58 & 0.81 \\
24228 & 103299 & 1994 & 646.95 & -0.04 & 64872.00 & 569252.71 & 1.00 & 0.88 & 0.88 \\
21502 & 102876 & 1994 & 31.37 & 0.07 & 1853.00 & 16846.55 & 1.69 & 0.54 & 0.91 \\
9745 & 101186 & 1994 & 110.84 & -0.03 & 11810.00 & 113994.66 & 0.94 & 1.03 & 0.97 \\
9917 & 101212 & 1994 & 436.30 & -0.07 & 46700.00 & 401963.42 & 0.93 & 0.92 & 0.86 \\
37705 & 107004 & 1994 & 5.02 & 0.02 & 723.00 & 6313.43 & 0.69 & 1.26 & 0.87 \\
12876 & 101603 & 1994 & 580.43 & -0.08 & 59016.00 & 519323.97 & 0.98 & 0.89 & 0.88 \\
22994 & 103101 & 1994 & 120.98 & 0.03 & 11710.00 & 107522.67 & 1.03 & 0.89 & 0.92 \\
539 & 100075 & 1994 & 714.67 & 0.17 & 68724.00 & 645122.89 & 1.04 & 0.90 & 0.94 \\
13834 & 101769 & 1994 & 2197.50 & -0.06 & 225720.00 & 2095270.72 & 0.97 & 0.95 & 0.93 \\
21745 & 102949 & 1994 & 1291.49 & -0.10 & 135225.00 & 1084102.98 & 0.96 & 0.84 & 0.80 \\
3078 & 100408 & 1994 & 13.43 & 0.02 & 1342.00 & 12481.86 & 1.00 & 0.93 & 0.93 \\
21763 & 102951 & 1994 & 1994.19 & 0.03 & 199517.00 & 1736786.87 & 1.00 & 0.87 & 0.87 \\
9078 & 101111 & 1994 & 108.40 & 0.00 & 16429.00 & 136548.09 & 0.66 & 1.26 & 0.83 \\
14442 & 101858 & 1994 & 87.53 & -0.16 & 9883.00 & 93910.90 & 0.89 & 1.07 & 0.95 \\
62998 & 500466 & 1994 & 225.12 & -0.06 & 11320.00 & 102082.26 & 1.99 & 0.45 & 0.90 \\
27307 & 105268 & 1994 & 22.89 & 0.01 & 2225.00 & 20319.84 & 1.03 & 0.89 & 0.91 \\
22917 & 103085 & 1994 & 100.11 & -0.02 & 10011.00 & 93514.60 & 1.00 & 0.93 & 0.93 \\
13477 & 101741 & 1994 & 456.46 & -0.05 & 48994.00 & 396024.34 & 0.93 & 0.87 & 0.81 \\
24027 & 103255 & 1994 & 96.67 & -0.02 & 8444.00 & 88594.43 & 1.14 & 0.92 & 1.05 \\
9274 & 101127 & 1994 & 334.53 & -0.08 & 29919.00 & 276679.50 & 1.12 & 0.83 & 0.92 \\
1530 & 100213 & 1994 & 176.21 & -0.10 & 19208.00 & 176298.12 & 0.92 & 1.00 & 0.92 \\
25 & 100003 & 1994 & 198.15 & -0.05 & 19296.00 & 174275.31 & 1.03 & 0.88 & 0.90 \\
3039 & 100400 & 1994 & 13.93 & -0.07 & 2570.00 & 25822.05 & 0.54 & 1.85 & 1.00 \\
6934 & 100973 & 1994 & 4.58 & 0.05 & 406.00 & 3341.35 & 1.13 & 0.73 & 0.82 \\
15019 & 101943 & 1994 & 22.64 & -0.05 & 3543.00 & 30261.22 & 0.64 & 1.34 & 0.85 \\
23925 & 103242 & 1994 & 5.00 & 0.00 & 492.00 & 4829.06 & 1.02 & 0.97 & 0.98 \\
14237 & 101835 & 1994 & 122.20 & 0.01 & 11975.00 & 116148.62 & 1.02 & 0.95 & 0.97 \\
6580 & 100892 & 1994 & 58.90 & 0.03 & 5683.00 & 55673.29 & 1.04 & 0.95 & 0.98 \\
21807 & 102952 & 1994 & 601.90 & -0.05 & 60407.00 & 550844.43 & 1.00 & 0.92 & 0.91 \\
23976 & 103252 & 1994 & 142.49 & 0.02 & 12870.00 & 124935.99 & 1.11 & 0.88 & 0.97 \\
7040 & 100992 & 1994 & 280.88 & -0.12 & 31930.00 & 229449.07 & 0.88 & 0.82 & 0.72 \\
8895 & 101104 & 1994 & 26.60 & 0.04 & 2863.00 & 25234.40 & 0.93 & 0.95 & 0.88 \\
23955 & 103251 & 1994 & 142.53 & -0.01 & 12881.00 & 118701.33 & 1.11 & 0.83 & 0.92 \\
24049 & 103259 & 1994 & 153.09 & 0.04 & 9040.00 & 72770.15 & 1.69 & 0.48 & 0.80 \\
2360 & 100320 & 1994 & 36.28 & -0.07 & 3593.00 & 35784.11 & 1.01 & 0.99 & 1.00 \\
1741 & 100227 & 1994 & 112.60 & 0.05 & 11345.00 & 109776.77 & 0.99 & 0.97 & 0.97 \\
2471 & 100333 & 1994 & 93.20 & 0.00 & 9318.00 & 82606.13 & 1.00 & 0.89 & 0.89 \\
10347 & 101283 & 1994 & 143.74 & 0.01 & 14356.00 & 130248.27 & 1.00 & 0.91 & 0.91 \\
47405 & 210770 & 1994 & 207.00 & -0.05 & 20659.00 & 206517.10 & 1.00 & 1.00 & 1.00 \\
21627 & 102924 & 1994 & 4.54 & -0.04 & 414.00 & 3618.49 & 1.10 & 0.80 & 0.87 \\
21616 & 102901 & 1994 & 34.24 & -0.02 & 3436.00 & 33988.93 & 1.00 & 0.99 & 0.99 \\
7697 & 101055 & 1994 & 2993.40 & -0.02 & 296198.00 & 2760835.73 & 1.01 & 0.92 & 0.93 \\
411 & 100055 & 1994 & 4081.58 & 0.04 & 422958.00 & 3531926.27 & 0.97 & 0.87 & 0.84 \\
15108 & 101958 & 1994 & 61.31 & 0.00 & 6346.00 & 60447.96 & 0.97 & 0.99 & 0.95 \\
7272 & 101018 & 1994 & 3287.00 & 0.10 & 225876.00 & 2129664.04 & 1.46 & 0.65 & 0.94 \\
1512 & 100209 & 1994 & 982.50 & -0.08 & 98250.00 & 855779.88 & 1.00 & 0.87 & 0.87 \\
2032 & 100284 & 1994 & 5.59 & 0.01 & 574.00 & 5073.41 & 0.97 & 0.91 & 0.88 \\
21701 & 102940 & 1994 & 70.72 & -0.03 & 6813.00 & 60554.28 & 1.04 & 0.86 & 0.89 \\
2031 & 100283 & 1994 & 5.75 & 0.01 & 601.00 & 4880.26 & 0.96 & 0.85 & 0.81 \\
21631 & 102937 & 1994 & 13.05 & 0.04 & 1332.00 & 11676.12 & 0.98 & 0.89 & 0.88 \\
3100 & 100409 & 1994 & 32.18 & 0.02 & 3218.00 & 28010.74 & 1.00 & 0.87 & 0.87 \\
15039 & 101953 & 1994 & 189.15 & 0.04 & 18897.00 & 180941.84 & 1.00 & 0.96 & 0.96 \\
2029 & 100281 & 1994 & 10.01 & -0.03 & 1060.00 & 10503.16 & 0.94 & 1.05 & 0.99 \\
24080 & 103264 & 1994 & 683.89 & -0.08 & 72553.00 & 654289.45 & 0.94 & 0.96 & 0.90 \\
21671 & 102939 & 1994 & 553.99 & -0.00 & 55270.00 & 523948.36 & 1.00 & 0.95 & 0.95 \\
24092 & 103266 & 1994 & 66.62 & -0.05 & 6137.00 & 60979.50 & 1.09 & 0.92 & 0.99 \\
2030 & 100282 & 1994 & 5.58 & 0.03 & 552.00 & 4794.06 & 1.01 & 0.86 & 0.87 \\
15090 & 101956 & 1994 & 501.24 & -0.02 & 50151.00 & 420869.05 & 1.00 & 0.84 & 0.84 \\
7521 & 101043 & 1994 & 635.20 & -0.00 & 58419.00 & 572495.86 & 1.09 & 0.90 & 0.98 \\
23181 & 103144 & 1994 & 99.30 & -0.04 & 8244.00 & 85748.24 & 1.20 & 0.86 & 1.04 \\
21103 & 102832 & 1994 & 23.67 & -0.01 & 2367.00 & 23597.61 & 1.00 & 1.00 & 1.00 \\
9344 & 101132 & 1994 & 46.88 & -0.07 & 4687.00 & 38820.43 & 1.00 & 0.83 & 0.83 \\
15282 & 101978 & 1994 & 68.53 & -0.03 & 7517.00 & 65580.53 & 0.91 & 0.96 & 0.87 \\
14348 & 101851 & 1994 & 605.65 & -0.13 & 60994.00 & 559979.13 & 0.99 & 0.92 & 0.92 \\
12819 & 101601 & 1994 & 646.96 & -0.08 & 65265.00 & 609573.45 & 0.99 & 0.94 & 0.93 \\
21135 & 102833 & 1994 & 6.95 & 0.02 & 695.00 & 6902.35 & 1.00 & 0.99 & 0.99 \\
15291 & 101980 & 1994 & 24.38 & -0.03 & 2144.00 & 21628.48 & 1.14 & 0.89 & 1.01 \\
15295 & 101982 & 1994 & 226.48 & -0.06 & 22630.00 & 223275.03 & 1.00 & 0.99 & 0.99 \\
21092 & 102829 & 1994 & 32.17 & -0.06 & 4075.00 & 37153.61 & 0.79 & 1.16 & 0.91 \\
1852 & 100245 & 1994 & 237.92 & 0.02 & 23792.00 & 193673.00 & 1.00 & 0.81 & 0.81 \\
3348 & 100425 & 1994 & 559.68 & 0.01 & 43536.00 & 417320.48 & 1.29 & 0.75 & 0.96 \\
9882 & 101200 & 1994 & 18.41 & -0.15 & 2145.00 & 21869.60 & 0.86 & 1.19 & 1.02 \\
6424 & 100868 & 1994 & 77.90 & 0.03 & 7590.00 & 77367.09 & 1.03 & 0.99 & 1.02 \\
21189 & 102837 & 1994 & 57.67 & -0.06 & 5724.00 & 52695.11 & 1.01 & 0.91 & 0.92 \\
7386 & 101038 & 1994 & 3207.80 & 0.04 & 329618.00 & 2747751.79 & 0.97 & 0.86 & 0.83 \\
24497 & 103329 & 1994 & 269.92 & 0.03 & 34239.00 & 308759.30 & 0.79 & 1.14 & 0.90 \\
2412 & 100323 & 1994 & 58.38 & 0.05 & 5864.00 & 56878.68 & 1.00 & 0.97 & 0.97 \\
3325 & 100424 & 1994 & 42.60 & 0.03 & 3707.00 & 36695.57 & 1.15 & 0.86 & 0.99 \\
21157 & 102835 & 1994 & 54.83 & -0.06 & 6461.00 & 58045.55 & 0.85 & 1.06 & 0.90 \\
15277 & 101977 & 1994 & 30.03 & -0.02 & 2983.00 & 27589.36 & 1.01 & 0.92 & 0.92 \\
24529 & 103339 & 1994 & 414.77 & -0.01 & 41449.00 & 389718.33 & 1.00 & 0.94 & 0.94 \\
13260 & 101714 & 1994 & 33.50 & -0.04 & 3355.00 & 33360.18 & 1.00 & 1.00 & 0.99 \\
13267 & 101716 & 1994 & 11.37 & -0.01 & 1096.00 & 10878.13 & 1.04 & 0.96 & 0.99 \\
21083 & 102828 & 1994 & 84.41 & -0.02 & 9189.00 & 78799.80 & 0.92 & 0.93 & 0.86 \\
1881 & 100247 & 1994 & 545.85 & -0.08 & 54584.00 & 451426.99 & 1.00 & 0.83 & 0.83 \\
25231 & 103463 & 1994 & 158.51 & -0.09 & 14945.00 & 131696.50 & 1.06 & 0.83 & 0.88 \\
24567 & 103369 & 1994 & 112.95 & -0.07 & 11296.00 & 107364.21 & 1.00 & 0.95 & 0.95 \\
13611 & 101748 & 1994 & 9.11 & -0.02 & 898.00 & 8938.18 & 1.01 & 0.98 & 1.00 \\
10561 & 101299 & 1994 & 537.96 & 0.03 & 53812.00 & 523927.60 & 1.00 & 0.97 & 0.97 \\
2392 & 100322 & 1994 & 121.93 & 0.03 & 11956.00 & 113854.86 & 1.02 & 0.93 & 0.95 \\
24587 & 103370 & 1994 & 5.43 & 0.05 & 492.00 & 4555.30 & 1.10 & 0.84 & 0.93 \\
9824 & 101194 & 1994 & 61.43 & -0.09 & 6267.00 & 50906.60 & 0.98 & 0.83 & 0.81 \\
12798 & 101600 & 1994 & 146.94 & 0.01 & 14640.00 & 138893.55 & 1.00 & 0.95 & 0.95 \\
6373 & 100856 & 1994 & 120.33 & -0.05 & 12034.00 & 105403.68 & 1.00 & 0.88 & 0.88 \\
47914 & 222809 & 1994 & 208.52 & 0.01 & 20899.00 & 183122.42 & 1.00 & 0.88 & 0.88 \\
15331 & 101987 & 1994 & 293.72 & -0.05 & 28838.00 & 283338.67 & 1.02 & 0.96 & 0.98 \\
47246 & 200344 & 1994 & 895.81 & -0.04 & 89581.00 & 835115.73 & 1.00 & 0.93 & 0.93 \\
23152 & 103136 & 1994 & 113.50 & -0.00 & 11880.00 & 109658.65 & 0.96 & 0.97 & 0.92 \\
6403 & 100864 & 1994 & 477.87 & -0.13 & 50631.00 & 468071.19 & 0.94 & 0.98 & 0.92 \\
10517 & 101298 & 1994 & 410.72 & -0.05 & 41502.00 & 329846.63 & 0.99 & 0.80 & 0.79 \\
21010 & 102823 & 1994 & 22.94 & -0.02 & 2335.00 & 20038.60 & 0.98 & 0.87 & 0.86 \\
21060 & 102827 & 1994 & 62.24 & -0.00 & 5099.00 & 43978.55 & 1.22 & 0.71 & 0.86 \\
153 & 100016 & 1994 & 23.16 & 0.04 & 2347.00 & 22792.12 & 0.99 & 0.98 & 0.97 \\
21053 & 102825 & 1994 & 91.60 & 0.02 & 9590.00 & 81763.50 & 0.96 & 0.89 & 0.85 \\
21018 & 102824 & 1994 & 38.95 & -0.01 & 3838.00 & 32433.32 & 1.01 & 0.83 & 0.85 \\
24555 & 103366 & 1994 & 48.36 & 0.00 & 6673.00 & 55837.08 & 0.72 & 1.15 & 0.84 \\
10502 & 101294 & 1994 & 5.09 & 0.01 & 509.00 & 4974.81 & 1.00 & 0.98 & 0.98 \\
24399 & 103319 & 1994 & 127.90 & -0.03 & 13429.00 & 130784.50 & 0.95 & 1.02 & 0.97 \\
6482 & 100876 & 1994 & 5.85 & 0.01 & 584.00 & 4854.41 & 1.00 & 0.83 & 0.83 \\
48251 & 240057 & 1994 & 10.95 & 0.14 & 1091.00 & 9022.65 & 1.00 & 0.82 & 0.83 \\
3251 & 100419 & 1994 & 2.25 & 0.05 & 194.00 & 1652.18 & 1.16 & 0.73 & 0.85 \\
21324 & 102852 & 1994 & 176.14 & 0.08 & 17568.00 & 152276.90 & 1.00 & 0.86 & 0.87 \\
21317 & 102848 & 1994 & 74.38 & -0.03 & 7873.00 & 61844.97 & 0.94 & 0.83 & 0.79 \\
651 & 100087 & 1994 & 3667.56 & -0.06 & 400634.00 & 3321065.77 & 0.92 & 0.91 & 0.83 \\
2429 & 100324 & 1994 & 52.58 & -0.00 & 5258.00 & 47284.04 & 1.00 & 0.90 & 0.90 \\
13547 & 101743 & 1994 & 2504.31 & -0.07 & 274281.00 & 2389633.02 & 0.91 & 0.95 & 0.87 \\
21305 & 102847 & 1994 & 36.80 & 0.03 & 3024.00 & 28286.72 & 1.22 & 0.77 & 0.94 \\
23163 & 103138 & 1994 & 395.07 & 0.01 & 40646.00 & 367229.13 & 0.97 & 0.93 & 0.90 \\
10456 & 101286 & 1994 & 271.18 & -0.01 & 27127.00 & 250042.34 & 1.00 & 0.92 & 0.92 \\
21355 & 102854 & 1994 & 132.50 & 0.23 & 13305.00 & 129987.30 & 1.00 & 0.98 & 0.98 \\
6499 & 100878 & 1994 & 900.65 & 0.00 & 93768.00 & 857218.47 & 0.96 & 0.95 & 0.91 \\
24311 & 103308 & 1994 & 3144.26 & -0.03 & 314426.00 & 2643920.66 & 1.00 & 0.84 & 0.84 \\
1964 & 100263 & 1994 & 7.87 & -0.02 & 787.00 & 7157.52 & 1.00 & 0.91 & 0.91 \\
14270 & 101842 & 1994 & 243.60 & -0.01 & 25865.00 & 240724.34 & 0.94 & 0.99 & 0.93 \\
21388 & 102861 & 1994 & 65.27 & -0.00 & 6585.00 & 63046.81 & 0.99 & 0.97 & 0.96 \\
9899 & 101211 & 1994 & 33.74 & -0.03 & 3575.00 & 34131.72 & 0.94 & 1.01 & 0.95 \\
13861 & 101781 & 1994 & 643.66 & -0.03 & 63173.00 & 509698.96 & 1.02 & 0.79 & 0.81 \\
2435 & 100330 & 1994 & 87.06 & 0.07 & 8026.00 & 68872.01 & 1.08 & 0.79 & 0.86 \\
24337 & 103315 & 1994 & 28.11 & -0.01 & 2914.00 & 23384.88 & 0.96 & 0.83 & 0.80 \\
9317 & 101131 & 1994 & 338.87 & -0.02 & 33886.00 & 271829.65 & 1.00 & 0.80 & 0.80 \\
15169 & 101964 & 1994 & 41.89 & 0.06 & 4189.00 & 40988.72 & 1.00 & 0.98 & 0.98 \\
48341 & 240065 & 1994 & 176.12 & 0.05 & 18195.00 & 158293.54 & 0.97 & 0.90 & 0.87 \\
21296 & 102846 & 1994 & 58.20 & 0.03 & 5982.00 & 53461.57 & 0.97 & 0.92 & 0.89 \\
13742 & 101762 & 1994 & 1583.47 & 0.09 & 159000.00 & 1451385.78 & 1.00 & 0.92 & 0.91 \\
15230 & 101970 & 1994 & 47.56 & 0.01 & 4355.00 & 38346.01 & 1.09 & 0.81 & 0.88 \\
1920 & 100250 & 1994 & 75.79 & -0.07 & 7579.00 & 61987.19 & 1.00 & 0.82 & 0.82 \\
6450 & 100875 & 1994 & 195.11 & -0.05 & 19864.00 & 182412.23 & 0.98 & 0.93 & 0.92 \\
13276 & 101717 & 1994 & 12.65 & -0.02 & 1261.00 & 11776.48 & 1.00 & 0.93 & 0.93 \\
24467 & 103328 & 1994 & 219.21 & 0.05 & 18017.00 & 182212.49 & 1.22 & 0.83 & 1.01 \\
23068 & 103110 & 1994 & 56.68 & 0.00 & 5918.00 & 49215.40 & 0.96 & 0.87 & 0.83 \\
15246 & 101972 & 1994 & 110.02 & -0.02 & 9881.00 & 89319.52 & 1.11 & 0.81 & 0.90 \\
14323 & 101850 & 1994 & 56.12 & -0.04 & 5607.00 & 45255.09 & 1.00 & 0.81 & 0.81 \\
406 & 100054 & 1994 & 2.29 & -0.06 & 273.00 & 2549.50 & 0.84 & 1.11 & 0.93 \\
1474 & 100207 & 1994 & 2105.85 & -0.04 & 210585.00 & 1807090.13 & 1.00 & 0.86 & 0.86 \\
24414 & 103321 & 1994 & 133.40 & 0.00 & 14345.00 & 137044.12 & 0.93 & 1.03 & 0.96 \\
8814 & 101100 & 1994 & 50.80 & 0.00 & 4360.00 & 36047.85 & 1.17 & 0.71 & 0.83 \\
21271 & 102844 & 1994 & 13.87 & 0.02 & 1287.00 & 12340.18 & 1.08 & 0.89 & 0.96 \\
24415 & 103326 & 1994 & 560.50 & 0.02 & 45759.00 & 403575.73 & 1.22 & 0.72 & 0.88 \\
456 & 100068 & 1994 & 24.46 & 0.00 & 2255.00 & 21422.83 & 1.08 & 0.88 & 0.95 \\
47330 & 210203 & 1994 & 239.11 & 0.03 & 22598.00 & 215562.11 & 1.06 & 0.90 & 0.95 \\
21247 & 102843 & 1994 & 5.65 & 0.03 & 525.00 & 4961.42 & 1.08 & 0.88 & 0.95 \\
21244 & 102842 & 1994 & 7.16 & 0.02 & 1060.00 & 9996.48 & 0.68 & 1.40 & 0.94 \\
24445 & 103327 & 1994 & 798.90 & 0.01 & 83214.00 & 726005.70 & 0.96 & 0.91 & 0.87 \\
4 & 100001 & 1994 & 876.19 & -0.01 & 92961.00 & 896403.43 & 0.94 & 1.02 & 0.96 \\
1924 & 100251 & 1994 & 123.60 & -0.13 & 12563.00 & 109857.48 & 0.98 & 0.89 & 0.87 \\
21878 & 102964 & 1994 & 34.94 & 0.02 & 3504.00 & 29465.27 & 1.00 & 0.84 & 0.84 \\
9490 & 101140 & 1994 & 444.53 & -0.09 & 44452.00 & 383775.23 & 1.00 & 0.86 & 0.86 \\
2215 & 100296 & 1994 & 6.53 & 0.01 & 728.00 & 6389.73 & 0.90 & 0.98 & 0.88 \\
22327 & 103007 & 1994 & 842.03 & -0.01 & 83933.00 & 807644.53 & 1.00 & 0.96 & 0.96 \\
14701 & 101911 & 1994 & 99.61 & 0.04 & 9961.00 & 95371.30 & 1.00 & 0.96 & 0.96 \\
24970 & 103402 & 1994 & 138.53 & -0.04 & 14000.00 & 129554.93 & 0.99 & 0.94 & 0.93 \\
14146 & 101809 & 1994 & 51.64 & -0.11 & 5051.00 & 52407.94 & 1.02 & 1.01 & 1.04 \\
24679 & 103375 & 1994 & 274.17 & 0.14 & 27590.00 & 249785.92 & 0.99 & 0.91 & 0.91 \\
22371 & 103008 & 1994 & 50.21 & -0.08 & 5045.00 & 49058.70 & 1.00 & 0.98 & 0.97 \\
14145 & 101807 & 1994 & 66.88 & -0.05 & 7889.00 & 59910.19 & 0.85 & 0.90 & 0.76 \\
7140 & 100998 & 1994 & 102.19 & -0.03 & 9769.00 & 83854.63 & 1.05 & 0.82 & 0.86 \\
2746 & 100357 & 1994 & 152.34 & -0.10 & 15271.00 & 149283.58 & 1.00 & 0.98 & 0.98 \\
2715 & 100355 & 1994 & 224.76 & -0.01 & 22540.00 & 223454.58 & 1.00 & 0.99 & 0.99 \\
2227 & 100298 & 1994 & 207.93 & 0.00 & 20582.00 & 185119.55 & 1.01 & 0.89 & 0.90 \\
10123 & 101262 & 1994 & 7.01 & -0.04 & 648.00 & 5545.08 & 1.08 & 0.79 & 0.86 \\
8993 & 101108 & 1994 & 664.80 & -0.03 & 71232.00 & 551246.89 & 0.93 & 0.83 & 0.77 \\
22291 & 103005 & 1994 & 90.00 & 0.00 & 8842.00 & 80037.55 & 1.02 & 0.89 & 0.91 \\
14667 & 101908 & 1994 & 5.80 & 0.03 & 574.00 & 5716.13 & 1.01 & 0.99 & 1.00 \\
6785 & 100954 & 1994 & 73.91 & -0.03 & 7373.00 & 59497.79 & 1.00 & 0.81 & 0.81 \\
24897 & 103394 & 1994 & 28.00 & 0.01 & 2800.00 & 23570.18 & 1.00 & 0.84 & 0.84 \\
22407 & 103011 & 1994 & 75.70 & -0.01 & 7910.00 & 70766.92 & 0.96 & 0.93 & 0.89 \\
23472 & 103179 & 1994 & 172.30 & 0.00 & 17574.00 & 148355.45 & 0.98 & 0.86 & 0.84 \\
14735 & 101912 & 1994 & 753.53 & 0.09 & 75353.00 & 742817.81 & 1.00 & 0.99 & 0.99 \\
7584 & 101047 & 1994 & 185.60 & 0.05 & 17520.00 & 171515.79 & 1.06 & 0.92 & 0.98 \\
7159 & 101000 & 1994 & 553.62 & -0.00 & 60066.00 & 524803.84 & 0.92 & 0.95 & 0.87 \\
14825 & 101916 & 1994 & 74.10 & -0.08 & 7410.00 & 66031.30 & 1.00 & 0.89 & 0.89 \\
13373 & 101733 & 1994 & 33.20 & 0.01 & 3480.00 & 33848.97 & 0.95 & 1.02 & 0.97 \\
6742 & 100947 & 1994 & 455.23 & -0.11 & 52591.00 & 471502.27 & 0.87 & 1.04 & 0.90 \\
1818 & 100244 & 1994 & 69.06 & -0.07 & 6912.00 & 55699.70 & 1.00 & 0.81 & 0.81 \\
2823 & 100362 & 1994 & 63.52 & 0.04 & 6384.00 & 61947.82 & 1.00 & 0.98 & 0.97 \\
590 & 100079 & 1994 & 746.41 & -0.03 & 76507.00 & 715962.27 & 0.98 & 0.96 & 0.94 \\
13324 & 101728 & 1994 & 26.38 & 0.03 & 2637.00 & 23733.13 & 1.00 & 0.90 & 0.90 \\
2194 & 100295 & 1994 & 10.20 & 0.05 & 1022.00 & 8926.99 & 1.00 & 0.88 & 0.87 \\
1808 & 100238 & 1994 & 268.50 & -0.07 & 27005.00 & 236089.64 & 0.99 & 0.88 & 0.87 \\
6763 & 100953 & 1994 & 8.46 & -0.11 & 863.00 & 7368.79 & 0.98 & 0.87 & 0.85 \\
565 & 100076 & 1994 & 161.94 & -0.05 & 14166.00 & 122277.37 & 1.14 & 0.76 & 0.86 \\
23493 & 103182 & 1994 & 121.50 & -0.00 & 12958.00 & 109338.79 & 0.94 & 0.90 & 0.84 \\
2308 & 100315 & 1994 & 226.23 & -0.01 & 22652.00 & 217323.19 & 1.00 & 0.96 & 0.96 \\
14799 & 101914 & 1994 & 25.70 & -0.01 & 2570.00 & 25144.42 & 1.00 & 0.98 & 0.98 \\
2790 & 100358 & 1994 & 2112.97 & -0.07 & 207209.00 & 1782989.04 & 1.02 & 0.84 & 0.86 \\
23329 & 103164 & 1994 & 1.50 & -0.37 & 235.00 & 2071.79 & 0.64 & 1.38 & 0.88 \\
14767 & 101913 & 1994 & 36.90 & -0.04 & 3690.00 & 36425.81 & 1.00 & 0.99 & 0.99 \\
24876 & 103383 & 1994 & 820.25 & 0.05 & 79348.00 & 711712.84 & 1.03 & 0.87 & 0.90 \\
2241 & 100299 & 1994 & 56.09 & -0.02 & 5526.00 & 49514.93 & 1.01 & 0.88 & 0.90 \\
1678 & 100223 & 1994 & 313.01 & -0.00 & 31393.00 & 272531.22 & 1.00 & 0.87 & 0.87 \\
23407 & 103175 & 1994 & 877.11 & 0.01 & 86012.00 & 848175.91 & 1.02 & 0.97 & 0.99 \\
22631 & 103027 & 1994 & 226.40 & -0.02 & 25549.00 & 231605.10 & 0.89 & 1.02 & 0.91 \\
7488 & 101042 & 1994 & 102.40 & 0.04 & 7125.00 & 71837.75 & 1.44 & 0.70 & 1.01 \\
1780 & 100237 & 1994 & 67.72 & -0.01 & 8476.00 & 76774.86 & 0.80 & 1.13 & 0.91 \\
7108 & 100997 & 1994 & 43.62 & -0.04 & 4928.00 & 37289.45 & 0.89 & 0.85 & 0.76 \\
7604 & 101048 & 1994 & 1340.70 & -0.00 & 125436.00 & 1157508.96 & 1.07 & 0.86 & 0.92 \\
23377 & 103174 & 1994 & 1015.53 & -0.08 & 103557.00 & 904161.16 & 0.98 & 0.89 & 0.87 \\
14126 & 101805 & 1994 & 1223.58 & 0.00 & 117102.00 & 1105570.19 & 1.04 & 0.90 & 0.94 \\
14109 & 101804 & 1994 & 234.24 & -0.05 & 25235.00 & 213856.71 & 0.93 & 0.91 & 0.85 \\
24935 & 103395 & 1994 & 120.78 & 0.03 & 12078.00 & 97639.42 & 1.00 & 0.81 & 0.81 \\
2267 & 100305 & 1994 & 6.42 & -0.00 & 601.00 & 5288.15 & 1.07 & 0.82 & 0.88 \\
10041 & 101256 & 1994 & 7.70 & -0.12 & 854.00 & 7833.46 & 0.90 & 1.02 & 0.92 \\
22675 & 103028 & 1994 & 2287.50 & -0.04 & 255684.00 & 2242470.00 & 0.89 & 0.98 & 0.88 \\
24719 & 103376 & 1994 & 4158.03 & -0.01 & 432530.00 & 3519645.27 & 0.96 & 0.85 & 0.81 \\
9675 & 101165 & 1994 & 158.96 & 0.01 & 14877.00 & 151448.44 & 1.07 & 0.95 & 1.02 \\
518 & 100072 & 1994 & 3703.51 & 0.01 & 319088.00 & 2636952.29 & 1.16 & 0.71 & 0.83 \\
2625 & 100348 & 1994 & 77.70 & -0.02 & 7852.00 & 79468.57 & 0.99 & 1.02 & 1.01 \\
2644 & 100350 & 1994 & 59.34 & -0.07 & 5727.00 & 55837.66 & 1.04 & 0.94 & 0.97 \\
9720 & 101179 & 1994 & 100.08 & 0.05 & 8511.00 & 72291.77 & 1.18 & 0.72 & 0.85 \\
2663 & 100351 & 1994 & 152.03 & -0.09 & 16553.00 & 161732.66 & 0.92 & 1.06 & 0.98 \\
6865 & 100966 & 1994 & 22.20 & 0.07 & 2153.00 & 20731.12 & 1.03 & 0.93 & 0.96 \\
6829 & 100962 & 1994 & 1121.57 & 0.01 & 111233.00 & 942180.55 & 1.01 & 0.84 & 0.85 \\
14091 & 101802 & 1994 & 188.94 & -0.06 & 20853.00 & 177414.35 & 0.91 & 0.94 & 0.85 \\
13029 & 101622 & 1994 & 725.99 & -0.07 & 71366.00 & 573476.76 & 1.02 & 0.79 & 0.80 \\
22480 & 103015 & 1994 & 26.26 & -0.00 & 2675.00 & 24070.44 & 0.98 & 0.92 & 0.90 \\
22462 & 103014 & 1994 & 138.29 & -0.00 & 13076.00 & 124374.47 & 1.06 & 0.90 & 0.95 \\
2248 & 100302 & 1994 & 5.38 & -0.08 & 542.00 & 5405.81 & 0.99 & 1.01 & 1.00 \\
6873 & 100967 & 1994 & 210.27 & 0.07 & 18644.00 & 209121.04 & 1.13 & 0.99 & 1.12 \\
24756 & 103377 & 1994 & 851.60 & -0.06 & 90320.00 & 684307.02 & 0.94 & 0.80 & 0.76 \\
51902 & 300102 & 1994 & 13.03 & -0.01 & 1337.00 & 12563.50 & 0.97 & 0.96 & 0.94 \\
13347 & 101729 & 1994 & 221.33 & -0.03 & 23469.00 & 193636.09 & 0.94 & 0.87 & 0.83 \\
22594 & 103024 & 1994 & 118.03 & 0.00 & 12302.00 & 113769.03 & 0.96 & 0.96 & 0.92 \\
9656 & 101161 & 1994 & 189.25 & 0.04 & 13393.00 & 122705.82 & 1.41 & 0.65 & 0.92 \\
22563 & 103021 & 1994 & 33.53 & -0.04 & 3355.00 & 32112.61 & 1.00 & 0.96 & 0.96 \\
22711 & 103034 & 1994 & 182.80 & -0.05 & 20215.00 & 189721.04 & 0.90 & 1.04 & 0.94 \\
23441 & 103177 & 1994 & 105.75 & 0.01 & 10262.00 & 86071.79 & 1.03 & 0.81 & 0.84 \\
2683 & 100352 & 1994 & 97.65 & -0.00 & 9653.00 & 96527.50 & 1.01 & 0.99 & 1.00 \\
27510 & 105284 & 1994 & 1.17 & -0.14 & 119.00 & 959.21 & 0.98 & 0.82 & 0.81 \\
2606 & 100347 & 1994 & 277.59 & 0.01 & 28086.00 & 281003.99 & 0.99 & 1.01 & 1.00 \\
22522 & 103017 & 1994 & 302.28 & 0.01 & 36171.00 & 332849.30 & 0.84 & 1.10 & 0.92 \\
2603 & 100346 & 1994 & 3.97 & 0.00 & 400.00 & 3615.43 & 0.99 & 0.91 & 0.90 \\
2254 & 100303 & 1994 & 70.09 & -0.01 & 7546.00 & 64072.59 & 0.93 & 0.91 & 0.85 \\
6814 & 100961 & 1994 & 8.84 & -0.07 & 1384.00 & 12901.26 & 0.64 & 1.46 & 0.93 \\
30954 & 105842 & 1994 & 7.72 & 0.01 & 700.00 & 6991.80 & 1.10 & 0.91 & 1.00 \\
12963 & 101617 & 1994 & 69.81 & -0.10 & 7121.00 & 55989.08 & 0.98 & 0.80 & 0.79 \\
14222 & 101834 & 1994 & 16.20 & -0.07 & 1705.00 & 16194.46 & 0.95 & 1.00 & 0.95 \\
25131 & 103436 & 1994 & 5.03 & 0.08 & 497.00 & 4507.26 & 1.01 & 0.90 & 0.91 \\
2090 & 100290 & 1994 & 253.85 & -0.00 & 27657.00 & 257919.81 & 0.92 & 1.02 & 0.93 \\
8921 & 101105 & 1994 & 11.10 & 0.04 & 1315.00 & 8995.15 & 0.84 & 0.81 & 0.68 \\
7424 & 101039 & 1994 & 1468.20 & 0.04 & 145046.00 & 1185441.86 & 1.01 & 0.81 & 0.82 \\
14947 & 101925 & 1994 & 190.79 & -0.03 & 19458.00 & 179167.24 & 0.98 & 0.94 & 0.92 \\
22855 & 103073 & 1994 & 410.23 & -0.02 & 41019.00 & 393772.13 & 1.00 & 0.96 & 0.96 \\
48206 & 240051 & 1994 & 105.07 & -0.03 & 12051.00 & 103752.86 & 0.87 & 0.99 & 0.86 \\
25068 & 103429 & 1994 & 496.36 & 0.10 & 49636.00 & 410484.33 & 1.00 & 0.83 & 0.83 \\
47792 & 221485 & 1994 & 90.17 & -0.03 & 8797.00 & 87974.86 & 1.03 & 0.98 & 1.00 \\
23291 & 103158 & 1994 & 811.00 & -0.04 & 81098.00 & 797758.37 & 1.00 & 0.98 & 0.98 \\
21984 & 102983 & 1994 & 596.91 & -0.08 & 73128.00 & 598060.27 & 0.82 & 1.00 & 0.82 \\
2531 & 100337 & 1994 & 107.56 & -0.03 & 10386.00 & 100814.18 & 1.04 & 0.94 & 0.97 \\
14191 & 101820 & 1994 & 183.97 & -0.00 & 18758.00 & 165173.35 & 0.98 & 0.90 & 0.88 \\
6664 & 100908 & 1994 & 101.20 & -0.03 & 10463.00 & 93472.03 & 0.97 & 0.92 & 0.89 \\
10224 & 101275 & 1994 & 58.81 & 0.00 & 7198.00 & 70472.45 & 0.82 & 1.20 & 0.98 \\
21951 & 102981 & 1994 & 59.53 & -0.07 & 6195.00 & 55293.30 & 0.96 & 0.93 & 0.89 \\
9982 & 101216 & 1994 & 46.85 & 0.00 & 4721.00 & 43026.73 & 0.99 & 0.92 & 0.91 \\
63172 & 500486 & 1994 & 131.61 & -0.03 & 14100.00 & 123593.86 & 0.93 & 0.94 & 0.88 \\
13163 & 101698 & 1994 & 108.10 & 0.04 & 10810.00 & 98242.33 & 1.00 & 0.91 & 0.91 \\
384 & 100048 & 1994 & 325.90 & -0.00 & 30603.00 & 280843.01 & 1.06 & 0.86 & 0.92 \\
13911 & 101787 & 1994 & 312.08 & -0.00 & 28422.00 & 273915.19 & 1.10 & 0.88 & 0.96 \\
22823 & 103067 & 1994 & 36.28 & 0.03 & 3691.00 & 35240.89 & 0.98 & 0.97 & 0.95 \\
14979 & 101926 & 1994 & 77.97 & -0.03 & 6877.00 & 60510.87 & 1.13 & 0.78 & 0.88 \\
23276 & 103154 & 1994 & 169.80 & -0.00 & 16975.00 & 143478.75 & 1.00 & 0.84 & 0.85 \\
23871 & 103226 & 1994 & 74.45 & -0.04 & 6699.00 & 65115.44 & 1.11 & 0.87 & 0.97 \\
47806 & 222027 & 1994 & 89.00 & 0.01 & 5471.00 & 53141.42 & 1.63 & 0.60 & 0.97 \\
10252 & 101276 & 1994 & 191.80 & 0.10 & 19180.00 & 169395.17 & 1.00 & 0.88 & 0.88 \\
23345 & 103166 & 1994 & 5.53 & -0.20 & 577.00 & 5305.60 & 0.96 & 0.96 & 0.92 \\
23244 & 103152 & 1994 & 796.12 & 0.06 & 77333.00 & 734239.41 & 1.03 & 0.92 & 0.95 \\
14992 & 101930 & 1994 & 525.32 & -0.03 & 52532.00 & 446991.73 & 1.00 & 0.85 & 0.85 \\
23889 & 103228 & 1994 & 37.52 & -0.05 & 3748.00 & 34161.80 & 1.00 & 0.91 & 0.91 \\
21848 & 102957 & 1994 & 112.46 & -0.01 & 11226.00 & 104195.42 & 1.00 & 0.93 & 0.93 \\
23905 & 103232 & 1994 & 3.90 & 0.02 & 332.00 & 3441.02 & 1.17 & 0.88 & 1.04 \\
15003 & 101933 & 1994 & 104.53 & 0.03 & 8818.00 & 88067.39 & 1.19 & 0.84 & 1.00 \\
12909 & 101606 & 1994 & 1007.01 & 0.03 & 100700.00 & 964625.24 & 1.00 & 0.96 & 0.96 \\
113 & 100009 & 1994 & 62.20 & -0.09 & 6463.00 & 59095.20 & 0.96 & 0.95 & 0.91 \\
27332 & 105269 & 1994 & 31.80 & 0.02 & 2945.00 & 27331.67 & 1.08 & 0.86 & 0.93 \\
13878 & 101785 & 1994 & 672.96 & 0.11 & 61679.00 & 528279.51 & 1.09 & 0.79 & 0.86 \\
2988 & 100395 & 1994 & 379.00 & 0.00 & 37837.00 & 324203.28 & 1.00 & 0.86 & 0.86 \\
3018 & 100398 & 1994 & 71.80 & -0.18 & 9120.00 & 69463.57 & 0.79 & 0.97 & 0.76 \\
6631 & 100906 & 1994 & 605.38 & 0.01 & 61016.00 & 527055.11 & 0.99 & 0.87 & 0.86 \\
21886 & 102969 & 1994 & 142.65 & 0.02 & 15292.00 & 140301.74 & 0.93 & 0.98 & 0.92 \\
22874 & 103074 & 1994 & 93.09 & -0.04 & 10530.00 & 78160.53 & 0.88 & 0.84 & 0.74 \\
23853 & 103224 & 1994 & 90.77 & 0.04 & 7973.00 & 67948.86 & 1.14 & 0.75 & 0.85 \\
25141 & 103439 & 1994 & 85.26 & 0.05 & 7550.00 & 74731.51 & 1.13 & 0.88 & 0.99 \\
14463 & 101861 & 1994 & 348.50 & -0.02 & 35592.00 & 322480.94 & 0.98 & 0.93 & 0.91 \\
27499 & 105283 & 1994 & 1.90 & -0.02 & 207.00 & 1614.74 & 0.92 & 0.85 & 0.78 \\
23547 & 103186 & 1994 & 12.00 & 0.03 & 811.00 & 6970.56 & 1.48 & 0.58 & 0.86 \\
22041 & 102988 & 1994 & 100.35 & -0.11 & 10717.00 & 101049.97 & 0.94 & 1.01 & 0.94 \\
2875 & 100368 & 1994 & 88.17 & 0.02 & 8663.00 & 74112.71 & 1.02 & 0.84 & 0.86 \\
23540 & 103184 & 1994 & 1167.70 & -0.03 & 124366.00 & 1009147.40 & 0.94 & 0.86 & 0.81 \\
2328 & 100319 & 1994 & 190.50 & 0.02 & 18836.00 & 181200.58 & 1.01 & 0.95 & 0.96 \\
24639 & 103373 & 1994 & 143.80 & -0.10 & 14385.00 & 130632.39 & 1.00 & 0.91 & 0.91 \\
14507 & 101871 & 1994 & 156.40 & -0.01 & 12794.00 & 130133.17 & 1.22 & 0.83 & 1.02 \\
27454 & 105279 & 1994 & 26.80 & 0.01 & 2131.00 & 19580.96 & 1.26 & 0.73 & 0.92 \\
22102 & 102990 & 1994 & 123.31 & -0.04 & 13188.00 & 124022.20 & 0.94 & 1.01 & 0.94 \\
9212 & 101119 & 1994 & 39.27 & 0.00 & 3927.00 & 33316.00 & 1.00 & 0.85 & 0.85 \\
22788 & 103065 & 1994 & 303.09 & -0.00 & 30182.00 & 277474.61 & 1.00 & 0.92 & 0.92 \\
22136 & 102993 & 1994 & 502.98 & -0.03 & 44885.00 & 398091.69 & 1.12 & 0.79 & 0.89 \\
8962 & 101107 & 1994 & 1193.80 & -0.07 & 120779.00 & 1069379.11 & 0.99 & 0.90 & 0.89 \\
13376 & 101736 & 1994 & 5.12 & 0.01 & 410.00 & 3420.42 & 1.25 & 0.67 & 0.83 \\
27460 & 105280 & 1994 & 24.80 & 0.02 & 2119.00 & 18867.31 & 1.17 & 0.76 & 0.89 \\
22180 & 102994 & 1994 & 45.18 & -0.03 & 5288.00 & 46683.27 & 0.85 & 1.03 & 0.88 \\
48174 & 240040 & 1994 & 193.20 & -0.02 & 20348.00 & 163978.43 & 0.95 & 0.85 & 0.81 \\
23514 & 103183 & 1994 & 302.50 & -0.04 & 32170.00 & 282886.17 & 0.94 & 0.94 & 0.88 \\
2161 & 100293 & 1994 & 39.05 & -0.10 & 4645.00 & 34156.76 & 0.84 & 0.87 & 0.74 \\
2856 & 100366 & 1994 & 8.90 & -0.11 & 972.00 & 8939.86 & 0.92 & 1.00 & 0.92 \\
2847 & 100365 & 1994 & 91.49 & 0.01 & 9202.00 & 86241.74 & 0.99 & 0.94 & 0.94 \\
2884 & 100369 & 1994 & 163.21 & 0.01 & 18351.00 & 161714.37 & 0.89 & 0.99 & 0.88 \\
60 & 100004 & 1994 & 631.63 & -0.10 & 68111.00 & 550063.74 & 0.93 & 0.87 & 0.81 \\
9539 & 101149 & 1994 & 310.08 & 0.05 & 31008.00 & 273278.45 & 1.00 & 0.88 & 0.88 \\
6681 & 100910 & 1994 & 50.90 & -0.00 & 5316.00 & 44843.60 & 0.96 & 0.88 & 0.84 \\
22007 & 102985 & 1994 & 114.44 & -0.03 & 13013.00 & 116143.73 & 0.88 & 1.01 & 0.89 \\
23620 & 103204 & 1994 & 37.94 & -0.03 & 3794.00 & 36452.82 & 1.00 & 0.96 & 0.96 \\
13949 & 101789 & 1994 & 316.29 & 0.01 & 26934.00 & 261988.79 & 1.17 & 0.83 & 0.97 \\
9067 & 101110 & 1994 & 76.30 & 0.14 & 5974.00 & 49771.24 & 1.28 & 0.65 & 0.83 \\
6905 & 100968 & 1994 & 17.70 & 0.04 & 1773.00 & 16798.28 & 1.00 & 0.95 & 0.95 \\
23651 & 103205 & 1994 & 13.70 & -0.03 & 877.00 & 7912.05 & 1.56 & 0.58 & 0.90 \\
21996 & 102984 & 1994 & 364.70 & -0.00 & 41934.00 & 391875.83 & 0.87 & 1.07 & 0.93 \\
14502 & 101867 & 1994 & 128.16 & -0.04 & 14421.00 & 134030.79 & 0.89 & 1.05 & 0.93 \\
13930 & 101788 & 1994 & 250.34 & 0.01 & 30772.00 & 303532.36 & 0.81 & 1.21 & 0.99 \\
2909 & 100379 & 1994 & 279.51 & -0.00 & 28596.00 & 267558.35 & 0.98 & 0.96 & 0.94 \\
14904 & 101921 & 1994 & 32.80 & -0.01 & 3256.00 & 30776.36 & 1.01 & 0.94 & 0.95 \\
23303 & 103160 & 1994 & 68.00 & -0.14 & 6810.00 & 60492.38 & 1.00 & 0.89 & 0.89 \\
47502 & 212408 & 1994 & 250.65 & 0.06 & 14693.00 & 146098.35 & 1.71 & 0.58 & 0.99 \\
22010 & 102987 & 1994 & 291.31 & 0.02 & 30047.00 & 295262.18 & 0.97 & 1.01 & 0.98 \\
14159 & 101819 & 1994 & 108.75 & 0.01 & 11589.00 & 100929.07 & 0.94 & 0.93 & 0.87 \\
6695 & 100913 & 1994 & 22.10 & 0.00 & 1991.00 & 19407.40 & 1.11 & 0.88 & 0.97 \\
23566 & 103193 & 1994 & 26.63 & 0.07 & 2743.00 & 21886.78 & 0.97 & 0.82 & 0.80 \\
13120 & 101668 & 1994 & 40.20 & 0.01 & 4343.00 & 39349.33 & 0.93 & 0.98 & 0.91 \\
9143 & 101115 & 1994 & 2060.60 & 0.01 & 187265.00 & 1770166.30 & 1.10 & 0.86 & 0.95 \\
2893 & 100371 & 1994 & 10.47 & -0.04 & 1216.00 & 10160.68 & 0.86 & 0.97 & 0.84 \\
47533 & 212658 & 1994 & 100.52 & 0.00 & 10066.00 & 90336.76 & 1.00 & 0.90 & 0.90 \\
2534 & 100343 & 1994 & 96.24 & -0.00 & 9250.00 & 90339.68 & 1.04 & 0.94 & 0.98 \\
2130 & 100292 & 1994 & 130.83 & -0.04 & 14950.00 & 151274.39 & 0.88 & 1.16 & 1.01 \\
12944 & 101616 & 1994 & 4458.43 & -0.03 & 451423.00 & 3881421.05 & 0.99 & 0.87 & 0.86 \\
2898 & 100373 & 1994 & 77.25 & -0.15 & 8464.00 & 66376.62 & 0.91 & 0.86 & 0.78 \\
14869 & 101919 & 1994 & 132.99 & 0.06 & 12095.00 & 114903.62 & 1.10 & 0.86 & 0.95 \\
23136 & 103134 & 1994 & 236.25 & 0.00 & 24941.00 & 246745.43 & 0.95 & 1.04 & 0.99 \\
15473 & 101998 & 1994 & 333.92 & 0.02 & 32181.00 & 324241.04 & 1.04 & 0.97 & 1.01 \\
19015 & 102543 & 1994 & 21.73 & -0.03 & 2357.00 & 20181.41 & 0.92 & 0.93 & 0.86 \\
5672 & 100785 & 1994 & 33.70 & -0.00 & 3132.00 & 31152.09 & 1.08 & 0.92 & 0.99 \\
4432 & 100625 & 1994 & 267.91 & -0.01 & 26766.00 & 263051.42 & 1.00 & 0.98 & 0.98 \\
8423 & 101086 & 1994 & 276.10 & 0.15 & 19849.00 & 159897.23 & 1.39 & 0.58 & 0.81 \\
18689 & 102503 & 1994 & 161.00 & 0.12 & 16099.00 & 154924.79 & 1.00 & 0.96 & 0.96 \\
26578 & 103592 & 1994 & 1.23 & 0.08 & 123.00 & 988.37 & 1.00 & 0.80 & 0.80 \\
12384 & 101538 & 1994 & 9.02 & -0.03 & 901.00 & 8610.58 & 1.00 & 0.96 & 0.96 \\
16685 & 102178 & 1994 & 97.29 & 0.03 & 9403.00 & 80794.53 & 1.03 & 0.83 & 0.86 \\
4447 & 100633 & 1994 & 124.26 & 0.02 & 12426.00 & 115909.95 & 1.00 & 0.93 & 0.93 \\
4472 & 100634 & 1994 & 591.69 & -0.01 & 59169.00 & 550325.18 & 1.00 & 0.93 & 0.93 \\
11500 & 101425 & 1994 & 9.19 & -0.05 & 1035.00 & 9723.33 & 0.89 & 1.06 & 0.94 \\
18649 & 102500 & 1994 & 598.89 & -0.06 & 56558.00 & 584980.61 & 1.06 & 0.98 & 1.03 \\
49182 & 240243 & 1994 & 570.64 & -0.17 & 54892.00 & 476251.62 & 1.04 & 0.83 & 0.87 \\
18678 & 102502 & 1994 & 396.91 & 0.05 & 31929.00 & 370675.88 & 1.24 & 0.93 & 1.16 \\
5642 & 100784 & 1994 & 543.48 & 0.05 & 54347.00 & 520449.39 & 1.00 & 0.96 & 0.96 \\
16652 & 102175 & 1994 & 21.62 & 0.01 & 2162.00 & 20621.70 & 1.00 & 0.95 & 0.95 \\
12404 & 101539 & 1994 & 184.45 & 0.01 & 17711.00 & 163661.37 & 1.04 & 0.89 & 0.92 \\
11360 & 101398 & 1994 & 88.28 & -0.16 & 10449.00 & 75951.32 & 0.84 & 0.86 & 0.73 \\
25623 & 103498 & 1994 & 90.18 & 0.01 & 7735.00 & 77380.37 & 1.17 & 0.86 & 1.00 \\
4376 & 100614 & 1994 & 202.32 & -0.03 & 21371.00 & 192175.72 & 0.95 & 0.95 & 0.90 \\
49100 & 240222 & 1994 & 359.30 & -0.03 & 36569.00 & 325893.38 & 0.98 & 0.91 & 0.89 \\
18795 & 102522 & 1994 & 241.77 & 0.00 & 24182.00 & 215184.83 & 1.00 & 0.89 & 0.89 \\
8004 & 101069 & 1994 & 306.70 & 0.01 & 31126.00 & 292554.22 & 0.99 & 0.95 & 0.94 \\
11478 & 101422 & 1994 & 31.90 & -0.02 & 2711.00 & 22969.08 & 1.18 & 0.72 & 0.85 \\
5704 & 100789 & 1994 & 7.02 & -0.05 & 583.00 & 6877.54 & 1.20 & 0.98 & 1.18 \\
4394 & 100622 & 1994 & 44.39 & 0.06 & 3961.00 & 36359.21 & 1.12 & 0.82 & 0.92 \\
18782 & 102515 & 1994 & 102.69 & 0.00 & 9729.00 & 81744.68 & 1.06 & 0.80 & 0.84 \\
11428 & 101402 & 1994 & 30.17 & -0.04 & 3124.00 & 26189.01 & 0.97 & 0.87 & 0.84 \\
18769 & 102508 & 1994 & 111.47 & 0.01 & 11141.00 & 98356.60 & 1.00 & 0.88 & 0.88 \\
16621 & 102166 & 1994 & 202.50 & -0.10 & 20205.00 & 178763.45 & 1.00 & 0.88 & 0.88 \\
18751 & 102507 & 1994 & 409.81 & -0.04 & 43761.00 & 397741.63 & 0.94 & 0.97 & 0.91 \\
16642 & 102173 & 1994 & 30.80 & -0.04 & 3870.00 & 25794.68 & 0.80 & 0.84 & 0.67 \\
1179 & 100159 & 1994 & 117.29 & -0.03 & 11195.00 & 103908.34 & 1.05 & 0.89 & 0.93 \\
18634 & 102493 & 1994 & 170.93 & -0.07 & 17345.00 & 155624.85 & 0.99 & 0.91 & 0.90 \\
18631 & 102492 & 1994 & 100.84 & -0.08 & 10377.00 & 83131.53 & 0.97 & 0.82 & 0.80 \\
11537 & 101427 & 1994 & 158.85 & -0.07 & 14550.00 & 118224.58 & 1.09 & 0.74 & 0.81 \\
18520 & 102470 & 1994 & 457.16 & -0.06 & 56370.00 & 505690.99 & 0.81 & 1.11 & 0.90 \\
4556 & 100639 & 1994 & 215.28 & -0.09 & 23518.00 & 210063.38 & 0.92 & 0.98 & 0.89 \\
18507 & 102469 & 1994 & 157.30 & -0.02 & 16885.00 & 151032.41 & 0.93 & 0.96 & 0.89 \\
18466 & 102462 & 1994 & 3.17 & -0.04 & 317.00 & 2941.74 & 1.00 & 0.93 & 0.93 \\
18449 & 102461 & 1994 & 745.81 & 0.05 & 60869.00 & 542933.51 & 1.23 & 0.73 & 0.89 \\
18534 & 102471 & 1994 & 87.69 & -0.05 & 9771.00 & 77946.67 & 0.90 & 0.89 & 0.80 \\
18441 & 102457 & 1994 & 2.80 & 0.00 & 273.00 & 2204.44 & 1.02 & 0.79 & 0.81 \\
57806 & 401070 & 1994 & 21.35 & -0.02 & 2229.00 & 20683.08 & 0.96 & 0.97 & 0.93 \\
18423 & 102452 & 1994 & 71.88 & -0.06 & 6087.00 & 50900.99 & 1.18 & 0.71 & 0.84 \\
4590 & 100642 & 1994 & 618.73 & -0.03 & 60929.00 & 581323.36 & 1.02 & 0.94 & 0.95 \\
16769 & 102191 & 1994 & 35.90 & -0.02 & 3590.00 & 34214.04 & 1.00 & 0.95 & 0.95 \\
12298 & 101532 & 1994 & 56.17 & -0.03 & 7921.00 & 79289.45 & 0.71 & 1.41 & 1.00 \\
18393 & 102447 & 1994 & 196.73 & 0.00 & 19673.00 & 159512.99 & 1.00 & 0.81 & 0.81 \\
1147 & 100157 & 1994 & 96.08 & 0.01 & 9453.00 & 86400.67 & 1.02 & 0.90 & 0.91 \\
18438 & 102455 & 1994 & 16.49 & -0.01 & 1684.00 & 16366.49 & 0.98 & 0.99 & 0.97 \\
4483 & 100635 & 1994 & 17.40 & 0.10 & 1740.00 & 16410.70 & 1.00 & 0.94 & 0.94 \\
5598 & 100773 & 1994 & 663.00 & 0.03 & 64156.00 & 638609.52 & 1.03 & 0.96 & 1.00 \\
18539 & 102474 & 1994 & 215.34 & -0.02 & 22223.00 & 219293.99 & 0.97 & 1.02 & 0.99 \\
5634 & 100780 & 1994 & 38.18 & 0.00 & 3739.00 & 33239.80 & 1.02 & 0.87 & 0.89 \\
18600 & 102491 & 1994 & 78.82 & -0.08 & 7728.00 & 68173.23 & 1.02 & 0.86 & 0.88 \\
18584 & 102490 & 1994 & 36.58 & -0.01 & 3658.00 & 36305.36 & 1.00 & 0.99 & 0.99 \\
12353 & 101537 & 1994 & 192.50 & 0.01 & 19290.00 & 183554.81 & 1.00 & 0.95 & 0.95 \\
26515 & 103590 & 1994 & 55.52 & 0.05 & 5551.00 & 53208.75 & 1.00 & 0.96 & 0.96 \\
16715 & 102179 & 1994 & 40.03 & 0.03 & 3982.00 & 36761.42 & 1.01 & 0.92 & 0.92 \\
18572 & 102489 & 1994 & 11.57 & 0.08 & 1075.00 & 11026.06 & 1.08 & 0.95 & 1.03 \\
16719 & 102182 & 1994 & 73.41 & -0.01 & 6654.00 & 64688.35 & 1.10 & 0.88 & 0.97 \\
12302 & 101534 & 1994 & 422.64 & -0.01 & 45180.00 & 419025.41 & 0.94 & 0.99 & 0.93 \\
8391 & 101085 & 1994 & 202.40 & -0.07 & 23462.00 & 208173.98 & 0.86 & 1.03 & 0.89 \\
18558 & 102483 & 1994 & 82.35 & -0.02 & 9176.00 & 82686.26 & 0.90 & 1.00 & 0.90 \\
57786 & 401058 & 1994 & 24.77 & -0.10 & 3005.00 & 23744.47 & 0.82 & 0.96 & 0.79 \\
25680 & 103510 & 1994 & 80.88 & -0.06 & 8153.00 & 72678.27 & 0.99 & 0.90 & 0.89 \\
12320 & 101536 & 1994 & 394.44 & -0.04 & 39537.00 & 364881.74 & 1.00 & 0.93 & 0.92 \\
16736 & 102183 & 1994 & 117.67 & -0.01 & 10906.00 & 105279.23 & 1.08 & 0.89 & 0.97 \\
4522 & 100637 & 1994 & 889.53 & 0.02 & 88953.00 & 771582.55 & 1.00 & 0.87 & 0.87 \\
18548 & 102482 & 1994 & 63.95 & -0.01 & 6392.00 & 58461.79 & 1.00 & 0.91 & 0.91 \\
5617 & 100775 & 1994 & 232.92 & 0.03 & 24501.00 & 247845.35 & 0.95 & 1.06 & 1.01 \\
5722 & 100790 & 1994 & 12.98 & -0.00 & 1391.00 & 12954.06 & 0.93 & 1.00 & 0.93 \\
18812 & 102523 & 1994 & 771.93 & -0.01 & 72405.00 & 712720.86 & 1.07 & 0.92 & 0.98 \\
11254 & 101380 & 1994 & 230.65 & -0.04 & 24484.00 & 227953.87 & 0.94 & 0.99 & 0.93 \\
5849 & 100808 & 1994 & 71.07 & -0.13 & 7159.00 & 70919.38 & 0.99 & 1.00 & 0.99 \\
16277 & 102121 & 1994 & 6.31 & -0.02 & 854.00 & 6299.53 & 0.74 & 1.00 & 0.74 \\
11265 & 101381 & 1994 & 53.33 & -0.09 & 5909.00 & 56161.13 & 0.90 & 1.05 & 0.95 \\
19264 & 102578 & 1994 & 95.37 & -0.03 & 17751.00 & 152346.62 & 0.54 & 1.60 & 0.86 \\
19248 & 102575 & 1994 & 81.23 & -0.03 & 7961.00 & 70369.07 & 1.02 & 0.87 & 0.88 \\
16302 & 102124 & 1994 & 636.77 & 0.01 & 70527.00 & 673451.49 & 0.90 & 1.06 & 0.95 \\
25548 & 103496 & 1994 & 264.80 & -0.04 & 24370.00 & 234108.70 & 1.09 & 0.88 & 0.96 \\
19297 & 102588 & 1994 & 151.71 & -0.03 & 15171.00 & 143126.18 & 1.00 & 0.94 & 0.94 \\
4207 & 100575 & 1994 & 17.68 & -0.06 & 1829.00 & 17384.43 & 0.97 & 0.98 & 0.95 \\
11278 & 101390 & 1994 & 1718.14 & -0.02 & 148400.00 & 1425327.23 & 1.16 & 0.83 & 0.96 \\
25579 & 103497 & 1994 & 10.47 & -0.01 & 947.00 & 9477.61 & 1.11 & 0.91 & 1.00 \\
16341 & 102129 & 1994 & 5.37 & -0.02 & 570.00 & 5380.78 & 0.94 & 1.00 & 0.94 \\
49057 & 240212 & 1994 & 861.32 & 0.14 & 84656.00 & 805883.71 & 1.02 & 0.94 & 0.95 \\
26688 & 103600 & 1994 & 3.23 & -0.17 & 323.00 & 2853.32 & 1.00 & 0.88 & 0.88 \\
4220 & 100589 & 1994 & 52.44 & -0.01 & 4980.00 & 43376.54 & 1.05 & 0.83 & 0.87 \\
5818 & 100804 & 1994 & 233.93 & 0.04 & 23360.00 & 217960.90 & 1.00 & 0.93 & 0.93 \\
4124 & 100559 & 1994 & 18.70 & -0.04 & 1995.00 & 18422.35 & 0.94 & 0.99 & 0.92 \\
19395 & 102600 & 1994 & 191.51 & -0.01 & 19132.00 & 174506.12 & 1.00 & 0.91 & 0.91 \\
16229 & 102102 & 1994 & 407.00 & -0.03 & 40696.00 & 372530.83 & 1.00 & 0.92 & 0.92 \\
8524 & 101089 & 1994 & 10.10 & -0.00 & 640.00 & 5482.16 & 1.58 & 0.54 & 0.86 \\
19361 & 102599 & 1994 & 501.95 & 0.15 & 49493.00 & 435399.73 & 1.01 & 0.87 & 0.88 \\
19327 & 102597 & 1994 & 10.19 & 0.06 & 962.00 & 9573.25 & 1.06 & 0.94 & 1.00 \\
19323 & 102594 & 1994 & 15.50 & 0.02 & 2251.00 & 22293.85 & 0.69 & 1.44 & 0.99 \\
7938 & 101067 & 1994 & 167.80 & 0.15 & 19726.00 & 156000.53 & 0.85 & 0.93 & 0.79 \\
53454 & 350572 & 1994 & 37.85 & -0.00 & 3865.00 & 38207.20 & 0.98 & 1.01 & 0.99 \\
19317 & 102591 & 1994 & 39.90 & -0.04 & 4257.00 & 40450.45 & 0.94 & 1.01 & 0.95 \\
316 & 100036 & 1994 & 61.23 & -0.05 & 6104.00 & 54077.44 & 1.00 & 0.88 & 0.89 \\
11213 & 101376 & 1994 & 164.70 & -0.01 & 16780.00 & 168143.36 & 0.98 & 1.02 & 1.00 \\
1248 & 100167 & 1994 & 130.83 & 0.02 & 13884.00 & 133953.83 & 0.94 & 1.02 & 0.96 \\
16267 & 102113 & 1994 & 179.43 & -0.04 & 17941.00 & 187196.97 & 1.00 & 1.04 & 1.04 \\
5865 & 100809 & 1994 & 2447.88 & -0.04 & 244530.00 & 1997684.66 & 1.00 & 0.82 & 0.82 \\
11243 & 101379 & 1994 & 137.77 & 0.01 & 14159.00 & 133541.15 & 0.97 & 0.97 & 0.94 \\
26752 & 103605 & 1994 & 66.52 & -0.03 & 7926.00 & 66975.26 & 0.84 & 1.01 & 0.85 \\
16251 & 102105 & 1994 & 72.10 & -0.05 & 7215.00 & 63227.57 & 1.00 & 0.88 & 0.88 \\
827 & 100098 & 1994 & 6.70 & 0.03 & 635.00 & 5840.87 & 1.06 & 0.87 & 0.92 \\
11347 & 101397 & 1994 & 30.05 & -0.21 & 3578.00 & 27271.71 & 0.84 & 0.91 & 0.76 \\
18985 & 102540 & 1994 & 6.47 & -0.05 & 612.00 & 5929.45 & 1.06 & 0.92 & 0.97 \\
16458 & 102150 & 1994 & 364.91 & -0.06 & 36491.00 & 294271.02 & 1.00 & 0.81 & 0.81 \\
16489 & 102151 & 1994 & 11.74 & 0.02 & 1174.00 & 10276.08 & 1.00 & 0.88 & 0.88 \\
18953 & 102531 & 1994 & 6.29 & 0.01 & 629.00 & 6094.90 & 1.00 & 0.97 & 0.97 \\
189 & 100018 & 1994 & 24.01 & -0.04 & 2415.00 & 20715.07 & 0.99 & 0.86 & 0.86 \\
16511 & 102152 & 1994 & 107.47 & 0.02 & 10735.00 & 102919.35 & 1.00 & 0.96 & 0.96 \\
4298 & 100603 & 1994 & 458.47 & 0.08 & 45810.00 & 417862.99 & 1.00 & 0.91 & 0.91 \\
26612 & 103593 & 1994 & 12668.30 & -0.00 & 1266830.00 & 11665694.76 & 1.00 & 0.92 & 0.92 \\
18905 & 102527 & 1994 & 85.00 & -0.04 & 9927.00 & 88189.25 & 0.86 & 1.04 & 0.89 \\
44392 & 109300 & 1994 & 154.85 & -0.02 & 15900.00 & 137805.04 & 0.97 & 0.89 & 0.87 \\
18874 & 102525 & 1994 & 74.95 & -0.04 & 7540.00 & 68379.97 & 0.99 & 0.91 & 0.91 \\
18843 & 102524 & 1994 & 209.93 & 0.01 & 24520.00 & 231895.22 & 0.86 & 1.10 & 0.95 \\
5746 & 100791 & 1994 & 35.01 & -0.01 & 3463.00 & 31886.35 & 1.01 & 0.91 & 0.92 \\
19017 & 102544 & 1994 & 166.15 & -0.04 & 12744.00 & 110381.26 & 1.30 & 0.66 & 0.87 \\
12512 & 101545 & 1994 & 41.73 & 0.01 & 4172.00 & 34675.40 & 1.00 & 0.83 & 0.83 \\
16408 & 102134 & 1994 & 11.01 & 0.02 & 1070.00 & 9486.88 & 1.03 & 0.86 & 0.89 \\
8488 & 101088 & 1994 & 156.60 & 0.17 & 15220.00 & 125779.77 & 1.03 & 0.80 & 0.83 \\
4248 & 100598 & 1994 & 247.07 & -0.02 & 24706.00 & 221922.32 & 1.00 & 0.90 & 0.90 \\
19139 & 102551 & 1994 & 33.50 & -0.04 & 3350.00 & 29539.16 & 1.00 & 0.88 & 0.88 \\
19132 & 102550 & 1994 & 48.44 & 0.01 & 4585.00 & 42572.46 & 1.06 & 0.88 & 0.93 \\
16356 & 102130 & 1994 & 439.81 & 0.01 & 46907.00 & 378691.53 & 0.94 & 0.86 & 0.81 \\
26644 & 103595 & 1994 & 87.67 & -0.01 & 8766.00 & 84034.49 & 1.00 & 0.96 & 0.96 \\
19073 & 102548 & 1994 & 115.41 & -0.02 & 9998.00 & 89319.89 & 1.15 & 0.77 & 0.89 \\
19065 & 102547 & 1994 & 86.49 & -0.02 & 8283.00 & 70818.21 & 1.04 & 0.82 & 0.85 \\
16391 & 102132 & 1994 & 31.65 & 0.01 & 3135.00 & 30129.14 & 1.01 & 0.95 & 0.96 \\
19057 & 102546 & 1994 & 66.14 & -0.00 & 5830.00 & 58315.00 & 1.13 & 0.88 & 1.00 \\
5778 & 100792 & 1994 & 6.93 & 0.00 & 1049.00 & 9137.79 & 0.66 & 1.32 & 0.87 \\
19041 & 102545 & 1994 & 12.65 & -0.04 & 1800.00 & 17814.52 & 0.70 & 1.41 & 0.99 \\
7972 & 101068 & 1994 & 36441.50 & -0.02 & 3616300.00 & 33768617.16 & 1.01 & 0.93 & 0.93 \\
26478 & 103582 & 1994 & 11.08 & -0.05 & 1036.00 & 9741.91 & 1.07 & 0.88 & 0.94 \\
25689 & 103514 & 1994 & 869.04 & -0.02 & 85904.00 & 737956.54 & 1.01 & 0.85 & 0.86 \\
857 & 100099 & 1994 & 10.00 & 0.05 & 936.00 & 8860.96 & 1.07 & 0.89 & 0.95 \\
17132 & 102258 & 1994 & 718.87 & -0.02 & 75494.00 & 668900.54 & 0.95 & 0.93 & 0.89 \\
12096 & 101503 & 1994 & 225.81 & 0.01 & 22600.00 & 196964.56 & 1.00 & 0.87 & 0.87 \\
17142 & 102259 & 1994 & 200.96 & 0.00 & 20214.00 & 182323.15 & 0.99 & 0.91 & 0.90 \\
5319 & 100753 & 1994 & 1160.21 & -0.01 & 145818.00 & 1087808.50 & 0.80 & 0.94 & 0.75 \\
4959 & 100697 & 1994 & 37.68 & -0.06 & 3967.00 & 36049.28 & 0.95 & 0.96 & 0.91 \\
17671 & 102342 & 1994 & 35.20 & -0.00 & 3523.00 & 34062.71 & 1.00 & 0.97 & 0.97 \\
17650 & 102334 & 1994 & 58.19 & 0.02 & 5165.00 & 44112.90 & 1.13 & 0.76 & 0.85 \\
4954 & 100696 & 1994 & 4.33 & 0.03 & 500.00 & 4121.60 & 0.87 & 0.95 & 0.82 \\
12059 & 101494 & 1994 & 151.87 & -0.07 & 14242.00 & 135163.68 & 1.07 & 0.89 & 0.95 \\
4992 & 100698 & 1994 & 44.34 & 0.04 & 4315.00 & 36921.00 & 1.03 & 0.83 & 0.86 \\
268 & 100030 & 1994 & 76.42 & -0.08 & 7583.00 & 64468.81 & 1.01 & 0.84 & 0.85 \\
17614 & 102321 & 1994 & 142.60 & -0.03 & 14260.00 & 132277.82 & 1.00 & 0.93 & 0.93 \\
17577 & 102319 & 1994 & 603.92 & 0.01 & 60392.00 & 575517.72 & 1.00 & 0.95 & 0.95 \\
11950 & 101473 & 1994 & 640.70 & -0.01 & 64000.00 & 578027.78 & 1.00 & 0.90 & 0.90 \\
17156 & 102261 & 1994 & 832.14 & -0.03 & 88070.00 & 807742.82 & 0.94 & 0.97 & 0.92 \\
17543 & 102318 & 1994 & 2820.21 & -0.03 & 282021.00 & 2651474.49 & 1.00 & 0.94 & 0.94 \\
17702 & 102349 & 1994 & 95.52 & 0.01 & 9315.00 & 79950.02 & 1.03 & 0.84 & 0.86 \\
26281 & 103558 & 1994 & 116.32 & -0.02 & 12516.00 & 121544.72 & 0.93 & 1.04 & 0.97 \\
11838 & 101463 & 1994 & 92.00 & 0.01 & 9333.00 & 89432.71 & 0.99 & 0.97 & 0.96 \\
17793 & 102358 & 1994 & 43.83 & 0.05 & 4380.00 & 42608.75 & 1.00 & 0.97 & 0.97 \\
11869 & 101464 & 1994 & 330.34 & -0.15 & 34385.00 & 277996.41 & 0.96 & 0.84 & 0.81 \\
4933 & 100695 & 1994 & 59.93 & 0.00 & 6134.00 & 49841.54 & 0.98 & 0.83 & 0.81 \\
17069 & 102241 & 1994 & 169.88 & -0.05 & 18132.00 & 150442.90 & 0.94 & 0.89 & 0.83 \\
11899 & 101465 & 1994 & 29.26 & -0.03 & 2976.00 & 27736.11 & 0.98 & 0.95 & 0.93 \\
17777 & 102357 & 1994 & 45.41 & 0.05 & 4384.00 & 36270.02 & 1.04 & 0.80 & 0.83 \\
25955 & 103531 & 1994 & 445.53 & -0.02 & 44553.00 & 420799.47 & 1.00 & 0.94 & 0.94 \\
17100 & 102257 & 1994 & 122.13 & 0.01 & 12315.00 & 116748.11 & 0.99 & 0.96 & 0.95 \\
17746 & 102356 & 1994 & 1.01 & -0.11 & 101.00 & 992.54 & 1.00 & 0.98 & 0.98 \\
8123 & 101076 & 1994 & 32.10 & -0.10 & 5313.00 & 42709.35 & 0.60 & 1.33 & 0.80 \\
45875 & 200153 & 1994 & 1.89 & 0.03 & 186.00 & 1645.79 & 1.01 & 0.87 & 0.88 \\
12110 & 101507 & 1994 & 46.70 & -0.04 & 4672.00 & 44657.25 & 1.00 & 0.96 & 0.96 \\
5340 & 100754 & 1994 & 407.43 & -0.01 & 40596.00 & 381407.37 & 1.00 & 0.94 & 0.94 \\
17725 & 102350 & 1994 & 10.55 & 0.03 & 826.00 & 7342.48 & 1.28 & 0.70 & 0.89 \\
8240 & 101080 & 1994 & 61.30 & -0.07 & 7036.00 & 52009.86 & 0.87 & 0.85 & 0.74 \\
12113 & 101508 & 1994 & 15.19 & -0.09 & 2825.00 & 27308.13 & 0.54 & 1.80 & 0.97 \\
17185 & 102270 & 1994 & 105.20 & -0.01 & 10509.00 & 96911.84 & 1.00 & 0.92 & 0.92 \\
5285 & 100746 & 1994 & 495.18 & 0.00 & 49553.00 & 478467.22 & 1.00 & 0.97 & 0.97 \\
17414 & 102305 & 1994 & 182.99 & -0.04 & 17620.00 & 176248.94 & 1.04 & 0.96 & 1.00 \\
12027 & 101488 & 1994 & 29.47 & -0.01 & 2891.00 & 27345.96 & 1.02 & 0.93 & 0.95 \\
17399 & 102286 & 1994 & 28.40 & -0.03 & 2772.00 & 25395.35 & 1.02 & 0.89 & 0.92 \\
17246 & 102274 & 1994 & 131.26 & -0.02 & 13074.00 & 107391.36 & 1.00 & 0.82 & 0.82 \\
12044 & 101491 & 1994 & 55.03 & -0.14 & 5362.00 & 50528.48 & 1.03 & 0.92 & 0.94 \\
17378 & 102284 & 1994 & 147.39 & -0.01 & 14462.00 & 121686.14 & 1.02 & 0.83 & 0.84 \\
5151 & 100727 & 1994 & 439.51 & -0.03 & 45698.00 & 448576.55 & 0.96 & 1.02 & 0.98 \\
26134 & 103544 & 1994 & 785.68 & -0.00 & 78568.00 & 761898.97 & 1.00 & 0.97 & 0.97 \\
5169 & 100730 & 1994 & 338.08 & -0.05 & 34017.00 & 323499.80 & 0.99 & 0.96 & 0.95 \\
5209 & 100736 & 1994 & 243.11 & -0.07 & 22857.00 & 236817.60 & 1.06 & 0.97 & 1.04 \\
8172 & 101078 & 1994 & 3.70 & 0.02 & 364.00 & 3484.98 & 1.02 & 0.94 & 0.96 \\
973 & 100113 & 1994 & 757.02 & -0.03 & 74910.00 & 749424.54 & 1.01 & 0.99 & 1.00 \\
17304 & 102280 & 1994 & 1261.39 & -0.01 & 126633.00 & 1060767.19 & 1.00 & 0.84 & 0.84 \\
17286 & 102278 & 1994 & 205.82 & -0.02 & 20797.00 & 197709.14 & 0.99 & 0.96 & 0.95 \\
57961 & 410010 & 1994 & 85.50 & -0.04 & 8430.00 & 71091.87 & 1.01 & 0.83 & 0.84 \\
284 & 100033 & 1994 & 38.83 & -0.05 & 3872.00 & 35343.35 & 1.00 & 0.91 & 0.91 \\
26168 & 103545 & 1994 & 7958.98 & -0.01 & 795890.00 & 7283901.37 & 1.00 & 0.92 & 0.92 \\
5108 & 100724 & 1994 & 32.92 & -0.03 & 1986.00 & 18396.92 & 1.66 & 0.56 & 0.93 \\
5242 & 100741 & 1994 & 136.51 & -0.01 & 15004.00 & 133995.66 & 0.91 & 0.98 & 0.89 \\
5045 & 100710 & 1994 & 12.52 & 0.03 & 1228.00 & 11024.45 & 1.02 & 0.88 & 0.90 \\
26009 & 103533 & 1994 & 708.08 & 0.10 & 70808.00 & 586139.52 & 1.00 & 0.83 & 0.83 \\
57927 & 410003 & 1994 & 144.03 & -0.08 & 14403.00 & 130895.05 & 1.00 & 0.91 & 0.91 \\
8145 & 101077 & 1994 & 7.60 & -0.07 & 781.00 & 6482.87 & 0.97 & 0.85 & 0.83 \\
26023 & 103535 & 1994 & 69.65 & 0.02 & 6965.00 & 69416.80 & 1.00 & 1.00 & 1.00 \\
26239 & 103547 & 1994 & 2109.28 & 0.04 & 210928.00 & 1918629.90 & 1.00 & 0.91 & 0.91 \\
26208 & 103546 & 1994 & 10672.97 & -0.00 & 1140630.00 & 9211242.05 & 0.94 & 0.86 & 0.81 \\
17492 & 102314 & 1994 & 244.47 & -0.01 & 42941.00 & 415633.20 & 0.57 & 1.70 & 0.97 \\
12010 & 101477 & 1994 & 146.28 & -0.00 & 12494.00 & 119998.19 & 1.17 & 0.82 & 0.96 \\
8203 & 101079 & 1994 & 96.70 & 0.00 & 9798.00 & 83844.67 & 0.99 & 0.87 & 0.86 \\
26053 & 103536 & 1994 & 31.41 & 0.01 & 3141.00 & 30785.64 & 1.00 & 0.98 & 0.98 \\
5075 & 100715 & 1994 & 86.83 & -0.12 & 8441.00 & 79787.97 & 1.03 & 0.92 & 0.95 \\
5077 & 100723 & 1994 & 15.64 & 0.01 & 1269.00 & 12640.54 & 1.23 & 0.81 & 1.00 \\
17481 & 102313 & 1994 & 347.63 & -0.10 & 34623.00 & 336570.81 & 1.00 & 0.97 & 0.97 \\
1017 & 100127 & 1994 & 249.25 & 0.28 & 24924.00 & 213763.81 & 1.00 & 0.86 & 0.86 \\
17431 & 102306 & 1994 & 1094.83 & -0.03 & 107678.00 & 990639.75 & 1.02 & 0.90 & 0.92 \\
1067 & 100150 & 1994 & 10.00 & -0.06 & 1019.00 & 9782.17 & 0.98 & 0.98 & 0.96 \\
17806 & 102364 & 1994 & 238.59 & 0.01 & 23000.00 & 222724.28 & 1.04 & 0.93 & 0.97 \\
4697 & 100667 & 1994 & 11.20 & -0.04 & 1200.00 & 9539.18 & 0.93 & 0.85 & 0.79 \\
887 & 100101 & 1994 & 2.41 & -0.02 & 301.00 & 2845.56 & 0.80 & 1.18 & 0.95 \\
5500 & 100769 & 1994 & 357.00 & 0.01 & 35700.00 & 350237.07 & 1.00 & 0.98 & 0.98 \\
18265 & 102419 & 1994 & 247.20 & -0.00 & 24755.00 & 206535.12 & 1.00 & 0.84 & 0.83 \\
26380 & 103573 & 1994 & 119.02 & -0.02 & 11928.00 & 109643.57 & 1.00 & 0.92 & 0.92 \\
16871 & 102213 & 1994 & 132.63 & -0.03 & 13499.00 & 115786.05 & 0.98 & 0.87 & 0.86 \\
4734 & 100670 & 1994 & 62.00 & 0.06 & 5934.00 & 60694.27 & 1.04 & 0.98 & 1.02 \\
25854 & 103525 & 1994 & 3265.16 & 0.01 & 326515.00 & 3248234.23 & 1.00 & 0.99 & 0.99 \\
11675 & 101456 & 1994 & 33.72 & -0.10 & 3810.00 & 33767.93 & 0.89 & 1.00 & 0.89 \\
12209 & 101519 & 1994 & 136.86 & -0.02 & 12617.00 & 138204.35 & 1.08 & 1.01 & 1.10 \\
26369 & 103572 & 1994 & 21.70 & -0.00 & 2208.00 & 21443.01 & 0.98 & 0.99 & 0.97 \\
25821 & 103524 & 1994 & 9356.35 & 0.01 & 935630.00 & 9329993.05 & 1.00 & 1.00 & 1.00 \\
8092 & 101074 & 1994 & 79.00 & -0.07 & 7673.00 & 61818.58 & 1.03 & 0.78 & 0.81 \\
25787 & 103523 & 1994 & 967.29 & -0.02 & 96729.00 & 918837.16 & 1.00 & 0.95 & 0.95 \\
25753 & 103521 & 1994 & 296.93 & -0.01 & 29693.00 & 297197.46 & 1.00 & 1.00 & 1.00 \\
4620 & 100644 & 1994 & 52.70 & 0.04 & 5062.00 & 47874.33 & 1.04 & 0.91 & 0.95 \\
25721 & 103520 & 1994 & 86.17 & 0.03 & 8617.00 & 85881.71 & 1.00 & 1.00 & 1.00 \\
18349 & 102446 & 1994 & 2.72 & -0.02 & 271.00 & 2264.91 & 1.00 & 0.83 & 0.84 \\
4636 & 100659 & 1994 & 150.46 & -0.01 & 15042.00 & 148363.18 & 1.00 & 0.99 & 0.99 \\
8353 & 101084 & 1994 & 227.40 & 0.11 & 22315.00 & 195288.26 & 1.02 & 0.86 & 0.88 \\
18348 & 102441 & 1994 & 326.30 & -0.13 & 32641.00 & 262827.32 & 1.00 & 0.81 & 0.81 \\
16841 & 102197 & 1994 & 43.94 & 0.03 & 4393.00 & 42113.23 & 1.00 & 0.96 & 0.96 \\
49261 & 240261 & 1994 & 145.24 & 0.03 & 12916.00 & 116075.47 & 1.12 & 0.80 & 0.90 \\
5531 & 100771 & 1994 & 33.00 & -0.08 & 3044.00 & 31493.71 & 1.08 & 0.95 & 1.03 \\
1117 & 100154 & 1994 & 168.53 & -0.06 & 18890.00 & 178377.33 & 0.89 & 1.06 & 0.94 \\
18315 & 102425 & 1994 & 698.30 & 0.02 & 69846.00 & 564118.74 & 1.00 & 0.81 & 0.81 \\
57828 & 401081 & 1994 & 33.52 & -0.10 & 4231.00 & 29289.77 & 0.79 & 0.87 & 0.69 \\
4670 & 100660 & 1994 & 102.68 & 0.03 & 12525.00 & 121155.23 & 0.82 & 1.18 & 0.97 \\
57835 & 401082 & 1994 & 10.99 & -0.12 & 1267.00 & 11881.77 & 0.87 & 1.08 & 0.94 \\
12239 & 101530 & 1994 & 338.94 & 0.02 & 30368.00 & 280181.37 & 1.12 & 0.83 & 0.92 \\
26427 & 103580 & 1994 & 93.18 & -0.02 & 9318.00 & 89045.32 & 1.00 & 0.96 & 0.96 \\
1132 & 100155 & 1994 & 88.40 & -0.01 & 8706.00 & 71319.00 & 1.02 & 0.81 & 0.82 \\
26336 & 103570 & 1994 & 11.17 & -0.02 & 728.00 & 7564.87 & 1.53 & 0.68 & 1.04 \\
4756 & 100671 & 1994 & 78.39 & -0.01 & 7541.00 & 66726.72 & 1.04 & 0.85 & 0.88 \\
18174 & 102414 & 1994 & 302.80 & -0.00 & 30922.00 & 291466.18 & 0.98 & 0.96 & 0.94 \\
11775 & 101461 & 1994 & 870.54 & -0.01 & 86609.00 & 774723.75 & 1.01 & 0.89 & 0.89 \\
17898 & 102372 & 1994 & 1626.87 & 0.03 & 167725.00 & 1416431.70 & 0.97 & 0.87 & 0.84 \\
4865 & 100687 & 1994 & 78.04 & 0.00 & 7255.00 & 72161.60 & 1.08 & 0.92 & 0.99 \\
923 & 100111 & 1994 & 38.20 & -0.02 & 3820.00 & 38006.80 & 1.00 & 0.99 & 0.99 \\
4870 & 100688 & 1994 & 74.42 & -0.08 & 6661.00 & 60621.95 & 1.12 & 0.81 & 0.91 \\
11806 & 101462 & 1994 & 180.58 & 0.06 & 10650.00 & 92684.90 & 1.70 & 0.51 & 0.87 \\
17878 & 102371 & 1994 & 7.85 & -0.02 & 877.00 & 8278.28 & 0.90 & 1.05 & 0.94 \\
5396 & 100760 & 1994 & 816.65 & -0.05 & 100120.00 & 869570.05 & 0.82 & 1.06 & 0.87 \\
12153 & 101513 & 1994 & 56.95 & 0.01 & 5939.00 & 53268.54 & 0.96 & 0.94 & 0.90 \\
4875 & 100690 & 1994 & 32.62 & -0.05 & 3018.00 & 27325.43 & 1.08 & 0.84 & 0.91 \\
4883 & 100691 & 1994 & 153.32 & -0.03 & 16836.00 & 154488.35 & 0.91 & 1.01 & 0.92 \\
45313 & 200022 & 1994 & 192.45 & -0.12 & 20151.00 & 157104.98 & 0.96 & 0.82 & 0.78 \\
17837 & 102365 & 1994 & 279.06 & -0.04 & 25122.00 & 241448.57 & 1.11 & 0.87 & 0.96 \\
17034 & 102231 & 1994 & 515.07 & -0.05 & 51507.00 & 459636.59 & 1.00 & 0.89 & 0.89 \\
12119 & 101511 & 1994 & 134.82 & -0.11 & 14133.00 & 127497.73 & 0.95 & 0.95 & 0.90 \\
257 & 100022 & 1994 & 34.13 & -0.07 & 3413.00 & 32020.48 & 1.00 & 0.94 & 0.94 \\
4903 & 100692 & 1994 & 310.85 & -0.02 & 30437.00 & 263274.52 & 1.02 & 0.85 & 0.86 \\
57896 & 401372 & 1994 & 10.01 & -0.10 & 1111.00 & 9195.56 & 0.90 & 0.92 & 0.83 \\
8271 & 101081 & 1994 & 94.20 & -0.04 & 12054.00 & 86276.46 & 0.78 & 0.92 & 0.72 \\
16962 & 102224 & 1994 & 606.38 & -0.01 & 60538.00 & 588065.87 & 1.00 & 0.97 & 0.97 \\
11742 & 101460 & 1994 & 2526.94 & 0.01 & 253039.00 & 2094032.82 & 1.00 & 0.83 & 0.83 \\
25888 & 103526 & 1994 & 1074.80 & -0.02 & 107480.00 & 984532.41 & 1.00 & 0.92 & 0.92 \\
8312 & 101082 & 1994 & 796.10 & 0.21 & 68904.00 & 578295.52 & 1.16 & 0.73 & 0.84 \\
18124 & 102404 & 1994 & 178.42 & 0.01 & 16680.00 & 163505.94 & 1.07 & 0.92 & 0.98 \\
18115 & 102399 & 1994 & 6.10 & -0.01 & 595.00 & 4902.84 & 1.03 & 0.80 & 0.82 \\
25917 & 103529 & 1994 & 596.05 & -0.01 & 59605.00 & 571462.64 & 1.00 & 0.96 & 0.96 \\
5440 & 100763 & 1994 & 269.37 & -0.10 & 26746.00 & 214154.85 & 1.01 & 0.80 & 0.80 \\
18075 & 102396 & 1994 & 18.27 & -0.09 & 1637.00 & 13915.04 & 1.12 & 0.76 & 0.85 \\
18037 & 102387 & 1994 & 14.78 & 0.01 & 1478.00 & 14742.61 & 1.00 & 1.00 & 1.00 \\
18002 & 102386 & 1994 & 106.72 & -0.03 & 10579.00 & 97511.74 & 1.01 & 0.91 & 0.92 \\
12184 & 101518 & 1994 & 39.19 & -0.02 & 3892.00 & 37644.85 & 1.01 & 0.96 & 0.97 \\
4835 & 100685 & 1994 & 5.83 & 0.08 & 511.00 & 4844.18 & 1.14 & 0.83 & 0.95 \\
11708 & 101457 & 1994 & 182.23 & 0.06 & 21756.00 & 195986.11 & 0.84 & 1.08 & 0.90 \\
17988 & 102383 & 1994 & 6.34 & -0.03 & 687.00 & 5595.71 & 0.92 & 0.88 & 0.81 \\
17962 & 102377 & 1994 & 90.10 & -0.04 & 9001.00 & 74309.13 & 1.00 & 0.82 & 0.83 \\
19429 & 102601 & 1994 & 2010.91 & -0.01 & 201176.00 & 1830509.73 & 1.00 & 0.91 & 0.91 \\
25339 & 103475 & 1994 & 2.02 & -0.08 & 181.00 & 1527.18 & 1.11 & 0.76 & 0.84 \\
20104 & 102667 & 1994 & 1533.70 & 0.11 & 155371.00 & 1341606.79 & 0.99 & 0.87 & 0.86 \\
7865 & 101064 & 1994 & 314.20 & -0.15 & 36015.00 & 264076.45 & 0.87 & 0.84 & 0.73 \\
20043 & 102664 & 1994 & 408.43 & -0.01 & 41057.00 & 374624.03 & 0.99 & 0.92 & 0.91 \\
25337 & 103474 & 1994 & 5.93 & -0.07 & 590.00 & 5635.35 & 1.00 & 0.95 & 0.96 \\
5191 & 100731 & 1994 & 6696.02 & -0.03 & 710246.00 & 5760464.58 & 0.94 & 0.86 & 0.81 \\
20704 & 102784 & 1994 & 4631.86 & -0.01 & 401208.00 & 3784520.65 & 1.15 & 0.82 & 0.94 \\
19974 & 102660 & 1994 & 38.98 & 0.01 & 3893.00 & 32667.23 & 1.00 & 0.84 & 0.84 \\
15908 & 102059 & 1994 & 190.09 & -0.04 & 19012.00 & 161593.06 & 1.00 & 0.85 & 0.85 \\
6078 & 100822 & 1994 & 13.19 & -0.01 & 1119.00 & 10518.03 & 1.18 & 0.80 & 0.94 \\
15492 & 101999 & 1994 & 905.11 & 0.03 & 90086.00 & 868888.12 & 1.00 & 0.96 & 0.96 \\
20665 & 102783 & 1994 & 1001.43 & -0.08 & 104055.00 & 812087.36 & 0.96 & 0.81 & 0.78 \\
20137 & 102669 & 1994 & 19.99 & 0.01 & 2105.00 & 19655.42 & 0.95 & 0.98 & 0.93 \\
20182 & 102676 & 1994 & 27.45 & 0.02 & 2725.00 & 24371.98 & 1.01 & 0.89 & 0.89 \\
8708 & 101095 & 1994 & 180.10 & 0.08 & 19166.00 & 155087.69 & 0.94 & 0.86 & 0.81 \\
3550 & 100455 & 1994 & 3.81 & -0.01 & 386.00 & 3179.77 & 0.99 & 0.84 & 0.82 \\
20169 & 102673 & 1994 & 94.73 & 0.02 & 9296.00 & 75018.40 & 1.02 & 0.79 & 0.81 \\
20148 & 102671 & 1994 & 39.88 & 0.01 & 3873.00 & 39153.07 & 1.03 & 0.98 & 1.01 \\
6109 & 100823 & 1994 & 24.80 & -0.05 & 2344.00 & 22878.75 & 1.06 & 0.92 & 0.98 \\
15839 & 102043 & 1994 & 103.60 & 0.03 & 9718.00 & 88601.06 & 1.07 & 0.86 & 0.91 \\
3810 & 100485 & 1994 & 58.21 & -0.03 & 6575.00 & 47583.24 & 0.89 & 0.82 & 0.72 \\
737 & 100093 & 1994 & 294.25 & -0.05 & 26210.00 & 254284.81 & 1.12 & 0.86 & 0.97 \\
26974 & 103642 & 1994 & 37.38 & -0.03 & 3738.00 & 35890.12 & 1.00 & 0.96 & 0.96 \\
3856 & 100506 & 1994 & 19.92 & -0.07 & 1813.00 & 17082.57 & 1.10 & 0.86 & 0.94 \\
19944 & 102659 & 1994 & 1648.38 & 0.00 & 164838.00 & 1551416.18 & 1.00 & 0.94 & 0.94 \\
25424 & 103485 & 1994 & 16.32 & 0.01 & 1760.00 & 18098.33 & 0.93 & 1.11 & 1.03 \\
16020 & 102073 & 1994 & 4225.83 & 0.01 & 422583.00 & 3449058.66 & 1.00 & 0.82 & 0.82 \\
6296 & 100847 & 1994 & 2.34 & -0.16 & 122.00 & 1210.34 & 1.92 & 0.52 & 0.99 \\
7795 & 101061 & 1994 & 319.70 & 0.01 & 31286.00 & 285990.23 & 1.02 & 0.89 & 0.91 \\
20762 & 102789 & 1994 & 242.86 & 0.00 & 25604.00 & 212977.40 & 0.95 & 0.88 & 0.83 \\
12634 & 101561 & 1994 & 13.88 & 0.02 & 1388.00 & 12742.89 & 1.00 & 0.92 & 0.92 \\
15467 & 101996 & 1994 & 20.66 & 0.00 & 2047.00 & 17822.18 & 1.01 & 0.86 & 0.87 \\
3912 & 100514 & 1994 & 75.88 & -0.08 & 7588.00 & 64962.68 & 1.00 & 0.86 & 0.86 \\
11018 & 101360 & 1994 & 940.21 & -0.00 & 98225.00 & 834904.00 & 0.96 & 0.89 & 0.85 \\
25428 & 103487 & 1994 & 13.51 & -0.03 & 1233.00 & 11359.41 & 1.10 & 0.84 & 0.92 \\
26922 & 103628 & 1994 & 501.97 & -0.03 & 48964.00 & 448682.69 & 1.03 & 0.89 & 0.92 \\
8596 & 101091 & 1994 & 85.50 & 0.06 & 10329.00 & 77723.40 & 0.83 & 0.91 & 0.75 \\
10665 & 101307 & 1994 & 15.79 & 0.09 & 1576.00 & 14218.47 & 1.00 & 0.90 & 0.90 \\
25420 & 103484 & 1994 & 17.81 & -0.02 & 1614.00 & 14713.19 & 1.10 & 0.83 & 0.91 \\
6047 & 100821 & 1994 & 14.85 & -0.05 & 1216.00 & 11728.21 & 1.22 & 0.79 & 0.96 \\
25306 & 103466 & 1994 & 150.54 & 0.01 & 13474.00 & 122742.36 & 1.12 & 0.82 & 0.91 \\
10697 & 101312 & 1994 & 39.89 & 0.09 & 3969.00 & 35199.41 & 1.01 & 0.88 & 0.89 \\
48978 & 240197 & 1994 & 168.49 & -0.10 & 15060.00 & 144876.57 & 1.12 & 0.86 & 0.96 \\
12759 & 101593 & 1994 & 104.63 & -0.05 & 11370.00 & 88512.30 & 0.92 & 0.85 & 0.78 \\
26966 & 103640 & 1994 & 106.89 & 0.00 & 6684.00 & 60657.45 & 1.60 & 0.57 & 0.91 \\
3864 & 100507 & 1994 & 15.79 & 0.05 & 999.00 & 10196.05 & 1.58 & 0.65 & 1.02 \\
12665 & 101562 & 1994 & 90.08 & 0.02 & 7564.00 & 63581.70 & 1.19 & 0.71 & 0.84 \\
25404 & 103483 & 1994 & 115.26 & 0.00 & 12313.00 & 117201.07 & 0.94 & 1.02 & 0.95 \\
20460 & 102749 & 1994 & 55.62 & -0.02 & 5776.00 & 54049.98 & 0.96 & 0.97 & 0.94 \\
74555 & 601136 & 1994 & 18.96 & -0.01 & 1906.00 & 18870.01 & 0.99 & 1.00 & 0.99 \\
333 & 100040 & 1994 & 5.81 & 0.01 & 594.00 & 5685.84 & 0.98 & 0.98 & 0.96 \\
20516 & 102761 & 1994 & 9165.62 & -0.02 & 934670.00 & 7790598.60 & 0.98 & 0.85 & 0.83 \\
3616 & 100463 & 1994 & 1.78 & -0.06 & 189.00 & 1609.54 & 0.94 & 0.90 & 0.85 \\
12714 & 101590 & 1994 & 23.24 & 0.01 & 2360.00 & 20766.83 & 0.98 & 0.89 & 0.88 \\
20296 & 102715 & 1994 & 2252.21 & 0.02 & 205653.00 & 1853658.69 & 1.10 & 0.82 & 0.90 \\
171 & 100017 & 1994 & 19.93 & -0.02 & 1992.00 & 19570.91 & 1.00 & 0.98 & 0.98 \\
25347 & 103478 & 1994 & 25.94 & 0.02 & 2504.00 & 20985.55 & 1.04 & 0.81 & 0.84 \\
25343 & 103477 & 1994 & 16.97 & -0.03 & 1602.00 & 15041.78 & 1.06 & 0.89 & 0.94 \\
20284 & 102709 & 1994 & 2.50 & -0.09 & 263.00 & 2370.27 & 0.95 & 0.95 & 0.90 \\
15693 & 102015 & 1994 & 308.30 & -0.03 & 33555.00 & 307917.21 & 0.92 & 1.00 & 0.92 \\
15567 & 102005 & 1994 & 591.94 & -0.05 & 59679.00 & 557759.23 & 0.99 & 0.94 & 0.93 \\
3675 & 100468 & 1994 & 128.54 & -0.08 & 16253.00 & 142119.06 & 0.79 & 1.11 & 0.87 \\
20376 & 102732 & 1994 & 167.43 & -0.07 & 18221.00 & 149917.67 & 0.92 & 0.90 & 0.82 \\
20381 & 102733 & 1994 & 3785.24 & -0.08 & 426923.00 & 3897198.28 & 0.89 & 1.03 & 0.91 \\
10794 & 101331 & 1994 & 23.41 & 0.00 & 2724.00 & 25848.52 & 0.86 & 1.10 & 0.95 \\
15633 & 102010 & 1994 & 1291.13 & 0.04 & 123846.00 & 1214900.41 & 1.04 & 0.94 & 0.98 \\
27007 & 103644 & 1994 & 187.67 & -0.04 & 18767.00 & 180245.00 & 1.00 & 0.96 & 0.96 \\
7827 & 101062 & 1994 & 555.70 & 0.02 & 53480.00 & 470508.76 & 1.04 & 0.85 & 0.88 \\
27033 & 103645 & 1994 & 295.13 & 0.09 & 29513.00 & 287592.97 & 1.00 & 0.97 & 0.97 \\
15612 & 102009 & 1994 & 73.14 & 0.04 & 7315.00 & 66228.23 & 1.00 & 0.91 & 0.91 \\
6191 & 100829 & 1994 & 392.81 & -0.00 & 39162.00 & 333499.25 & 1.00 & 0.85 & 0.85 \\
15582 & 102007 & 1994 & 374.21 & 0.00 & 34866.00 & 296446.95 & 1.07 & 0.79 & 0.85 \\
10762 & 101330 & 1994 & 670.78 & 0.00 & 62909.00 & 591259.38 & 1.07 & 0.88 & 0.94 \\
1362 & 100192 & 1994 & 37.73 & 0.05 & 3773.00 & 36821.91 & 1.00 & 0.98 & 0.98 \\
25370 & 103479 & 1994 & 27.89 & 0.01 & 2751.00 & 23824.75 & 1.01 & 0.85 & 0.87 \\
8673 & 101094 & 1994 & 211.50 & 0.00 & 17374.00 & 166233.35 & 1.22 & 0.79 & 0.96 \\
12707 & 101589 & 1994 & 84.75 & 0.06 & 7350.00 & 70871.96 & 1.15 & 0.84 & 0.96 \\
3718 & 100475 & 1994 & 91.19 & -0.12 & 9008.00 & 85566.45 & 1.01 & 0.94 & 0.95 \\
20241 & 102695 & 1994 & 1.73 & -0.24 & 221.00 & 1535.23 & 0.78 & 0.89 & 0.69 \\
6139 & 100824 & 1994 & 5.00 & -0.01 & 528.00 & 4631.83 & 0.95 & 0.93 & 0.88 \\
20593 & 102774 & 1994 & 845.41 & -0.05 & 77498.00 & 729204.23 & 1.09 & 0.86 & 0.94 \\
701 & 100091 & 1994 & 7.07 & -0.08 & 657.00 & 6298.26 & 1.08 & 0.89 & 0.96 \\
15752 & 102017 & 1994 & 1603.63 & -0.01 & 164082.00 & 1312187.53 & 0.98 & 0.82 & 0.80 \\
96657 & 611002 & 1994 & 1966.31 & -0.02 & 195170.00 & 1766812.38 & 1.01 & 0.90 & 0.91 \\
15783 & 102018 & 1994 & 397.61 & -0.01 & 38711.00 & 381562.12 & 1.03 & 0.96 & 0.99 \\
1343 & 100190 & 1994 & 850.27 & -0.01 & 85027.00 & 768551.98 & 1.00 & 0.90 & 0.90 \\
3778 & 100481 & 1994 & 20.33 & -0.04 & 2179.00 & 17881.93 & 0.93 & 0.88 & 0.82 \\
15523 & 102000 & 1994 & 479.18 & -0.04 & 48265.00 & 467121.73 & 0.99 & 0.97 & 0.97 \\
20197 & 102688 & 1994 & 56.19 & -0.11 & 5675.00 & 50735.77 & 0.99 & 0.90 & 0.89 \\
15798 & 102026 & 1994 & 34.44 & -0.18 & 3242.00 & 28019.61 & 1.06 & 0.81 & 0.86 \\
6143 & 100825 & 1994 & 5.03 & -0.01 & 517.00 & 4779.49 & 0.97 & 0.95 & 0.92 \\
61324 & 500037 & 1994 & 1110.18 & -0.02 & 107614.00 & 1047060.87 & 1.03 & 0.94 & 0.97 \\
6252 & 100833 & 1994 & 348.02 & -0.07 & 34770.00 & 320367.79 & 1.00 & 0.92 & 0.92 \\
8659 & 101093 & 1994 & 7.50 & -0.22 & 1212.00 & 7309.16 & 0.62 & 0.97 & 0.60 \\
12699 & 101588 & 1994 & 317.23 & 0.00 & 28330.00 & 242936.24 & 1.12 & 0.77 & 0.86 \\
27063 & 103647 & 1994 & 5.09 & -0.04 & 760.00 & 6838.28 & 0.67 & 1.34 & 0.90 \\
6156 & 100827 & 1994 & 5.09 & -0.02 & 516.00 & 4244.69 & 0.99 & 0.83 & 0.82 \\
25374 & 103481 & 1994 & 41.18 & 0.05 & 3977.00 & 36660.54 & 1.04 & 0.89 & 0.92 \\
25394 & 103482 & 1994 & 4.30 & -0.06 & 446.00 & 4023.88 & 0.97 & 0.93 & 0.90 \\
26981 & 103643 & 1994 & 51.10 & -0.01 & 5109.00 & 47070.08 & 1.00 & 0.92 & 0.92 \\
11202 & 101375 & 1994 & 9.70 & -0.06 & 966.00 & 9468.71 & 1.00 & 0.98 & 0.98 \\
703 & 100092 & 1994 & 22.14 & 0.01 & 2044.00 & 18171.03 & 1.08 & 0.82 & 0.89 \\
25341 & 103476 & 1994 & 3.21 & -0.02 & 306.00 & 2881.11 & 1.05 & 0.90 & 0.94 \\
20249 & 102696 & 1994 & 72.80 & -0.08 & 6840.00 & 63626.87 & 1.06 & 0.87 & 0.93 \\
15718 & 102016 & 1994 & 979.17 & -0.07 & 98672.00 & 903502.66 & 0.99 & 0.92 & 0.92 \\
65079 & 500660 & 1994 & 291.56 & -0.03 & 38175.00 & 287709.68 & 0.76 & 0.99 & 0.75 \\
10747 & 101322 & 1994 & 111.64 & -0.06 & 10354.00 & 103508.50 & 1.08 & 0.93 & 1.00 \\
20553 & 102767 & 1994 & 792.38 & -0.04 & 80366.00 & 770900.88 & 0.99 & 0.97 & 0.96 \\
15453 & 101992 & 1994 & 1187.43 & 0.00 & 118295.00 & 1021032.84 & 1.00 & 0.86 & 0.86 \\
26956 & 103638 & 1994 & 73.48 & 0.00 & 7348.00 & 71819.12 & 1.00 & 0.98 & 0.98 \\
25491 & 103494 & 1994 & 272.76 & -0.04 & 22499.00 & 225053.63 & 1.21 & 0.83 & 1.00 \\
19549 & 102624 & 1994 & 123.85 & -0.01 & 9802.00 & 84623.66 & 1.26 & 0.68 & 0.86 \\
3425 & 100434 & 1994 & 4.10 & -0.08 & 411.00 & 4111.50 & 1.00 & 1.00 & 1.00 \\
16136 & 102085 & 1994 & 1751.05 & 0.01 & 181769.00 & 1426519.44 & 0.96 & 0.81 & 0.78 \\
19563 & 102628 & 1994 & 164.97 & 0.00 & 17775.00 & 165831.80 & 0.93 & 1.01 & 0.93 \\
19570 & 102633 & 1994 & 213.11 & -0.01 & 23061.00 & 206641.99 & 0.92 & 0.97 & 0.90 \\
4057 & 100544 & 1994 & 34.28 & 0.10 & 3428.00 & 33888.75 & 1.00 & 0.99 & 0.99 \\
26788 & 103607 & 1994 & 384.83 & -0.01 & 35176.00 & 339651.43 & 1.09 & 0.88 & 0.97 \\
19598 & 102636 & 1994 & 777.06 & -0.02 & 39558.00 & 352360.94 & 1.96 & 0.45 & 0.89 \\
8561 & 101090 & 1994 & 175.40 & 0.20 & 11469.00 & 95759.37 & 1.53 & 0.55 & 0.83 \\
4043 & 100543 & 1994 & 401.68 & -0.23 & 40168.00 & 397666.90 & 1.00 & 0.99 & 0.99 \\
19630 & 102639 & 1994 & 105.57 & 0.03 & 11562.00 & 111333.96 & 0.91 & 1.05 & 0.96 \\
19651 & 102641 & 1994 & 241.10 & -0.01 & 24417.00 & 238469.78 & 0.99 & 0.99 & 0.98 \\
1443 & 100200 & 1994 & 14.87 & 0.10 & 1487.00 & 13998.75 & 1.00 & 0.94 & 0.94 \\
11116 & 101368 & 1994 & 438.36 & 0.00 & 43663.00 & 412215.44 & 1.00 & 0.94 & 0.94 \\
65056 & 500659 & 1994 & 20.36 & -0.10 & 2601.00 & 16395.59 & 0.78 & 0.81 & 0.63 \\
12777 & 101595 & 1994 & 402.43 & -0.08 & 37799.00 & 371767.62 & 1.06 & 0.92 & 0.98 \\
26815 & 103608 & 1994 & 40.82 & 0.00 & 3667.00 & 35293.69 & 1.11 & 0.86 & 0.96 \\
15391 & 101989 & 1994 & 69.29 & -0.06 & 5290.00 & 50457.80 & 1.31 & 0.73 & 0.95 \\
12590 & 101557 & 1994 & 20.48 & 0.03 & 2058.00 & 17644.25 & 0.99 & 0.86 & 0.86 \\
5969 & 100815 & 1994 & 9.92 & 0.02 & 1081.00 & 9728.23 & 0.92 & 0.98 & 0.90 \\
19664 & 102645 & 1994 & 148.60 & 0.00 & 14982.00 & 120500.08 & 0.99 & 0.81 & 0.80 \\
3456 & 100439 & 1994 & 46.00 & -0.17 & 5398.00 & 53986.79 & 0.85 & 1.17 & 1.00 \\
10600 & 101300 & 1994 & 2543.34 & -0.12 & 254595.00 & 2148681.12 & 1.00 & 0.84 & 0.84 \\
3420 & 100432 & 1994 & 16.88 & -0.12 & 3054.00 & 28233.67 & 0.55 & 1.67 & 0.92 \\
16222 & 102099 & 1994 & 5.80 & -0.00 & 588.00 & 4864.87 & 0.99 & 0.84 & 0.83 \\
4105 & 100552 & 1994 & 235.31 & 0.07 & 17922.00 & 167117.39 & 1.31 & 0.71 & 0.93 \\
11198 & 101374 & 1994 & 6.60 & -0.05 & 659.00 & 6479.64 & 1.00 & 0.98 & 0.98 \\
16195 & 102090 & 1994 & 62.84 & -0.06 & 6141.00 & 57738.10 & 1.02 & 0.92 & 0.94 \\
12794 & 101596 & 1994 & 137.06 & -0.08 & 15368.00 & 138681.07 & 0.89 & 1.01 & 0.90 \\
16164 & 102089 & 1994 & 160.51 & -0.00 & 17463.00 & 146823.04 & 0.92 & 0.91 & 0.84 \\
20972 & 102814 & 1994 & 18.98 & -0.03 & 1956.00 & 19330.00 & 0.97 & 1.02 & 0.99 \\
74622 & 601142 & 1994 & 196.70 & -0.11 & 29593.00 & 270005.87 & 0.66 & 1.37 & 0.91 \\
19486 & 102607 & 1994 & 970.87 & -0.06 & 97277.00 & 933715.10 & 1.00 & 0.96 & 0.96 \\
11190 & 101370 & 1994 & 15.40 & -0.05 & 1700.00 & 15938.09 & 0.91 & 1.03 & 0.94 \\
19503 & 102608 & 1994 & 28.07 & 0.00 & 2814.00 & 26900.14 & 1.00 & 0.96 & 0.96 \\
11152 & 101369 & 1994 & 426.73 & -0.03 & 48247.00 & 466367.05 & 0.88 & 1.09 & 0.97 \\
19543 & 102614 & 1994 & 176.14 & -0.03 & 19201.00 & 187881.38 & 0.92 & 1.07 & 0.98 \\
19536 & 102612 & 1994 & 65.21 & 0.01 & 7198.00 & 67798.08 & 0.91 & 1.04 & 0.94 \\
20933 & 102812 & 1994 & 44.80 & -0.02 & 4900.00 & 41602.17 & 0.91 & 0.93 & 0.85 \\
12565 & 101554 & 1994 & 202.66 & -0.01 & 20992.00 & 195852.96 & 0.97 & 0.97 & 0.93 \\
53093 & 338393 & 1994 & 5.80 & 0.01 & 546.00 & 5712.58 & 1.06 & 0.98 & 1.05 \\
19533 & 102611 & 1994 & 18.68 & -0.00 & 1824.00 & 17916.47 & 1.02 & 0.96 & 0.98 \\
20942 & 102813 & 1994 & 126.39 & 0.03 & 11576.00 & 108691.22 & 1.09 & 0.86 & 0.94 \\
26756 & 103606 & 1994 & 41.32 & -0.09 & 4667.00 & 41438.99 & 0.89 & 1.00 & 0.89 \\
797 & 100097 & 1994 & 5.10 & 0.03 & 618.00 & 5852.82 & 0.83 & 1.15 & 0.95 \\
11082 & 101367 & 1994 & 196.17 & -0.03 & 20570.00 & 192435.66 & 0.95 & 0.98 & 0.94 \\
20465 & 102757 & 1994 & 1504.73 & 0.02 & 100001.00 & 1036126.11 & 1.50 & 0.69 & 1.04 \\
6325 & 100849 & 1994 & 64.48 & -0.10 & 6444.00 & 58211.69 & 1.00 & 0.90 & 0.90 \\
27123 & 105243 & 1994 & 17.23 & 0.07 & 1828.00 & 17676.14 & 0.94 & 1.03 & 0.97 \\
12604 & 101560 & 1994 & 8.73 & 0.01 & 862.00 & 8572.12 & 1.01 & 0.98 & 0.99 \\
19759 & 102651 & 1994 & 270.17 & 0.01 & 27016.00 & 231630.31 & 1.00 & 0.86 & 0.86 \\
6016 & 100820 & 1994 & 12.43 & 0.02 & 842.00 & 8329.30 & 1.48 & 0.67 & 0.99 \\
53399 & 346113 & 1994 & 232.51 & -0.11 & 29770.00 & 226822.52 & 0.78 & 0.98 & 0.76 \\
10635 & 101302 & 1994 & 239.68 & -0.05 & 24025.00 & 210948.88 & 1.00 & 0.88 & 0.88 \\
26908 & 103621 & 1994 & 49.69 & -0.04 & 5245.00 & 42115.92 & 0.95 & 0.85 & 0.80 \\
767 & 100096 & 1994 & 9.20 & 0.02 & 838.00 & 7947.51 & 1.10 & 0.86 & 0.95 \\
679 & 100090 & 1994 & 71.43 & -0.02 & 7285.00 & 68841.11 & 0.98 & 0.96 & 0.94 \\
3966 & 100535 & 1994 & 206.10 & -0.04 & 20684.00 & 196387.86 & 1.00 & 0.95 & 0.95 \\
25275 & 103464 & 1994 & 677.15 & 0.03 & 61691.00 & 606723.02 & 1.10 & 0.90 & 0.98 \\
12770 & 101594 & 1994 & 260.52 & -0.08 & 28708.00 & 249374.06 & 0.91 & 0.96 & 0.87 \\
20801 & 102795 & 1994 & 141.41 & -0.01 & 13959.00 & 119367.20 & 1.01 & 0.84 & 0.86 \\
8733 & 101096 & 1994 & 104.20 & 0.03 & 8892.00 & 81883.58 & 1.17 & 0.79 & 0.92 \\
19872 & 102654 & 1994 & 646.15 & -0.07 & 64615.00 & 608447.85 & 1.00 & 0.94 & 0.94 \\
4009 & 100538 & 1994 & 489.68 & -0.04 & 52885.00 & 479953.86 & 0.93 & 0.98 & 0.91 \\
15447 & 101991 & 1994 & 49.26 & -0.02 & 4231.00 & 39244.58 & 1.16 & 0.80 & 0.93 \\
15419 & 101990 & 1994 & 195.20 & 0.02 & 22378.00 & 195149.61 & 0.87 & 1.00 & 0.87 \\
65199 & 500670 & 1994 & 634.75 & -0.01 & 60338.00 & 540177.63 & 1.05 & 0.85 & 0.90 \\
12060 & 101494 & 1995 & 257.61 & 0.27 & 25857.00 & 249291.05 & 1.00 & 0.97 & 0.96 \\
12154 & 101513 & 1995 & 91.70 & 0.35 & 8451.00 & 86215.26 & 1.09 & 0.94 & 1.02 \\
10562 & 101299 & 1995 & 813.02 & 0.35 & 80332.00 & 728717.93 & 1.01 & 0.90 & 0.91 \\
29727 & 105643 & 1995 & 46.79 & 0.37 & 4697.00 & 44285.17 & 1.00 & 0.95 & 0.94 \\
74787 & 601171 & 1995 & 173.28 & 0.37 & 15677.00 & 155402.08 & 1.11 & 0.90 & 0.99 \\
7458 & 101040 & 1995 & 2145.60 & 0.29 & 208824.00 & 1777682.92 & 1.03 & 0.83 & 0.85 \\
10503 & 101294 & 1995 & 6.78 & 0.33 & 662.00 & 6781.73 & 1.02 & 1.00 & 1.02 \\
29571 & 105616 & 1995 & 2.52 & 0.35 & 241.00 & 2406.25 & 1.05 & 0.95 & 1.00 \\
96658 & 611002 & 1995 & 2398.20 & 0.31 & 266426.00 & 2390495.48 & 0.90 & 1.00 & 0.90 \\
11743 & 101460 & 1995 & 5313.64 & 0.32 & 499176.00 & 4317868.21 & 1.06 & 0.81 & 0.86 \\
28203 & 105393 & 1995 & 4.20 & 0.23 & 411.00 & 4080.16 & 1.02 & 0.97 & 0.99 \\
28720 & 105472 & 1995 & 75.59 & 0.03 & 6606.00 & 63728.45 & 1.14 & 0.84 & 0.96 \\
13367 & 101730 & 1995 & 67.90 & 0.24 & 7063.00 & 57538.96 & 0.96 & 0.85 & 0.81 \\
9540 & 101149 & 1995 & 576.00 & 0.46 & 52406.00 & 519333.18 & 1.10 & 0.90 & 0.99 \\
138 & 100010 & 1995 & 1195.05 & 0.30 & 100331.00 & 1022240.22 & 1.19 & 0.86 & 1.02 \\
9213 & 101119 & 1995 & 43.81 & 0.27 & 4381.00 & 37114.31 & 1.00 & 0.85 & 0.85 \\
9382 & 101133 & 1995 & 432.50 & 0.40 & 28139.00 & 273626.60 & 1.54 & 0.63 & 0.97 \\
12877 & 101603 & 1995 & 687.37 & 0.22 & 59010.00 & 545562.61 & 1.16 & 0.79 & 0.92 \\
12760 & 101593 & 1995 & 85.12 & 0.27 & 8348.00 & 72328.58 & 1.02 & 0.85 & 0.87 \\
10636 & 101302 & 1995 & 275.50 & 0.27 & 27376.00 & 241255.65 & 1.01 & 0.88 & 0.88 \\
29207 & 105544 & 1995 & 31.14 & 0.01 & 2720.00 & 25095.06 & 1.14 & 0.81 & 0.92 \\
12833 & 101602 & 1995 & 1242.19 & 0.20 & 107490.00 & 956678.57 & 1.16 & 0.77 & 0.89 \\
10666 & 101307 & 1995 & 29.02 & 0.16 & 2530.00 & 23851.34 & 1.15 & 0.82 & 0.94 \\
28162 & 105386 & 1995 & 145.41 & 0.02 & 11078.00 & 118489.15 & 1.31 & 0.81 & 1.07 \\
28164 & 105387 & 1995 & 113.63 & 0.04 & 9963.00 & 85808.11 & 1.14 & 0.76 & 0.86 \\
29653 & 105631 & 1995 & 3.70 & 0.36 & 357.00 & 3475.00 & 1.04 & 0.94 & 0.97 \\
29759 & 105645 & 1995 & 81.00 & -0.00 & 8100.00 & 74560.72 & 1.00 & 0.92 & 0.92 \\
12185 & 101518 & 1995 & 61.34 & 0.36 & 5482.00 & 61477.06 & 1.12 & 1.00 & 1.12 \\
10763 & 101330 & 1995 & 2010.70 & 0.48 & 173426.00 & 1459633.34 & 1.16 & 0.73 & 0.84 \\
29788 & 105647 & 1995 & 46.20 & 0.31 & 4668.00 & 38028.51 & 0.99 & 0.82 & 0.81 \\
8173 & 101078 & 1995 & 27.80 & 0.47 & 2201.00 & 24637.01 & 1.26 & 0.89 & 1.12 \\
29677 & 105635 & 1995 & 75.39 & 0.27 & 6780.00 & 55129.40 & 1.11 & 0.73 & 0.81 \\
258 & 100022 & 1995 & 43.33 & 0.40 & 4333.00 & 41596.83 & 1.00 & 0.96 & 0.96 \\
13409 & 101738 & 1995 & 498.94 & 0.27 & 45937.00 & 453206.80 & 1.09 & 0.91 & 0.99 \\
11709 & 101457 & 1995 & 362.38 & 0.29 & 36304.00 & 342653.33 & 1.00 & 0.95 & 0.94 \\
9479 & 101139 & 1995 & 45.57 & 0.26 & 4544.00 & 37856.25 & 1.00 & 0.83 & 0.83 \\
29749 & 105644 & 1995 & 37.36 & 0.34 & 3811.00 & 33360.79 & 0.98 & 0.89 & 0.88 \\
28712 & 105471 & 1995 & 64.92 & 0.01 & 6341.00 & 56252.97 & 1.02 & 0.87 & 0.89 \\
8124 & 101076 & 1995 & 27.90 & 0.26 & 3226.00 & 31657.10 & 0.86 & 1.13 & 0.98 \\
12077 & 101497 & 1995 & 1558.00 & 0.29 & 155505.00 & 1337712.73 & 1.00 & 0.86 & 0.86 \\
29231 & 105558 & 1995 & 1.80 & 0.07 & 158.00 & 1299.22 & 1.14 & 0.72 & 0.82 \\
9275 & 101127 & 1995 & 418.90 & 0.24 & 40324.00 & 331652.48 & 1.04 & 0.79 & 0.82 \\
8786 & 101098 & 1995 & 47.20 & 0.29 & 3885.00 & 41047.26 & 1.21 & 0.87 & 1.06 \\
28702 & 105469 & 1995 & 5.19 & 0.24 & 504.00 & 4781.22 & 1.03 & 0.92 & 0.95 \\
13478 & 101741 & 1995 & 632.40 & 0.36 & 58296.00 & 523237.01 & 1.08 & 0.83 & 0.90 \\
7796 & 101061 & 1995 & 594.20 & 0.52 & 54270.00 & 488574.93 & 1.09 & 0.82 & 0.90 \\
26 & 100003 & 1995 & 329.29 & 0.44 & 29660.00 & 283654.83 & 1.11 & 0.86 & 0.96 \\
8241 & 101080 & 1995 & 78.40 & 0.27 & 11244.00 & 78622.31 & 0.70 & 1.00 & 0.70 \\
11870 & 101464 & 1995 & 235.80 & 0.09 & 29619.00 & 264107.90 & 0.80 & 1.12 & 0.89 \\
13548 & 101743 & 1995 & 3057.41 & 0.36 & 299575.00 & 2689873.01 & 1.02 & 0.88 & 0.90 \\
28693 & 105465 & 1995 & 14.76 & -0.00 & 1523.00 & 14995.91 & 0.97 & 1.02 & 0.98 \\
9318 & 101131 & 1995 & 624.80 & 0.38 & 59899.00 & 502946.43 & 1.04 & 0.80 & 0.84 \\
7732 & 101056 & 1995 & 14818.60 & 0.35 & 1212123.00 & 12298008.60 & 1.22 & 0.83 & 1.01 \\
9408 & 101134 & 1995 & 163.40 & 0.13 & 21236.00 & 199338.58 & 0.77 & 1.22 & 0.94 \\
11900 & 101465 & 1995 & 38.97 & 0.27 & 3891.00 & 36407.46 & 1.00 & 0.93 & 0.94 \\
9491 & 101140 & 1995 & 535.50 & 0.30 & 49736.00 & 509110.05 & 1.08 & 0.95 & 1.02 \\
8734 & 101096 & 1995 & 84.00 & 0.28 & 11369.00 & 98124.96 & 0.74 & 1.17 & 0.86 \\
8204 & 101079 & 1995 & 131.50 & 0.21 & 9079.00 & 93807.72 & 1.45 & 0.71 & 1.03 \\
28701 & 105468 & 1995 & 7.18 & 0.30 & 717.00 & 7166.15 & 1.00 & 1.00 & 1.00 \\
9254 & 101124 & 1995 & 16.33 & 0.23 & 1580.00 & 14677.48 & 1.03 & 0.90 & 0.93 \\
12750 & 101592 & 1995 & 577.49 & 0.33 & 57661.00 & 503575.05 & 1.00 & 0.87 & 0.87 \\
11985 & 101476 & 1995 & 2147.10 & 0.30 & 190675.00 & 1667191.18 & 1.13 & 0.78 & 0.87 \\
29233 & 105561 & 1995 & 17.99 & 0.17 & 1607.00 & 17061.89 & 1.12 & 0.95 & 1.06 \\
269 & 100030 & 1995 & 79.79 & 0.24 & 7928.00 & 74175.31 & 1.01 & 0.93 & 0.94 \\
10748 & 101322 & 1995 & 120.53 & 0.24 & 11505.00 & 120483.56 & 1.05 & 1.00 & 1.05 \\
12011 & 101477 & 1995 & 248.50 & 0.32 & 20777.00 & 197587.39 & 1.20 & 0.80 & 0.95 \\
12097 & 101503 & 1995 & 261.70 & 0.34 & 22802.00 & 216617.83 & 1.15 & 0.83 & 0.95 \\
7387 & 101038 & 1995 & 4278.70 & 0.37 & 400925.00 & 3522284.13 & 1.07 & 0.82 & 0.88 \\
9522 & 101142 & 1995 & 262.30 & 0.16 & 27625.00 & 228002.41 & 0.95 & 0.87 & 0.83 \\
11951 & 101473 & 1995 & 969.11 & 0.34 & 89284.00 & 892388.39 & 1.09 & 0.92 & 1.00 \\
10698 & 101312 & 1995 & 250.25 & 0.42 & 19638.00 & 195773.36 & 1.27 & 0.78 & 1.00 \\
8272 & 101081 & 1995 & 251.10 & 0.47 & 16610.00 & 154002.34 & 1.51 & 0.61 & 0.93 \\
28771 & 105476 & 1995 & 8.68 & 0.07 & 576.00 & 5725.43 & 1.51 & 0.66 & 0.99 \\
12028 & 101488 & 1995 & 41.04 & 0.47 & 3694.00 & 35024.77 & 1.11 & 0.85 & 0.95 \\
11776 & 101461 & 1995 & 884.12 & 0.26 & 106395.00 & 901190.53 & 0.83 & 1.02 & 0.85 \\
12111 & 101507 & 1995 & 53.79 & 0.31 & 6040.00 & 57091.42 & 0.89 & 1.06 & 0.95 \\
53094 & 338393 & 1995 & 8.43 & 0.26 & 703.00 & 7489.60 & 1.20 & 0.89 & 1.07 \\
12120 & 101511 & 1995 & 136.10 & 0.38 & 13752.00 & 136997.01 & 0.99 & 1.01 & 1.00 \\
5 & 100001 & 1995 & 1254.11 & 0.31 & 118454.00 & 1213192.72 & 1.06 & 0.97 & 1.02 \\
7425 & 101039 & 1995 & 2207.50 & 0.36 & 201966.00 & 1713847.37 & 1.09 & 0.78 & 0.85 \\
11807 & 101462 & 1995 & 413.54 & 0.33 & 32495.00 & 317299.12 & 1.27 & 0.77 & 0.98 \\
29641 & 105630 & 1995 & 9.50 & 0.30 & 1008.00 & 8118.63 & 0.94 & 0.85 & 0.81 \\
28645 & 105458 & 1995 & 294.60 & 0.00 & 27493.00 & 264752.98 & 1.07 & 0.90 & 0.96 \\
285 & 100033 & 1995 & 38.99 & 0.11 & 3820.00 & 36963.54 & 1.02 & 0.95 & 0.97 \\
154 & 100016 & 1995 & 46.06 & 0.34 & 4899.00 & 48949.11 & 0.94 & 1.06 & 1.00 \\
10457 & 101286 & 1995 & 345.42 & 0.25 & 34547.00 & 323709.83 & 1.00 & 0.94 & 0.94 \\
10727 & 101320 & 1995 & 4.04 & 0.06 & 358.00 & 3358.63 & 1.13 & 0.83 & 0.94 \\
10601 & 101300 & 1995 & 2917.01 & 0.25 & 277434.00 & 2417922.27 & 1.05 & 0.83 & 0.87 \\
29612 & 105627 & 1995 & 93.60 & 0.24 & 7368.00 & 80364.76 & 1.27 & 0.86 & 1.09 \\
9456 & 101137 & 1995 & 8.00 & 0.33 & 761.00 & 7391.51 & 1.05 & 0.92 & 0.97 \\
11203 & 101375 & 1995 & 9.42 & 0.22 & 973.00 & 9640.82 & 0.97 & 1.02 & 0.99 \\
11019 & 101360 & 1995 & 1090.36 & 0.25 & 112345.00 & 1011969.06 & 0.97 & 0.93 & 0.90 \\
28868 & 105496 & 1995 & 5.33 & 0.08 & 548.00 & 5531.19 & 0.97 & 1.04 & 1.01 \\
10206 & 101274 & 1995 & 63.09 & 0.26 & 4790.00 & 47907.45 & 1.32 & 0.76 & 1.00 \\
8457 & 101087 & 1995 & 26.10 & 0.24 & 3182.00 & 30393.95 & 0.82 & 1.16 & 0.96 \\
190 & 100018 & 1995 & 47.70 & 0.32 & 5022.00 & 49456.05 & 0.95 & 1.04 & 0.98 \\
10192 & 101268 & 1995 & 651.80 & 0.46 & 46536.00 & 387543.97 & 1.40 & 0.59 & 0.83 \\
53400 & 346113 & 1995 & 382.42 & 0.32 & 38787.00 & 381697.89 & 0.99 & 1.00 & 0.98 \\
11348 & 101397 & 1995 & 27.76 & 0.27 & 2770.00 & 24901.46 & 1.00 & 0.90 & 0.90 \\
7585 & 101047 & 1995 & 232.80 & 0.36 & 21016.00 & 231389.16 & 1.11 & 0.99 & 1.10 \\
11051 & 101364 & 1995 & 56.48 & 0.28 & 5236.00 & 48772.95 & 1.08 & 0.86 & 0.93 \\
12501 & 101544 & 1995 & 7.41 & -0.00 & 739.00 & 6804.69 & 1.00 & 0.92 & 0.92 \\
7973 & 101068 & 1995 & 53702.30 & 0.37 & 5351916.00 & 46981203.03 & 1.00 & 0.87 & 0.88 \\
11083 & 101367 & 1995 & 228.62 & 0.25 & 21847.00 & 225444.69 & 1.05 & 0.99 & 1.03 \\
12945 & 101616 & 1995 & 7171.19 & 0.30 & 651640.00 & 5598479.95 & 1.10 & 0.78 & 0.86 \\
28630 & 105457 & 1995 & 66.44 & -0.03 & 6405.00 & 59815.39 & 1.04 & 0.90 & 0.93 \\
13067 & 101626 & 1995 & 699.66 & 0.22 & 60810.00 & 498365.51 & 1.15 & 0.71 & 0.82 \\
10225 & 101275 & 1995 & 108.28 & 0.31 & 8410.00 & 85001.35 & 1.29 & 0.78 & 1.01 \\
12605 & 101560 & 1995 & 11.34 & 0.25 & 1153.00 & 9283.56 & 0.98 & 0.82 & 0.81 \\
13121 & 101668 & 1995 & 55.23 & 0.25 & 4826.00 & 50541.16 & 1.14 & 0.92 & 1.05 \\
9079 & 101111 & 1995 & 322.30 & 0.47 & 19668.00 & 174044.20 & 1.64 & 0.54 & 0.88 \\
11452 & 101414 & 1995 & 15.89 & 0.19 & 1597.00 & 15370.82 & 1.00 & 0.97 & 0.96 \\
12405 & 101539 & 1995 & 490.68 & 0.25 & 37772.00 & 320888.09 & 1.30 & 0.65 & 0.85 \\
28557 & 105443 & 1995 & 33.25 & 0.28 & 3244.00 & 27112.26 & 1.02 & 0.82 & 0.84 \\
7660 & 101054 & 1995 & 3213.70 & 0.33 & 242137.00 & 2414240.44 & 1.33 & 0.75 & 1.00 \\
11429 & 101402 & 1995 & 25.36 & 0.22 & 3259.00 & 29425.42 & 0.78 & 1.16 & 0.90 \\
12439 & 101541 & 1995 & 73.63 & 0.20 & 7363.00 & 67546.28 & 1.00 & 0.92 & 0.92 \\
8005 & 101069 & 1995 & 689.00 & 0.37 & 57834.00 & 577578.99 & 1.19 & 0.84 & 1.00 \\
9983 & 101216 & 1995 & 60.19 & 0.29 & 5916.00 & 54709.38 & 1.02 & 0.91 & 0.92 \\
13164 & 101698 & 1995 & 174.80 & 0.31 & 15505.00 & 163761.54 & 1.13 & 0.94 & 1.06 \\
13151 & 101681 & 1995 & 212.70 & 0.30 & 20071.00 & 179820.36 & 1.06 & 0.85 & 0.90 \\
49101 & 240222 & 1995 & 504.65 & 0.38 & 45963.00 & 459532.64 & 1.10 & 0.91 & 1.00 \\
11361 & 101398 & 1995 & 96.74 & 0.31 & 9579.00 & 94330.40 & 1.01 & 0.98 & 0.98 \\
385 & 100048 & 1995 & 348.02 & 0.31 & 35902.00 & 332462.73 & 0.97 & 0.96 & 0.93 \\
28609 & 105450 & 1995 & 4.29 & 0.15 & 429.00 & 3924.94 & 1.00 & 0.92 & 0.91 \\
53455 & 350572 & 1995 & 44.25 & 0.30 & 4340.00 & 44370.36 & 1.02 & 1.00 & 1.02 \\
12566 & 101554 & 1995 & 254.95 & 0.36 & 22514.00 & 249019.57 & 1.13 & 0.98 & 1.11 \\
11244 & 101379 & 1995 & 214.45 & 0.37 & 19225.00 & 198509.31 & 1.12 & 0.93 & 1.03 \\
8994 & 101108 & 1995 & 438.20 & 0.27 & 53094.00 & 404335.02 & 0.83 & 0.92 & 0.76 \\
8963 & 101107 & 1995 & 1274.50 & 0.22 & 132200.00 & 1119974.49 & 0.96 & 0.88 & 0.85 \\
28564 & 105444 & 1995 & 8.58 & 0.32 & 836.00 & 7785.09 & 1.03 & 0.91 & 0.93 \\
12542 & 101553 & 1995 & 33.74 & -0.03 & 2647.00 & 26752.79 & 1.27 & 0.79 & 1.01 \\
11191 & 101370 & 1995 & 11.17 & 0.24 & 1712.00 & 16302.70 & 0.65 & 1.46 & 0.95 \\
10124 & 101262 & 1995 & 10.06 & 0.28 & 911.00 & 8639.78 & 1.10 & 0.86 & 0.95 \\
317 & 100036 & 1995 & 67.95 & 0.33 & 6736.00 & 62786.64 & 1.01 & 0.92 & 0.93 \\
28991 & 105511 & 1995 & 10.17 & 0.18 & 603.00 & 6311.94 & 1.69 & 0.62 & 1.05 \\
7939 & 101067 & 1995 & 65.30 & 0.21 & 7387.00 & 70981.04 & 0.88 & 1.09 & 0.96 \\
7635 & 101050 & 1995 & 155.30 & 0.32 & 12836.00 & 106193.53 & 1.21 & 0.68 & 0.83 \\
10141 & 101263 & 1995 & 373.24 & 0.29 & 34772.00 & 302205.66 & 1.07 & 0.81 & 0.87 \\
10162 & 101264 & 1995 & 218.21 & 0.31 & 19829.00 & 175751.60 & 1.10 & 0.81 & 0.89 \\
11199 & 101374 & 1995 & 5.69 & 0.26 & 571.00 & 5176.06 & 1.00 & 0.91 & 0.91 \\
12995 & 101618 & 1995 & 359.73 & 0.18 & 32280.00 & 270496.69 & 1.11 & 0.75 & 0.84 \\
28629 & 105454 & 1995 & 53.76 & 0.33 & 5542.00 & 53772.07 & 0.97 & 1.00 & 0.97 \\
11255 & 101380 & 1995 & 276.05 & 0.33 & 27259.00 & 262305.25 & 1.01 & 0.95 & 0.96 \\
11153 & 101369 & 1995 & 526.41 & 0.33 & 52050.00 & 503469.59 & 1.01 & 0.96 & 0.97 \\
12591 & 101557 & 1995 & 28.26 & 0.31 & 2608.00 & 23445.60 & 1.08 & 0.83 & 0.90 \\
9031 & 101109 & 1995 & 203.30 & 0.32 & 18945.00 & 173022.80 & 1.07 & 0.85 & 0.91 \\
13044 & 101623 & 1995 & 1539.79 & 0.22 & 133320.00 & 1334729.26 & 1.15 & 0.87 & 1.00 \\
29062 & 105523 & 1995 & 10.09 & 0.01 & 1009.00 & 9345.16 & 1.00 & 0.93 & 0.93 \\
11117 & 101368 & 1995 & 486.37 & 0.23 & 51388.00 & 398052.57 & 0.95 & 0.82 & 0.77 \\
7605 & 101048 & 1995 & 2872.70 & 0.40 & 183480.00 & 1922512.94 & 1.57 & 0.67 & 1.05 \\
8489 & 101088 & 1995 & 892.30 & 0.69 & 116970.00 & 717905.56 & 0.76 & 0.80 & 0.61 \\
11311 & 101393 & 1995 & 166.93 & 0.35 & 17096.00 & 161153.04 & 0.98 & 0.97 & 0.94 \\
30955 & 105842 & 1995 & 29.84 & 0.19 & 2973.00 & 27836.35 & 1.00 & 0.93 & 0.94 \\
95 & 100006 & 1995 & 3259.80 & 0.32 & 300652.00 & 2901371.78 & 1.08 & 0.89 & 0.97 \\
29033 & 105522 & 1995 & 29.90 & 0.02 & 2552.00 & 23133.52 & 1.17 & 0.77 & 0.91 \\
12513 & 101545 & 1995 & 72.24 & 0.34 & 7228.00 & 66198.87 & 1.00 & 0.92 & 0.92 \\
13030 & 101622 & 1995 & 979.44 & 0.29 & 194823.00 & 1948332.88 & 0.50 & 1.99 & 1.00 \\
49058 & 240212 & 1995 & 1310.57 & 0.44 & 103897.00 & 1207920.88 & 1.26 & 0.92 & 1.16 \\
12583 & 101555 & 1995 & 5.79 & 0.02 & 688.00 & 7024.76 & 0.84 & 1.21 & 1.02 \\
28935 & 105507 & 1995 & 207.45 & 0.02 & 18541.00 & 193747.69 & 1.12 & 0.93 & 1.04 \\
11266 & 101381 & 1995 & 58.68 & 0.37 & 5525.00 & 57479.66 & 1.06 & 0.98 & 1.04 \\
51903 & 300102 & 1995 & 14.69 & 0.35 & 1310.00 & 14376.51 & 1.12 & 0.98 & 1.10 \\
52031 & 301299 & 1995 & 336.09 & 0.46 & 33113.00 & 288390.41 & 1.01 & 0.86 & 0.87 \\
9918 & 101212 & 1995 & 499.94 & 0.31 & 70957.00 & 681039.97 & 0.70 & 1.36 & 0.96 \\
8024 & 101071 & 1995 & 1150.20 & 0.34 & 107640.00 & 940536.34 & 1.07 & 0.82 & 0.87 \\
53631 & 355027 & 1995 & 174.70 & 0.36 & 18042.00 & 149312.65 & 0.97 & 0.85 & 0.83 \\
28425 & 105424 & 1995 & 362.35 & 0.35 & 29738.00 & 306594.86 & 1.22 & 0.85 & 1.03 \\
10378 & 101284 & 1995 & 46.48 & 0.32 & 3674.00 & 32187.58 & 1.26 & 0.69 & 0.88 \\
29889 & 105656 & 1995 & 272.40 & 0.32 & 26994.00 & 250697.64 & 1.01 & 0.92 & 0.93 \\
12708 & 101589 & 1995 & 114.42 & 0.26 & 11440.00 & 102887.42 & 1.00 & 0.90 & 0.90 \\
49262 & 240261 & 1995 & 206.44 & 0.27 & 20598.00 & 174561.44 & 1.00 & 0.85 & 0.85 \\
7489 & 101042 & 1995 & 528.80 & 0.38 & 47085.00 & 433593.97 & 1.12 & 0.82 & 0.92 \\
28454 & 105426 & 1995 & 134.32 & 0.18 & 10297.00 & 96720.70 & 1.30 & 0.72 & 0.94 \\
13325 & 101728 & 1995 & 49.60 & 0.42 & 5039.00 & 44591.63 & 0.98 & 0.90 & 0.88 \\
12700 & 101588 & 1995 & 608.67 & 0.50 & 63227.00 & 489351.35 & 0.96 & 0.80 & 0.77 \\
9721 & 101179 & 1995 & 160.20 & 0.29 & 13229.00 & 130426.37 & 1.21 & 0.81 & 0.99 \\
28512 & 105431 & 1995 & 72.14 & 0.01 & 8880.00 & 88569.02 & 0.81 & 1.23 & 1.00 \\
13311 & 101723 & 1995 & 20.23 & 0.27 & 1928.00 & 17430.89 & 1.05 & 0.86 & 0.90 \\
8871 & 101103 & 1995 & 61.80 & 0.13 & 6762.00 & 54893.95 & 0.91 & 0.89 & 0.81 \\
9178 & 101116 & 1995 & 1510.70 & 0.31 & 124969.00 & 1286440.75 & 1.21 & 0.85 & 1.03 \\
9700 & 101167 & 1995 & 239.52 & 0.49 & 21113.00 & 184750.99 & 1.13 & 0.77 & 0.88 \\
12225 & 101528 & 1995 & 55.65 & 0.03 & 5464.00 & 44342.44 & 1.02 & 0.80 & 0.81 \\
29860 & 105655 & 1995 & 69.48 & 0.22 & 5835.00 & 59397.62 & 1.19 & 0.85 & 1.02 \\
8313 & 101082 & 1995 & 1448.70 & 0.46 & 106679.00 & 968293.53 & 1.36 & 0.67 & 0.91 \\
10795 & 101331 & 1995 & 66.88 & 0.34 & 6687.00 & 64685.62 & 1.00 & 0.97 & 0.97 \\
29817 & 105652 & 1995 & 123.38 & 0.23 & 10609.00 & 118590.28 & 1.16 & 0.96 & 1.12 \\
29843 & 105654 & 1995 & 28.34 & 0.36 & 2454.00 & 22875.29 & 1.15 & 0.81 & 0.93 \\
11676 & 101456 & 1995 & 45.41 & 0.44 & 4540.00 & 43144.62 & 1.00 & 0.95 & 0.95 \\
8846 & 101102 & 1995 & 39.50 & 0.12 & 3950.00 & 39337.84 & 1.00 & 1.00 & 1.00 \\
364 & 100041 & 1995 & 55.89 & 0.35 & 5466.00 & 55504.33 & 1.02 & 0.99 & 1.02 \\
10825 & 101334 & 1995 & 9.99 & 0.54 & 999.00 & 8256.66 & 1.00 & 0.83 & 0.83 \\
12210 & 101519 & 1995 & 114.27 & 0.22 & 12116.00 & 109500.95 & 0.94 & 0.96 & 0.90 \\
28243 & 105399 & 1995 & 1.30 & 0.07 & 67.00 & 689.42 & 1.94 & 0.53 & 1.03 \\
13348 & 101729 & 1995 & 278.30 & 0.27 & 26197.00 & 276038.86 & 1.06 & 0.99 & 1.05 \\
9657 & 101161 & 1995 & 363.97 & 0.32 & 33270.00 & 290983.16 & 1.09 & 0.80 & 0.87 \\
12216 & 101523 & 1995 & 422.28 & 0.27 & 41901.00 & 353736.04 & 1.01 & 0.84 & 0.84 \\
9676 & 101165 & 1995 & 223.61 & 0.36 & 22227.00 & 225463.31 & 1.01 & 1.01 & 1.01 \\
172 & 100017 & 1995 & 23.52 & 0.33 & 2475.00 & 24743.67 & 0.95 & 1.05 & 1.00 \\
12715 & 101590 & 1995 & 20.52 & 0.23 & 2042.00 & 18943.00 & 1.00 & 0.92 & 0.93 \\
28860 & 105489 & 1995 & 1.92 & 0.03 & 152.00 & 1318.57 & 1.26 & 0.69 & 0.87 \\
8063 & 101073 & 1995 & 2350.10 & 0.38 & 210782.00 & 1988973.76 & 1.11 & 0.85 & 0.94 \\
10857 & 101340 & 1995 & 4754.49 & 0.34 & 475449.00 & 3917061.38 & 1.00 & 0.82 & 0.82 \\
61 & 100004 & 1995 & 709.22 & 0.31 & 69201.00 & 686891.85 & 1.02 & 0.97 & 0.99 \\
12303 & 101534 & 1995 & 521.16 & 0.26 & 46938.00 & 459562.10 & 1.11 & 0.88 & 0.98 \\
9825 & 101194 & 1995 & 91.82 & 0.32 & 7278.00 & 76265.85 & 1.26 & 0.83 & 1.05 \\
8896 & 101104 & 1995 & 89.40 & 0.34 & 9094.00 & 73479.19 & 0.98 & 0.82 & 0.81 \\
10968 & 101357 & 1995 & 91.21 & 0.33 & 16613.00 & 151445.02 & 0.55 & 1.66 & 0.91 \\
9855 & 101198 & 1995 & 125.36 & 0.24 & 10851.00 & 87905.10 & 1.16 & 0.70 & 0.81 \\
8625 & 101092 & 1995 & 586.90 & 0.53 & 32058.00 & 294540.86 & 1.83 & 0.50 & 0.92 \\
12321 & 101536 & 1995 & 464.59 & 0.28 & 41976.00 & 443150.56 & 1.11 & 0.95 & 1.06 \\
9883 & 101200 & 1995 & 25.08 & 0.46 & 2028.00 & 23068.03 & 1.24 & 0.92 & 1.14 \\
7866 & 101064 & 1995 & 313.60 & 0.11 & 35392.00 & 289652.15 & 0.89 & 0.92 & 0.82 \\
12354 & 101537 & 1995 & 202.70 & 0.35 & 18805.00 & 190782.86 & 1.08 & 0.94 & 1.01 \\
12910 & 101606 & 1995 & 1705.08 & 0.34 & 143920.00 & 1347823.85 & 1.18 & 0.79 & 0.94 \\
49183 & 240243 & 1995 & 540.43 & 0.09 & 56968.00 & 553229.92 & 0.95 & 1.02 & 0.97 \\
28554 & 105439 & 1995 & 155.77 & 0.28 & 13124.00 & 144373.80 & 1.19 & 0.93 & 1.10 \\
28555 & 105440 & 1995 & 55.94 & 0.22 & 4421.00 & 47299.21 & 1.27 & 0.85 & 1.07 \\
11501 & 101425 & 1995 & 13.25 & 0.31 & 1334.00 & 11991.03 & 0.99 & 0.90 & 0.90 \\
10253 & 101276 & 1995 & 353.00 & 0.38 & 27698.00 & 275468.79 & 1.27 & 0.78 & 0.99 \\
12666 & 101562 & 1995 & 141.79 & 0.34 & 11965.00 & 117580.37 & 1.19 & 0.83 & 0.98 \\
11479 & 101422 & 1995 & 27.76 & 0.15 & 3306.00 & 34549.80 & 0.84 & 1.24 & 1.05 \\
114 & 100009 & 1995 & 83.37 & 0.34 & 7602.00 & 76750.35 & 1.10 & 0.92 & 1.01 \\
7828 & 101062 & 1995 & 708.70 & 0.11 & 52899.00 & 504117.12 & 1.34 & 0.71 & 0.95 \\
13268 & 101716 & 1995 & 22.48 & 0.31 & 2030.00 & 17694.31 & 1.11 & 0.79 & 0.87 \\
9144 & 101115 & 1995 & 3658.30 & 0.37 & 298494.00 & 3139644.74 & 1.23 & 0.86 & 1.05 \\
7698 & 101055 & 1995 & 6278.80 & 0.40 & 645121.00 & 5424693.70 & 0.97 & 0.86 & 0.84 \\
10317 & 101279 & 1995 & 47.39 & 0.17 & 4739.00 & 43027.52 & 1.00 & 0.91 & 0.91 \\
28527 & 105436 & 1995 & 12.82 & 0.20 & 646.00 & 6490.76 & 1.98 & 0.51 & 1.00 \\
28547 & 105438 & 1995 & 49.69 & 0.27 & 4915.00 & 46142.67 & 1.01 & 0.93 & 0.94 \\
412 & 100055 & 1995 & 5306.58 & 0.32 & 496452.00 & 4813096.10 & 1.07 & 0.91 & 0.97 \\
13261 & 101714 & 1995 & 32.38 & 0.17 & 3210.00 & 32111.71 & 1.01 & 0.99 & 1.00 \\
29193 & 105536 & 1995 & 125.20 & 0.55 & 7569.00 & 66896.40 & 1.65 & 0.53 & 0.88 \\
9746 & 101186 & 1995 & 170.24 & 0.34 & 14690.00 & 156440.84 & 1.16 & 0.92 & 1.06 \\
334 & 100040 & 1995 & 10.91 & 0.53 & 1049.00 & 9647.43 & 1.04 & 0.88 & 0.92 \\
11555 & 101430 & 1995 & 97.05 & 0.27 & 5433.00 & 55119.08 & 1.79 & 0.57 & 1.01 \\
9764 & 101192 & 1995 & 8.78 & 0.24 & 765.00 & 6965.96 & 1.15 & 0.79 & 0.91 \\
7522 & 101043 & 1995 & 1249.90 & 0.36 & 134327.00 & 1140740.07 & 0.93 & 0.91 & 0.85 \\
10288 & 101278 & 1995 & 575.07 & 0.13 & 40806.00 & 407895.24 & 1.41 & 0.71 & 1.00 \\
13277 & 101717 & 1995 & 20.83 & 0.29 & 1853.00 & 18488.77 & 1.12 & 0.89 & 1.00 \\
9794 & 101193 & 1995 & 70.70 & 0.32 & 4844.00 & 46534.58 & 1.46 & 0.66 & 0.96 \\
8754 & 101097 & 1995 & 451.60 & 0.30 & 45681.00 & 390353.47 & 0.99 & 0.86 & 0.85 \\
24588 & 103370 & 1995 & 16.61 & 0.68 & 1307.00 & 13864.27 & 1.27 & 0.83 & 1.06 \\
57962 & 410010 & 1995 & 107.75 & 0.31 & 10773.00 & 92962.08 & 1.00 & 0.86 & 0.86 \\
17305 & 102280 & 1995 & 1726.55 & 0.35 & 161815.00 & 1557696.01 & 1.07 & 0.90 & 0.96 \\
2885 & 100369 & 1995 & 193.24 & 0.37 & 18629.00 & 169833.89 & 1.04 & 0.88 & 0.91 \\
22011 & 102987 & 1995 & 289.10 & 0.40 & 31102.00 & 302086.22 & 0.93 & 1.04 & 0.97 \\
26169 & 103545 & 1995 & 11012.95 & 0.39 & 1027216.00 & 10202073.53 & 1.07 & 0.93 & 0.99 \\
5170 & 100730 & 1995 & 571.49 & 0.48 & 53487.00 & 457132.21 & 1.07 & 0.80 & 0.85 \\
26135 & 103544 & 1995 & 1570.89 & 0.36 & 157089.00 & 1518151.59 & 1.00 & 0.97 & 0.97 \\
5152 & 100727 & 1995 & 512.87 & 0.31 & 48387.00 & 510174.44 & 1.06 & 0.99 & 1.05 \\
2894 & 100371 & 1995 & 9.64 & 0.22 & 1055.00 & 9889.34 & 0.91 & 1.03 & 0.94 \\
17379 & 102284 & 1995 & 214.19 & 0.20 & 20273.00 & 211513.17 & 1.06 & 0.99 & 1.04 \\
2899 & 100373 & 1995 & 70.94 & 0.04 & 6386.00 & 61053.60 & 1.11 & 0.86 & 0.96 \\
2910 & 100379 & 1995 & 367.35 & 0.31 & 36615.00 & 359951.18 & 1.00 & 0.98 & 0.98 \\
5192 & 100731 & 1995 & 9368.95 & 0.35 & 823994.00 & 7867555.75 & 1.14 & 0.84 & 0.95 \\
17400 & 102286 & 1995 & 32.20 & 0.19 & 3223.00 & 27095.90 & 1.00 & 0.84 & 0.84 \\
5129 & 100726 & 1995 & 2383.20 & 0.42 & 183744.00 & 1991929.41 & 1.30 & 0.84 & 1.08 \\
21985 & 102983 & 1995 & 276.85 & 0.31 & 31169.00 & 307515.32 & 0.89 & 1.11 & 0.99 \\
5109 & 100724 & 1995 & 47.14 & 0.34 & 4236.00 & 46712.86 & 1.11 & 0.99 & 1.10 \\
17432 & 102306 & 1995 & 1955.75 & 0.39 & 164647.00 & 1595563.69 & 1.19 & 0.82 & 0.97 \\
21952 & 102981 & 1995 & 60.48 & 0.19 & 7148.00 & 63910.06 & 0.85 & 1.06 & 0.89 \\
58039 & 410060 & 1995 & 41.07 & -0.00 & 4445.00 & 39766.50 & 0.92 & 0.97 & 0.89 \\
1018 & 100127 & 1995 & 709.78 & 0.45 & 71853.00 & 572185.81 & 0.99 & 0.81 & 0.80 \\
17482 & 102313 & 1995 & 420.54 & 0.30 & 37639.00 & 399903.99 & 1.12 & 0.95 & 1.06 \\
5078 & 100723 & 1995 & 27.07 & 0.33 & 2189.00 & 24503.27 & 1.24 & 0.91 & 1.12 \\
63173 & 500486 & 1995 & 191.18 & 0.23 & 16196.00 & 165208.84 & 1.18 & 0.86 & 1.02 \\
26054 & 103536 & 1995 & 98.34 & 0.38 & 9834.00 & 96724.28 & 1.00 & 0.98 & 0.98 \\
26024 & 103535 & 1995 & 254.23 & 0.34 & 25423.00 & 252475.79 & 1.00 & 0.99 & 0.99 \\
17415 & 102305 & 1995 & 195.51 & 0.43 & 19002.00 & 189801.22 & 1.03 & 0.97 & 1.00 \\
22042 & 102988 & 1995 & 22.41 & 0.23 & 2801.00 & 25457.81 & 0.80 & 1.14 & 0.91 \\
17287 & 102278 & 1995 & 277.30 & 0.32 & 27686.00 & 268595.00 & 1.00 & 0.97 & 0.97 \\
45876 & 200153 & 1995 & 5.98 & 0.31 & 379.00 & 3424.35 & 1.58 & 0.57 & 0.90 \\
5341 & 100754 & 1995 & 566.54 & 0.36 & 56061.00 & 541984.97 & 1.01 & 0.96 & 0.97 \\
24988 & 103406 & 1995 & 1206.31 & 0.34 & 121884.00 & 1150849.96 & 0.99 & 0.95 & 0.94 \\
17133 & 102258 & 1995 & 883.16 & 0.28 & 92832.00 & 872878.13 & 0.95 & 0.99 & 0.94 \\
22181 & 102994 & 1995 & 59.52 & 0.34 & 5431.00 & 61023.42 & 1.10 & 1.03 & 1.12 \\
5320 & 100753 & 1995 & 1425.95 & 0.32 & 161377.00 & 1382535.58 & 0.88 & 0.97 & 0.86 \\
57908 & 402003 & 1995 & 59.25 & 0.29 & 5714.00 & 46566.02 & 1.04 & 0.79 & 0.81 \\
17143 & 102259 & 1995 & 302.84 & 0.29 & 30840.00 & 278971.11 & 0.98 & 0.92 & 0.90 \\
2848 & 100365 & 1995 & 121.56 & 0.39 & 12189.00 & 120509.88 & 1.00 & 0.99 & 0.99 \\
2857 & 100366 & 1995 & 4.08 & 0.09 & 659.00 & 5977.57 & 0.62 & 1.46 & 0.91 \\
17157 & 102261 & 1995 & 1092.68 & 0.35 & 109299.00 & 1025646.51 & 1.00 & 0.94 & 0.94 \\
17186 & 102270 & 1995 & 147.03 & 0.29 & 15142.00 & 149033.05 & 0.97 & 1.01 & 0.98 \\
5286 & 100746 & 1995 & 806.75 & 0.29 & 64209.00 & 675579.75 & 1.26 & 0.84 & 1.05 \\
974 & 100113 & 1995 & 938.69 & 0.34 & 98712.00 & 933898.18 & 0.95 & 0.99 & 0.95 \\
5210 & 100736 & 1995 & 360.78 & 0.62 & 27373.00 & 353027.13 & 1.32 & 0.98 & 1.29 \\
2876 & 100368 & 1995 & 134.82 & 0.32 & 12960.00 & 118411.88 & 1.04 & 0.88 & 0.91 \\
17247 & 102274 & 1995 & 152.48 & 0.27 & 14022.00 & 149037.48 & 1.09 & 0.98 & 1.06 \\
5243 & 100741 & 1995 & 149.95 & 0.23 & 13922.00 & 140230.30 & 1.08 & 0.94 & 1.01 \\
2949 & 100389 & 1995 & 199.47 & 0.33 & 19910.00 & 182288.97 & 1.00 & 0.91 & 0.92 \\
22103 & 102990 & 1995 & 398.91 & 0.38 & 33169.00 & 347336.35 & 1.20 & 0.87 & 1.05 \\
5265 & 100745 & 1995 & 3232.00 & 0.28 & 302389.00 & 2501436.25 & 1.07 & 0.77 & 0.83 \\
26209 & 103546 & 1995 & 13646.27 & 0.30 & 1300700.00 & 10829379.86 & 1.05 & 0.79 & 0.83 \\
22137 & 102993 & 1995 & 924.95 & 0.38 & 74734.00 & 742279.71 & 1.24 & 0.80 & 0.99 \\
26240 & 103547 & 1995 & 3624.12 & 0.38 & 362412.00 & 3419340.51 & 1.00 & 0.94 & 0.94 \\
1591 & 100217 & 1995 & 106.01 & 0.25 & 9054.00 & 89340.79 & 1.17 & 0.84 & 0.99 \\
57928 & 410003 & 1995 & 182.07 & 0.47 & 16488.00 & 162324.59 & 1.10 & 0.89 & 0.98 \\
45802 & 200142 & 1995 & 9.63 & 0.23 & 1667.00 & 8372.17 & 0.58 & 0.87 & 0.50 \\
17493 & 102314 & 1995 & 331.04 & 0.24 & 53853.00 & 510141.79 & 0.61 & 1.54 & 0.95 \\
25069 & 103429 & 1995 & 827.47 & 0.55 & 82845.00 & 737709.52 & 1.00 & 0.89 & 0.89 \\
21808 & 102952 & 1995 & 877.25 & 0.51 & 87769.00 & 824602.85 & 1.00 & 0.94 & 0.94 \\
4904 & 100692 & 1995 & 758.17 & 0.46 & 65476.00 & 647857.26 & 1.16 & 0.85 & 0.99 \\
17807 & 102364 & 1995 & 341.51 & 0.35 & 29593.00 & 313127.31 & 1.15 & 0.92 & 1.06 \\
17838 & 102365 & 1995 & 414.86 & 0.37 & 34928.00 & 365966.04 & 1.19 & 0.88 & 1.05 \\
1531 & 100213 & 1995 & 212.79 & 0.31 & 18775.00 & 185417.95 & 1.13 & 0.87 & 0.99 \\
4884 & 100691 & 1995 & 273.15 & 0.38 & 24730.00 & 251458.12 & 1.10 & 0.92 & 1.02 \\
17866 & 102367 & 1995 & 606.96 & 0.29 & 51581.00 & 535958.24 & 1.18 & 0.88 & 1.04 \\
21764 & 102951 & 1995 & 3067.99 & 0.31 & 305632.00 & 2695084.38 & 1.00 & 0.88 & 0.88 \\
17879 & 102371 & 1995 & 13.12 & 0.29 & 1319.00 & 11548.41 & 0.99 & 0.88 & 0.88 \\
4871 & 100688 & 1995 & 104.48 & 0.35 & 9339.00 & 96976.18 & 1.12 & 0.93 & 1.04 \\
62999 & 500466 & 1995 & 154.03 & 0.30 & 14453.00 & 148577.12 & 1.07 & 0.96 & 1.03 \\
21702 & 102940 & 1995 & 138.48 & 0.28 & 12207.00 & 127672.48 & 1.13 & 0.92 & 1.05 \\
18038 & 102387 & 1995 & 24.25 & 0.18 & 2620.00 & 22251.84 & 0.93 & 0.92 & 0.85 \\
18003 & 102386 & 1995 & 162.45 & 0.34 & 14867.00 & 144206.51 & 1.09 & 0.89 & 0.97 \\
4836 & 100685 & 1995 & 8.49 & 0.10 & 730.00 & 8377.71 & 1.16 & 0.99 & 1.15 \\
1513 & 100209 & 1995 & 1740.36 & 0.44 & 174036.00 & 1588084.48 & 1.00 & 0.91 & 0.91 \\
17989 & 102383 & 1995 & 7.40 & 0.30 & 728.00 & 6519.13 & 1.02 & 0.88 & 0.90 \\
17794 & 102358 & 1995 & 41.41 & 0.10 & 4520.00 & 40689.84 & 0.92 & 0.98 & 0.90 \\
3079 & 100408 & 1995 & 137.88 & 0.33 & 13788.00 & 134664.66 & 1.00 & 0.98 & 0.98 \\
45269 & 200005 & 1995 & 1.46 & 0.22 & 116.00 & 1105.85 & 1.26 & 0.76 & 0.95 \\
21746 & 102949 & 1995 & 1565.13 & 0.32 & 149385.00 & 1492166.79 & 1.05 & 0.95 & 1.00 \\
17899 & 102372 & 1995 & 3301.56 & 0.34 & 282707.00 & 2799321.66 & 1.17 & 0.85 & 0.99 \\
4866 & 100687 & 1995 & 130.51 & 0.37 & 11255.00 & 126875.23 & 1.16 & 0.97 & 1.13 \\
17963 & 102377 & 1995 & 134.78 & 0.43 & 11692.00 & 121975.17 & 1.15 & 0.90 & 1.04 \\
22212 & 102996 & 1995 & 225.03 & 0.37 & 20050.00 & 200486.37 & 1.12 & 0.89 & 1.00 \\
21849 & 102957 & 1995 & 185.78 & 0.32 & 17160.00 & 164965.02 & 1.08 & 0.89 & 0.96 \\
5046 & 100710 & 1995 & 183.17 & 0.36 & 17229.00 & 167032.76 & 1.06 & 0.91 & 0.97 \\
1033 & 100128 & 1995 & 18.05 & 0.22 & 1723.00 & 16395.56 & 1.05 & 0.91 & 0.95 \\
25132 & 103436 & 1995 & 9.24 & 0.39 & 815.00 & 6936.28 & 1.13 & 0.75 & 0.85 \\
5022 & 100701 & 1995 & 919.17 & 0.11 & 88577.00 & 817785.71 & 1.04 & 0.89 & 0.92 \\
17544 & 102318 & 1995 & 3505.30 & 0.35 & 350530.00 & 3277888.73 & 1.00 & 0.94 & 0.94 \\
2989 & 100395 & 1995 & 477.43 & 0.36 & 43735.00 & 442621.57 & 1.09 & 0.93 & 1.01 \\
5012 & 100700 & 1995 & 429.14 & 0.24 & 38676.00 & 349257.90 & 1.11 & 0.81 & 0.90 \\
17578 & 102319 & 1995 & 693.35 & 0.29 & 69335.00 & 662203.50 & 1.00 & 0.96 & 0.96 \\
17615 & 102321 & 1995 & 169.57 & 0.30 & 16957.00 & 162055.33 & 1.00 & 0.96 & 0.96 \\
4993 & 100698 & 1995 & 66.96 & 0.15 & 6829.00 & 53920.84 & 0.98 & 0.81 & 0.79 \\
4934 & 100695 & 1995 & 75.37 & 0.31 & 7058.00 & 64831.77 & 1.07 & 0.86 & 0.92 \\
17747 & 102356 & 1995 & 1.25 & 0.29 & 114.00 & 1090.54 & 1.10 & 0.87 & 0.96 \\
21879 & 102964 & 1995 & 73.78 & 0.30 & 6617.00 & 65257.51 & 1.12 & 0.88 & 0.99 \\
17726 & 102350 & 1995 & 27.81 & 0.29 & 2215.00 & 22404.84 & 1.26 & 0.81 & 1.01 \\
17703 & 102349 & 1995 & 192.53 & 0.26 & 16736.00 & 171322.95 & 1.15 & 0.89 & 1.02 \\
25956 & 103531 & 1995 & 506.36 & 0.25 & 50636.00 & 472888.59 & 1.00 & 0.93 & 0.93 \\
4960 & 100697 & 1995 & 73.25 & 0.37 & 5690.00 & 62544.26 & 1.29 & 0.85 & 1.10 \\
3019 & 100398 & 1995 & 71.65 & 0.31 & 7240.00 & 72755.62 & 0.99 & 1.02 & 1.00 \\
17672 & 102342 & 1995 & 48.11 & 0.36 & 4089.00 & 45679.68 & 1.18 & 0.95 & 1.12 \\
25142 & 103439 & 1995 & 114.22 & 0.35 & 11344.00 & 105519.42 & 1.01 & 0.92 & 0.93 \\
21887 & 102969 & 1995 & 234.02 & 0.36 & 15734.00 & 188408.03 & 1.49 & 0.81 & 1.20 \\
17651 & 102334 & 1995 & 102.94 & 0.27 & 10193.00 & 96882.15 & 1.01 & 0.94 & 0.95 \\
4955 & 100696 & 1995 & 5.03 & 0.30 & 420.00 & 4705.47 & 1.20 & 0.94 & 1.12 \\
18076 & 102396 & 1995 & 31.83 & 0.26 & 2816.00 & 25959.96 & 1.13 & 0.82 & 0.92 \\
17101 & 102257 & 1995 & 146.86 & 0.37 & 14503.00 & 135744.69 & 1.01 & 0.92 & 0.94 \\
5350 & 100757 & 1995 & 26.17 & 0.01 & 2909.00 & 27205.64 & 0.90 & 1.04 & 0.94 \\
828 & 100098 & 1995 & 10.70 & 0.52 & 1046.00 & 10460.57 & 1.02 & 0.98 & 1.00 \\
22676 & 103028 & 1995 & 2881.46 & 0.19 & 283976.00 & 2740538.54 & 1.01 & 0.95 & 0.97 \\
16490 & 102151 & 1995 & 14.21 & 0.25 & 1422.00 & 12696.19 & 1.00 & 0.89 & 0.89 \\
2645 & 100350 & 1995 & 84.88 & 0.24 & 8659.00 & 84017.11 & 0.98 & 0.99 & 0.97 \\
16512 & 102152 & 1995 & 147.87 & 0.30 & 13561.00 & 140376.18 & 1.09 & 0.95 & 1.04 \\
22632 & 103027 & 1995 & 418.49 & 0.35 & 48132.00 & 478425.90 & 0.87 & 1.14 & 0.99 \\
16535 & 102154 & 1995 & 141.83 & 0.29 & 12504.00 & 114312.70 & 1.13 & 0.81 & 0.91 \\
1679 & 100223 & 1995 & 669.18 & 0.37 & 66917.00 & 609984.04 & 1.00 & 0.91 & 0.91 \\
5723 & 100790 & 1995 & 25.25 & 0.45 & 2525.00 & 25040.84 & 1.00 & 0.99 & 0.99 \\
2626 & 100348 & 1995 & 93.70 & 0.28 & 9494.00 & 94978.85 & 0.99 & 1.01 & 1.00 \\
2664 & 100351 & 1995 & 144.71 & 0.24 & 14695.00 & 147314.16 & 0.98 & 1.02 & 1.00 \\
16610 & 102163 & 1995 & 859.82 & 0.24 & 87877.00 & 741764.48 & 0.98 & 0.86 & 0.84 \\
5705 & 100789 & 1995 & 11.80 & 0.65 & 809.00 & 9932.89 & 1.46 & 0.84 & 1.23 \\
22595 & 103024 & 1995 & 158.78 & 0.32 & 12302.00 & 147174.93 & 1.29 & 0.93 & 1.20 \\
26579 & 103592 & 1995 & 130.49 & 0.59 & 13169.00 & 116777.77 & 0.99 & 0.89 & 0.89 \\
57735 & 401015 & 1995 & 117.56 & 0.01 & 7256.00 & 71919.78 & 1.62 & 0.61 & 0.99 \\
16622 & 102166 & 1995 & 236.47 & 0.35 & 20240.00 & 215941.85 & 1.17 & 0.91 & 1.07 \\
16653 & 102175 & 1995 & 43.47 & 0.19 & 3454.00 & 33548.17 & 1.26 & 0.77 & 0.97 \\
22564 & 103021 & 1995 & 38.58 & 0.24 & 3857.00 & 37514.98 & 1.00 & 0.97 & 0.97 \\
46262 & 200205 & 1995 & 100.26 & 0.26 & 9587.00 & 83177.35 & 1.05 & 0.83 & 0.87 \\
22552 & 103019 & 1995 & 271.45 & 0.30 & 27204.00 & 221777.04 & 1.00 & 0.82 & 0.82 \\
26613 & 103593 & 1995 & 18786.14 & 0.41 & 1878614.00 & 17438628.99 & 1.00 & 0.93 & 0.93 \\
22712 & 103034 & 1995 & 217.97 & 0.31 & 20723.00 & 195566.48 & 1.05 & 0.90 & 0.94 \\
16409 & 102134 & 1995 & 34.02 & 0.29 & 3180.00 & 31783.05 & 1.07 & 0.93 & 1.00 \\
2535 & 100343 & 1995 & 178.07 & 0.32 & 15703.00 & 157188.56 & 1.13 & 0.88 & 1.00 \\
22789 & 103065 & 1995 & 377.15 & 0.34 & 33619.00 & 379681.60 & 1.12 & 1.01 & 1.13 \\
16196 & 102090 & 1995 & 173.91 & 0.33 & 17008.00 & 157851.90 & 1.02 & 0.91 & 0.93 \\
16230 & 102102 & 1995 & 592.60 & 0.30 & 59260.00 & 542470.44 & 1.00 & 0.92 & 0.92 \\
16252 & 102105 & 1995 & 102.40 & 0.32 & 14035.00 & 96292.38 & 0.73 & 0.94 & 0.69 \\
16268 & 102113 & 1995 & 179.26 & 0.32 & 17930.00 & 187312.57 & 1.00 & 1.04 & 1.04 \\
24877 & 103383 & 1995 & 1774.63 & 0.46 & 167558.00 & 1442852.47 & 1.06 & 0.81 & 0.86 \\
5866 & 100809 & 1995 & 2654.73 & 0.24 & 271145.00 & 2766620.39 & 0.98 & 1.04 & 1.02 \\
26718 & 103601 & 1995 & 73.78 & 0.39 & 7378.00 & 62588.94 & 1.00 & 0.85 & 0.85 \\
5850 & 100808 & 1995 & 89.57 & 0.30 & 7999.00 & 80457.89 & 1.12 & 0.90 & 1.01 \\
16278 & 102121 & 1995 & 9.19 & 0.32 & 919.00 & 8094.57 & 1.00 & 0.88 & 0.88 \\
2607 & 100347 & 1995 & 513.67 & 0.37 & 49896.00 & 493019.00 & 1.03 & 0.96 & 0.99 \\
16400 & 102133 & 1995 & 35.14 & 0.00 & 3125.00 & 31248.33 & 1.12 & 0.89 & 1.00 \\
2604 & 100346 & 1995 & 4.97 & 0.32 & 497.00 & 4232.72 & 1.00 & 0.85 & 0.85 \\
5779 & 100792 & 1995 & 31.55 & 0.43 & 3154.00 & 27608.89 & 1.00 & 0.88 & 0.88 \\
16392 & 102132 & 1995 & 39.69 & 0.29 & 3923.00 & 36928.94 & 1.01 & 0.93 & 0.94 \\
26645 & 103595 & 1995 & 100.15 & 0.25 & 10179.00 & 93776.96 & 0.98 & 0.94 & 0.92 \\
16357 & 102130 & 1995 & 604.50 & 0.29 & 58248.00 & 503898.84 & 1.04 & 0.83 & 0.87 \\
26659 & 103597 & 1995 & 149.36 & 0.38 & 14935.00 & 127551.04 & 1.00 & 0.85 & 0.85 \\
26689 & 103600 & 1995 & 5.41 & 0.40 & 541.00 & 5249.03 & 1.00 & 0.97 & 0.97 \\
24898 & 103394 & 1995 & 35.24 & 0.20 & 3158.00 & 30806.24 & 1.12 & 0.87 & 0.98 \\
22741 & 103057 & 1995 & 2360.28 & 0.22 & 219010.00 & 1955665.85 & 1.08 & 0.83 & 0.89 \\
16303 & 102124 & 1995 & 846.20 & 0.30 & 79625.00 & 828433.13 & 1.06 & 0.98 & 1.04 \\
2684 & 100352 & 1995 & 159.28 & 0.35 & 15994.00 & 153915.75 & 1.00 & 0.97 & 0.96 \\
16686 & 102178 & 1995 & 159.24 & 0.45 & 15812.00 & 152649.37 & 1.01 & 0.96 & 0.97 \\
22292 & 103005 & 1995 & 89.73 & 0.12 & 8932.00 & 71468.66 & 1.00 & 0.80 & 0.80 \\
5501 & 100769 & 1995 & 680.00 & 0.29 & 69165.00 & 691381.64 & 0.98 & 1.02 & 1.00 \\
888 & 100101 & 1995 & 49.50 & 0.79 & 4933.00 & 49334.81 & 1.00 & 1.00 & 1.00 \\
24971 & 103402 & 1995 & 174.77 & 0.31 & 16908.00 & 156433.74 & 1.03 & 0.90 & 0.93 \\
26381 & 103573 & 1995 & 171.97 & 0.37 & 16484.00 & 156748.36 & 1.04 & 0.91 & 0.95 \\
26370 & 103572 & 1995 & 25.27 & 0.32 & 2465.00 & 25760.63 & 1.03 & 1.02 & 1.05 \\
16872 & 102213 & 1995 & 230.67 & 0.27 & 16381.00 & 159734.70 & 1.41 & 0.69 & 0.98 \\
2791 & 100358 & 1995 & 2152.25 & 0.21 & 220568.00 & 2043453.62 & 0.98 & 0.95 & 0.93 \\
22271 & 102999 & 1995 & 526.54 & 0.36 & 41756.00 & 383691.17 & 1.26 & 0.73 & 0.92 \\
26337 & 103570 & 1995 & 12.49 & 0.17 & 1210.00 & 11391.02 & 1.03 & 0.91 & 0.94 \\
5463 & 100764 & 1995 & 186.70 & 0.23 & 19010.00 & 173576.02 & 0.98 & 0.93 & 0.91 \\
5441 & 100763 & 1995 & 413.81 & 0.42 & 37954.00 & 387091.78 & 1.09 & 0.94 & 1.02 \\
2813 & 100360 & 1995 & 570.85 & 0.34 & 52267.00 & 480083.74 & 1.09 & 0.84 & 0.92 \\
17070 & 102241 & 1995 & 206.42 & 0.33 & 20235.00 & 212346.65 & 1.02 & 1.03 & 1.05 \\
5364 & 100758 & 1995 & 56.76 & 0.03 & 5195.00 & 49170.01 & 1.09 & 0.87 & 0.95 \\
930 & 100112 & 1995 & 672.09 & 0.37 & 63075.00 & 573979.06 & 1.07 & 0.85 & 0.91 \\
57897 & 401372 & 1995 & 9.51 & 0.22 & 982.00 & 9132.16 & 0.97 & 0.96 & 0.93 \\
16842 & 102197 & 1995 & 60.47 & 0.33 & 5309.00 & 54471.78 & 1.14 & 0.90 & 1.03 \\
17035 & 102231 & 1995 & 603.92 & 0.28 & 60254.00 & 577440.24 & 1.00 & 0.96 & 0.96 \\
5397 & 100760 & 1995 & 864.56 & 0.29 & 102066.00 & 917184.19 & 0.85 & 1.06 & 0.90 \\
924 & 100111 & 1995 & 44.84 & 0.32 & 4397.00 & 45067.01 & 1.02 & 1.01 & 1.02 \\
16963 & 102224 & 1995 & 971.23 & 0.29 & 97123.00 & 956816.14 & 1.00 & 0.99 & 0.99 \\
26287 & 103564 & 1995 & 992.57 & 0.22 & 100936.00 & 904609.17 & 0.98 & 0.91 & 0.90 \\
22240 & 102997 & 1995 & 2450.03 & 0.43 & 300894.00 & 2477829.80 & 0.81 & 1.01 & 0.82 \\
17078 & 102255 & 1995 & 131.72 & 0.26 & 13037.00 & 119924.53 & 1.01 & 0.91 & 0.92 \\
57836 & 401082 & 1995 & 8.66 & 0.19 & 896.00 & 8208.32 & 0.97 & 0.95 & 0.92 \\
22328 & 103007 & 1995 & 1069.67 & 0.34 & 105794.00 & 1004883.63 & 1.01 & 0.94 & 0.95 \\
26548 & 103591 & 1995 & 61.74 & 0.40 & 6158.00 & 57106.05 & 1.00 & 0.92 & 0.93 \\
5643 & 100784 & 1995 & 1227.93 & 0.33 & 122793.00 & 1133897.24 & 1.00 & 0.92 & 0.92 \\
5635 & 100780 & 1995 & 43.46 & 0.29 & 4153.00 & 41076.40 & 1.05 & 0.95 & 0.99 \\
22523 & 103017 & 1995 & 532.30 & 0.37 & 54520.00 & 567584.82 & 0.98 & 1.07 & 1.04 \\
26516 & 103590 & 1995 & 294.02 & 0.28 & 29480.00 & 280181.74 & 1.00 & 0.95 & 0.95 \\
16716 & 102179 & 1995 & 51.92 & 0.21 & 5172.00 & 49640.76 & 1.00 & 0.96 & 0.96 \\
22496 & 103016 & 1995 & 29.43 & 0.42 & 2330.00 & 24403.51 & 1.26 & 0.83 & 1.05 \\
16720 & 102182 & 1995 & 147.41 & 0.34 & 14045.00 & 138698.80 & 1.05 & 0.94 & 0.99 \\
5618 & 100775 & 1995 & 359.94 & 0.35 & 35079.00 & 366122.45 & 1.03 & 1.02 & 1.04 \\
16737 & 102183 & 1995 & 119.11 & 0.33 & 15386.00 & 146040.85 & 0.77 & 1.23 & 0.95 \\
22481 & 103015 & 1995 & 49.11 & 0.29 & 4388.00 & 44764.82 & 1.12 & 0.91 & 1.02 \\
5599 & 100773 & 1995 & 946.00 & 0.34 & 94600.00 & 929605.87 & 1.00 & 0.98 & 0.98 \\
22463 & 103014 & 1995 & 211.54 & 0.28 & 19834.00 & 209587.23 & 1.07 & 0.99 & 1.06 \\
1638 & 100219 & 1995 & 5.00 & 0.24 & 387.00 & 3504.64 & 1.29 & 0.70 & 0.91 \\
57829 & 401081 & 1995 & 28.52 & 0.01 & 3270.00 & 27522.41 & 0.87 & 0.97 & 0.84 \\
5532 & 100771 & 1995 & 46.00 & 0.12 & 4506.00 & 45022.80 & 1.02 & 0.98 & 1.00 \\
22372 & 103008 & 1995 & 60.69 & 0.26 & 6045.00 & 59758.94 & 1.00 & 0.98 & 0.99 \\
16813 & 102193 & 1995 & 294.04 & 0.00 & 25933.00 & 268919.69 & 1.13 & 0.91 & 1.04 \\
26428 & 103580 & 1995 & 166.72 & 0.30 & 16596.00 & 151723.60 & 1.00 & 0.91 & 0.91 \\
858 & 100099 & 1995 & 23.80 & 0.61 & 2152.00 & 21519.03 & 1.11 & 0.90 & 1.00 \\
26479 & 103582 & 1995 & 14.30 & 0.37 & 1256.00 & 12078.26 & 1.14 & 0.84 & 0.96 \\
22408 & 103011 & 1995 & 88.80 & 0.28 & 9050.00 & 82898.58 & 0.98 & 0.93 & 0.92 \\
16786 & 102192 & 1995 & 23.60 & 0.33 & 2018.00 & 20980.51 & 1.17 & 0.89 & 1.04 \\
2716 & 100355 & 1995 & 372.47 & 0.37 & 28104.00 & 298967.44 & 1.33 & 0.80 & 1.06 \\
2747 & 100357 & 1995 & 217.40 & 0.36 & 16974.00 & 195724.80 & 1.28 & 0.90 & 1.15 \\
3101 & 100409 & 1995 & 190.47 & 0.32 & 19047.00 & 182594.65 & 1.00 & 0.96 & 0.96 \\
21672 & 102939 & 1995 & 1063.44 & 0.37 & 82671.00 & 878590.73 & 1.29 & 0.83 & 1.06 \\
19571 & 102633 & 1995 & 259.13 & 0.21 & 25269.00 & 244092.21 & 1.03 & 0.94 & 0.97 \\
4058 & 100544 & 1995 & 47.69 & 0.21 & 4259.00 & 43529.05 & 1.12 & 0.91 & 1.02 \\
20934 & 102812 & 1995 & 51.64 & 0.32 & 4794.00 & 43977.67 & 1.08 & 0.85 & 0.92 \\
19589 & 102635 & 1995 & 221.27 & 0.35 & 20518.00 & 180884.73 & 1.08 & 0.82 & 0.88 \\
19599 & 102636 & 1995 & 528.85 & 0.30 & 52150.00 & 475770.97 & 1.01 & 0.90 & 0.91 \\
20922 & 102802 & 1995 & 1053.29 & 0.20 & 109509.00 & 898072.43 & 0.96 & 0.85 & 0.82 \\
4044 & 100543 & 1995 & 521.25 & 0.41 & 50222.00 & 534199.65 & 1.04 & 1.02 & 1.06 \\
19631 & 102639 & 1995 & 101.21 & 0.10 & 10281.00 & 99679.10 & 0.98 & 0.98 & 0.97 \\
19652 & 102641 & 1995 & 390.83 & 0.28 & 37459.00 & 379442.16 & 1.04 & 0.97 & 1.01 \\
19665 & 102645 & 1995 & 229.45 & 0.38 & 21268.00 & 216081.64 & 1.08 & 0.94 & 1.02 \\
65057 & 500659 & 1995 & 20.34 & 0.09 & 2044.00 & 20934.30 & 0.99 & 1.03 & 1.02 \\
19564 & 102628 & 1995 & 195.69 & 0.37 & 17747.00 & 182579.50 & 1.10 & 0.93 & 1.03 \\
4010 & 100538 & 1995 & 603.25 & 0.24 & 60565.00 & 531321.43 & 1.00 & 0.88 & 0.88 \\
19760 & 102651 & 1995 & 457.26 & 0.37 & 45726.00 & 436187.14 & 1.00 & 0.95 & 0.95 \\
20852 & 102797 & 1995 & 53.14 & 0.17 & 5463.00 & 49318.70 & 0.97 & 0.93 & 0.90 \\
3967 & 100535 & 1995 & 267.20 & 0.35 & 21818.00 & 252985.41 & 1.22 & 0.95 & 1.16 \\
25276 & 103464 & 1995 & 1067.81 & 0.32 & 92189.00 & 971779.65 & 1.16 & 0.91 & 1.05 \\
65200 & 500670 & 1995 & 509.00 & -0.09 & 57027.00 & 480106.24 & 0.89 & 0.94 & 0.84 \\
25492 & 103494 & 1995 & 317.57 & 0.32 & 28168.00 & 290100.43 & 1.13 & 0.91 & 1.03 \\
20943 & 102813 & 1995 & 270.97 & 0.21 & 25080.00 & 216201.26 & 1.08 & 0.80 & 0.86 \\
19217 & 102570 & 1995 & 206.89 & 0.51 & 20689.00 & 199002.83 & 1.00 & 0.96 & 0.96 \\
19249 & 102575 & 1995 & 142.43 & 0.59 & 13048.00 & 130735.78 & 1.09 & 0.92 & 1.00 \\
19265 & 102578 & 1995 & 123.51 & 0.24 & 22984.00 & 208973.99 & 0.54 & 1.69 & 0.91 \\
19277 & 102579 & 1995 & 635.62 & 0.48 & 62049.00 & 538330.39 & 1.02 & 0.85 & 0.87 \\
19298 & 102588 & 1995 & 209.60 & 0.48 & 20227.00 & 204810.19 & 1.04 & 0.98 & 1.01 \\
21019 & 102824 & 1995 & 63.00 & 0.26 & 6035.00 & 57708.08 & 1.04 & 0.92 & 0.96 \\
4171 & 100567 & 1995 & 2393.89 & 0.33 & 239389.00 & 2032823.80 & 1.00 & 0.85 & 0.85 \\
21011 & 102823 & 1995 & 33.20 & 0.28 & 3197.00 & 31467.78 & 1.04 & 0.95 & 0.98 \\
1249 & 100167 & 1995 & 194.28 & 0.36 & 17269.00 & 180020.32 & 1.13 & 0.93 & 1.04 \\
25232 & 103463 & 1995 & 143.35 & 0.04 & 15200.00 & 125758.75 & 0.94 & 0.88 & 0.83 \\
19318 & 102591 & 1995 & 53.61 & 0.35 & 4537.00 & 50963.64 & 1.18 & 0.95 & 1.12 \\
1444 & 100200 & 1995 & 23.43 & 0.12 & 2219.00 & 22089.80 & 1.06 & 0.94 & 1.00 \\
19324 & 102594 & 1995 & 7.42 & 0.28 & 1118.00 & 11205.62 & 0.66 & 1.51 & 1.00 \\
19328 & 102597 & 1995 & 15.70 & 0.37 & 1565.00 & 14210.74 & 1.00 & 0.90 & 0.91 \\
3391 & 100431 & 1995 & 162.25 & 0.59 & 12383.00 & 117591.10 & 1.31 & 0.72 & 0.95 \\
20973 & 102814 & 1995 & 21.96 & 0.24 & 2085.00 & 20202.01 & 1.05 & 0.92 & 0.97 \\
19544 & 102614 & 1995 & 175.16 & 0.28 & 17248.00 & 176414.57 & 1.02 & 1.01 & 1.02 \\
20989 & 102818 & 1995 & 117.67 & 0.29 & 11479.00 & 102167.38 & 1.03 & 0.87 & 0.89 \\
19537 & 102612 & 1995 & 78.75 & 0.38 & 7343.00 & 74904.06 & 1.07 & 0.95 & 1.02 \\
19504 & 102608 & 1995 & 27.48 & 0.12 & 2665.00 & 27600.17 & 1.03 & 1.00 & 1.04 \\
21000 & 102821 & 1995 & 491.02 & 0.36 & 50205.00 & 455686.77 & 0.98 & 0.93 & 0.91 \\
19873 & 102654 & 1995 & 1028.21 & 0.44 & 102821.00 & 996877.37 & 1.00 & 0.97 & 0.97 \\
19487 & 102607 & 1995 & 1046.51 & 0.19 & 108654.00 & 993374.00 & 0.96 & 0.95 & 0.91 \\
4106 & 100552 & 1995 & 425.37 & 0.30 & 46029.00 & 403967.60 & 0.92 & 0.95 & 0.88 \\
19430 & 102601 & 1995 & 2466.24 & 0.27 & 226837.00 & 2299017.38 & 1.09 & 0.93 & 1.01 \\
19396 & 102600 & 1995 & 358.49 & 0.32 & 31014.00 & 306814.74 & 1.16 & 0.86 & 0.99 \\
4125 & 100559 & 1995 & 27.09 & 0.36 & 2793.00 & 28163.00 & 0.97 & 1.04 & 1.01 \\
19464 & 102606 & 1995 & 5213.78 & 0.27 & 487581.00 & 4488077.12 & 1.07 & 0.86 & 0.92 \\
3491 & 100441 & 1995 & 660.96 & 0.28 & 65102.00 & 555226.50 & 1.02 & 0.84 & 0.85 \\
20614 & 102775 & 1995 & 501.66 & 0.34 & 48403.00 & 399537.08 & 1.04 & 0.80 & 0.83 \\
1344 & 100190 & 1995 & 1225.39 & 0.33 & 122539.00 & 1090388.43 & 1.00 & 0.89 & 0.89 \\
3573 & 100457 & 1995 & 34.35 & -0.01 & 3866.00 & 38867.40 & 0.89 & 1.13 & 1.01 \\
20242 & 102695 & 1995 & 1.30 & 0.11 & 160.00 & 1575.20 & 0.81 & 1.21 & 0.98 \\
3749 & 100480 & 1995 & 26.45 & 0.37 & 2180.00 & 20033.63 & 1.21 & 0.76 & 0.92 \\
61325 & 500037 & 1995 & 2247.77 & 0.29 & 201384.00 & 2028025.13 & 1.12 & 0.90 & 1.01 \\
20594 & 102774 & 1995 & 1387.78 & 0.39 & 120465.00 & 1143575.91 & 1.15 & 0.82 & 0.95 \\
20250 & 102696 & 1995 & 186.40 & 0.20 & 16327.00 & 157803.50 & 1.14 & 0.85 & 0.97 \\
65080 & 500660 & 1995 & 175.42 & -0.27 & 21196.00 & 205819.43 & 0.83 & 1.17 & 0.97 \\
25395 & 103482 & 1995 & 4.18 & 0.24 & 433.00 & 4400.31 & 0.97 & 1.05 & 1.02 \\
25375 & 103481 & 1995 & 73.29 & 0.21 & 7337.00 & 63168.96 & 1.00 & 0.86 & 0.86 \\
20554 & 102767 & 1995 & 1089.80 & 0.37 & 100395.00 & 1024632.03 & 1.09 & 0.94 & 1.02 \\
3719 & 100475 & 1995 & 93.88 & 0.20 & 9280.00 & 94493.40 & 1.01 & 1.01 & 1.02 \\
20285 & 102709 & 1995 & 2.20 & 0.39 & 218.00 & 1868.94 & 1.01 & 0.85 & 0.86 \\
25344 & 103477 & 1995 & 38.70 & 0.29 & 3410.00 & 33839.23 & 1.13 & 0.87 & 0.99 \\
20297 & 102715 & 1995 & 3357.86 & 0.31 & 284040.00 & 2783536.25 & 1.18 & 0.83 & 0.98 \\
1363 & 100192 & 1995 & 47.77 & 0.33 & 4776.00 & 46988.12 & 1.00 & 0.98 & 0.98 \\
20444 & 102744 & 1995 & 236.88 & 0.34 & 22897.00 & 205841.53 & 1.03 & 0.87 & 0.90 \\
25371 & 103479 & 1995 & 43.24 & 0.25 & 4484.00 & 36836.39 & 0.96 & 0.85 & 0.82 \\
20517 & 102761 & 1995 & 14574.45 & 0.34 & 1328182.00 & 13211827.05 & 1.10 & 0.91 & 0.99 \\
3779 & 100481 & 1995 & 27.33 & 0.31 & 2617.00 & 22393.69 & 1.04 & 0.82 & 0.86 \\
20414 & 102737 & 1995 & 118.51 & 0.02 & 10194.00 & 87941.34 & 1.16 & 0.74 & 0.86 \\
20382 & 102733 & 1995 & 4070.47 & 0.25 & 390305.00 & 3641275.37 & 1.04 & 0.89 & 0.93 \\
3617 & 100463 & 1995 & 2.57 & 0.27 & 179.00 & 2145.33 & 1.44 & 0.83 & 1.20 \\
20365 & 102728 & 1995 & 470.22 & 0.20 & 47767.00 & 416120.67 & 0.98 & 0.88 & 0.87 \\
3676 & 100468 & 1995 & 139.29 & 0.24 & 13928.00 & 135427.68 & 1.00 & 0.97 & 0.97 \\
25348 & 103478 & 1995 & 57.81 & 0.39 & 5009.00 & 47589.03 & 1.15 & 0.82 & 0.95 \\
20413 & 102736 & 1995 & 251.11 & -0.02 & 22050.00 & 208119.66 & 1.14 & 0.83 & 0.94 \\
25549 & 103496 & 1995 & 337.27 & 0.37 & 33940.00 & 341017.58 & 0.99 & 1.01 & 1.00 \\
20198 & 102688 & 1995 & 74.86 & 0.33 & 7469.00 & 72542.49 & 1.00 & 0.97 & 0.97 \\
3560 & 100456 & 1995 & 2.08 & 0.23 & 169.00 & 1611.27 & 1.23 & 0.77 & 0.95 \\
25429 & 103487 & 1995 & 23.98 & 0.20 & 2134.00 & 21629.35 & 1.12 & 0.90 & 1.01 \\
25425 & 103485 & 1995 & 28.57 & 0.33 & 2897.00 & 26078.54 & 0.99 & 0.91 & 0.90 \\
19917 & 102655 & 1995 & 1090.39 & 0.21 & 109039.00 & 960220.68 & 1.00 & 0.88 & 0.88 \\
3913 & 100514 & 1995 & 88.76 & 0.29 & 8876.00 & 76436.87 & 1.00 & 0.86 & 0.86 \\
25421 & 103484 & 1995 & 52.94 & 0.30 & 4413.00 & 40888.49 & 1.20 & 0.77 & 0.93 \\
20763 & 102789 & 1995 & 350.30 & 0.37 & 33131.00 & 279492.74 & 1.06 & 0.80 & 0.84 \\
65130 & 500664 & 1995 & 1585.51 & 0.29 & 139981.00 & 1213621.01 & 1.13 & 0.77 & 0.87 \\
25405 & 103483 & 1995 & 172.05 & 0.29 & 16970.00 & 166724.10 & 1.01 & 0.97 & 0.98 \\
25307 & 103466 & 1995 & 267.26 & 0.32 & 21000.00 & 211118.58 & 1.27 & 0.79 & 1.01 \\
3873 & 100508 & 1995 & 21.42 & 0.30 & 1715.00 & 18372.77 & 1.25 & 0.86 & 1.07 \\
3865 & 100507 & 1995 & 26.71 & 0.37 & 1934.00 & 18542.18 & 1.38 & 0.69 & 0.96 \\
3857 & 100506 & 1995 & 48.76 & 0.36 & 3660.00 & 40265.04 & 1.33 & 0.83 & 1.10 \\
19945 & 102659 & 1995 & 2854.55 & 0.55 & 285454.00 & 2638778.41 & 1.00 & 0.92 & 0.92 \\
1414 & 100196 & 1995 & 515.84 & 0.35 & 51583.00 & 413320.90 & 1.00 & 0.80 & 0.80 \\
19975 & 102660 & 1995 & 146.45 & 0.38 & 7754.00 & 68490.00 & 1.89 & 0.47 & 0.88 \\
3850 & 100505 & 1995 & 6.90 & 0.04 & 679.00 & 5977.32 & 1.02 & 0.87 & 0.88 \\
61283 & 500027 & 1995 & 136.48 & 0.12 & 13690.00 & 118755.64 & 1.00 & 0.87 & 0.87 \\
20170 & 102673 & 1995 & 174.34 & 0.34 & 17410.00 & 145565.46 & 1.00 & 0.83 & 0.84 \\
20149 & 102671 & 1995 & 55.25 & 0.37 & 4933.00 & 52707.76 & 1.12 & 0.95 & 1.07 \\
3551 & 100455 & 1995 & 4.35 & 0.26 & 387.00 & 4086.80 & 1.12 & 0.94 & 1.06 \\
61239 & 500005 & 1995 & 3.67 & 0.21 & 323.00 & 2752.40 & 1.14 & 0.75 & 0.85 \\
20183 & 102676 & 1995 & 38.50 & 0.33 & 3834.00 & 35789.45 & 1.00 & 0.93 & 0.93 \\
20138 & 102669 & 1995 & 24.98 & 0.32 & 2299.00 & 24169.04 & 1.09 & 0.97 & 1.05 \\
20705 & 102784 & 1995 & 8937.23 & 0.40 & 713875.00 & 7097233.04 & 1.25 & 0.79 & 0.99 \\
20105 & 102667 & 1995 & 3236.84 & 0.35 & 323684.00 & 2900930.70 & 1.00 & 0.90 & 0.90 \\
20044 & 102664 & 1995 & 936.42 & 0.29 & 57158.00 & 524274.77 & 1.64 & 0.56 & 0.92 \\
20010 & 102663 & 1995 & 2285.26 & 0.56 & 228526.00 & 1965770.72 & 1.00 & 0.86 & 0.86 \\
20666 & 102783 & 1995 & 1245.36 & 0.30 & 116860.00 & 1015860.66 & 1.07 & 0.82 & 0.87 \\
4208 & 100575 & 1995 & 19.28 & 0.27 & 1969.00 & 19947.92 & 0.98 & 1.03 & 1.01 \\
4222 & 100590 & 1995 & 17.56 & 0.41 & 1064.00 & 11170.88 & 1.65 & 0.64 & 1.05 \\
21054 & 102825 & 1995 & 111.17 & 0.28 & 11233.00 & 107247.63 & 0.99 & 0.96 & 0.95 \\
18350 & 102446 & 1995 & 4.66 & 0.44 & 405.00 & 3989.90 & 1.15 & 0.86 & 0.99 \\
21488 & 102873 & 1995 & 166.82 & 0.19 & 15748.00 & 155394.84 & 1.06 & 0.93 & 0.99 \\
25194 & 103460 & 1995 & 442.85 & 0.50 & 36985.00 & 382517.83 & 1.20 & 0.86 & 1.03 \\
4621 & 100644 & 1995 & 61.59 & 0.28 & 5904.00 & 59249.57 & 1.04 & 0.96 & 1.00 \\
39140 & 107611 & 1995 & 74.57 & 0.05 & 7457.00 & 73462.34 & 1.00 & 0.99 & 0.99 \\
21448 & 102872 & 1995 & 302.12 & 0.18 & 24706.00 & 247127.48 & 1.22 & 0.82 & 1.00 \\
13580 & 101744 & 1995 & 1790.39 & 0.26 & 175320.00 & 1539183.00 & 1.02 & 0.86 & 0.88 \\
25690 & 103514 & 1995 & 1102.76 & 0.37 & 110276.00 & 996761.16 & 1.00 & 0.90 & 0.90 \\
1148 & 100157 & 1995 & 153.03 & 0.37 & 12836.00 & 125311.05 & 1.19 & 0.82 & 0.98 \\
18394 & 102447 & 1995 & 278.61 & 0.34 & 27861.00 & 235857.45 & 1.00 & 0.85 & 0.85 \\
4591 & 100642 & 1995 & 837.99 & 0.35 & 83984.00 & 726276.57 & 1.00 & 0.87 & 0.86 \\
18424 & 102452 & 1995 & 99.28 & 0.33 & 9677.00 & 91449.80 & 1.03 & 0.92 & 0.95 \\
18450 & 102461 & 1995 & 2028.11 & 0.50 & 173196.00 & 1765959.47 & 1.17 & 0.87 & 1.02 \\
21417 & 102871 & 1995 & 97.48 & 0.19 & 8280.00 & 82486.01 & 1.18 & 0.85 & 1.00 \\
18559 & 102483 & 1995 & 74.44 & 0.16 & 7990.00 & 72401.48 & 0.93 & 0.97 & 0.91 \\
4523 & 100637 & 1995 & 1384.41 & 0.30 & 138441.00 & 1354842.34 & 1.00 & 0.98 & 0.98 \\
18549 & 102482 & 1995 & 99.45 & 0.47 & 9616.00 & 80629.05 & 1.03 & 0.81 & 0.84 \\
21356 & 102854 & 1995 & 1144.42 & 0.43 & 114015.00 & 1129053.71 & 1.00 & 0.99 & 0.99 \\
3236 & 100417 & 1995 & 1.66 & 0.32 & 155.00 & 1511.66 & 1.07 & 0.91 & 0.98 \\
1133 & 100155 & 1995 & 126.99 & 0.40 & 11421.00 & 101407.86 & 1.11 & 0.80 & 0.89 \\
21389 & 102861 & 1995 & 66.31 & 0.28 & 5990.00 & 63603.59 & 1.11 & 0.96 & 1.06 \\
18521 & 102470 & 1995 & 551.38 & 0.33 & 47975.00 & 489969.91 & 1.15 & 0.89 & 1.02 \\
21407 & 102862 & 1995 & 7.23 & 0.26 & 714.00 & 6219.51 & 1.01 & 0.86 & 0.87 \\
4557 & 100639 & 1995 & 251.45 & 0.36 & 24281.00 & 248282.79 & 1.04 & 0.99 & 1.02 \\
21413 & 102870 & 1995 & 56.18 & 0.16 & 5242.00 & 42022.06 & 1.07 & 0.75 & 0.80 \\
18540 & 102474 & 1995 & 252.84 & 0.28 & 25598.00 & 256301.69 & 0.99 & 1.01 & 1.00 \\
18573 & 102489 & 1995 & 15.99 & 0.37 & 1400.00 & 15713.16 & 1.14 & 0.98 & 1.12 \\
25722 & 103520 & 1995 & 271.00 & 0.38 & 27100.00 & 269084.74 & 1.00 & 0.99 & 0.99 \\
4671 & 100660 & 1995 & 152.28 & 0.33 & 19146.00 & 198675.07 & 0.80 & 1.30 & 1.04 \\
4803 & 100682 & 1995 & 79.66 & 0.26 & 7966.00 & 65015.44 & 1.00 & 0.82 & 0.82 \\
1068 & 100150 & 1995 & 11.58 & 0.27 & 1148.00 & 11533.45 & 1.01 & 1.00 & 1.00 \\
25918 & 103529 & 1995 & 909.45 & 0.36 & 90945.00 & 867377.49 & 1.00 & 0.95 & 0.95 \\
18116 & 102399 & 1995 & 9.23 & 0.28 & 720.00 & 7043.21 & 1.28 & 0.76 & 0.98 \\
18125 & 102404 & 1995 & 260.10 & 0.37 & 23813.00 & 256442.14 & 1.09 & 0.99 & 1.08 \\
25889 & 103526 & 1995 & 1471.10 & 0.36 & 147110.00 & 1324896.04 & 1.00 & 0.90 & 0.90 \\
4757 & 100671 & 1995 & 118.08 & 0.36 & 12079.00 & 107796.58 & 0.98 & 0.91 & 0.89 \\
21617 & 102901 & 1995 & 38.67 & 0.20 & 4014.00 & 41260.26 & 0.96 & 1.07 & 1.03 \\
18175 & 102414 & 1995 & 767.50 & 0.34 & 68630.00 & 585100.65 & 1.12 & 0.76 & 0.85 \\
3136 & 100411 & 1995 & 4210.50 & 0.36 & 421050.00 & 3906899.08 & 1.00 & 0.93 & 0.93 \\
38980 & 107470 & 1995 & 1.07 & 0.06 & 107.00 & 962.64 & 1.00 & 0.90 & 0.90 \\
21591 & 102895 & 1995 & 604.91 & 0.25 & 59858.00 & 580899.93 & 1.01 & 0.96 & 0.97 \\
25855 & 103525 & 1995 & 6674.39 & 0.38 & 667439.00 & 6613776.96 & 1.00 & 0.99 & 0.99 \\
4735 & 100670 & 1995 & 81.80 & 0.39 & 8243.00 & 77213.46 & 0.99 & 0.94 & 0.94 \\
21492 & 102875 & 1995 & 25.05 & 0.29 & 2308.00 & 24380.30 & 1.09 & 0.97 & 1.06 \\
25754 & 103521 & 1995 & 496.33 & 0.37 & 49633.00 & 489294.21 & 1.00 & 0.99 & 0.99 \\
21503 & 102876 & 1995 & 52.59 & 0.23 & 4941.00 & 47943.82 & 1.06 & 0.91 & 0.97 \\
21527 & 102893 & 1995 & 33.77 & 0.12 & 4120.00 & 40255.72 & 0.82 & 1.19 & 0.98 \\
18316 & 102425 & 1995 & 1123.20 & 0.35 & 90720.00 & 765081.18 & 1.24 & 0.68 & 0.84 \\
25788 & 103523 & 1995 & 1523.21 & 0.34 & 152321.00 & 1437275.76 & 1.00 & 0.94 & 0.94 \\
18266 & 102419 & 1995 & 262.80 & 0.24 & 27253.00 & 242314.02 & 0.96 & 0.92 & 0.89 \\
1098 & 100153 & 1995 & 178.87 & 0.34 & 14072.00 & 122355.05 & 1.27 & 0.68 & 0.87 \\
25822 & 103524 & 1995 & 17371.83 & 0.37 & 1510635.00 & 15716283.52 & 1.15 & 0.90 & 1.04 \\
18222 & 102417 & 1995 & 944.80 & 0.37 & 88252.00 & 788877.57 & 1.07 & 0.83 & 0.89 \\
4698 & 100667 & 1995 & 12.35 & 0.24 & 1242.00 & 9916.24 & 0.99 & 0.80 & 0.80 \\
16165 & 102089 & 1995 & 169.08 & 0.31 & 15910.00 & 169336.06 & 1.06 & 1.00 & 1.06 \\
18585 & 102490 & 1995 & 44.51 & 0.33 & 4134.00 & 45212.10 & 1.08 & 1.02 & 1.09 \\
18632 & 102492 & 1995 & 129.60 & 0.35 & 13399.00 & 112634.34 & 0.97 & 0.87 & 0.84 \\
21226 & 102838 & 1995 & 263.10 & 0.29 & 23995.00 & 258358.32 & 1.10 & 0.98 & 1.08 \\
44393 & 109300 & 1995 & 222.12 & 0.39 & 19884.00 & 203103.55 & 1.12 & 0.91 & 1.02 \\
18906 & 102527 & 1995 & 123.95 & 0.36 & 12230.00 & 120337.79 & 1.01 & 0.97 & 0.98 \\
21190 & 102837 & 1995 & 81.65 & 0.50 & 6718.00 & 72953.71 & 1.22 & 0.89 & 1.09 \\
4299 & 100603 & 1995 & 793.08 & 0.11 & 46422.00 & 469726.10 & 1.71 & 0.59 & 1.01 \\
18954 & 102531 & 1995 & 10.53 & 0.33 & 1052.00 & 9829.41 & 1.00 & 0.93 & 0.93 \\
21158 & 102835 & 1995 & 58.20 & 0.46 & 5698.00 & 58446.95 & 1.02 & 1.00 & 1.03 \\
18986 & 102540 & 1995 & 7.22 & 0.11 & 673.00 & 6784.60 & 1.07 & 0.94 & 1.01 \\
19018 & 102544 & 1995 & 258.82 & 0.37 & 23622.00 & 215661.69 & 1.10 & 0.83 & 0.91 \\
21136 & 102833 & 1995 & 11.64 & 0.36 & 1098.00 & 10608.54 & 1.06 & 0.91 & 0.97 \\
19058 & 102546 & 1995 & 92.50 & 0.34 & 7400.00 & 77501.34 & 1.25 & 0.84 & 1.05 \\
19066 & 102547 & 1995 & 135.20 & 0.33 & 11815.00 & 108712.47 & 1.14 & 0.80 & 0.92 \\
21104 & 102832 & 1995 & 31.59 & 0.33 & 2936.00 & 29008.50 & 1.08 & 0.92 & 0.99 \\
25580 & 103497 & 1995 & 15.99 & 0.24 & 1482.00 & 13565.93 & 1.08 & 0.85 & 0.92 \\
21061 & 102827 & 1995 & 132.77 & 0.32 & 10200.00 & 111384.24 & 1.30 & 0.84 & 1.09 \\
19198 & 102563 & 1995 & 500.56 & 0.30 & 49858.00 & 427526.67 & 1.00 & 0.85 & 0.86 \\
18875 & 102525 & 1995 & 128.62 & 0.30 & 12298.00 & 114865.09 & 1.05 & 0.89 & 0.93 \\
19140 & 102551 & 1995 & 41.66 & 0.25 & 3509.00 & 35531.31 & 1.19 & 0.85 & 1.01 \\
19133 & 102550 & 1995 & 74.08 & 0.22 & 7452.00 & 69387.03 & 0.99 & 0.94 & 0.93 \\
19102 & 102549 & 1995 & 74.11 & 0.36 & 6977.00 & 67625.40 & 1.06 & 0.91 & 0.97 \\
19074 & 102548 & 1995 & 184.78 & 0.37 & 19021.00 & 179114.76 & 0.97 & 0.97 & 0.94 \\
18601 & 102491 & 1995 & 93.20 & 0.22 & 9448.00 & 87819.45 & 0.99 & 0.94 & 0.93 \\
21238 & 102841 & 1995 & 77.52 & 0.24 & 7273.00 & 70216.33 & 1.07 & 0.91 & 0.97 \\
18813 & 102523 & 1995 & 637.19 & 0.19 & 59435.00 & 587593.93 & 1.07 & 0.92 & 0.99 \\
4484 & 100635 & 1995 & 21.59 & 0.17 & 2159.00 & 20304.06 & 1.00 & 0.94 & 0.94 \\
18635 & 102493 & 1995 & 200.00 & 0.27 & 22451.00 & 199664.27 & 0.89 & 1.00 & 0.89 \\
21325 & 102852 & 1995 & 158.51 & 0.20 & 15844.00 & 133651.74 & 1.00 & 0.84 & 0.84 \\
4473 & 100634 & 1995 & 848.73 & 0.33 & 84873.00 & 840471.39 & 1.00 & 0.99 & 0.99 \\
3252 & 100419 & 1995 & 3.35 & 0.34 & 314.00 & 2976.60 & 1.07 & 0.89 & 0.95 \\
25662 & 103500 & 1995 & 156.00 & 0.04 & 12946.00 & 121872.42 & 1.21 & 0.78 & 0.94 \\
18650 & 102500 & 1995 & 783.98 & 0.30 & 73108.00 & 795604.40 & 1.07 & 1.01 & 1.09 \\
21318 & 102848 & 1995 & 83.31 & 0.22 & 7862.00 & 86599.91 & 1.06 & 1.04 & 1.10 \\
21297 & 102846 & 1995 & 82.95 & 0.38 & 8024.00 & 73619.81 & 1.03 & 0.89 & 0.92 \\
4448 & 100633 & 1995 & 358.52 & 0.31 & 35852.00 & 353501.85 & 1.00 & 0.99 & 0.99 \\
18679 & 102502 & 1995 & 485.01 & 0.36 & 43265.00 & 483308.27 & 1.12 & 1.00 & 1.12 \\
4433 & 100625 & 1995 & 410.70 & 0.29 & 38612.00 & 385048.34 & 1.06 & 0.94 & 1.00 \\
59028 & 410401 & 1995 & 240.82 & 0.09 & 23521.00 & 252683.74 & 1.02 & 1.05 & 1.07 \\
4351 & 100611 & 1995 & 1271.45 & 0.34 & 120496.00 & 1115769.47 & 1.06 & 0.88 & 0.93 \\
25624 & 103498 & 1995 & 98.41 & 0.29 & 6618.00 & 62576.71 & 1.49 & 0.64 & 0.95 \\
4377 & 100614 & 1995 & 276.96 & 0.37 & 27095.00 & 250954.25 & 1.02 & 0.91 & 0.93 \\
18796 & 102522 & 1995 & 284.00 & 0.29 & 27350.00 & 270119.96 & 1.04 & 0.95 & 0.99 \\
18844 & 102524 & 1995 & 378.72 & 0.37 & 33615.00 & 350459.47 & 1.13 & 0.93 & 1.04 \\
1180 & 100159 & 1995 & 154.80 & 0.30 & 13496.00 & 137239.29 & 1.15 & 0.89 & 1.02 \\
18783 & 102515 & 1995 & 126.85 & 0.28 & 10400.00 & 118204.84 & 1.22 & 0.93 & 1.14 \\
18770 & 102508 & 1995 & 164.79 & 0.28 & 14228.00 & 158338.54 & 1.16 & 0.96 & 1.11 \\
1475 & 100207 & 1995 & 2488.64 & 0.32 & 234108.00 & 2391658.22 & 1.06 & 0.96 & 1.02 \\
21245 & 102842 & 1995 & 7.71 & 0.29 & 703.00 & 7806.16 & 1.10 & 1.01 & 1.11 \\
4395 & 100622 & 1995 & 110.53 & 0.37 & 10625.00 & 92788.58 & 1.04 & 0.84 & 0.87 \\
2532 & 100337 & 1995 & 114.69 & 0.32 & 11466.00 & 112201.72 & 1.00 & 0.98 & 0.98 \\
26757 & 103606 & 1995 & 25.09 & 0.22 & 2509.00 & 22726.50 & 1.00 & 0.91 & 0.91 \\
23652 & 103205 & 1995 & 18.49 & 0.27 & 1849.00 & 17419.05 & 1.00 & 0.94 & 0.94 \\
6808 & 100958 & 1995 & 1.43 & 0.36 & 143.00 & 1332.29 & 1.00 & 0.93 & 0.93 \\
14654 & 101906 & 1995 & 6.36 & 0.09 & 635.00 & 5822.66 & 1.00 & 0.92 & 0.92 \\
2131 & 100292 & 1995 & 120.62 & 0.19 & 12061.00 & 117848.56 & 1.00 & 0.98 & 0.98 \\
14668 & 101908 & 1995 & 6.49 & 0.26 & 604.00 & 6749.28 & 1.08 & 1.04 & 1.12 \\
23567 & 103193 & 1995 & 39.07 & 0.38 & 3772.00 & 37334.21 & 1.04 & 0.96 & 0.99 \\
6786 & 100954 & 1995 & 207.52 & 0.31 & 12709.00 & 128686.34 & 1.63 & 0.62 & 1.01 \\
14702 & 101911 & 1995 & 135.98 & 0.19 & 14114.00 & 133450.54 & 0.96 & 0.98 & 0.95 \\
23548 & 103186 & 1995 & 41.93 & 0.40 & 3140.00 & 26608.39 & 1.34 & 0.63 & 0.85 \\
6830 & 100962 & 1995 & 1647.57 & 0.27 & 164757.00 & 1430331.96 & 1.00 & 0.87 & 0.87 \\
14736 & 101912 & 1995 & 1876.47 & 0.27 & 187647.00 & 1627071.87 & 1.00 & 0.87 & 0.87 \\
14768 & 101913 & 1995 & 45.40 & 0.31 & 4564.00 & 44230.29 & 0.99 & 0.97 & 0.97 \\
23541 & 103184 & 1995 & 1133.89 & 0.19 & 120147.00 & 1113796.45 & 0.94 & 0.98 & 0.93 \\
14800 & 101914 & 1995 & 36.40 & 0.35 & 3696.00 & 35350.70 & 0.98 & 0.97 & 0.96 \\
24640 & 103373 & 1995 & 167.39 & 0.41 & 16553.00 & 156483.16 & 1.01 & 0.93 & 0.95 \\
591 & 100079 & 1995 & 1052.76 & 0.35 & 105001.00 & 993212.22 & 1.00 & 0.94 & 0.95 \\
2162 & 100293 & 1995 & 45.31 & 0.23 & 5055.00 & 46103.04 & 0.90 & 1.02 & 0.91 \\
27500 & 105283 & 1995 & 3.00 & 0.20 & 255.00 & 2297.42 & 1.18 & 0.77 & 0.90 \\
798 & 100097 & 1995 & 7.90 & 0.55 & 799.00 & 7989.84 & 0.99 & 1.01 & 1.00 \\
1819 & 100244 & 1995 & 100.57 & 0.36 & 10050.00 & 93218.06 & 1.00 & 0.93 & 0.93 \\
27470 & 105281 & 1995 & 29.70 & 0.17 & 2644.00 & 24583.38 & 1.12 & 0.83 & 0.93 \\
6764 & 100953 & 1995 & 10.43 & 0.36 & 962.00 & 8826.78 & 1.08 & 0.85 & 0.92 \\
2195 & 100295 & 1995 & 12.45 & 0.19 & 1269.00 & 10495.32 & 0.98 & 0.84 & 0.83 \\
27511 & 105284 & 1995 & 1.27 & -0.01 & 104.00 & 1144.44 & 1.22 & 0.90 & 1.10 \\
27528 & 105286 & 1995 & 49.14 & 0.23 & 4237.00 & 37735.17 & 1.16 & 0.77 & 0.89 \\
6935 & 100973 & 1995 & 7.13 & 0.30 & 715.00 & 6071.64 & 1.00 & 0.85 & 0.85 \\
47807 & 222027 & 1995 & 89.86 & 0.22 & 9514.00 & 86046.83 & 0.94 & 0.96 & 0.90 \\
14464 & 101861 & 1995 & 426.93 & 0.34 & 39111.00 & 380218.88 & 1.09 & 0.89 & 0.97 \\
27630 & 105304 & 1995 & 4.24 & 0.38 & 412.00 & 4120.66 & 1.03 & 0.97 & 1.00 \\
23791 & 103213 & 1995 & 709.00 & 0.27 & 64320.00 & 624741.53 & 1.10 & 0.88 & 0.97 \\
27601 & 105303 & 1995 & 120.84 & 0.03 & 10956.00 & 110604.13 & 1.10 & 0.92 & 1.01 \\
2091 & 100290 & 1995 & 313.73 & 0.34 & 35070.00 & 337116.92 & 0.89 & 1.07 & 0.96 \\
27586 & 105293 & 1995 & 2.09 & 0.29 & 188.00 & 1806.70 & 1.11 & 0.87 & 0.96 \\
14503 & 101867 & 1995 & 164.65 & 0.36 & 13159.00 & 147435.50 & 1.25 & 0.90 & 1.12 \\
6906 & 100968 & 1995 & 24.40 & 0.31 & 2087.00 & 22044.43 & 1.17 & 0.90 & 1.06 \\
27526 & 105285 & 1995 & 195.69 & 0.29 & 13981.00 & 145336.73 & 1.40 & 0.74 & 1.04 \\
566 & 100076 & 1995 & 309.22 & 0.34 & 22549.00 & 205942.38 & 1.37 & 0.67 & 0.91 \\
14550 & 101876 & 1995 & 138.84 & 0.23 & 13890.00 & 138585.41 & 1.00 & 1.00 & 1.00 \\
6874 & 100967 & 1995 & 266.33 & 0.20 & 24125.00 & 235042.51 & 1.10 & 0.88 & 0.97 \\
23755 & 103212 & 1995 & 1327.66 & 0.27 & 108976.00 & 984274.70 & 1.22 & 0.74 & 0.90 \\
6866 & 100966 & 1995 & 28.19 & 0.28 & 2537.00 & 28222.98 & 1.11 & 1.00 & 1.11 \\
23719 & 103209 & 1995 & 599.06 & 0.35 & 46150.00 & 449620.80 & 1.30 & 0.75 & 0.97 \\
23685 & 103208 & 1995 & 1061.01 & 0.32 & 91632.00 & 867598.89 & 1.16 & 0.82 & 0.95 \\
27537 & 105287 & 1995 & 33.96 & 0.41 & 3277.00 & 27625.08 & 1.04 & 0.81 & 0.84 \\
14508 & 101871 & 1995 & 247.56 & 0.39 & 19236.00 & 223910.22 & 1.29 & 0.90 & 1.16 \\
14443 & 101858 & 1995 & 104.30 & 0.35 & 9493.00 & 91796.89 & 1.10 & 0.88 & 0.97 \\
1809 & 100238 & 1995 & 226.52 & 0.08 & 25454.00 & 243775.67 & 0.89 & 1.08 & 0.96 \\
14826 & 101916 & 1995 & 90.40 & 0.32 & 8965.00 & 79361.12 & 1.01 & 0.88 & 0.89 \\
6581 & 100892 & 1995 & 97.11 & 0.34 & 9377.00 & 84952.20 & 1.04 & 0.87 & 0.91 \\
15040 & 101953 & 1995 & 235.35 & 0.35 & 21133.00 & 239655.33 & 1.11 & 1.02 & 1.13 \\
15057 & 101955 & 1995 & 4135.20 & 0.30 & 301403.00 & 2690952.99 & 1.37 & 0.65 & 0.89 \\
23378 & 103174 & 1995 & 1173.95 & 0.31 & 117041.00 & 1091210.21 & 1.00 & 0.93 & 0.93 \\
2268 & 100305 & 1995 & 12.01 & 0.29 & 1159.00 & 9360.31 & 1.04 & 0.78 & 0.81 \\
47406 & 210770 & 1995 & 297.00 & 0.38 & 43875.00 & 438338.56 & 0.68 & 1.48 & 1.00 \\
15091 & 101956 & 1995 & 900.56 & 0.34 & 67100.00 & 658011.91 & 1.34 & 0.73 & 0.98 \\
15109 & 101958 & 1995 & 109.31 & 0.36 & 7915.00 & 81596.21 & 1.38 & 0.75 & 1.03 \\
6545 & 100890 & 1995 & 442.80 & 0.35 & 41401.00 & 414603.41 & 1.07 & 0.94 & 1.00 \\
6536 & 100889 & 1995 & 40.80 & 0.26 & 3905.00 & 39456.43 & 1.04 & 0.97 & 1.01 \\
27308 & 105268 & 1995 & 50.35 & 0.28 & 4821.00 & 46629.75 & 1.04 & 0.93 & 0.97 \\
23364 & 103172 & 1995 & 39.64 & 0.30 & 3317.00 & 31702.91 & 1.19 & 0.80 & 0.96 \\
27270 & 105260 & 1995 & 122.80 & 0.33 & 11224.00 & 100860.74 & 1.09 & 0.82 & 0.90 \\
37706 & 107004 & 1995 & 10.88 & 0.35 & 1258.00 & 12323.97 & 0.86 & 1.13 & 0.98 \\
23346 & 103166 & 1995 & 4.67 & -0.02 & 573.00 & 5482.48 & 0.82 & 1.17 & 0.96 \\
6500 & 100878 & 1995 & 1227.93 & 0.39 & 90035.00 & 1214705.46 & 1.36 & 0.99 & 1.35 \\
15170 & 101964 & 1995 & 70.54 & 0.49 & 7054.00 & 70778.48 & 1.00 & 1.00 & 1.00 \\
24720 & 103376 & 1995 & 4849.04 & 0.20 & 520690.00 & 3955378.95 & 0.93 & 0.82 & 0.76 \\
652 & 100087 & 1995 & 4630.02 & 0.30 & 466130.00 & 3938477.96 & 0.99 & 0.85 & 0.84 \\
27214 & 105256 & 1995 & 11.64 & 0.27 & 1054.00 & 8534.30 & 1.10 & 0.73 & 0.81 \\
15212 & 101968 & 1995 & 84.50 & 0.33 & 7553.00 & 73520.05 & 1.12 & 0.87 & 0.97 \\
47331 & 210203 & 1995 & 989.33 & 0.38 & 92575.00 & 887667.94 & 1.07 & 0.90 & 0.96 \\
15141 & 101963 & 1995 & 734.10 & 0.28 & 59499.00 & 606612.74 & 1.23 & 0.83 & 1.02 \\
27461 & 105280 & 1995 & 50.78 & 0.23 & 4793.00 & 50073.85 & 1.06 & 0.99 & 1.04 \\
23408 & 103175 & 1995 & 1162.81 & 0.36 & 115978.00 & 1103928.97 & 1.00 & 0.95 & 0.95 \\
15004 & 101933 & 1995 & 183.13 & 0.38 & 17407.00 & 167548.34 & 1.05 & 0.91 & 0.96 \\
6723 & 100925 & 1995 & 28.00 & 0.42 & 2223.00 & 18647.67 & 1.26 & 0.67 & 0.84 \\
2216 & 100296 & 1995 & 13.29 & 0.29 & 970.00 & 10065.07 & 1.37 & 0.76 & 1.04 \\
27455 & 105279 & 1995 & 38.16 & 0.28 & 3809.00 & 36469.55 & 1.00 & 0.96 & 0.96 \\
24680 & 103375 & 1995 & 603.23 & 0.38 & 56649.00 & 496330.57 & 1.06 & 0.82 & 0.88 \\
27425 & 105278 & 1995 & 3.59 & -0.03 & 340.00 & 3483.26 & 1.05 & 0.97 & 1.02 \\
27396 & 105276 & 1995 & 46.58 & 0.40 & 2478.00 & 22127.40 & 1.88 & 0.48 & 0.89 \\
6696 & 100913 & 1995 & 23.42 & 0.09 & 2328.00 & 23282.30 & 1.01 & 0.99 & 1.00 \\
14870 & 101919 & 1995 & 195.39 & 0.46 & 18252.00 & 177400.47 & 1.07 & 0.91 & 0.97 \\
23473 & 103179 & 1995 & 332.46 & 0.51 & 24238.00 & 234164.65 & 1.37 & 0.70 & 0.97 \\
47503 & 212408 & 1995 & 715.49 & 0.27 & 56519.00 & 571383.63 & 1.27 & 0.80 & 1.01 \\
626 & 100085 & 1995 & 6250.74 & 0.34 & 384783.00 & 3336825.66 & 1.62 & 0.53 & 0.87 \\
6682 & 100910 & 1995 & 63.00 & 0.33 & 6340.00 & 58386.25 & 0.99 & 0.93 & 0.92 \\
14907 & 101922 & 1995 & 39.20 & 0.32 & 3277.00 & 32579.53 & 1.20 & 0.83 & 0.99 \\
6665 & 100908 & 1995 & 123.22 & 0.26 & 12450.00 & 113751.28 & 0.99 & 0.92 & 0.91 \\
2249 & 100302 & 1995 & 4.19 & 0.07 & 429.00 & 3948.41 & 0.98 & 0.94 & 0.92 \\
27366 & 105275 & 1995 & 61.92 & 0.47 & 3697.00 & 33715.48 & 1.67 & 0.54 & 0.91 \\
27333 & 105269 & 1995 & 74.58 & 0.19 & 7467.00 & 73942.53 & 1.00 & 0.99 & 0.99 \\
14948 & 101925 & 1995 & 227.15 & 0.26 & 20812.00 & 217479.37 & 1.09 & 0.96 & 1.04 \\
2255 & 100303 & 1995 & 86.70 & 0.33 & 9005.00 & 72155.42 & 0.96 & 0.83 & 0.80 \\
23442 & 103177 & 1995 & 212.48 & 0.31 & 20609.00 & 176077.82 & 1.03 & 0.83 & 0.85 \\
14980 & 101926 & 1995 & 136.59 & 0.36 & 13141.00 & 132033.53 & 1.04 & 0.97 & 1.00 \\
6632 & 100906 & 1995 & 1051.81 & 0.36 & 100213.00 & 902140.75 & 1.05 & 0.86 & 0.90 \\
14993 & 101930 & 1995 & 673.73 & 0.32 & 62833.00 & 625644.68 & 1.07 & 0.93 & 1.00 \\
14905 & 101921 & 1995 & 42.74 & 0.29 & 3784.00 & 38856.68 & 1.13 & 0.91 & 1.03 \\
27659 & 105309 & 1995 & 382.00 & 0.32 & 50528.00 & 413788.57 & 0.76 & 1.08 & 0.82 \\
27685 & 105310 & 1995 & 10.07 & 0.36 & 1020.00 & 8923.74 & 0.99 & 0.89 & 0.87 \\
28010 & 105369 & 1995 & 149.43 & 0.01 & 14276.00 & 137668.97 & 1.05 & 0.92 & 0.96 \\
24416 & 103326 & 1995 & 949.61 & 0.32 & 78239.00 & 791134.18 & 1.21 & 0.83 & 1.01 \\
28007 & 105366 & 1995 & 23.49 & 0.34 & 1760.00 & 20690.45 & 1.33 & 0.88 & 1.18 \\
27978 & 105364 & 1995 & 23.90 & 0.36 & 1609.00 & 15103.26 & 1.49 & 0.63 & 0.94 \\
13862 & 101781 & 1995 & 809.85 & 0.30 & 80985.00 & 708334.36 & 1.00 & 0.87 & 0.87 \\
24400 & 103319 & 1995 & 154.85 & 0.38 & 14435.00 & 153567.76 & 1.07 & 0.99 & 1.06 \\
7235 & 101015 & 1995 & 745.00 & 0.38 & 55774.00 & 486693.57 & 1.34 & 0.65 & 0.87 \\
13879 & 101785 & 1995 & 948.46 & 0.42 & 94845.00 & 908061.77 & 1.00 & 0.96 & 0.96 \\
48237 & 240056 & 1995 & 129.30 & 0.19 & 12275.00 & 103961.29 & 1.05 & 0.80 & 0.85 \\
48207 & 240051 & 1995 & 132.16 & 0.36 & 13435.00 & 126053.65 & 0.98 & 0.95 & 0.94 \\
13912 & 101787 & 1995 & 462.23 & 0.24 & 28867.00 & 294287.92 & 1.60 & 0.64 & 1.02 \\
13931 & 101788 & 1995 & 191.65 & 0.32 & 18304.00 & 198474.17 & 1.05 & 1.04 & 1.08 \\
28039 & 105370 & 1995 & 49.71 & -0.00 & 4185.00 & 39974.55 & 1.19 & 0.80 & 0.96 \\
13950 & 101789 & 1995 & 414.38 & 0.37 & 32864.00 & 347535.16 & 1.26 & 0.84 & 1.06 \\
48175 & 240040 & 1995 & 322.50 & 0.29 & 27000.00 & 270303.37 & 1.19 & 0.84 & 1.00 \\
13973 & 101794 & 1995 & 220.72 & 0.36 & 16628.00 & 181957.10 & 1.33 & 0.82 & 1.09 \\
24338 & 103315 & 1995 & 35.16 & 0.23 & 3344.00 & 31827.47 & 1.05 & 0.91 & 0.95 \\
24312 & 103308 & 1995 & 4081.86 & 0.27 & 408148.00 & 3576232.60 & 1.00 & 0.88 & 0.88 \\
39889 & 107928 & 1995 & 48.93 & -0.00 & 4035.00 & 37059.24 & 1.21 & 0.76 & 0.92 \\
27946 & 105358 & 1995 & 41.04 & 0.48 & 2863.00 & 24777.80 & 1.43 & 0.60 & 0.87 \\
27941 & 105354 & 1995 & 35.43 & 0.34 & 2220.00 & 20215.70 & 1.60 & 0.57 & 0.91 \\
27927 & 105353 & 1995 & 8.00 & 0.27 & 623.00 & 5629.37 & 1.28 & 0.70 & 0.90 \\
27921 & 105352 & 1995 & 139.10 & 0.22 & 12876.00 & 137644.32 & 1.08 & 0.99 & 1.07 \\
7160 & 101000 & 1995 & 665.42 & 0.36 & 66579.00 & 582271.64 & 1.00 & 0.88 & 0.87 \\
14017 & 101800 & 1995 & 520.83 & 0.29 & 49161.00 & 467288.96 & 1.06 & 0.90 & 0.95 \\
27895 & 105343 & 1995 & 26.88 & 0.02 & 2733.00 & 25353.52 & 0.98 & 0.94 & 0.93 \\
24446 & 103327 & 1995 & 1243.23 & 0.34 & 107004.00 & 1137013.75 & 1.16 & 0.91 & 1.06 \\
1925 & 100251 & 1995 & 161.38 & 0.32 & 14285.00 & 145570.29 & 1.13 & 0.90 & 1.02 \\
13612 & 101748 & 1995 & 10.59 & 0.28 & 1034.00 & 10012.76 & 1.02 & 0.95 & 0.97 \\
24568 & 103369 & 1995 & 105.82 & 0.30 & 11582.00 & 117869.37 & 0.91 & 1.11 & 1.02 \\
28133 & 105384 & 1995 & 44.39 & 0.04 & 3584.00 & 38215.36 & 1.24 & 0.86 & 1.07 \\
28119 & 105383 & 1995 & 34.92 & 0.31 & 3028.00 & 27736.78 & 1.15 & 0.79 & 0.92 \\
1882 & 100247 & 1995 & 654.78 & 0.25 & 65469.00 & 585735.24 & 1.00 & 0.89 & 0.89 \\
1853 & 100245 & 1995 & 285.72 & 0.21 & 28565.00 & 252949.73 & 1.00 & 0.89 & 0.89 \\
28090 & 105382 & 1995 & 18.83 & 0.24 & 1899.00 & 17791.47 & 0.99 & 0.94 & 0.94 \\
28088 & 105381 & 1995 & 24.73 & 0.34 & 2004.00 & 21424.93 & 1.23 & 0.87 & 1.07 \\
7343 & 101023 & 1995 & 10492.10 & 0.32 & 922705.00 & 8351181.39 & 1.14 & 0.80 & 0.91 \\
24542 & 103347 & 1995 & 2.70 & 0.14 & 380.00 & 3447.59 & 0.71 & 1.28 & 0.91 \\
13835 & 101769 & 1995 & 2546.44 & 0.31 & 254644.00 & 2043644.12 & 1.00 & 0.80 & 0.80 \\
48342 & 240065 & 1995 & 249.31 & 0.30 & 18195.00 & 206213.57 & 1.37 & 0.83 & 1.13 \\
24530 & 103339 & 1995 & 594.74 & 0.43 & 57493.00 & 571774.30 & 1.03 & 0.96 & 0.99 \\
28073 & 105375 & 1995 & 17.82 & 0.02 & 1712.00 & 16426.55 & 1.04 & 0.92 & 0.96 \\
13743 & 101762 & 1995 & 3159.82 & 0.37 & 300825.00 & 2962208.78 & 1.05 & 0.94 & 0.98 \\
24498 & 103329 & 1995 & 378.73 & 0.25 & 34031.00 & 338094.45 & 1.11 & 0.89 & 0.99 \\
24468 & 103328 & 1995 & 367.17 & 0.34 & 31503.00 & 325295.16 & 1.17 & 0.89 & 1.03 \\
55027 & 400049 & 1995 & 184.78 & 0.16 & 19352.00 & 161966.34 & 0.95 & 0.88 & 0.84 \\
13799 & 101764 & 1995 & 180.26 & 0.31 & 18027.00 & 147645.30 & 1.00 & 0.82 & 0.82 \\
1921 & 100250 & 1995 & 90.03 & 0.26 & 9007.00 & 84501.58 & 1.00 & 0.94 & 0.94 \\
7273 & 101018 & 1995 & 7801.90 & 0.28 & 647833.00 & 5938583.17 & 1.20 & 0.76 & 0.92 \\
28068 & 105372 & 1995 & 25.72 & 0.25 & 2527.00 & 23050.28 & 1.02 & 0.90 & 0.91 \\
23854 & 103224 & 1995 & 167.46 & 0.50 & 14913.00 & 139746.04 & 1.12 & 0.83 & 0.94 \\
7141 & 100998 & 1995 & 134.17 & 0.26 & 13417.00 & 121909.91 & 1.00 & 0.91 & 0.91 \\
24260 & 103301 & 1995 & 956.10 & 0.30 & 73670.00 & 774817.61 & 1.30 & 0.81 & 1.05 \\
23998 & 103253 & 1995 & 97.70 & 0.19 & 8716.00 & 83285.74 & 1.12 & 0.85 & 0.96 \\
23977 & 103252 & 1995 & 187.91 & 0.32 & 16208.00 & 175057.13 & 1.16 & 0.93 & 1.08 \\
7041 & 100992 & 1995 & 248.35 & 0.40 & 25224.00 & 240314.22 & 0.98 & 0.97 & 0.95 \\
23956 & 103251 & 1995 & 180.03 & 0.26 & 16325.00 & 168150.34 & 1.10 & 0.93 & 1.03 \\
14271 & 101842 & 1995 & 460.36 & 0.37 & 40015.00 & 412019.39 & 1.15 & 0.89 & 1.03 \\
14312 & 101849 & 1995 & 30.91 & 0.24 & 3103.00 & 25751.46 & 1.00 & 0.83 & 0.83 \\
14324 & 101850 & 1995 & 71.75 & 0.23 & 6935.00 & 66717.81 & 1.03 & 0.93 & 0.96 \\
23926 & 103242 & 1995 & 13.00 & 0.30 & 1277.00 & 12090.80 & 1.02 & 0.93 & 0.95 \\
47915 & 222809 & 1995 & 247.72 & 0.32 & 24581.00 & 248433.42 & 1.01 & 1.00 & 1.01 \\
27705 & 105317 & 1995 & 5.55 & 0.33 & 451.00 & 4157.44 & 1.23 & 0.75 & 0.92 \\
14238 & 101835 & 1995 & 220.56 & 0.37 & 19899.00 & 202479.55 & 1.11 & 0.92 & 1.02 \\
23906 & 103232 & 1995 & 7.45 & 0.34 & 724.00 & 7019.55 & 1.03 & 0.94 & 0.97 \\
14349 & 101851 & 1995 & 744.62 & 0.23 & 75439.00 & 652206.84 & 0.99 & 0.88 & 0.86 \\
6986 & 100981 & 1995 & 26.39 & 0.02 & 2582.00 & 21441.15 & 1.02 & 0.81 & 0.83 \\
6981 & 100980 & 1995 & 48.30 & 0.28 & 5114.00 & 44596.53 & 0.94 & 0.92 & 0.87 \\
14383 & 101853 & 1995 & 248.60 & 0.30 & 23984.00 & 208788.07 & 1.04 & 0.84 & 0.87 \\
23890 & 103228 & 1995 & 47.81 & 0.35 & 4437.00 & 41847.88 & 1.08 & 0.88 & 0.94 \\
27702 & 105313 & 1995 & 6.64 & 0.22 & 684.00 & 5402.15 & 0.97 & 0.81 & 0.79 \\
6974 & 100978 & 1995 & 56.73 & 0.15 & 5475.00 & 51780.81 & 1.04 & 0.91 & 0.95 \\
6972 & 100977 & 1995 & 15.86 & 0.40 & 1433.00 & 13656.16 & 1.11 & 0.86 & 0.95 \\
6966 & 100974 & 1995 & 14.24 & 0.44 & 1350.00 & 13538.31 & 1.05 & 0.95 & 1.00 \\
14399 & 101854 & 1995 & 1388.72 & 0.41 & 131553.00 & 1228553.28 & 1.06 & 0.88 & 0.93 \\
23872 & 103226 & 1995 & 84.61 & 0.31 & 8315.00 & 83137.97 & 1.02 & 0.98 & 1.00 \\
540 & 100075 & 1995 & 2304.11 & 0.42 & 224555.00 & 1944421.82 & 1.03 & 0.84 & 0.87 \\
27693 & 105311 & 1995 & 20.75 & 0.55 & 2101.00 & 21095.74 & 0.99 & 1.02 & 1.00 \\
24028 & 103255 & 1995 & 121.93 & 0.25 & 10576.00 & 115565.39 & 1.15 & 0.95 & 1.09 \\
14192 & 101820 & 1995 & 213.77 & 0.28 & 20086.00 & 197161.61 & 1.06 & 0.92 & 0.98 \\
1978 & 100278 & 1995 & 2.32 & 0.34 & 226.00 & 2001.72 & 1.03 & 0.86 & 0.89 \\
48064 & 235413 & 1995 & 20.44 & 0.52 & 1891.00 & 17770.23 & 1.08 & 0.87 & 0.94 \\
14092 & 101802 & 1995 & 222.49 & 0.32 & 19946.00 & 206216.57 & 1.12 & 0.93 & 1.03 \\
24253 & 103300 & 1995 & 45.11 & -0.03 & 7405.00 & 37779.74 & 0.61 & 0.84 & 0.51 \\
27870 & 105336 & 1995 & 20.83 & 0.33 & 1536.00 & 16014.96 & 1.36 & 0.77 & 1.04 \\
24229 & 103299 & 1995 & 847.73 & 0.42 & 66630.00 & 772264.02 & 1.27 & 0.91 & 1.16 \\
27848 & 105333 & 1995 & 5.88 & 0.37 & 539.00 & 4959.61 & 1.09 & 0.84 & 0.92 \\
2002 & 100280 & 1995 & 42.65 & 0.22 & 4327.00 & 40963.65 & 0.99 & 0.96 & 0.95 \\
7109 & 100997 & 1995 & 54.64 & 0.29 & 5712.00 & 56773.64 & 0.96 & 1.04 & 0.99 \\
14110 & 101804 & 1995 & 298.87 & 0.36 & 27603.00 & 280472.44 & 1.08 & 0.94 & 1.02 \\
24194 & 103296 & 1995 & 1454.16 & 0.25 & 121225.00 & 1441709.53 & 1.20 & 0.99 & 1.19 \\
519 & 100072 & 1995 & 7377.44 & 0.35 & 713927.00 & 6150861.49 & 1.03 & 0.83 & 0.86 \\
14223 & 101834 & 1995 & 13.76 & 0.38 & 1367.00 & 13355.83 & 1.01 & 0.97 & 0.98 \\
27819 & 105332 & 1995 & 36.60 & 0.45 & 3697.00 & 36512.31 & 0.99 & 1.00 & 0.99 \\
14127 & 101805 & 1995 & 1988.13 & 0.36 & 194222.00 & 1834682.96 & 1.02 & 0.92 & 0.94 \\
14147 & 101809 & 1995 & 44.91 & 0.24 & 5196.00 & 47293.52 & 0.86 & 1.05 & 0.91 \\
27790 & 105331 & 1995 & 1.74 & 0.20 & 139.00 & 1239.06 & 1.25 & 0.71 & 0.89 \\
7078 & 100996 & 1995 & 763.33 & 0.36 & 76333.00 & 630338.38 & 1.00 & 0.83 & 0.83 \\
24109 & 103267 & 1995 & 190.18 & -0.01 & 19772.00 & 191306.41 & 0.96 & 1.01 & 0.97 \\
27784 & 105327 & 1995 & 12.80 & 0.31 & 927.00 & 10458.35 & 1.38 & 0.82 & 1.13 \\
24093 & 103266 & 1995 & 132.66 & 0.60 & 10210.00 & 105989.14 & 1.30 & 0.80 & 1.04 \\
27776 & 105326 & 1995 & 24.30 & 0.26 & 2103.00 & 23328.54 & 1.16 & 0.96 & 1.11 \\
24081 & 103264 & 1995 & 757.98 & 0.25 & 78109.00 & 751618.62 & 0.97 & 0.99 & 0.96 \\
27766 & 105322 & 1995 & 19.44 & 0.31 & 1944.00 & 18954.14 & 1.00 & 0.98 & 0.98 \\
14160 & 101819 & 1995 & 140.70 & 0.35 & 13124.00 & 123467.40 & 1.07 & 0.88 & 0.94 \\
24050 & 103259 & 1995 & 652.49 & 0.33 & 55331.00 & 505613.83 & 1.18 & 0.77 & 0.91 \\
6451 & 100875 & 1995 & 174.80 & 0.20 & 17730.00 & 169621.12 & 0.99 & 0.97 & 0.96 \\
23515 & 103183 & 1995 & 375.14 & 0.36 & 36034.00 & 353820.66 & 1.04 & 0.94 & 0.98 \\
23069 & 103110 & 1995 & 67.64 & 0.30 & 6630.00 & 64535.87 & 1.02 & 0.95 & 0.97 \\
15474 & 101998 & 1995 & 423.74 & 0.21 & 43037.00 & 399919.48 & 0.98 & 0.94 & 0.93 \\
26923 & 103628 & 1995 & 650.47 & 0.14 & 61240.00 & 606741.54 & 1.06 & 0.93 & 0.99 \\
24796 & 103380 & 1995 & 3982.84 & 0.23 & 403377.00 & 3347017.77 & 0.99 & 0.84 & 0.83 \\
16021 & 102073 & 1995 & 6577.03 & 0.32 & 657703.00 & 5340727.64 & 1.00 & 0.81 & 0.81 \\
2472 & 100333 & 1995 & 122.85 & 0.32 & 12266.00 & 110615.05 & 1.00 & 0.90 & 0.90 \\
6048 & 100821 & 1995 & 29.53 & 0.40 & 2293.00 & 25296.75 & 1.29 & 0.86 & 1.10 \\
23164 & 103138 & 1995 & 806.12 & 0.37 & 70752.00 & 699541.24 & 1.14 & 0.87 & 0.99 \\
27092 & 103652 & 1995 & 467.75 & 0.24 & 46779.00 & 401192.11 & 1.00 & 0.86 & 0.86 \\
15524 & 102000 & 1995 & 635.46 & 0.31 & 64072.00 & 616136.26 & 0.99 & 0.97 & 0.96 \\
74556 & 601136 & 1995 & 24.98 & 0.27 & 2501.00 & 23656.22 & 1.00 & 0.95 & 0.95 \\
6253 & 100833 & 1995 & 419.27 & 0.36 & 41907.00 & 398741.76 & 1.00 & 0.95 & 0.95 \\
704 & 100092 & 1995 & 60.40 & 0.46 & 5729.00 & 57295.82 & 1.05 & 0.95 & 1.00 \\
26957 & 103638 & 1995 & 133.60 & 0.33 & 13320.00 & 122828.77 & 1.00 & 0.92 & 0.92 \\
27087 & 103649 & 1995 & 3.58 & 0.26 & 392.00 & 2940.31 & 0.91 & 0.82 & 0.75 \\
22943 & 103090 & 1995 & 643.64 & 0.33 & 68769.00 & 721906.73 & 0.94 & 1.12 & 1.05 \\
23182 & 103144 & 1995 & 176.43 & 0.30 & 14349.00 & 159804.18 & 1.23 & 0.91 & 1.11 \\
39762 & 107868 & 1995 & 36.19 & 0.03 & 3164.00 & 30828.58 & 1.14 & 0.85 & 0.97 \\
27124 & 105243 & 1995 & 19.58 & 0.09 & 1888.00 & 19196.09 & 1.04 & 0.98 & 1.02 \\
15392 & 101989 & 1995 & 108.27 & 0.34 & 10901.00 & 104287.28 & 0.99 & 0.96 & 0.96 \\
23277 & 103154 & 1995 & 438.70 & 0.31 & 44000.00 & 373025.65 & 1.00 & 0.85 & 0.85 \\
768 & 100096 & 1995 & 12.90 & 0.40 & 1266.00 & 12664.91 & 1.02 & 0.98 & 1.00 \\
15420 & 101990 & 1995 & 204.73 & 0.25 & 25972.00 & 204243.55 & 0.79 & 1.00 & 0.79 \\
26909 & 103621 & 1995 & 58.21 & 0.24 & 5595.00 & 55699.68 & 1.04 & 0.96 & 1.00 \\
23245 & 103152 & 1995 & 1065.09 & 0.31 & 106865.00 & 998753.03 & 1.00 & 0.94 & 0.93 \\
27064 & 103647 & 1995 & 7.55 & 0.28 & 1094.00 & 10514.56 & 0.69 & 1.39 & 0.96 \\
6017 & 100820 & 1995 & 26.81 & 0.37 & 2135.00 & 23289.87 & 1.26 & 0.87 & 1.09 \\
6326 & 100849 & 1995 & 73.98 & 0.21 & 7394.00 & 67375.57 & 1.00 & 0.91 & 0.91 \\
680 & 100090 & 1995 & 172.70 & 0.38 & 15587.00 & 155875.03 & 1.11 & 0.90 & 1.00 \\
2361 & 100320 & 1995 & 40.17 & 0.20 & 4044.00 & 38776.51 & 0.99 & 0.97 & 0.96 \\
15454 & 101992 & 1995 & 1620.89 & 0.33 & 159763.00 & 1526539.34 & 1.01 & 0.94 & 0.96 \\
15468 & 101996 & 1995 & 34.01 & 0.34 & 2556.00 & 25254.13 & 1.33 & 0.74 & 0.99 \\
6297 & 100847 & 1995 & 2.86 & 0.18 & 280.00 & 2240.99 & 1.02 & 0.78 & 0.80 \\
15448 & 101991 & 1995 & 68.79 & 0.33 & 6014.00 & 58675.17 & 1.14 & 0.85 & 0.98 \\
23137 & 103134 & 1995 & 327.20 & 0.36 & 23626.00 & 325403.46 & 1.38 & 0.99 & 1.38 \\
26967 & 103640 & 1995 & 174.10 & 0.34 & 13847.00 & 154251.27 & 1.26 & 0.89 & 1.11 \\
27034 & 103645 & 1995 & 566.72 & 0.30 & 59647.00 & 542614.60 & 0.95 & 0.96 & 0.91 \\
15247 & 101972 & 1995 & 117.70 & 0.34 & 10182.00 & 112293.26 & 1.16 & 0.95 & 1.10 \\
15694 & 102015 & 1995 & 361.83 & 0.33 & 37814.00 & 362852.63 & 0.96 & 1.00 & 0.96 \\
22995 & 103101 & 1995 & 208.00 & 0.27 & 19264.00 & 185088.17 & 1.08 & 0.89 & 0.96 \\
26982 & 103643 & 1995 & 49.21 & 0.29 & 4913.00 & 49082.28 & 1.00 & 1.00 & 1.00 \\
6110 & 100823 & 1995 & 32.56 & 0.25 & 2598.00 & 27148.47 & 1.25 & 0.83 & 1.04 \\
2430 & 100324 & 1995 & 56.79 & 0.20 & 5653.00 & 49229.12 & 1.00 & 0.87 & 0.87 \\
26975 & 103642 & 1995 & 26.11 & 0.21 & 2599.00 & 25871.31 & 1.00 & 0.99 & 1.00 \\
6157 & 100827 & 1995 & 7.10 & 0.36 & 640.00 & 5902.19 & 1.11 & 0.83 & 0.92 \\
15719 & 102016 & 1995 & 1138.91 & 0.28 & 119482.00 & 1013517.17 & 0.95 & 0.89 & 0.85 \\
23026 & 103103 & 1995 & 111.90 & 0.27 & 10829.00 & 111147.49 & 1.03 & 0.99 & 1.03 \\
15799 & 102026 & 1995 & 26.57 & -0.05 & 2753.00 & 25013.87 & 0.96 & 0.94 & 0.91 \\
6144 & 100825 & 1995 & 7.53 & 0.43 & 669.00 & 6866.50 & 1.13 & 0.91 & 1.03 \\
2436 & 100330 & 1995 & 353.10 & 0.35 & 28997.00 & 272528.05 & 1.22 & 0.77 & 0.94 \\
6140 & 100824 & 1995 & 6.43 & 0.25 & 619.00 & 6121.09 & 1.04 & 0.95 & 0.99 \\
15753 & 102017 & 1995 & 2239.75 & 0.39 & 191856.00 & 1985173.08 & 1.17 & 0.89 & 1.03 \\
15784 & 102018 & 1995 & 531.62 & 0.40 & 51656.00 & 487251.60 & 1.03 & 0.92 & 0.94 \\
47182 & 200342 & 1995 & 5745.80 & 0.27 & 446910.00 & 4150058.01 & 1.29 & 0.72 & 0.93 \\
15674 & 102013 & 1995 & 282.55 & 0.44 & 27362.00 & 234445.74 & 1.03 & 0.83 & 0.86 \\
6192 & 100829 & 1995 & 608.21 & 0.30 & 49173.00 & 480720.10 & 1.24 & 0.79 & 0.98 \\
2393 & 100322 & 1995 & 308.30 & 0.32 & 29675.00 & 278939.99 & 1.04 & 0.90 & 0.94 \\
6226 & 100831 & 1995 & 265.04 & 0.21 & 25315.00 & 235955.52 & 1.05 & 0.89 & 0.93 \\
22958 & 103099 & 1995 & 178.00 & 0.36 & 16079.00 & 133864.98 & 1.11 & 0.75 & 0.83 \\
15939 & 102061 & 1995 & 1.20 & 0.24 & 71.00 & 598.23 & 1.69 & 0.50 & 0.84 \\
15568 & 102005 & 1995 & 725.93 & 0.34 & 72814.00 & 702043.05 & 1.00 & 0.97 & 0.96 \\
15583 & 102007 & 1995 & 831.50 & 0.37 & 71164.00 & 687372.85 & 1.17 & 0.83 & 0.97 \\
15909 & 102059 & 1995 & 291.27 & 0.40 & 26669.00 & 242321.88 & 1.09 & 0.83 & 0.91 \\
6079 & 100822 & 1995 & 21.97 & 0.28 & 2188.00 & 22421.18 & 1.00 & 1.02 & 1.02 \\
2413 & 100323 & 1995 & 73.54 & 0.32 & 6740.00 & 73064.96 & 1.09 & 0.99 & 1.08 \\
23131 & 103131 & 1995 & 125.59 & 0.32 & 11846.00 & 121831.49 & 1.06 & 0.97 & 1.03 \\
27008 & 103644 & 1995 & 217.62 & 0.30 & 21759.00 & 216025.61 & 1.00 & 0.99 & 0.99 \\
15634 & 102010 & 1995 & 3459.79 & 0.41 & 331252.00 & 3152511.77 & 1.04 & 0.91 & 0.95 \\
23127 & 103130 & 1995 & 75.57 & 0.31 & 7997.00 & 70002.63 & 0.94 & 0.93 & 0.88 \\
15886 & 102052 & 1995 & 245.57 & 0.14 & 24080.00 & 217640.47 & 1.02 & 0.89 & 0.90 \\
22977 & 103100 & 1995 & 318.50 & 0.34 & 29595.00 & 299634.71 & 1.08 & 0.94 & 1.01 \\
738 & 100093 & 1995 & 494.10 & 0.49 & 46843.00 & 463979.48 & 1.05 & 0.94 & 0.99 \\
23292 & 103158 & 1995 & 1002.60 & 0.32 & 100270.00 & 1023863.93 & 1.00 & 1.02 & 1.02 \\
15493 & 101999 & 1995 & 1402.36 & 0.39 & 120237.00 & 1110613.78 & 1.17 & 0.79 & 0.92 \\
22918 & 103085 & 1995 & 88.62 & 0.21 & 8862.00 & 78957.11 & 1.00 & 0.89 & 0.89 \\
5978 & 100817 & 1995 & 6.52 & 0.36 & 631.00 & 5900.61 & 1.03 & 0.91 & 0.94 \\
26839 & 103609 & 1995 & 18.17 & 0.29 & 1816.00 & 17480.84 & 1.00 & 0.96 & 0.96 \\
27196 & 105252 & 1995 & 20.71 & 0.52 & 2198.00 & 19870.05 & 0.94 & 0.96 & 0.90 \\
2329 & 100319 & 1995 & 201.89 & 0.16 & 19849.00 & 195583.17 & 1.02 & 0.97 & 0.99 \\
22875 & 103074 & 1995 & 90.03 & 0.26 & 7920.00 & 78083.80 & 1.14 & 0.87 & 0.99 \\
1762 & 100228 & 1995 & 141.06 & 0.29 & 12501.00 & 116912.70 & 1.13 & 0.83 & 0.94 \\
15278 & 101977 & 1995 & 45.37 & 0.44 & 4337.00 & 40221.71 & 1.05 & 0.89 & 0.93 \\
24836 & 103381 & 1995 & 14551.25 & 0.38 & 1429720.00 & 12511445.79 & 1.02 & 0.86 & 0.88 \\
26816 & 103608 & 1995 & 62.72 & 0.36 & 6272.00 & 60986.17 & 1.00 & 0.97 & 0.97 \\
24757 & 103377 & 1995 & 1026.05 & 0.27 & 100623.00 & 839665.08 & 1.02 & 0.82 & 0.83 \\
6425 & 100868 & 1995 & 114.02 & 0.30 & 9881.00 & 102590.79 & 1.15 & 0.90 & 1.04 \\
15283 & 101978 & 1995 & 67.41 & 0.33 & 6712.00 & 70383.55 & 1.00 & 1.04 & 1.05 \\
2309 & 100315 & 1995 & 271.12 & 0.31 & 24728.00 & 281642.21 & 1.10 & 1.04 & 1.14 \\
15332 & 101987 & 1995 & 377.81 & 0.36 & 37355.00 & 373214.55 & 1.01 & 0.99 & 1.00 \\
27205 & 105253 & 1995 & 15.45 & 0.28 & 1616.00 & 12384.69 & 0.96 & 0.80 & 0.77 \\
26789 & 103607 & 1995 & 547.04 & 0.27 & 54704.00 & 499088.95 & 1.00 & 0.91 & 0.91 \\
15292 & 101980 & 1995 & 26.04 & 0.26 & 2438.00 & 23566.27 & 1.07 & 0.90 & 0.97 \\
6404 & 100864 & 1995 & 520.86 & 0.26 & 52172.00 & 430868.02 & 1.00 & 0.83 & 0.83 \\
47247 & 200344 & 1995 & 1127.66 & 0.35 & 112766.00 & 1066310.89 & 1.00 & 0.95 & 0.95 \\
16072 & 102079 & 1995 & 739.14 & 0.21 & 70194.00 & 586075.12 & 1.05 & 0.79 & 0.83 \\
15296 & 101982 & 1995 & 273.16 & 0.33 & 27277.00 & 261837.65 & 1.00 & 0.96 & 0.96 \\
15362 & 101988 & 1995 & 132.83 & 0.34 & 10709.00 & 101483.55 & 1.24 & 0.76 & 0.95 \\
74623 & 601142 & 1995 & 252.13 & 0.37 & 34637.00 & 313549.50 & 0.73 & 1.24 & 0.91 \\
23304 & 103160 & 1995 & 61.50 & 0.16 & 6150.00 & 59443.55 & 1.00 & 0.97 & 0.97 \\
5937 & 100812 & 1995 & 408.61 & 0.28 & 40918.00 & 392096.92 & 1.00 & 0.96 & 0.96 \\
6374 & 100856 & 1995 & 140.91 & 0.27 & 14096.00 & 132442.10 & 1.00 & 0.94 & 0.94 \\
27186 & 105250 & 1995 & 3.26 & 0.21 & 273.00 & 2985.20 & 1.19 & 0.92 & 1.09 \\
1742 & 100227 & 1995 & 134.82 & 0.25 & 12175.00 & 121656.22 & 1.11 & 0.90 & 1.00 \\
22856 & 103073 & 1995 & 614.20 & 0.50 & 61415.00 & 518763.83 & 1.00 & 0.84 & 0.84 \\
26887 & 103620 & 1995 & 96.60 & 0.30 & 8797.00 & 82801.43 & 1.10 & 0.86 & 0.94 \\
6007 & 100818 & 1995 & 540.60 & 0.12 & 57208.00 & 576087.47 & 0.94 & 1.07 & 1.01 \\
31150 & 105864 & 1996 & 92.21 & 0.21 & 11934.00 & 112828.13 & 0.77 & 1.22 & 0.95 \\
21094 & 102829 & 1996 & 37.01 & 0.19 & 3695.00 & 35591.80 & 1.00 & 0.96 & 0.96 \\
20496 & 102760 & 1996 & 715.41 & 0.07 & 48371.00 & 433695.75 & 1.48 & 0.61 & 0.90 \\
23927 & 103242 & 1996 & 24.19 & 0.33 & 2408.00 & 23202.32 & 1.00 & 0.96 & 0.96 \\
22825 & 103067 & 1996 & 57.73 & 0.23 & 5782.00 & 55840.10 & 1.00 & 0.97 & 0.97 \\
9901 & 101211 & 1996 & 114.20 & 0.28 & 12850.00 & 98758.73 & 0.89 & 0.86 & 0.77 \\
22959 & 103099 & 1996 & 305.80 & 0.22 & 27498.00 & 220647.13 & 1.11 & 0.72 & 0.80 \\
24110 & 103267 & 1996 & 232.18 & 0.21 & 23130.00 & 239680.01 & 1.00 & 1.03 & 1.04 \\
24569 & 103369 & 1996 & 90.45 & 0.19 & 9616.00 & 97228.37 & 0.94 & 1.07 & 1.01 \\
21062 & 102827 & 1996 & 166.42 & 0.20 & 16638.00 & 161050.62 & 1.00 & 0.97 & 0.97 \\
20555 & 102767 & 1996 & 1489.30 & 0.21 & 120678.00 & 1349166.10 & 1.23 & 0.91 & 1.12 \\
30318 & 105735 & 1996 & 73.32 & 0.16 & 6009.00 & 62703.90 & 1.22 & 0.86 & 1.04 \\
22596 & 103024 & 1996 & 192.73 & 0.27 & 17174.00 & 170193.65 & 1.12 & 0.88 & 0.99 \\
31185 & 105866 & 1996 & 347.69 & -0.02 & 28010.00 & 289844.99 & 1.24 & 0.83 & 1.03 \\
9457 & 101137 & 1996 & 9.40 & 0.17 & 909.00 & 8720.65 & 1.03 & 0.93 & 0.96 \\
30319 & 105737 & 1996 & 45.45 & 0.17 & 4560.00 & 45068.11 & 1.00 & 0.99 & 0.99 \\
32356 & 106014 & 1996 & 20.38 & 0.23 & 1932.00 & 20184.80 & 1.05 & 0.99 & 1.04 \\
32369 & 106018 & 1996 & 86.91 & 0.01 & 8120.00 & 72258.70 & 1.07 & 0.83 & 0.89 \\
9919 & 101212 & 1996 & 523.66 & 0.13 & 55126.00 & 539518.94 & 0.95 & 1.03 & 0.98 \\
30347 & 105740 & 1996 & 156.01 & 0.04 & 15601.00 & 131379.11 & 1.00 & 0.84 & 0.84 \\
30363 & 105742 & 1996 & 165.42 & 0.17 & 16528.00 & 150431.80 & 1.00 & 0.91 & 0.91 \\
32400 & 106023 & 1996 & 1.64 & 0.03 & 164.00 & 1375.40 & 1.00 & 0.84 & 0.84 \\
24531 & 103339 & 1996 & 677.12 & 0.13 & 66976.00 & 663992.94 & 1.01 & 0.98 & 0.99 \\
24160 & 103294 & 1996 & 304.52 & 0.25 & 26638.00 & 233189.92 & 1.14 & 0.77 & 0.88 \\
9984 & 101216 & 1996 & 65.65 & 0.18 & 6900.00 & 59678.30 & 0.95 & 0.91 & 0.86 \\
23027 & 103103 & 1996 & 154.40 & 0.25 & 15012.00 & 150125.07 & 1.03 & 0.97 & 1.00 \\
31146 & 105862 & 1996 & 35.44 & 0.03 & 3570.00 & 33096.03 & 0.99 & 0.93 & 0.93 \\
31134 & 105861 & 1996 & 56.42 & 0.24 & 5659.00 & 53615.76 & 1.00 & 0.95 & 0.95 \\
24094 & 103266 & 1996 & 303.70 & 0.14 & 27068.00 & 248163.85 & 1.12 & 0.82 & 0.92 \\
25349 & 103478 & 1996 & 111.01 & 0.25 & 11372.00 & 109336.83 & 0.98 & 0.98 & 0.96 \\
22565 & 103021 & 1996 & 39.59 & 0.21 & 3959.00 & 38775.98 & 1.00 & 0.98 & 0.98 \\
30465 & 105761 & 1996 & 119.46 & 0.25 & 11934.00 & 113469.54 & 1.00 & 0.95 & 0.95 \\
31490 & 105899 & 1996 & 5.34 & -0.02 & 534.00 & 4988.45 & 1.00 & 0.93 & 0.93 \\
9383 & 101133 & 1996 & 622.10 & 0.30 & 62224.00 & 597365.45 & 1.00 & 0.96 & 0.96 \\
24029 & 103255 & 1996 & 147.47 & 0.19 & 14856.00 & 143645.21 & 0.99 & 0.97 & 0.97 \\
23978 & 103252 & 1996 & 227.67 & 0.19 & 20986.00 & 211722.65 & 1.08 & 0.93 & 1.01 \\
21055 & 102825 & 1996 & 91.08 & 0.17 & 9130.00 & 86873.04 & 1.00 & 0.95 & 0.95 \\
22996 & 103101 & 1996 & 299.41 & 0.19 & 28985.00 & 282528.86 & 1.03 & 0.94 & 0.97 \\
24612 & 103372 & 1996 & 277.60 & 0.22 & 24325.00 & 224780.79 & 1.14 & 0.81 & 0.92 \\
23999 & 103253 & 1996 & 128.66 & 0.22 & 11289.00 & 119534.61 & 1.14 & 0.93 & 1.06 \\
20518 & 102761 & 1996 & 22413.48 & 0.21 & 2057236.00 & 19505182.22 & 1.09 & 0.87 & 0.95 \\
22857 & 103073 & 1996 & 787.38 & 0.25 & 78739.00 & 674248.32 & 1.00 & 0.86 & 0.86 \\
30437 & 105760 & 1996 & 102.55 & 0.21 & 10202.00 & 98732.27 & 1.01 & 0.96 & 0.97 \\
24557 & 103366 & 1996 & 94.22 & 0.23 & 10611.00 & 99884.85 & 0.89 & 1.06 & 0.94 \\
23957 & 103251 & 1996 & 220.74 & 0.21 & 20261.00 & 220929.99 & 1.09 & 1.00 & 1.09 \\
21012 & 102823 & 1996 & 37.55 & 0.30 & 3755.00 & 35647.32 & 1.00 & 0.95 & 0.95 \\
24082 & 103264 & 1996 & 773.17 & 0.19 & 76202.00 & 750812.38 & 1.01 & 0.97 & 0.99 \\
21085 & 102828 & 1996 & 98.34 & 0.24 & 9825.00 & 96016.91 & 1.00 & 0.98 & 0.98 \\
21020 & 102824 & 1996 & 85.25 & 0.21 & 8525.00 & 86553.78 & 1.00 & 1.02 & 1.02 \\
9480 & 101139 & 1996 & 42.38 & 0.05 & 4269.00 & 39080.21 & 0.99 & 0.92 & 0.92 \\
30369 & 105746 & 1996 & 38.18 & 0.28 & 3818.00 & 33848.18 & 1.00 & 0.89 & 0.89 \\
22978 & 103100 & 1996 & 394.70 & 0.22 & 38663.00 & 386653.53 & 1.02 & 0.98 & 1.00 \\
30383 & 105748 & 1996 & 55.50 & 0.17 & 5549.00 & 45142.78 & 1.00 & 0.81 & 0.81 \\
31462 & 105895 & 1996 & 59.59 & 0.03 & 3099.00 & 28622.47 & 1.92 & 0.48 & 0.92 \\
31079 & 105857 & 1996 & 10.81 & 0.03 & 826.00 & 8504.53 & 1.31 & 0.79 & 1.03 \\
25345 & 103477 & 1996 & 46.40 & 0.13 & 4536.00 & 44934.58 & 1.02 & 0.97 & 0.99 \\
21001 & 102821 & 1996 & 422.41 & 0.12 & 42521.00 & 403758.02 & 0.99 & 0.96 & 0.95 \\
30427 & 105758 & 1996 & 68.87 & 0.20 & 6941.00 & 67400.60 & 0.99 & 0.98 & 0.97 \\
24051 & 103259 & 1996 & 1548.22 & 0.24 & 125211.00 & 1275733.65 & 1.24 & 0.82 & 1.02 \\
24543 & 103347 & 1996 & 3.20 & 0.20 & 317.00 & 3162.37 & 1.01 & 0.99 & 1.00 \\
22790 & 103065 & 1996 & 375.32 & 0.22 & 37593.00 & 360038.38 & 1.00 & 0.96 & 0.96 \\
30417 & 105757 & 1996 & 1.92 & 0.18 & 192.00 & 1719.46 & 1.00 & 0.89 & 0.90 \\
24401 & 103319 & 1996 & 191.43 & 0.23 & 16965.00 & 184715.86 & 1.13 & 0.96 & 1.09 \\
10749 & 101322 & 1996 & 120.75 & 0.16 & 12131.00 & 116825.04 & 1.00 & 0.97 & 0.96 \\
24447 & 103327 & 1996 & 1662.76 & 0.20 & 150545.00 & 1659561.65 & 1.10 & 1.00 & 1.10 \\
24367 & 103318 & 1996 & 116.56 & 0.01 & 11659.00 & 113904.97 & 1.00 & 0.98 & 0.98 \\
10637 & 101302 & 1996 & 307.27 & 0.23 & 27200.00 & 300174.93 & 1.13 & 0.98 & 1.10 \\
39763 & 107868 & 1996 & 71.13 & 0.22 & 7113.00 & 65340.87 & 1.00 & 0.92 & 0.92 \\
40068 & 108021 & 1996 & 193.96 & 0.27 & 19301.00 & 167757.35 & 1.00 & 0.86 & 0.87 \\
24797 & 103380 & 1996 & 4060.48 & 0.19 & 403868.00 & 3976622.23 & 1.01 & 0.98 & 0.98 \\
24339 & 103315 & 1996 & 45.15 & 0.24 & 4515.00 & 39498.70 & 1.00 & 0.87 & 0.87 \\
32658 & 106047 & 1996 & 7.33 & -0.04 & 729.00 & 6977.91 & 1.01 & 0.95 & 0.96 \\
20651 & 102777 & 1996 & 2841.90 & 0.04 & 217837.00 & 2205723.02 & 1.30 & 0.78 & 1.01 \\
20803 & 102795 & 1996 & 160.24 & 0.19 & 14669.00 & 141602.01 & 1.09 & 0.88 & 0.97 \\
25277 & 103464 & 1996 & 1316.58 & 0.18 & 131658.00 & 1280026.13 & 1.00 & 0.97 & 0.97 \\
24899 & 103394 & 1996 & 47.59 & 0.13 & 4405.00 & 35662.56 & 1.08 & 0.75 & 0.81 \\
24313 & 103308 & 1996 & 5001.76 & 0.20 & 500176.00 & 4525181.33 & 1.00 & 0.90 & 0.90 \\
22713 & 103034 & 1996 & 258.18 & 0.18 & 26496.00 & 254620.49 & 0.97 & 0.99 & 0.96 \\
22742 & 103057 & 1996 & 3032.74 & 0.20 & 259936.00 & 3035133.39 & 1.17 & 1.00 & 1.17 \\
22919 & 103085 & 1996 & 105.55 & 0.27 & 10533.00 & 99479.19 & 1.00 & 0.94 & 0.94 \\
22677 & 103028 & 1996 & 3450.34 & 0.09 & 324705.00 & 3132122.21 & 1.06 & 0.91 & 0.96 \\
32774 & 106062 & 1996 & 2.44 & 0.00 & 244.00 & 2043.99 & 1.00 & 0.84 & 0.84 \\
31313 & 105878 & 1996 & 126.08 & 0.01 & 9677.00 & 103823.32 & 1.30 & 0.82 & 1.07 \\
31435 & 105886 & 1996 & 88.60 & -0.02 & 8923.00 & 84097.12 & 0.99 & 0.95 & 0.94 \\
10699 & 101312 & 1996 & 891.79 & 0.27 & 74769.00 & 716732.49 & 1.19 & 0.80 & 0.96 \\
25308 & 103466 & 1996 & 513.23 & 0.23 & 51323.00 & 489803.79 & 1.00 & 0.95 & 0.95 \\
31305 & 105877 & 1996 & 2.18 & 0.22 & 130.00 & 1169.82 & 1.68 & 0.54 & 0.90 \\
20706 & 102784 & 1996 & 13893.40 & 0.13 & 1153494.00 & 11927365.35 & 1.20 & 0.86 & 1.03 \\
40004 & 107994 & 1996 & 35.10 & -0.01 & 3816.00 & 34443.97 & 0.92 & 0.98 & 0.90 \\
40032 & 108013 & 1996 & 161.40 & 0.24 & 15112.00 & 155342.23 & 1.07 & 0.96 & 1.03 \\
31298 & 105876 & 1996 & 199.76 & 0.05 & 20719.00 & 165501.67 & 0.96 & 0.83 & 0.80 \\
24417 & 103326 & 1996 & 1451.61 & 0.21 & 122727.00 & 1297569.10 & 1.18 & 0.89 & 1.06 \\
32804 & 106066 & 1996 & 19.44 & 0.00 & 1958.00 & 17592.64 & 0.99 & 0.91 & 0.90 \\
20667 & 102783 & 1996 & 1273.23 & 0.15 & 124016.00 & 1047635.28 & 1.03 & 0.82 & 0.84 \\
20764 & 102789 & 1996 & 622.52 & 0.25 & 55263.00 & 467890.78 & 1.13 & 0.75 & 0.85 \\
10667 & 101307 & 1996 & 128.54 & 0.31 & 9032.00 & 80411.03 & 1.42 & 0.63 & 0.89 \\
31445 & 105890 & 1996 & 32.63 & 0.29 & 3289.00 & 28367.30 & 0.99 & 0.87 & 0.86 \\
21105 & 102832 & 1996 & 34.65 & 0.28 & 3571.00 & 34475.40 & 0.97 & 1.00 & 0.97 \\
31492 & 105900 & 1996 & 2.29 & 0.18 & 229.00 & 2214.11 & 1.00 & 0.97 & 0.97 \\
22944 & 103090 & 1996 & 837.82 & 0.16 & 68769.00 & 806402.14 & 1.22 & 0.96 & 1.17 \\
24837 & 103381 & 1996 & 18418.62 & 0.14 & 1827894.00 & 17387072.17 & 1.01 & 0.94 & 0.95 \\
24230 & 103299 & 1996 & 845.03 & 0.12 & 93511.00 & 819042.83 & 0.90 & 0.97 & 0.88 \\
24469 & 103328 & 1996 & 603.14 & 0.19 & 49567.00 & 508795.12 & 1.22 & 0.84 & 1.03 \\
9444 & 101135 & 1996 & 936.20 & 0.11 & 93643.00 & 798361.48 & 1.00 & 0.85 & 0.85 \\
20974 & 102814 & 1996 & 26.51 & 0.26 & 2095.00 & 21292.39 & 1.27 & 0.80 & 1.02 \\
24195 & 103296 & 1996 & 1683.89 & 0.18 & 163857.00 & 1459238.79 & 1.03 & 0.87 & 0.89 \\
24499 & 103329 & 1996 & 500.88 & 0.25 & 47479.00 & 524863.93 & 1.05 & 1.05 & 1.11 \\
20990 & 102818 & 1996 & 136.33 & 0.23 & 13615.00 & 114603.66 & 1.00 & 0.84 & 0.84 \\
22876 & 103074 & 1996 & 96.20 & 0.17 & 6278.00 & 65708.08 & 1.53 & 0.68 & 1.05 \\
52032 & 301299 & 1996 & 639.88 & 0.24 & 63912.00 & 547069.43 & 1.00 & 0.85 & 0.86 \\
10728 & 101320 & 1996 & 33.83 & 0.40 & 2607.00 & 22010.47 & 1.30 & 0.65 & 0.84 \\
22633 & 103027 & 1996 & 607.91 & 0.02 & 54139.00 & 586372.36 & 1.12 & 0.96 & 1.08 \\
31535 & 105907 & 1996 & 4.08 & 0.00 & 400.00 & 3943.70 & 1.02 & 0.97 & 0.99 \\
20853 & 102797 & 1996 & 55.10 & 0.18 & 5613.00 & 48958.73 & 0.98 & 0.89 & 0.87 \\
20615 & 102775 & 1996 & 834.80 & 0.21 & 80964.00 & 648858.85 & 1.03 & 0.78 & 0.80 \\
10043 & 101256 & 1996 & 8.58 & 0.11 & 881.00 & 7388.99 & 0.97 & 0.86 & 0.84 \\
20881 & 102798 & 1996 & 897.00 & 0.01 & 89632.00 & 864205.88 & 1.00 & 0.96 & 0.96 \\
20909 & 102799 & 1996 & 571.06 & 0.05 & 56860.00 & 551494.64 & 1.00 & 0.97 & 0.97 \\
20923 & 102802 & 1996 & 532.42 & -0.01 & 80719.00 & 749553.80 & 0.66 & 1.41 & 0.93 \\
32541 & 106039 & 1996 & 4.81 & 0.28 & 481.00 & 4291.61 & 1.00 & 0.89 & 0.89 \\
32497 & 106036 & 1996 & 1.54 & -0.04 & 154.00 & 1274.87 & 1.00 & 0.83 & 0.83 \\
32440 & 106026 & 1996 & 3.96 & -0.00 & 397.00 & 3765.80 & 1.00 & 0.95 & 0.95 \\
39890 & 107928 & 1996 & 196.16 & 0.33 & 25168.00 & 241242.56 & 0.78 & 1.23 & 0.96 \\
32413 & 106025 & 1996 & 5.88 & 0.24 & 561.00 & 5334.89 & 1.05 & 0.91 & 0.95 \\
24261 & 103301 & 1996 & 1185.79 & 0.16 & 118083.00 & 1130139.95 & 1.00 & 0.95 & 0.96 \\
24878 & 103383 & 1996 & 2236.05 & 0.08 & 228148.00 & 2205179.00 & 0.98 & 0.99 & 0.97 \\
32513 & 106038 & 1996 & 49.19 & 0.03 & 4919.00 & 42392.43 & 1.00 & 0.86 & 0.86 \\
20944 & 102813 & 1996 & 351.02 & 0.23 & 35102.00 & 305766.83 & 1.00 & 0.87 & 0.87 \\
32290 & 106009 & 1996 & 7.56 & -0.02 & 623.00 & 5392.25 & 1.21 & 0.71 & 0.87 \\
23495 & 103182 & 1996 & 241.19 & 0.22 & 20439.00 & 173012.48 & 1.18 & 0.72 & 0.85 \\
30912 & 105836 & 1996 & 67.29 & 0.26 & 6987.00 & 63547.83 & 0.96 & 0.94 & 0.91 \\
30930 & 105838 & 1996 & 4.82 & 0.17 & 421.00 & 4489.24 & 1.14 & 0.93 & 1.07 \\
22213 & 102996 & 1996 & 283.82 & 0.16 & 27518.00 & 241185.18 & 1.03 & 0.85 & 0.88 \\
9579 & 101151 & 1996 & 157.40 & 0.13 & 15736.00 & 132930.59 & 1.00 & 0.84 & 0.84 \\
23246 & 103152 & 1996 & 1237.37 & 0.17 & 136291.00 & 1249375.78 & 0.91 & 1.01 & 0.92 \\
21747 & 102949 & 1996 & 1774.05 & 0.20 & 165698.00 & 1740492.49 & 1.07 & 0.98 & 1.05 \\
21703 & 102940 & 1996 & 232.84 & 0.15 & 20912.00 & 217559.54 & 1.11 & 0.93 & 1.04 \\
23474 & 103179 & 1996 & 1423.91 & 0.20 & 144873.00 & 1295466.68 & 0.98 & 0.91 & 0.89 \\
23278 & 103154 & 1996 & 707.10 & 0.07 & 73115.00 & 652180.09 & 0.97 & 0.92 & 0.89 \\
22182 & 102994 & 1996 & 62.45 & 0.12 & 6243.00 & 60297.71 & 1.00 & 0.97 & 0.97 \\
21765 & 102951 & 1996 & 4200.99 & 0.22 & 418841.00 & 3472965.19 & 1.00 & 0.83 & 0.83 \\
23293 & 103158 & 1996 & 1167.35 & 0.21 & 134592.00 & 1307065.28 & 0.87 & 1.12 & 0.97 \\
24681 & 103375 & 1996 & 874.40 & 0.06 & 86575.00 & 771191.23 & 1.01 & 0.88 & 0.89 \\
39513 & 107716 & 1996 & 57.03 & 0.21 & 4924.00 & 51962.45 & 1.16 & 0.91 & 1.06 \\
23443 & 103177 & 1996 & 298.31 & 0.15 & 29740.00 & 266097.72 & 1.00 & 0.89 & 0.89 \\
23305 & 103160 & 1996 & 63.12 & 0.25 & 6147.00 & 55824.02 & 1.03 & 0.88 & 0.91 \\
21809 & 102952 & 1996 & 971.47 & 0.10 & 96124.00 & 949339.98 & 1.01 & 0.98 & 0.99 \\
9627 & 101160 & 1996 & 380.51 & 0.32 & 27866.00 & 281122.19 & 1.37 & 0.74 & 1.01 \\
24989 & 103406 & 1996 & 1287.87 & 0.12 & 128787.00 & 1222585.63 & 1.00 & 0.95 & 0.95 \\
30948 & 105841 & 1996 & 26.17 & 0.32 & 1843.00 & 18006.81 & 1.42 & 0.69 & 0.98 \\
23213 & 103145 & 1996 & 124.40 & 0.27 & 10056.00 & 111368.64 & 1.24 & 0.90 & 1.11 \\
9747 & 101186 & 1996 & 239.84 & 0.21 & 25686.00 & 230695.82 & 0.93 & 0.96 & 0.90 \\
21558 & 102894 & 1996 & 174.50 & 0.43 & 17445.00 & 162515.95 & 1.00 & 0.93 & 0.93 \\
10349 & 101283 & 1996 & 710.76 & 0.18 & 65700.00 & 545048.18 & 1.08 & 0.77 & 0.83 \\
30907 & 105811 & 1996 & 3.12 & 0.20 & 286.00 & 2802.61 & 1.09 & 0.90 & 0.98 \\
23165 & 103138 & 1996 & 1239.98 & 0.23 & 107328.00 & 1103747.60 & 1.16 & 0.89 & 1.03 \\
31922 & 105961 & 1996 & 11.57 & -0.00 & 878.00 & 8437.70 & 1.32 & 0.73 & 0.96 \\
21592 & 102895 & 1996 & 824.72 & 0.25 & 85308.00 & 820305.13 & 0.97 & 0.99 & 0.96 \\
31912 & 105960 & 1996 & 19.66 & -0.01 & 1338.00 & 11824.53 & 1.47 & 0.60 & 0.88 \\
31031 & 105851 & 1996 & 15.60 & -0.04 & 1318.00 & 13087.48 & 1.18 & 0.84 & 0.99 \\
23542 & 103184 & 1996 & 1149.58 & 0.19 & 97272.00 & 1013206.95 & 1.18 & 0.88 & 1.04 \\
21673 & 102939 & 1996 & 1908.77 & 0.22 & 156006.00 & 1722540.54 & 1.22 & 0.90 & 1.10 \\
31018 & 105848 & 1996 & 42.21 & 0.15 & 4061.00 & 40160.56 & 1.04 & 0.95 & 0.99 \\
21618 & 102901 & 1996 & 44.56 & 0.15 & 4395.00 & 42189.25 & 1.01 & 0.95 & 0.96 \\
31010 & 105847 & 1996 & 23.98 & 0.21 & 2310.00 & 23003.95 & 1.04 & 0.96 & 1.00 \\
24641 & 103373 & 1996 & 182.24 & 0.19 & 18137.00 & 170545.64 & 1.00 & 0.94 & 0.94 \\
21629 & 102924 & 1996 & 37.48 & 0.04 & 3751.00 & 35589.40 & 1.00 & 0.95 & 0.95 \\
23183 & 103144 & 1996 & 214.28 & 0.18 & 23440.00 & 217891.40 & 0.91 & 1.02 & 0.93 \\
30994 & 105846 & 1996 & 111.51 & 0.28 & 8873.00 & 86794.78 & 1.26 & 0.78 & 0.98 \\
23516 & 103183 & 1996 & 443.37 & 0.21 & 41796.00 & 404040.15 & 1.06 & 0.91 & 0.97 \\
10318 & 101279 & 1996 & 45.80 & 0.20 & 4580.00 & 44312.49 & 1.00 & 0.97 & 0.97 \\
31026 & 105849 & 1996 & 18.96 & -0.02 & 1779.00 & 17794.14 & 1.07 & 0.94 & 1.00 \\
24721 & 103376 & 1996 & 4667.20 & 0.15 & 476542.00 & 4576537.74 & 0.98 & 0.98 & 0.96 \\
10226 & 101275 & 1996 & 294.44 & 0.17 & 26905.00 & 275241.73 & 1.09 & 0.93 & 1.02 \\
22012 & 102987 & 1996 & 370.28 & 0.19 & 38235.00 & 338831.38 & 0.97 & 0.92 & 0.89 \\
31601 & 105916 & 1996 & 152.77 & 0.10 & 15757.00 & 138450.74 & 0.97 & 0.91 & 0.88 \\
31860 & 105949 & 1996 & 52.75 & 0.03 & 5339.00 & 47007.79 & 0.99 & 0.89 & 0.88 \\
21953 & 102981 & 1996 & 77.61 & 0.17 & 7740.00 & 71194.05 & 1.00 & 0.92 & 0.92 \\
25070 & 103429 & 1996 & 983.64 & 0.08 & 98364.00 & 882375.30 & 1.00 & 0.90 & 0.90 \\
31641 & 105920 & 1996 & 365.09 & 0.14 & 69468.00 & 693672.11 & 0.53 & 1.90 & 1.00 \\
21986 & 102983 & 1996 & 204.42 & 0.21 & 23658.00 & 219783.18 & 0.86 & 1.08 & 0.93 \\
31787 & 105936 & 1996 & 41.04 & 0.19 & 3579.00 & 35481.61 & 1.15 & 0.86 & 0.99 \\
10193 & 101268 & 1996 & 1032.09 & 0.17 & 111832.00 & 902514.31 & 0.92 & 0.87 & 0.81 \\
21998 & 102984 & 1996 & 186.69 & 0.17 & 20282.00 & 208882.10 & 0.92 & 1.12 & 1.03 \\
31675 & 105930 & 1996 & 34.29 & -0.04 & 2286.00 & 23774.17 & 1.50 & 0.69 & 1.04 \\
10207 & 101274 & 1996 & 153.94 & 0.18 & 14322.00 & 143444.42 & 1.07 & 0.93 & 1.00 \\
23347 & 103166 & 1996 & 4.61 & 0.15 & 458.00 & 4120.44 & 1.01 & 0.89 & 0.90 \\
9722 & 101179 & 1996 & 231.82 & 0.25 & 23182.00 & 212727.34 & 1.00 & 0.92 & 0.92 \\
21850 & 102957 & 1996 & 217.94 & 0.18 & 21839.00 & 215938.93 & 1.00 & 0.99 & 0.99 \\
9658 & 101161 & 1996 & 536.90 & 0.29 & 51749.00 & 506092.39 & 1.04 & 0.94 & 0.98 \\
21880 & 102964 & 1996 & 102.37 & 0.23 & 10523.00 & 93745.83 & 0.97 & 0.92 & 0.89 \\
23409 & 103175 & 1996 & 1507.83 & 0.19 & 144380.00 & 1443905.02 & 1.04 & 0.96 & 1.00 \\
22138 & 102993 & 1996 & 1378.06 & 0.31 & 117289.00 & 1206081.19 & 1.17 & 0.88 & 1.03 \\
25143 & 103439 & 1996 & 83.48 & 0.24 & 8730.00 & 84797.56 & 0.96 & 1.02 & 0.97 \\
23379 & 103174 & 1996 & 1181.74 & 0.20 & 110223.00 & 1076228.27 & 1.07 & 0.91 & 0.98 \\
24758 & 103377 & 1996 & 1125.49 & 0.15 & 108315.00 & 1037122.31 & 1.04 & 0.92 & 0.96 \\
30983 & 105845 & 1996 & 2.40 & 0.20 & 242.00 & 2428.15 & 0.99 & 1.01 & 1.00 \\
21888 & 102969 & 1996 & 390.89 & 0.24 & 33915.00 & 360993.67 & 1.15 & 0.92 & 1.06 \\
25133 & 103436 & 1996 & 10.40 & 0.11 & 1112.00 & 8847.31 & 0.94 & 0.85 & 0.80 \\
9677 & 101165 & 1996 & 394.67 & 0.22 & 36301.00 & 384532.14 & 1.09 & 0.97 & 1.06 \\
30976 & 105843 & 1996 & 5.96 & 0.19 & 488.00 & 4186.77 & 1.22 & 0.70 & 0.86 \\
30956 & 105842 & 1996 & 37.95 & 0.13 & 3795.00 & 36251.96 & 1.00 & 0.96 & 0.96 \\
22104 & 102990 & 1996 & 556.71 & 0.20 & 51501.00 & 536205.86 & 1.08 & 0.96 & 1.04 \\
22043 & 102988 & 1996 & 26.80 & 0.26 & 2495.00 & 26155.72 & 1.07 & 0.98 & 1.05 \\
23365 & 103172 & 1996 & 62.75 & 0.18 & 4993.00 & 44802.68 & 1.26 & 0.71 & 0.90 \\
31036 & 105852 & 1996 & 16.84 & 0.00 & 1708.00 & 15790.77 & 0.99 & 0.94 & 0.92 \\
22241 & 102997 & 1996 & 3273.50 & 0.12 & 327350.00 & 3365552.75 & 1.00 & 1.03 & 1.03 \\
9765 & 101192 & 1996 & 15.81 & 0.15 & 1565.00 & 13266.76 & 1.01 & 0.84 & 0.85 \\
23549 & 103186 & 1996 & 271.30 & 0.35 & 17909.00 & 159674.12 & 1.51 & 0.59 & 0.89 \\
23855 & 103224 & 1996 & 276.26 & 0.15 & 26641.00 & 236808.48 & 1.04 & 0.86 & 0.89 \\
25195 & 103460 & 1996 & 715.96 & 0.16 & 71596.00 & 670011.11 & 1.00 & 0.94 & 0.94 \\
21246 & 102842 & 1996 & 8.71 & 0.13 & 738.00 & 8025.32 & 1.18 & 0.92 & 1.09 \\
23822 & 103214 & 1996 & 1308.16 & 0.20 & 130816.00 & 1216276.34 & 1.00 & 0.93 & 0.93 \\
32234 & 106007 & 1996 & 9.99 & 0.04 & 750.00 & 7172.65 & 1.33 & 0.72 & 0.96 \\
21249 & 102843 & 1996 & 124.61 & 0.23 & 12470.00 & 120120.79 & 1.00 & 0.96 & 0.96 \\
21273 & 102844 & 1996 & 220.12 & 0.23 & 22036.00 & 214210.18 & 1.00 & 0.97 & 0.97 \\
30559 & 105770 & 1996 & 34.61 & -0.01 & 3461.00 & 32996.31 & 1.00 & 0.95 & 0.95 \\
9884 & 101200 & 1996 & 30.85 & 0.21 & 3085.00 & 28243.69 & 1.00 & 0.92 & 0.92 \\
21307 & 102847 & 1996 & 82.43 & 0.22 & 7251.00 & 74080.18 & 1.14 & 0.90 & 1.02 \\
23098 & 103122 & 1996 & 17.10 & -0.03 & 1148.00 & 11339.71 & 1.49 & 0.66 & 0.99 \\
32100 & 105983 & 1996 & 104.00 & -0.02 & 9779.00 & 99637.23 & 1.06 & 0.96 & 1.02 \\
32096 & 105982 & 1996 & 7.84 & 0.04 & 543.00 & 5609.18 & 1.44 & 0.72 & 1.03 \\
22409 & 103011 & 1996 & 105.20 & 0.23 & 10570.00 & 97699.79 & 1.00 & 0.93 & 0.92 \\
32068 & 105980 & 1996 & 31.43 & 0.03 & 2182.00 & 18396.90 & 1.44 & 0.59 & 0.84 \\
23792 & 103213 & 1996 & 759.13 & 0.14 & 75913.00 & 657677.90 & 1.00 & 0.87 & 0.87 \\
30648 & 105781 & 1996 & 57.83 & 0.39 & 4538.00 & 43699.29 & 1.27 & 0.76 & 0.96 \\
21298 & 102846 & 1996 & 82.29 & 0.20 & 8156.00 & 80234.96 & 1.01 & 0.98 & 0.98 \\
30549 & 105769 & 1996 & 13.54 & -0.01 & 1377.00 & 12430.93 & 0.98 & 0.92 & 0.90 \\
22464 & 103014 & 1996 & 314.90 & 0.24 & 26189.00 & 291292.31 & 1.20 & 0.93 & 1.11 \\
21137 & 102833 & 1996 & 18.13 & 0.14 & 1863.00 & 18058.56 & 0.97 & 1.00 & 0.97 \\
23907 & 103232 & 1996 & 57.04 & 0.27 & 3500.00 & 31616.35 & 1.63 & 0.55 & 0.90 \\
21159 & 102835 & 1996 & 66.54 & 0.18 & 6390.00 & 63101.23 & 1.04 & 0.95 & 0.99 \\
30493 & 105762 & 1996 & 87.59 & 0.07 & 9201.00 & 71199.39 & 0.95 & 0.81 & 0.77 \\
10504 & 101294 & 1996 & 12.20 & 0.23 & 1060.00 & 9900.45 & 1.15 & 0.81 & 0.93 \\
22524 & 103017 & 1996 & 860.74 & 0.23 & 70468.00 & 761828.35 & 1.22 & 0.89 & 1.08 \\
21191 & 102837 & 1996 & 115.67 & 0.31 & 11188.00 & 105741.52 & 1.03 & 0.91 & 0.95 \\
10474 & 101287 & 1996 & 291.92 & 0.12 & 26518.00 & 250209.36 & 1.10 & 0.86 & 0.94 \\
23891 & 103228 & 1996 & 57.07 & 0.12 & 5569.00 & 51221.32 & 1.02 & 0.90 & 0.92 \\
21227 & 102838 & 1996 & 289.34 & 0.11 & 28934.00 & 262864.72 & 1.00 & 0.91 & 0.91 \\
10458 & 101286 & 1996 & 417.73 & 0.22 & 78810.00 & 372095.67 & 0.53 & 0.89 & 0.47 \\
30521 & 105763 & 1996 & 92.31 & 0.22 & 9212.00 & 83873.54 & 1.00 & 0.91 & 0.91 \\
9492 & 101140 & 1996 & 651.30 & 0.21 & 65127.00 & 625846.37 & 1.00 & 0.96 & 0.96 \\
10073 & 101258 & 1996 & 1390.27 & 0.20 & 125094.00 & 1102753.04 & 1.11 & 0.79 & 0.88 \\
21239 & 102841 & 1996 & 83.53 & 0.11 & 7273.00 & 73367.80 & 1.15 & 0.88 & 1.01 \\
51904 & 300102 & 1996 & 16.49 & 0.22 & 1583.00 & 15552.22 & 1.04 & 0.94 & 0.98 \\
22482 & 103015 & 1996 & 71.43 & 0.21 & 5582.00 & 60448.12 & 1.28 & 0.85 & 1.08 \\
23873 & 103226 & 1996 & 89.18 & 0.10 & 8819.00 & 82016.58 & 1.01 & 0.92 & 0.93 \\
23128 & 103130 & 1996 & 94.12 & 0.15 & 11129.00 & 106620.28 & 0.85 & 1.13 & 0.96 \\
32303 & 106010 & 1996 & 6.83 & -0.01 & 555.00 & 4792.97 & 1.23 & 0.70 & 0.86 \\
30748 & 105793 & 1996 & 412.90 & 0.22 & 43227.00 & 352805.85 & 0.96 & 0.85 & 0.82 \\
9523 & 101142 & 1996 & 237.50 & 0.09 & 23746.00 & 198855.06 & 1.00 & 0.84 & 0.84 \\
31991 & 105967 & 1996 & 1.27 & 0.02 & 120.00 & 1084.39 & 1.06 & 0.85 & 0.90 \\
21449 & 102872 & 1996 & 316.82 & 0.18 & 31815.00 & 270753.89 & 1.00 & 0.85 & 0.85 \\
23138 & 103134 & 1996 & 396.38 & 0.22 & 37027.00 & 397178.74 & 1.07 & 1.00 & 1.07 \\
24972 & 103402 & 1996 & 203.34 & 0.22 & 18242.00 & 191805.88 & 1.11 & 0.94 & 1.05 \\
52126 & 301859 & 1996 & 21.19 & 0.23 & 2575.00 & 21421.57 & 0.82 & 1.01 & 0.83 \\
31933 & 105963 & 1996 & 111.70 & 0.22 & 10605.00 & 101551.48 & 1.05 & 0.91 & 0.96 \\
21489 & 102873 & 1996 & 145.58 & 0.14 & 14537.00 & 133902.58 & 1.00 & 0.92 & 0.92 \\
23622 & 103204 & 1996 & 134.78 & 0.08 & 13478.00 & 128087.65 & 1.00 & 0.95 & 0.95 \\
23653 & 103205 & 1996 & 21.69 & 0.15 & 2169.00 & 20141.10 & 1.00 & 0.93 & 0.93 \\
22272 & 102999 & 1996 & 867.56 & 0.11 & 88508.00 & 821434.19 & 0.98 & 0.95 & 0.93 \\
21504 & 102876 & 1996 & 57.49 & 0.21 & 5749.00 & 57110.41 & 1.00 & 0.99 & 0.99 \\
23568 & 103193 & 1996 & 44.83 & 0.14 & 4480.00 & 41297.71 & 1.00 & 0.92 & 0.92 \\
21528 & 102893 & 1996 & 28.60 & -0.03 & 2964.00 & 28478.83 & 0.96 & 1.00 & 0.96 \\
30879 & 105806 & 1996 & 10.79 & 0.22 & 689.00 & 6932.36 & 1.57 & 0.64 & 1.01 \\
23154 & 103136 & 1996 & 297.86 & 0.17 & 23688.00 & 249877.53 & 1.26 & 0.84 & 1.05 \\
10163 & 101264 & 1996 & 250.65 & 0.21 & 24749.00 & 212517.50 & 1.01 & 0.85 & 0.86 \\
30903 & 105809 & 1996 & 3.22 & 0.09 & 320.00 & 3163.14 & 1.01 & 0.98 & 0.99 \\
10379 & 101284 & 1996 & 188.28 & 0.30 & 14376.00 & 127274.24 & 1.31 & 0.68 & 0.89 \\
9795 & 101193 & 1996 & 120.36 & 0.20 & 12404.00 & 118873.79 & 0.97 & 0.99 & 0.96 \\
9826 & 101194 & 1996 & 106.09 & 0.08 & 11170.00 & 104104.31 & 0.95 & 0.98 & 0.93 \\
23686 & 103208 & 1996 & 1243.65 & 0.20 & 124365.00 & 1098131.28 & 1.00 & 0.88 & 0.88 \\
30795 & 105803 & 1996 & 230.30 & -0.01 & 22460.00 & 211296.58 & 1.03 & 0.92 & 0.94 \\
9856 & 101198 & 1996 & 158.33 & 0.12 & 16857.00 & 151170.64 & 0.94 & 0.95 & 0.90 \\
30823 & 105804 & 1996 & 46.70 & -0.02 & 4550.00 & 40176.06 & 1.03 & 0.86 & 0.88 \\
30851 & 105805 & 1996 & 30.96 & -0.00 & 3073.00 & 28861.75 & 1.01 & 0.93 & 0.94 \\
23756 & 103212 & 1996 & 1769.57 & 0.18 & 176957.00 & 1544399.28 & 1.00 & 0.87 & 0.87 \\
10414 & 101285 & 1996 & 120.85 & 0.29 & 8785.00 & 98968.87 & 1.38 & 0.82 & 1.13 \\
22373 & 103008 & 1996 & 68.10 & 0.18 & 6982.00 & 68157.47 & 0.98 & 1.00 & 0.98 \\
10125 & 101262 & 1996 & 12.06 & 0.19 & 1122.00 & 11527.07 & 1.07 & 0.96 & 1.03 \\
21326 & 102852 & 1996 & 210.40 & 0.22 & 19858.00 & 172054.55 & 1.06 & 0.82 & 0.87 \\
9505 & 101141 & 1996 & 4281.80 & 0.17 & 428176.00 & 3469473.81 & 1.00 & 0.81 & 0.81 \\
32056 & 105978 & 1996 & 92.65 & -0.04 & 9243.00 & 87020.84 & 1.00 & 0.94 & 0.94 \\
21390 & 102861 & 1996 & 76.17 & 0.24 & 7098.00 & 75765.14 & 1.07 & 0.99 & 1.07 \\
22329 & 103007 & 1996 & 1284.93 & 0.21 & 133385.00 & 1218012.24 & 0.96 & 0.95 & 0.91 \\
21414 & 102870 & 1996 & 36.09 & 0.16 & 3591.00 & 29047.72 & 1.01 & 0.80 & 0.81 \\
23720 & 103209 & 1996 & 513.79 & 0.01 & 51379.00 & 473653.36 & 1.00 & 0.92 & 0.92 \\
52098 & 301571 & 1996 & 80.09 & 0.05 & 7564.00 & 76455.08 & 1.06 & 0.95 & 1.01 \\
21418 & 102871 & 1996 & 141.38 & 0.16 & 14133.00 & 134410.71 & 1.00 & 0.95 & 0.95 \\
13581 & 101744 & 1996 & 1705.46 & 0.08 & 174132.00 & 1548114.52 & 0.98 & 0.91 & 0.89 \\
25691 & 103514 & 1996 & 1424.95 & 0.17 & 142495.00 & 1305803.83 & 1.00 & 0.92 & 0.92 \\
15494 & 101999 & 1996 & 2290.24 & 0.28 & 186948.00 & 1862822.65 & 1.23 & 0.81 & 1.00 \\
15475 & 101998 & 1996 & 644.63 & 0.35 & 58546.00 & 643039.51 & 1.10 & 1.00 & 1.10 \\
27093 & 103652 & 1996 & 494.60 & 0.15 & 50520.00 & 419822.79 & 0.98 & 0.85 & 0.83 \\
27112 & 103658 & 1996 & 1202.40 & 0.13 & 124570.00 & 1052320.13 & 0.97 & 0.88 & 0.84 \\
15455 & 101992 & 1996 & 1429.53 & 0.19 & 144237.00 & 1473970.75 & 0.99 & 1.03 & 1.02 \\
12761 & 101593 & 1996 & 93.74 & 0.20 & 9299.00 & 84242.52 & 1.01 & 0.90 & 0.91 \\
15449 & 101991 & 1996 & 103.02 & 0.16 & 9750.00 & 99298.03 & 1.06 & 0.96 & 1.02 \\
15421 & 101990 & 1996 & 175.10 & 0.14 & 19506.00 & 174955.35 & 0.90 & 1.00 & 0.90 \\
15393 & 101989 & 1996 & 161.02 & 0.21 & 14323.00 & 134986.56 & 1.12 & 0.84 & 0.94 \\
47183 & 200342 & 1996 & 7080.10 & 0.17 & 693418.00 & 6673802.49 & 1.02 & 0.94 & 0.96 \\
15363 & 101988 & 1996 & 231.42 & 0.21 & 17340.00 & 161741.49 & 1.33 & 0.70 & 0.93 \\
27125 & 105243 & 1996 & 31.71 & 0.36 & 2065.00 & 17353.84 & 1.54 & 0.55 & 0.84 \\
15525 & 102000 & 1996 & 574.35 & 0.10 & 61329.00 & 618598.64 & 0.94 & 1.08 & 1.01 \\
28721 & 105472 & 1996 & 130.66 & 0.18 & 13244.00 & 132010.30 & 0.99 & 1.01 & 1.00 \\
28743 & 105475 & 1996 & 238.03 & 0.28 & 26903.00 & 229019.42 & 0.88 & 0.96 & 0.85 \\
26976 & 103642 & 1996 & 14.05 & 0.18 & 1405.00 & 13641.47 & 1.00 & 0.97 & 0.97 \\
15695 & 102015 & 1996 & 374.97 & 0.17 & 33736.00 & 383748.85 & 1.11 & 1.02 & 1.14 \\
12709 & 101589 & 1996 & 140.25 & 0.17 & 13782.00 & 135609.43 & 1.02 & 0.97 & 0.98 \\
26983 & 103643 & 1996 & 25.83 & 0.18 & 2583.00 & 24928.56 & 1.00 & 0.97 & 0.97 \\
12716 & 101590 & 1996 & 15.12 & 0.15 & 1491.00 & 14524.55 & 1.01 & 0.96 & 0.97 \\
15635 & 102010 & 1996 & 5993.78 & 0.14 & 630115.00 & 5703040.84 & 0.95 & 0.95 & 0.91 \\
28831 & 105479 & 1996 & 19.50 & -0.01 & 1948.00 & 16828.41 & 1.00 & 0.86 & 0.86 \\
15614 & 102009 & 1996 & 306.57 & 0.20 & 30662.00 & 293810.60 & 1.00 & 0.96 & 0.96 \\
15584 & 102007 & 1996 & 2318.88 & 0.24 & 178884.00 & 1648855.83 & 1.30 & 0.71 & 0.92 \\
15569 & 102005 & 1996 & 827.86 & 0.21 & 82283.00 & 787155.25 & 1.01 & 0.95 & 0.96 \\
27009 & 103644 & 1996 & 205.84 & 0.19 & 20584.00 & 200921.19 & 1.00 & 0.98 & 0.98 \\
27035 & 103645 & 1996 & 755.62 & 0.26 & 75305.00 & 645631.15 & 1.00 & 0.85 & 0.86 \\
28772 & 105476 & 1996 & 24.79 & 0.30 & 1838.00 & 20578.13 & 1.35 & 0.83 & 1.12 \\
12701 & 101588 & 1996 & 617.55 & 0.07 & 61788.00 & 586846.47 & 1.00 & 0.95 & 0.95 \\
27194 & 105251 & 1996 & 74.82 & 0.13 & 7482.00 & 66813.43 & 1.00 & 0.89 & 0.89 \\
28675 & 105463 & 1996 & 252.23 & 0.23 & 22160.00 & 210987.37 & 1.14 & 0.84 & 0.95 \\
28646 & 105458 & 1996 & 391.42 & 0.20 & 38335.00 & 375348.97 & 1.02 & 0.96 & 0.98 \\
28631 & 105457 & 1996 & 99.34 & 0.25 & 10023.00 & 94732.04 & 0.99 & 0.95 & 0.95 \\
15171 & 101964 & 1996 & 150.16 & 0.26 & 15016.00 & 141480.29 & 1.00 & 0.94 & 0.94 \\
37707 & 107004 & 1996 & 27.48 & 0.22 & 2265.00 & 19513.74 & 1.21 & 0.71 & 0.86 \\
12878 & 101603 & 1996 & 827.13 & 0.24 & 85417.00 & 794525.60 & 0.97 & 0.96 & 0.93 \\
15142 & 101963 & 1996 & 1092.13 & 0.21 & 109257.00 & 1017492.18 & 1.00 & 0.93 & 0.93 \\
47376 & 210681 & 1996 & 23797.90 & 0.22 & 1965860.00 & 16816747.46 & 1.21 & 0.71 & 0.86 \\
15110 & 101958 & 1996 & 225.56 & 0.22 & 22575.00 & 222006.26 & 1.00 & 0.98 & 0.98 \\
15092 & 101956 & 1996 & 1468.30 & 0.19 & 146954.00 & 1417453.17 & 1.00 & 0.97 & 0.96 \\
15041 & 101953 & 1996 & 278.21 & 0.18 & 27635.00 & 269785.36 & 1.01 & 0.97 & 0.98 \\
27271 & 105260 & 1996 & 199.09 & 0.26 & 19909.00 & 181075.31 & 1.00 & 0.91 & 0.91 \\
27309 & 105268 & 1996 & 90.97 & 0.18 & 9142.00 & 82459.51 & 1.00 & 0.91 & 0.90 \\
15005 & 101933 & 1996 & 271.70 & 0.12 & 26431.00 & 245609.62 & 1.03 & 0.90 & 0.93 \\
27187 & 105250 & 1996 & 4.10 & 0.27 & 488.00 & 4646.83 & 0.84 & 1.13 & 0.95 \\
47332 & 210203 & 1996 & 1681.73 & 0.20 & 159818.00 & 1598234.61 & 1.05 & 0.95 & 1.00 \\
27197 & 105252 & 1996 & 117.56 & 0.32 & 12378.00 & 113059.30 & 0.95 & 0.96 & 0.91 \\
12800 & 101600 & 1996 & 945.84 & 0.21 & 93316.00 & 868532.95 & 1.01 & 0.92 & 0.93 \\
15333 & 101987 & 1996 & 406.95 & 0.19 & 40328.00 & 369603.53 & 1.01 & 0.91 & 0.92 \\
28713 & 105471 & 1996 & 76.06 & 0.11 & 7799.00 & 74451.84 & 0.98 & 0.98 & 0.95 \\
47248 & 200344 & 1996 & 1322.98 & 0.22 & 124247.00 & 1056469.18 & 1.06 & 0.80 & 0.85 \\
15297 & 101982 & 1996 & 236.08 & 0.15 & 23581.00 & 223396.10 & 1.00 & 0.95 & 0.95 \\
12821 & 101601 & 1996 & 940.44 & 0.25 & 93864.00 & 860014.77 & 1.00 & 0.91 & 0.92 \\
15279 & 101977 & 1996 & 58.33 & 0.18 & 5557.00 & 58329.50 & 1.05 & 1.00 & 1.05 \\
12834 & 101602 & 1996 & 1596.15 & 0.18 & 128300.00 & 1118872.08 & 1.24 & 0.70 & 0.87 \\
15248 & 101972 & 1996 & 270.60 & 0.25 & 27060.00 & 241327.46 & 1.00 & 0.89 & 0.89 \\
28694 & 105465 & 1996 & 23.69 & 0.22 & 2163.00 & 20184.68 & 1.10 & 0.85 & 0.93 \\
15232 & 101970 & 1996 & 52.80 & 0.17 & 5284.00 & 48982.08 & 1.00 & 0.93 & 0.93 \\
15213 & 101968 & 1996 & 100.95 & 0.14 & 10095.00 & 97425.53 & 1.00 & 0.97 & 0.97 \\
12911 & 101606 & 1996 & 2857.58 & 0.24 & 293488.00 & 2729682.22 & 0.97 & 0.96 & 0.93 \\
15720 & 102016 & 1996 & 1266.35 & 0.22 & 136624.00 & 1176967.03 & 0.93 & 0.93 & 0.86 \\
15754 & 102017 & 1996 & 3116.95 & 0.25 & 312665.00 & 2828858.76 & 1.00 & 0.91 & 0.90 \\
29073 & 105525 & 1996 & 6.23 & 0.25 & 622.00 & 5079.41 & 1.00 & 0.82 & 0.82 \\
29063 & 105523 & 1996 & 40.41 & 0.25 & 4041.00 & 40031.83 & 1.00 & 0.99 & 0.99 \\
26646 & 103595 & 1996 & 104.57 & 0.12 & 10457.00 & 103278.51 & 1.00 & 0.99 & 0.99 \\
29034 & 105522 & 1996 & 51.52 & 0.19 & 4679.00 & 44358.81 & 1.10 & 0.86 & 0.95 \\
28999 & 105512 & 1996 & 4.17 & 0.19 & 425.00 & 4119.31 & 0.98 & 0.99 & 0.97 \\
26660 & 103597 & 1996 & 260.66 & 0.19 & 24789.00 & 222603.33 & 1.05 & 0.85 & 0.90 \\
12514 & 101545 & 1996 & 139.08 & 0.27 & 13908.00 & 123704.41 & 1.00 & 0.89 & 0.89 \\
49059 & 240212 & 1996 & 1631.90 & 0.12 & 144977.00 & 1539430.72 & 1.13 & 0.94 & 1.06 \\
26690 & 103600 & 1996 & 10.18 & 0.35 & 1017.00 & 9039.38 & 1.00 & 0.89 & 0.89 \\
16304 & 102124 & 1996 & 995.80 & 0.24 & 95859.00 & 999576.66 & 1.04 & 1.00 & 1.04 \\
16269 & 102113 & 1996 & 218.16 & 0.12 & 22903.00 & 212564.81 & 0.95 & 0.97 & 0.93 \\
26719 & 103601 & 1996 & 94.85 & 0.19 & 9485.00 & 80003.75 & 1.00 & 0.84 & 0.84 \\
12543 & 101553 & 1996 & 57.74 & 0.21 & 4727.00 & 50185.22 & 1.22 & 0.87 & 1.06 \\
16253 & 102105 & 1996 & 136.42 & 0.21 & 13636.00 & 128060.87 & 1.00 & 0.94 & 0.94 \\
16231 & 102102 & 1996 & 719.02 & 0.06 & 71972.00 & 628017.45 & 1.00 & 0.87 & 0.87 \\
16358 & 102130 & 1996 & 687.20 & 0.17 & 65561.00 & 630944.51 & 1.05 & 0.92 & 0.96 \\
16213 & 102095 & 1996 & 40.68 & 0.01 & 4105.00 & 33613.38 & 0.99 & 0.83 & 0.82 \\
16393 & 102132 & 1996 & 41.47 & 0.16 & 4002.00 & 40056.88 & 1.04 & 0.97 & 1.00 \\
16410 & 102134 & 1996 & 66.61 & 0.29 & 5663.00 & 51647.93 & 1.18 & 0.78 & 0.91 \\
49102 & 240222 & 1996 & 667.44 & 0.22 & 61372.00 & 613710.26 & 1.09 & 0.92 & 1.00 \\
16601 & 102159 & 1996 & 1.53 & 0.21 & 154.00 & 1440.72 & 0.99 & 0.94 & 0.94 \\
16557 & 102156 & 1996 & 1.55 & 0.46 & 153.00 & 1342.92 & 1.01 & 0.87 & 0.88 \\
29172 & 105534 & 1996 & 282.93 & 0.29 & 27378.00 & 243413.28 & 1.03 & 0.86 & 0.89 \\
29144 & 105533 & 1996 & 117.93 & 0.15 & 10435.00 & 89653.27 & 1.13 & 0.76 & 0.86 \\
16513 & 102152 & 1996 & 180.03 & 0.17 & 18003.00 & 178316.48 & 1.00 & 0.99 & 0.99 \\
16491 & 102151 & 1996 & 17.34 & 0.24 & 1734.00 & 16506.86 & 1.00 & 0.95 & 0.95 \\
16460 & 102150 & 1996 & 211.52 & 0.02 & 21152.00 & 180130.42 & 1.00 & 0.85 & 0.85 \\
29124 & 105531 & 1996 & 20.03 & 0.15 & 2554.00 & 18340.84 & 0.78 & 0.92 & 0.72 \\
29122 & 105529 & 1996 & 256.60 & -0.02 & 25635.00 & 249946.98 & 1.00 & 0.97 & 0.98 \\
29101 & 105527 & 1996 & 1.81 & 0.01 & 162.00 & 1530.60 & 1.11 & 0.85 & 0.94 \\
12502 & 101544 & 1996 & 20.64 & 0.24 & 2064.00 & 19551.10 & 1.00 & 0.95 & 0.95 \\
26614 & 103593 & 1996 & 25374.47 & 0.20 & 2403165.00 & 23662613.57 & 1.06 & 0.93 & 0.98 \\
16401 & 102133 & 1996 & 42.70 & 0.05 & 4440.00 & 36629.54 & 0.96 & 0.86 & 0.82 \\
28861 & 105489 & 1996 & 2.64 & 0.05 & 261.00 & 2502.05 & 1.01 & 0.95 & 0.96 \\
28992 & 105511 & 1996 & 25.09 & 0.32 & 2509.00 & 24770.80 & 1.00 & 0.99 & 0.99 \\
16166 & 102089 & 1996 & 172.06 & 0.18 & 16121.00 & 174594.43 & 1.07 & 1.01 & 1.08 \\
16022 & 102073 & 1996 & 8595.15 & 0.17 & 859507.00 & 7160347.22 & 1.00 & 0.83 & 0.83 \\
12636 & 101561 & 1996 & 75.22 & 0.21 & 7528.00 & 68567.59 & 1.00 & 0.91 & 0.91 \\
48994 & 240198 & 1996 & 1213.51 & 0.24 & 121785.00 & 1248566.65 & 1.00 & 1.03 & 1.03 \\
26958 & 103638 & 1996 & 102.39 & 0.19 & 10245.00 & 96210.96 & 1.00 & 0.94 & 0.94 \\
48980 & 240197 & 1996 & 670.74 & 0.20 & 69114.00 & 679031.75 & 0.97 & 1.01 & 0.98 \\
12667 & 101562 & 1996 & 232.36 & 0.18 & 14740.00 & 144565.52 & 1.58 & 0.62 & 0.98 \\
15940 & 102061 & 1996 & 5.40 & 0.32 & 545.00 & 5117.47 & 0.99 & 0.95 & 0.94 \\
15910 & 102059 & 1996 & 442.93 & 0.22 & 41185.00 & 346850.35 & 1.08 & 0.78 & 0.84 \\
15887 & 102052 & 1996 & 473.80 & 0.10 & 41833.00 & 418360.29 & 1.13 & 0.88 & 1.00 \\
26968 & 103640 & 1996 & 200.26 & 0.19 & 20023.00 & 196179.54 & 1.00 & 0.98 & 0.98 \\
15841 & 102043 & 1996 & 189.76 & 0.19 & 19013.00 & 180173.41 & 1.00 & 0.95 & 0.95 \\
15800 & 102026 & 1996 & 28.36 & 0.04 & 2855.00 & 24963.84 & 0.99 & 0.88 & 0.87 \\
16197 & 102090 & 1996 & 284.38 & 0.18 & 30305.00 & 277613.95 & 0.94 & 0.98 & 0.92 \\
49007 & 240199 & 1996 & 374.93 & 0.31 & 49327.00 & 418784.25 & 0.76 & 1.12 & 0.85 \\
26910 & 103621 & 1996 & 74.93 & 0.20 & 7127.00 & 77729.75 & 1.05 & 1.04 & 1.09 \\
28981 & 105510 & 1996 & 109.03 & -0.01 & 11079.00 & 97231.12 & 0.98 & 0.89 & 0.88 \\
28955 & 105508 & 1996 & 12.70 & 0.04 & 1270.00 & 11511.56 & 1.00 & 0.91 & 0.91 \\
28936 & 105507 & 1996 & 440.05 & 0.22 & 33423.00 & 308115.87 & 1.32 & 0.70 & 0.92 \\
16147 & 102087 & 1996 & 381.06 & 0.11 & 38411.00 & 362596.98 & 0.99 & 0.95 & 0.94 \\
26758 & 103606 & 1996 & 20.26 & 0.23 & 2011.00 & 16702.36 & 1.01 & 0.82 & 0.83 \\
12567 & 101554 & 1996 & 303.59 & 0.16 & 27163.00 & 296419.13 & 1.12 & 0.98 & 1.09 \\
12584 & 101555 & 1996 & 7.38 & 0.20 & 641.00 & 6767.78 & 1.15 & 0.92 & 1.06 \\
26790 & 103607 & 1996 & 569.76 & 0.17 & 58533.00 & 550863.14 & 0.97 & 0.97 & 0.94 \\
16104 & 102080 & 1996 & 649.89 & 0.23 & 66989.00 & 604750.31 & 0.97 & 0.93 & 0.90 \\
12592 & 101557 & 1996 & 35.60 & 0.20 & 3557.00 & 31406.24 & 1.00 & 0.88 & 0.88 \\
26817 & 103608 & 1996 & 56.57 & 0.21 & 5667.00 & 54274.07 & 1.00 & 0.96 & 0.96 \\
26840 & 103609 & 1996 & 36.70 & 0.26 & 3665.00 & 35838.73 & 1.00 & 0.98 & 0.98 \\
16073 & 102079 & 1996 & 751.66 & 0.15 & 72585.00 & 625593.11 & 1.04 & 0.83 & 0.86 \\
26888 & 103620 & 1996 & 127.01 & 0.22 & 12035.00 & 110837.84 & 1.06 & 0.87 & 0.92 \\
12606 & 101560 & 1996 & 18.29 & 0.27 & 1828.00 & 15627.02 & 1.00 & 0.85 & 0.85 \\
26580 & 103592 & 1996 & 100.47 & 0.04 & 10099.00 & 92875.55 & 0.99 & 0.92 & 0.92 \\
14994 & 101930 & 1996 & 798.39 & 0.17 & 78969.00 & 778840.02 & 1.01 & 0.98 & 0.99 \\
27857 & 105335 & 1996 & 81.85 & 0.19 & 8357.00 & 75936.09 & 0.98 & 0.93 & 0.91 \\
13349 & 101729 & 1996 & 292.18 & 0.17 & 31885.00 & 247673.06 & 0.92 & 0.85 & 0.78 \\
28273 & 105400 & 1996 & 2.28 & 0.32 & 228.00 & 1992.20 & 1.00 & 0.87 & 0.87 \\
28244 & 105399 & 1996 & 13.71 & 0.30 & 1371.00 & 12180.39 & 1.00 & 0.89 & 0.89 \\
14093 & 101802 & 1996 & 265.82 & 0.21 & 22482.00 & 237726.49 & 1.18 & 0.89 & 1.06 \\
48065 & 235413 & 1996 & 36.42 & 0.17 & 3642.00 & 33949.74 & 1.00 & 0.93 & 0.93 \\
27871 & 105336 & 1996 & 33.73 & 0.30 & 3373.00 & 32330.92 & 1.00 & 0.96 & 0.96 \\
27891 & 105342 & 1996 & 39.92 & 0.03 & 4210.00 & 36562.57 & 0.95 & 0.92 & 0.87 \\
27896 & 105343 & 1996 & 147.65 & 0.21 & 15276.00 & 147738.55 & 0.97 & 1.00 & 0.97 \\
48137 & 240027 & 1996 & 38.51 & 0.42 & 2647.00 & 27183.90 & 1.45 & 0.71 & 1.03 \\
14018 & 101800 & 1996 & 516.86 & 0.12 & 49688.00 & 417699.96 & 1.04 & 0.81 & 0.84 \\
27922 & 105352 & 1996 & 170.80 & 0.16 & 15339.00 & 177921.72 & 1.11 & 1.04 & 1.16 \\
27928 & 105353 & 1996 & 9.74 & 0.22 & 974.00 & 9453.23 & 1.00 & 0.97 & 0.97 \\
13378 & 101736 & 1996 & 28.59 & 0.21 & 2858.00 & 27859.44 & 1.00 & 0.97 & 0.97 \\
27849 & 105333 & 1996 & 9.00 & 0.20 & 898.00 & 8238.92 & 1.00 & 0.92 & 0.92 \\
27942 & 105354 & 1996 & 104.98 & 0.15 & 9128.00 & 81798.35 & 1.15 & 0.78 & 0.90 \\
27820 & 105332 & 1996 & 98.80 & 0.28 & 9909.00 & 98458.56 & 1.00 & 1.00 & 0.99 \\
28528 & 105436 & 1996 & 75.63 & 0.23 & 7563.00 & 71856.06 & 1.00 & 0.95 & 0.95 \\
14224 & 101834 & 1996 & 19.77 & 0.27 & 1885.00 & 19964.06 & 1.05 & 1.01 & 1.06 \\
14193 & 101820 & 1996 & 229.14 & 0.14 & 28856.00 & 280558.19 & 0.79 & 1.22 & 0.97 \\
14161 & 101819 & 1996 & 158.81 & 0.20 & 14198.00 & 137269.55 & 1.12 & 0.86 & 0.97 \\
27767 & 105322 & 1996 & 27.29 & 0.13 & 2729.00 & 23928.46 & 1.00 & 0.88 & 0.88 \\
13312 & 101723 & 1996 & 24.16 & 0.26 & 2415.00 & 20901.40 & 1.00 & 0.87 & 0.87 \\
28484 & 105427 & 1996 & 57.98 & 0.15 & 5278.00 & 47099.96 & 1.10 & 0.81 & 0.89 \\
27777 & 105326 & 1996 & 28.30 & 0.22 & 2635.00 & 28441.89 & 1.07 & 1.01 & 1.08 \\
28455 & 105426 & 1996 & 187.30 & 0.22 & 16318.00 & 166089.34 & 1.15 & 0.89 & 1.02 \\
27785 & 105327 & 1996 & 20.66 & 0.31 & 1683.00 & 16111.70 & 1.23 & 0.78 & 0.96 \\
13326 & 101728 & 1996 & 89.20 & 0.18 & 9528.00 & 89326.03 & 0.94 & 1.00 & 0.94 \\
14128 & 101805 & 1996 & 1784.10 & 0.01 & 181663.00 & 1755650.49 & 0.98 & 0.98 & 0.97 \\
28426 & 105424 & 1996 & 828.27 & 0.22 & 48208.00 & 467422.02 & 1.72 & 0.56 & 0.97 \\
27791 & 105331 & 1996 & 2.31 & 0.19 & 232.00 & 2100.89 & 1.00 & 0.91 & 0.91 \\
28333 & 105416 & 1996 & 10.90 & 0.04 & 918.00 & 8475.89 & 1.19 & 0.78 & 0.92 \\
14111 & 101804 & 1996 & 305.53 & 0.08 & 28537.00 & 257799.94 & 1.07 & 0.84 & 0.90 \\
14239 & 101835 & 1996 & 477.46 & 0.23 & 44912.00 & 458768.12 & 1.06 & 0.96 & 1.02 \\
27947 & 105358 & 1996 & 243.98 & 0.28 & 20723.00 & 195992.38 & 1.18 & 0.80 & 0.95 \\
28231 & 105397 & 1996 & 37.10 & 0.34 & 3718.00 & 34085.71 & 1.00 & 0.92 & 0.92 \\
48253 & 240057 & 1996 & 155.09 & 0.16 & 14557.00 & 128421.11 & 1.07 & 0.83 & 0.88 \\
13549 & 101743 & 1996 & 4070.65 & 0.28 & 393744.00 & 4031576.89 & 1.03 & 0.99 & 1.02 \\
13727 & 101759 & 1996 & 95.84 & -0.04 & 9684.00 & 90593.06 & 0.99 & 0.95 & 0.94 \\
48343 & 240065 & 1996 & 277.37 & 0.20 & 26905.00 & 262450.47 & 1.03 & 0.95 & 0.98 \\
28089 & 105381 & 1996 & 29.70 & 0.16 & 2849.00 & 28002.54 & 1.04 & 0.94 & 0.98 \\
28091 & 105382 & 1996 & 30.61 & 0.19 & 2698.00 & 26917.49 & 1.13 & 0.88 & 1.00 \\
28167 & 105390 & 1996 & 368.91 & 0.02 & 37800.00 & 339193.95 & 0.98 & 0.92 & 0.90 \\
28166 & 105389 & 1996 & 151.05 & 0.01 & 15523.00 & 138158.55 & 0.97 & 0.91 & 0.89 \\
28165 & 105387 & 1996 & 147.50 & 0.13 & 13512.00 & 125249.60 & 1.09 & 0.85 & 0.93 \\
28120 & 105383 & 1996 & 46.20 & 0.18 & 4606.00 & 40599.69 & 1.00 & 0.88 & 0.88 \\
28134 & 105384 & 1996 & 54.67 & 0.20 & 4932.00 & 56567.11 & 1.11 & 1.03 & 1.15 \\
13613 & 101748 & 1996 & 12.56 & 0.21 & 1256.00 & 12138.86 & 1.00 & 0.97 & 0.97 \\
28069 & 105372 & 1996 & 45.79 & 0.22 & 4289.00 & 42221.08 & 1.07 & 0.92 & 0.98 \\
13800 & 101764 & 1996 & 280.99 & 0.27 & 28101.00 & 228018.39 & 1.00 & 0.81 & 0.81 \\
13836 & 101769 & 1996 & 2714.51 & 0.16 & 222979.00 & 2229913.21 & 1.22 & 0.82 & 1.00 \\
13974 & 101794 & 1996 & 262.37 & 0.11 & 26237.00 & 245880.41 & 1.00 & 0.94 & 0.94 \\
28223 & 105394 & 1996 & 190.50 & 0.15 & 19012.00 & 183087.37 & 1.00 & 0.96 & 0.96 \\
28204 & 105393 & 1996 & 15.92 & 0.23 & 1591.00 & 14826.07 & 1.00 & 0.93 & 0.93 \\
13951 & 101789 & 1996 & 515.66 & 0.15 & 41245.00 & 450045.00 & 1.25 & 0.87 & 1.09 \\
48176 & 240040 & 1996 & 708.10 & 0.30 & 65049.00 & 617650.87 & 1.09 & 0.87 & 0.95 \\
13932 & 101788 & 1996 & 253.07 & 0.22 & 19049.00 & 216481.27 & 1.33 & 0.86 & 1.14 \\
13913 & 101787 & 1996 & 558.46 & 0.19 & 46103.00 & 521559.65 & 1.21 & 0.93 & 1.13 \\
48208 & 240051 & 1996 & 171.41 & 0.29 & 17220.00 & 161851.32 & 1.00 & 0.94 & 0.94 \\
13880 & 101785 & 1996 & 1107.24 & 0.20 & 110724.00 & 1024919.77 & 1.00 & 0.93 & 0.93 \\
13863 & 101781 & 1996 & 915.13 & 0.14 & 91513.00 & 756553.65 & 1.00 & 0.83 & 0.83 \\
27979 & 105364 & 1996 & 45.33 & 0.09 & 4300.00 & 38481.87 & 1.05 & 0.85 & 0.89 \\
28008 & 105366 & 1996 & 22.40 & 0.08 & 2205.00 & 18992.56 & 1.02 & 0.85 & 0.86 \\
28011 & 105369 & 1996 & 225.25 & 0.23 & 20891.00 & 202870.40 & 1.08 & 0.90 & 0.97 \\
13479 & 101741 & 1996 & 1100.54 & 0.26 & 103318.00 & 989068.44 & 1.07 & 0.90 & 0.96 \\
28040 & 105370 & 1996 & 78.76 & 0.12 & 7652.00 & 76520.17 & 1.03 & 0.97 & 1.00 \\
13510 & 101742 & 1996 & 2061.32 & 0.18 & 200235.00 & 1672591.96 & 1.03 & 0.81 & 0.84 \\
14981 & 101926 & 1996 & 248.71 & 0.26 & 21584.00 & 203456.62 & 1.15 & 0.82 & 0.94 \\
28535 & 105437 & 1996 & 1524.36 & 0.18 & 149622.00 & 1430654.03 & 1.02 & 0.94 & 0.96 \\
27749 & 105321 & 1996 & 25.15 & -0.04 & 2359.00 & 20461.34 & 1.07 & 0.81 & 0.87 \\
28565 & 105444 & 1996 & 7.90 & 0.17 & 814.00 & 8142.56 & 0.97 & 1.03 & 1.00 \\
14801 & 101914 & 1996 & 43.70 & 0.21 & 4360.00 & 42298.10 & 1.00 & 0.97 & 0.97 \\
47618 & 215696 & 1996 & 127.59 & 0.03 & 9683.00 & 94473.10 & 1.32 & 0.74 & 0.98 \\
14769 & 101913 & 1996 & 39.50 & 0.20 & 3813.00 & 37774.21 & 1.04 & 0.96 & 0.99 \\
14737 & 101912 & 1996 & 2300.08 & 0.21 & 230008.00 & 1932148.66 & 1.00 & 0.84 & 0.84 \\
12996 & 101618 & 1996 & 423.48 & 0.19 & 42606.00 & 354184.82 & 0.99 & 0.84 & 0.83 \\
14703 & 101911 & 1996 & 252.16 & 0.31 & 25216.00 & 243748.51 & 1.00 & 0.97 & 0.97 \\
14669 & 101908 & 1996 & 7.88 & 0.16 & 665.00 & 7001.63 & 1.18 & 0.89 & 1.05 \\
14655 & 101906 & 1996 & 14.45 & 0.41 & 1458.00 & 12536.23 & 0.99 & 0.87 & 0.86 \\
14625 & 101903 & 1996 & 1495.12 & 0.21 & 149549.00 & 1216960.38 & 1.00 & 0.81 & 0.81 \\
27501 & 105283 & 1996 & 4.51 & 0.15 & 364.00 & 3203.07 & 1.24 & 0.71 & 0.88 \\
13031 & 101622 & 1996 & 1262.01 & 0.19 & 126088.00 & 1127630.33 & 1.00 & 0.89 & 0.89 \\
27512 & 105284 & 1996 & 1.15 & -0.18 & 115.00 & 1086.38 & 1.00 & 0.94 & 0.94 \\
27529 & 105286 & 1996 & 124.73 & 0.06 & 7569.00 & 66463.42 & 1.65 & 0.53 & 0.88 \\
28581 & 105448 & 1996 & 20.60 & 0.36 & 2062.00 & 19057.60 & 1.00 & 0.93 & 0.92 \\
28610 & 105450 & 1996 & 7.61 & 0.13 & 795.00 & 6811.71 & 0.96 & 0.89 & 0.86 \\
47455 & 211485 & 1996 & 290.15 & 0.20 & 23337.00 & 255846.36 & 1.24 & 0.88 & 1.10 \\
14949 & 101925 & 1996 & 257.15 & 0.21 & 23390.00 & 245409.42 & 1.10 & 0.95 & 1.05 \\
27334 & 105269 & 1996 & 133.22 & 0.22 & 13123.00 & 130493.53 & 1.02 & 0.98 & 0.99 \\
27367 & 105275 & 1996 & 125.08 & 0.17 & 12786.00 & 106573.28 & 0.98 & 0.85 & 0.83 \\
47467 & 212027 & 1996 & 16.71 & 0.08 & 1416.00 & 13484.01 & 1.18 & 0.81 & 0.95 \\
14908 & 101922 & 1996 & 70.35 & 0.18 & 11396.00 & 115584.30 & 0.62 & 1.64 & 1.01 \\
12946 & 101616 & 1996 & 10589.14 & 0.24 & 858767.00 & 8232084.55 & 1.23 & 0.78 & 0.96 \\
14871 & 101919 & 1996 & 311.53 & 0.22 & 24705.00 & 273025.35 & 1.26 & 0.88 & 1.11 \\
27397 & 105276 & 1996 & 197.85 & 0.29 & 19642.00 & 175481.70 & 1.01 & 0.89 & 0.89 \\
27426 & 105278 & 1996 & 5.44 & 0.28 & 569.00 & 5441.75 & 0.96 & 1.00 & 0.96 \\
27456 & 105279 & 1996 & 46.07 & 0.20 & 4290.00 & 43842.28 & 1.07 & 0.95 & 1.02 \\
47535 & 212658 & 1996 & 1021.04 & 0.25 & 101928.00 & 919640.22 & 1.00 & 0.90 & 0.90 \\
14827 & 101916 & 1996 & 163.50 & 0.30 & 16242.00 & 136656.96 & 1.01 & 0.84 & 0.84 \\
27462 & 105280 & 1996 & 62.07 & 0.24 & 4861.00 & 52626.98 & 1.28 & 0.85 & 1.08 \\
36493 & 106541 & 1996 & 5.78 & 0.23 & 476.00 & 4761.60 & 1.21 & 0.82 & 1.00 \\
27471 & 105281 & 1996 & 43.19 & 0.36 & 5708.00 & 55350.35 & 0.76 & 1.28 & 0.97 \\
27538 & 105287 & 1996 & 133.25 & 0.22 & 7428.00 & 72697.51 & 1.79 & 0.55 & 0.98 \\
14400 & 101854 & 1996 & 2047.46 & 0.39 & 204133.00 & 1732018.03 & 1.00 & 0.85 & 0.85 \\
27694 & 105311 & 1996 & 26.95 & 0.06 & 2606.00 & 24927.80 & 1.03 & 0.92 & 0.96 \\
28548 & 105438 & 1996 & 100.64 & 0.23 & 8531.00 & 79523.62 & 1.18 & 0.79 & 0.93 \\
14384 & 101853 & 1996 & 353.97 & 0.31 & 35457.00 & 285294.54 & 1.00 & 0.81 & 0.80 \\
14350 & 101851 & 1996 & 1007.62 & 0.29 & 100660.00 & 942165.60 & 1.00 & 0.94 & 0.94 \\
47916 & 222809 & 1996 & 132.00 & 0.20 & 13200.00 & 121064.53 & 1.00 & 0.92 & 0.92 \\
13262 & 101714 & 1996 & 30.30 & 0.16 & 3033.00 & 28580.59 & 1.00 & 0.94 & 0.94 \\
14325 & 101850 & 1996 & 90.98 & 0.28 & 9102.00 & 89094.80 & 1.00 & 0.98 & 0.98 \\
13269 & 101716 & 1996 & 36.40 & 0.17 & 3584.00 & 38125.95 & 1.02 & 1.05 & 1.06 \\
14313 & 101849 & 1996 & 24.38 & 0.15 & 2482.00 & 22305.11 & 0.98 & 0.91 & 0.90 \\
14272 & 101842 & 1996 & 898.64 & 0.25 & 82946.00 & 854035.42 & 1.08 & 0.95 & 1.03 \\
13278 & 101717 & 1996 & 41.46 & 0.22 & 3915.00 & 40645.23 & 1.06 & 0.98 & 1.04 \\
27660 & 105309 & 1996 & 397.50 & 0.39 & 26613.00 & 264339.66 & 1.49 & 0.67 & 0.99 \\
13045 & 101623 & 1996 & 1858.50 & 0.22 & 185264.00 & 1695581.93 & 1.00 & 0.91 & 0.92 \\
14444 & 101858 & 1996 & 136.58 & 0.33 & 13641.00 & 125199.02 & 1.00 & 0.92 & 0.92 \\
47808 & 222027 & 1996 & 120.77 & 0.19 & 10900.00 & 110749.25 & 1.11 & 0.92 & 1.02 \\
27557 & 105291 & 1996 & 187.24 & 0.23 & 16240.00 & 137090.38 & 1.15 & 0.73 & 0.84 \\
13068 & 101626 & 1996 & 758.96 & 0.14 & 78828.00 & 632112.07 & 0.96 & 0.83 & 0.80 \\
14551 & 101876 & 1996 & 200.23 & 0.26 & 20031.00 & 189283.32 & 1.00 & 0.95 & 0.94 \\
47741 & 221051 & 1996 & 4045.90 & 0.08 & 420535.00 & 3691623.89 & 0.96 & 0.91 & 0.88 \\
14509 & 101871 & 1996 & 391.72 & 0.22 & 38375.00 & 383607.27 & 1.02 & 0.98 & 1.00 \\
13122 & 101668 & 1996 & 70.15 & 0.21 & 7015.00 & 64495.11 & 1.00 & 0.92 & 0.92 \\
13152 & 101681 & 1996 & 267.90 & 0.12 & 26795.00 & 218374.40 & 1.00 & 0.82 & 0.81 \\
27602 & 105303 & 1996 & 176.38 & 0.22 & 16361.00 & 169626.96 & 1.08 & 0.96 & 1.04 \\
13165 & 101698 & 1996 & 251.80 & 0.18 & 25178.00 & 238649.91 & 1.00 & 0.95 & 0.95 \\
47794 & 221485 & 1996 & 73.47 & 0.21 & 7303.00 & 71822.57 & 1.01 & 0.98 & 0.98 \\
14465 & 101861 & 1996 & 540.65 & 0.30 & 49762.00 & 490886.40 & 1.09 & 0.91 & 0.99 \\
13178 & 101703 & 1996 & 30875.47 & 0.19 & 2487259.00 & 24873651.44 & 1.24 & 0.81 & 1.00 \\
28558 & 105443 & 1996 & 45.47 & 0.18 & 4549.00 & 39316.94 & 1.00 & 0.86 & 0.86 \\
12440 & 101541 & 1996 & 75.16 & 0.19 & 7516.00 & 73773.75 & 1.00 & 0.98 & 0.98 \\
16623 & 102166 & 1996 & 220.90 & 0.19 & 27506.00 & 257177.04 & 0.80 & 1.16 & 0.93 \\
19019 & 102544 & 1996 & 419.00 & 0.27 & 41724.00 & 411918.25 & 1.00 & 0.98 & 0.99 \\
18987 & 102540 & 1996 & 8.43 & 0.18 & 813.00 & 7888.75 & 1.04 & 0.94 & 0.97 \\
18955 & 102531 & 1996 & 15.96 & 0.15 & 1479.00 & 14598.97 & 1.08 & 0.91 & 0.99 \\
18942 & 102529 & 1996 & 103.22 & -0.07 & 10331.00 & 95823.88 & 1.00 & 0.93 & 0.93 \\
30134 & 105702 & 1996 & 95.40 & 0.00 & 9524.00 & 92626.67 & 1.00 & 0.97 & 0.97 \\
30121 & 105701 & 1996 & 48.53 & 0.32 & 4852.00 & 42951.01 & 1.00 & 0.89 & 0.89 \\
18923 & 102528 & 1996 & 106.98 & 0.02 & 10336.00 & 100460.72 & 1.04 & 0.94 & 0.97 \\
18907 & 102527 & 1996 & 200.84 & 0.24 & 19963.00 & 183212.04 & 1.01 & 0.91 & 0.92 \\
44394 & 109300 & 1996 & 297.72 & 0.25 & 24924.00 & 270738.29 & 1.19 & 0.91 & 1.09 \\
18876 & 102525 & 1996 & 184.07 & 0.19 & 17099.00 & 164288.50 & 1.08 & 0.89 & 0.96 \\
30111 & 105700 & 1996 & 484.19 & 0.24 & 48425.00 & 465683.68 & 1.00 & 0.96 & 0.96 \\
30103 & 105694 & 1996 & 13.80 & 0.04 & 1341.00 & 12544.22 & 1.03 & 0.91 & 0.94 \\
18814 & 102523 & 1996 & 456.11 & 0.13 & 45646.00 & 429921.34 & 1.00 & 0.94 & 0.94 \\
18797 & 102522 & 1996 & 329.60 & 0.18 & 32198.00 & 287141.45 & 1.02 & 0.87 & 0.89 \\
11362 & 101398 & 1996 & 147.90 & 0.31 & 15863.00 & 138566.20 & 0.93 & 0.94 & 0.87 \\
25625 & 103498 & 1996 & 89.60 & 0.07 & 7342.00 & 73423.47 & 1.22 & 0.82 & 1.00 \\
18771 & 102508 & 1996 & 218.64 & 0.26 & 21926.00 & 212817.58 & 1.00 & 0.97 & 0.97 \\
19059 & 102546 & 1996 & 82.56 & 0.21 & 4830.00 & 46702.33 & 1.71 & 0.57 & 0.97 \\
19075 & 102548 & 1996 & 226.01 & 0.23 & 22664.00 & 218458.73 & 1.00 & 0.97 & 0.96 \\
11245 & 101379 & 1996 & 352.46 & 0.23 & 31765.00 & 288560.51 & 1.11 & 0.82 & 0.91 \\
19266 & 102578 & 1996 & 113.48 & 0.21 & 11351.00 & 110940.62 & 1.00 & 0.98 & 0.98 \\
11256 & 101380 & 1996 & 326.33 & 0.20 & 31174.00 & 291257.83 & 1.05 & 0.89 & 0.93 \\
19250 & 102575 & 1996 & 214.35 & 0.18 & 21438.00 & 199303.56 & 1.00 & 0.93 & 0.93 \\
19218 & 102570 & 1996 & 245.51 & 0.11 & 24395.00 & 229156.72 & 1.01 & 0.93 & 0.94 \\
25550 & 103496 & 1996 & 371.10 & 0.22 & 36228.00 & 362298.51 & 1.02 & 0.98 & 1.00 \\
11280 & 101390 & 1996 & 2791.24 & 0.23 & 279083.00 & 2582947.27 & 1.00 & 0.93 & 0.93 \\
25581 & 103497 & 1996 & 18.10 & 0.10 & 1931.00 & 17488.97 & 0.94 & 0.97 & 0.91 \\
11312 & 101393 & 1996 & 228.80 & 0.22 & 23622.00 & 237945.33 & 0.97 & 1.04 & 1.01 \\
19199 & 102563 & 1996 & 445.91 & 0.08 & 46907.00 & 387120.75 & 0.95 & 0.87 & 0.83 \\
19175 & 102559 & 1996 & 95.09 & 0.13 & 9730.00 & 80642.95 & 0.98 & 0.85 & 0.83 \\
19141 & 102551 & 1996 & 36.83 & 0.28 & 3618.00 & 34854.17 & 1.02 & 0.95 & 0.96 \\
19103 & 102549 & 1996 & 110.08 & 0.24 & 8454.00 & 84196.44 & 1.30 & 0.76 & 1.00 \\
19067 & 102547 & 1996 & 119.57 & 0.22 & 11972.00 & 113328.44 & 1.00 & 0.95 & 0.95 \\
11430 & 101402 & 1996 & 12.70 & 0.06 & 1547.00 & 11945.52 & 0.82 & 0.94 & 0.77 \\
18691 & 102503 & 1996 & 268.20 & 0.12 & 26823.00 & 249691.57 & 1.00 & 0.93 & 0.93 \\
18560 & 102483 & 1996 & 87.15 & 0.17 & 8975.00 & 88068.14 & 0.97 & 1.01 & 0.98 \\
18550 & 102482 & 1996 & 116.37 & 0.28 & 11625.00 & 101000.08 & 1.00 & 0.87 & 0.87 \\
29997 & 105676 & 1996 & 360.52 & 0.40 & 21842.00 & 218713.23 & 1.65 & 0.61 & 1.00 \\
29985 & 105665 & 1996 & 75.52 & 0.17 & 7552.00 & 70474.83 & 1.00 & 0.93 & 0.93 \\
18541 & 102474 & 1996 & 274.61 & 0.17 & 27971.00 & 264759.91 & 0.98 & 0.96 & 0.95 \\
18522 & 102470 & 1996 & 668.81 & 0.23 & 70873.00 & 596082.37 & 0.94 & 0.89 & 0.84 \\
18468 & 102462 & 1996 & 8.55 & 0.44 & 855.00 & 8305.24 & 1.00 & 0.97 & 0.97 \\
11556 & 101430 & 1996 & 115.90 & 0.19 & 8784.00 & 85914.81 & 1.32 & 0.74 & 0.98 \\
18451 & 102461 & 1996 & 5938.51 & 0.10 & 587215.00 & 4917514.14 & 1.01 & 0.83 & 0.84 \\
18425 & 102452 & 1996 & 113.99 & 0.16 & 9315.00 & 94991.15 & 1.22 & 0.83 & 1.02 \\
18395 & 102447 & 1996 & 415.60 & 0.21 & 41620.00 & 370632.41 & 1.00 & 0.89 & 0.89 \\
39141 & 107611 & 1996 & 153.23 & 0.25 & 15323.00 & 143962.92 & 1.00 & 0.94 & 0.94 \\
18351 & 102446 & 1996 & 7.39 & 0.19 & 738.00 & 7028.31 & 1.00 & 0.95 & 0.95 \\
18567 & 102486 & 1996 & 41.91 & 0.07 & 4182.00 & 34882.97 & 1.00 & 0.83 & 0.83 \\
18753 & 102507 & 1996 & 503.53 & 0.22 & 50475.00 & 498052.43 & 1.00 & 0.99 & 0.99 \\
18574 & 102489 & 1996 & 29.62 & 0.28 & 2962.00 & 29192.38 & 1.00 & 0.99 & 0.99 \\
11453 & 101414 & 1996 & 55.63 & 0.26 & 5563.00 & 54072.76 & 1.00 & 0.97 & 0.97 \\
18680 & 102502 & 1996 & 675.94 & 0.28 & 71511.00 & 668746.04 & 0.95 & 0.99 & 0.94 \\
11480 & 101422 & 1996 & 31.00 & 0.22 & 2725.00 & 29729.23 & 1.14 & 0.96 & 1.09 \\
30096 & 105686 & 1996 & 455.70 & 0.33 & 45664.00 & 378008.24 & 1.00 & 0.83 & 0.83 \\
30094 & 105685 & 1996 & 20.96 & -0.02 & 1855.00 & 18560.21 & 1.13 & 0.89 & 1.00 \\
30080 & 105682 & 1996 & 30.54 & -0.03 & 2360.00 & 22095.63 & 1.29 & 0.72 & 0.94 \\
30075 & 105681 & 1996 & 241.87 & -0.01 & 24452.00 & 193949.83 & 0.99 & 0.80 & 0.79 \\
18651 & 102500 & 1996 & 642.94 & 0.10 & 64142.00 & 589157.71 & 1.00 & 0.92 & 0.92 \\
25663 & 103500 & 1996 & 132.30 & 0.03 & 16742.00 & 134291.83 & 0.79 & 1.02 & 0.80 \\
11502 & 101425 & 1996 & 16.66 & 0.22 & 1728.00 & 15126.65 & 0.96 & 0.91 & 0.88 \\
30042 & 105679 & 1996 & 5.82 & -0.02 & 639.00 & 5607.51 & 0.91 & 0.96 & 0.88 \\
30014 & 105678 & 1996 & 3.79 & 0.04 & 284.00 & 2587.45 & 1.33 & 0.68 & 0.91 \\
30005 & 105677 & 1996 & 1.33 & 0.14 & 133.00 & 1308.15 & 1.00 & 0.98 & 0.98 \\
18586 & 102490 & 1996 & 53.28 & 0.22 & 5328.00 & 52491.38 & 1.00 & 0.99 & 0.99 \\
11215 & 101376 & 1996 & 194.40 & 0.11 & 19436.00 & 188823.19 & 1.00 & 0.97 & 0.97 \\
25493 & 103494 & 1996 & 348.90 & 0.23 & 28759.00 & 287609.72 & 1.21 & 0.82 & 1.00 \\
20150 & 102671 & 1996 & 72.49 & 0.22 & 8414.00 & 71922.97 & 0.86 & 0.99 & 0.85 \\
10954 & 101356 & 1996 & 284.12 & 0.21 & 28975.00 & 264621.27 & 0.98 & 0.93 & 0.91 \\
20139 & 102669 & 1996 & 29.74 & 0.22 & 3055.00 & 30194.97 & 0.97 & 1.02 & 0.99 \\
20106 & 102667 & 1996 & 4203.99 & 0.18 & 420399.00 & 3853776.71 & 1.00 & 0.92 & 0.92 \\
10969 & 101357 & 1996 & 134.71 & 0.17 & 23304.00 & 194646.70 & 0.58 & 1.44 & 0.84 \\
20045 & 102664 & 1996 & 1409.83 & 0.25 & 136255.00 & 1148271.21 & 1.03 & 0.81 & 0.84 \\
20011 & 102663 & 1996 & 3291.80 & 0.14 & 324536.00 & 3125382.41 & 1.01 & 0.95 & 0.96 \\
19976 & 102660 & 1996 & 539.67 & 0.28 & 52529.00 & 467479.52 & 1.03 & 0.87 & 0.89 \\
30222 & 105718 & 1996 & 137.74 & 0.24 & 13443.00 & 132837.33 & 1.02 & 0.96 & 0.99 \\
30207 & 105716 & 1996 & 638.16 & 0.27 & 64589.00 & 640962.83 & 0.99 & 1.00 & 0.99 \\
19946 & 102659 & 1996 & 4363.42 & 0.17 & 436342.00 & 4103318.43 & 1.00 & 0.94 & 0.94 \\
25406 & 103483 & 1996 & 307.71 & 0.20 & 30779.00 & 285144.02 & 1.00 & 0.93 & 0.93 \\
30201 & 105708 & 1996 & 105.49 & 0.16 & 10086.00 & 86826.01 & 1.05 & 0.82 & 0.86 \\
25422 & 103484 & 1996 & 89.45 & 0.15 & 8847.00 & 85302.68 & 1.01 & 0.95 & 0.96 \\
10893 & 101345 & 1996 & 682.15 & 0.15 & 70510.00 & 634324.03 & 0.97 & 0.93 & 0.90 \\
25396 & 103482 & 1996 & 8.23 & 0.26 & 843.00 & 7453.56 & 0.98 & 0.91 & 0.88 \\
20467 & 102757 & 1996 & 11837.22 & 0.11 & 1184286.00 & 11432055.33 & 1.00 & 0.97 & 0.97 \\
20445 & 102744 & 1996 & 558.09 & 0.30 & 49998.00 & 463945.37 & 1.12 & 0.83 & 0.93 \\
10796 & 101331 & 1996 & 96.37 & 0.25 & 7862.00 & 80768.94 & 1.23 & 0.84 & 1.03 \\
10826 & 101334 & 1996 & 142.86 & 0.49 & 10248.00 & 88550.32 & 1.39 & 0.62 & 0.86 \\
20415 & 102737 & 1996 & 211.66 & 0.22 & 18438.00 & 186465.59 & 1.15 & 0.88 & 1.01 \\
20383 & 102733 & 1996 & 4175.42 & 0.21 & 401319.00 & 3661734.74 & 1.04 & 0.88 & 0.91 \\
20366 & 102728 & 1996 & 447.23 & 0.19 & 46134.00 & 422410.40 & 0.97 & 0.94 & 0.92 \\
20332 & 102716 & 1996 & 1390.71 & 0.21 & 132468.00 & 1234083.34 & 1.05 & 0.89 & 0.93 \\
30259 & 105721 & 1996 & 49.20 & 0.26 & 4921.00 & 44086.58 & 1.00 & 0.90 & 0.90 \\
20286 & 102709 & 1996 & 2.80 & 0.06 & 279.00 & 2766.97 & 1.00 & 0.99 & 0.99 \\
25372 & 103479 & 1996 & 47.90 & 0.05 & 4684.00 & 46258.16 & 1.02 & 0.97 & 0.99 \\
25376 & 103481 & 1996 & 70.33 & -0.02 & 6894.00 & 69032.89 & 1.02 & 0.98 & 1.00 \\
20251 & 102696 & 1996 & 514.15 & 0.23 & 51415.00 & 507578.95 & 1.00 & 0.99 & 0.99 \\
10858 & 101340 & 1996 & 7165.37 & 0.20 & 716537.00 & 6020866.64 & 1.00 & 0.84 & 0.84 \\
30229 & 105719 & 1996 & 26.90 & 0.11 & 2682.00 & 25647.64 & 1.00 & 0.95 & 0.96 \\
19918 & 102655 & 1996 & 1467.74 & 0.24 & 140934.00 & 1258958.57 & 1.04 & 0.86 & 0.89 \\
25430 & 103487 & 1996 & 33.12 & 0.38 & 3375.00 & 31832.54 & 0.98 & 0.96 & 0.94 \\
19551 & 102624 & 1996 & 452.80 & 0.23 & 39568.00 & 419760.69 & 1.14 & 0.93 & 1.06 \\
19545 & 102614 & 1996 & 184.03 & 0.19 & 17892.00 & 190732.39 & 1.03 & 1.04 & 1.07 \\
11154 & 101369 & 1996 & 623.37 & 0.19 & 60541.00 & 615447.84 & 1.03 & 0.99 & 1.02 \\
30170 & 105704 & 1996 & 53.65 & 0.45 & 5535.00 & 46857.48 & 0.97 & 0.87 & 0.85 \\
19538 & 102612 & 1996 & 108.48 & 0.20 & 9724.00 & 100900.57 & 1.12 & 0.93 & 1.04 \\
19505 & 102608 & 1996 & 104.64 & 0.36 & 10645.00 & 98642.28 & 0.98 & 0.94 & 0.93 \\
19488 & 102607 & 1996 & 1021.46 & 0.13 & 101739.00 & 1020452.34 & 1.00 & 1.00 & 1.00 \\
19465 & 102606 & 1996 & 6008.90 & 0.22 & 601414.00 & 5766865.44 & 1.00 & 0.96 & 0.96 \\
11192 & 101370 & 1996 & 10.46 & 0.18 & 1127.00 & 11274.07 & 0.93 & 1.08 & 1.00 \\
11200 & 101374 & 1996 & 6.30 & 0.15 & 607.00 & 6052.58 & 1.04 & 0.96 & 1.00 \\
19431 & 102601 & 1996 & 2906.71 & 0.20 & 290671.00 & 2771140.38 & 1.00 & 0.95 & 0.95 \\
11204 & 101375 & 1996 & 10.90 & 0.17 & 1092.00 & 10637.87 & 1.00 & 0.98 & 0.97 \\
19397 & 102600 & 1996 & 502.32 & 0.12 & 50232.00 & 498446.69 & 1.00 & 0.99 & 0.99 \\
19329 & 102597 & 1996 & 26.91 & 0.11 & 2250.00 & 21404.26 & 1.20 & 0.80 & 0.95 \\
19319 & 102591 & 1996 & 70.00 & 0.21 & 7048.00 & 60592.20 & 0.99 & 0.87 & 0.86 \\
19565 & 102628 & 1996 & 278.53 & 0.22 & 20579.00 & 224612.66 & 1.35 & 0.81 & 1.09 \\
25426 & 103485 & 1996 & 24.65 & 0.07 & 2314.00 & 21924.85 & 1.07 & 0.89 & 0.95 \\
30174 & 105705 & 1996 & 106.26 & 0.27 & 10039.00 & 96720.32 & 1.06 & 0.91 & 0.96 \\
19590 & 102635 & 1996 & 323.80 & 0.25 & 29820.00 & 264896.86 & 1.09 & 0.82 & 0.89 \\
19874 & 102654 & 1996 & 1440.19 & 0.20 & 142888.00 & 1369195.61 & 1.01 & 0.95 & 0.96 \\
30200 & 105707 & 1996 & 16.05 & 0.00 & 1605.00 & 15873.90 & 1.00 & 0.99 & 0.99 \\
11020 & 101360 & 1996 & 1081.56 & 0.13 & 114233.00 & 1062285.63 & 0.95 & 0.98 & 0.93 \\
19795 & 102652 & 1996 & 2488.82 & 0.26 & 248773.00 & 2128997.44 & 1.00 & 0.86 & 0.86 \\
19761 & 102651 & 1996 & 807.65 & 0.23 & 80745.00 & 795740.48 & 1.00 & 0.99 & 0.99 \\
19730 & 102650 & 1996 & 7818.13 & 0.20 & 781813.00 & 6436192.18 & 1.00 & 0.82 & 0.82 \\
11052 & 101364 & 1996 & 60.30 & 0.18 & 5701.00 & 54672.86 & 1.06 & 0.91 & 0.96 \\
11084 & 101367 & 1996 & 216.67 & 0.12 & 22324.00 & 225870.36 & 0.97 & 1.04 & 1.01 \\
19666 & 102645 & 1996 & 307.17 & 0.25 & 30410.00 & 279235.15 & 1.01 & 0.91 & 0.92 \\
19653 & 102641 & 1996 & 381.90 & 0.19 & 38183.00 & 377836.47 & 1.00 & 0.99 & 0.99 \\
11118 & 101368 & 1996 & 461.15 & 0.09 & 46084.00 & 455682.03 & 1.00 & 0.99 & 0.99 \\
19632 & 102639 & 1996 & 209.08 & 0.38 & 20908.00 & 175330.65 & 1.00 & 0.84 & 0.84 \\
19600 & 102636 & 1996 & 632.95 & 0.19 & 59997.00 & 600001.59 & 1.05 & 0.95 & 1.00 \\
19572 & 102633 & 1996 & 351.56 & 0.19 & 33277.00 & 331903.75 & 1.06 & 0.94 & 1.00 \\
25723 & 103520 & 1996 & 786.90 & 0.24 & 78690.00 & 785382.72 & 1.00 & 1.00 & 1.00 \\
17134 & 102258 & 1996 & 716.68 & 0.18 & 77613.00 & 800686.59 & 0.92 & 1.12 & 1.03 \\
12078 & 101497 & 1996 & 1865.48 & 0.17 & 186548.00 & 1561319.42 & 1.00 & 0.84 & 0.84 \\
12098 & 101503 & 1996 & 423.18 & 0.29 & 39548.00 & 371897.57 & 1.07 & 0.88 & 0.94 \\
45866 & 200151 & 1996 & 139.60 & 0.47 & 11198.00 & 95761.77 & 1.25 & 0.69 & 0.86 \\
45877 & 200153 & 1996 & 5.84 & 0.24 & 774.00 & 7479.26 & 0.75 & 1.28 & 0.97 \\
17102 & 102257 & 1996 & 539.71 & 0.23 & 45168.00 & 477709.20 & 1.19 & 0.89 & 1.06 \\
17071 & 102241 & 1996 & 203.28 & 0.11 & 20277.00 & 174072.17 & 1.00 & 0.86 & 0.86 \\
17036 & 102231 & 1996 & 687.46 & 0.21 & 67306.00 & 640856.31 & 1.02 & 0.93 & 0.95 \\
12121 & 101511 & 1996 & 111.83 & 0.11 & 11351.00 & 109981.23 & 0.99 & 0.98 & 0.97 \\
16964 & 102224 & 1996 & 1713.35 & 0.18 & 171335.00 & 1658414.60 & 1.00 & 0.97 & 0.97 \\
12155 & 101513 & 1996 & 116.81 & 0.20 & 11211.00 & 118976.65 & 1.04 & 1.02 & 1.06 \\
26288 & 103564 & 1996 & 1184.17 & 0.24 & 109013.00 & 1077057.31 & 1.09 & 0.91 & 0.99 \\
12186 & 101518 & 1996 & 576.16 & 0.24 & 57616.00 & 558960.22 & 1.00 & 0.97 & 0.97 \\
26306 & 103567 & 1996 & 2044.95 & 0.19 & 212472.00 & 1835572.79 & 0.96 & 0.90 & 0.86 \\
17144 & 102259 & 1996 & 352.03 & 0.18 & 36240.00 & 315566.79 & 0.97 & 0.90 & 0.87 \\
26338 & 103570 & 1996 & 14.29 & 0.16 & 1324.00 & 14919.84 & 1.08 & 1.04 & 1.13 \\
12061 & 101494 & 1996 & 303.05 & 0.20 & 30490.00 & 280960.15 & 0.99 & 0.93 & 0.92 \\
17187 & 102270 & 1996 & 208.88 & 0.20 & 24116.00 & 214004.48 & 0.87 & 1.02 & 0.89 \\
29572 & 105616 & 1996 & 5.54 & 0.04 & 560.00 & 5223.22 & 0.99 & 0.94 & 0.93 \\
26136 & 103544 & 1996 & 3530.86 & 0.21 & 353086.00 & 3332825.52 & 1.00 & 0.94 & 0.94 \\
26170 & 103545 & 1996 & 15700.01 & 0.24 & 1475789.00 & 14319694.11 & 1.06 & 0.91 & 0.97 \\
17306 & 102280 & 1996 & 2411.56 & 0.26 & 211295.00 & 2168691.18 & 1.14 & 0.90 & 1.03 \\
17288 & 102278 & 1996 & 363.90 & 0.26 & 36457.00 & 353387.70 & 1.00 & 0.97 & 0.97 \\
17248 & 102274 & 1996 & 400.29 & 0.30 & 28716.00 & 291182.38 & 1.39 & 0.73 & 1.01 \\
29232 & 105558 & 1996 & 2.00 & 0.15 & 254.00 & 2458.17 & 0.79 & 1.23 & 0.97 \\
45790 & 200140 & 1996 & 900.69 & 0.12 & 84972.00 & 691641.22 & 1.06 & 0.77 & 0.81 \\
29208 & 105544 & 1996 & 52.24 & 0.20 & 4791.00 & 41250.67 & 1.09 & 0.79 & 0.86 \\
26210 & 103546 & 1996 & 16738.11 & 0.26 & 1673811.00 & 15276244.41 & 1.00 & 0.91 & 0.91 \\
26241 & 103547 & 1996 & 5532.03 & 0.26 & 495120.00 & 4843481.24 & 1.12 & 0.88 & 0.98 \\
17158 & 102261 & 1996 & 1357.38 & 0.21 & 129433.00 & 1338955.48 & 1.05 & 0.99 & 1.03 \\
17380 & 102284 & 1996 & 245.22 & 0.15 & 23402.00 & 245989.01 & 1.05 & 1.00 & 1.05 \\
12211 & 101519 & 1996 & 94.26 & 0.12 & 9905.00 & 85228.68 & 0.95 & 0.90 & 0.86 \\
24589 & 103370 & 1996 & 74.12 & 0.29 & 6742.00 & 62648.44 & 1.10 & 0.85 & 0.93 \\
16721 & 102182 & 1996 & 108.67 & 0.20 & 10862.00 & 112392.84 & 1.00 & 1.03 & 1.03 \\
16717 & 102179 & 1996 & 25.13 & 0.22 & 2276.00 & 24415.19 & 1.10 & 0.97 & 1.07 \\
26517 & 103590 & 1996 & 219.17 & 0.05 & 21926.00 & 210059.81 & 1.00 & 0.96 & 0.96 \\
12355 & 101537 & 1996 & 374.16 & 0.17 & 39528.00 & 374829.65 & 0.95 & 1.00 & 0.95 \\
49184 & 240243 & 1996 & 433.45 & -0.03 & 43259.00 & 358479.36 & 1.00 & 0.83 & 0.83 \\
26549 & 103591 & 1996 & 89.87 & 0.17 & 8875.00 & 84467.09 & 1.01 & 0.94 & 0.95 \\
16687 & 102178 & 1996 & 248.16 & 0.31 & 23378.00 & 230534.11 & 1.06 & 0.93 & 0.99 \\
16654 & 102175 & 1996 & 48.20 & 0.22 & 4335.00 & 45129.13 & 1.11 & 0.94 & 1.04 \\
16644 & 102173 & 1996 & 40.23 & 0.20 & 3750.00 & 40398.63 & 1.07 & 1.00 & 1.08 \\
12406 & 101539 & 1996 & 622.35 & 0.20 & 59804.00 & 586075.54 & 1.04 & 0.94 & 0.98 \\
16873 & 102213 & 1996 & 310.08 & 0.26 & 31127.00 & 296716.55 & 1.00 & 0.96 & 0.95 \\
29178 & 105535 & 1996 & 154.62 & 0.23 & 14278.00 & 125622.26 & 1.08 & 0.81 & 0.88 \\
26371 & 103572 & 1996 & 32.08 & 0.17 & 3307.00 & 32314.36 & 0.97 & 1.01 & 0.98 \\
26382 & 103573 & 1996 & 241.25 & 0.23 & 23026.00 & 210410.21 & 1.05 & 0.87 & 0.91 \\
12217 & 101523 & 1996 & 375.20 & 0.19 & 37248.00 & 309055.95 & 1.01 & 0.82 & 0.83 \\
12226 & 101528 & 1996 & 69.25 & 0.27 & 7175.00 & 57747.88 & 0.97 & 0.83 & 0.80 \\
16843 & 102197 & 1996 & 73.07 & 0.24 & 7307.00 & 65152.60 & 1.00 & 0.89 & 0.89 \\
12241 & 101530 & 1996 & 1627.05 & 0.21 & 162676.00 & 1386711.53 & 1.00 & 0.85 & 0.85 \\
49263 & 240261 & 1996 & 256.25 & 0.16 & 25604.00 & 207267.99 & 1.00 & 0.81 & 0.81 \\
26429 & 103580 & 1996 & 291.76 & 0.18 & 26376.00 & 255923.71 & 1.11 & 0.88 & 0.97 \\
16814 & 102193 & 1996 & 417.40 & 0.22 & 41639.00 & 396927.53 & 1.00 & 0.95 & 0.95 \\
29194 & 105536 & 1996 & 580.14 & 0.28 & 67594.00 & 534152.42 & 0.86 & 0.92 & 0.79 \\
16771 & 102191 & 1996 & 37.35 & 0.21 & 3347.00 & 36982.59 & 1.12 & 0.99 & 1.10 \\
26480 & 103582 & 1996 & 15.90 & 0.11 & 1594.00 & 14378.52 & 1.00 & 0.90 & 0.90 \\
12029 & 101488 & 1996 & 151.34 & 0.14 & 14852.00 & 123911.42 & 1.02 & 0.82 & 0.83 \\
17401 & 102286 & 1996 & 35.55 & 0.13 & 3555.00 & 29241.64 & 1.00 & 0.82 & 0.82 \\
17416 & 102305 & 1996 & 208.43 & 0.21 & 22024.00 & 220327.23 & 0.95 & 1.06 & 1.00 \\
25890 & 103526 & 1996 & 2053.77 & 0.23 & 205377.00 & 1848598.46 & 1.00 & 0.90 & 0.90 \\
29760 & 105645 & 1996 & 117.44 & 0.13 & 11743.00 & 107274.38 & 1.00 & 0.91 & 0.91 \\
29750 & 105644 & 1996 & 110.28 & 0.24 & 11024.00 & 105216.02 & 1.00 & 0.95 & 0.95 \\
29728 & 105643 & 1996 & 149.12 & 0.26 & 14591.00 & 141844.36 & 1.02 & 0.95 & 0.97 \\
18077 & 102396 & 1996 & 36.04 & 0.26 & 3604.00 & 33506.55 & 1.00 & 0.93 & 0.93 \\
18039 & 102387 & 1996 & 17.57 & 0.14 & 1761.00 & 15359.88 & 1.00 & 0.87 & 0.87 \\
18004 & 102386 & 1996 & 203.83 & 0.17 & 20588.00 & 197051.35 & 0.99 & 0.97 & 0.96 \\
17990 & 102383 & 1996 & 8.11 & 0.20 & 804.00 & 8061.11 & 1.01 & 0.99 & 1.00 \\
11710 & 101457 & 1996 & 501.59 & 0.18 & 46505.00 & 467486.79 & 1.08 & 0.93 & 1.01 \\
17964 & 102377 & 1996 & 185.59 & 0.22 & 18559.00 & 176106.46 & 1.00 & 0.95 & 0.95 \\
17930 & 102376 & 1996 & 16.59 & 0.16 & 1659.00 & 13886.10 & 1.00 & 0.84 & 0.84 \\
45270 & 200005 & 1996 & 2.23 & 0.17 & 214.00 & 2182.37 & 1.04 & 0.98 & 1.02 \\
11744 & 101460 & 1996 & 7075.40 & 0.18 & 655225.00 & 6670055.76 & 1.08 & 0.94 & 1.02 \\
17900 & 102372 & 1996 & 4194.97 & 0.16 & 393393.00 & 3888098.27 & 1.07 & 0.93 & 0.99 \\
18117 & 102399 & 1996 & 11.51 & 0.18 & 1200.00 & 11265.67 & 0.96 & 0.98 & 0.94 \\
11777 & 101461 & 1996 & 878.57 & 0.21 & 82498.00 & 854501.05 & 1.06 & 0.97 & 1.04 \\
18126 & 102404 & 1996 & 452.90 & 0.22 & 39178.00 & 406791.62 & 1.16 & 0.90 & 1.04 \\
29818 & 105652 & 1996 & 170.90 & 0.23 & 17090.00 & 169553.20 & 1.00 & 0.99 & 0.99 \\
29939 & 105659 & 1996 & 39.48 & -0.02 & 3948.00 & 38818.01 & 1.00 & 0.98 & 0.98 \\
29924 & 105658 & 1996 & 298.70 & 0.19 & 30622.00 & 271624.11 & 0.98 & 0.91 & 0.89 \\
25755 & 103521 & 1996 & 1015.73 & 0.22 & 101573.00 & 1006522.53 & 1.00 & 0.99 & 0.99 \\
29919 & 105657 & 1996 & 30.00 & 0.18 & 1931.00 & 20200.59 & 1.55 & 0.67 & 1.05 \\
29890 & 105656 & 1996 & 295.60 & 0.12 & 29609.00 & 288466.57 & 1.00 & 0.98 & 0.97 \\
29861 & 105655 & 1996 & 217.77 & 0.40 & 15035.00 & 141692.70 & 1.45 & 0.65 & 0.94 \\
25789 & 103523 & 1996 & 2098.84 & 0.18 & 209884.00 & 1955291.18 & 1.00 & 0.93 & 0.93 \\
29844 & 105654 & 1996 & 77.71 & 0.41 & 5666.00 & 55778.29 & 1.37 & 0.72 & 0.98 \\
18267 & 102419 & 1996 & 296.83 & 0.19 & 29670.00 & 246444.73 & 1.00 & 0.83 & 0.83 \\
25823 & 103524 & 1996 & 30331.85 & 0.23 & 3033185.00 & 30341756.70 & 1.00 & 1.00 & 1.00 \\
25856 & 103525 & 1996 & 11426.42 & 0.23 & 1142641.00 & 11352837.48 & 1.00 & 0.99 & 0.99 \\
38981 & 107470 & 1996 & 2.29 & 0.28 & 229.00 & 1940.08 & 1.00 & 0.85 & 0.85 \\
11677 & 101456 & 1996 & 48.80 & 0.15 & 4715.00 & 47253.51 & 1.03 & 0.97 & 1.00 \\
18176 & 102414 & 1996 & 957.09 & 0.27 & 80431.00 & 805812.36 & 1.19 & 0.84 & 1.00 \\
29789 & 105647 & 1996 & 82.01 & 0.14 & 8206.00 & 79950.46 & 1.00 & 0.97 & 0.97 \\
17880 & 102371 & 1996 & 31.97 & 0.29 & 3167.00 & 29144.94 & 1.01 & 0.91 & 0.92 \\
17867 & 102367 & 1996 & 841.68 & 0.22 & 83648.00 & 792920.09 & 1.01 & 0.94 & 0.95 \\
17616 & 102321 & 1996 & 232.75 & 0.22 & 23275.00 & 207657.95 & 1.00 & 0.89 & 0.89 \\
17579 & 102319 & 1996 & 784.26 & 0.19 & 78426.00 & 731659.12 & 1.00 & 0.93 & 0.93 \\
17545 & 102318 & 1996 & 3984.19 & 0.20 & 398385.00 & 3396755.12 & 1.00 & 0.85 & 0.85 \\
11952 & 101473 & 1996 & 1432.27 & 0.26 & 142819.00 & 1331154.58 & 1.00 & 0.93 & 0.93 \\
11986 & 101476 & 1996 & 2615.10 & 0.19 & 238919.00 & 1971955.87 & 1.09 & 0.75 & 0.83 \\
17494 & 102314 & 1996 & 409.16 & 0.18 & 56729.00 & 532710.60 & 0.72 & 1.30 & 0.94 \\
17483 & 102313 & 1996 & 468.82 & 0.16 & 46962.00 & 440681.18 & 1.00 & 0.94 & 0.94 \\
26025 & 103535 & 1996 & 517.39 & 0.21 & 51739.00 & 514687.15 & 1.00 & 0.99 & 0.99 \\
26055 & 103536 & 1996 & 234.52 & 0.23 & 23525.00 & 235264.76 & 1.00 & 1.00 & 1.00 \\
17433 & 102306 & 1996 & 3385.45 & 0.26 & 327185.00 & 3227822.43 & 1.03 & 0.95 & 0.99 \\
29613 & 105627 & 1996 & 132.65 & 0.25 & 13234.00 & 128172.39 & 1.00 & 0.97 & 0.97 \\
29601 & 105623 & 1996 & 24.29 & 0.38 & 4714.00 & 45802.86 & 0.52 & 1.89 & 0.97 \\
12012 & 101477 & 1996 & 237.10 & 0.08 & 32879.00 & 239163.89 & 0.72 & 1.01 & 0.73 \\
29642 & 105630 & 1996 & 5.40 & 0.01 & 585.00 & 4956.53 & 0.92 & 0.92 & 0.85 \\
17652 & 102334 & 1996 & 137.28 & 0.21 & 13196.00 & 130795.55 & 1.04 & 0.95 & 0.99 \\
29654 & 105631 & 1996 & 5.20 & 0.07 & 527.00 & 5172.32 & 0.99 & 0.99 & 0.98 \\
17839 & 102365 & 1996 & 549.20 & 0.21 & 54805.00 & 523183.00 & 1.00 & 0.95 & 0.95 \\
17808 & 102364 & 1996 & 627.40 & 0.22 & 62358.00 & 616265.81 & 1.01 & 0.98 & 0.99 \\
11808 & 101462 & 1996 & 573.62 & 0.26 & 47043.00 & 523937.02 & 1.22 & 0.91 & 1.11 \\
17795 & 102358 & 1996 & 12.84 & 0.04 & 1696.00 & 16441.51 & 0.76 & 1.28 & 0.97 \\
10764 & 101330 & 1996 & 3836.95 & 0.29 & 326936.00 & 2800769.77 & 1.17 & 0.73 & 0.86 \\
17748 & 102356 & 1996 & 1.64 & 0.09 & 161.00 & 1481.93 & 1.02 & 0.91 & 0.92 \\
11901 & 101465 & 1996 & 48.16 & 0.21 & 4821.00 & 45661.05 & 1.00 & 0.95 & 0.95 \\
17727 & 102350 & 1996 & 54.69 & 0.22 & 4554.00 & 46527.83 & 1.20 & 0.85 & 1.02 \\
11932 & 101466 & 1996 & 259.71 & -0.11 & 25973.00 & 240614.96 & 1.00 & 0.93 & 0.93 \\
17704 & 102349 & 1996 & 289.08 & 0.18 & 27088.00 & 265719.64 & 1.07 & 0.92 & 0.98 \\
25957 & 103531 & 1996 & 847.90 & 0.34 & 84790.00 & 739816.42 & 1.00 & 0.87 & 0.87 \\
17673 & 102342 & 1996 & 60.27 & 0.25 & 6027.00 & 59814.67 & 1.00 & 0.99 & 0.99 \\
25919 & 103529 & 1996 & 1285.41 & 0.23 & 128541.00 & 1226012.51 & 1.00 & 0.95 & 0.95 \\
16738 & 102183 & 1996 & 66.73 & 0.04 & 10037.00 & 89201.49 & 0.66 & 1.34 & 0.89 \\
8923 & 101105 & 1996 & 13.00 & 0.10 & 1048.00 & 10325.48 & 1.24 & 0.79 & 0.99 \\
5047 & 100710 & 1996 & 215.19 & 0.25 & 21519.00 & 199135.75 & 1.00 & 0.93 & 0.93 \\
7426 & 101039 & 1996 & 2797.20 & 0.18 & 253841.00 & 2100305.98 & 1.10 & 0.75 & 0.83 \\
8273 & 101081 & 1996 & 337.50 & 0.32 & 25944.00 & 239131.37 & 1.30 & 0.71 & 0.92 \\
7974 & 101068 & 1996 & 53988.70 & 0.12 & 5291051.00 & 47762440.07 & 1.02 & 0.88 & 0.90 \\
58040 & 410060 & 1996 & 31.91 & 0.10 & 3838.00 & 38308.33 & 0.83 & 1.20 & 1.00 \\
5079 & 100723 & 1996 & 31.17 & 0.17 & 3149.00 & 30599.96 & 0.99 & 0.98 & 0.97 \\
9179 & 101116 & 1996 & 1267.90 & 0.11 & 123868.00 & 1301864.77 & 1.02 & 1.03 & 1.05 \\
3521 & 100453 & 1996 & 118.40 & 0.18 & 13343.00 & 133442.83 & 0.89 & 1.13 & 1.00 \\
829 & 100098 & 1996 & 10.25 & 0.15 & 973.00 & 9682.91 & 1.05 & 0.94 & 1.00 \\
5110 & 100724 & 1996 & 49.65 & 0.18 & 5013.00 & 48992.60 & 0.99 & 0.99 & 0.98 \\
7042 & 100992 & 1996 & 410.87 & 0.23 & 32241.00 & 311790.13 & 1.27 & 0.76 & 0.97 \\
5153 & 100727 & 1996 & 499.36 & 0.11 & 50657.00 & 475932.99 & 0.99 & 0.95 & 0.94 \\
3458 & 100439 & 1996 & 49.76 & 0.01 & 5964.00 & 57321.91 & 0.83 & 1.15 & 0.96 \\
799 & 100097 & 1996 & 6.78 & 0.10 & 682.00 & 6786.76 & 0.99 & 1.00 & 1.00 \\
5171 & 100730 & 1996 & 775.17 & 0.12 & 79920.00 & 762514.94 & 0.97 & 0.98 & 0.95 \\
9032 & 101109 & 1996 & 182.90 & 0.11 & 22314.00 & 215694.50 & 0.82 & 1.18 & 0.97 \\
7940 & 101067 & 1996 & 109.80 & 0.22 & 9164.00 & 77723.80 & 1.20 & 0.71 & 0.85 \\
57963 & 410010 & 1996 & 111.71 & 0.19 & 12296.00 & 100560.68 & 0.91 & 0.90 & 0.82 \\
3443 & 100435 & 1996 & 22.68 & 0.04 & 2627.00 & 18216.02 & 0.86 & 0.80 & 0.69 \\
769 & 100096 & 1996 & 18.81 & 0.11 & 1681.00 & 16468.29 & 1.12 & 0.88 & 0.98 \\
5130 & 100726 & 1996 & 3099.52 & 0.11 & 285828.00 & 2858448.22 & 1.08 & 0.92 & 1.00 \\
5193 & 100731 & 1996 & 12151.21 & 0.19 & 1057499.00 & 9652092.87 & 1.15 & 0.79 & 0.91 \\
859 & 100099 & 1996 & 30.94 & 0.15 & 3006.00 & 29331.52 & 1.03 & 0.95 & 0.98 \\
8242 & 101080 & 1996 & 107.00 & 0.13 & 9359.00 & 87914.77 & 1.14 & 0.82 & 0.94 \\
7079 & 100996 & 1996 & 1031.92 & 0.23 & 95887.00 & 949855.41 & 1.08 & 0.92 & 0.99 \\
1019 & 100127 & 1996 & 1356.71 & 0.19 & 136483.00 & 1219173.34 & 0.99 & 0.90 & 0.89 \\
8847 & 101102 & 1996 & 157.20 & 0.21 & 23417.00 & 224770.30 & 0.67 & 1.43 & 0.96 \\
4905 & 100692 & 1996 & 1523.84 & 0.10 & 146738.00 & 1392710.72 & 1.04 & 0.91 & 0.95 \\
975 & 100113 & 1996 & 1016.23 & 0.17 & 105850.00 & 1012769.17 & 0.96 & 1.00 & 0.96 \\
1979 & 100278 & 1996 & 3.52 & -0.05 & 345.00 & 3138.48 & 1.02 & 0.89 & 0.91 \\
9145 & 101115 & 1996 & 5778.80 & 0.23 & 425170.00 & 4994595.96 & 1.36 & 0.86 & 1.17 \\
3618 & 100463 & 1996 & 5.92 & 0.38 & 592.00 & 5731.17 & 1.00 & 0.97 & 0.97 \\
5023 & 100701 & 1996 & 435.56 & 0.06 & 48104.00 & 463243.39 & 0.91 & 1.06 & 0.96 \\
4935 & 100695 & 1996 & 92.94 & 0.22 & 9294.00 & 80723.91 & 1.00 & 0.87 & 0.87 \\
3586 & 100460 & 1996 & 23.89 & 0.00 & 3601.00 & 36136.29 & 0.66 & 1.51 & 1.00 \\
4961 & 100697 & 1996 & 66.91 & 0.13 & 6619.00 & 64862.71 & 1.01 & 0.97 & 0.98 \\
3574 & 100457 & 1996 & 55.61 & 0.22 & 5009.00 & 49486.56 & 1.11 & 0.89 & 0.99 \\
58068 & 410075 & 1996 & 115.00 & 0.70 & 12093.00 & 102196.55 & 0.95 & 0.89 & 0.85 \\
4994 & 100698 & 1996 & 95.90 & 0.25 & 9329.00 & 85188.49 & 1.03 & 0.89 & 0.91 \\
2003 & 100280 & 1996 & 52.49 & 0.16 & 5595.00 & 44190.40 & 0.94 & 0.84 & 0.79 \\
889 & 100101 & 1996 & 156.03 & 0.14 & 16337.00 & 162631.09 & 0.96 & 1.04 & 1.00 \\
8735 & 101096 & 1996 & 86.20 & 0.27 & 7798.00 & 78152.93 & 1.11 & 0.91 & 1.00 \\
3561 & 100456 & 1996 & 2.21 & 0.20 & 218.00 & 2131.09 & 1.01 & 0.96 & 0.98 \\
5013 & 100700 & 1996 & 405.48 & 0.12 & 44099.00 & 417708.38 & 0.92 & 1.03 & 0.95 \\
3552 & 100455 & 1996 & 5.42 & 0.16 & 496.00 & 4342.33 & 1.09 & 0.80 & 0.88 \\
931 & 100112 & 1996 & 920.28 & 0.23 & 84986.00 & 817261.46 & 1.08 & 0.89 & 0.96 \\
3422 & 100432 & 1996 & 7.74 & 0.05 & 777.00 & 7390.82 & 1.00 & 0.95 & 0.95 \\
7001 & 100982 & 1996 & 1602.04 & 0.11 & 95069.00 & 823564.82 & 1.69 & 0.51 & 0.87 \\
5211 & 100736 & 1996 & 471.04 & 0.01 & 44287.00 & 457591.82 & 1.06 & 0.97 & 1.03 \\
8355 & 101084 & 1996 & 491.30 & 0.16 & 42012.00 & 464245.97 & 1.17 & 0.94 & 1.11 \\
3327 & 100424 & 1996 & 58.14 & 0.09 & 6838.00 & 59454.14 & 0.85 & 1.02 & 0.87 \\
5365 & 100758 & 1996 & 77.25 & 0.21 & 6718.00 & 68646.68 & 1.15 & 0.89 & 1.02 \\
27 & 100003 & 1996 & 482.64 & 0.09 & 45241.00 & 443539.29 & 1.07 & 0.92 & 0.98 \\
57898 & 401372 & 1996 & 8.27 & 0.10 & 964.00 & 8099.55 & 0.86 & 0.98 & 0.84 \\
5398 & 100760 & 1996 & 1052.69 & 0.13 & 111663.00 & 978929.94 & 0.94 & 0.93 & 0.88 \\
653 & 100087 & 1996 & 5411.85 & 0.17 & 530543.00 & 5118755.93 & 1.02 & 0.95 & 0.96 \\
57873 & 401319 & 1996 & 63.08 & 0.20 & 6056.00 & 51813.65 & 1.04 & 0.82 & 0.86 \\
5442 & 100763 & 1996 & 687.46 & 0.10 & 66563.00 & 602198.70 & 1.03 & 0.88 & 0.90 \\
5351 & 100757 & 1996 & 28.17 & 0.13 & 2852.00 & 28984.60 & 0.99 & 1.03 & 1.02 \\
6907 & 100968 & 1996 & 33.90 & 0.27 & 3386.00 & 31154.48 & 1.00 & 0.92 & 0.92 \\
2132 & 100292 & 1996 & 120.05 & 0.24 & 11977.00 & 118326.18 & 1.00 & 0.99 & 0.99 \\
57845 & 401145 & 1996 & 13.80 & 0.03 & 1385.00 & 11285.67 & 1.00 & 0.82 & 0.81 \\
3253 & 100419 & 1996 & 5.53 & 0.17 & 374.00 & 3387.05 & 1.48 & 0.61 & 0.91 \\
57837 & 401082 & 1996 & 19.25 & 0.32 & 1890.00 & 16320.04 & 1.02 & 0.85 & 0.86 \\
5464 & 100764 & 1996 & 236.49 & 0.22 & 22476.00 & 223991.19 & 1.05 & 0.95 & 1.00 \\
53632 & 355027 & 1996 & 159.30 & 0.17 & 26623.00 & 243509.24 & 0.60 & 1.53 & 0.91 \\
6936 & 100973 & 1996 & 14.45 & 0.23 & 1445.00 & 11863.01 & 1.00 & 0.82 & 0.82 \\
74624 & 601142 & 1996 & 308.04 & 0.16 & 50528.00 & 491720.79 & 0.61 & 1.60 & 0.97 \\
739 & 100093 & 1996 & 582.60 & 0.04 & 56293.00 & 556505.27 & 1.03 & 0.96 & 0.99 \\
6987 & 100981 & 1996 & 41.89 & 0.17 & 4286.00 & 38080.54 & 0.98 & 0.91 & 0.89 \\
74557 & 601136 & 1996 & 27.74 & 0.19 & 2776.00 & 26605.08 & 1.00 & 0.96 & 0.96 \\
8314 & 101082 & 1996 & 1242.30 & 0.15 & 120503.00 & 965370.67 & 1.03 & 0.78 & 0.80 \\
3392 & 100431 & 1996 & 188.09 & 0.07 & 24561.00 & 207492.41 & 0.77 & 1.10 & 0.84 \\
57929 & 410003 & 1996 & 451.80 & 0.32 & 37083.00 & 345919.19 & 1.22 & 0.77 & 0.93 \\
5266 & 100745 & 1996 & 2682.86 & 0.05 & 297373.00 & 2390348.88 & 0.90 & 0.89 & 0.80 \\
74585 & 601139 & 1996 & 2483.10 & 0.15 & 252061.00 & 2185647.27 & 0.99 & 0.88 & 0.87 \\
6982 & 100980 & 1996 & 81.17 & 0.25 & 5348.00 & 51947.07 & 1.52 & 0.64 & 0.97 \\
6975 & 100978 & 1996 & 76.33 & 0.09 & 7916.00 & 74450.77 & 0.96 & 0.98 & 0.94 \\
7459 & 101040 & 1996 & 2099.00 & 0.15 & 214649.00 & 1963610.62 & 0.98 & 0.94 & 0.91 \\
5287 & 100746 & 1996 & 1110.93 & 0.17 & 96159.00 & 992075.07 & 1.16 & 0.89 & 1.03 \\
7905 & 101065 & 1996 & 635.00 & 0.06 & 65372.00 & 575655.45 & 0.97 & 0.91 & 0.88 \\
705 & 100092 & 1996 & 138.87 & 0.12 & 13125.00 & 127967.74 & 1.06 & 0.92 & 0.97 \\
155 & 100016 & 1996 & 60.55 & 0.20 & 6121.00 & 61020.51 & 0.99 & 1.01 & 1.00 \\
3350 & 100425 & 1996 & 1658.55 & 0.23 & 192522.00 & 1808300.35 & 0.86 & 1.09 & 0.94 \\
4107 & 100552 & 1996 & 501.35 & 0.11 & 46365.00 & 477665.98 & 1.08 & 0.95 & 1.03 \\
5321 & 100753 & 1996 & 1890.92 & 0.20 & 192588.00 & 1649836.16 & 0.98 & 0.87 & 0.86 \\
681 & 100090 & 1996 & 288.80 & 0.12 & 26920.00 & 263798.47 & 1.07 & 0.91 & 0.98 \\
5342 & 100754 & 1996 & 771.09 & 0.19 & 77793.00 & 741617.37 & 0.99 & 0.96 & 0.95 \\
2092 & 100290 & 1996 & 296.74 & 0.19 & 46290.00 & 441316.36 & 0.64 & 1.49 & 0.95 \\
259 & 100022 & 1996 & 56.50 & 0.23 & 5718.00 & 55549.50 & 0.99 & 0.98 & 0.97 \\
4885 & 100691 & 1996 & 489.94 & 0.23 & 43115.00 & 412397.44 & 1.14 & 0.84 & 0.96 \\
4877 & 100690 & 1996 & 470.56 & 0.30 & 47110.00 & 465779.48 & 1.00 & 0.99 & 0.99 \\
3677 & 100468 & 1996 & 146.29 & 0.24 & 14683.00 & 143166.29 & 1.00 & 0.98 & 0.98 \\
1514 & 100209 & 1996 & 2645.05 & 0.17 & 227816.00 & 2265883.45 & 1.16 & 0.86 & 0.99 \\
3968 & 100535 & 1996 & 270.81 & 0.17 & 27064.00 & 267396.10 & 1.00 & 0.99 & 0.99 \\
8816 & 101100 & 1996 & 562.80 & 0.49 & 35464.00 & 326950.61 & 1.59 & 0.58 & 0.92 \\
1476 & 100207 & 1996 & 2749.84 & 0.20 & 261091.00 & 2650129.79 & 1.05 & 0.96 & 1.02 \\
7236 & 101015 & 1996 & 724.70 & 0.02 & 72206.00 & 670325.11 & 1.00 & 0.92 & 0.93 \\
1445 & 100200 & 1996 & 57.00 & 0.39 & 5747.00 & 53018.45 & 0.99 & 0.93 & 0.92 \\
65058 & 500659 & 1996 & 25.89 & 0.34 & 2981.00 & 23200.94 & 0.87 & 0.90 & 0.78 \\
7344 & 101023 & 1996 & 14258.90 & 0.21 & 1198349.00 & 10614588.34 & 1.19 & 0.74 & 0.89 \\
3914 & 100514 & 1996 & 60.61 & 0.17 & 6023.00 & 58225.07 & 1.01 & 0.96 & 0.97 \\
4345 & 100610 & 1996 & 1266.46 & 0.17 & 121144.00 & 1020231.61 & 1.05 & 0.81 & 0.84 \\
173 & 100017 & 1996 & 36.37 & 0.24 & 3845.00 & 35653.26 & 0.95 & 0.98 & 0.93 \\
4352 & 100611 & 1996 & 1063.16 & 0.06 & 104709.00 & 912136.82 & 1.02 & 0.86 & 0.87 \\
1854 & 100245 & 1996 & 315.87 & 0.18 & 31670.00 & 308989.50 & 1.00 & 0.98 & 0.98 \\
4378 & 100614 & 1996 & 405.15 & 0.20 & 37753.00 & 377545.38 & 1.07 & 0.93 & 1.00 \\
65081 & 500660 & 1996 & 153.25 & -0.25 & 17093.00 & 139197.31 & 0.90 & 0.91 & 0.81 \\
4396 & 100622 & 1996 & 245.47 & 0.18 & 22101.00 & 218019.12 & 1.11 & 0.89 & 0.99 \\
3874 & 100508 & 1996 & 11.35 & 0.04 & 1135.00 & 10242.42 & 1.00 & 0.90 & 0.90 \\
3866 & 100507 & 1996 & 24.30 & 0.07 & 2430.00 & 21411.68 & 1.00 & 0.88 & 0.88 \\
1364 & 100192 & 1996 & 61.77 & 0.36 & 6177.00 & 60999.75 & 1.00 & 0.99 & 0.99 \\
8125 & 101076 & 1996 & 28.10 & 0.16 & 3762.00 & 35694.33 & 0.75 & 1.27 & 0.95 \\
4434 & 100625 & 1996 & 520.79 & 0.19 & 49069.00 & 488276.31 & 1.06 & 0.94 & 1.00 \\
3858 & 100506 & 1996 & 28.80 & 0.02 & 2880.00 & 27311.99 & 1.00 & 0.95 & 0.95 \\
1415 & 100196 & 1996 & 815.40 & 0.17 & 81693.00 & 659674.93 & 1.00 & 0.81 & 0.81 \\
1532 & 100213 & 1996 & 228.66 & 0.21 & 21232.00 & 221903.43 & 1.08 & 0.97 & 1.05 \\
4280 & 100600 & 1996 & 93.07 & 0.23 & 10353.00 & 103538.53 & 0.90 & 1.11 & 1.00 \\
1820 & 100244 & 1996 & 128.25 & 0.07 & 12857.00 & 117252.51 & 1.00 & 0.91 & 0.91 \\
4126 & 100559 & 1996 & 40.28 & 0.22 & 4028.00 & 39815.82 & 1.00 & 0.99 & 0.99 \\
1743 & 100227 & 1996 & 152.76 & 0.18 & 15276.00 & 141467.32 & 1.00 & 0.93 & 0.93 \\
4172 & 100567 & 1996 & 2422.58 & 0.14 & 239280.00 & 2108076.59 & 1.01 & 0.87 & 0.88 \\
1680 & 100223 & 1996 & 1213.68 & 0.19 & 103201.00 & 1049313.76 & 1.18 & 0.86 & 1.02 \\
4059 & 100544 & 1996 & 161.73 & 0.44 & 16173.00 & 160997.61 & 1.00 & 1.00 & 1.00 \\
62 & 100004 & 1996 & 877.21 & 0.21 & 87723.00 & 841954.58 & 1.00 & 0.96 & 0.96 \\
1639 & 100219 & 1996 & 3.10 & 0.03 & 379.00 & 3360.81 & 0.82 & 1.08 & 0.89 \\
8064 & 101073 & 1996 & 3387.20 & 0.24 & 263514.00 & 2907679.63 & 1.29 & 0.86 & 1.10 \\
1625 & 100218 & 1996 & 12.20 & -0.05 & 1250.00 & 10013.44 & 0.98 & 0.82 & 0.80 \\
7310 & 101020 & 1996 & 1642.00 & 0.15 & 158985.00 & 1467295.65 & 1.03 & 0.89 & 0.92 \\
55028 & 400049 & 1996 & 225.50 & 0.20 & 24756.00 & 202303.83 & 0.91 & 0.90 & 0.82 \\
1763 & 100228 & 1996 & 165.79 & 0.16 & 16578.00 & 134891.80 & 1.00 & 0.81 & 0.81 \\
4209 & 100575 & 1996 & 17.14 & 0.21 & 1719.00 & 17997.96 & 1.00 & 1.05 & 1.05 \\
4045 & 100543 & 1996 & 631.26 & 0.28 & 63126.00 & 600522.27 & 1.00 & 0.95 & 0.95 \\
4223 & 100590 & 1996 & 52.92 & 0.04 & 5292.00 & 43853.14 & 1.00 & 0.83 & 0.83 \\
8094 & 101074 & 1996 & 153.10 & 0.34 & 10725.00 & 101801.54 & 1.43 & 0.66 & 0.95 \\
1782 & 100237 & 1996 & 98.21 & 0.26 & 9817.00 & 96249.09 & 1.00 & 0.98 & 0.98 \\
1563 & 100214 & 1996 & 148.23 & 0.10 & 14807.00 & 121675.61 & 1.00 & 0.82 & 0.82 \\
7274 & 101018 & 1996 & 9569.70 & 0.15 & 835910.00 & 8802371.88 & 1.14 & 0.92 & 1.05 \\
9112 & 101112 & 1996 & 1601.90 & 0.17 & 158643.00 & 1423090.87 & 1.01 & 0.89 & 0.90 \\
191 & 100018 & 1996 & 65.50 & 0.25 & 6606.00 & 61985.94 & 0.99 & 0.95 & 0.94 \\
4011 & 100538 & 1996 & 658.09 & 0.23 & 68900.00 & 645033.21 & 0.96 & 0.98 & 0.94 \\
8787 & 101098 & 1996 & 84.00 & 0.13 & 6339.00 & 55533.59 & 1.33 & 0.66 & 0.88 \\
1345 & 100190 & 1996 & 1619.59 & 0.23 & 161296.00 & 1336307.40 & 1.00 & 0.83 & 0.83 \\
2163 & 100293 & 1996 & 37.03 & 0.18 & 3651.00 & 34546.41 & 1.01 & 0.93 & 0.95 \\
65131 & 500664 & 1996 & 1922.18 & 0.18 & 192846.00 & 1630396.39 & 1.00 & 0.85 & 0.85 \\
3851 & 100505 & 1996 & 4.43 & -0.06 & 442.00 & 4316.33 & 1.00 & 0.98 & 0.98 \\
1926 & 100251 & 1996 & 141.11 & 0.05 & 14442.00 & 122467.37 & 0.98 & 0.87 & 0.85 \\
7142 & 100998 & 1996 & 157.90 & 0.15 & 14469.00 & 144459.24 & 1.09 & 0.91 & 1.00 \\
4699 & 100667 & 1996 & 13.36 & 0.18 & 1272.00 & 11309.25 & 1.05 & 0.85 & 0.89 \\
8205 & 101079 & 1996 & 239.70 & 0.32 & 14236.00 & 144904.40 & 1.68 & 0.60 & 1.02 \\
4736 & 100670 & 1996 & 91.30 & 0.22 & 9135.00 & 86926.97 & 1.00 & 0.95 & 0.95 \\
7388 & 101038 & 1996 & 4903.20 & 0.19 & 456548.00 & 3911310.09 & 1.07 & 0.80 & 0.86 \\
4758 & 100671 & 1996 & 190.23 & 0.18 & 18735.00 & 170770.79 & 1.02 & 0.90 & 0.91 \\
1134 & 100155 & 1996 & 234.45 & 0.32 & 23557.00 & 224018.50 & 1.00 & 0.96 & 0.95 \\
3750 & 100480 & 1996 & 52.63 & 0.23 & 4385.00 & 42721.11 & 1.20 & 0.81 & 0.97 \\
1149 & 100157 & 1996 & 330.00 & 0.22 & 34498.00 & 321018.00 & 0.96 & 0.97 & 0.93 \\
4804 & 100682 & 1996 & 75.21 & 0.05 & 7175.00 & 63498.72 & 1.05 & 0.84 & 0.88 \\
1099 & 100153 & 1996 & 207.11 & 0.16 & 20701.00 & 178993.82 & 1.00 & 0.86 & 0.86 \\
4837 & 100685 & 1996 & 12.23 & 0.29 & 1037.00 & 10149.54 & 1.18 & 0.83 & 0.98 \\
8755 & 101097 & 1996 & 437.00 & 0.05 & 49349.00 & 459446.23 & 0.89 & 1.05 & 0.93 \\
7110 & 100997 & 1996 & 61.76 & 0.18 & 5995.00 & 55858.21 & 1.03 & 0.90 & 0.93 \\
3720 & 100475 & 1996 & 112.32 & 0.21 & 10434.00 & 106558.21 & 1.08 & 0.95 & 1.02 \\
53095 & 338393 & 1996 & 12.66 & 0.22 & 1125.00 & 12002.58 & 1.13 & 0.95 & 1.07 \\
1069 & 100150 & 1996 & 14.75 & 0.12 & 1517.00 & 13996.41 & 0.97 & 0.95 & 0.92 \\
4867 & 100687 & 1996 & 195.28 & 0.23 & 19726.00 & 192812.59 & 0.99 & 0.99 & 0.98 \\
8006 & 101069 & 1996 & 1597.50 & 0.23 & 138629.00 & 1365832.62 & 1.15 & 0.85 & 0.99 \\
4872 & 100688 & 1996 & 136.81 & 0.22 & 13534.00 & 133131.65 & 1.01 & 0.97 & 0.98 \\
61326 & 500037 & 1996 & 3710.38 & 0.23 & 340198.00 & 3441745.47 & 1.09 & 0.93 & 1.01 \\
4672 & 100660 & 1996 & 221.15 & 0.22 & 26127.00 & 246278.65 & 0.85 & 1.11 & 0.94 \\
3780 & 100481 & 1996 & 37.23 & 0.19 & 3356.00 & 33615.48 & 1.11 & 0.90 & 1.00 \\
2608 & 100347 & 1996 & 704.49 & 0.22 & 70354.00 & 694749.69 & 1.00 & 0.99 & 0.99 \\
7197 & 101013 & 1996 & 15405.90 & 0.18 & 1457542.00 & 11692337.44 & 1.06 & 0.76 & 0.80 \\
96659 & 611002 & 1996 & 2951.57 & 0.18 & 312859.00 & 2838460.35 & 0.94 & 0.96 & 0.91 \\
4474 & 100634 & 1996 & 1064.10 & 0.21 & 105127.00 & 964901.51 & 1.01 & 0.91 & 0.92 \\
4485 & 100635 & 1996 & 68.07 & 0.38 & 5615.00 & 54297.04 & 1.21 & 0.80 & 0.97 \\
1883 & 100247 & 1996 & 830.73 & 0.16 & 83090.00 & 773699.95 & 1.00 & 0.93 & 0.93 \\
3812 & 100485 & 1996 & 111.33 & 0.26 & 10010.00 & 91374.56 & 1.11 & 0.82 & 0.91 \\
1261 & 100168 & 1996 & 2.31 & 0.18 & 231.00 & 1884.30 & 1.00 & 0.81 & 0.82 \\
61240 & 500005 & 1996 & 28.00 & 0.45 & 2885.00 & 23142.72 & 0.97 & 0.83 & 0.80 \\
4524 & 100637 & 1996 & 1585.28 & 0.13 & 148781.00 & 1435545.17 & 1.07 & 0.91 & 0.96 \\
1250 & 100167 & 1996 & 252.29 & 0.21 & 23853.00 & 250844.07 & 1.06 & 0.99 & 1.05 \\
8025 & 101071 & 1996 & 1695.10 & 0.18 & 171093.00 & 1607327.60 & 0.99 & 0.95 & 0.94 \\
4558 & 100639 & 1996 & 309.55 & 0.22 & 26685.00 & 291548.21 & 1.16 & 0.94 & 1.09 \\
9080 & 101111 & 1996 & 510.60 & 0.36 & 42385.00 & 418814.85 & 1.20 & 0.82 & 0.99 \\
7161 & 101000 & 1996 & 839.77 & 0.21 & 83054.00 & 811187.91 & 1.01 & 0.97 & 0.98 \\
61284 & 500027 & 1996 & 309.78 & 0.24 & 29883.00 & 269218.29 & 1.04 & 0.87 & 0.90 \\
9069 & 101110 & 1996 & 20.60 & 0.07 & 3237.00 & 28273.90 & 0.64 & 1.37 & 0.87 \\
1219 & 100166 & 1996 & 3789.07 & 0.14 & 360425.00 & 3259932.68 & 1.05 & 0.86 & 0.90 \\
8174 & 101078 & 1996 & 39.10 & 0.25 & 3294.00 & 34757.75 & 1.19 & 0.89 & 1.06 \\
4592 & 100642 & 1996 & 948.98 & 0.17 & 102743.00 & 955295.29 & 0.92 & 1.01 & 0.93 \\
1181 & 100159 & 1996 & 254.90 & 0.36 & 25273.00 & 229292.31 & 1.01 & 0.90 & 0.91 \\
4638 & 100659 & 1996 & 207.89 & 0.22 & 20789.00 & 205738.36 & 1.00 & 0.99 & 0.99 \\
4449 & 100633 & 1996 & 536.76 & 0.21 & 50989.00 & 509996.28 & 1.05 & 0.95 & 1.00 \\
8393 & 101085 & 1996 & 255.70 & 0.11 & 21112.00 & 237929.48 & 1.21 & 0.93 & 1.13 \\
57909 & 402003 & 1996 & 102.16 & 0.26 & 9857.00 & 81223.27 & 1.04 & 0.80 & 0.82 \\
7523 & 101043 & 1996 & 1854.00 & 0.16 & 193921.00 & 1789945.38 & 0.96 & 0.97 & 0.92 \\
7586 & 101047 & 1996 & 255.10 & 0.21 & 25513.00 & 244420.80 & 1.00 & 0.96 & 0.96 \\
2886 & 100369 & 1996 & 221.92 & 0.22 & 23199.00 & 209701.50 & 0.96 & 0.94 & 0.90 \\
6080 & 100822 & 1996 & 15.90 & 0.14 & 2163.00 & 22202.05 & 0.74 & 1.40 & 1.03 \\
2877 & 100368 & 1996 & 146.71 & 0.19 & 15095.00 & 132537.75 & 0.97 & 0.90 & 0.88 \\
6744 & 100947 & 1996 & 1283.96 & 0.17 & 118771.00 & 1005735.44 & 1.08 & 0.78 & 0.85 \\
8897 & 101104 & 1996 & 72.60 & 0.01 & 9032.00 & 79086.27 & 0.80 & 1.09 & 0.88 \\
6141 & 100824 & 1996 & 9.45 & 0.22 & 907.00 & 8838.12 & 1.04 & 0.94 & 0.97 \\
8964 & 101107 & 1996 & 1333.90 & 0.17 & 129505.00 & 1293301.63 & 1.03 & 0.97 & 1.00 \\
8675 & 101094 & 1996 & 401.50 & 0.22 & 38229.00 & 415351.52 & 1.05 & 1.03 & 1.09 \\
6145 & 100825 & 1996 & 51.59 & 0.31 & 3419.00 & 33551.19 & 1.51 & 0.65 & 0.98 \\
2849 & 100365 & 1996 & 161.79 & 0.21 & 16973.00 & 157445.92 & 0.95 & 0.97 & 0.93 \\
6158 & 100827 & 1996 & 23.76 & 0.34 & 1685.00 & 16558.59 & 1.41 & 0.70 & 0.98 \\
2825 & 100362 & 1996 & 94.35 & 0.14 & 9761.00 & 97804.88 & 0.97 & 1.04 & 1.00 \\
6193 & 100829 & 1996 & 795.55 & 0.17 & 78100.00 & 739519.14 & 1.02 & 0.93 & 0.95 \\
2362 & 100320 & 1996 & 48.39 & 0.28 & 4965.00 & 45379.04 & 0.97 & 0.94 & 0.91 \\
2814 & 100360 & 1996 & 781.98 & 0.17 & 81577.00 & 685173.25 & 0.96 & 0.88 & 0.84 \\
2900 & 100373 & 1996 & 80.12 & 0.24 & 8789.00 & 67393.16 & 0.91 & 0.84 & 0.77 \\
458 & 100068 & 1996 & 60.02 & 0.09 & 5443.00 & 56811.25 & 1.10 & 0.95 & 1.04 \\
96 & 100006 & 1996 & 4606.54 & 0.22 & 458710.00 & 3781510.86 & 1.00 & 0.82 & 0.82 \\
5867 & 100809 & 1996 & 2711.68 & 0.16 & 269877.00 & 2646173.56 & 1.00 & 0.98 & 0.98 \\
3020 & 100398 & 1996 & 55.33 & -0.02 & 5516.00 & 47833.97 & 1.00 & 0.86 & 0.87 \\
7797 & 101061 & 1996 & 1252.90 & 0.23 & 126654.00 & 1182798.15 & 0.99 & 0.94 & 0.93 \\
9242 & 101122 & 1996 & 166.10 & 0.01 & 16044.00 & 148326.10 & 1.04 & 0.89 & 0.92 \\
2990 & 100395 & 1996 & 572.69 & 0.22 & 57033.00 & 512469.66 & 1.00 & 0.89 & 0.90 \\
6765 & 100953 & 1996 & 13.25 & 0.22 & 1142.00 & 12214.48 & 1.16 & 0.92 & 1.07 \\
63174 & 500486 & 1996 & 300.94 & 0.19 & 24179.00 & 249984.48 & 1.24 & 0.83 & 1.03 \\
5938 & 100812 & 1996 & 464.38 & 0.22 & 40864.00 & 413406.52 & 1.14 & 0.89 & 1.01 \\
2330 & 100319 & 1996 & 191.63 & 0.21 & 19246.00 & 177820.35 & 1.00 & 0.93 & 0.92 \\
5971 & 100815 & 1996 & 139.84 & 0.22 & 12124.00 & 125639.25 & 1.15 & 0.90 & 1.04 \\
5979 & 100817 & 1996 & 19.59 & 0.22 & 1798.00 & 17717.40 & 1.09 & 0.90 & 0.99 \\
115 & 100009 & 1996 & 152.39 & 0.17 & 14516.00 & 140344.22 & 1.05 & 0.92 & 0.97 \\
53456 & 350572 & 1996 & 44.92 & 0.19 & 4283.00 & 45294.40 & 1.05 & 1.01 & 1.06 \\
2310 & 100315 & 1996 & 311.15 & 0.22 & 31079.00 & 302427.23 & 1.00 & 0.97 & 0.97 \\
413 & 100055 & 1996 & 6332.10 & 0.20 & 610066.00 & 5709296.83 & 1.04 & 0.90 & 0.94 \\
9319 & 101131 & 1996 & 856.20 & 0.16 & 85623.00 & 734877.81 & 1.00 & 0.86 & 0.86 \\
6452 & 100875 & 1996 & 170.74 & 0.30 & 17075.00 & 164262.73 & 1.00 & 0.96 & 0.96 \\
2685 & 100352 & 1996 & 274.31 & 0.27 & 27443.00 & 261445.89 & 1.00 & 0.95 & 0.95 \\
6633 & 100906 & 1996 & 1260.72 & 0.16 & 117413.00 & 1265711.08 & 1.07 & 1.00 & 1.08 \\
9346 & 101132 & 1996 & 233.10 & 0.39 & 23306.00 & 208358.12 & 1.00 & 0.89 & 0.89 \\
8626 & 101092 & 1996 & 794.30 & 0.37 & 76370.00 & 717794.30 & 1.04 & 0.90 & 0.94 \\
7661 & 101054 & 1996 & 4450.80 & 0.20 & 444123.00 & 4107244.01 & 1.00 & 0.92 & 0.92 \\
6501 & 100878 & 1996 & 1526.07 & 0.24 & 152598.00 & 1468968.64 & 1.00 & 0.96 & 0.96 \\
2665 & 100351 & 1996 & 130.52 & 0.18 & 13102.00 & 131024.51 & 1.00 & 1.00 & 1.00 \\
6537 & 100889 & 1996 & 48.05 & 0.21 & 4696.00 & 46961.20 & 1.02 & 0.98 & 1.00 \\
6546 & 100890 & 1996 & 594.90 & 0.25 & 57938.00 & 553629.06 & 1.03 & 0.93 & 0.96 \\
318 & 100036 & 1996 & 82.94 & 0.25 & 8294.00 & 79154.04 & 1.00 & 0.95 & 0.95 \\
6 & 100001 & 1996 & 1655.86 & 0.22 & 154472.00 & 1561856.31 & 1.07 & 0.94 & 1.01 \\
2646 & 100350 & 1996 & 98.63 & 0.14 & 10499.00 & 104210.05 & 0.94 & 1.06 & 0.99 \\
7699 & 101055 & 1996 & 9272.60 & 0.17 & 960666.00 & 8179871.51 & 0.97 & 0.88 & 0.85 \\
6589 & 100900 & 1996 & 67.78 & 0.17 & 5930.00 & 51674.34 & 1.14 & 0.76 & 0.87 \\
6582 & 100892 & 1996 & 143.95 & 0.10 & 14750.00 & 145832.25 & 0.98 & 1.01 & 0.99 \\
2627 & 100348 & 1996 & 116.51 & 0.16 & 11643.00 & 109255.73 & 1.00 & 0.94 & 0.94 \\
6578 & 100891 & 1996 & 239.41 & 0.20 & 25036.00 & 227063.23 & 0.96 & 0.95 & 0.91 \\
6697 & 100913 & 1996 & 22.34 & 0.11 & 2184.00 & 20187.32 & 1.02 & 0.90 & 0.92 \\
139 & 100010 & 1996 & 1497.94 & 0.19 & 135168.00 & 1556152.94 & 1.11 & 1.04 & 1.15 \\
6426 & 100868 & 1996 & 137.40 & 0.17 & 13772.00 & 131780.78 & 1.00 & 0.96 & 0.96 \\
2414 & 100323 & 1996 & 158.02 & 0.24 & 12855.00 & 108963.38 & 1.23 & 0.69 & 0.85 \\
7606 & 101048 & 1996 & 4708.50 & 0.17 & 486492.00 & 4176440.47 & 0.97 & 0.89 & 0.86 \\
6227 & 100831 & 1996 & 268.79 & 0.18 & 25155.00 & 241795.40 & 1.07 & 0.90 & 0.96 \\
6254 & 100833 & 1996 & 562.41 & 0.24 & 58812.00 & 542377.28 & 0.96 & 0.96 & 0.92 \\
386 & 100048 & 1996 & 416.33 & 0.27 & 38442.00 & 360512.01 & 1.08 & 0.87 & 0.94 \\
6683 & 100910 & 1996 & 78.26 & 0.18 & 7826.00 & 70116.15 & 1.00 & 0.90 & 0.90 \\
74788 & 601171 & 1996 & 271.09 & 0.23 & 25173.00 & 260753.44 & 1.08 & 0.96 & 1.04 \\
6298 & 100847 & 1996 & 2.99 & 0.11 & 306.00 & 2625.09 & 0.98 & 0.88 & 0.86 \\
2437 & 100330 & 1996 & 1326.32 & 0.22 & 132632.00 & 1219754.98 & 1.00 & 0.92 & 0.92 \\
6666 & 100908 & 1996 & 121.70 & 0.16 & 12030.00 & 118939.74 & 1.01 & 0.98 & 0.99 \\
2748 & 100357 & 1996 & 144.62 & 0.05 & 14513.00 & 139159.15 & 1.00 & 0.96 & 0.96 \\
6327 & 100849 & 1996 & 75.83 & 0.21 & 8295.00 & 67030.15 & 0.91 & 0.88 & 0.81 \\
53401 & 346113 & 1996 & 387.55 & 0.14 & 42374.00 & 361708.24 & 0.91 & 0.93 & 0.85 \\
286 & 100033 & 1996 & 265.43 & 0.18 & 26553.00 & 263193.51 & 1.00 & 0.99 & 0.99 \\
6375 & 100856 & 1996 & 146.18 & 0.16 & 15801.00 & 151924.35 & 0.93 & 1.04 & 0.96 \\
2717 & 100355 & 1996 & 715.06 & 0.24 & 71769.00 & 675300.54 & 1.00 & 0.94 & 0.94 \\
7636 & 101050 & 1996 & 147.40 & 0.16 & 14688.00 & 138744.89 & 1.00 & 0.94 & 0.94 \\
8598 & 101091 & 1996 & 252.80 & 0.13 & 31142.00 & 297971.39 & 0.81 & 1.18 & 0.96 \\
7733 & 101056 & 1996 & 23896.50 & 0.20 & 1920564.00 & 19300587.15 & 1.24 & 0.81 & 1.00 \\
2473 & 100333 & 1996 & 130.56 & 0.18 & 13050.00 & 124555.93 & 1.00 & 0.95 & 0.95 \\
3024 & 100399 & 1996 & 203.44 & 0.20 & 21636.00 & 171293.48 & 0.94 & 0.84 & 0.79 \\
3209 & 100415 & 1996 & 166.64 & 0.26 & 17736.00 & 164393.86 & 0.94 & 0.99 & 0.93 \\
2269 & 100305 & 1996 & 19.95 & 0.24 & 1977.00 & 17308.05 & 1.01 & 0.87 & 0.88 \\
7554 & 101045 & 1996 & 15238.40 & 0.19 & 1559603.00 & 12821433.22 & 0.98 & 0.84 & 0.82 \\
6875 & 100967 & 1996 & 253.57 & 0.11 & 25338.00 & 242773.58 & 1.00 & 0.96 & 0.96 \\
541 & 100075 & 1996 & 2723.95 & 0.14 & 275056.00 & 2531601.14 & 0.99 & 0.93 & 0.92 \\
6867 & 100966 & 1996 & 30.73 & 0.10 & 3073.00 & 29596.01 & 1.00 & 0.96 & 0.96 \\
6787 & 100954 & 1996 & 779.62 & 0.23 & 52415.00 & 474612.39 & 1.49 & 0.61 & 0.91 \\
592 & 100079 & 1996 & 1305.61 & 0.20 & 127244.00 & 1238294.28 & 1.03 & 0.95 & 0.97 \\
5619 & 100775 & 1996 & 457.84 & 0.18 & 40767.00 & 420375.78 & 1.12 & 0.92 & 1.03 \\
5706 & 100789 & 1996 & 26.59 & 0.11 & 2791.00 & 25747.51 & 0.95 & 0.97 & 0.92 \\
74672 & 601149 & 1996 & 26.04 & 0.29 & 2515.00 & 23385.25 & 1.04 & 0.90 & 0.93 \\
2250 & 100302 & 1996 & 2.93 & 0.11 & 317.00 & 2788.88 & 0.92 & 0.95 & 0.88 \\
520 & 100072 & 1996 & 11915.24 & 0.16 & 1154854.00 & 10449970.87 & 1.03 & 0.88 & 0.90 \\
3102 & 100409 & 1996 & 383.03 & 0.23 & 38303.00 & 357808.45 & 1.00 & 0.93 & 0.93 \\
74681 & 601151 & 1996 & 5.89 & 0.16 & 578.00 & 5489.52 & 1.02 & 0.93 & 0.95 \\
9214 & 101119 & 1996 & 43.62 & 0.17 & 4357.00 & 38925.64 & 1.00 & 0.89 & 0.89 \\
63000 & 500466 & 1996 & 181.93 & 0.18 & 17413.00 & 174138.00 & 1.04 & 0.96 & 1.00 \\
8710 & 101095 & 1996 & 84.50 & 0.14 & 9685.00 & 74558.61 & 0.87 & 0.88 & 0.77 \\
6809 & 100958 & 1996 & 2.16 & 0.21 & 216.00 & 2007.22 & 1.00 & 0.93 & 0.93 \\
2217 & 100296 & 1996 & 21.10 & 0.25 & 2113.00 & 21152.28 & 1.00 & 1.00 & 1.00 \\
2256 & 100303 & 1996 & 141.88 & 0.20 & 15103.00 & 139162.77 & 0.94 & 0.98 & 0.92 \\
5820 & 100804 & 1996 & 876.70 & 0.26 & 85597.00 & 842361.94 & 1.02 & 0.96 & 0.98 \\
2229 & 100298 & 1996 & 401.93 & 0.16 & 40092.00 & 373587.93 & 1.00 & 0.93 & 0.93 \\
3137 & 100411 & 1996 & 6546.94 & 0.25 & 654694.00 & 6266611.77 & 1.00 & 0.96 & 0.96 \\
8995 & 101108 & 1996 & 538.00 & 0.14 & 52466.00 & 478891.42 & 1.03 & 0.89 & 0.91 \\
5724 & 100790 & 1996 & 77.39 & 0.14 & 7763.00 & 76823.63 & 1.00 & 0.99 & 0.99 \\
6831 & 100962 & 1996 & 2263.71 & 0.08 & 226371.00 & 1987475.42 & 1.00 & 0.88 & 0.88 \\
2243 & 100299 & 1996 & 113.69 & 0.20 & 11352.00 & 103370.36 & 1.00 & 0.91 & 0.91 \\
567 & 100076 & 1996 & 507.69 & 0.18 & 50869.00 & 413435.27 & 1.00 & 0.81 & 0.81 \\
5644 & 100784 & 1996 & 2252.00 & 0.22 & 207321.00 & 2067676.68 & 1.09 & 0.92 & 1.00 \\
5780 & 100792 & 1996 & 270.80 & 0.29 & 27080.00 & 223464.00 & 1.00 & 0.83 & 0.83 \\
7867 & 101064 & 1996 & 354.70 & 0.20 & 46849.00 & 359930.34 & 0.76 & 1.01 & 0.77 \\
8425 & 101086 & 1996 & 275.30 & 0.15 & 31687.00 & 287089.70 & 0.87 & 1.04 & 0.91 \\
3080 & 100408 & 1996 & 128.70 & 0.13 & 12617.00 & 125023.29 & 1.02 & 0.97 & 0.99 \\
2196 & 100295 & 1996 & 12.60 & 0.09 & 1310.00 & 10711.42 & 0.96 & 0.85 & 0.82 \\
23721 & 103209 & 1997 & 383.01 & 0.11 & 38301.00 & 342023.98 & 1.00 & 0.89 & 0.89 \\
49060 & 240212 & 1997 & 2663.38 & 0.32 & 227279.00 & 2303212.18 & 1.17 & 0.86 & 1.01 \\
1855 & 100245 & 1997 & 358.15 & 0.23 & 36413.00 & 326588.88 & 0.98 & 0.91 & 0.90 \\
39891 & 107928 & 1997 & 861.01 & 0.32 & 86101.00 & 826049.18 & 1.00 & 0.96 & 0.96 \\
12637 & 101561 & 1997 & 118.23 & 0.26 & 8958.00 & 104725.44 & 1.32 & 0.89 & 1.17 \\
23687 & 103208 & 1997 & 1564.83 & 0.24 & 156783.00 & 1406889.84 & 1.00 & 0.90 & 0.90 \\
7734 & 101056 & 1997 & 36656.81 & 0.30 & 3428368.00 & 33036137.81 & 1.07 & 0.90 & 0.96 \\
2218 & 100296 & 1997 & 40.20 & 0.25 & 4347.00 & 42844.46 & 0.92 & 1.07 & 0.99 \\
12356 & 101537 & 1997 & 786.70 & 0.32 & 80750.00 & 787153.49 & 0.97 & 1.00 & 0.97 \\
22997 & 103101 & 1997 & 445.46 & 0.27 & 42208.00 & 422098.29 & 1.06 & 0.95 & 1.00 \\
22979 & 103100 & 1997 & 496.28 & 0.28 & 48527.00 & 485284.33 & 1.02 & 0.98 & 1.00 \\
49008 & 240199 & 1997 & 566.83 & 0.09 & 83269.00 & 685893.51 & 0.68 & 1.21 & 0.82 \\
24544 & 103347 & 1997 & 3.60 & 0.25 & 363.00 & 3613.77 & 0.99 & 1.00 & 1.00 \\
1884 & 100247 & 1997 & 866.01 & 0.13 & 89647.00 & 766741.83 & 0.97 & 0.89 & 0.86 \\
12702 & 101588 & 1997 & 324.67 & -0.03 & 32466.00 & 303435.50 & 1.00 & 0.93 & 0.93 \\
22960 & 103099 & 1997 & 506.94 & 0.32 & 46501.00 & 455527.04 & 1.09 & 0.90 & 0.98 \\
2438 & 100330 & 1997 & 3793.01 & 0.34 & 379301.00 & 3530309.76 & 1.00 & 0.93 & 0.93 \\
24558 & 103366 & 1997 & 105.58 & 0.34 & 15583.00 & 154909.94 & 0.68 & 1.47 & 0.99 \\
23475 & 103179 & 1997 & 1012.59 & 0.06 & 111064.00 & 1041820.94 & 0.91 & 1.03 & 0.94 \\
24570 & 103369 & 1997 & 89.64 & 0.31 & 8964.00 & 88753.43 & 1.00 & 0.99 & 0.99 \\
24590 & 103370 & 1997 & 153.48 & 0.15 & 15348.00 & 150973.28 & 1.00 & 0.98 & 0.98 \\
23028 & 103103 & 1997 & 197.37 & 0.31 & 18553.00 & 194559.47 & 1.06 & 0.99 & 1.05 \\
22945 & 103090 & 1997 & 1010.05 & 0.29 & 97756.00 & 924537.15 & 1.03 & 0.92 & 0.95 \\
22920 & 103085 & 1997 & 165.47 & 0.26 & 12981.00 & 129893.70 & 1.27 & 0.78 & 1.00 \\
24613 & 103372 & 1997 & 721.30 & 0.39 & 61619.00 & 529126.17 & 1.17 & 0.73 & 0.86 \\
39769 & 107870 & 1997 & 233.19 & 0.03 & 23127.00 & 217921.74 & 1.01 & 0.93 & 0.94 \\
12668 & 101562 & 1997 & 366.96 & 0.27 & 32587.00 & 315722.33 & 1.13 & 0.86 & 0.97 \\
1764 & 100228 & 1997 & 198.69 & 0.23 & 19869.00 & 161221.06 & 1.00 & 0.81 & 0.81 \\
7700 & 101055 & 1997 & 10509.37 & 0.19 & 1009067.00 & 10429057.41 & 1.04 & 0.99 & 1.03 \\
24798 & 103380 & 1997 & 5033.18 & 0.32 & 478221.00 & 4756873.56 & 1.05 & 0.95 & 0.99 \\
23517 & 103183 & 1997 & 623.37 & 0.32 & 60837.00 & 609253.26 & 1.02 & 0.98 & 1.00 \\
39764 & 107868 & 1997 & 96.08 & 0.30 & 9608.00 & 81688.99 & 1.00 & 0.85 & 0.85 \\
2133 & 100292 & 1997 & 174.19 & 0.36 & 17419.00 & 160433.04 & 1.00 & 0.92 & 0.92 \\
23550 & 103186 & 1997 & 532.23 & 0.26 & 47554.00 & 467843.71 & 1.12 & 0.88 & 0.98 \\
2537 & 100343 & 1997 & 374.40 & 0.33 & 31500.00 & 334382.40 & 1.19 & 0.89 & 1.06 \\
22743 & 103057 & 1997 & 4094.72 & 0.24 & 377819.00 & 3296073.80 & 1.08 & 0.80 & 0.87 \\
7868 & 101064 & 1997 & 211.60 & -0.39 & 29748.00 & 196903.48 & 0.71 & 0.93 & 0.66 \\
1744 & 100227 & 1997 & 178.34 & 0.22 & 17834.00 & 159266.25 & 1.00 & 0.89 & 0.89 \\
24838 & 103381 & 1997 & 22489.94 & 0.23 & 2172591.00 & 19877700.89 & 1.04 & 0.88 & 0.91 \\
12912 & 101606 & 1997 & 4143.34 & 0.20 & 418419.00 & 4041004.90 & 0.99 & 0.98 & 0.97 \\
49264 & 240261 & 1997 & 287.25 & 0.19 & 28725.00 & 248418.13 & 1.00 & 0.86 & 0.86 \\
24879 & 103383 & 1997 & 1688.07 & -0.09 & 184822.00 & 1645020.00 & 0.91 & 0.97 & 0.89 \\
23543 & 103184 & 1997 & 1267.63 & 0.26 & 132864.00 & 1101948.74 & 0.95 & 0.87 & 0.83 \\
24759 & 103377 & 1997 & 1277.84 & 0.19 & 132151.00 & 1156428.68 & 0.97 & 0.90 & 0.88 \\
22791 & 103065 & 1997 & 382.32 & 0.18 & 38031.00 & 353664.33 & 1.01 & 0.93 & 0.93 \\
2164 & 100293 & 1997 & 51.36 & 0.29 & 4431.00 & 42155.96 & 1.16 & 0.82 & 0.95 \\
23654 & 103205 & 1997 & 39.42 & 0.42 & 3938.00 & 36969.28 & 1.00 & 0.94 & 0.94 \\
48995 & 240198 & 1997 & 1409.24 & 0.28 & 140692.00 & 1435217.90 & 1.00 & 1.02 & 1.02 \\
2474 & 100333 & 1997 & 135.92 & 0.29 & 13605.00 & 129863.58 & 1.00 & 0.96 & 0.95 \\
24642 & 103373 & 1997 & 170.69 & 0.11 & 19387.00 & 180605.17 & 0.88 & 1.06 & 0.93 \\
39862 & 107892 & 1997 & 133.91 & 0.06 & 13391.00 & 123898.06 & 1.00 & 0.93 & 0.93 \\
49236 & 240254 & 1997 & 1493.40 & 0.02 & 150396.00 & 1377440.63 & 0.99 & 0.92 & 0.92 \\
2093 & 100290 & 1997 & 360.80 & 0.27 & 45386.00 & 447202.25 & 0.79 & 1.24 & 0.99 \\
2197 & 100295 & 1997 & 13.65 & 0.14 & 1427.00 & 12344.98 & 0.96 & 0.90 & 0.87 \\
23623 & 103204 & 1997 & 187.45 & 0.28 & 13479.00 & 140064.22 & 1.39 & 0.75 & 1.04 \\
2503 & 100336 & 1997 & 22.89 & 0.36 & 2257.00 & 22305.91 & 1.01 & 0.97 & 0.99 \\
12879 & 101603 & 1997 & 1037.86 & 0.30 & 104018.00 & 991915.19 & 1.00 & 0.96 & 0.95 \\
23599 & 103202 & 1997 & 52.64 & 0.01 & 4619.00 & 44575.27 & 1.14 & 0.85 & 0.97 \\
22877 & 103074 & 1997 & 103.97 & 0.29 & 7687.00 & 75487.10 & 1.35 & 0.73 & 0.98 \\
1821 & 100244 & 1997 & 129.90 & 0.01 & 12724.00 & 118913.13 & 1.02 & 0.92 & 0.93 \\
23596 & 103199 & 1997 & 8.00 & -0.01 & 673.00 & 6328.92 & 1.19 & 0.79 & 0.94 \\
48981 & 240197 & 1997 & 751.71 & 0.28 & 77896.00 & 737731.85 & 0.97 & 0.98 & 0.95 \\
24682 & 103375 & 1997 & 986.07 & -0.01 & 102364.00 & 895395.36 & 0.96 & 0.91 & 0.87 \\
23569 & 103193 & 1997 & 59.97 & 0.36 & 6018.00 & 54084.70 & 1.00 & 0.90 & 0.90 \\
22858 & 103073 & 1997 & 1151.48 & 0.44 & 114928.00 & 1044449.45 & 1.00 & 0.91 & 0.91 \\
1783 & 100237 & 1997 & 183.96 & 0.29 & 18416.00 & 172974.83 & 1.00 & 0.94 & 0.94 \\
24722 & 103376 & 1997 & 5171.01 & 0.29 & 511515.00 & 4798284.63 & 1.01 & 0.93 & 0.94 \\
22826 & 103067 & 1997 & 81.52 & 0.32 & 8131.00 & 78073.32 & 1.00 & 0.96 & 0.96 \\
23496 & 103182 & 1997 & 340.01 & 0.26 & 33774.00 & 311793.69 & 1.01 & 0.92 & 0.92 \\
23099 & 103122 & 1997 & 96.67 & 0.25 & 8499.00 & 75421.92 & 1.14 & 0.78 & 0.89 \\
8026 & 101071 & 1997 & 1763.07 & 0.27 & 167440.00 & 1648866.29 & 1.05 & 0.94 & 0.98 \\
23306 & 103160 & 1997 & 93.00 & 0.29 & 9302.00 & 85256.03 & 1.00 & 0.92 & 0.92 \\
23294 & 103158 & 1997 & 1851.10 & 0.28 & 185107.00 & 1724504.53 & 1.00 & 0.93 & 0.93 \\
24196 & 103296 & 1997 & 1767.95 & 0.22 & 171733.00 & 1802414.27 & 1.03 & 1.02 & 1.05 \\
24000 & 103253 & 1997 & 163.70 & 0.30 & 14992.00 & 151869.44 & 1.09 & 0.93 & 1.01 \\
40179 & 108071 & 1997 & 27.07 & -0.01 & 2707.00 & 26832.45 & 1.00 & 0.99 & 0.99 \\
24231 & 103299 & 1997 & 781.06 & -0.04 & 84293.00 & 781280.07 & 0.93 & 1.00 & 0.93 \\
23279 & 103154 & 1997 & 1166.70 & 0.06 & 116665.00 & 1118741.00 & 1.00 & 0.96 & 0.96 \\
12780 & 101595 & 1997 & 1258.54 & 0.17 & 125854.00 & 1179327.79 & 1.00 & 0.94 & 0.94 \\
1980 & 100278 & 1997 & 7.33 & 0.24 & 733.00 & 7186.54 & 1.00 & 0.98 & 0.98 \\
23979 & 103252 & 1997 & 276.29 & 0.30 & 25354.00 & 268917.09 & 1.09 & 0.97 & 1.06 \\
12568 & 101554 & 1997 & 376.24 & 0.21 & 36688.00 & 364047.49 & 1.03 & 0.97 & 0.99 \\
23958 & 103251 & 1997 & 245.00 & 0.21 & 23848.00 & 244808.30 & 1.03 & 1.00 & 1.03 \\
23928 & 103242 & 1997 & 47.32 & 0.32 & 4538.00 & 44286.11 & 1.04 & 0.94 & 0.98 \\
23247 & 103152 & 1997 & 1826.06 & 0.31 & 182606.00 & 1694777.32 & 1.00 & 0.93 & 0.93 \\
24280 & 103304 & 1997 & 22.54 & 0.21 & 2260.00 & 19267.05 & 1.00 & 0.85 & 0.85 \\
23410 & 103175 & 1997 & 1949.51 & 0.13 & 194823.00 & 1770157.39 & 1.00 & 0.91 & 0.91 \\
2331 & 100319 & 1997 & 217.62 & 0.32 & 21619.00 & 214499.03 & 1.01 & 0.99 & 0.99 \\
23380 & 103174 & 1997 & 1657.68 & 0.24 & 164271.00 & 1417964.61 & 1.01 & 0.86 & 0.86 \\
2004 & 100280 & 1997 & 46.29 & 0.19 & 4628.00 & 45452.20 & 1.00 & 0.98 & 0.98 \\
24083 & 103264 & 1997 & 814.93 & 0.22 & 84049.00 & 836958.11 & 0.97 & 1.03 & 1.00 \\
24111 & 103267 & 1997 & 117.17 & 0.23 & 15611.00 & 156314.21 & 0.75 & 1.33 & 1.00 \\
2311 & 100315 & 1997 & 395.82 & 0.28 & 39601.00 & 388473.22 & 1.00 & 0.98 & 0.98 \\
12752 & 101592 & 1997 & 501.74 & 0.19 & 50174.00 & 452850.16 & 1.00 & 0.90 & 0.90 \\
24052 & 103259 & 1997 & 2504.93 & 0.26 & 241695.00 & 2295153.17 & 1.04 & 0.92 & 0.95 \\
12515 & 101545 & 1997 & 329.90 & 0.29 & 32990.00 & 313354.58 & 1.00 & 0.95 & 0.95 \\
7798 & 101061 & 1997 & 1564.21 & 0.25 & 152969.00 & 1368536.64 & 1.02 & 0.87 & 0.89 \\
49103 & 240222 & 1997 & 843.24 & 0.29 & 84686.00 & 816248.41 & 1.00 & 0.97 & 0.96 \\
12544 & 101553 & 1997 & 328.16 & 0.05 & 30569.00 & 302673.34 & 1.07 & 0.92 & 0.99 \\
7975 & 101068 & 1997 & 63766.16 & 0.27 & 6374135.00 & 56868751.05 & 1.00 & 0.89 & 0.89 \\
40257 & 108083 & 1997 & 33.09 & 0.04 & 3224.00 & 30771.92 & 1.03 & 0.93 & 0.95 \\
7941 & 101067 & 1997 & 68.32 & 0.16 & 8499.00 & 73986.50 & 0.80 & 1.08 & 0.87 \\
23348 & 103166 & 1997 & 3.49 & -0.01 & 425.00 & 3300.38 & 0.82 & 0.95 & 0.78 \\
24161 & 103294 & 1997 & 470.51 & 0.35 & 38408.00 & 324765.64 & 1.23 & 0.69 & 0.85 \\
23366 & 103172 & 1997 & 69.15 & 0.12 & 6634.00 & 61561.04 & 1.04 & 0.89 & 0.93 \\
2270 & 100305 & 1997 & 49.95 & 0.27 & 6665.00 & 52186.53 & 0.75 & 1.04 & 0.78 \\
12503 & 101544 & 1997 & 46.80 & 0.37 & 4680.00 & 45029.59 & 1.00 & 0.96 & 0.96 \\
24030 & 103255 & 1997 & 166.73 & 0.21 & 15564.00 & 162739.44 & 1.07 & 0.98 & 1.05 \\
1967 & 100263 & 1997 & 13.31 & 0.13 & 1331.00 & 12144.48 & 1.00 & 0.91 & 0.91 \\
23908 & 103232 & 1997 & 116.87 & 0.30 & 11571.00 & 115643.29 & 1.01 & 0.99 & 1.00 \\
12407 & 101539 & 1997 & 726.78 & 0.26 & 71028.00 & 721151.29 & 1.02 & 0.99 & 1.02 \\
23823 & 103214 & 1997 & 1597.12 & 0.19 & 159712.00 & 1433660.37 & 1.00 & 0.90 & 0.90 \\
23444 & 103177 & 1997 & 374.31 & 0.15 & 37624.00 & 366404.18 & 0.99 & 0.98 & 0.97 \\
12607 & 101560 & 1997 & 29.15 & 0.32 & 2915.00 & 26566.72 & 1.00 & 0.91 & 0.91 \\
1927 & 100251 & 1997 & 152.16 & 0.23 & 14279.00 & 139302.35 & 1.07 & 0.92 & 0.98 \\
12717 & 101590 & 1997 & 13.19 & 0.19 & 1319.00 & 12916.41 & 1.00 & 0.98 & 0.98 \\
2395 & 100322 & 1997 & 1421.13 & 0.27 & 141918.00 & 1313795.91 & 1.00 & 0.92 & 0.93 \\
24448 & 103327 & 1997 & 1815.63 & 0.31 & 193475.00 & 2007935.91 & 0.94 & 1.11 & 1.04 \\
40005 & 107994 & 1997 & 194.37 & 0.06 & 10875.00 & 96889.76 & 1.79 & 0.50 & 0.89 \\
2244 & 100299 & 1997 & 154.60 & 0.28 & 15557.00 & 147579.97 & 0.99 & 0.95 & 0.95 \\
24500 & 103329 & 1997 & 856.27 & 0.36 & 62237.00 & 686486.06 & 1.38 & 0.80 & 1.10 \\
23793 & 103213 & 1997 & 848.73 & 0.24 & 83158.00 & 691230.82 & 1.02 & 0.81 & 0.83 \\
24095 & 103266 & 1997 & 335.87 & 0.11 & 31472.00 & 322229.96 & 1.07 & 0.96 & 1.02 \\
12710 & 101589 & 1997 & 95.44 & 0.09 & 9543.00 & 86525.84 & 1.00 & 0.91 & 0.91 \\
12822 & 101601 & 1997 & 1123.51 & 0.19 & 117079.00 & 1072634.98 & 0.96 & 0.95 & 0.92 \\
23757 & 103212 & 1997 & 1924.89 & 0.23 & 192489.00 & 1691028.84 & 1.00 & 0.88 & 0.88 \\
2415 & 100323 & 1997 & 890.36 & 0.32 & 76646.00 & 746035.57 & 1.16 & 0.84 & 0.97 \\
24470 & 103328 & 1997 & 1028.07 & 0.10 & 77732.00 & 763128.78 & 1.32 & 0.74 & 0.98 \\
24532 & 103339 & 1997 & 655.41 & 0.15 & 68198.00 & 627138.66 & 0.96 & 0.96 & 0.92 \\
24418 & 103326 & 1997 & 1718.42 & 0.29 & 184482.00 & 1850478.32 & 0.93 & 1.08 & 1.00 \\
23139 & 103134 & 1997 & 515.80 & 0.32 & 51580.00 & 513092.16 & 1.00 & 0.99 & 0.99 \\
24314 & 103308 & 1997 & 6494.23 & 0.27 & 649422.00 & 5720564.04 & 1.00 & 0.88 & 0.88 \\
23892 & 103228 & 1997 & 56.75 & 0.13 & 5675.00 & 54565.47 & 1.00 & 0.96 & 0.96 \\
23184 & 103144 & 1997 & 238.58 & 0.21 & 27416.00 & 245352.77 & 0.87 & 1.03 & 0.89 \\
12441 & 101541 & 1997 & 98.16 & 0.42 & 9816.00 & 97685.62 & 1.00 & 1.00 & 1.00 \\
7830 & 101062 & 1997 & 2480.20 & 0.21 & 219606.00 & 2184728.71 & 1.13 & 0.88 & 0.99 \\
2257 & 100303 & 1997 & 161.79 & 0.26 & 19202.00 & 153476.54 & 0.84 & 0.95 & 0.80 \\
8007 & 101069 & 1997 & 3869.65 & 0.33 & 350016.00 & 3415289.66 & 1.11 & 0.88 & 0.98 \\
23166 & 103138 & 1997 & 2014.00 & 0.33 & 192226.00 & 1937701.14 & 1.05 & 0.96 & 1.01 \\
23856 & 103224 & 1997 & 262.28 & 0.03 & 24497.00 & 247482.24 & 1.07 & 0.94 & 1.01 \\
40069 & 108021 & 1997 & 271.35 & 0.19 & 27135.00 & 225602.95 & 1.00 & 0.83 & 0.83 \\
7906 & 101065 & 1997 & 551.02 & 0.15 & 54232.00 & 514630.61 & 1.02 & 0.93 & 0.95 \\
12801 & 101600 & 1997 & 1965.15 & 0.26 & 196770.00 & 1943044.13 & 1.00 & 0.99 & 0.99 \\
24368 & 103318 & 1997 & 603.60 & 0.37 & 60854.00 & 595717.62 & 0.99 & 0.99 & 0.98 \\
23155 & 103136 & 1997 & 399.88 & 0.19 & 39990.00 & 367823.07 & 1.00 & 0.92 & 0.92 \\
40033 & 108013 & 1997 & 186.71 & 0.29 & 18473.00 & 184738.47 & 1.01 & 0.99 & 1.00 \\
24402 & 103319 & 1997 & 301.19 & 0.32 & 29547.00 & 301954.63 & 1.02 & 1.00 & 1.02 \\
12593 & 101557 & 1997 & 58.24 & 0.40 & 4844.00 & 49457.94 & 1.20 & 0.85 & 1.02 \\
23874 & 103226 & 1997 & 105.90 & 0.31 & 10590.00 & 104626.57 & 1.00 & 0.99 & 0.99 \\
40041 & 108018 & 1997 & 177.34 & 0.04 & 17733.00 & 147691.85 & 1.00 & 0.83 & 0.83 \\
2363 & 100320 & 1997 & 87.73 & 0.33 & 8170.00 & 78714.78 & 1.07 & 0.90 & 0.96 \\
24900 & 103394 & 1997 & 49.88 & 0.23 & 4852.00 & 41980.73 & 1.03 & 0.84 & 0.87 \\
27858 & 105335 & 1997 & 151.72 & 0.27 & 15253.00 & 147100.77 & 0.99 & 0.97 & 0.96 \\
27850 & 105333 & 1997 & 17.49 & 0.29 & 1748.00 & 17251.56 & 1.00 & 0.99 & 0.99 \\
27821 & 105332 & 1997 & 103.62 & 0.10 & 10464.00 & 103537.31 & 0.99 & 1.00 & 0.99 \\
11397 & 101400 & 1997 & 443.06 & 0.02 & 43466.00 & 431664.48 & 1.02 & 0.97 & 0.99 \\
27792 & 105331 & 1997 & 3.95 & 0.23 & 256.00 & 2640.86 & 1.54 & 0.67 & 1.03 \\
521 & 100072 & 1997 & 14283.05 & 0.13 & 1452906.00 & 13003212.83 & 0.98 & 0.91 & 0.89 \\
27786 & 105327 & 1997 & 39.90 & 0.24 & 4002.00 & 36899.89 & 1.00 & 0.92 & 0.92 \\
27872 & 105336 & 1997 & 51.68 & 0.20 & 5184.00 & 43467.93 & 1.00 & 0.84 & 0.84 \\
27778 & 105326 & 1997 & 51.40 & 0.33 & 5133.00 & 51053.07 & 1.00 & 0.99 & 0.99 \\
27750 & 105321 & 1997 & 77.05 & 0.21 & 7358.00 & 73577.71 & 1.05 & 0.95 & 1.00 \\
74682 & 601151 & 1997 & 8.26 & 0.29 & 745.00 & 6273.72 & 1.11 & 0.76 & 0.84 \\
11431 & 101402 & 1997 & 10.58 & 0.17 & 1241.00 & 11531.18 & 0.85 & 1.09 & 0.93 \\
27707 & 105317 & 1997 & 266.98 & 0.40 & 22150.00 & 209011.94 & 1.21 & 0.78 & 0.94 \\
27695 & 105311 & 1997 & 42.67 & 0.23 & 4267.00 & 40631.92 & 1.00 & 0.95 & 0.95 \\
27768 & 105322 & 1997 & 37.31 & 0.25 & 3731.00 & 35839.36 & 1.00 & 0.96 & 0.96 \\
27687 & 105310 & 1997 & 26.66 & 0.12 & 2512.00 & 21946.20 & 1.06 & 0.82 & 0.87 \\
11369 & 101399 & 1997 & 148.45 & 0.01 & 14555.00 & 145558.77 & 1.02 & 0.98 & 1.00 \\
11363 & 101398 & 1997 & 180.72 & 0.29 & 18701.00 & 180853.43 & 0.97 & 1.00 & 0.97 \\
28092 & 105382 & 1997 & 29.89 & 0.31 & 2729.00 & 26802.05 & 1.10 & 0.90 & 0.98 \\
28070 & 105372 & 1997 & 45.68 & 0.25 & 5075.00 & 44061.48 & 0.90 & 0.96 & 0.87 \\
485 & 100071 & 1997 & 4703.97 & 0.24 & 456887.00 & 3714258.24 & 1.03 & 0.79 & 0.81 \\
8491 & 101088 & 1997 & 2301.95 & 0.14 & 289607.00 & 1926662.90 & 0.79 & 0.84 & 0.67 \\
28012 & 105369 & 1997 & 366.09 & 0.28 & 26825.00 & 266698.82 & 1.36 & 0.73 & 0.99 \\
27892 & 105342 & 1997 & 34.30 & 0.23 & 5548.00 & 50941.99 & 0.62 & 1.49 & 0.92 \\
28009 & 105366 & 1997 & 17.87 & 0.19 & 1712.00 & 17044.26 & 1.04 & 0.95 & 1.00 \\
27948 & 105358 & 1997 & 415.13 & 0.31 & 41513.00 & 395218.53 & 1.00 & 0.95 & 0.95 \\
27943 & 105354 & 1997 & 194.86 & 0.20 & 18640.00 & 174138.05 & 1.05 & 0.89 & 0.93 \\
27929 & 105353 & 1997 & 110.01 & 0.32 & 10945.00 & 108368.68 & 1.01 & 0.99 & 0.99 \\
27923 & 105352 & 1997 & 200.51 & 0.28 & 21004.00 & 196949.21 & 0.95 & 0.98 & 0.94 \\
8459 & 101087 & 1997 & 647.63 & 0.62 & 36285.00 & 375130.34 & 1.78 & 0.58 & 1.03 \\
27906 & 105346 & 1997 & 994.44 & 0.18 & 88513.00 & 854930.59 & 1.12 & 0.86 & 0.97 \\
27897 & 105343 & 1997 & 199.86 & 0.30 & 19854.00 & 194244.63 & 1.01 & 0.97 & 0.98 \\
27980 & 105364 & 1997 & 61.25 & 0.21 & 5893.00 & 55993.61 & 1.04 & 0.91 & 0.95 \\
11281 & 101390 & 1997 & 3658.13 & 0.33 & 279083.00 & 3497316.67 & 1.31 & 0.96 & 1.25 \\
11454 & 101414 & 1997 & 74.24 & 0.31 & 7313.00 & 73129.08 & 1.02 & 0.99 & 1.00 \\
27368 & 105275 & 1997 & 109.06 & 0.16 & 10905.00 & 97107.04 & 1.00 & 0.89 & 0.89 \\
11557 & 101430 & 1997 & 157.19 & 0.31 & 10456.00 & 89759.25 & 1.50 & 0.57 & 0.86 \\
27357 & 105271 & 1997 & 14.60 & 0.01 & 1040.00 & 10020.07 & 1.40 & 0.69 & 0.96 \\
27335 & 105269 & 1997 & 257.28 & 0.32 & 27398.00 & 255463.64 & 0.94 & 0.99 & 0.93 \\
27310 & 105268 & 1997 & 162.15 & 0.29 & 16464.00 & 155874.11 & 0.98 & 0.96 & 0.95 \\
628 & 100085 & 1997 & 13598.25 & 0.23 & 1213506.00 & 11862386.76 & 1.12 & 0.87 & 0.98 \\
27398 & 105276 & 1997 & 546.38 & 0.29 & 54637.00 & 470295.90 & 1.00 & 0.86 & 0.86 \\
27272 & 105260 & 1997 & 255.35 & 0.23 & 25535.00 & 255405.64 & 1.00 & 1.00 & 1.00 \\
27243 & 105259 & 1997 & 13.20 & 0.02 & 1202.00 & 12251.30 & 1.10 & 0.93 & 1.02 \\
37708 & 107004 & 1997 & 109.23 & 0.35 & 8964.00 & 95120.64 & 1.22 & 0.87 & 1.06 \\
654 & 100087 & 1997 & 6453.76 & 0.25 & 636114.00 & 5731479.00 & 1.01 & 0.89 & 0.90 \\
8426 & 101086 & 1997 & 203.67 & 0.06 & 27731.00 & 205008.21 & 0.73 & 1.01 & 0.74 \\
27661 & 105309 & 1997 & 465.31 & 0.24 & 45263.00 & 423204.93 & 1.03 & 0.91 & 0.93 \\
27632 & 105306 & 1997 & 4.97 & 0.03 & 391.00 & 3682.06 & 1.27 & 0.74 & 0.94 \\
27603 & 105303 & 1997 & 194.23 & 0.26 & 19041.00 & 180162.37 & 1.02 & 0.93 & 0.95 \\
11481 & 101422 & 1997 & 32.94 & 0.29 & 4044.00 & 36489.86 & 0.81 & 1.11 & 0.90 \\
568 & 100076 & 1997 & 625.71 & 0.19 & 57071.00 & 536572.20 & 1.10 & 0.86 & 0.94 \\
27539 & 105287 & 1997 & 138.23 & 0.07 & 16269.00 & 139358.70 & 0.85 & 1.01 & 0.86 \\
27530 & 105286 & 1997 & 157.09 & 0.16 & 15814.00 & 133721.61 & 0.99 & 0.85 & 0.85 \\
74673 & 601149 & 1997 & 67.65 & 0.21 & 6765.00 & 63909.25 & 1.00 & 0.94 & 0.94 \\
11503 & 101425 & 1997 & 27.38 & 0.30 & 2548.00 & 22561.33 & 1.07 & 0.82 & 0.89 \\
593 & 100079 & 1997 & 1650.19 & 0.24 & 165023.00 & 1553023.80 & 1.00 & 0.94 & 0.94 \\
27472 & 105281 & 1997 & 290.46 & 0.25 & 29224.00 & 243341.59 & 0.99 & 0.84 & 0.83 \\
27463 & 105280 & 1997 & 94.62 & 0.32 & 9676.00 & 89237.62 & 0.98 & 0.94 & 0.92 \\
27457 & 105279 & 1997 & 47.97 & 0.26 & 4555.00 & 46522.47 & 1.05 & 0.97 & 1.02 \\
28121 & 105383 & 1997 & 73.73 & 0.29 & 7028.00 & 64535.48 & 1.05 & 0.88 & 0.92 \\
28135 & 105384 & 1997 & 73.03 & 0.30 & 6908.00 & 70679.43 & 1.06 & 0.97 & 1.02 \\
11268 & 101381 & 1997 & 55.23 & 0.29 & 5523.00 & 53282.68 & 1.00 & 0.96 & 0.96 \\
28862 & 105489 & 1997 & 3.71 & 0.35 & 475.00 & 4708.85 & 0.78 & 1.27 & 0.99 \\
336 & 100040 & 1997 & 391.88 & 0.24 & 36240.00 & 340659.98 & 1.08 & 0.87 & 0.94 \\
28851 & 105487 & 1997 & 82.99 & 0.01 & 6015.00 & 62424.22 & 1.38 & 0.75 & 1.04 \\
28832 & 105479 & 1997 & 48.36 & 0.20 & 4842.00 & 41027.17 & 1.00 & 0.85 & 0.85 \\
28804 & 105478 & 1997 & 33.63 & 0.24 & 3363.00 & 27878.46 & 1.00 & 0.83 & 0.83 \\
11021 & 101360 & 1997 & 1334.19 & 0.38 & 128772.00 & 1160204.29 & 1.04 & 0.87 & 0.90 \\
28744 & 105475 & 1997 & 582.81 & 0.17 & 58146.00 & 513887.95 & 1.00 & 0.88 & 0.88 \\
28722 & 105472 & 1997 & 135.43 & 0.14 & 14475.00 & 142334.61 & 0.94 & 1.05 & 0.98 \\
28714 & 105471 & 1997 & 79.69 & 0.10 & 10167.00 & 94286.68 & 0.78 & 1.18 & 0.93 \\
53402 & 346113 & 1997 & 347.40 & 0.16 & 36843.00 & 338824.40 & 0.94 & 0.98 & 0.92 \\
28695 & 105465 & 1997 & 42.14 & 0.39 & 4440.00 & 41494.13 & 0.95 & 0.98 & 0.93 \\
28685 & 105464 & 1997 & 30.85 & 0.17 & 3137.00 & 28276.08 & 0.98 & 0.92 & 0.90 \\
28773 & 105476 & 1997 & 58.64 & 0.19 & 4592.00 & 42818.65 & 1.28 & 0.73 & 0.93 \\
28676 & 105463 & 1997 & 283.87 & 0.08 & 30968.00 & 280747.29 & 0.92 & 0.99 & 0.91 \\
28869 & 105498 & 1997 & 118.95 & -0.12 & 11895.00 & 117525.65 & 1.00 & 0.99 & 0.99 \\
28879 & 105502 & 1997 & 1472.93 & 0.25 & 147293.00 & 1227358.82 & 1.00 & 0.83 & 0.83 \\
29064 & 105523 & 1997 & 101.42 & 0.24 & 9774.00 & 97629.04 & 1.04 & 0.96 & 1.00 \\
10923 & 101354 & 1997 & 1016.82 & 0.30 & 101681.00 & 881912.64 & 1.00 & 0.87 & 0.87 \\
29035 & 105522 & 1997 & 73.39 & 0.19 & 7149.00 & 65087.94 & 1.03 & 0.89 & 0.91 \\
29027 & 105520 & 1997 & 24.92 & 0.01 & 2565.00 & 22148.10 & 0.97 & 0.89 & 0.86 \\
29000 & 105512 & 1997 & 6.54 & 0.37 & 718.00 & 7189.34 & 0.91 & 1.10 & 1.00 \\
28993 & 105511 & 1997 & 51.83 & 0.22 & 4596.00 & 49696.11 & 1.13 & 0.96 & 1.08 \\
10981 & 101358 & 1997 & 450.94 & 0.27 & 45094.00 & 443349.94 & 1.00 & 0.98 & 0.98 \\
319 & 100036 & 1997 & 117.13 & 0.36 & 10229.00 & 110727.44 & 1.15 & 0.95 & 1.08 \\
28956 & 105508 & 1997 & 20.52 & 0.23 & 2059.00 & 18460.14 & 1.00 & 0.90 & 0.90 \\
28937 & 105507 & 1997 & 588.20 & 0.33 & 58850.00 & 550833.64 & 1.00 & 0.94 & 0.94 \\
28921 & 105506 & 1997 & 29.99 & 0.01 & 2565.00 & 25634.42 & 1.17 & 0.85 & 1.00 \\
8627 & 101092 & 1997 & 1663.98 & 0.46 & 140534.00 & 1487500.95 & 1.18 & 0.89 & 1.06 \\
10970 & 101357 & 1997 & 306.19 & 0.26 & 30618.00 & 261380.17 & 1.00 & 0.85 & 0.85 \\
28982 & 105510 & 1997 & 100.30 & 0.06 & 12274.00 & 102353.14 & 0.82 & 1.02 & 0.83 \\
28647 & 105458 & 1997 & 719.20 & 0.22 & 71920.00 & 676225.00 & 1.00 & 0.94 & 0.94 \\
74789 & 601171 & 1997 & 431.42 & 0.32 & 39262.00 & 380653.69 & 1.10 & 0.88 & 0.97 \\
28632 & 105457 & 1997 & 167.47 & 0.33 & 16747.00 & 165932.85 & 1.00 & 0.99 & 0.99 \\
11201 & 101374 & 1997 & 7.97 & 0.25 & 797.00 & 7798.36 & 1.00 & 0.98 & 0.98 \\
11205 & 101375 & 1997 & 15.79 & 0.21 & 1579.00 & 15480.76 & 1.00 & 0.98 & 0.98 \\
11216 & 101376 & 1997 & 254.88 & 0.12 & 25488.00 & 248613.33 & 1.00 & 0.98 & 0.98 \\
11193 & 101370 & 1997 & 12.57 & 0.35 & 1257.00 & 11806.70 & 1.00 & 0.94 & 0.94 \\
28224 & 105394 & 1997 & 73.62 & 0.11 & 9094.00 & 83034.52 & 0.81 & 1.13 & 0.91 \\
28205 & 105393 & 1997 & 18.11 & 0.19 & 1811.00 & 16553.26 & 1.00 & 0.91 & 0.91 \\
28196 & 105391 & 1997 & 13.20 & 0.29 & 1274.00 & 12247.83 & 1.04 & 0.93 & 0.96 \\
459 & 100068 & 1997 & 77.25 & 0.33 & 7042.00 & 69277.64 & 1.10 & 0.90 & 0.98 \\
11246 & 101379 & 1997 & 547.01 & 0.33 & 54700.00 & 531751.34 & 1.00 & 0.97 & 0.97 \\
53457 & 350572 & 1997 & 57.82 & 0.37 & 6003.00 & 57242.88 & 0.96 & 0.99 & 0.95 \\
11257 & 101380 & 1997 & 379.26 & 0.24 & 37926.00 & 349769.93 & 1.00 & 0.92 & 0.92 \\
28427 & 105424 & 1997 & 916.53 & 0.34 & 77670.00 & 778918.53 & 1.18 & 0.85 & 1.00 \\
11155 & 101369 & 1997 & 951.54 & 0.43 & 88109.00 & 859324.72 & 1.08 & 0.90 & 0.98 \\
387 & 100048 & 1997 & 643.39 & 0.31 & 61192.00 & 538502.93 & 1.05 & 0.84 & 0.88 \\
11085 & 101367 & 1997 & 292.96 & 0.36 & 29652.00 & 243344.03 & 0.99 & 0.83 & 0.82 \\
36494 & 106541 & 1997 & 15.28 & 0.15 & 1528.00 & 14409.47 & 1.00 & 0.94 & 0.94 \\
28611 & 105450 & 1997 & 9.93 & 0.28 & 867.00 & 10076.36 & 1.14 & 1.02 & 1.16 \\
28582 & 105448 & 1997 & 60.22 & 0.19 & 3267.00 & 33801.96 & 1.84 & 0.56 & 1.03 \\
28549 & 105438 & 1997 & 152.02 & 0.22 & 15411.00 & 149292.10 & 0.99 & 0.98 & 0.97 \\
11119 & 101368 & 1997 & 596.70 & 0.48 & 55644.00 & 590927.92 & 1.07 & 0.99 & 1.06 \\
28536 & 105437 & 1997 & 2496.47 & 0.24 & 244545.00 & 2383872.72 & 1.02 & 0.95 & 0.97 \\
414 & 100055 & 1997 & 7989.82 & 0.25 & 769775.00 & 7506200.60 & 1.04 & 0.94 & 0.98 \\
28529 & 105436 & 1997 & 102.34 & 0.21 & 10739.00 & 107960.68 & 0.95 & 1.05 & 1.01 \\
28514 & 105431 & 1997 & 78.19 & 0.36 & 10668.00 & 107104.24 & 0.73 & 1.37 & 1.00 \\
28485 & 105427 & 1997 & 66.07 & 0.08 & 6593.00 & 57877.63 & 1.00 & 0.88 & 0.88 \\
8564 & 101090 & 1997 & 569.37 & 0.00 & 80130.00 & 538174.07 & 0.71 & 0.95 & 0.67 \\
28456 & 105426 & 1997 & 354.82 & 0.35 & 32334.00 & 303877.58 & 1.10 & 0.86 & 0.94 \\
27207 & 105253 & 1997 & 45.54 & 0.22 & 2399.00 & 19970.01 & 1.90 & 0.44 & 0.83 \\
27198 & 105252 & 1997 & 122.37 & 0.24 & 12305.00 & 124866.58 & 0.99 & 1.02 & 1.01 \\
27188 & 105250 & 1997 & 7.32 & 0.27 & 762.00 & 7220.58 & 0.96 & 0.99 & 0.95 \\
25551 & 103496 & 1997 & 432.60 & 0.24 & 39205.00 & 440890.92 & 1.10 & 1.02 & 1.12 \\
12062 & 101494 & 1997 & 587.42 & 0.16 & 33950.00 & 324552.80 & 1.73 & 0.55 & 0.96 \\
1251 & 100167 & 1997 & 384.49 & 0.30 & 30967.00 & 386791.39 & 1.24 & 1.01 & 1.25 \\
25494 & 103494 & 1997 & 382.40 & 0.22 & 36553.00 & 355083.35 & 1.05 & 0.93 & 0.97 \\
1262 & 100168 & 1997 & 1.59 & 0.28 & 169.00 & 1672.20 & 0.94 & 1.05 & 0.99 \\
1269 & 100171 & 1997 & 1803.84 & 0.26 & 166860.00 & 1720986.38 & 1.08 & 0.95 & 1.03 \\
12079 & 101497 & 1997 & 2404.72 & 0.29 & 240772.00 & 2081268.11 & 1.00 & 0.87 & 0.86 \\
25431 & 103487 & 1997 & 44.47 & 0.09 & 4446.00 & 40851.65 & 1.00 & 0.92 & 0.92 \\
65132 & 500664 & 1997 & 2397.50 & 0.28 & 218633.00 & 1823829.43 & 1.10 & 0.76 & 0.83 \\
12099 & 101503 & 1997 & 327.27 & 0.02 & 34681.00 & 313723.40 & 0.94 & 0.96 & 0.90 \\
25407 & 103483 & 1997 & 542.19 & 0.21 & 54219.00 & 533379.70 & 1.00 & 0.98 & 0.98 \\
1346 & 100190 & 1997 & 2335.98 & 0.28 & 192341.00 & 1902879.51 & 1.21 & 0.81 & 0.99 \\
8126 & 101076 & 1997 & 26.61 & 0.18 & 2778.00 & 26052.99 & 0.96 & 0.98 & 0.94 \\
25582 & 103497 & 1997 & 21.80 & 0.12 & 2690.00 & 21726.68 & 0.81 & 1.00 & 0.81 \\
8175 & 101078 & 1997 & 49.97 & 0.24 & 6123.00 & 48959.37 & 0.82 & 0.98 & 0.80 \\
38982 & 107470 & 1997 & 4.26 & 0.27 & 426.00 & 3933.77 & 1.00 & 0.92 & 0.92 \\
25857 & 103525 & 1997 & 18835.81 & 0.33 & 1883580.00 & 18669180.28 & 1.00 & 0.99 & 0.99 \\
25824 & 103524 & 1997 & 49357.62 & 0.33 & 4935762.00 & 48286065.49 & 1.00 & 0.98 & 0.98 \\
1100 & 100153 & 1997 & 232.46 & 0.32 & 24941.00 & 243449.46 & 0.93 & 1.05 & 0.98 \\
11987 & 101476 & 1997 & 3286.16 & 0.27 & 295385.00 & 2547609.71 & 1.11 & 0.78 & 0.86 \\
25790 & 103523 & 1997 & 3137.48 & 0.27 & 313747.00 & 2964199.42 & 1.00 & 0.94 & 0.94 \\
25756 & 103521 & 1997 & 1795.54 & 0.30 & 179554.00 & 1784559.74 & 1.00 & 0.99 & 0.99 \\
25724 & 103520 & 1997 & 2365.33 & 0.36 & 236533.00 & 2349882.52 & 1.00 & 0.99 & 0.99 \\
8206 & 101079 & 1997 & 158.23 & 0.10 & 17910.00 & 163182.33 & 0.88 & 1.03 & 0.91 \\
39142 & 107611 & 1997 & 474.10 & 0.26 & 44355.00 & 414274.90 & 1.07 & 0.87 & 0.93 \\
12013 & 101477 & 1997 & 194.23 & 0.12 & 25538.00 & 227908.84 & 0.76 & 1.17 & 0.89 \\
25692 & 103514 & 1997 & 2099.72 & 0.31 & 209971.00 & 1928460.46 & 1.00 & 0.92 & 0.92 \\
1150 & 100157 & 1997 & 625.89 & 0.33 & 53946.00 & 569090.69 & 1.16 & 0.91 & 1.05 \\
1182 & 100159 & 1997 & 425.00 & 0.25 & 32500.00 & 337451.89 & 1.31 & 0.79 & 1.04 \\
1135 & 100155 & 1997 & 658.20 & 0.32 & 66241.00 & 629645.96 & 0.99 & 0.96 & 0.95 \\
25377 & 103481 & 1997 & 65.68 & 0.14 & 6568.00 & 59485.45 & 1.00 & 0.91 & 0.91 \\
12122 & 101511 & 1997 & 109.83 & 0.21 & 10983.00 & 94709.36 & 1.00 & 0.86 & 0.86 \\
1593 & 100217 & 1997 & 135.85 & 0.13 & 13963.00 & 124299.70 & 0.97 & 0.91 & 0.89 \\
12218 & 101523 & 1997 & 1100.01 & 0.25 & 56573.00 & 466789.11 & 1.94 & 0.42 & 0.83 \\
39623 & 107830 & 1997 & 993.00 & 0.03 & 95514.00 & 769369.74 & 1.04 & 0.77 & 0.81 \\
39633 & 107832 & 1997 & 115.90 & 0.02 & 11637.00 & 109013.09 & 1.00 & 0.94 & 0.94 \\
12227 & 101528 & 1997 & 80.70 & 0.19 & 8071.00 & 73251.34 & 1.00 & 0.91 & 0.91 \\
24990 & 103406 & 1997 & 1598.69 & 0.30 & 159869.00 & 1313130.19 & 1.00 & 0.82 & 0.82 \\
1626 & 100218 & 1997 & 20.60 & 0.75 & 2035.00 & 20361.00 & 1.01 & 0.99 & 1.00 \\
12242 & 101530 & 1997 & 2126.72 & 0.23 & 212671.00 & 1814954.91 & 1.00 & 0.85 & 0.85 \\
24938 & 103395 & 1997 & 198.54 & 0.21 & 19507.00 & 156567.02 & 1.02 & 0.79 & 0.80 \\
1681 & 100223 & 1997 & 2419.34 & 0.31 & 241934.00 & 2314088.75 & 1.00 & 0.96 & 0.96 \\
39660 & 107833 & 1997 & 574.86 & 0.02 & 58920.00 & 589218.13 & 0.98 & 1.02 & 1.00 \\
39671 & 107834 & 1997 & 178.98 & -0.26 & 18641.00 & 175194.61 & 0.96 & 0.98 & 0.94 \\
49288 & 240266 & 1997 & 4.38 & 0.02 & 347.00 & 3345.21 & 1.26 & 0.76 & 0.96 \\
39677 & 107835 & 1997 & 488.09 & 0.02 & 48764.00 & 420222.25 & 1.00 & 0.86 & 0.86 \\
1564 & 100214 & 1997 & 151.83 & 0.25 & 14067.00 & 139250.59 & 1.08 & 0.92 & 0.99 \\
25071 & 103429 & 1997 & 1067.77 & 0.07 & 106777.00 & 960858.33 & 1.00 & 0.90 & 0.90 \\
25109 & 103432 & 1997 & 1442.62 & 0.20 & 144261.00 & 1287448.07 & 1.00 & 0.89 & 0.89 \\
25350 & 103478 & 1997 & 230.03 & 0.34 & 23002.00 & 221589.08 & 1.00 & 0.96 & 0.96 \\
25309 & 103466 & 1997 & 936.41 & 0.26 & 82509.00 & 803295.60 & 1.13 & 0.86 & 0.97 \\
1416 & 100196 & 1997 & 1162.20 & 0.24 & 114963.00 & 980863.28 & 1.01 & 0.84 & 0.85 \\
25278 & 103464 & 1997 & 1356.86 & 0.04 & 165026.00 & 1367735.65 & 0.82 & 1.01 & 0.83 \\
65059 & 500659 & 1997 & 24.97 & -0.03 & 2998.00 & 25530.61 & 0.83 & 1.02 & 0.85 \\
1446 & 100200 & 1997 & 141.25 & 0.20 & 9946.00 & 98342.53 & 1.42 & 0.70 & 0.99 \\
25234 & 103463 & 1997 & 84.48 & -0.38 & 7945.00 & 69973.80 & 1.06 & 0.83 & 0.88 \\
1477 & 100207 & 1997 & 3386.24 & 0.30 & 309469.00 & 3158581.65 & 1.09 & 0.93 & 1.02 \\
25196 & 103460 & 1997 & 782.41 & 0.08 & 75398.00 & 736448.35 & 1.04 & 0.94 & 0.98 \\
1515 & 100209 & 1997 & 3573.79 & 0.21 & 333102.00 & 3193351.46 & 1.07 & 0.89 & 0.96 \\
1533 & 100213 & 1997 & 245.78 & 0.24 & 23674.00 & 236744.67 & 1.04 & 0.96 & 1.00 \\
39514 & 107716 & 1997 & 76.70 & 0.37 & 7662.00 & 76345.96 & 1.00 & 1.00 & 1.00 \\
25144 & 103439 & 1997 & 71.10 & 0.07 & 7882.00 & 63096.32 & 0.90 & 0.89 & 0.80 \\
12187 & 101518 & 1997 & 865.06 & 0.36 & 92201.00 & 919914.27 & 0.94 & 1.06 & 1.00 \\
25134 & 103436 & 1997 & 13.77 & 0.34 & 1377.00 & 13248.44 & 1.00 & 0.96 & 0.96 \\
25891 & 103526 & 1997 & 2795.30 & 0.31 & 275518.00 & 2471982.69 & 1.01 & 0.88 & 0.90 \\
1070 & 100150 & 1997 & 18.60 & 0.25 & 1800.00 & 16641.11 & 1.03 & 0.89 & 0.92 \\
11953 & 101473 & 1997 & 2515.03 & 0.34 & 204408.00 & 2077037.57 & 1.23 & 0.83 & 1.02 \\
26969 & 103640 & 1997 & 213.14 & 0.31 & 20645.00 & 218497.11 & 1.03 & 1.03 & 1.06 \\
11711 & 101457 & 1997 & 591.35 & 0.20 & 59135.00 & 552547.00 & 1.00 & 0.93 & 0.93 \\
26924 & 103628 & 1997 & 807.63 & 0.05 & 104165.00 & 1017095.06 & 0.78 & 1.26 & 0.98 \\
26911 & 103621 & 1997 & 99.50 & 0.28 & 9330.00 & 96040.92 & 1.07 & 0.97 & 1.03 \\
770 & 100096 & 1997 & 24.62 & 0.27 & 2292.00 & 19816.96 & 1.07 & 0.80 & 0.86 \\
26889 & 103620 & 1997 & 179.60 & 0.32 & 16525.00 & 160938.11 & 1.09 & 0.90 & 0.97 \\
740 & 100093 & 1997 & 592.06 & 0.09 & 60698.00 & 554953.01 & 0.98 & 0.94 & 0.91 \\
26841 & 103609 & 1997 & 46.01 & 0.28 & 4601.00 & 44550.89 & 1.00 & 0.97 & 0.97 \\
11745 & 101460 & 1997 & 8329.54 & 0.18 & 832803.00 & 7936032.53 & 1.00 & 0.95 & 0.95 \\
26791 & 103607 & 1997 & 602.52 & 0.15 & 60252.00 & 580649.69 & 1.00 & 0.96 & 0.96 \\
800 & 100097 & 1997 & 4.60 & 0.25 & 452.00 & 4207.45 & 1.02 & 0.91 & 0.93 \\
26720 & 103601 & 1997 & 133.74 & 0.41 & 13374.00 & 107445.91 & 1.00 & 0.80 & 0.80 \\
26691 & 103600 & 1997 & 21.58 & 0.39 & 2158.00 & 18770.03 & 1.00 & 0.87 & 0.87 \\
11778 & 101461 & 1997 & 1227.00 & 0.35 & 117512.00 & 1180409.08 & 1.04 & 0.96 & 1.00 \\
26818 & 103608 & 1997 & 62.16 & 0.31 & 6216.00 & 60754.01 & 1.00 & 0.98 & 0.98 \\
74558 & 601136 & 1997 & 36.52 & 0.27 & 3651.00 & 35352.31 & 1.00 & 0.97 & 0.97 \\
8315 & 101082 & 1997 & 1631.54 & 0.31 & 105991.00 & 1472098.57 & 1.54 & 0.90 & 1.39 \\
74625 & 601142 & 1997 & 310.93 & 0.16 & 49476.00 & 437428.42 & 0.63 & 1.41 & 0.88 \\
27126 & 105243 & 1997 & 32.23 & 0.19 & 3426.00 & 30090.09 & 0.94 & 0.93 & 0.88 \\
682 & 100090 & 1997 & 362.91 & 0.23 & 35749.00 & 328768.41 & 1.02 & 0.91 & 0.92 \\
27113 & 103658 & 1997 & 1233.80 & 0.18 & 123512.00 & 1126740.23 & 1.00 & 0.91 & 0.91 \\
706 & 100092 & 1997 & 172.65 & 0.15 & 17199.00 & 155973.18 & 1.00 & 0.90 & 0.91 \\
27036 & 103645 & 1997 & 1082.14 & 0.18 & 108214.00 & 1006069.27 & 1.00 & 0.93 & 0.93 \\
11678 & 101456 & 1997 & 61.64 & 0.37 & 5395.00 & 57005.12 & 1.14 & 0.92 & 1.06 \\
27010 & 103644 & 1997 & 259.49 & 0.33 & 22896.00 & 251710.98 & 1.13 & 0.97 & 1.10 \\
74586 & 601139 & 1997 & 3452.83 & 0.26 & 336275.00 & 3197757.56 & 1.03 & 0.93 & 0.95 \\
26984 & 103643 & 1997 & 27.68 & 0.34 & 2431.00 & 25855.95 & 1.14 & 0.93 & 1.06 \\
27094 & 103652 & 1997 & 564.50 & 0.23 & 56725.00 & 539024.56 & 1.00 & 0.95 & 0.95 \\
26647 & 103595 & 1997 & 128.53 & 0.24 & 12853.00 & 122643.95 & 1.00 & 0.95 & 0.95 \\
26615 & 103593 & 1997 & 38245.98 & 0.33 & 3824598.00 & 34818449.44 & 1.00 & 0.91 & 0.91 \\
830 & 100098 & 1997 & 6.37 & 0.20 & 643.00 & 6018.56 & 0.99 & 0.95 & 0.94 \\
38627 & 107302 & 1997 & 15.75 & 0.03 & 1604.00 & 15545.56 & 0.98 & 0.99 & 0.97 \\
26211 & 103546 & 1997 & 20914.56 & 0.22 & 2067807.00 & 20133439.76 & 1.01 & 0.96 & 0.97 \\
976 & 100113 & 1997 & 1272.49 & 0.33 & 130104.00 & 1299517.05 & 0.98 & 1.02 & 1.00 \\
26171 & 103545 & 1997 & 24376.46 & 0.32 & 2437646.00 & 22316321.16 & 1.00 & 0.92 & 0.92 \\
11902 & 101465 & 1997 & 66.18 & 0.28 & 6618.00 & 62492.63 & 1.00 & 0.94 & 0.94 \\
26137 & 103544 & 1997 & 6596.46 & 0.31 & 659641.00 & 6115166.46 & 1.00 & 0.93 & 0.93 \\
26103 & 103539 & 1997 & 684.65 & 0.25 & 60460.00 & 550031.34 & 1.13 & 0.80 & 0.91 \\
26242 & 103547 & 1997 & 8054.60 & 0.26 & 744525.00 & 7473707.57 & 1.08 & 0.93 & 1.00 \\
26056 & 103536 & 1997 & 579.29 & 0.33 & 49961.00 & 507648.80 & 1.16 & 0.88 & 1.02 \\
26026 & 103535 & 1997 & 892.51 & 0.27 & 85293.00 & 814748.86 & 1.05 & 0.91 & 0.96 \\
11933 & 101466 & 1997 & 128.69 & -0.13 & 15250.00 & 150214.63 & 0.84 & 1.17 & 0.99 \\
25958 & 103531 & 1997 & 1181.41 & 0.15 & 112431.00 & 931182.67 & 1.05 & 0.79 & 0.83 \\
1020 & 100127 & 1997 & 1971.45 & 0.04 & 202946.00 & 1752017.15 & 0.97 & 0.89 & 0.86 \\
932 & 100112 & 1997 & 1446.15 & 0.28 & 136689.00 & 1329249.55 & 1.06 & 0.92 & 0.97 \\
11872 & 101464 & 1997 & 387.58 & 0.30 & 30974.00 & 266323.74 & 1.25 & 0.69 & 0.86 \\
26289 & 103564 & 1997 & 1377.28 & 0.23 & 131325.00 & 1284280.24 & 1.05 & 0.93 & 0.98 \\
26581 & 103592 & 1997 & 61.05 & -0.01 & 6105.00 & 60661.64 & 1.00 & 0.99 & 0.99 \\
26550 & 103591 & 1997 & 146.75 & 0.28 & 14675.00 & 139736.85 & 1.00 & 0.95 & 0.95 \\
8274 & 101081 & 1997 & 548.73 & 0.42 & 44798.00 & 487652.69 & 1.22 & 0.89 & 1.09 \\
11809 & 101462 & 1997 & 1194.68 & 0.28 & 115912.00 & 1101758.87 & 1.03 & 0.92 & 0.95 \\
26518 & 103590 & 1997 & 172.21 & 0.19 & 18897.00 & 155588.63 & 0.91 & 0.90 & 0.82 \\
26481 & 103582 & 1997 & 15.57 & -0.02 & 1558.00 & 14485.75 & 1.00 & 0.93 & 0.93 \\
73360 & 600006 & 1997 & 103.27 & 0.30 & 10467.00 & 84748.79 & 0.99 & 0.82 & 0.81 \\
860 & 100099 & 1997 & 31.33 & 0.14 & 3150.00 & 29809.00 & 0.99 & 0.95 & 0.95 \\
26430 & 103580 & 1997 & 416.97 & 0.25 & 41000.00 & 367529.76 & 1.02 & 0.88 & 0.90 \\
26398 & 103579 & 1997 & 268.81 & 0.20 & 25991.00 & 220024.31 & 1.03 & 0.82 & 0.85 \\
11841 & 101463 & 1997 & 390.11 & 0.32 & 54717.00 & 510558.40 & 0.71 & 1.31 & 0.93 \\
26383 & 103573 & 1997 & 348.43 & 0.27 & 32710.00 & 321866.97 & 1.07 & 0.92 & 0.98 \\
890 & 100101 & 1997 & 236.14 & 0.20 & 22910.00 & 210665.84 & 1.03 & 0.89 & 0.92 \\
26372 & 103572 & 1997 & 42.72 & 0.33 & 3615.00 & 41495.48 & 1.18 & 0.97 & 1.15 \\
26339 & 103570 & 1997 & 17.83 & 0.18 & 1783.00 & 14505.78 & 1.00 & 0.81 & 0.81 \\
26307 & 103567 & 1997 & 2609.09 & 0.29 & 250817.00 & 2465793.70 & 1.04 & 0.95 & 0.98 \\
25920 & 103529 & 1997 & 2141.81 & 0.36 & 214181.00 & 2060017.74 & 1.00 & 0.96 & 0.96 \\
25626 & 103498 & 1997 & 116.60 & 0.32 & 9577.00 & 96314.32 & 1.22 & 0.83 & 1.01 \\
21630 & 102924 & 1997 & 33.60 & 0.09 & 3888.00 & 38462.62 & 0.86 & 1.14 & 0.99 \\
22714 & 103034 & 1997 & 314.99 & 0.22 & 43570.00 & 385292.70 & 0.72 & 1.22 & 0.88 \\
17249 & 102274 & 1997 & 1871.59 & 0.35 & 184193.00 & 1737545.26 & 1.02 & 0.93 & 0.94 \\
5245 & 100741 & 1997 & 284.77 & 0.14 & 30863.00 & 275364.16 & 0.92 & 0.97 & 0.89 \\
47885 & 222658 & 1997 & 57.97 & 0.05 & 5797.00 & 54896.59 & 1.00 & 0.95 & 0.95 \\
57930 & 410003 & 1997 & 1190.18 & 0.17 & 119018.00 & 1086892.46 & 1.00 & 0.91 & 0.91 \\
5267 & 100745 & 1997 & 2433.06 & 0.26 & 233165.00 & 2390524.46 & 1.04 & 0.98 & 1.03 \\
5212 & 100736 & 1997 & 396.59 & 0.06 & 47396.00 & 432561.87 & 0.84 & 1.09 & 0.91 \\
14385 & 101853 & 1997 & 615.07 & 0.38 & 45782.00 & 454810.87 & 1.34 & 0.74 & 0.99 \\
17188 & 102270 & 1997 & 304.48 & 0.36 & 31726.00 & 318695.89 & 0.96 & 1.05 & 1.00 \\
5288 & 100746 & 1997 & 1491.41 & 0.27 & 136974.00 & 1445677.80 & 1.09 & 0.97 & 1.06 \\
14401 & 101854 & 1997 & 3815.97 & 0.38 & 296117.00 & 2962878.22 & 1.29 & 0.78 & 1.00 \\
17159 & 102261 & 1997 & 2478.59 & 0.26 & 196329.00 & 1898478.14 & 1.26 & 0.77 & 0.97 \\
57917 & 402004 & 1997 & 27.44 & 0.23 & 2571.00 & 22829.64 & 1.07 & 0.83 & 0.89 \\
6976 & 100978 & 1997 & 78.76 & 0.20 & 8312.00 & 68978.39 & 0.95 & 0.88 & 0.83 \\
17145 & 102259 & 1997 & 422.35 & 0.20 & 39007.00 & 357889.13 & 1.08 & 0.85 & 0.92 \\
14351 & 101851 & 1997 & 1423.07 & 0.23 & 143217.00 & 1246688.31 & 0.99 & 0.88 & 0.87 \\
17289 & 102278 & 1997 & 460.60 & 0.21 & 45968.00 & 437334.29 & 1.00 & 0.95 & 0.95 \\
14300 & 101843 & 1997 & 356.10 & 0.02 & 35751.00 & 345771.56 & 1.00 & 0.97 & 0.97 \\
5131 & 100726 & 1997 & 4076.49 & 0.32 & 349085.00 & 3644864.17 & 1.17 & 0.89 & 1.04 \\
17402 & 102286 & 1997 & 37.94 & 0.11 & 3794.00 & 33750.90 & 1.00 & 0.89 & 0.89 \\
17381 & 102284 & 1997 & 307.40 & 0.26 & 30139.00 & 276263.14 & 1.02 & 0.90 & 0.92 \\
5154 & 100727 & 1997 & 498.65 & 0.25 & 56121.00 & 533333.66 & 0.89 & 1.07 & 0.95 \\
14314 & 101849 & 1997 & 21.14 & 0.16 & 2561.00 & 21179.11 & 0.83 & 1.00 & 0.83 \\
5172 & 100730 & 1997 & 810.40 & 0.13 & 87960.00 & 783509.51 & 0.92 & 0.97 & 0.89 \\
6988 & 100981 & 1997 & 49.73 & 0.25 & 4889.00 & 47523.34 & 1.02 & 0.96 & 0.97 \\
14326 & 101850 & 1997 & 171.95 & 0.24 & 17243.00 & 153171.27 & 1.00 & 0.89 & 0.89 \\
17307 & 102280 & 1997 & 3834.21 & 0.29 & 383007.00 & 3590223.92 & 1.00 & 0.94 & 0.94 \\
57964 & 410010 & 1997 & 143.38 & 0.30 & 14335.00 & 133433.24 & 1.00 & 0.93 & 0.93 \\
47917 & 222809 & 1997 & 152.38 & 0.28 & 15238.00 & 140537.38 & 1.00 & 0.92 & 0.92 \\
5194 & 100731 & 1997 & 15721.87 & 0.26 & 1484418.00 & 13817846.21 & 1.06 & 0.88 & 0.93 \\
7002 & 100982 & 1997 & 1592.10 & 0.26 & 159210.00 & 1330004.56 & 1.00 & 0.84 & 0.84 \\
7014 & 100985 & 1997 & 171.22 & 0.30 & 17242.00 & 142361.96 & 0.99 & 0.83 & 0.83 \\
57910 & 402003 & 1997 & 169.65 & 0.23 & 16334.00 & 146510.73 & 1.04 & 0.86 & 0.90 \\
17135 & 102258 & 1997 & 973.75 & 0.30 & 60552.00 & 591989.58 & 1.61 & 0.61 & 0.98 \\
5322 & 100753 & 1997 & 2384.09 & 0.29 & 232363.00 & 2029668.91 & 1.03 & 0.85 & 0.87 \\
6908 & 100968 & 1997 & 55.84 & 0.22 & 5584.00 & 53300.89 & 1.00 & 0.95 & 0.95 \\
16874 & 102213 & 1997 & 765.38 & 0.43 & 76538.00 & 717252.36 & 1.00 & 0.94 & 0.94 \\
14510 & 101871 & 1997 & 526.75 & 0.23 & 50056.00 & 525698.88 & 1.05 & 1.00 & 1.05 \\
57846 & 401145 & 1997 & 27.00 & 0.06 & 2710.00 & 26014.08 & 1.00 & 0.96 & 0.96 \\
16844 & 102197 & 1997 & 94.20 & 0.28 & 8178.00 & 79259.63 & 1.15 & 0.84 & 0.97 \\
5465 & 100764 & 1997 & 356.47 & 0.33 & 33542.00 & 314577.91 & 1.06 & 0.88 & 0.94 \\
57838 & 401082 & 1997 & 16.30 & 0.19 & 1390.00 & 15556.74 & 1.17 & 0.95 & 1.12 \\
47742 & 221051 & 1997 & 3932.23 & 0.19 & 401368.00 & 3501507.74 & 0.98 & 0.89 & 0.87 \\
16788 & 102192 & 1997 & 125.11 & 0.33 & 12511.00 & 109085.49 & 1.00 & 0.87 & 0.87 \\
16772 & 102191 & 1997 & 49.25 & 0.29 & 4927.00 & 48423.06 & 1.00 & 0.98 & 0.98 \\
5534 & 100771 & 1997 & 97.63 & 0.19 & 9507.00 & 95120.06 & 1.03 & 0.97 & 1.00 \\
5443 & 100763 & 1997 & 679.29 & 0.19 & 61515.00 & 621826.39 & 1.10 & 0.92 & 1.01 \\
47809 & 222027 & 1997 & 118.09 & 0.11 & 12286.00 & 112387.62 & 0.96 & 0.95 & 0.91 \\
5343 & 100754 & 1997 & 1154.58 & 0.31 & 110464.00 & 1086792.45 & 1.05 & 0.94 & 0.98 \\
45878 & 200153 & 1997 & 10.96 & 0.30 & 1096.00 & 10034.23 & 1.00 & 0.92 & 0.92 \\
17103 & 102257 & 1997 & 605.52 & 0.32 & 74721.00 & 748827.52 & 0.81 & 1.24 & 1.00 \\
17080 & 102255 & 1997 & 132.72 & 0.24 & 18226.00 & 179659.95 & 0.73 & 1.35 & 0.99 \\
6937 & 100973 & 1997 & 28.34 & 0.25 & 3131.00 & 31315.93 & 0.91 & 1.10 & 1.00 \\
14445 & 101858 & 1997 & 176.18 & 0.27 & 16665.00 & 159363.02 & 1.06 & 0.90 & 0.96 \\
17072 & 102241 & 1997 & 183.94 & 0.09 & 19194.00 & 182756.99 & 0.96 & 0.99 & 0.95 \\
57899 & 401372 & 1997 & 9.29 & 0.23 & 961.00 & 9168.40 & 0.97 & 0.99 & 0.95 \\
5366 & 100758 & 1997 & 85.17 & 0.31 & 8110.00 & 80008.06 & 1.05 & 0.94 & 0.99 \\
17037 & 102231 & 1997 & 757.00 & 0.17 & 78886.00 & 709834.22 & 0.96 & 0.94 & 0.90 \\
14466 & 101861 & 1997 & 727.17 & 0.26 & 64633.00 & 607281.09 & 1.13 & 0.84 & 0.94 \\
5399 & 100760 & 1997 & 892.54 & 0.14 & 92562.00 & 880618.84 & 0.96 & 0.99 & 0.95 \\
17001 & 102230 & 1997 & 42.70 & 0.24 & 4291.00 & 36807.87 & 1.00 & 0.86 & 0.86 \\
16965 & 102224 & 1997 & 2789.29 & 0.25 & 278927.00 & 2663590.12 & 1.00 & 0.95 & 0.95 \\
5111 & 100724 & 1997 & 65.79 & 0.33 & 6609.00 & 55688.54 & 1.00 & 0.85 & 0.84 \\
47944 & 225413 & 1997 & 13.38 & 0.04 & 1338.00 & 12788.09 & 1.00 & 0.96 & 0.96 \\
5080 & 100723 & 1997 & 35.65 & 0.33 & 3571.00 & 35102.06 & 1.00 & 0.98 & 0.98 \\
14094 & 101802 & 1997 & 340.15 & 0.30 & 31293.00 & 304124.87 & 1.09 & 0.89 & 0.97 \\
18040 & 102387 & 1997 & 60.01 & 0.40 & 6001.00 & 57409.09 & 1.00 & 0.96 & 0.96 \\
18005 & 102386 & 1997 & 196.90 & 0.31 & 19690.00 & 195520.94 & 1.00 & 0.99 & 0.99 \\
4838 & 100685 & 1997 & 15.69 & 0.14 & 1952.00 & 19076.99 & 0.80 & 1.22 & 0.98 \\
7111 & 100997 & 1997 & 77.18 & 0.31 & 7387.00 & 66110.71 & 1.04 & 0.86 & 0.89 \\
17965 & 102377 & 1997 & 140.06 & -0.09 & 18626.00 & 161590.24 & 0.75 & 1.15 & 0.87 \\
17931 & 102376 & 1997 & 19.56 & 0.28 & 1956.00 & 17817.05 & 1.00 & 0.91 & 0.91 \\
17881 & 102371 & 1997 & 139.92 & 0.38 & 8266.00 & 78829.67 & 1.69 & 0.56 & 0.95 \\
4868 & 100687 & 1997 & 274.44 & 0.33 & 24592.00 & 264329.41 & 1.12 & 0.96 & 1.07 \\
14129 & 101805 & 1997 & 1382.89 & 0.13 & 154574.00 & 1435372.45 & 0.89 & 1.04 & 0.93 \\
4873 & 100688 & 1997 & 171.55 & 0.31 & 15239.00 & 173491.07 & 1.13 & 1.01 & 1.14 \\
17868 & 102367 & 1997 & 995.23 & 0.22 & 98659.00 & 983740.64 & 1.01 & 0.99 & 1.00 \\
4886 & 100691 & 1997 & 658.54 & 0.31 & 62169.00 & 558179.50 & 1.06 & 0.85 & 0.90 \\
4805 & 100682 & 1997 & 82.81 & 0.25 & 7430.00 & 81335.31 & 1.11 & 0.98 & 1.09 \\
48066 & 235413 & 1997 & 48.01 & 0.21 & 4943.00 & 49438.70 & 0.97 & 1.03 & 1.00 \\
4593 & 100642 & 1997 & 1310.50 & 0.24 & 102743.00 & 1208646.99 & 1.28 & 0.92 & 1.18 \\
18352 & 102446 & 1997 & 12.15 & 0.26 & 1215.00 & 11148.57 & 1.00 & 0.92 & 0.92 \\
4639 & 100659 & 1997 & 301.43 & 0.32 & 30143.00 & 292780.09 & 1.00 & 0.97 & 0.97 \\
4673 & 100660 & 1997 & 364.80 & 0.29 & 38817.00 & 386026.13 & 0.94 & 1.06 & 0.99 \\
7143 & 100998 & 1997 & 140.72 & 0.20 & 15047.00 & 139331.56 & 0.94 & 0.99 & 0.93 \\
18224 & 102417 & 1997 & 1650.28 & 0.10 & 173974.00 & 1389143.98 & 0.95 & 0.84 & 0.80 \\
4737 & 100670 & 1997 & 87.90 & 0.23 & 8790.00 & 84362.49 & 1.00 & 0.96 & 0.96 \\
18177 & 102414 & 1997 & 1349.11 & 0.26 & 126773.00 & 1051411.16 & 1.06 & 0.78 & 0.83 \\
4759 & 100671 & 1997 & 584.91 & 0.24 & 55634.00 & 486063.85 & 1.05 & 0.83 & 0.87 \\
18127 & 102404 & 1997 & 718.23 & 0.33 & 74788.00 & 751382.86 & 0.96 & 1.05 & 1.00 \\
18118 & 102399 & 1997 & 13.69 & 0.17 & 1327.00 & 13104.04 & 1.03 & 0.96 & 0.99 \\
4700 & 100667 & 1997 & 17.30 & 0.29 & 1753.00 & 14937.38 & 0.99 & 0.86 & 0.85 \\
7080 & 100996 & 1997 & 1514.40 & 0.33 & 138807.00 & 1429544.47 & 1.09 & 0.94 & 1.03 \\
17840 & 102365 & 1997 & 710.74 & 0.25 & 71563.00 & 695211.84 & 0.99 & 0.98 & 0.97 \\
17617 & 102321 & 1997 & 328.02 & 0.33 & 32800.00 & 292212.87 & 1.00 & 0.89 & 0.89 \\
47998 & 225696 & 1997 & 11.50 & 0.01 & 1111.00 & 10688.65 & 1.04 & 0.93 & 0.96 \\
7043 & 100992 & 1997 & 539.78 & 0.26 & 55512.00 & 524846.44 & 0.97 & 0.97 & 0.95 \\
14240 & 101835 & 1997 & 880.76 & 0.33 & 80314.00 & 814250.18 & 1.10 & 0.92 & 1.01 \\
17580 & 102319 & 1997 & 1025.00 & 0.37 & 102500.00 & 1006018.72 & 1.00 & 0.98 & 0.98 \\
5014 & 100700 & 1997 & 377.62 & 0.26 & 40413.00 & 403965.16 & 0.93 & 1.07 & 1.00 \\
17546 & 102318 & 1997 & 5257.07 & 0.37 & 525707.00 & 4941783.25 & 1.00 & 0.94 & 0.94 \\
14225 & 101834 & 1997 & 44.55 & 0.36 & 3861.00 & 37453.12 & 1.15 & 0.84 & 0.97 \\
5024 & 100701 & 1997 & 352.69 & 0.11 & 39633.00 & 316819.71 & 0.89 & 0.90 & 0.80 \\
5048 & 100710 & 1997 & 289.89 & 0.27 & 28989.00 & 278393.15 & 1.00 & 0.96 & 0.96 \\
14273 & 101842 & 1997 & 1645.55 & 0.32 & 152579.00 & 1574165.19 & 1.08 & 0.96 & 1.03 \\
17495 & 102314 & 1997 & 460.78 & 0.11 & 79723.00 & 683598.38 & 0.58 & 1.48 & 0.86 \\
58041 & 410060 & 1997 & 25.58 & 0.18 & 2778.00 & 26926.10 & 0.92 & 1.05 & 0.97 \\
17484 & 102313 & 1997 & 425.89 & 0.17 & 48212.00 & 456696.85 & 0.88 & 1.07 & 0.95 \\
47972 & 225484 & 1997 & 114.97 & 0.20 & 11724.00 & 104157.52 & 0.98 & 0.91 & 0.89 \\
47985 & 225687 & 1997 & 550.95 & 0.03 & 73033.00 & 587352.74 & 0.75 & 1.07 & 0.80 \\
4995 & 100698 & 1997 & 115.86 & 0.11 & 11838.00 & 110439.21 & 0.98 & 0.95 & 0.93 \\
17653 & 102334 & 1997 & 201.10 & 0.32 & 18731.00 & 193221.91 & 1.07 & 0.96 & 1.03 \\
45331 & 200039 & 1997 & 20.34 & 0.00 & 1825.00 & 17021.39 & 1.11 & 0.84 & 0.93 \\
14162 & 101819 & 1997 & 203.14 & 0.33 & 18849.00 & 182097.04 & 1.08 & 0.90 & 0.97 \\
17809 & 102364 & 1997 & 980.04 & 0.33 & 92282.00 & 930921.58 & 1.06 & 0.95 & 1.01 \\
4906 & 100692 & 1997 & 1998.64 & 0.21 & 197282.00 & 1762225.72 & 1.01 & 0.88 & 0.89 \\
17796 & 102358 & 1997 & 4.50 & 0.14 & 449.00 & 4305.05 & 1.00 & 0.96 & 0.96 \\
4936 & 100695 & 1997 & 145.83 & 0.20 & 14583.00 & 132010.45 & 1.00 & 0.91 & 0.91 \\
17749 & 102356 & 1997 & 2.17 & 0.27 & 219.00 & 2023.87 & 0.99 & 0.93 & 0.92 \\
17728 & 102350 & 1997 & 104.38 & 0.32 & 9828.00 & 98951.25 & 1.06 & 0.95 & 1.01 \\
14194 & 101820 & 1997 & 260.68 & 0.31 & 33600.00 & 336015.96 & 0.78 & 1.29 & 1.00 \\
17705 & 102349 & 1997 & 422.03 & 0.30 & 40324.00 & 398121.09 & 1.05 & 0.94 & 0.99 \\
4962 & 100697 & 1997 & 71.62 & 0.23 & 7030.00 & 69996.18 & 1.02 & 0.98 & 1.00 \\
17674 & 102342 & 1997 & 108.62 & 0.37 & 8824.00 & 96239.40 & 1.23 & 0.89 & 1.09 \\
17434 & 102306 & 1997 & 7940.19 & 0.33 & 794916.00 & 7007137.97 & 1.00 & 0.88 & 0.88 \\
14552 & 101876 & 1997 & 228.06 & 0.30 & 23310.00 & 223650.74 & 0.98 & 0.98 & 0.96 \\
5601 & 100773 & 1997 & 1437.26 & 0.23 & 150943.00 & 1413457.60 & 0.95 & 0.98 & 0.94 \\
6876 & 100967 & 1997 & 331.81 & 0.39 & 33316.00 & 295590.97 & 1.00 & 0.89 & 0.89 \\
14872 & 101919 & 1997 & 527.80 & 0.30 & 40709.00 & 405674.41 & 1.30 & 0.77 & 1.00 \\
15615 & 102009 & 1997 & 426.69 & 0.23 & 38364.00 & 377766.56 & 1.11 & 0.89 & 0.98 \\
15585 & 102007 & 1997 & 4796.20 & 0.20 & 454119.00 & 4044151.32 & 1.06 & 0.84 & 0.89 \\
15570 & 102005 & 1997 & 979.57 & 0.23 & 95550.00 & 922313.71 & 1.03 & 0.94 & 0.97 \\
6228 & 100831 & 1997 & 259.33 & 0.16 & 25978.00 & 238398.54 & 1.00 & 0.92 & 0.92 \\
6255 & 100833 & 1997 & 804.10 & 0.29 & 75608.00 & 738665.19 & 1.06 & 0.92 & 0.98 \\
15526 & 102000 & 1997 & 626.93 & 0.29 & 62663.00 & 594224.53 & 1.00 & 0.95 & 0.95 \\
15636 & 102010 & 1997 & 9272.46 & 0.20 & 940987.00 & 8305001.04 & 0.99 & 0.90 & 0.88 \\
15495 & 101999 & 1997 & 3669.01 & 0.23 & 377387.00 & 2944399.26 & 0.97 & 0.80 & 0.78 \\
15476 & 101998 & 1997 & 875.09 & 0.31 & 89740.00 & 801461.81 & 0.98 & 0.92 & 0.89 \\
6299 & 100847 & 1997 & 3.28 & 0.15 & 328.00 & 2944.76 & 1.00 & 0.90 & 0.90 \\
6667 & 100908 & 1997 & 120.73 & 0.20 & 12072.00 & 119019.15 & 1.00 & 0.99 & 0.99 \\
14909 & 101922 & 1997 & 151.76 & 0.27 & 14299.00 & 142383.96 & 1.06 & 0.94 & 1.00 \\
15450 & 101991 & 1997 & 142.13 & 0.37 & 14406.00 & 142590.27 & 0.99 & 1.00 & 0.99 \\
6684 & 100910 & 1997 & 118.68 & 0.28 & 11884.00 & 106130.04 & 1.00 & 0.89 & 0.89 \\
6328 & 100849 & 1997 & 77.84 & 0.12 & 7797.00 & 74884.65 & 1.00 & 0.96 & 0.96 \\
6698 & 100913 & 1997 & 69.87 & 0.44 & 7013.00 & 61461.53 & 1.00 & 0.88 & 0.88 \\
15676 & 102013 & 1997 & 1140.09 & 0.26 & 111749.00 & 1023058.34 & 1.02 & 0.90 & 0.92 \\
15888 & 102052 & 1997 & 435.52 & 0.12 & 48097.00 & 480986.81 & 0.91 & 1.10 & 1.00 \\
6112 & 100823 & 1997 & 49.81 & 0.31 & 4981.00 & 49854.88 & 1.00 & 1.00 & 1.00 \\
14828 & 101916 & 1997 & 278.59 & 0.26 & 20900.00 & 196810.49 & 1.33 & 0.71 & 0.94 \\
15786 & 102018 & 1997 & 650.29 & 0.24 & 65208.00 & 637988.00 & 1.00 & 0.98 & 0.98 \\
6194 & 100829 & 1997 & 866.12 & 0.20 & 80197.00 & 846315.58 & 1.08 & 0.98 & 1.06 \\
15755 & 102017 & 1997 & 6046.57 & 0.31 & 455335.00 & 4524094.28 & 1.33 & 0.75 & 0.99 \\
6142 & 100824 & 1997 & 11.84 & 0.12 & 1223.00 & 11292.59 & 0.97 & 0.95 & 0.92 \\
6146 & 100825 & 1997 & 165.50 & 0.16 & 16196.00 & 132334.35 & 1.02 & 0.80 & 0.82 \\
47536 & 212658 & 1997 & 1838.65 & 0.18 & 183865.00 & 1682731.87 & 1.00 & 0.92 & 0.92 \\
15696 & 102015 & 1997 & 432.59 & 0.25 & 45728.00 & 416938.76 & 0.95 & 0.96 & 0.91 \\
15721 & 102016 & 1997 & 1859.61 & 0.48 & 187906.00 & 1699800.91 & 0.99 & 0.91 & 0.90 \\
15422 & 101990 & 1997 & 184.51 & 0.23 & 14514.00 & 135139.27 & 1.27 & 0.73 & 0.93 \\
15394 & 101989 & 1997 & 216.14 & 0.23 & 20160.00 & 201721.18 & 1.07 & 0.93 & 1.00 \\
14995 & 101930 & 1997 & 1021.03 & 0.28 & 99474.00 & 994757.12 & 1.03 & 0.97 & 1.00 \\
15172 & 101964 & 1997 & 412.59 & 0.29 & 41259.00 & 381502.17 & 1.00 & 0.92 & 0.92 \\
6502 & 100878 & 1997 & 1980.94 & 0.31 & 152598.00 & 1914027.29 & 1.30 & 0.97 & 1.25 \\
15006 & 101933 & 1997 & 284.33 & -0.00 & 31142.00 & 265809.75 & 0.91 & 0.93 & 0.85 \\
15143 & 101963 & 1997 & 1523.60 & 0.22 & 145233.00 & 1396995.61 & 1.05 & 0.92 & 0.96 \\
15214 & 101968 & 1997 & 73.43 & 0.13 & 11015.00 & 105663.85 & 0.67 & 1.44 & 0.96 \\
6538 & 100889 & 1997 & 54.28 & 0.29 & 5441.00 & 55628.77 & 1.00 & 1.02 & 1.02 \\
6547 & 100890 & 1997 & 796.09 & 0.28 & 82658.00 & 826598.90 & 0.96 & 1.04 & 1.00 \\
15093 & 101956 & 1997 & 2085.96 & 0.24 & 202845.00 & 1996139.02 & 1.03 & 0.96 & 0.98 \\
6583 & 100892 & 1997 & 179.50 & 0.13 & 17716.00 & 164899.38 & 1.01 & 0.92 & 0.93 \\
6579 & 100891 & 1997 & 264.56 & 0.26 & 28242.00 & 265549.37 & 0.94 & 1.00 & 0.94 \\
47408 & 210770 & 1997 & 849.16 & 0.28 & 149000.00 & 1478521.02 & 0.57 & 1.74 & 0.99 \\
15042 & 101953 & 1997 & 388.88 & 0.27 & 38888.00 & 379635.23 & 1.00 & 0.98 & 0.98 \\
15111 & 101958 & 1997 & 435.69 & 0.32 & 39225.00 & 358128.55 & 1.11 & 0.82 & 0.91 \\
6634 & 100906 & 1997 & 1531.19 & 0.24 & 153097.00 & 1527461.72 & 1.00 & 1.00 & 1.00 \\
47333 & 210203 & 1997 & 2906.31 & 0.31 & 290631.00 & 2806653.96 & 1.00 & 0.97 & 0.97 \\
14982 & 101926 & 1997 & 364.93 & 0.32 & 35859.00 & 358595.51 & 1.02 & 0.98 & 1.00 \\
47184 & 200342 & 1997 & 7913.75 & 0.21 & 785483.00 & 6845953.55 & 1.01 & 0.87 & 0.87 \\
47468 & 212027 & 1997 & 27.64 & 0.25 & 2501.00 & 24407.88 & 1.11 & 0.88 & 0.98 \\
15364 & 101988 & 1997 & 526.31 & 0.28 & 33173.00 & 323054.13 & 1.59 & 0.61 & 0.97 \\
6376 & 100856 & 1997 & 190.99 & 0.27 & 19099.00 & 181943.32 & 1.00 & 0.95 & 0.95 \\
15334 & 101987 & 1997 & 438.19 & 0.33 & 44816.00 & 423105.58 & 0.98 & 0.97 & 0.94 \\
14950 & 101925 & 1997 & 343.33 & 0.27 & 30776.00 & 315844.33 & 1.12 & 0.92 & 1.03 \\
47249 & 200344 & 1997 & 1667.26 & 0.29 & 145529.00 & 1525726.37 & 1.15 & 0.92 & 1.05 \\
15298 & 101982 & 1997 & 288.46 & 0.26 & 22738.00 & 281045.16 & 1.27 & 0.97 & 1.24 \\
6427 & 100868 & 1997 & 177.03 & 0.31 & 16054.00 & 165115.76 & 1.10 & 0.93 & 1.03 \\
47456 & 211485 & 1997 & 527.77 & 0.29 & 43283.00 & 439746.34 & 1.22 & 0.83 & 1.02 \\
15249 & 101972 & 1997 & 719.97 & 0.24 & 67826.00 & 574497.79 & 1.06 & 0.80 & 0.85 \\
15233 & 101970 & 1997 & 64.89 & 0.27 & 6724.00 & 65329.20 & 0.96 & 1.01 & 0.97 \\
6453 & 100875 & 1997 & 192.94 & 0.33 & 18996.00 & 187408.04 & 1.02 & 0.97 & 0.99 \\
6081 & 100822 & 1997 & 18.80 & 0.12 & 1880.00 & 18886.48 & 1.00 & 1.00 & 1.00 \\
15911 & 102059 & 1997 & 594.80 & 0.18 & 61126.00 & 532201.19 & 0.97 & 0.89 & 0.87 \\
14802 & 101914 & 1997 & 64.62 & 0.29 & 5619.00 & 55188.27 & 1.15 & 0.85 & 0.98 \\
15941 & 102061 & 1997 & 8.59 & 0.21 & 859.00 & 6988.29 & 1.00 & 0.81 & 0.81 \\
6810 & 100958 & 1997 & 3.73 & 0.27 & 291.00 & 2902.59 & 1.28 & 0.78 & 1.00 \\
14626 & 101903 & 1997 & 1300.00 & 0.09 & 137193.00 & 1267234.89 & 0.95 & 0.97 & 0.92 \\
5707 & 100789 & 1997 & 20.75 & 0.02 & 2294.00 & 22009.07 & 0.90 & 1.06 & 0.96 \\
16602 & 102159 & 1997 & 1.42 & 0.00 & 154.00 & 1465.54 & 0.92 & 1.03 & 0.95 \\
16588 & 102157 & 1997 & 2.76 & 0.19 & 288.00 & 2885.19 & 0.96 & 1.05 & 1.00 \\
6832 & 100962 & 1997 & 2541.26 & 0.15 & 254126.00 & 2260547.78 & 1.00 & 0.89 & 0.89 \\
5725 & 100790 & 1997 & 183.36 & 0.28 & 17811.00 & 148631.98 & 1.03 & 0.81 & 0.83 \\
16537 & 102154 & 1997 & 214.91 & 0.26 & 19745.00 & 174138.14 & 1.09 & 0.81 & 0.88 \\
14656 & 101906 & 1997 & 44.30 & 0.32 & 4461.00 & 39081.09 & 0.99 & 0.88 & 0.88 \\
16514 & 102152 & 1997 & 233.56 & 0.35 & 21660.00 & 225307.83 & 1.08 & 0.96 & 1.04 \\
6788 & 100954 & 1997 & 1686.81 & 0.31 & 134869.00 & 1105487.59 & 1.25 & 0.66 & 0.82 \\
14670 & 101908 & 1997 & 9.10 & 0.18 & 853.00 & 7956.91 & 1.07 & 0.87 & 0.93 \\
16612 & 102163 & 1997 & 1364.59 & 0.24 & 113755.00 & 964037.69 & 1.20 & 0.71 & 0.85 \\
16624 & 102166 & 1997 & 229.28 & 0.24 & 20495.00 & 229246.18 & 1.12 & 1.00 & 1.12 \\
16739 & 102183 & 1997 & 59.57 & 0.11 & 6791.00 & 56349.04 & 0.88 & 0.95 & 0.83 \\
16722 & 102182 & 1997 & 88.22 & 0.29 & 10007.00 & 101105.10 & 0.88 & 1.15 & 1.01 \\
6868 & 100966 & 1997 & 31.86 & 0.15 & 3211.00 & 30724.17 & 0.99 & 0.96 & 0.96 \\
5620 & 100775 & 1997 & 648.77 & 0.36 & 49039.00 & 574270.49 & 1.32 & 0.89 & 1.17 \\
16718 & 102179 & 1997 & 42.91 & 0.37 & 4269.00 & 40769.47 & 1.01 & 0.95 & 0.96 \\
5645 & 100784 & 1997 & 4087.00 & 0.28 & 507700.00 & 3636720.11 & 0.81 & 0.89 & 0.72 \\
16688 & 102178 & 1997 & 619.37 & 0.31 & 58373.00 & 575189.78 & 1.06 & 0.93 & 0.99 \\
5675 & 100785 & 1997 & 282.58 & 0.36 & 29236.00 & 292152.85 & 0.97 & 1.03 & 1.00 \\
46264 & 200205 & 1997 & 110.51 & -0.13 & 11971.00 & 90903.20 & 0.92 & 0.82 & 0.76 \\
16645 & 102173 & 1997 & 64.44 & 0.35 & 6298.00 & 54895.40 & 1.02 & 0.85 & 0.87 \\
16492 & 102151 & 1997 & 21.91 & 0.29 & 2191.00 & 20477.02 & 1.00 & 0.93 & 0.93 \\
48138 & 240027 & 1997 & 56.00 & 0.31 & 6009.00 & 58324.01 & 0.93 & 1.04 & 0.97 \\
16461 & 102150 & 1997 & 163.41 & 0.26 & 16341.00 & 156489.42 & 1.00 & 0.96 & 0.96 \\
5939 & 100812 & 1997 & 499.86 & 0.15 & 54125.00 & 474300.35 & 0.92 & 0.95 & 0.88 \\
16105 & 102080 & 1997 & 1030.34 & 0.24 & 103034.00 & 893886.86 & 1.00 & 0.87 & 0.87 \\
5972 & 100815 & 1997 & 281.69 & 0.21 & 27638.00 & 265267.90 & 1.02 & 0.94 & 0.96 \\
5980 & 100817 & 1997 & 40.48 & 0.35 & 3468.00 & 35895.44 & 1.17 & 0.89 & 1.04 \\
16148 & 102087 & 1997 & 341.36 & 0.28 & 34039.00 & 293970.30 & 1.00 & 0.86 & 0.86 \\
14770 & 101913 & 1997 & 32.27 & 0.24 & 4305.00 & 40525.97 & 0.75 & 1.26 & 0.94 \\
6009 & 100818 & 1997 & 305.57 & -0.13 & 40589.00 & 333242.75 & 0.75 & 1.09 & 0.82 \\
47619 & 215696 & 1997 & 214.41 & 0.17 & 21786.00 & 204030.71 & 0.98 & 0.95 & 0.94 \\
6019 & 100820 & 1997 & 121.66 & 0.32 & 12166.00 & 121881.70 & 1.00 & 1.00 & 1.00 \\
16023 & 102073 & 1997 & 11277.56 & 0.29 & 1127756.00 & 9972471.26 & 1.00 & 0.88 & 0.88 \\
6050 & 100821 & 1997 & 71.94 & 0.30 & 6799.00 & 69451.10 & 1.06 & 0.97 & 1.02 \\
15979 & 102062 & 1997 & 332.21 & 0.19 & 33261.00 & 291782.26 & 1.00 & 0.88 & 0.88 \\
16074 & 102079 & 1997 & 690.36 & 0.19 & 70963.00 & 593890.95 & 0.97 & 0.86 & 0.84 \\
16167 & 102089 & 1997 & 201.52 & 0.28 & 19352.00 & 187707.48 & 1.04 & 0.93 & 0.97 \\
16198 & 102090 & 1997 & 503.19 & 0.31 & 48436.00 & 464023.50 & 1.04 & 0.92 & 0.96 \\
14738 & 101912 & 1997 & 2605.07 & 0.28 & 260507.00 & 2692420.09 & 1.00 & 1.03 & 1.03 \\
32833 & 106067 & 1997 & 24.78 & 0.27 & 1662.00 & 16540.14 & 1.49 & 0.67 & 1.00 \\
16411 & 102134 & 1997 & 135.88 & 0.17 & 13213.00 & 121396.70 & 1.03 & 0.89 & 0.92 \\
16402 & 102133 & 1997 & 48.91 & 0.06 & 5292.00 & 42925.70 & 0.92 & 0.88 & 0.81 \\
16394 & 102132 & 1997 & 57.74 & 0.27 & 5363.00 & 53317.71 & 1.08 & 0.92 & 0.99 \\
5781 & 100792 & 1997 & 1179.36 & 0.20 & 126314.00 & 1108199.43 & 0.93 & 0.94 & 0.88 \\
16359 & 102130 & 1997 & 835.10 & 0.26 & 81812.00 & 809956.34 & 1.02 & 0.97 & 0.99 \\
5821 & 100804 & 1997 & 1978.81 & 0.28 & 169359.00 & 1666625.92 & 1.17 & 0.84 & 0.98 \\
14704 & 101911 & 1997 & 915.26 & 0.29 & 91526.00 & 773463.13 & 1.00 & 0.85 & 0.85 \\
16305 & 102124 & 1997 & 1545.70 & 0.40 & 144842.00 & 1501757.03 & 1.07 & 0.97 & 1.04 \\
16270 & 102113 & 1997 & 222.30 & 0.10 & 22220.00 & 212425.62 & 1.00 & 0.96 & 0.96 \\
5868 & 100809 & 1997 & 2614.77 & 0.11 & 278166.00 & 2385526.20 & 0.94 & 0.91 & 0.86 \\
16254 & 102105 & 1997 & 191.90 & 0.27 & 19191.00 & 172996.04 & 1.00 & 0.90 & 0.90 \\
5900 & 100811 & 1997 & 1943.99 & 0.16 & 191241.00 & 1637126.66 & 1.02 & 0.84 & 0.86 \\
16214 & 102095 & 1997 & 41.85 & 0.24 & 3744.00 & 32230.68 & 1.12 & 0.77 & 0.86 \\
16439 & 102145 & 1997 & 104.39 & 0.04 & 8335.00 & 86904.43 & 1.25 & 0.83 & 1.04 \\
2609 & 100347 & 1997 & 1006.71 & 0.32 & 96284.00 & 1004618.16 & 1.05 & 1.00 & 1.04 \\
18396 & 102447 & 1997 & 816.37 & 0.29 & 81637.00 & 712408.94 & 1.00 & 0.87 & 0.87 \\
14019 & 101800 & 1997 & 897.17 & 0.32 & 91998.00 & 825773.68 & 0.98 & 0.92 & 0.90 \\
21320 & 102848 & 1997 & 86.03 & 0.28 & 8345.00 & 80910.90 & 1.03 & 0.94 & 0.97 \\
3254 & 100419 & 1997 & 9.54 & 0.16 & 996.00 & 8331.94 & 0.96 & 0.87 & 0.84 \\
21308 & 102847 & 1997 & 101.40 & 0.25 & 10450.00 & 103300.58 & 0.97 & 1.02 & 0.99 \\
21299 & 102846 & 1997 & 101.76 & 0.28 & 10350.00 & 102717.31 & 0.98 & 1.01 & 0.99 \\
21274 & 102844 & 1997 & 497.76 & 0.34 & 49951.00 & 487160.92 & 1.00 & 0.98 & 0.98 \\
21250 & 102843 & 1997 & 353.06 & 0.32 & 35582.00 & 345563.79 & 0.99 & 0.98 & 0.97 \\
21327 & 102852 & 1997 & 248.85 & 0.19 & 24918.00 & 229828.54 & 1.00 & 0.92 & 0.92 \\
21192 & 102837 & 1997 & 483.67 & 0.44 & 49872.00 & 457322.33 & 0.97 & 0.95 & 0.92 \\
7491 & 101042 & 1997 & 4480.44 & 0.28 & 436758.00 & 4215960.54 & 1.03 & 0.94 & 0.97 \\
13327 & 101728 & 1997 & 122.40 & 0.04 & 11969.00 & 119697.98 & 1.02 & 0.98 & 1.00 \\
21160 & 102835 & 1997 & 109.65 & 0.27 & 11198.00 & 103307.70 & 0.98 & 0.94 & 0.92 \\
3328 & 100424 & 1997 & 39.92 & 0.14 & 5065.00 & 50651.02 & 0.79 & 1.27 & 1.00 \\
21138 & 102833 & 1997 & 20.93 & 0.15 & 2093.00 & 20916.55 & 1.00 & 1.00 & 1.00 \\
21106 & 102832 & 1997 & 60.47 & 0.31 & 6047.00 & 60478.30 & 1.00 & 1.00 & 1.00 \\
21228 & 102838 & 1997 & 259.20 & 0.04 & 25439.00 & 247990.77 & 1.02 & 0.96 & 0.97 \\
21358 & 102854 & 1997 & 756.86 & 0.14 & 75810.00 & 755466.27 & 1.00 & 1.00 & 1.00 \\
21391 & 102861 & 1997 & 91.44 & 0.27 & 8599.00 & 89584.47 & 1.06 & 0.98 & 1.04 \\
21619 & 102901 & 1997 & 66.09 & 0.30 & 6719.00 & 66433.71 & 0.98 & 1.01 & 0.99 \\
21593 & 102895 & 1997 & 1449.04 & 0.24 & 150101.00 & 1455523.49 & 0.97 & 1.00 & 0.97 \\
3180 & 100413 & 1997 & 36.65 & 0.14 & 3665.00 & 31179.87 & 1.00 & 0.85 & 0.85 \\
3238 & 100417 & 1997 & 6.16 & 0.23 & 663.00 & 6425.42 & 0.93 & 1.04 & 0.97 \\
21505 & 102876 & 1997 & 76.85 & 0.28 & 7702.00 & 77226.35 & 1.00 & 1.00 & 1.00 \\
21450 & 102872 & 1997 & 533.55 & 0.18 & 53354.00 & 455366.83 & 1.00 & 0.85 & 0.85 \\
21419 & 102871 & 1997 & 199.52 & 0.25 & 19952.00 & 178711.17 & 1.00 & 0.90 & 0.90 \\
13270 & 101716 & 1997 & 39.65 & -0.02 & 4316.00 & 37521.29 & 0.92 & 0.95 & 0.87 \\
3210 & 100415 & 1997 & 507.74 & 0.25 & 55186.00 & 504017.24 & 0.92 & 0.99 & 0.91 \\
7524 & 101043 & 1997 & 1981.52 & 0.19 & 180824.00 & 2001570.99 & 1.10 & 1.01 & 1.11 \\
13279 & 101717 & 1997 & 53.40 & 0.33 & 5427.00 & 53407.80 & 0.98 & 1.00 & 0.98 \\
21494 & 102875 & 1997 & 29.46 & 0.30 & 3084.00 & 30747.94 & 0.96 & 1.04 & 1.00 \\
21095 & 102829 & 1997 & 42.78 & 0.16 & 3817.00 & 37711.75 & 1.12 & 0.88 & 0.99 \\
21086 & 102828 & 1997 & 136.09 & 0.24 & 12417.00 & 127058.60 & 1.10 & 0.93 & 1.02 \\
21063 & 102827 & 1997 & 224.12 & 0.30 & 21386.00 & 223628.98 & 1.05 & 1.00 & 1.05 \\
3493 & 100441 & 1997 & 834.60 & 0.23 & 86149.00 & 750499.43 & 0.97 & 0.90 & 0.87 \\
20765 & 102789 & 1997 & 1176.17 & 0.26 & 108276.00 & 920173.44 & 1.09 & 0.78 & 0.85 \\
3522 & 100453 & 1997 & 138.05 & 0.24 & 13814.00 & 123013.52 & 1.00 & 0.89 & 0.89 \\
20726 & 102788 & 1997 & 280.02 & 0.00 & 34531.00 & 324162.92 & 0.81 & 1.16 & 0.94 \\
7427 & 101039 & 1997 & 3808.03 & 0.30 & 360914.00 & 2971108.95 & 1.06 & 0.78 & 0.82 \\
20707 & 102784 & 1997 & 19444.95 & 0.26 & 1809239.00 & 17495181.14 & 1.07 & 0.90 & 0.97 \\
13455 & 101740 & 1997 & 19200.19 & 0.23 & 1902933.00 & 15949598.71 & 1.01 & 0.83 & 0.84 \\
20804 & 102795 & 1997 & 308.74 & 0.24 & 28624.00 & 247717.79 & 1.08 & 0.80 & 0.87 \\
20668 & 102783 & 1997 & 1507.22 & 0.26 & 132467.00 & 1196249.77 & 1.14 & 0.79 & 0.90 \\
3553 & 100455 & 1997 & 7.14 & 0.21 & 636.00 & 5342.76 & 1.12 & 0.75 & 0.84 \\
3562 & 100456 & 1997 & 2.54 & 0.25 & 235.00 & 2571.38 & 1.08 & 1.01 & 1.09 \\
20616 & 102775 & 1997 & 1436.86 & 0.26 & 139116.00 & 1221035.08 & 1.03 & 0.85 & 0.88 \\
20596 & 102774 & 1997 & 4495.48 & 0.23 & 417258.00 & 4301954.40 & 1.08 & 0.96 & 1.03 \\
3575 & 100457 & 1997 & 92.22 & 0.22 & 9083.00 & 81158.29 & 1.02 & 0.88 & 0.89 \\
13480 & 101741 & 1997 & 2004.61 & 0.35 & 199218.00 & 1992216.89 & 1.01 & 0.99 & 1.00 \\
3587 & 100460 & 1997 & 27.96 & 0.15 & 2974.00 & 25766.27 & 0.94 & 0.92 & 0.87 \\
20652 & 102777 & 1997 & 3744.32 & 0.10 & 388026.00 & 3743186.88 & 0.96 & 1.00 & 0.96 \\
3459 & 100439 & 1997 & 44.77 & 0.05 & 5604.00 & 50906.23 & 0.80 & 1.14 & 0.91 \\
20854 & 102797 & 1997 & 69.97 & 0.28 & 6997.00 & 61947.16 & 1.00 & 0.89 & 0.89 \\
21056 & 102825 & 1997 & 88.55 & 0.14 & 9601.00 & 92467.76 & 0.92 & 1.04 & 0.96 \\
7460 & 101040 & 1997 & 2557.43 & 0.33 & 235808.00 & 2444370.66 & 1.08 & 0.96 & 1.04 \\
21021 & 102824 & 1997 & 100.20 & 0.26 & 9558.00 & 87710.14 & 1.05 & 0.88 & 0.92 \\
21013 & 102823 & 1997 & 52.28 & 0.25 & 4988.00 & 48012.15 & 1.05 & 0.92 & 0.96 \\
13379 & 101736 & 1997 & 42.88 & 0.26 & 4288.00 & 41009.38 & 1.00 & 0.96 & 0.96 \\
21002 & 102821 & 1997 & 314.12 & 0.11 & 37041.00 & 344835.10 & 0.85 & 1.10 & 0.93 \\
3393 & 100431 & 1997 & 228.95 & 0.13 & 28588.00 & 222626.70 & 0.80 & 0.97 & 0.78 \\
20991 & 102818 & 1997 & 164.99 & 0.32 & 15806.00 & 147333.48 & 1.04 & 0.89 & 0.93 \\
20975 & 102814 & 1997 & 55.99 & 0.40 & 4937.00 & 48054.54 & 1.13 & 0.86 & 0.97 \\
20945 & 102813 & 1997 & 536.44 & 0.33 & 51838.00 & 457650.68 & 1.03 & 0.85 & 0.88 \\
20936 & 102812 & 1997 & 92.23 & 0.23 & 9199.00 & 83170.10 & 1.00 & 0.90 & 0.90 \\
20924 & 102802 & 1997 & 710.62 & 0.30 & 69600.00 & 622558.57 & 1.02 & 0.88 & 0.89 \\
3444 & 100435 & 1997 & 9.36 & -0.56 & 1522.00 & 7614.12 & 0.61 & 0.81 & 0.50 \\
13411 & 101738 & 1997 & 573.11 & 0.11 & 55183.00 & 457460.70 & 1.04 & 0.80 & 0.83 \\
20910 & 102799 & 1997 & 646.97 & 0.19 & 66141.00 & 616069.13 & 0.98 & 0.95 & 0.93 \\
20882 & 102798 & 1997 & 970.89 & 0.17 & 96082.00 & 930404.70 & 1.01 & 0.96 & 0.97 \\
29102 & 105527 & 1997 & 3.90 & 0.32 & 206.00 & 2038.79 & 1.90 & 0.52 & 0.99 \\
20556 & 102767 & 1997 & 2372.08 & 0.34 & 217012.00 & 2171622.73 & 1.09 & 0.92 & 1.00 \\
21674 & 102939 & 1997 & 3304.29 & 0.32 & 303700.00 & 3190620.04 & 1.09 & 0.97 & 1.05 \\
12997 & 101618 & 1997 & 579.39 & 0.25 & 58181.00 & 515715.80 & 1.00 & 0.89 & 0.89 \\
2718 & 100355 & 1997 & 1793.77 & 0.33 & 123428.00 & 1293360.88 & 1.45 & 0.72 & 1.05 \\
13011 & 101621 & 1997 & 4231.44 & 0.25 & 434960.00 & 3399436.33 & 0.97 & 0.80 & 0.78 \\
2749 & 100357 & 1997 & 206.55 & 0.36 & 20996.00 & 203144.36 & 0.98 & 0.98 & 0.97 \\
22330 & 103007 & 1997 & 1615.73 & 0.27 & 158730.00 & 1585421.63 & 1.02 & 0.98 & 1.00 \\
22294 & 103005 & 1997 & 126.19 & 0.21 & 12319.00 & 111638.92 & 1.02 & 0.88 & 0.91 \\
13032 & 101622 & 1997 & 1305.91 & 0.15 & 130970.00 & 1247815.17 & 1.00 & 0.96 & 0.95 \\
2793 & 100358 & 1997 & 1718.58 & 0.14 & 189905.00 & 1533419.45 & 0.90 & 0.89 & 0.81 \\
22242 & 102997 & 1997 & 4565.90 & 0.22 & 487432.00 & 3907172.16 & 0.94 & 0.86 & 0.80 \\
13046 & 101623 & 1997 & 2448.49 & 0.24 & 246711.00 & 2028836.26 & 0.99 & 0.83 & 0.82 \\
2815 & 100360 & 1997 & 784.28 & 0.16 & 90268.00 & 733788.34 & 0.87 & 0.94 & 0.81 \\
22273 & 102999 & 1997 & 989.10 & 0.17 & 102873.00 & 906180.19 & 0.96 & 0.92 & 0.88 \\
22214 & 102996 & 1997 & 320.10 & 0.22 & 33198.00 & 317938.18 & 0.96 & 0.99 & 0.96 \\
7637 & 101050 & 1997 & 174.98 & 0.23 & 15614.00 & 144964.47 & 1.12 & 0.83 & 0.93 \\
22483 & 103015 & 1997 & 80.52 & 0.38 & 7174.00 & 76799.32 & 1.12 & 0.95 & 1.07 \\
2628 & 100348 & 1997 & 152.31 & 0.22 & 14609.00 & 147601.90 & 1.04 & 0.97 & 1.01 \\
22678 & 103028 & 1997 & 4188.70 & 0.12 & 404188.00 & 3668467.29 & 1.04 & 0.88 & 0.91 \\
2647 & 100350 & 1997 & 120.53 & 0.33 & 12053.00 & 123910.77 & 1.00 & 1.03 & 1.03 \\
22634 & 103027 & 1997 & 798.66 & 0.28 & 84055.00 & 774278.95 & 0.95 & 0.97 & 0.92 \\
7662 & 101054 & 1997 & 6529.61 & 0.32 & 550868.00 & 5771952.58 & 1.19 & 0.88 & 1.05 \\
12947 & 101616 & 1997 & 14625.87 & 0.24 & 1476792.00 & 13396824.90 & 0.99 & 0.92 & 0.91 \\
22597 & 103024 & 1997 & 258.35 & 0.31 & 25835.00 & 224131.96 & 1.00 & 0.87 & 0.87 \\
22465 & 103014 & 1997 & 470.14 & 0.34 & 42940.00 & 437369.39 & 1.09 & 0.93 & 1.02 \\
2666 & 100351 & 1997 & 154.59 & 0.31 & 15940.00 & 151212.27 & 0.97 & 0.98 & 0.95 \\
22525 & 103017 & 1997 & 1452.75 & 0.33 & 129957.00 & 1296813.63 & 1.12 & 0.89 & 1.00 \\
22498 & 103016 & 1997 & 281.51 & 0.24 & 14981.00 & 143833.52 & 1.88 & 0.51 & 0.96 \\
22566 & 103021 & 1997 & 49.73 & 0.28 & 4973.00 & 47812.94 & 1.00 & 0.96 & 0.96 \\
2826 & 100362 & 1997 & 100.41 & 0.26 & 9002.00 & 94011.08 & 1.12 & 0.94 & 1.04 \\
22183 & 102994 & 1997 & 74.93 & 0.27 & 7655.00 & 68067.75 & 0.98 & 0.91 & 0.89 \\
2850 & 100365 & 1997 & 234.12 & 0.34 & 22499.00 & 203872.50 & 1.04 & 0.87 & 0.91 \\
13153 & 101681 & 1997 & 313.60 & 0.18 & 31362.00 & 268723.69 & 1.00 & 0.86 & 0.86 \\
21881 & 102964 & 1997 & 352.50 & 0.35 & 29798.00 & 251356.02 & 1.18 & 0.71 & 0.84 \\
21851 & 102957 & 1997 & 307.40 & 0.33 & 28806.00 & 295535.15 & 1.07 & 0.96 & 1.03 \\
13166 & 101698 & 1997 & 379.10 & 0.28 & 37906.00 & 352506.73 & 1.00 & 0.93 & 0.93 \\
21889 & 102969 & 1997 & 599.23 & 0.32 & 47657.00 & 574349.46 & 1.26 & 0.96 & 1.21 \\
21810 & 102952 & 1997 & 862.20 & 0.07 & 98287.00 & 917237.78 & 0.88 & 1.06 & 0.93 \\
3051 & 100401 & 1997 & 389.03 & 0.27 & 38381.00 & 311776.43 & 1.01 & 0.80 & 0.81 \\
21766 & 102951 & 1997 & 6832.80 & 0.30 & 568736.00 & 4995900.16 & 1.20 & 0.73 & 0.88 \\
63001 & 500466 & 1997 & 215.01 & 0.23 & 21501.00 & 193346.48 & 1.00 & 0.90 & 0.90 \\
21748 & 102949 & 1997 & 2230.20 & 0.33 & 221273.00 & 2169923.35 & 1.01 & 0.97 & 0.98 \\
3081 & 100408 & 1997 & 149.37 & 0.34 & 14936.00 & 134166.17 & 1.00 & 0.90 & 0.90 \\
21704 & 102940 & 1997 & 519.94 & 0.26 & 49164.00 & 488798.73 & 1.06 & 0.94 & 0.99 \\
7555 & 101045 & 1997 & 16977.45 & 0.27 & 1665038.00 & 15996827.01 & 1.02 & 0.94 & 0.96 \\
13179 & 101703 & 1997 & 39717.67 & 0.30 & 3971767.00 & 32432836.38 & 1.00 & 0.82 & 0.82 \\
2991 & 100395 & 1997 & 681.42 & 0.25 & 65978.00 & 644954.76 & 1.03 & 0.95 & 0.98 \\
21906 & 102979 & 1997 & 95.96 & 0.30 & 9799.00 & 88996.21 & 0.98 & 0.93 & 0.91 \\
2951 & 100389 & 1997 & 405.33 & 0.28 & 38644.00 & 320705.49 & 1.05 & 0.79 & 0.83 \\
13069 & 101626 & 1997 & 994.93 & 0.29 & 99215.00 & 874886.95 & 1.00 & 0.88 & 0.88 \\
2859 & 100366 & 1997 & 1.64 & 0.03 & 143.00 & 1290.69 & 1.14 & 0.79 & 0.90 \\
22139 & 102993 & 1997 & 2218.09 & 0.37 & 211710.00 & 2016391.19 & 1.05 & 0.91 & 0.95 \\
22105 & 102990 & 1997 & 963.59 & 0.37 & 86683.00 & 857916.69 & 1.11 & 0.89 & 0.99 \\
2878 & 100368 & 1997 & 198.20 & 0.25 & 19918.00 & 169411.56 & 1.00 & 0.85 & 0.85 \\
22044 & 102988 & 1997 & 39.04 & 0.32 & 3505.00 & 33521.26 & 1.11 & 0.86 & 0.96 \\
7587 & 101047 & 1997 & 320.09 & 0.34 & 32009.00 & 308283.74 & 1.00 & 0.96 & 0.96 \\
22013 & 102987 & 1997 & 522.76 & 0.08 & 56362.00 & 485183.76 & 0.93 & 0.93 & 0.86 \\
2887 & 100369 & 1997 & 306.90 & 0.33 & 29876.00 & 284527.16 & 1.03 & 0.93 & 0.95 \\
21999 & 102984 & 1997 & 150.12 & 0.28 & 17377.00 & 164159.98 & 0.86 & 1.09 & 0.94 \\
2912 & 100379 & 1997 & 581.88 & 0.28 & 55946.00 & 571886.71 & 1.04 & 0.98 & 1.02 \\
13123 & 101668 & 1997 & 96.49 & 0.29 & 9749.00 & 86625.42 & 0.99 & 0.90 & 0.89 \\
21987 & 102983 & 1997 & 144.26 & 0.32 & 16424.00 & 145664.04 & 0.88 & 1.01 & 0.89 \\
21954 & 102981 & 1997 & 83.55 & 0.17 & 9047.00 & 82991.73 & 0.92 & 0.99 & 0.92 \\
63175 & 500486 & 1997 & 385.55 & 0.16 & 40450.00 & 371441.14 & 0.95 & 0.96 & 0.92 \\
3103 & 100409 & 1997 & 589.83 & 0.19 & 61736.00 & 550109.65 & 0.96 & 0.93 & 0.89 \\
20519 & 102761 & 1997 & 32837.59 & 0.24 & 3144769.00 & 31373546.66 & 1.04 & 0.96 & 1.00 \\
3619 & 100463 & 1997 & 8.05 & 0.24 & 1065.00 & 9400.13 & 0.76 & 1.17 & 0.88 \\
20497 & 102760 & 1997 & 1088.70 & 0.23 & 108843.00 & 994316.69 & 1.00 & 0.91 & 0.91 \\
19076 & 102548 & 1997 & 570.30 & 0.34 & 43325.00 & 434568.48 & 1.32 & 0.76 & 1.00 \\
19068 & 102547 & 1997 & 172.59 & 0.31 & 14022.00 & 153476.28 & 1.23 & 0.89 & 1.09 \\
19060 & 102546 & 1997 & 40.41 & 0.31 & 4865.00 & 49140.90 & 0.83 & 1.22 & 1.01 \\
19044 & 102545 & 1997 & 203.34 & 0.29 & 17980.00 & 181568.03 & 1.13 & 0.89 & 1.01 \\
19020 & 102544 & 1997 & 752.20 & 0.34 & 56195.00 & 639866.01 & 1.34 & 0.85 & 1.14 \\
4281 & 100600 & 1997 & 122.19 & 0.18 & 12219.00 & 111664.24 & 1.00 & 0.91 & 0.91 \\
18988 & 102540 & 1997 & 14.14 & 0.22 & 2483.00 & 24234.55 & 0.57 & 1.71 & 0.98 \\
13837 & 101769 & 1997 & 2682.51 & 0.16 & 268251.00 & 2629468.86 & 1.00 & 0.98 & 0.98 \\
7237 & 101015 & 1997 & 658.51 & 0.26 & 66835.00 & 614146.47 & 0.99 & 0.93 & 0.92 \\
18956 & 102531 & 1997 & 20.21 & 0.33 & 2021.00 & 19037.19 & 1.00 & 0.94 & 0.94 \\
18943 & 102529 & 1997 & 56.76 & -0.17 & 5875.00 & 57587.29 & 0.97 & 1.01 & 0.98 \\
4301 & 100603 & 1997 & 3433.00 & 0.21 & 327515.00 & 3222437.86 & 1.05 & 0.94 & 0.98 \\
18924 & 102528 & 1997 & 170.78 & 0.32 & 15979.00 & 150068.89 & 1.07 & 0.88 & 0.94 \\
13881 & 101785 & 1997 & 1150.39 & 0.07 & 120972.00 & 1055444.27 & 0.95 & 0.92 & 0.87 \\
18908 & 102527 & 1997 & 260.03 & 0.33 & 25875.00 & 253671.42 & 1.00 & 0.98 & 0.98 \\
44395 & 109300 & 1997 & 561.36 & 0.30 & 48418.00 & 461166.74 & 1.16 & 0.82 & 0.95 \\
13864 & 101781 & 1997 & 1010.23 & 0.20 & 101023.00 & 959727.38 & 1.00 & 0.95 & 0.95 \\
19104 & 102549 & 1997 & 214.56 & 0.37 & 16625.00 & 172800.59 & 1.29 & 0.81 & 1.04 \\
7275 & 101018 & 1997 & 13428.38 & 0.27 & 1287730.00 & 11487268.37 & 1.04 & 0.86 & 0.89 \\
4173 & 100567 & 1997 & 2565.27 & 0.23 & 256527.00 & 2508587.12 & 1.00 & 0.98 & 0.98 \\
19299 & 102588 & 1997 & 457.24 & 0.08 & 45724.00 & 425444.58 & 1.00 & 0.93 & 0.93 \\
19279 & 102579 & 1997 & 1516.22 & -0.04 & 183751.00 & 1526880.00 & 0.83 & 1.01 & 0.83 \\
19251 & 102575 & 1997 & 172.20 & -0.01 & 13965.00 & 131198.26 & 1.23 & 0.76 & 0.94 \\
7311 & 101020 & 1997 & 2109.63 & 0.27 & 158561.00 & 1870701.02 & 1.33 & 0.89 & 1.18 \\
19135 & 102550 & 1997 & 39.01 & 0.27 & 3339.00 & 36246.02 & 1.17 & 0.93 & 1.09 \\
19219 & 102570 & 1997 & 262.28 & 0.19 & 25961.00 & 251733.04 & 1.01 & 0.96 & 0.97 \\
4210 & 100575 & 1997 & 13.51 & 0.23 & 1350.00 & 13449.57 & 1.00 & 1.00 & 1.00 \\
4224 & 100590 & 1997 & 71.78 & 0.18 & 7329.00 & 61540.86 & 0.98 & 0.86 & 0.84 \\
19200 & 102563 & 1997 & 371.24 & 0.13 & 37864.00 & 326770.68 & 0.98 & 0.88 & 0.86 \\
19176 & 102559 & 1997 & 116.90 & 0.36 & 11349.00 & 104614.58 & 1.03 & 0.89 & 0.92 \\
13801 & 101764 & 1997 & 482.16 & 0.26 & 48858.00 & 433767.97 & 0.99 & 0.90 & 0.89 \\
48239 & 240056 & 1997 & 220.25 & 0.27 & 19772.00 & 159102.06 & 1.11 & 0.72 & 0.80 \\
18846 & 102524 & 1997 & 1430.91 & 0.33 & 127945.00 & 1263930.81 & 1.12 & 0.88 & 0.99 \\
18815 & 102523 & 1997 & 659.65 & 0.25 & 65965.00 & 622336.06 & 1.00 & 0.94 & 0.94 \\
4486 & 100635 & 1997 & 151.55 & 0.25 & 15155.00 & 140091.30 & 1.00 & 0.92 & 0.92 \\
18587 & 102490 & 1997 & 79.80 & 0.33 & 7982.00 & 79091.54 & 1.00 & 0.99 & 0.99 \\
13975 & 101794 & 1997 & 288.12 & 0.27 & 26550.00 & 282116.05 & 1.09 & 0.98 & 1.06 \\
18575 & 102489 & 1997 & 288.20 & 0.30 & 28825.00 & 285675.76 & 1.00 & 0.99 & 0.99 \\
18568 & 102486 & 1997 & 35.85 & 0.25 & 3291.00 & 31640.48 & 1.09 & 0.88 & 0.96 \\
18561 & 102483 & 1997 & 125.80 & 0.27 & 12854.00 & 126065.71 & 0.98 & 1.00 & 0.98 \\
18551 & 102482 & 1997 & 355.97 & 0.44 & 35597.00 & 309826.61 & 1.00 & 0.87 & 0.87 \\
18542 & 102474 & 1997 & 265.04 & 0.15 & 26504.00 & 242860.44 & 1.00 & 0.92 & 0.92 \\
18523 & 102470 & 1997 & 1064.26 & 0.31 & 111857.00 & 915257.13 & 0.95 & 0.86 & 0.82 \\
18510 & 102469 & 1997 & 302.42 & 0.27 & 30250.00 & 278970.55 & 1.00 & 0.92 & 0.92 \\
7162 & 101000 & 1997 & 1216.42 & 0.34 & 117402.00 & 1208553.76 & 1.04 & 0.99 & 1.03 \\
18469 & 102462 & 1997 & 11.12 & 0.07 & 968.00 & 10283.53 & 1.15 & 0.92 & 1.06 \\
18452 & 102461 & 1997 & 5540.75 & -0.03 & 630031.00 & 5351625.60 & 0.88 & 0.97 & 0.85 \\
4525 & 100637 & 1997 & 1249.72 & 0.13 & 124972.00 & 1010551.73 & 1.00 & 0.81 & 0.81 \\
4475 & 100634 & 1997 & 1324.92 & 0.26 & 132492.00 & 1301560.20 & 1.00 & 0.98 & 0.98 \\
18652 & 102500 & 1997 & 629.90 & 0.11 & 69523.00 & 696369.67 & 0.91 & 1.11 & 1.00 \\
13952 & 101789 & 1997 & 733.36 & 0.24 & 66122.00 & 688066.39 & 1.11 & 0.94 & 1.04 \\
4353 & 100611 & 1997 & 1053.80 & 0.28 & 101183.00 & 910836.08 & 1.04 & 0.86 & 0.90 \\
48209 & 240051 & 1997 & 268.89 & 0.29 & 26889.00 & 258413.20 & 1.00 & 0.96 & 0.96 \\
4379 & 100614 & 1997 & 561.43 & 0.33 & 53320.00 & 547726.30 & 1.05 & 0.98 & 1.03 \\
4397 & 100622 & 1997 & 583.98 & 0.24 & 58398.00 & 539050.00 & 1.00 & 0.92 & 0.92 \\
13914 & 101787 & 1997 & 742.78 & 0.28 & 74425.00 & 690782.36 & 1.00 & 0.93 & 0.93 \\
18772 & 102508 & 1997 & 284.73 & 0.19 & 26812.00 & 283533.01 & 1.06 & 1.00 & 1.06 \\
18692 & 102503 & 1997 & 337.40 & 0.30 & 33743.00 & 329119.68 & 1.00 & 0.98 & 0.98 \\
4435 & 100625 & 1997 & 749.63 & 0.35 & 74963.00 & 700061.12 & 1.00 & 0.93 & 0.93 \\
13933 & 101788 & 1997 & 618.16 & 0.32 & 56120.00 & 562980.55 & 1.10 & 0.91 & 1.00 \\
18681 & 102502 & 1997 & 791.28 & -0.06 & 93635.00 & 842416.85 & 0.85 & 1.06 & 0.90 \\
18671 & 102501 & 1997 & 378.51 & 0.25 & 43922.00 & 317096.26 & 0.86 & 0.84 & 0.72 \\
4450 & 100633 & 1997 & 783.91 & 0.25 & 78390.00 & 728960.42 & 1.00 & 0.93 & 0.93 \\
48177 & 240040 & 1997 & 572.29 & -0.01 & 69354.00 & 587483.83 & 0.83 & 1.03 & 0.85 \\
19320 & 102591 & 1997 & 147.89 & 0.25 & 12945.00 & 119095.39 & 1.14 & 0.81 & 0.92 \\
19330 & 102597 & 1997 & 30.52 & 0.01 & 3168.00 & 29861.97 & 0.96 & 0.98 & 0.94 \\
19364 & 102599 & 1997 & 2406.59 & 0.17 & 239712.00 & 2299505.41 & 1.00 & 0.96 & 0.96 \\
20200 & 102688 & 1997 & 87.67 & 0.29 & 7174.00 & 82888.52 & 1.22 & 0.95 & 1.16 \\
13582 & 101744 & 1997 & 1234.27 & 0.13 & 125234.00 & 1183143.45 & 0.99 & 0.96 & 0.94 \\
3781 & 100481 & 1997 & 49.61 & 0.32 & 4627.00 & 42523.89 & 1.07 & 0.86 & 0.92 \\
20185 & 102676 & 1997 & 132.38 & 0.26 & 11753.00 & 115262.20 & 1.13 & 0.87 & 0.98 \\
61285 & 500027 & 1997 & 298.45 & 0.22 & 30003.00 & 276730.01 & 0.99 & 0.93 & 0.92 \\
20172 & 102673 & 1997 & 434.22 & 0.22 & 42367.00 & 384028.80 & 1.02 & 0.88 & 0.91 \\
48444 & 240085 & 1997 & 41.75 & 0.00 & 8078.00 & 72959.34 & 0.52 & 1.75 & 0.90 \\
61313 & 500028 & 1997 & 79.39 & -0.17 & 7942.00 & 68343.84 & 1.00 & 0.86 & 0.86 \\
20151 & 102671 & 1997 & 102.86 & 0.27 & 10347.00 & 101805.39 & 0.99 & 0.99 & 0.98 \\
20140 & 102669 & 1997 & 39.01 & 0.28 & 3727.00 & 38555.30 & 1.05 & 0.99 & 1.03 \\
3813 & 100485 & 1997 & 209.43 & 0.33 & 17929.00 & 173549.69 & 1.17 & 0.83 & 0.97 \\
13614 & 101748 & 1997 & 17.76 & 0.23 & 1740.00 & 17216.10 & 1.02 & 0.97 & 0.99 \\
20107 & 102667 & 1997 & 8186.16 & 0.37 & 818616.00 & 7129326.79 & 1.00 & 0.87 & 0.87 \\
20046 & 102664 & 1997 & 2088.22 & 0.28 & 208822.00 & 1759495.77 & 1.00 & 0.84 & 0.84 \\
20012 & 102663 & 1997 & 3538.49 & 0.02 & 353849.00 & 3191360.09 & 1.00 & 0.90 & 0.90 \\
19977 & 102660 & 1997 & 2344.55 & 0.28 & 234455.00 & 2056696.40 & 1.00 & 0.88 & 0.88 \\
7389 & 101038 & 1997 & 6139.97 & 0.31 & 596074.00 & 5055478.68 & 1.03 & 0.82 & 0.85 \\
3751 & 100480 & 1997 & 79.63 & 0.33 & 7186.00 & 65509.67 & 1.11 & 0.82 & 0.91 \\
61327 & 500037 & 1997 & 5330.42 & 0.27 & 506257.00 & 4496101.76 & 1.05 & 0.84 & 0.89 \\
43075 & 109059 & 1997 & 101.82 & -0.10 & 11870.00 & 88145.26 & 0.86 & 0.87 & 0.74 \\
20468 & 102757 & 1997 & 16091.41 & 0.22 & 1433273.00 & 15187343.51 & 1.12 & 0.94 & 1.06 \\
13511 & 101742 & 1997 & 1911.45 & 0.12 & 197105.00 & 1638895.19 & 0.97 & 0.86 & 0.83 \\
20446 & 102744 & 1997 & 2241.42 & 0.29 & 207117.00 & 1891403.24 & 1.08 & 0.84 & 0.91 \\
20384 & 102733 & 1997 & 4771.05 & 0.27 & 432327.00 & 4190831.27 & 1.10 & 0.88 & 0.97 \\
20367 & 102728 & 1997 & 426.78 & 0.15 & 43937.00 & 418593.23 & 0.97 & 0.98 & 0.95 \\
3678 & 100468 & 1997 & 204.39 & 0.38 & 20439.00 & 189987.29 & 1.00 & 0.93 & 0.93 \\
20299 & 102715 & 1997 & 5463.79 & 0.23 & 487537.00 & 3981521.11 & 1.12 & 0.73 & 0.82 \\
61419 & 500064 & 1997 & 6.80 & 0.08 & 685.00 & 5810.67 & 0.99 & 0.85 & 0.85 \\
20287 & 102709 & 1997 & 3.06 & 0.25 & 305.00 & 3197.53 & 1.00 & 1.05 & 1.05 \\
13550 & 101743 & 1997 & 6003.65 & 0.30 & 587107.00 & 5271477.96 & 1.02 & 0.88 & 0.90 \\
3721 & 100475 & 1997 & 152.41 & 0.19 & 15325.00 & 144581.37 & 0.99 & 0.95 & 0.94 \\
48490 & 240090 & 1997 & 30.21 & 0.01 & 2803.00 & 25392.51 & 1.08 & 0.84 & 0.91 \\
20252 & 102696 & 1997 & 1511.23 & 0.29 & 151123.00 & 1485515.99 & 1.00 & 0.98 & 0.98 \\
3852 & 100505 & 1997 & 2.62 & -0.45 & 341.00 & 2112.26 & 0.77 & 0.81 & 0.62 \\
18426 & 102452 & 1997 & 118.11 & 0.27 & 12137.00 & 119322.49 & 0.97 & 1.01 & 0.98 \\
19947 & 102659 & 1997 & 5103.74 & 0.10 & 466307.00 & 4576490.55 & 1.09 & 0.90 & 0.98 \\
3867 & 100507 & 1997 & 24.96 & 0.22 & 2490.00 & 22961.83 & 1.00 & 0.92 & 0.92 \\
19591 & 102635 & 1997 & 536.73 & 0.33 & 49409.00 & 467627.23 & 1.09 & 0.87 & 0.95 \\
19573 & 102633 & 1997 & 502.84 & 0.35 & 47947.00 & 457568.55 & 1.05 & 0.91 & 0.95 \\
4060 & 100544 & 1997 & 489.57 & 0.27 & 48945.00 & 485328.80 & 1.00 & 0.99 & 0.99 \\
13728 & 101759 & 1997 & 81.75 & 0.10 & 8794.00 & 84596.81 & 0.93 & 1.03 & 0.96 \\
19552 & 102624 & 1997 & 736.21 & 0.31 & 59177.00 & 581211.07 & 1.24 & 0.79 & 0.98 \\
19546 & 102614 & 1997 & 191.13 & 0.23 & 20216.00 & 192438.20 & 0.95 & 1.01 & 0.95 \\
4046 & 100543 & 1997 & 722.63 & 0.24 & 72267.00 & 690032.86 & 1.00 & 0.95 & 0.95 \\
19539 & 102612 & 1997 & 124.73 & 0.24 & 14189.00 & 140408.20 & 0.88 & 1.13 & 0.99 \\
19489 & 102607 & 1997 & 936.97 & 0.07 & 105444.00 & 894944.35 & 0.89 & 0.96 & 0.85 \\
19466 & 102606 & 1997 & 7187.59 & 0.23 & 755159.00 & 6928812.27 & 0.95 & 0.96 & 0.92 \\
13745 & 101762 & 1997 & 6010.51 & 0.13 & 534131.00 & 4682340.02 & 1.13 & 0.78 & 0.88 \\
4108 & 100552 & 1997 & 558.62 & 0.15 & 62450.00 & 540763.49 & 0.89 & 0.97 & 0.87 \\
19432 & 102601 & 1997 & 4053.00 & 0.38 & 352230.00 & 3950580.92 & 1.15 & 0.97 & 1.12 \\
19398 & 102600 & 1997 & 734.43 & 0.34 & 68886.00 & 688883.03 & 1.07 & 0.94 & 1.00 \\
19506 & 102608 & 1997 & 222.69 & 0.19 & 17501.00 & 189933.04 & 1.27 & 0.85 & 1.09 \\
19601 & 102636 & 1997 & 729.81 & 0.17 & 71642.00 & 697350.21 & 1.02 & 0.96 & 0.97 \\
19633 & 102639 & 1997 & 218.12 & 0.21 & 20877.00 & 205274.14 & 1.04 & 0.94 & 0.98 \\
19919 & 102655 & 1997 & 1492.66 & 0.11 & 149266.00 & 1330281.68 & 1.00 & 0.89 & 0.89 \\
7345 & 101023 & 1997 & 21177.74 & 0.30 & 1971628.00 & 16835568.43 & 1.07 & 0.79 & 0.85 \\
3915 & 100514 & 1997 & 70.07 & 0.29 & 6961.00 & 63777.77 & 1.01 & 0.91 & 0.92 \\
19875 & 102654 & 1997 & 1757.88 & 0.21 & 175788.00 & 1636872.45 & 1.00 & 0.93 & 0.93 \\
3969 & 100535 & 1997 & 413.72 & 0.50 & 41457.00 & 388814.05 & 1.00 & 0.94 & 0.94 \\
19796 & 102652 & 1997 & 3446.72 & 0.24 & 344671.00 & 2965008.99 & 1.00 & 0.86 & 0.86 \\
19762 & 102651 & 1997 & 1908.32 & 0.33 & 190832.00 & 1867624.42 & 1.00 & 0.98 & 0.98 \\
19731 & 102650 & 1997 & 12771.18 & 0.29 & 1277118.00 & 11152551.74 & 1.00 & 0.87 & 0.87 \\
4012 & 100538 & 1997 & 1131.83 & 0.33 & 123182.00 & 1043165.82 & 0.92 & 0.92 & 0.85 \\
48344 & 240065 & 1997 & 460.21 & 0.29 & 46021.00 & 416881.69 & 1.00 & 0.91 & 0.91 \\
19667 & 102645 & 1997 & 342.27 & 0.28 & 37367.00 & 366464.24 & 0.92 & 1.07 & 0.98 \\
3859 & 100506 & 1997 & 18.27 & 0.05 & 2073.00 & 19162.32 & 0.88 & 1.05 & 0.92 \\
29125 & 105531 & 1997 & 23.29 & 0.05 & 2359.00 & 19545.69 & 0.99 & 0.84 & 0.83 \\
15059 & 101955 & 1997 & 9731.68 & 0.26 & 881158.00 & 7933899.08 & 1.10 & 0.82 & 0.90 \\
10668 & 101307 & 1997 & 428.73 & 0.19 & 37505.00 & 343787.03 & 1.14 & 0.80 & 0.92 \\
10765 & 101330 & 1997 & 7275.89 & 0.53 & 644459.00 & 6471638.53 & 1.13 & 0.89 & 1.00 \\
30112 & 105700 & 1997 & 509.08 & 0.18 & 53759.00 & 479031.73 & 0.95 & 0.94 & 0.89 \\
32041 & 105977 & 1997 & 980.58 & 0.24 & 88750.00 & 758764.40 & 1.10 & 0.77 & 0.85 \\
31011 & 105847 & 1997 & 29.69 & 0.30 & 2952.00 & 26210.80 & 1.01 & 0.88 & 0.89 \\
32026 & 105974 & 1997 & 33.04 & -0.00 & 2617.00 & 26608.19 & 1.26 & 0.81 & 1.02 \\
29516 & 105604 & 1997 & 9.85 & 0.02 & 996.00 & 9587.45 & 0.99 & 0.97 & 0.96 \\
30995 & 105846 & 1997 & 171.55 & 0.18 & 17899.00 & 158637.88 & 0.96 & 0.92 & 0.89 \\
31019 & 105848 & 1997 & 63.97 & 0.33 & 6351.00 & 56872.81 & 1.01 & 0.89 & 0.90 \\
31027 & 105849 & 1997 & 30.18 & 0.15 & 3409.00 & 28187.94 & 0.89 & 0.93 & 0.83 \\
9723 & 101179 & 1997 & 368.49 & 0.36 & 36849.00 & 338063.65 & 1.00 & 0.92 & 0.92 \\
30097 & 105686 & 1997 & 192.30 & -0.10 & 25941.00 & 243094.92 & 0.74 & 1.26 & 0.94 \\
30095 & 105685 & 1997 & 38.60 & 0.27 & 3341.00 & 31210.87 & 1.16 & 0.81 & 0.93 \\
33641 & 106158 & 1997 & 15.43 & 0.02 & 1490.00 & 13079.56 & 1.04 & 0.85 & 0.88 \\
31032 & 105851 & 1997 & 26.05 & 0.25 & 2486.00 & 22584.25 & 1.05 & 0.87 & 0.91 \\
30104 & 105694 & 1997 & 23.87 & 0.23 & 2332.00 & 22303.53 & 1.02 & 0.93 & 0.96 \\
31037 & 105852 & 1997 & 30.85 & 0.24 & 3023.00 & 29221.53 & 1.02 & 0.95 & 0.97 \\
34530 & 106250 & 1997 & 33.99 & 0.02 & 3244.00 & 30057.50 & 1.05 & 0.88 & 0.93 \\
30984 & 105845 & 1997 & 5.90 & 0.32 & 427.00 & 4006.67 & 1.38 & 0.68 & 0.94 \\
32066 & 105979 & 1997 & 56.85 & -0.00 & 5231.00 & 49314.59 & 1.09 & 0.87 & 0.94 \\
34539 & 106251 & 1997 & 34.02 & 0.02 & 3366.00 & 31399.00 & 1.01 & 0.92 & 0.93 \\
30122 & 105701 & 1997 & 246.16 & 0.30 & 22700.00 & 220726.09 & 1.08 & 0.90 & 0.97 \\
32057 & 105978 & 1997 & 116.11 & 0.13 & 14376.00 & 143763.64 & 0.81 & 1.24 & 1.00 \\
192 & 100018 & 1997 & 124.81 & 0.36 & 12481.00 & 119432.62 & 1.00 & 0.96 & 0.96 \\
97 & 100006 & 1997 & 6671.30 & 0.31 & 660601.00 & 5313672.94 & 1.01 & 0.80 & 0.80 \\
9180 & 101116 & 1997 & 1708.67 & 0.32 & 128755.00 & 1440062.32 & 1.33 & 0.84 & 1.12 \\
9702 & 101167 & 1997 & 371.51 & 0.22 & 42837.00 & 323837.95 & 0.87 & 0.87 & 0.76 \\
30977 & 105843 & 1997 & 18.57 & 0.34 & 1951.00 & 19880.81 & 0.95 & 1.07 & 1.02 \\
35727 & 106392 & 1997 & 66.72 & 0.03 & 5883.00 & 55631.52 & 1.13 & 0.83 & 0.95 \\
35732 & 106394 & 1997 & 11.68 & 0.01 & 1180.00 & 11648.69 & 0.99 & 1.00 & 0.99 \\
32069 & 105980 & 1997 & 103.03 & 0.17 & 10303.00 & 100854.42 & 1.00 & 0.98 & 0.98 \\
30088 & 105684 & 1997 & 37.68 & 0.16 & 4261.00 & 35904.71 & 0.88 & 0.95 & 0.84 \\
33695 & 106162 & 1997 & 6.70 & 0.22 & 558.00 & 4846.76 & 1.20 & 0.72 & 0.87 \\
33698 & 106163 & 1997 & 54.42 & 0.31 & 9126.00 & 80793.50 & 0.60 & 1.48 & 0.89 \\
35664 & 106384 & 1997 & 2.50 & 0.04 & 250.00 & 2089.23 & 1.00 & 0.84 & 0.84 \\
34468 & 106244 & 1997 & 2.52 & -0.01 & 252.00 & 2438.03 & 1.00 & 0.97 & 0.97 \\
31934 & 105963 & 1997 & 139.34 & 0.25 & 12687.00 & 122826.05 & 1.10 & 0.88 & 0.97 \\
35271 & 106341 & 1997 & 50.30 & 0.03 & 4983.00 & 42088.62 & 1.01 & 0.84 & 0.84 \\
31923 & 105961 & 1997 & 12.02 & 0.16 & 1181.00 & 11047.50 & 1.02 & 0.92 & 0.94 \\
9766 & 101192 & 1997 & 27.87 & 0.19 & 2924.00 & 22498.44 & 0.95 & 0.81 & 0.77 \\
31913 & 105960 & 1997 & 40.67 & 0.28 & 4025.00 & 40252.07 & 1.01 & 0.99 & 1.00 \\
35302 & 106345 & 1997 & 3.51 & -0.01 & 351.00 & 3447.37 & 1.00 & 0.98 & 0.98 \\
31908 & 105959 & 1997 & 4.67 & 0.02 & 375.00 & 3723.70 & 1.25 & 0.80 & 0.99 \\
34425 & 106239 & 1997 & 1.91 & 0.03 & 178.00 & 1723.45 & 1.08 & 0.90 & 0.97 \\
35657 & 106382 & 1997 & 32.72 & 0.02 & 3288.00 & 32593.24 & 1.00 & 1.00 & 0.99 \\
35661 & 106383 & 1997 & 83.89 & 0.02 & 9176.00 & 90647.09 & 0.91 & 1.08 & 0.99 \\
35700 & 106391 & 1997 & 72.07 & 0.00 & 7207.00 & 70548.67 & 1.00 & 0.98 & 0.98 \\
9748 & 101186 & 1997 & 350.64 & 0.26 & 32842.00 & 318364.85 & 1.07 & 0.91 & 0.97 \\
31992 & 105967 & 1997 & 1.37 & 0.04 & 177.00 & 1434.27 & 0.77 & 1.05 & 0.81 \\
10126 & 101262 & 1997 & 16.92 & 0.32 & 1746.00 & 17462.87 & 0.97 & 1.03 & 1.00 \\
33665 & 106160 & 1997 & 15.94 & 0.15 & 1347.00 & 13223.48 & 1.18 & 0.83 & 0.98 \\
35697 & 106388 & 1997 & 44.40 & 0.02 & 3808.00 & 36134.54 & 1.17 & 0.81 & 0.95 \\
8996 & 101108 & 1997 & 824.79 & 0.27 & 82748.00 & 761195.24 & 1.00 & 0.92 & 0.92 \\
30081 & 105682 & 1997 & 84.55 & 0.33 & 7414.00 & 72982.10 & 1.14 & 0.86 & 0.98 \\
9146 & 101115 & 1997 & 9872.11 & 0.33 & 865274.00 & 8939760.85 & 1.14 & 0.91 & 1.03 \\
34521 & 106249 & 1997 & 38.95 & 0.02 & 3493.00 & 35657.41 & 1.12 & 0.92 & 1.02 \\
30076 & 105681 & 1997 & 294.69 & 0.14 & 29301.00 & 234528.40 & 1.01 & 0.80 & 0.80 \\
31961 & 105964 & 1997 & 36.58 & -0.00 & 3252.00 & 29468.18 & 1.12 & 0.81 & 0.91 \\
34477 & 106247 & 1997 & 3.72 & 0.00 & 434.00 & 3914.22 & 0.86 & 1.05 & 0.90 \\
33692 & 106161 & 1997 & 9.67 & 0.26 & 725.00 & 7584.30 & 1.33 & 0.78 & 1.05 \\
35670 & 106386 & 1997 & 186.26 & 0.01 & 18974.00 & 180190.35 & 0.98 & 0.97 & 0.95 \\
8711 & 101095 & 1997 & 22.39 & 0.02 & 3447.00 & 29912.45 & 0.65 & 1.34 & 0.87 \\
31969 & 105965 & 1997 & 9.07 & 0.21 & 907.00 & 8398.43 & 1.00 & 0.93 & 0.93 \\
30043 & 105679 & 1997 & 26.15 & 0.19 & 2345.00 & 19385.57 & 1.12 & 0.74 & 0.83 \\
30135 & 105702 & 1997 & 101.73 & 0.10 & 10167.00 & 93576.71 & 1.00 & 0.92 & 0.92 \\
33409 & 106133 & 1997 & 29.10 & 0.02 & 2444.00 & 22790.41 & 1.19 & 0.78 & 0.93 \\
30852 & 105805 & 1997 & 36.91 & 0.11 & 3238.00 & 33760.67 & 1.14 & 0.91 & 1.04 \\
33416 & 106135 & 1997 & 34.55 & 0.02 & 3455.00 & 31806.56 & 1.00 & 0.92 & 0.92 \\
33443 & 106136 & 1997 & 70.04 & 0.23 & 5642.00 & 52609.64 & 1.24 & 0.75 & 0.93 \\
30880 & 105806 & 1997 & 37.72 & 0.24 & 3276.00 & 28831.57 & 1.15 & 0.76 & 0.88 \\
30208 & 105716 & 1997 & 313.22 & -0.20 & 45507.00 & 375846.63 & 0.69 & 1.20 & 0.83 \\
33470 & 106137 & 1997 & 4.01 & 0.24 & 257.00 & 2600.50 & 1.56 & 0.65 & 1.01 \\
10797 & 101331 & 1997 & 155.54 & 0.29 & 15554.00 & 131815.33 & 1.00 & 0.85 & 0.85 \\
30202 & 105708 & 1997 & 130.95 & 0.24 & 19016.00 & 132629.22 & 0.69 & 1.01 & 0.70 \\
35182 & 106333 & 1997 & 62.30 & 0.01 & 5027.00 & 47016.62 & 1.24 & 0.75 & 0.94 \\
34548 & 106254 & 1997 & 22.19 & 0.02 & 2980.00 & 19352.26 & 0.74 & 0.87 & 0.65 \\
30897 & 105807 & 1997 & 15.17 & 0.38 & 1282.00 & 12314.89 & 1.18 & 0.81 & 0.96 \\
10208 & 101274 & 1997 & 210.23 & 0.33 & 19276.00 & 196202.92 & 1.09 & 0.93 & 1.02 \\
33334 & 106116 & 1997 & 25.78 & 0.19 & 2578.00 & 24626.45 & 1.00 & 0.96 & 0.96 \\
34628 & 106262 & 1997 & 379.79 & 0.02 & 37979.00 & 350001.63 & 1.00 & 0.92 & 0.92 \\
33345 & 106123 & 1997 & 393.10 & 0.26 & 38664.00 & 380243.08 & 1.02 & 0.97 & 0.98 \\
174 & 100017 & 1997 & 81.07 & 0.33 & 8107.00 & 77148.02 & 1.00 & 0.95 & 0.95 \\
9580 & 101151 & 1997 & 212.50 & 0.24 & 21251.00 & 190095.44 & 1.00 & 0.89 & 0.89 \\
30230 & 105719 & 1997 & 39.68 & -0.01 & 2505.00 & 20051.57 & 1.58 & 0.51 & 0.80 \\
140 & 100010 & 1997 & 1805.15 & 0.25 & 169693.00 & 1681801.02 & 1.06 & 0.93 & 0.99 \\
34613 & 106258 & 1997 & 6.23 & 0.02 & 579.00 & 5467.17 & 1.08 & 0.88 & 0.94 \\
33352 & 106124 & 1997 & 22.24 & -0.02 & 1962.00 & 17212.86 & 1.13 & 0.77 & 0.88 \\
30223 & 105718 & 1997 & 73.84 & -0.14 & 9828.00 & 80781.62 & 0.75 & 1.09 & 0.82 \\
34617 & 106261 & 1997 & 133.74 & 0.02 & 13490.00 & 125905.36 & 0.99 & 0.94 & 0.93 \\
10475 & 101287 & 1997 & 421.04 & 0.27 & 39783.00 & 355677.97 & 1.06 & 0.84 & 0.89 \\
10227 & 101275 & 1997 & 448.40 & 0.32 & 40283.00 & 395211.83 & 1.11 & 0.88 & 0.98 \\
33393 & 106128 & 1997 & 72.00 & 0.01 & 5322.00 & 51336.92 & 1.35 & 0.71 & 0.96 \\
30824 & 105804 & 1997 & 76.30 & 0.24 & 7261.00 & 69691.56 & 1.05 & 0.91 & 0.96 \\
10164 & 101264 & 1997 & 247.30 & 0.12 & 24730.00 & 216440.64 & 1.00 & 0.88 & 0.88 \\
9628 & 101160 & 1997 & 585.94 & 0.24 & 58116.00 & 537675.74 & 1.01 & 0.92 & 0.93 \\
35209 & 106334 & 1997 & 116.98 & 0.02 & 13489.00 & 125615.35 & 0.87 & 1.07 & 0.93 \\
33559 & 106149 & 1997 & 116.47 & 0.33 & 10925.00 & 102304.41 & 1.07 & 0.88 & 0.94 \\
32128 & 105984 & 1997 & 65.55 & 0.02 & 6192.00 & 57856.58 & 1.06 & 0.88 & 0.93 \\
10505 & 101294 & 1997 & 23.12 & 0.17 & 2053.00 & 17981.64 & 1.13 & 0.78 & 0.88 \\
33571 & 106150 & 1997 & 16.90 & 0.03 & 1690.00 & 15300.99 & 1.00 & 0.91 & 0.91 \\
30943 & 105840 & 1997 & 8.18 & 0.31 & 632.00 & 6450.23 & 1.29 & 0.79 & 1.02 \\
30143 & 105703 & 1997 & 192.62 & 0.28 & 13512.00 & 154329.06 & 1.43 & 0.80 & 1.14 \\
35757 & 106401 & 1997 & 285.27 & 0.04 & 28527.00 & 274284.77 & 1.00 & 0.96 & 0.96 \\
33582 & 106151 & 1997 & 587.88 & 0.23 & 60275.00 & 582811.30 & 0.98 & 0.99 & 0.97 \\
32101 & 105983 & 1997 & 159.86 & 0.32 & 15493.00 & 154938.92 & 1.03 & 0.97 & 1.00 \\
9678 & 101165 & 1997 & 931.65 & 0.32 & 59697.00 & 589250.62 & 1.56 & 0.63 & 0.99 \\
30949 & 105841 & 1997 & 40.75 & 0.04 & 4095.00 & 38760.84 & 1.00 & 0.95 & 0.95 \\
32097 & 105982 & 1997 & 31.51 & 0.30 & 2892.00 & 28015.39 & 1.09 & 0.89 & 0.97 \\
29862 & 105655 & 1997 & 382.22 & 0.27 & 36910.00 & 319676.09 & 1.04 & 0.84 & 0.87 \\
30904 & 105809 & 1997 & 4.89 & 0.26 & 433.00 & 4488.85 & 1.13 & 0.92 & 1.04 \\
35244 & 106336 & 1997 & 28.00 & 0.46 & 2861.00 & 28611.52 & 0.98 & 1.02 & 1.00 \\
32207 & 106000 & 1997 & 96.43 & 0.34 & 9643.00 & 93290.24 & 1.00 & 0.97 & 0.97 \\
28 & 100003 & 1997 & 612.00 & 0.18 & 61200.00 & 586483.38 & 1.00 & 0.96 & 0.96 \\
30908 & 105811 & 1997 & 4.51 & 0.26 & 431.00 & 4246.85 & 1.05 & 0.94 & 0.99 \\
33529 & 106144 & 1997 & 15.56 & -0.00 & 1223.00 & 11081.12 & 1.27 & 0.71 & 0.91 \\
10194 & 101268 & 1997 & 1758.34 & 0.27 & 175834.00 & 1470930.01 & 1.00 & 0.84 & 0.84 \\
30911 & 105834 & 1997 & 12.95 & 0.02 & 1307.00 & 13069.39 & 0.99 & 1.01 & 1.00 \\
30175 & 105705 & 1997 & 118.64 & 0.06 & 11972.00 & 117687.02 & 0.99 & 0.99 & 0.98 \\
32179 & 105999 & 1997 & 6.88 & 0.01 & 688.00 & 6619.35 & 1.00 & 0.96 & 0.96 \\
30171 & 105704 & 1997 & 57.47 & 0.18 & 5741.00 & 57837.54 & 1.00 & 1.01 & 1.01 \\
32175 & 105997 & 1997 & 102.11 & 0.03 & 9990.00 & 102636.39 & 1.02 & 1.01 & 1.03 \\
33534 & 106147 & 1997 & 117.21 & 0.26 & 12993.00 & 111720.35 & 0.90 & 0.95 & 0.86 \\
35217 & 106335 & 1997 & 114.57 & 0.00 & 13612.00 & 110853.00 & 0.84 & 0.97 & 0.81 \\
33548 & 106148 & 1997 & 70.47 & 0.26 & 7109.00 & 67264.91 & 0.99 & 0.95 & 0.95 \\
8965 & 101107 & 1997 & 1428.00 & 0.09 & 143007.00 & 1341992.30 & 1.00 & 0.94 & 0.94 \\
30931 & 105838 & 1997 & 8.45 & 0.33 & 759.00 & 7823.19 & 1.11 & 0.93 & 1.03 \\
30913 & 105836 & 1997 & 99.24 & 0.18 & 8723.00 & 78193.62 & 1.14 & 0.79 & 0.90 \\
33725 & 106164 & 1997 & 10.90 & -0.02 & 1090.00 & 10699.60 & 1.00 & 0.98 & 0.98 \\
35314 & 106346 & 1997 & 1.20 & -0.01 & 120.00 & 1084.25 & 1.00 & 0.91 & 0.90 \\
29925 & 105658 & 1997 & 368.14 & 0.27 & 31781.00 & 309648.31 & 1.16 & 0.84 & 0.97 \\
9902 & 101211 & 1997 & 252.19 & 0.24 & 24352.00 & 223602.43 & 1.04 & 0.89 & 0.92 \\
29920 & 105657 & 1997 & 43.78 & 0.26 & 3946.00 & 42699.79 & 1.11 & 0.98 & 1.08 \\
31306 & 105877 & 1997 & 6.61 & 0.29 & 497.00 & 4730.28 & 1.33 & 0.72 & 0.95 \\
33995 & 106197 & 1997 & 30.35 & -0.00 & 3035.00 & 26895.41 & 1.00 & 0.89 & 0.89 \\
35367 & 106355 & 1997 & 22.41 & -0.04 & 2091.00 & 19682.42 & 1.07 & 0.88 & 0.94 \\
31299 & 105876 & 1997 & 172.10 & 0.15 & 18755.00 & 156844.90 & 0.92 & 0.91 & 0.84 \\
31314 & 105878 & 1997 & 217.56 & 0.35 & 21756.00 & 202340.76 & 1.00 & 0.93 & 0.93 \\
35380 & 106356 & 1997 & 15.50 & 0.00 & 1678.00 & 15910.13 & 0.92 & 1.03 & 0.95 \\
31536 & 105907 & 1997 & 4.13 & 0.13 & 456.00 & 4398.01 & 0.91 & 1.06 & 0.96 \\
63 & 100004 & 1997 & 1217.90 & 0.35 & 122838.00 & 1214239.88 & 0.99 & 1.00 & 0.99 \\
34257 & 106216 & 1997 & 125.48 & 0.02 & 9977.00 & 102918.26 & 1.26 & 0.82 & 1.03 \\
29790 & 105647 & 1997 & 120.60 & 0.35 & 12060.00 & 118756.84 & 1.00 & 0.98 & 0.98 \\
34022 & 106198 & 1997 & 329.65 & 0.01 & 31423.00 & 307013.69 & 1.05 & 0.93 & 0.98 \\
31526 & 105905 & 1997 & 21.96 & 0.08 & 2196.00 & 21672.93 & 1.00 & 0.99 & 0.99 \\
33968 & 106195 & 1997 & 58.45 & -0.00 & 5845.00 & 47714.41 & 1.00 & 0.82 & 0.82 \\
31602 & 105916 & 1997 & 163.67 & 0.15 & 17001.00 & 152365.80 & 0.96 & 0.93 & 0.90 \\
29729 & 105643 & 1997 & 371.40 & 0.29 & 37140.00 & 352665.21 & 1.00 & 0.95 & 0.95 \\
33925 & 106192 & 1997 & 260.64 & 0.02 & 24357.00 & 221576.87 & 1.07 & 0.85 & 0.91 \\
53096 & 338393 & 1997 & 17.78 & 0.16 & 1762.00 & 16389.69 & 1.01 & 0.92 & 0.93 \\
9113 & 101112 & 1997 & 1895.47 & 0.29 & 182703.00 & 1721433.47 & 1.04 & 0.91 & 0.94 \\
33952 & 106193 & 1997 & 38.45 & -0.02 & 3854.00 & 35570.79 & 1.00 & 0.93 & 0.92 \\
9885 & 101200 & 1997 & 36.08 & 0.23 & 3608.00 & 35214.46 & 1.00 & 0.98 & 0.98 \\
10603 & 101300 & 1997 & 2220.90 & 0.25 & 198390.00 & 1738803.31 & 1.12 & 0.78 & 0.88 \\
31608 & 105917 & 1997 & 17.68 & 0.31 & 1995.00 & 19796.31 & 0.89 & 1.12 & 0.99 \\
34284 & 106220 & 1997 & 145.86 & -0.03 & 13792.00 & 120848.75 & 1.06 & 0.83 & 0.88 \\
31290 & 105875 & 1997 & 46.91 & 0.03 & 4146.00 & 43097.85 & 1.13 & 0.92 & 1.04 \\
29761 & 105645 & 1997 & 63.37 & -0.08 & 6913.00 & 64640.71 & 0.92 & 1.02 & 0.94 \\
35346 & 106353 & 1997 & 18.56 & 0.02 & 1856.00 & 16328.51 & 1.00 & 0.88 & 0.88 \\
29751 & 105644 & 1997 & 157.14 & 0.29 & 15714.00 & 141086.59 & 1.00 & 0.90 & 0.90 \\
8756 & 101097 & 1997 & 441.03 & 0.26 & 35291.00 & 308342.53 & 1.25 & 0.70 & 0.87 \\
34049 & 106199 & 1997 & 59.69 & 0.04 & 5478.00 & 54780.11 & 1.09 & 0.92 & 1.00 \\
34057 & 106200 & 1997 & 92.37 & 0.05 & 8090.00 & 81222.07 & 1.14 & 0.88 & 1.00 \\
31463 & 105895 & 1997 & 79.59 & 0.29 & 8034.00 & 75049.45 & 0.99 & 0.94 & 0.93 \\
29839 & 105653 & 1997 & 6.96 & -0.01 & 696.00 & 6482.06 & 1.00 & 0.93 & 0.93 \\
9081 & 101111 & 1997 & 1351.76 & 0.46 & 97106.00 & 1086298.30 & 1.39 & 0.80 & 1.12 \\
34188 & 106211 & 1997 & 102.50 & 0.03 & 9498.00 & 89110.38 & 1.08 & 0.87 & 0.94 \\
9955 & 101215 & 1997 & 29.00 & 0.29 & 3304.00 & 31100.72 & 0.88 & 1.07 & 0.94 \\
31387 & 105881 & 1997 & 111.38 & 0.01 & 18909.00 & 177592.78 & 0.59 & 1.59 & 0.94 \\
31446 & 105890 & 1997 & 72.56 & 0.30 & 10263.00 & 85265.54 & 0.71 & 1.18 & 0.83 \\
31436 & 105886 & 1997 & 52.40 & 0.21 & 8179.00 & 83916.42 & 0.64 & 1.60 & 1.03 \\
29845 & 105654 & 1997 & 173.41 & 0.33 & 16349.00 & 164630.04 & 1.06 & 0.95 & 1.01 \\
8736 & 101096 & 1997 & 47.68 & 0.19 & 6182.00 & 53802.87 & 0.77 & 1.13 & 0.87 \\
31427 & 105883 & 1997 & 23.87 & 0.02 & 2399.00 & 19344.07 & 0.99 & 0.81 & 0.81 \\
31414 & 105882 & 1997 & 110.16 & 0.01 & 11157.00 & 105095.85 & 0.99 & 0.95 & 0.94 \\
34230 & 106214 & 1997 & 27.56 & 0.03 & 1989.00 & 19064.79 & 1.39 & 0.69 & 0.96 \\
35421 & 106360 & 1997 & 132.31 & 0.25 & 9030.00 & 81594.32 & 1.47 & 0.62 & 0.90 \\
34200 & 106212 & 1997 & 47.16 & 0.04 & 4661.00 & 46547.89 & 1.01 & 0.99 & 1.00 \\
31342 & 105879 & 1997 & 129.01 & 0.34 & 10807.00 & 102482.09 & 1.19 & 0.79 & 0.95 \\
52033 & 301299 & 1997 & 1572.04 & 0.32 & 146293.00 & 1343873.96 & 1.07 & 0.85 & 0.92 \\
9998 & 101251 & 1997 & 27.38 & 0.22 & 2259.00 & 18502.43 & 1.21 & 0.68 & 0.82 \\
35478 & 106364 & 1997 & 5.36 & 0.01 & 534.00 & 4699.92 & 1.00 & 0.88 & 0.88 \\
29891 & 105656 & 1997 & 291.80 & 0.23 & 29244.00 & 285303.01 & 1.00 & 0.98 & 0.98 \\
9070 & 101110 & 1997 & 17.14 & 0.09 & 2971.00 & 24544.41 & 0.58 & 1.43 & 0.83 \\
34218 & 106213 & 1997 & 3.03 & 0.00 & 329.00 & 3008.98 & 0.92 & 0.99 & 0.91 \\
31493 & 105900 & 1997 & 9.86 & 0.31 & 850.00 & 8491.77 & 1.16 & 0.86 & 1.00 \\
31491 & 105899 & 1997 & 2.52 & 0.27 & 239.00 & 2375.15 & 1.05 & 0.94 & 0.99 \\
29819 & 105652 & 1997 & 274.61 & 0.36 & 24292.00 & 263131.39 & 1.13 & 0.96 & 1.08 \\
9985 & 101216 & 1997 & 81.87 & 0.27 & 8185.00 & 75600.63 & 1.00 & 0.92 & 0.92 \\
31360 & 105880 & 1997 & 123.98 & 0.28 & 8149.00 & 74014.50 & 1.52 & 0.60 & 0.91 \\
31822 & 105945 & 1997 & 14.63 & 0.22 & 1443.00 & 13956.94 & 1.01 & 0.95 & 0.97 \\
29614 & 105627 & 1997 & 241.55 & 0.30 & 24155.00 & 235434.57 & 1.00 & 0.97 & 0.97 \\
31811 & 105943 & 1997 & 8.95 & 0.03 & 874.00 & 8769.26 & 1.02 & 0.98 & 1.00 \\
30006 & 105677 & 1997 & 19.22 & 0.19 & 1888.00 & 18876.63 & 1.02 & 0.98 & 1.00 \\
31158 & 105865 & 1997 & 157.02 & 0.22 & 15953.00 & 134022.46 & 0.98 & 0.85 & 0.84 \\
31147 & 105862 & 1997 & 134.70 & 0.35 & 10490.00 & 102066.17 & 1.28 & 0.76 & 0.97 \\
35609 & 106380 & 1997 & 18.80 & 0.02 & 1808.00 & 18207.33 & 1.04 & 0.97 & 1.01 \\
10074 & 101258 & 1997 & 1569.05 & 0.13 & 171694.00 & 1459995.49 & 0.91 & 0.93 & 0.85 \\
29986 & 105665 & 1997 & 98.25 & 0.26 & 8879.00 & 97115.83 & 1.11 & 0.99 & 1.09 \\
31796 & 105938 & 1997 & 34.07 & -0.02 & 2795.00 & 22873.28 & 1.22 & 0.67 & 0.82 \\
35599 & 106379 & 1997 & 16.09 & 0.00 & 2339.00 & 19107.86 & 0.69 & 1.19 & 0.82 \\
35330 & 106348 & 1997 & 11.08 & -0.01 & 1112.00 & 9877.39 & 1.00 & 0.89 & 0.89 \\
31803 & 105941 & 1997 & 10.53 & 0.01 & 573.00 & 5719.58 & 1.84 & 0.54 & 1.00 \\
31828 & 105946 & 1997 & 131.94 & 0.31 & 13194.00 & 118937.20 & 1.00 & 0.90 & 0.90 \\
33749 & 106167 & 1997 & 133.95 & -0.03 & 12279.00 & 108207.79 & 1.09 & 0.81 & 0.88 \\
33741 & 106165 & 1997 & 15.66 & 0.14 & 1566.00 & 15131.40 & 1.00 & 0.97 & 0.97 \\
10729 & 101320 & 1997 & 54.29 & 0.04 & 6297.00 & 48737.13 & 0.86 & 0.90 & 0.77 \\
35317 & 106347 & 1997 & 12.78 & 0.01 & 1278.00 & 11542.27 & 1.00 & 0.90 & 0.90 \\
34422 & 106238 & 1997 & 3.90 & 0.00 & 251.00 & 2375.94 & 1.55 & 0.61 & 0.95 \\
51905 & 300102 & 1997 & 22.27 & 0.35 & 1966.00 & 18991.19 & 1.13 & 0.85 & 0.97 \\
52147 & 302206 & 1997 & 69.30 & 0.03 & 5153.00 & 52038.55 & 1.34 & 0.75 & 1.01 \\
31861 & 105949 & 1997 & 184.37 & 0.32 & 18779.00 & 170509.55 & 0.98 & 0.92 & 0.91 \\
31135 & 105861 & 1997 & 211.50 & 0.37 & 13412.00 & 121397.33 & 1.58 & 0.57 & 0.91 \\
31850 & 105948 & 1997 & 89.54 & 0.33 & 8954.00 & 82232.11 & 1.00 & 0.92 & 0.92 \\
29602 & 105623 & 1997 & 25.40 & 0.16 & 4960.00 & 46749.30 & 0.51 & 1.84 & 0.94 \\
9796 & 101193 & 1997 & 211.12 & 0.27 & 16840.00 & 177980.40 & 1.25 & 0.84 & 1.06 \\
34392 & 106231 & 1997 & 238.50 & 0.03 & 17623.00 & 169475.66 & 1.35 & 0.71 & 0.96 \\
30015 & 105678 & 1997 & 12.78 & 0.22 & 1278.00 & 11065.68 & 1.00 & 0.87 & 0.87 \\
31839 & 105947 & 1997 & 116.25 & 0.16 & 11625.00 & 109020.21 & 1.00 & 0.94 & 0.94 \\
34419 & 106236 & 1997 & 68.34 & 0.03 & 6103.00 & 55009.44 & 1.12 & 0.80 & 0.90 \\
34365 & 106230 & 1997 & 50.80 & 0.01 & 3340.00 & 32084.21 & 1.52 & 0.63 & 0.96 \\
33761 & 106169 & 1997 & 5.49 & 0.01 & 549.00 & 5401.89 & 1.00 & 0.98 & 0.98 \\
31214 & 105867 & 1997 & 2.98 & 0.15 & 236.00 & 2285.20 & 1.26 & 0.77 & 0.97 \\
29952 & 105662 & 1997 & 35.28 & 0.02 & 4025.00 & 40109.30 & 0.88 & 1.14 & 1.00 \\
33870 & 106179 & 1997 & 52.77 & 0.01 & 5277.00 & 49323.86 & 1.00 & 0.93 & 0.93 \\
9857 & 101198 & 1997 & 194.09 & 0.16 & 19143.00 & 174647.94 & 1.01 & 0.90 & 0.91 \\
10044 & 101256 & 1997 & 8.92 & 0.14 & 916.00 & 7629.55 & 0.97 & 0.86 & 0.83 \\
33855 & 106176 & 1997 & 7.41 & 0.04 & 670.00 & 6037.99 & 1.11 & 0.82 & 0.90 \\
31676 & 105930 & 1997 & 80.91 & 0.17 & 8492.00 & 70076.61 & 0.95 & 0.87 & 0.83 \\
52086 & 301560 & 1997 & 373.38 & 0.01 & 30296.00 & 285498.79 & 1.23 & 0.76 & 0.94 \\
31670 & 105926 & 1997 & 1.88 & 0.33 & 188.00 & 1812.20 & 1.00 & 0.96 & 0.96 \\
35550 & 106374 & 1997 & 42.74 & 0.02 & 4080.00 & 41540.11 & 1.05 & 0.97 & 1.02 \\
29940 & 105659 & 1997 & 91.22 & 0.34 & 9122.00 & 90224.90 & 1.00 & 0.99 & 0.99 \\
33879 & 106180 & 1997 & 82.86 & 0.18 & 9273.00 & 82133.04 & 0.89 & 0.99 & 0.89 \\
9033 & 101109 & 1997 & 141.76 & 0.20 & 15038.00 & 141772.74 & 0.94 & 1.00 & 0.94 \\
34327 & 106223 & 1997 & 83.65 & 0.01 & 7983.00 & 79597.60 & 1.05 & 0.95 & 1.00 \\
29679 & 105635 & 1997 & 241.59 & 0.15 & 27881.00 & 243530.40 & 0.87 & 1.01 & 0.87 \\
33828 & 106173 & 1997 & 88.66 & 0.28 & 9491.00 & 94437.78 & 0.93 & 1.07 & 1.00 \\
9827 & 101194 & 1997 & 114.60 & 0.24 & 10739.00 & 112847.48 & 1.07 & 0.98 & 1.05 \\
34361 & 106226 & 1997 & 113.17 & 0.03 & 11309.00 & 102511.07 & 1.00 & 0.91 & 0.91 \\
29655 & 105631 & 1997 & 6.50 & 0.25 & 944.00 & 9438.75 & 0.69 & 1.45 & 1.00 \\
33788 & 106170 & 1997 & 98.18 & 0.01 & 8685.00 & 84407.11 & 1.13 & 0.86 & 0.97 \\
29968 & 105664 & 1997 & 8.09 & 0.29 & 809.00 & 6616.01 & 1.00 & 0.82 & 0.82 \\
31186 & 105866 & 1997 & 1771.42 & 0.33 & 154044.00 & 1622497.17 & 1.15 & 0.92 & 1.05 \\
33801 & 106172 & 1997 & 39.83 & 0.03 & 3735.00 & 34177.09 & 1.07 & 0.86 & 0.92 \\
10700 & 101312 & 1997 & 3265.01 & 0.15 & 310351.00 & 2538883.01 & 1.05 & 0.78 & 0.82 \\
260 & 100022 & 1997 & 73.90 & 0.29 & 7390.00 & 69652.66 & 1.00 & 0.94 & 0.94 \\
34347 & 106224 & 1997 & 11.40 & 0.02 & 1140.00 & 9807.07 & 1.00 & 0.86 & 0.86 \\
52099 & 301571 & 1997 & 113.61 & 0.14 & 12063.00 & 113901.40 & 0.94 & 1.00 & 0.94 \\
31722 & 105932 & 1997 & 108.83 & 0.01 & 8489.00 & 81686.35 & 1.28 & 0.75 & 0.96 \\
32357 & 106014 & 1997 & 42.31 & 0.18 & 4050.00 & 37573.45 & 1.04 & 0.89 & 0.93 \\
33596 & 106152 & 1997 & 79.90 & 0.01 & 5684.00 & 53294.14 & 1.41 & 0.67 & 0.94 \\
30291 & 105731 & 1997 & 93.12 & 0.33 & 9313.00 & 76050.59 & 1.00 & 0.82 & 0.82 \\
8898 & 101104 & 1997 & 70.55 & 0.31 & 4966.00 & 50176.05 & 1.42 & 0.71 & 1.01 \\
30494 & 105762 & 1997 & 156.99 & 0.22 & 15699.00 & 139231.82 & 1.00 & 0.89 & 0.89 \\
32514 & 106038 & 1997 & 217.99 & 0.27 & 21799.00 & 192165.41 & 1.00 & 0.88 & 0.88 \\
29195 & 105536 & 1997 & 348.56 & 0.07 & 41362.00 & 367688.80 & 0.84 & 1.05 & 0.89 \\
30384 & 105748 & 1997 & 93.54 & 0.26 & 9354.00 & 86031.44 & 1.00 & 0.92 & 0.92 \\
35057 & 106310 & 1997 & 10.08 & 0.03 & 1008.00 & 9844.33 & 1.00 & 0.98 & 0.98 \\
9243 & 101122 & 1997 & 81.89 & 0.18 & 8624.00 & 76842.82 & 0.95 & 0.94 & 0.89 \\
32732 & 106053 & 1997 & 9.58 & 0.10 & 1140.00 & 9534.59 & 0.84 & 1.00 & 0.84 \\
9410 & 101134 & 1997 & 127.60 & -0.55 & 12756.00 & 112145.35 & 1.00 & 0.88 & 0.88 \\
32542 & 106039 & 1997 & 7.30 & 0.33 & 728.00 & 6950.94 & 1.00 & 0.95 & 0.95 \\
32938 & 106083 & 1997 & 79.38 & 0.31 & 6939.00 & 64468.41 & 1.14 & 0.81 & 0.93 \\
9493 & 101140 & 1997 & 753.30 & 0.27 & 75328.00 & 749445.27 & 1.00 & 0.99 & 0.99 \\
34789 & 106277 & 1997 & 77.37 & 0.01 & 7731.00 & 63819.50 & 1.00 & 0.82 & 0.83 \\
32717 & 106052 & 1997 & 111.30 & 0.33 & 9759.00 & 106316.49 & 1.14 & 0.96 & 1.09 \\
34812 & 106278 & 1997 & 118.78 & 0.01 & 10526.00 & 89507.06 & 1.13 & 0.75 & 0.85 \\
10350 & 101283 & 1997 & 1285.82 & 0.26 & 122295.00 & 1021678.76 & 1.05 & 0.79 & 0.84 \\
30680 & 105783 & 1997 & 1277.67 & 0.19 & 103108.00 & 1028209.58 & 1.24 & 0.80 & 1.00 \\
29209 & 105544 & 1997 & 83.44 & 0.29 & 7569.00 & 66596.07 & 1.10 & 0.80 & 0.88 \\
32505 & 106037 & 1997 & 53.78 & -0.00 & 5378.00 & 47100.67 & 1.00 & 0.88 & 0.88 \\
32894 & 106081 & 1997 & 3.85 & 0.19 & 429.00 & 3402.97 & 0.90 & 0.88 & 0.79 \\
32911 & 106082 & 1997 & 89.11 & 0.40 & 6977.00 & 70052.21 & 1.28 & 0.79 & 1.00 \\
33224 & 106104 & 1997 & 1.28 & 0.02 & 80.00 & 812.46 & 1.60 & 0.63 & 1.02 \\
8676 & 101094 & 1997 & 426.17 & 0.15 & 45063.00 & 389768.40 & 0.95 & 0.91 & 0.86 \\
10827 & 101334 & 1997 & 166.93 & 0.07 & 16693.00 & 136856.13 & 1.00 & 0.82 & 0.82 \\
32751 & 106061 & 1997 & 125.71 & -0.01 & 11642.00 & 112803.55 & 1.08 & 0.90 & 0.97 \\
30392 & 105753 & 1997 & 56.09 & -0.01 & 4998.00 & 46612.53 & 1.12 & 0.83 & 0.93 \\
96660 & 611002 & 1997 & 3906.22 & 0.27 & 378187.00 & 3530891.58 & 1.03 & 0.90 & 0.93 \\
30286 & 105724 & 1997 & 28.26 & 0.20 & 2826.00 & 26811.57 & 1.00 & 0.95 & 0.95 \\
34986 & 106303 & 1997 & 4.31 & 0.01 & 355.00 & 3532.68 & 1.21 & 0.82 & 1.00 \\
30707 & 105788 & 1997 & 5.22 & 0.01 & 522.00 & 5212.85 & 1.00 & 1.00 & 1.00 \\
34722 & 106273 & 1997 & 22.79 & 0.00 & 2253.00 & 20468.43 & 1.01 & 0.90 & 0.91 \\
33186 & 106102 & 1997 & 1.26 & 0.00 & 112.00 & 941.27 & 1.12 & 0.75 & 0.84 \\
32498 & 106036 & 1997 & 2.60 & 0.00 & 193.00 & 1674.37 & 1.35 & 0.64 & 0.87 \\
29179 & 105535 & 1997 & 141.30 & 0.22 & 14301.00 & 137071.66 & 0.99 & 0.97 & 0.96 \\
35133 & 106326 & 1997 & 1.65 & -0.03 & 133.00 & 1159.36 & 1.24 & 0.70 & 0.87 \\
32877 & 106080 & 1997 & 12.81 & 0.37 & 1019.00 & 9930.15 & 1.26 & 0.78 & 0.97 \\
34968 & 106298 & 1997 & 12.26 & 0.00 & 1226.00 & 10569.36 & 1.00 & 0.86 & 0.86 \\
30320 & 105737 & 1997 & 49.78 & 0.22 & 4978.00 & 45950.74 & 1.00 & 0.92 & 0.92 \\
35063 & 106317 & 1997 & 69.86 & 0.01 & 6330.00 & 65118.71 & 1.10 & 0.93 & 1.03 \\
10319 & 101279 & 1997 & 50.20 & 0.24 & 5020.00 & 46262.11 & 1.00 & 0.92 & 0.92 \\
9320 & 101131 & 1997 & 1120.50 & 0.24 & 112054.00 & 1050957.22 & 1.00 & 0.94 & 0.94 \\
33043 & 106088 & 1997 & 6.49 & -0.01 & 737.00 & 6798.55 & 0.88 & 1.05 & 0.92 \\
29229 & 105546 & 1997 & 1.63 & 0.16 & 181.00 & 1764.78 & 0.90 & 1.09 & 0.98 \\
8662 & 101093 & 1997 & 61.14 & 0.31 & 3440.00 & 32064.22 & 1.78 & 0.52 & 0.93 \\
32659 & 106047 & 1997 & 6.32 & 0.06 & 634.00 & 5753.81 & 1.00 & 0.91 & 0.91 \\
30348 & 105740 & 1997 & 737.30 & 0.30 & 73730.00 & 645534.67 & 1.00 & 0.88 & 0.88 \\
9458 & 101137 & 1997 & 9.50 & 0.21 & 954.00 & 9075.38 & 1.00 & 0.96 & 0.95 \\
30560 & 105770 & 1997 & 50.55 & 0.24 & 5055.00 & 49912.96 & 1.00 & 0.99 & 0.99 \\
33070 & 106089 & 1997 & 7.75 & -0.00 & 484.00 & 4011.09 & 1.60 & 0.52 & 0.83 \\
30364 & 105742 & 1997 & 104.58 & 0.09 & 10721.00 & 103753.43 & 0.98 & 0.99 & 0.97 \\
9298 & 101128 & 1997 & 70.95 & 0.02 & 6904.00 & 70543.79 & 1.03 & 0.99 & 1.02 \\
32635 & 106044 & 1997 & 32.20 & 0.03 & 3397.00 & 31486.96 & 0.95 & 0.98 & 0.93 \\
30589 & 105775 & 1997 & 2214.28 & 0.05 & 172564.00 & 1516230.71 & 1.28 & 0.68 & 0.88 \\
34882 & 106284 & 1997 & 100.50 & 0.02 & 9307.00 & 79802.87 & 1.08 & 0.79 & 0.86 \\
32640 & 106045 & 1997 & 25.07 & -0.02 & 2289.00 & 21549.38 & 1.10 & 0.86 & 0.94 \\
32655 & 106046 & 1997 & 1.87 & 0.36 & 146.00 & 1575.87 & 1.28 & 0.84 & 1.08 \\
10859 & 101340 & 1997 & 10464.56 & 0.27 & 1046456.00 & 8646933.08 & 1.00 & 0.83 & 0.83 \\
34862 & 106283 & 1997 & 226.70 & 0.02 & 22671.00 & 196463.96 & 1.00 & 0.87 & 0.87 \\
8873 & 101103 & 1997 & 59.55 & 0.17 & 6477.00 & 60398.01 & 0.92 & 1.01 & 0.93 \\
32711 & 106051 & 1997 & 41.57 & 0.32 & 4198.00 & 40274.26 & 0.99 & 0.97 & 0.96 \\
33143 & 106097 & 1997 & 5.02 & 0.00 & 475.00 & 5029.05 & 1.06 & 1.00 & 1.06 \\
32558 & 106041 & 1997 & 21.77 & 0.16 & 2999.00 & 25621.87 & 0.73 & 1.18 & 0.85 \\
30649 & 105781 & 1997 & 163.48 & 0.22 & 14400.00 & 123277.69 & 1.14 & 0.75 & 0.86 \\
9481 & 101139 & 1997 & 42.96 & 0.27 & 4296.00 & 39532.56 & 1.00 & 0.92 & 0.92 \\
29221 & 105545 & 1997 & 389.36 & 0.46 & 33378.00 & 306009.79 & 1.17 & 0.79 & 0.92 \\
30640 & 105780 & 1997 & 142.62 & 0.02 & 13326.00 & 136347.65 & 1.07 & 0.96 & 1.02 \\
34824 & 106281 & 1997 & 38.55 & -0.01 & 3873.00 & 38200.66 & 1.00 & 0.99 & 0.99 \\
33120 & 106092 & 1997 & 752.07 & 0.01 & 75207.00 & 653886.95 & 1.00 & 0.87 & 0.87 \\
29261 & 105564 & 1997 & 7.66 & 0.04 & 746.00 & 7163.56 & 1.03 & 0.94 & 0.96 \\
30550 & 105769 & 1997 & 18.97 & 0.33 & 1668.00 & 14409.58 & 1.14 & 0.76 & 0.86 \\
34957 & 106297 & 1997 & 135.93 & 0.07 & 10221.00 & 96859.56 & 1.33 & 0.71 & 0.95 \\
30370 & 105746 & 1997 & 145.76 & 0.31 & 13449.00 & 124363.47 & 1.08 & 0.85 & 0.92 \\
32686 & 106049 & 1997 & 301.40 & 0.33 & 30063.00 & 298268.87 & 1.00 & 0.99 & 0.99 \\
30522 & 105763 & 1997 & 169.08 & 0.37 & 16908.00 & 160803.85 & 1.00 & 0.95 & 0.95 \\
35072 & 106318 & 1997 & 60.55 & -0.02 & 5023.00 & 56343.13 & 1.21 & 0.93 & 1.12 \\
34835 & 106282 & 1997 & 296.06 & 0.29 & 30038.00 & 258347.72 & 0.99 & 0.87 & 0.86 \\
34930 & 106294 & 1997 & 38.57 & 0.05 & 3480.00 & 31087.93 & 1.11 & 0.81 & 0.89 \\
32989 & 106085 & 1997 & 16.06 & 0.30 & 1200.00 & 13228.98 & 1.34 & 0.82 & 1.10 \\
8848 & 101102 & 1997 & 191.19 & 0.42 & 12118.00 & 127193.78 & 1.58 & 0.67 & 1.05 \\
32698 & 106050 & 1997 & 275.46 & 0.25 & 28271.00 & 250006.01 & 0.97 & 0.91 & 0.88 \\
30466 & 105761 & 1997 & 208.01 & 0.25 & 20801.00 & 206780.22 & 1.00 & 0.99 & 0.99 \\
287 & 100033 & 1997 & 321.89 & 0.14 & 31404.00 & 333103.57 & 1.02 & 1.03 & 1.06 \\
32401 & 106023 & 1997 & 8.36 & 0.26 & 836.00 & 7884.42 & 1.00 & 0.94 & 0.94 \\
30749 & 105793 & 1997 & 629.18 & 0.27 & 61565.00 & 534681.61 & 1.02 & 0.85 & 0.87 \\
32414 & 106025 & 1997 & 11.78 & 0.21 & 1168.00 & 9945.98 & 1.01 & 0.84 & 0.85 \\
32775 & 106062 & 1997 & 7.67 & 0.23 & 766.00 & 7019.17 & 1.00 & 0.92 & 0.92 \\
10894 & 101345 & 1997 & 514.74 & 0.14 & 60983.00 & 584021.45 & 0.84 & 1.13 & 0.96 \\
32441 & 106026 & 1997 & 22.23 & 0.28 & 2226.00 & 21318.35 & 1.00 & 0.96 & 0.96 \\
33244 & 106108 & 1997 & 44.93 & 0.27 & 3919.00 & 35403.18 & 1.15 & 0.79 & 0.90 \\
34651 & 106267 & 1997 & 9.62 & 0.01 & 942.00 & 9436.40 & 1.02 & 0.98 & 1.00 \\
30263 & 105722 & 1997 & 26.30 & -0.01 & 2724.00 & 24141.18 & 0.97 & 0.92 & 0.89 \\
8817 & 101100 & 1997 & 1133.39 & 0.52 & 108434.00 & 953410.23 & 1.05 & 0.84 & 0.88 \\
30413 & 105754 & 1997 & 27.21 & 0.04 & 2098.00 & 19605.05 & 1.30 & 0.72 & 0.93 \\
30781 & 105798 & 1997 & 234.50 & 0.01 & 18903.00 & 201338.84 & 1.24 & 0.86 & 1.07 \\
30260 & 105721 & 1997 & 131.29 & 0.32 & 9928.00 & 110595.40 & 1.32 & 0.84 & 1.11 \\
9542 & 101149 & 1997 & 1021.30 & 0.09 & 102133.00 & 1000283.76 & 1.00 & 0.98 & 0.98 \\
10459 & 101286 & 1997 & 551.30 & 0.26 & 55230.00 & 467507.94 & 1.00 & 0.85 & 0.85 \\
10380 & 101284 & 1997 & 653.18 & 0.16 & 60747.00 & 507252.11 & 1.08 & 0.78 & 0.84 \\
9384 & 101133 & 1997 & 1307.20 & 0.45 & 130723.00 & 1190122.85 & 1.00 & 0.91 & 0.91 \\
33280 & 106113 & 1997 & 66.88 & 0.33 & 5471.00 & 55898.83 & 1.22 & 0.84 & 1.02 \\
9347 & 101132 & 1997 & 212.10 & 0.19 & 21214.00 & 195588.60 & 1.00 & 0.92 & 0.92 \\
32852 & 106069 & 1997 & 46.31 & 0.01 & 2500.00 & 24419.32 & 1.85 & 0.53 & 0.98 \\
32859 & 106070 & 1997 & 859.20 & 0.13 & 96844.00 & 839656.19 & 0.89 & 0.98 & 0.87 \\
34657 & 106268 & 1997 & 157.20 & 0.01 & 13529.00 & 128944.15 & 1.16 & 0.82 & 0.95 \\
35017 & 106306 & 1997 & 10.74 & 0.02 & 985.00 & 9181.29 & 1.09 & 0.86 & 0.93 \\
7 & 100001 & 1997 & 2396.78 & 0.34 & 239678.00 & 2351146.87 & 1.00 & 0.98 & 0.98 \\
33275 & 106110 & 1997 & 213.95 & 0.03 & 19091.00 & 190831.77 & 1.12 & 0.89 & 1.00 \\
33307 & 106114 & 1997 & 52.34 & 0.23 & 5238.00 & 49598.86 & 1.00 & 0.95 & 0.95 \\
30796 & 105803 & 1997 & 439.96 & 0.27 & 42096.00 & 405477.04 & 1.05 & 0.92 & 0.96 \\
30438 & 105760 & 1997 & 190.08 & 0.25 & 19008.00 & 188690.62 & 1.00 & 0.99 & 0.99 \\
9215 & 101119 & 1997 & 51.98 & 0.29 & 5198.00 & 48949.35 & 1.00 & 0.94 & 0.94 \\
32869 & 106075 & 1997 & 104.61 & 0.31 & 17054.00 & 163525.61 & 0.61 & 1.56 & 0.96 \\
33234 & 106107 & 1997 & 85.55 & 0.25 & 7499.00 & 76744.26 & 1.14 & 0.90 & 1.02 \\
34673 & 106270 & 1997 & 10.24 & -0.01 & 1018.00 & 9295.34 & 1.01 & 0.91 & 0.91 \\
156 & 100016 & 1997 & 87.89 & 0.33 & 8789.00 & 88327.16 & 1.00 & 1.00 & 1.00 \\
8924 & 101105 & 1997 & 54.64 & 0.32 & 5597.00 & 53978.20 & 0.98 & 0.99 & 0.96 \\
31388 & 105881 & 1998 & 339.10 & -0.06 & 40731.00 & 379941.06 & 0.83 & 1.12 & 0.93 \\
7869 & 101064 & 1998 & 201.70 & -0.09 & 16484.00 & 161853.60 & 1.22 & 0.80 & 0.98 \\
52615 & 303175 & 1998 & 638.60 & 0.06 & 54129.00 & 501673.94 & 1.18 & 0.79 & 0.93 \\
34958 & 106297 & 1998 & 55.20 & -0.34 & 5812.00 & 56571.95 & 0.95 & 1.02 & 0.97 \\
30523 & 105763 & 1998 & 187.10 & -0.09 & 18338.00 & 164216.70 & 1.02 & 0.88 & 0.90 \\
34348 & 106224 & 1998 & 14.80 & -0.08 & 1159.00 & 11603.58 & 1.28 & 0.78 & 1.00 \\
20805 & 102795 & 1998 & 329.50 & -0.10 & 30788.00 & 288625.92 & 1.07 & 0.88 & 0.94 \\
2165 & 100293 & 1998 & 94.70 & 0.41 & 9374.00 & 91028.52 & 1.01 & 0.96 & 0.97 \\
30429 & 105758 & 1998 & 124.40 & -0.05 & 13485.00 & 117601.25 & 0.92 & 0.95 & 0.87 \\
20925 & 102802 & 1998 & 830.70 & 0.12 & 83071.00 & 742857.45 & 1.00 & 0.89 & 0.89 \\
48600 & 240113 & 1998 & 41.00 & 0.11 & 3963.00 & 39654.72 & 1.03 & 0.97 & 1.00 \\
34931 & 106294 & 1998 & 56.10 & -0.18 & 5369.00 & 51874.85 & 1.04 & 0.92 & 0.97 \\
2134 & 100292 & 1998 & 201.80 & 0.10 & 20321.00 & 187829.32 & 0.99 & 0.93 & 0.92 \\
23551 & 103186 & 1998 & 879.90 & 0.18 & 72784.00 & 788389.31 & 1.21 & 0.90 & 1.08 \\
23544 & 103184 & 1998 & 882.50 & -0.11 & 89623.00 & 751342.13 & 0.98 & 0.85 & 0.84 \\
31220 & 105868 & 1998 & 66.70 & -0.03 & 7025.00 & 65573.63 & 0.95 & 0.98 & 0.93 \\
20173 & 102673 & 1998 & 469.20 & -0.01 & 47291.00 & 438924.38 & 0.99 & 0.94 & 0.93 \\
31215 & 105867 & 1998 & 2.70 & 0.17 & 266.00 & 2660.48 & 1.02 & 0.99 & 1.00 \\
23929 & 103242 & 1998 & 72.30 & 0.18 & 5711.00 & 71415.30 & 1.27 & 0.99 & 1.25 \\
20855 & 102797 & 1998 & 62.70 & -0.10 & 7358.00 & 54847.45 & 0.85 & 0.87 & 0.75 \\
20883 & 102798 & 1998 & 975.60 & 0.15 & 95410.00 & 901126.60 & 1.02 & 0.92 & 0.94 \\
48445 & 240085 & 1998 & 148.80 & 0.13 & 14301.00 & 123466.61 & 1.04 & 0.83 & 0.86 \\
20152 & 102671 & 1998 & 157.00 & 0.32 & 13606.00 & 146666.54 & 1.15 & 0.93 & 1.08 \\
20911 & 102799 & 1998 & 557.80 & 0.01 & 56506.00 & 510108.08 & 0.99 & 0.91 & 0.90 \\
40504 & 108134 & 1998 & 20.70 & 0.09 & 2018.00 & 18642.14 & 1.03 & 0.90 & 0.92 \\
12569 & 101554 & 1998 & 683.70 & 0.15 & 68303.00 & 675523.04 & 1.00 & 0.99 & 0.99 \\
13412 & 101738 & 1998 & 728.00 & 0.34 & 57694.00 & 623632.23 & 1.26 & 0.86 & 1.08 \\
48607 & 240114 & 1998 & 122.90 & 0.31 & 12066.00 & 120629.80 & 1.02 & 0.98 & 1.00 \\
34328 & 106223 & 1998 & 93.80 & 0.10 & 9375.00 & 92700.23 & 1.00 & 0.99 & 0.99 \\
23959 & 103251 & 1998 & 329.30 & 0.44 & 29829.00 & 340677.07 & 1.10 & 1.03 & 1.14 \\
3445 & 100435 & 1998 & 6.30 & -0.32 & 761.00 & 6449.25 & 0.83 & 1.02 & 0.85 \\
20946 & 102813 & 1998 & 659.50 & -0.01 & 58024.00 & 541796.18 & 1.14 & 0.82 & 0.93 \\
10381 & 101284 & 1998 & 902.40 & 0.10 & 84396.00 & 866067.96 & 1.07 & 0.96 & 1.03 \\
30561 & 105770 & 1998 & 39.80 & 0.13 & 3858.00 & 38581.72 & 1.03 & 0.97 & 1.00 \\
34366 & 106230 & 1998 & 148.50 & 0.11 & 14836.00 & 147517.18 & 1.00 & 0.99 & 0.99 \\
7461 & 101040 & 1998 & 3145.40 & 0.22 & 280212.00 & 3123138.07 & 1.12 & 0.99 & 1.11 \\
24001 & 103253 & 1998 & 141.40 & -0.07 & 16464.00 & 145580.12 & 0.86 & 1.03 & 0.88 \\
31159 & 105865 & 1998 & 115.90 & 0.10 & 18195.00 & 110728.71 & 0.64 & 0.96 & 0.61 \\
20047 & 102664 & 1998 & 1631.90 & -0.12 & 156574.00 & 1543889.41 & 1.04 & 0.95 & 0.99 \\
12703 & 101588 & 1998 & 77.20 & 0.05 & 6939.00 & 68415.10 & 1.11 & 0.89 & 0.99 \\
2231 & 100298 & 1998 & 1042.30 & 0.18 & 103921.00 & 1005189.81 & 1.00 & 0.96 & 0.97 \\
21064 & 102827 & 1998 & 163.20 & 0.05 & 22631.00 & 199041.69 & 0.72 & 1.22 & 0.88 \\
10075 & 101258 & 1998 & 1545.10 & 0.15 & 136402.00 & 1291457.20 & 1.13 & 0.84 & 0.95 \\
21087 & 102828 & 1998 & 181.30 & 0.30 & 16475.00 & 172311.09 & 1.10 & 0.95 & 1.05 \\
20013 & 102663 & 1998 & 4695.60 & 0.42 & 470324.00 & 4561739.75 & 1.00 & 0.97 & 0.97 \\
2245 & 100299 & 1998 & 120.30 & -0.08 & 12041.00 & 113905.27 & 1.00 & 0.95 & 0.95 \\
40426 & 108119 & 1998 & 61.20 & 0.06 & 6122.00 & 58084.86 & 1.00 & 0.95 & 0.95 \\
13351 & 101729 & 1998 & 324.30 & 0.13 & 32416.00 & 259803.43 & 1.00 & 0.80 & 0.80 \\
21022 & 102824 & 1998 & 93.70 & 0.07 & 8937.00 & 89601.68 & 1.05 & 0.96 & 1.00 \\
34162 & 106210 & 1998 & 16.90 & -0.04 & 1565.00 & 15220.66 & 1.08 & 0.90 & 0.97 \\
40478 & 108122 & 1998 & 27.60 & -0.15 & 3075.00 & 27635.45 & 0.90 & 1.00 & 0.90 \\
20141 & 102669 & 1998 & 41.50 & 0.19 & 4229.00 & 44358.64 & 0.98 & 1.07 & 1.05 \\
48642 & 240116 & 1998 & 26.00 & 0.45 & 1492.00 & 13388.16 & 1.74 & 0.51 & 0.90 \\
20976 & 102814 & 1998 & 218.50 & -0.02 & 19357.00 & 191322.61 & 1.13 & 0.88 & 0.99 \\
2198 & 100295 & 1998 & 15.80 & 0.19 & 1521.00 & 14483.67 & 1.04 & 0.92 & 0.95 \\
23497 & 103182 & 1998 & 418.40 & 0.20 & 35333.00 & 354402.80 & 1.18 & 0.85 & 1.00 \\
34189 & 106211 & 1998 & 107.30 & 0.09 & 10727.00 & 100778.91 & 1.00 & 0.94 & 0.94 \\
23980 & 103252 & 1998 & 283.60 & 0.10 & 29367.00 & 288344.35 & 0.97 & 1.02 & 0.98 \\
3394 & 100431 & 1998 & 221.90 & 0.06 & 17061.00 & 201259.13 & 1.30 & 0.91 & 1.18 \\
2219 & 100296 & 1998 & 23.10 & 0.17 & 2452.00 & 23703.79 & 0.94 & 1.03 & 0.97 \\
23476 & 103179 & 1998 & 728.10 & -0.02 & 68373.00 & 735650.10 & 1.06 & 1.01 & 1.08 \\
13380 & 101736 & 1998 & 61.90 & 0.24 & 6191.00 & 58525.90 & 1.00 & 0.95 & 0.95 \\
34362 & 106226 & 1998 & 114.90 & 0.10 & 11827.00 & 110652.56 & 0.97 & 0.96 & 0.94 \\
54840 & 400018 & 1998 & 28.80 & 0.16 & 2872.00 & 26554.43 & 1.00 & 0.92 & 0.92 \\
20108 & 102667 & 1998 & 11209.90 & 0.25 & 1049713.00 & 9995289.51 & 1.07 & 0.89 & 0.95 \\
30551 & 105769 & 1998 & 30.10 & 0.38 & 2222.00 & 19409.92 & 1.35 & 0.64 & 0.87 \\
40452 & 108121 & 1998 & 112.60 & 0.04 & 8654.00 & 95680.46 & 1.30 & 0.85 & 1.11 \\
13615 & 101748 & 1998 & 22.90 & 0.17 & 2291.00 & 22710.94 & 1.00 & 0.99 & 0.99 \\
31187 & 105866 & 1998 & 4750.30 & 0.28 & 376427.00 & 4188456.81 & 1.26 & 0.88 & 1.11 \\
20727 & 102788 & 1998 & 293.80 & 0.20 & 29172.00 & 289645.02 & 1.01 & 0.99 & 0.99 \\
10045 & 101256 & 1998 & 8.30 & 0.12 & 906.00 & 9110.41 & 0.92 & 1.10 & 1.01 \\
49009 & 240199 & 1998 & 1550.40 & 0.07 & 126799.00 & 1339704.92 & 1.22 & 0.86 & 1.06 \\
3722 & 100475 & 1998 & 167.50 & 0.10 & 12920.00 & 162577.89 & 1.30 & 0.97 & 1.26 \\
34987 & 106303 & 1998 & 8.10 & 0.21 & 666.00 & 7855.00 & 1.22 & 0.97 & 1.18 \\
20469 & 102757 & 1998 & 19613.00 & 0.24 & 1896887.00 & 19901911.83 & 1.03 & 1.01 & 1.05 \\
61565 & 500096 & 1998 & 30.60 & -0.01 & 2990.00 & 29701.13 & 1.02 & 0.97 & 0.99 \\
31291 & 105875 & 1998 & 83.40 & 0.65 & 6278.00 & 85082.48 & 1.33 & 1.02 & 1.36 \\
61370 & 500047 & 1998 & 2.70 & 0.03 & 218.00 & 2276.46 & 1.24 & 0.84 & 1.04 \\
48491 & 240090 & 1998 & 41.20 & 0.09 & 4144.00 & 39786.64 & 0.99 & 0.97 & 0.96 \\
23857 & 103224 & 1998 & 220.40 & 0.12 & 20554.00 & 204473.54 & 1.07 & 0.93 & 0.99 \\
23722 & 103209 & 1998 & 322.00 & 0.07 & 30286.00 & 288075.70 & 1.06 & 0.89 & 0.95 \\
48560 & 240107 & 1998 & 9.70 & -0.37 & 982.00 & 9871.73 & 0.99 & 1.02 & 1.01 \\
23688 & 103208 & 1998 & 1439.90 & 0.07 & 143252.00 & 1249775.91 & 1.01 & 0.87 & 0.87 \\
12594 & 101557 & 1998 & 88.40 & 0.14 & 8032.00 & 82792.88 & 1.10 & 0.94 & 1.03 \\
3620 & 100463 & 1998 & 14.30 & 0.27 & 1422.00 & 13524.73 & 1.01 & 0.95 & 0.95 \\
12638 & 101561 & 1998 & 107.80 & -0.12 & 10628.00 & 105390.55 & 1.01 & 0.98 & 0.99 \\
20520 & 102761 & 1998 & 38428.90 & 0.15 & 3475006.00 & 40019318.82 & 1.11 & 1.04 & 1.15 \\
9999 & 101251 & 1998 & 42.20 & 0.11 & 3459.00 & 32910.90 & 1.22 & 0.78 & 0.95 \\
31277 & 105874 & 1998 & 38.20 & 0.20 & 2999.00 & 36190.12 & 1.27 & 0.95 & 1.21 \\
20498 & 102760 & 1998 & 1536.40 & 0.23 & 139600.00 & 1318006.31 & 1.10 & 0.86 & 0.94 \\
51950 & 300673 & 1998 & 22.60 & 0.03 & 2320.00 & 21234.46 & 0.97 & 0.94 & 0.92 \\
13512 & 101742 & 1998 & 3111.50 & 0.58 & 234945.00 & 2582383.89 & 1.32 & 0.83 & 1.10 \\
34258 & 106216 & 1998 & 154.10 & 0.21 & 14151.00 & 141979.79 & 1.09 & 0.92 & 1.00 \\
12608 & 101560 & 1998 & 34.10 & 0.23 & 2632.00 & 30973.53 & 1.30 & 0.91 & 1.18 \\
20300 & 102715 & 1998 & 5621.20 & 0.08 & 610082.00 & 5049524.99 & 0.92 & 0.90 & 0.83 \\
20334 & 102716 & 1998 & 2104.80 & -0.03 & 211265.00 & 1893902.29 & 1.00 & 0.90 & 0.90 \\
3679 & 100468 & 1998 & 283.80 & 0.20 & 28606.00 & 268182.07 & 0.99 & 0.94 & 0.94 \\
31315 & 105878 & 1998 & 273.40 & 0.15 & 21909.00 & 219445.58 & 1.25 & 0.80 & 1.00 \\
23794 & 103213 & 1998 & 1022.50 & 0.32 & 103750.00 & 985941.45 & 0.99 & 0.96 & 0.95 \\
30467 & 105761 & 1998 & 321.40 & 0.18 & 26652.00 & 274868.44 & 1.21 & 0.86 & 1.03 \\
61396 & 500048 & 1998 & 70.70 & 0.10 & 6504.00 & 65977.44 & 1.09 & 0.93 & 1.01 \\
61478 & 500082 & 1998 & 51.80 & -0.00 & 3610.00 & 30627.21 & 1.43 & 0.59 & 0.85 \\
10003 & 101252 & 1998 & 27.40 & 0.08 & 2477.00 & 24777.76 & 1.11 & 0.90 & 1.00 \\
20385 & 102733 & 1998 & 4805.20 & 0.08 & 501885.00 & 4375888.85 & 0.96 & 0.91 & 0.87 \\
31307 & 105877 & 1998 & 11.50 & 0.22 & 1145.00 & 10399.98 & 1.00 & 0.90 & 0.91 \\
34231 & 106214 & 1998 & 40.40 & -0.03 & 3338.00 & 34774.93 & 1.21 & 0.86 & 1.04 \\
23758 & 103212 & 1998 & 2037.90 & 0.15 & 202432.00 & 1688359.93 & 1.01 & 0.83 & 0.83 \\
34991 & 106305 & 1998 & 23.80 & 0.02 & 2333.00 & 23334.90 & 1.02 & 0.98 & 1.00 \\
2094 & 100290 & 1998 & 405.60 & 0.13 & 52084.00 & 482350.20 & 0.78 & 1.19 & 0.93 \\
20447 & 102744 & 1998 & 3370.70 & 0.08 & 337709.00 & 2934852.50 & 1.00 & 0.87 & 0.87 \\
31300 & 105876 & 1998 & 141.50 & 0.13 & 12541.00 & 130151.79 & 1.13 & 0.92 & 1.04 \\
35018 & 106306 & 1998 & 17.80 & 0.30 & 1645.00 & 16440.46 & 1.08 & 0.92 & 1.00 \\
61554 & 500094 & 1998 & 6.10 & 0.01 & 597.00 & 5567.18 & 1.02 & 0.91 & 0.93 \\
31152 & 105864 & 1998 & 36.10 & -0.37 & 3574.00 & 34994.62 & 1.01 & 0.97 & 0.98 \\
20186 & 102676 & 1998 & 215.70 & 0.16 & 19383.00 & 197307.33 & 1.11 & 0.91 & 1.02 \\
61328 & 500037 & 1998 & 7093.30 & 0.28 & 707133.00 & 6834916.52 & 1.00 & 0.96 & 0.97 \\
23875 & 103226 & 1998 & 133.50 & 0.28 & 9762.00 & 123880.74 & 1.37 & 0.93 & 1.27 \\
13456 & 101740 & 1998 & 25490.60 & 0.39 & 2063600.00 & 22822664.64 & 1.24 & 0.90 & 1.11 \\
23597 & 103199 & 1998 & 7.40 & -0.02 & 603.00 & 7418.32 & 1.23 & 1.00 & 1.23 \\
48982 & 240197 & 1998 & 1413.50 & 0.40 & 115555.00 & 1479355.21 & 1.22 & 1.05 & 1.28 \\
34969 & 106298 & 1998 & 16.70 & 0.06 & 2047.00 & 15664.78 & 0.82 & 0.94 & 0.77 \\
31361 & 105880 & 1998 & 94.50 & -0.04 & 9322.00 & 93212.03 & 1.01 & 0.99 & 1.00 \\
34318 & 106222 & 1998 & 16.90 & 0.09 & 1296.00 & 11123.41 & 1.30 & 0.66 & 0.86 \\
7428 & 101039 & 1998 & 4997.00 & 0.28 & 396442.00 & 4674850.71 & 1.26 & 0.94 & 1.18 \\
23570 & 103193 & 1998 & 63.30 & 0.11 & 5979.00 & 59789.46 & 1.06 & 0.94 & 1.00 \\
20708 & 102784 & 1998 & 25468.30 & 0.24 & 2300532.00 & 23243261.76 & 1.11 & 0.91 & 1.01 \\
20669 & 102783 & 1998 & 1650.00 & 0.14 & 154400.00 & 1388805.20 & 1.07 & 0.84 & 0.90 \\
13583 & 101744 & 1998 & 1246.20 & 0.23 & 104026.00 & 1141098.28 & 1.20 & 0.92 & 1.10 \\
23824 & 103214 & 1998 & 1834.70 & 0.27 & 179457.00 & 1691811.09 & 1.02 & 0.92 & 0.94 \\
3523 & 100453 & 1998 & 159.20 & 0.22 & 11292.00 & 132654.00 & 1.41 & 0.83 & 1.17 \\
9034 & 101109 & 1998 & 146.30 & -0.06 & 13618.00 & 147312.45 & 1.07 & 1.01 & 1.08 \\
23909 & 103232 & 1998 & 171.70 & 0.23 & 16277.00 & 162884.57 & 1.05 & 0.95 & 1.00 \\
34201 & 106212 & 1998 & 63.10 & 0.22 & 6726.00 & 66988.08 & 0.94 & 1.06 & 1.00 \\
3782 & 100481 & 1998 & 56.80 & 0.18 & 4280.00 & 51551.45 & 1.33 & 0.91 & 1.20 \\
12669 & 101562 & 1998 & 434.70 & 0.08 & 40214.00 & 408153.19 & 1.08 & 0.94 & 1.01 \\
3494 & 100441 & 1998 & 880.90 & 0.17 & 87727.00 & 832772.02 & 1.00 & 0.95 & 0.95 \\
31226 & 105869 & 1998 & 104.50 & 0.32 & 7640.00 & 99110.82 & 1.37 & 0.95 & 1.30 \\
34291 & 106221 & 1998 & 98.70 & -0.07 & 9315.00 & 97017.79 & 1.06 & 0.98 & 1.04 \\
20653 & 102777 & 1998 & 4203.20 & 0.11 & 404669.00 & 3643797.41 & 1.04 & 0.87 & 0.90 \\
20201 & 102688 & 1998 & 111.20 & 0.22 & 11249.00 & 109680.77 & 0.99 & 0.99 & 0.98 \\
23655 & 103205 & 1998 & 42.60 & -0.10 & 4146.00 & 41465.08 & 1.03 & 0.97 & 1.00 \\
48996 & 240198 & 1998 & 2594.60 & 0.10 & 230811.00 & 2492334.94 & 1.12 & 0.96 & 1.08 \\
13481 & 101741 & 1998 & 3504.10 & 0.35 & 282572.00 & 2340084.23 & 1.24 & 0.67 & 0.83 \\
31343 & 105879 & 1998 & 155.80 & 0.09 & 15517.00 & 150556.14 & 1.00 & 0.97 & 0.97 \\
3752 & 100480 & 1998 & 125.20 & 0.29 & 9514.00 & 97635.71 & 1.32 & 0.78 & 1.03 \\
34285 & 106220 & 1998 & 223.20 & 0.17 & 20746.00 & 207446.31 & 1.08 & 0.93 & 1.00 \\
61314 & 500028 & 1998 & 23.70 & -0.34 & 2271.00 & 19502.44 & 1.04 & 0.82 & 0.86 \\
3576 & 100457 & 1998 & 173.20 & 0.38 & 18078.00 & 161586.92 & 0.96 & 0.93 & 0.89 \\
31266 & 105873 & 1998 & 12.70 & -0.15 & 1140.00 & 10732.18 & 1.11 & 0.85 & 0.94 \\
7390 & 101038 & 1998 & 7365.30 & 0.26 & 619055.00 & 6208643.44 & 1.19 & 0.84 & 1.00 \\
20597 & 102774 & 1998 & 5544.40 & 0.19 & 528257.00 & 5428099.27 & 1.05 & 0.98 & 1.03 \\
34219 & 106213 & 1998 & 17.70 & 0.01 & 1540.00 & 15983.83 & 1.15 & 0.90 & 1.04 \\
20617 & 102775 & 1998 & 1840.50 & 0.12 & 170158.00 & 1598629.48 & 1.08 & 0.87 & 0.94 \\
2059 & 100287 & 1998 & 47.40 & 0.10 & 4702.00 & 43974.04 & 1.01 & 0.93 & 0.94 \\
3563 & 100456 & 1998 & 3.10 & 0.21 & 276.00 & 2367.48 & 1.12 & 0.76 & 0.86 \\
30439 & 105760 & 1998 & 334.90 & 0.32 & 27652.00 & 303715.73 & 1.21 & 0.91 & 1.10 \\
31262 & 105872 & 1998 & 10.30 & 0.12 & 851.00 & 7677.01 & 1.21 & 0.75 & 0.90 \\
23624 & 103204 & 1998 & 245.60 & 0.20 & 24650.00 & 229773.90 & 1.00 & 0.94 & 0.93 \\
48591 & 240112 & 1998 & 9.60 & 0.38 & 955.00 & 8608.69 & 1.01 & 0.90 & 0.90 \\
30495 & 105762 & 1998 & 228.40 & 0.15 & 20079.00 & 193677.99 & 1.14 & 0.85 & 0.96 \\
3554 & 100455 & 1998 & 8.90 & 0.21 & 821.00 & 6593.83 & 1.08 & 0.74 & 0.80 \\
23893 & 103228 & 1998 & 61.40 & 0.23 & 5490.00 & 50383.93 & 1.12 & 0.82 & 0.92 \\
30797 & 105803 & 1998 & 545.30 & 0.14 & 50977.00 & 500917.71 & 1.07 & 0.92 & 0.98 \\
21309 & 102847 & 1998 & 93.70 & 0.03 & 8605.00 & 93140.00 & 1.09 & 0.99 & 1.08 \\
21096 & 102829 & 1998 & 46.20 & -0.07 & 5821.00 & 43602.39 & 0.79 & 0.94 & 0.75 \\
23029 & 103103 & 1998 & 311.70 & 0.37 & 23831.00 & 282006.23 & 1.31 & 0.90 & 1.18 \\
22073 & 102989 & 1998 & 2038.20 & 0.14 & 165633.00 & 1962797.52 & 1.23 & 0.96 & 1.19 \\
22106 & 102990 & 1998 & 1749.40 & 0.11 & 139371.00 & 1677772.64 & 1.26 & 0.96 & 1.20 \\
2439 & 100330 & 1998 & 5672.30 & 0.04 & 439918.00 & 5248821.98 & 1.29 & 0.93 & 1.19 \\
41972 & 108886 & 1998 & 66.70 & -0.09 & 5701.00 & 68541.59 & 1.17 & 1.03 & 1.20 \\
30932 & 105838 & 1998 & 10.40 & 0.12 & 1038.00 & 10030.71 & 1.00 & 0.96 & 0.97 \\
22140 & 102993 & 1998 & 2905.10 & 0.16 & 274554.00 & 2590515.41 & 1.06 & 0.89 & 0.94 \\
22998 & 103101 & 1998 & 348.90 & -0.10 & 34509.00 & 338750.16 & 1.01 & 0.97 & 0.98 \\
2860 & 100366 & 1998 & 1.60 & -0.03 & 143.00 & 1657.88 & 1.12 & 1.04 & 1.16 \\
13070 & 101626 & 1998 & 1497.30 & 0.22 & 149212.00 & 1318354.83 & 1.00 & 0.88 & 0.88 \\
8966 & 101107 & 1998 & 1198.40 & 0.03 & 153202.00 & 1388793.04 & 0.78 & 1.16 & 0.91 \\
2851 & 100365 & 1998 & 328.20 & 0.24 & 32822.00 & 293271.73 & 1.00 & 0.89 & 0.89 \\
22980 & 103100 & 1998 & 562.90 & 0.26 & 56282.00 & 552122.56 & 1.00 & 0.98 & 0.98 \\
22184 & 102994 & 1998 & 93.80 & 0.25 & 7978.00 & 88924.50 & 1.18 & 0.95 & 1.11 \\
7735 & 101056 & 1998 & 48639.80 & 0.28 & 4424299.00 & 44447016.43 & 1.10 & 0.91 & 1.00 \\
22961 & 103099 & 1998 & 712.80 & 0.28 & 81318.00 & 620847.14 & 0.88 & 0.87 & 0.76 \\
34674 & 106270 & 1998 & 9.70 & 0.00 & 1015.00 & 8593.74 & 0.96 & 0.89 & 0.85 \\
30914 & 105836 & 1998 & 212.60 & 0.34 & 11180.00 & 112772.49 & 1.90 & 0.53 & 1.01 \\
22215 & 102996 & 1998 & 340.50 & 0.19 & 34681.00 & 332492.93 & 0.98 & 0.98 & 0.96 \\
2816 & 100360 & 1998 & 670.20 & 0.03 & 67019.00 & 608055.67 & 1.00 & 0.91 & 0.91 \\
30750 & 105793 & 1998 & 1288.10 & 0.57 & 95152.00 & 954753.84 & 1.35 & 0.74 & 1.00 \\
2807 & 100359 & 1998 & 347.30 & 0.22 & 34727.00 & 290525.31 & 1.00 & 0.84 & 0.84 \\
13047 & 101623 & 1998 & 2625.80 & 0.09 & 266552.00 & 2455462.80 & 0.99 & 0.94 & 0.92 \\
22946 & 103090 & 1998 & 1227.80 & 0.16 & 119015.00 & 1246754.59 & 1.03 & 1.02 & 1.05 \\
10195 & 101268 & 1998 & 2377.30 & 0.25 & 231399.00 & 2181123.25 & 1.03 & 0.92 & 0.94 \\
2879 & 100368 & 1998 & 260.40 & 0.23 & 26040.00 & 228784.40 & 1.00 & 0.88 & 0.88 \\
22045 & 102988 & 1998 & 40.70 & 0.03 & 3597.00 & 39918.20 & 1.13 & 0.98 & 1.11 \\
30938 & 105839 & 1998 & 7.40 & 0.38 & 738.00 & 7094.24 & 1.00 & 0.96 & 0.96 \\
7588 & 101047 & 1998 & 358.70 & 0.19 & 35303.00 & 350768.60 & 1.02 & 0.98 & 0.99 \\
23100 & 103122 & 1998 & 271.70 & 0.48 & 27171.00 & 256550.26 & 1.00 & 0.94 & 0.94 \\
34736 & 106275 & 1998 & 11.00 & -0.11 & 1177.00 & 10829.67 & 0.93 & 0.98 & 0.92 \\
23072 & 103110 & 1998 & 1964.90 & 0.04 & 193947.00 & 1886884.89 & 1.01 & 0.96 & 0.97 \\
34531 & 106250 & 1998 & 48.50 & 0.27 & 3983.00 & 39877.78 & 1.22 & 0.82 & 1.00 \\
21907 & 102979 & 1998 & 89.70 & -0.01 & 8102.00 & 77097.01 & 1.11 & 0.86 & 0.95 \\
2952 & 100389 & 1998 & 553.60 & 0.20 & 55307.00 & 547933.71 & 1.00 & 0.99 & 0.99 \\
34540 & 106251 & 1998 & 48.10 & 0.22 & 4742.00 & 47415.88 & 1.01 & 0.99 & 1.00 \\
30958 & 105842 & 1998 & 57.20 & -0.02 & 5756.00 & 52610.05 & 0.99 & 0.92 & 0.91 \\
12823 & 101601 & 1998 & 1151.10 & 0.02 & 118363.00 & 1092262.15 & 0.97 & 0.95 & 0.92 \\
21955 & 102981 & 1998 & 97.80 & 0.16 & 9381.00 & 93796.30 & 1.04 & 0.96 & 1.00 \\
21988 & 102983 & 1998 & 126.40 & 0.29 & 12460.00 & 129641.42 & 1.01 & 1.03 & 1.04 \\
10165 & 101264 & 1998 & 239.30 & 0.05 & 19913.00 & 185748.10 & 1.20 & 0.78 & 0.93 \\
30909 & 105811 & 1998 & 4.80 & 0.11 & 474.00 & 4743.57 & 1.01 & 0.99 & 1.00 \\
2913 & 100379 & 1998 & 687.40 & 0.12 & 64749.00 & 638869.01 & 1.06 & 0.93 & 0.99 \\
22000 & 102984 & 1998 & 112.70 & 0.16 & 12247.00 & 127501.74 & 0.92 & 1.13 & 1.04 \\
30950 & 105841 & 1998 & 18.20 & 0.07 & 1820.00 & 17482.49 & 1.00 & 0.96 & 0.96 \\
30708 & 105788 & 1998 & 9.10 & 0.36 & 911.00 & 8169.42 & 1.00 & 0.90 & 0.90 \\
2888 & 100369 & 1998 & 446.60 & 0.28 & 44699.00 & 377868.40 & 1.00 & 0.85 & 0.85 \\
41263 & 108710 & 1998 & 363.90 & 0.19 & 36361.00 & 334488.20 & 1.00 & 0.92 & 0.92 \\
22014 & 102987 & 1998 & 723.90 & 0.06 & 63816.00 & 644238.16 & 1.13 & 0.89 & 1.01 \\
30944 & 105840 & 1998 & 6.90 & 0.19 & 695.00 & 6829.25 & 0.99 & 0.99 & 0.98 \\
12836 & 101602 & 1998 & 2736.60 & 0.12 & 277406.00 & 2432608.01 & 0.99 & 0.89 & 0.88 \\
34707 & 106272 & 1998 & 1118.00 & 0.28 & 101198.00 & 1067395.85 & 1.10 & 0.95 & 1.05 \\
117 & 100009 & 1998 & 237.80 & 0.23 & 20460.00 & 248365.95 & 1.16 & 1.04 & 1.21 \\
98 & 100006 & 1998 & 6933.00 & 0.06 & 638418.00 & 6094999.39 & 1.09 & 0.88 & 0.95 \\
30905 & 105809 & 1998 & 4.40 & 0.07 & 389.00 & 4446.12 & 1.13 & 1.01 & 1.14 \\
2794 & 100358 & 1998 & 1566.50 & 0.16 & 156649.00 & 1355943.39 & 1.00 & 0.87 & 0.87 \\
22792 & 103065 & 1998 & 363.90 & 0.07 & 36322.00 & 338548.85 & 1.00 & 0.93 & 0.93 \\
22526 & 103017 & 1998 & 2250.10 & 0.28 & 227332.00 & 2212380.33 & 0.99 & 0.98 & 0.97 \\
34587 & 106257 & 1998 & 129.00 & 0.14 & 10132.00 & 119067.52 & 1.27 & 0.92 & 1.18 \\
30782 & 105798 & 1998 & 347.10 & 0.35 & 25230.00 & 314485.85 & 1.38 & 0.91 & 1.25 \\
2538 & 100343 & 1998 & 490.30 & 0.13 & 49025.00 & 471667.44 & 1.00 & 0.96 & 0.96 \\
2687 & 100352 & 1998 & 2100.20 & 0.25 & 157488.00 & 1529152.33 & 1.33 & 0.73 & 0.97 \\
22744 & 103057 & 1998 & 4194.10 & 0.05 & 478224.00 & 3821186.80 & 0.88 & 0.91 & 0.80 \\
34614 & 106258 & 1998 & 17.80 & 0.18 & 1084.00 & 11039.13 & 1.64 & 0.62 & 1.02 \\
7701 & 101055 & 1998 & 11440.40 & 0.27 & 1059323.00 & 10846568.19 & 1.08 & 0.95 & 1.02 \\
10228 & 101275 & 1998 & 639.20 & 0.38 & 49147.00 & 620368.32 & 1.30 & 0.97 & 1.26 \\
22555 & 103019 & 1998 & 579.10 & 0.23 & 57957.00 & 470125.01 & 1.00 & 0.81 & 0.81 \\
34652 & 106267 & 1998 & 10.80 & 0.12 & 1114.00 & 10894.12 & 0.97 & 1.01 & 0.98 \\
22567 & 103021 & 1998 & 50.50 & -0.01 & 5046.00 & 49610.73 & 1.00 & 0.98 & 0.98 \\
22499 & 103016 & 1998 & 453.30 & -0.08 & 45390.00 & 442355.85 & 1.00 & 0.98 & 0.97 \\
2667 & 100351 & 1998 & 190.90 & 0.35 & 19076.00 & 178989.58 & 1.00 & 0.94 & 0.94 \\
12948 & 101616 & 1998 & 20071.30 & 0.28 & 2025727.00 & 18877647.20 & 0.99 & 0.94 & 0.93 \\
30793 & 105802 & 1998 & 3.40 & -0.05 & 393.00 & 2908.41 & 0.87 & 0.86 & 0.74 \\
30825 & 105804 & 1998 & 154.90 & 0.35 & 13383.00 & 135532.85 & 1.16 & 0.87 & 1.01 \\
7663 & 101054 & 1998 & 10563.20 & 0.31 & 801755.00 & 8934520.24 & 1.32 & 0.85 & 1.11 \\
22598 & 103024 & 1998 & 310.80 & -0.05 & 27783.00 & 271111.39 & 1.12 & 0.87 & 0.98 \\
12913 & 101606 & 1998 & 5806.50 & 0.22 & 582317.00 & 5495865.73 & 1.00 & 0.95 & 0.94 \\
22635 & 103027 & 1998 & 1061.40 & 0.35 & 75760.00 & 1036962.06 & 1.40 & 0.98 & 1.37 \\
141 & 100010 & 1998 & 956.80 & 0.09 & 100112.00 & 958845.22 & 0.96 & 1.00 & 0.96 \\
2610 & 100347 & 1998 & 1429.70 & 0.28 & 142994.00 & 1382162.62 & 1.00 & 0.97 & 0.97 \\
2648 & 100350 & 1998 & 148.80 & 0.36 & 14881.00 & 148529.09 & 1.00 & 1.00 & 1.00 \\
34629 & 106262 & 1998 & 452.40 & 0.23 & 45392.00 & 439718.07 & 1.00 & 0.97 & 0.97 \\
2629 & 100348 & 1998 & 215.90 & -0.03 & 21617.00 & 212250.04 & 1.00 & 0.98 & 0.98 \\
34618 & 106261 & 1998 & 194.90 & 0.08 & 19493.00 & 188723.04 & 1.00 & 0.97 & 0.97 \\
2992 & 100395 & 1998 & 633.80 & 0.07 & 63021.00 & 613601.37 & 1.01 & 0.97 & 0.97 \\
22827 & 103067 & 1998 & 119.90 & 0.31 & 11944.00 & 115232.09 & 1.00 & 0.96 & 0.96 \\
30853 & 105805 & 1998 & 28.30 & 0.04 & 2852.00 & 25694.80 & 0.99 & 0.91 & 0.90 \\
13033 & 101622 & 1998 & 1427.70 & 0.19 & 150121.00 & 1280938.95 & 0.95 & 0.90 & 0.85 \\
7608 & 101048 & 1998 & 7336.80 & 0.39 & 673525.00 & 6741146.96 & 1.09 & 0.92 & 1.00 \\
30898 & 105807 & 1998 & 41.20 & 0.15 & 3165.00 & 37827.94 & 1.30 & 0.92 & 1.20 \\
2475 & 100333 & 1998 & 124.10 & 0.03 & 12198.00 & 118106.94 & 1.02 & 0.95 & 0.97 \\
22274 & 102999 & 1998 & 816.00 & -0.03 & 82061.00 & 785066.44 & 0.99 & 0.96 & 0.96 \\
22921 & 103085 & 1998 & 134.90 & -0.07 & 13490.00 & 130066.52 & 1.00 & 0.96 & 0.96 \\
30760 & 105794 & 1998 & 8.40 & 0.19 & 959.00 & 9106.80 & 0.88 & 1.08 & 0.95 \\
22890 & 103084 & 1998 & 445.80 & -0.14 & 44605.00 & 412896.02 & 1.00 & 0.93 & 0.93 \\
22295 & 103005 & 1998 & 127.00 & 0.07 & 12656.00 & 115788.49 & 1.00 & 0.91 & 0.91 \\
22331 & 103007 & 1998 & 1780.00 & 0.14 & 177294.00 & 1712264.37 & 1.00 & 0.96 & 0.97 \\
2750 & 100357 & 1998 & 307.20 & 0.27 & 21963.00 & 254559.03 & 1.40 & 0.83 & 1.16 \\
34658 & 106268 & 1998 & 202.90 & -0.00 & 22908.00 & 190721.41 & 0.89 & 0.94 & 0.83 \\
22484 & 103015 & 1998 & 204.10 & 0.35 & 21284.00 & 195989.40 & 0.96 & 0.96 & 0.92 \\
13012 & 101621 & 1998 & 4721.20 & 0.19 & 413501.00 & 3725110.70 & 1.14 & 0.79 & 0.90 \\
10209 & 101274 & 1998 & 237.60 & 0.03 & 20404.00 & 237898.08 & 1.16 & 1.00 & 1.17 \\
22375 & 103008 & 1998 & 113.70 & 0.10 & 11316.00 & 107752.97 & 1.00 & 0.95 & 0.95 \\
54594 & 377010 & 1998 & 1.90 & 0.06 & 121.00 & 1211.69 & 1.57 & 0.64 & 1.00 \\
10256 & 101276 & 1998 & 683.60 & -0.00 & 68362.00 & 570420.20 & 1.00 & 0.83 & 0.83 \\
22411 & 103011 & 1998 & 121.00 & 0.06 & 11805.00 & 114456.42 & 1.02 & 0.95 & 0.97 \\
2719 & 100355 & 1998 & 4381.30 & 0.28 & 322578.00 & 3284983.17 & 1.36 & 0.75 & 1.02 \\
12998 & 101618 & 1998 & 622.20 & -0.02 & 64488.00 & 591226.58 & 0.96 & 0.95 & 0.92 \\
34580 & 106256 & 1998 & 44.40 & -0.04 & 3609.00 & 37294.46 & 1.23 & 0.84 & 1.03 \\
7638 & 101050 & 1998 & 178.90 & 0.16 & 19198.00 & 178611.61 & 0.93 & 1.00 & 0.93 \\
22466 & 103014 & 1998 & 566.50 & 0.22 & 56622.00 & 557610.47 & 1.00 & 0.98 & 0.98 \\
12880 & 101603 & 1998 & 1249.80 & 0.19 & 126402.00 & 1144497.48 & 0.99 & 0.92 & 0.91 \\
54822 & 400017 & 1998 & 15.50 & 0.20 & 1474.00 & 14751.54 & 1.05 & 0.95 & 1.00 \\
34763 & 106276 & 1998 & 76.80 & 0.34 & 5091.00 & 50910.72 & 1.51 & 0.66 & 1.00 \\
34863 & 106283 & 1998 & 287.50 & -0.08 & 21558.00 & 215372.42 & 1.33 & 0.75 & 1.00 \\
21328 & 102852 & 1998 & 394.30 & 0.28 & 39526.00 & 375619.80 & 1.00 & 0.95 & 0.95 \\
3239 & 100417 & 1998 & 9.40 & 0.29 & 747.00 & 8819.90 & 1.26 & 0.94 & 1.18 \\
21359 & 102854 & 1998 & 678.30 & 0.27 & 67985.00 & 680168.90 & 1.00 & 1.00 & 1.00 \\
2271 & 100305 & 1998 & 69.30 & 0.11 & 9187.00 & 68354.01 & 0.75 & 0.99 & 0.74 \\
8997 & 101108 & 1998 & 748.00 & -0.07 & 91857.00 & 665724.78 & 0.81 & 0.89 & 0.72 \\
62168 & 500359 & 1998 & 205.90 & 0.21 & 14446.00 & 120374.58 & 1.43 & 0.58 & 0.83 \\
23367 & 103172 & 1998 & 62.00 & -0.04 & 6181.00 & 57148.25 & 1.00 & 0.92 & 0.92 \\
34836 & 106282 & 1998 & 405.50 & 0.15 & 40785.00 & 327885.22 & 0.99 & 0.81 & 0.80 \\
13280 & 101717 & 1998 & 74.60 & 0.24 & 7466.00 & 72452.02 & 1.00 & 0.97 & 0.97 \\
34442 & 106240 & 1998 & 152.70 & -0.13 & 14564.00 & 145600.75 & 1.05 & 0.95 & 1.00 \\
21392 & 102861 & 1998 & 223.40 & 0.24 & 21121.00 & 221889.08 & 1.06 & 0.99 & 1.05 \\
23349 & 103166 & 1998 & 4.80 & 0.59 & 381.00 & 4920.99 & 1.26 & 1.03 & 1.29 \\
31081 & 105857 & 1998 & 138.40 & -0.24 & 19694.00 & 129754.47 & 0.70 & 0.94 & 0.66 \\
13271 & 101716 & 1998 & 37.30 & 0.03 & 3428.00 & 38550.42 & 1.09 & 1.03 & 1.12 \\
23336 & 103165 & 1998 & 2.20 & 0.26 & 163.00 & 2195.79 & 1.35 & 1.00 & 1.35 \\
7525 & 101043 & 1998 & 2897.00 & 0.43 & 247032.00 & 2623765.18 & 1.17 & 0.91 & 1.06 \\
12753 & 101592 & 1998 & 331.20 & -0.01 & 33305.00 & 319893.08 & 0.99 & 0.97 & 0.96 \\
34469 & 106244 & 1998 & 4.00 & -0.03 & 397.00 & 3747.21 & 1.01 & 0.94 & 0.94 \\
34825 & 106281 & 1998 & 25.30 & 0.34 & 2121.00 & 24216.33 & 1.19 & 0.96 & 1.14 \\
21420 & 102871 & 1998 & 165.30 & -0.03 & 16687.00 & 163459.26 & 0.99 & 0.99 & 0.98 \\
13264 & 101714 & 1998 & 24.40 & 0.06 & 2443.00 & 21884.99 & 1.00 & 0.90 & 0.90 \\
21451 & 102872 & 1998 & 572.00 & 0.07 & 56878.00 & 518493.29 & 1.01 & 0.91 & 0.91 \\
31066 & 105854 & 1998 & 393.20 & 0.01 & 34927.00 & 402522.28 & 1.13 & 1.02 & 1.15 \\
23381 & 103174 & 1998 & 2117.70 & 0.27 & 212473.00 & 1869185.86 & 1.00 & 0.88 & 0.88 \\
23445 & 103177 & 1998 & 348.60 & 0.06 & 34640.00 & 331247.90 & 1.01 & 0.95 & 0.96 \\
21107 & 102832 & 1998 & 171.60 & 0.54 & 17167.00 & 152076.67 & 1.00 & 0.89 & 0.89 \\
54812 & 400015 & 1998 & 10.10 & 0.02 & 1104.00 & 8634.01 & 0.91 & 0.85 & 0.78 \\
21139 & 102833 & 1998 & 47.10 & 0.59 & 4710.00 & 43630.70 & 1.00 & 0.93 & 0.93 \\
12718 & 101590 & 1998 & 17.90 & 0.27 & 1424.00 & 18048.34 & 1.26 & 1.01 & 1.27 \\
31136 & 105861 & 1998 & 279.00 & -0.06 & 23859.00 & 253035.49 & 1.17 & 0.91 & 1.06 \\
21161 & 102835 & 1998 & 117.30 & 0.02 & 10318.00 & 92713.50 & 1.14 & 0.79 & 0.90 \\
13328 & 101728 & 1998 & 126.30 & 0.18 & 12645.00 & 125144.35 & 1.00 & 0.99 & 0.99 \\
7831 & 101062 & 1998 & 1239.30 & -0.19 & 162733.00 & 1356179.39 & 0.76 & 1.09 & 0.83 \\
7492 & 101042 & 1998 & 10368.60 & 0.36 & 734203.00 & 7288704.28 & 1.41 & 0.70 & 0.99 \\
30590 & 105775 & 1998 & 3688.50 & 0.05 & 368553.00 & 3125970.69 & 1.00 & 0.85 & 0.85 \\
21495 & 102875 & 1998 & 48.60 & 0.06 & 4577.00 & 45770.58 & 1.06 & 0.94 & 1.00 \\
34393 & 106231 & 1998 & 177.70 & -0.19 & 17853.00 & 178626.08 & 1.00 & 1.01 & 1.00 \\
10320 & 101279 & 1998 & 49.90 & 0.05 & 4981.00 & 46648.58 & 1.00 & 0.93 & 0.94 \\
34420 & 106236 & 1998 & 83.70 & 0.20 & 7317.00 & 83434.34 & 1.14 & 1.00 & 1.14 \\
13314 & 101723 & 1998 & 32.80 & 0.15 & 2843.00 & 28474.32 & 1.15 & 0.87 & 1.00 \\
2258 & 100303 & 1998 & 252.40 & 0.25 & 23725.00 & 197705.66 & 1.06 & 0.78 & 0.83 \\
21251 & 102843 & 1998 & 541.70 & 0.03 & 54248.00 & 536828.38 & 1.00 & 0.99 & 0.99 \\
54786 & 400014 & 1998 & 5.40 & 0.15 & 549.00 & 4760.46 & 0.98 & 0.88 & 0.87 \\
21275 & 102844 & 1998 & 878.40 & 0.21 & 90130.00 & 862447.30 & 0.97 & 0.98 & 0.96 \\
21300 & 102846 & 1998 & 101.60 & 0.16 & 10111.00 & 96608.56 & 1.00 & 0.95 & 0.96 \\
23411 & 103175 & 1998 & 1916.50 & 0.10 & 191423.00 & 1801105.20 & 1.00 & 0.94 & 0.94 \\
34423 & 106238 & 1998 & 3.90 & 0.03 & 336.00 & 4014.72 & 1.16 & 1.03 & 1.19 \\
22679 & 103028 & 1998 & 4857.20 & 0.18 & 344507.00 & 3996962.86 & 1.41 & 0.82 & 1.16 \\
21193 & 102837 & 1998 & 982.50 & 0.14 & 98253.00 & 932543.06 & 1.00 & 0.95 & 0.95 \\
7799 & 101061 & 1998 & 1683.70 & 0.28 & 153629.00 & 1605159.18 & 1.10 & 0.95 & 1.04 \\
21506 & 102876 & 1998 & 58.60 & -0.04 & 6810.00 & 68128.98 & 0.86 & 1.16 & 1.00 \\
2312 & 100315 & 1998 & 512.10 & 0.32 & 51182.00 & 479171.14 & 1.00 & 0.94 & 0.94 \\
23215 & 103145 & 1998 & 224.40 & 0.07 & 21164.00 & 234620.76 & 1.06 & 1.05 & 1.11 \\
21749 & 102949 & 1998 & 2355.50 & 0.09 & 200417.00 & 2244104.78 & 1.18 & 0.95 & 1.12 \\
31020 & 105848 & 1998 & 70.00 & -0.01 & 6984.00 & 65333.25 & 1.00 & 0.93 & 0.94 \\
21767 & 102951 & 1998 & 7633.90 & 0.06 & 829258.00 & 6961537.86 & 0.92 & 0.91 & 0.84 \\
23185 & 103144 & 1998 & 171.00 & 0.02 & 15359.00 & 177728.38 & 1.11 & 1.04 & 1.16 \\
63002 & 500466 & 1998 & 261.30 & 0.30 & 21501.00 & 249787.31 & 1.22 & 0.96 & 1.16 \\
3052 & 100401 & 1998 & 398.00 & 0.16 & 38952.00 & 326229.75 & 1.02 & 0.82 & 0.84 \\
13180 & 101703 & 1998 & 48497.60 & 0.23 & 4849758.00 & 39496839.80 & 1.00 & 0.81 & 0.81 \\
31012 & 105847 & 1998 & 49.80 & 0.09 & 4986.00 & 47603.84 & 1.00 & 0.96 & 0.95 \\
3043 & 100400 & 1998 & 1006.70 & 0.24 & 89162.00 & 867089.79 & 1.13 & 0.86 & 0.97 \\
31028 & 105849 & 1998 & 42.30 & 0.29 & 3829.00 & 38236.48 & 1.10 & 0.90 & 1.00 \\
21811 & 102952 & 1998 & 782.90 & 0.08 & 83586.00 & 773427.39 & 0.94 & 0.99 & 0.93 \\
30996 & 105846 & 1998 & 306.90 & 0.39 & 22408.00 & 278861.23 & 1.37 & 0.91 & 1.24 \\
30681 & 105783 & 1998 & 923.60 & -0.16 & 100541.00 & 939387.39 & 0.92 & 1.02 & 0.93 \\
10291 & 101278 & 1998 & 279.10 & -0.24 & 24481.00 & 225933.34 & 1.14 & 0.81 & 0.92 \\
13167 & 101698 & 1998 & 432.10 & 0.15 & 42899.00 & 417885.35 & 1.01 & 0.97 & 0.97 \\
23156 & 103136 & 1998 & 424.80 & 0.29 & 42512.00 & 424367.62 & 1.00 & 1.00 & 1.00 \\
12802 & 101600 & 1998 & 3092.10 & 0.23 & 309002.00 & 3043964.45 & 1.00 & 0.98 & 0.99 \\
30985 & 105845 & 1998 & 19.80 & 0.25 & 1966.00 & 19644.32 & 1.01 & 0.99 & 1.00 \\
21882 & 102964 & 1998 & 606.00 & 0.12 & 52610.00 & 542079.44 & 1.15 & 0.89 & 1.03 \\
13154 & 101681 & 1998 & 316.80 & 0.15 & 33214.00 & 278312.49 & 0.95 & 0.88 & 0.84 \\
30978 & 105843 & 1998 & 20.20 & 0.18 & 1996.00 & 19482.02 & 1.01 & 0.96 & 0.98 \\
10144 & 101263 & 1998 & 542.40 & 0.18 & 57928.00 & 474628.30 & 0.94 & 0.88 & 0.82 \\
2396 & 100322 & 1998 & 1453.80 & 0.06 & 146145.00 & 1352960.82 & 0.99 & 0.93 & 0.93 \\
3082 & 100408 & 1998 & 248.50 & 0.34 & 24987.00 & 240507.54 & 0.99 & 0.97 & 0.96 \\
21675 & 102939 & 1998 & 5592.70 & 0.27 & 469767.00 & 5228560.16 & 1.19 & 0.93 & 1.11 \\
12763 & 101593 & 1998 & 67.40 & 0.15 & 9675.00 & 85445.90 & 0.70 & 1.27 & 0.88 \\
3181 & 100413 & 1998 & 48.40 & 0.26 & 4843.00 & 40385.07 & 1.00 & 0.83 & 0.83 \\
30641 & 105780 & 1998 & 181.00 & 0.32 & 14706.00 & 172605.85 & 1.23 & 0.95 & 1.17 \\
2332 & 100319 & 1998 & 176.30 & -0.11 & 17596.00 & 174036.22 & 1.00 & 0.99 & 0.99 \\
23307 & 103160 & 1998 & 106.50 & -0.04 & 10725.00 & 107256.63 & 0.99 & 1.01 & 1.00 \\
34495 & 106248 & 1998 & 144.30 & -0.03 & 12479.00 & 111669.45 & 1.16 & 0.77 & 0.89 \\
21530 & 102893 & 1998 & 25.40 & -0.14 & 3566.00 & 30783.09 & 0.71 & 1.21 & 0.86 \\
48674 & 240118 & 1998 & 109.90 & -0.01 & 11294.00 & 110455.42 & 0.97 & 1.01 & 0.98 \\
30650 & 105781 & 1998 & 118.20 & -0.12 & 19262.00 & 180109.92 & 0.61 & 1.52 & 0.94 \\
21594 & 102895 & 1998 & 2089.30 & 0.24 & 207898.00 & 1986111.28 & 1.00 & 0.95 & 0.96 \\
3139 & 100411 & 1998 & 4890.00 & -0.01 & 488354.00 & 4783339.60 & 1.00 & 0.98 & 0.98 \\
34522 & 106249 & 1998 & 72.00 & 0.27 & 7032.00 & 70164.50 & 1.02 & 0.97 & 1.00 \\
23295 & 103158 & 1998 & 2603.10 & 0.29 & 254310.00 & 2526456.29 & 1.02 & 0.97 & 0.99 \\
21705 & 102940 & 1998 & 731.30 & -0.04 & 62926.00 & 691573.98 & 1.16 & 0.95 & 1.10 \\
21620 & 102901 & 1998 & 68.40 & 0.03 & 6810.00 & 63808.22 & 1.00 & 0.93 & 0.94 \\
23280 & 103154 & 1998 & 1015.20 & 0.14 & 100813.00 & 1007188.03 & 1.01 & 0.99 & 1.00 \\
10127 & 101262 & 1998 & 19.00 & 0.06 & 1906.00 & 18549.77 & 1.00 & 0.98 & 0.97 \\
34813 & 106278 & 1998 & 134.40 & 0.04 & 13417.00 & 125122.05 & 1.00 & 0.93 & 0.93 \\
13220 & 101708 & 1998 & 471.80 & 0.09 & 47334.00 & 464904.24 & 1.00 & 0.99 & 0.98 \\
3104 & 100409 & 1998 & 356.80 & -0.01 & 35904.00 & 348413.57 & 0.99 & 0.98 & 0.97 \\
31038 & 105852 & 1998 & 42.80 & 0.08 & 3665.00 & 37160.51 & 1.17 & 0.87 & 1.01 \\
23248 & 103152 & 1998 & 2402.10 & 0.20 & 228648.00 & 2304462.47 & 1.05 & 0.96 & 1.01 \\
12781 & 101595 & 1998 & 2051.50 & 0.64 & 179247.00 & 1758143.51 & 1.14 & 0.86 & 0.98 \\
31033 & 105851 & 1998 & 32.50 & 0.24 & 3300.00 & 31147.49 & 0.98 & 0.96 & 0.94 \\
34790 & 106277 & 1998 & 27.30 & 0.15 & 2831.00 & 28312.20 & 0.96 & 1.04 & 1.00 \\
2364 & 100320 & 1998 & 82.80 & -0.04 & 8254.00 & 80909.10 & 1.00 & 0.98 & 0.98 \\
7556 & 101045 & 1998 & 19453.70 & 0.23 & 1861113.00 & 17192276.91 & 1.05 & 0.88 & 0.92 \\
19978 & 102660 & 1998 & 4330.10 & 0.29 & 431728.00 & 4114139.32 & 1.00 & 0.95 & 0.95 \\
19948 & 102659 & 1998 & 5228.50 & 0.06 & 523414.00 & 4881597.06 & 1.00 & 0.93 & 0.93 \\
16740 & 102183 & 1998 & 73.80 & 0.27 & 7479.00 & 71568.75 & 0.99 & 0.97 & 0.96 \\
32358 & 106014 & 1998 & 52.40 & 0.03 & 5472.00 & 51300.77 & 0.96 & 0.98 & 0.94 \\
16723 & 102182 & 1998 & 57.90 & 0.09 & 5766.00 & 56878.61 & 1.00 & 0.98 & 0.99 \\
9216 & 101119 & 1998 & 42.20 & -0.05 & 4219.00 & 40829.82 & 1.00 & 0.97 & 0.97 \\
33308 & 106114 & 1998 & 58.50 & 0.14 & 5443.00 & 57017.69 & 1.07 & 0.97 & 1.05 \\
52293 & 302760 & 1998 & 16.90 & 0.08 & 1686.00 & 16693.94 & 1.00 & 0.99 & 0.99 \\
5646 & 100784 & 1998 & 4355.30 & 0.05 & 436460.00 & 4136525.15 & 1.00 & 0.95 & 0.95 \\
16689 & 102178 & 1998 & 877.60 & 0.19 & 86619.00 & 832546.36 & 1.01 & 0.95 & 0.96 \\
33281 & 106113 & 1998 & 141.40 & 0.28 & 14144.00 & 131627.87 & 1.00 & 0.93 & 0.93 \\
5676 & 100785 & 1998 & 842.40 & 0.33 & 63860.00 & 717242.70 & 1.32 & 0.85 & 1.12 \\
46265 & 200205 & 1998 & 68.50 & -0.18 & 9913.00 & 68962.88 & 0.69 & 1.01 & 0.70 \\
6869 & 100966 & 1998 & 23.80 & -0.05 & 2902.00 & 23609.78 & 0.82 & 0.99 & 0.81 \\
33335 & 106116 & 1998 & 22.10 & -0.12 & 2044.00 & 20437.88 & 1.08 & 0.92 & 1.00 \\
52470 & 302944 & 1998 & 24.40 & 0.09 & 2525.00 & 21793.60 & 0.97 & 0.89 & 0.86 \\
5602 & 100773 & 1998 & 1360.20 & -0.07 & 159657.00 & 1306570.58 & 0.85 & 0.96 & 0.82 \\
16845 & 102197 & 1998 & 60.30 & 0.24 & 5700.00 & 56826.23 & 1.06 & 0.94 & 1.00 \\
57839 & 401082 & 1998 & 8.60 & -0.18 & 861.00 & 7250.49 & 1.00 & 0.84 & 0.84 \\
52286 & 302732 & 1998 & 71.20 & 0.37 & 6703.00 & 62513.39 & 1.06 & 0.88 & 0.93 \\
47743 & 221051 & 1998 & 4946.10 & 0.36 & 455139.00 & 4435818.58 & 1.09 & 0.90 & 0.97 \\
32305 & 106010 & 1998 & 317.20 & 0.14 & 23647.00 & 234619.64 & 1.34 & 0.74 & 0.99 \\
5535 & 100771 & 1998 & 104.50 & -0.15 & 10524.00 & 91063.60 & 0.99 & 0.87 & 0.87 \\
16816 & 102193 & 1998 & 795.40 & 0.15 & 75961.00 & 759561.22 & 1.05 & 0.95 & 1.00 \\
33379 & 106127 & 1998 & 170.20 & 0.14 & 14682.00 & 157132.78 & 1.16 & 0.92 & 1.07 \\
5563 & 100772 & 1998 & 336.00 & -0.06 & 49272.00 & 423677.38 & 0.68 & 1.26 & 0.86 \\
16773 & 102191 & 1998 & 65.80 & 0.31 & 5503.00 & 61124.40 & 1.20 & 0.93 & 1.11 \\
9581 & 101151 & 1998 & 193.40 & 0.24 & 19285.00 & 179210.39 & 1.00 & 0.93 & 0.93 \\
6877 & 100967 & 1998 & 376.20 & 0.06 & 30806.00 & 282298.16 & 1.22 & 0.75 & 0.92 \\
14553 & 101876 & 1998 & 245.70 & 0.21 & 24573.00 & 244454.16 & 1.00 & 0.99 & 0.99 \\
33346 & 106123 & 1998 & 599.20 & 0.23 & 60555.00 & 593369.99 & 0.99 & 0.99 & 0.98 \\
33353 & 106124 & 1998 & 33.10 & 0.05 & 3481.00 & 33363.54 & 0.95 & 1.01 & 0.96 \\
33276 & 106110 & 1998 & 260.90 & 0.20 & 26061.00 & 258728.90 & 1.00 & 0.99 & 0.99 \\
52319 & 302763 & 1998 & 9.60 & 0.00 & 953.00 & 8197.10 & 1.01 & 0.85 & 0.86 \\
16538 & 102154 & 1998 & 250.90 & 0.18 & 24995.00 & 212204.60 & 1.00 & 0.85 & 0.85 \\
14657 & 101906 & 1998 & 44.50 & -0.06 & 4368.00 & 41161.92 & 1.02 & 0.92 & 0.94 \\
16515 & 102152 & 1998 & 342.40 & 0.18 & 34066.00 & 311567.53 & 1.01 & 0.91 & 0.91 \\
16493 & 102151 & 1998 & 18.90 & -0.01 & 2460.00 & 18620.64 & 0.77 & 0.99 & 0.76 \\
5726 & 100790 & 1998 & 272.00 & 0.18 & 26482.00 & 230258.36 & 1.03 & 0.85 & 0.87 \\
33225 & 106104 & 1998 & 1.20 & 0.22 & 122.00 & 1137.81 & 0.98 & 0.95 & 0.93 \\
52338 & 302780 & 1998 & 231.40 & 0.12 & 22652.00 & 198497.34 & 1.02 & 0.86 & 0.88 \\
16462 & 102150 & 1998 & 124.70 & -0.09 & 12646.00 & 118006.15 & 0.99 & 0.95 & 0.93 \\
16440 & 102145 & 1998 & 137.00 & 0.06 & 13690.00 & 132547.02 & 1.00 & 0.97 & 0.97 \\
16412 & 102134 & 1998 & 132.80 & 0.00 & 13250.00 & 126035.84 & 1.00 & 0.95 & 0.95 \\
16403 & 102133 & 1998 & 49.70 & 0.00 & 5052.00 & 47039.22 & 0.98 & 0.95 & 0.93 \\
33214 & 106103 & 1998 & 18.60 & 0.33 & 1869.00 & 18382.87 & 1.00 & 0.99 & 0.98 \\
14671 & 101908 & 1998 & 9.10 & 0.03 & 767.00 & 9550.09 & 1.19 & 1.05 & 1.25 \\
5504 & 100769 & 1998 & 2510.00 & -0.07 & 260186.00 & 2055761.28 & 0.96 & 0.82 & 0.79 \\
33235 & 106107 & 1998 & 91.50 & 0.15 & 9163.00 & 95591.79 & 1.00 & 1.04 & 1.04 \\
16646 & 102173 & 1998 & 93.30 & 0.12 & 9188.00 & 90963.55 & 1.02 & 0.97 & 0.99 \\
32402 & 106023 & 1998 & 8.00 & 0.12 & 1015.00 & 8114.80 & 0.79 & 1.01 & 0.80 \\
57738 & 401015 & 1998 & 4924.60 & 0.25 & 828111.00 & 7830804.09 & 0.59 & 1.59 & 0.95 \\
16613 & 102163 & 1998 & 1048.00 & 0.02 & 138240.00 & 1106055.49 & 0.76 & 1.06 & 0.80 \\
6833 & 100962 & 1998 & 2463.60 & 0.04 & 247312.00 & 2245876.75 & 1.00 & 0.91 & 0.91 \\
6811 & 100958 & 1998 & 4.80 & 0.18 & 454.00 & 4673.52 & 1.06 & 0.97 & 1.03 \\
14627 & 101903 & 1998 & 617.20 & -0.16 & 61173.00 & 520824.72 & 1.01 & 0.84 & 0.85 \\
9543 & 101149 & 1998 & 1147.70 & 0.30 & 130139.00 & 1219515.04 & 0.88 & 1.06 & 0.94 \\
16605 & 102160 & 1998 & 1.20 & -0.08 & 102.00 & 1113.33 & 1.18 & 0.93 & 1.09 \\
32443 & 106028 & 1998 & 28.70 & 0.35 & 2967.00 & 29642.73 & 0.97 & 1.03 & 1.00 \\
33266 & 106109 & 1998 & 32.60 & 0.01 & 3289.00 & 28383.18 & 0.99 & 0.87 & 0.86 \\
16395 & 102132 & 1998 & 81.80 & 0.34 & 8355.00 & 80233.99 & 0.98 & 0.98 & 0.96 \\
14511 & 101871 & 1998 & 482.70 & 0.10 & 48268.00 & 482935.20 & 1.00 & 1.00 & 1.00 \\
33549 & 106148 & 1998 & 94.50 & 0.35 & 6670.00 & 78171.97 & 1.42 & 0.83 & 1.17 \\
14386 & 101853 & 1998 & 652.70 & 0.00 & 65275.00 & 591070.17 & 1.00 & 0.91 & 0.91 \\
45805 & 200142 & 1998 & 58.70 & 0.23 & 7019.00 & 49127.42 & 0.84 & 0.84 & 0.70 \\
17189 & 102270 & 1998 & 1401.00 & 0.29 & 137405.00 & 1374001.46 & 1.02 & 0.98 & 1.00 \\
33535 & 106147 & 1998 & 69.40 & 0.02 & 5669.00 & 71482.62 & 1.22 & 1.03 & 1.26 \\
5289 & 100746 & 1998 & 1575.60 & 0.11 & 169221.00 & 1679549.06 & 0.93 & 1.07 & 0.99 \\
32176 & 105997 & 1998 & 55.90 & -0.17 & 6034.00 & 59917.57 & 0.93 & 1.07 & 0.99 \\
14402 & 101854 & 1998 & 8079.20 & 0.49 & 807921.00 & 7094840.29 & 1.00 & 0.88 & 0.88 \\
17160 & 102261 & 1998 & 2426.50 & 0.17 & 213551.00 & 2461674.83 & 1.14 & 1.01 & 1.15 \\
52213 & 302677 & 1998 & 1.90 & 0.07 & 182.00 & 1814.44 & 1.04 & 0.95 & 1.00 \\
57911 & 402003 & 1998 & 170.50 & 0.13 & 17863.00 & 156313.43 & 0.95 & 0.92 & 0.88 \\
32180 & 105999 & 1998 & 8.20 & -0.06 & 762.00 & 8346.80 & 1.08 & 1.02 & 1.10 \\
33530 & 106144 & 1998 & 38.00 & 0.08 & 3809.00 & 38566.19 & 1.00 & 1.01 & 1.01 \\
5323 & 100753 & 1998 & 2384.40 & 0.09 & 253314.00 & 2187211.23 & 0.94 & 0.92 & 0.86 \\
6977 & 100978 & 1998 & 86.10 & 0.10 & 8611.00 & 85332.40 & 1.00 & 0.99 & 0.99 \\
5268 & 100745 & 1998 & 2794.20 & 0.22 & 252452.00 & 2655902.22 & 1.11 & 0.95 & 1.05 \\
52187 & 302676 & 1998 & 1.40 & -0.16 & 135.00 & 1320.37 & 1.04 & 0.94 & 0.98 \\
7003 & 100982 & 1998 & 1622.40 & 0.13 & 161220.00 & 1462633.16 & 1.01 & 0.90 & 0.91 \\
17290 & 102278 & 1998 & 428.90 & 0.07 & 42698.00 & 410212.90 & 1.00 & 0.96 & 0.96 \\
6989 & 100981 & 1998 & 62.90 & 0.28 & 6289.00 & 61133.55 & 1.00 & 0.97 & 0.97 \\
33572 & 106150 & 1998 & 54.30 & 0.13 & 5498.00 & 51525.90 & 0.99 & 0.95 & 0.94 \\
14352 & 101851 & 1998 & 1798.20 & 0.24 & 179819.00 & 1583284.83 & 1.00 & 0.88 & 0.88 \\
33560 & 106149 & 1998 & 545.20 & 0.68 & 37731.00 & 442551.37 & 1.44 & 0.81 & 1.17 \\
33503 & 106143 & 1998 & 131.00 & -0.02 & 12806.00 & 123826.69 & 1.02 & 0.95 & 0.97 \\
32129 & 105984 & 1998 & 104.50 & 0.15 & 10579.00 & 101707.41 & 0.99 & 0.97 & 0.96 \\
29 & 100003 & 1998 & 636.80 & 0.02 & 63555.00 & 597930.38 & 1.00 & 0.94 & 0.94 \\
5246 & 100741 & 1998 & 185.20 & -0.22 & 19736.00 & 193600.39 & 0.94 & 1.05 & 0.98 \\
9660 & 101161 & 1998 & 1590.70 & 0.10 & 141644.00 & 1662952.82 & 1.12 & 1.05 & 1.17 \\
47886 & 222658 & 1998 & 131.20 & 0.21 & 13257.00 & 127307.99 & 0.99 & 0.97 & 0.96 \\
17223 & 102271 & 1998 & 1901.20 & 0.11 & 190380.00 & 1906508.18 & 1.00 & 1.00 & 1.00 \\
57931 & 410003 & 1998 & 1867.90 & -0.05 & 191238.00 & 1776134.16 & 0.98 & 0.95 & 0.93 \\
32143 & 105987 & 1998 & 261.30 & 0.29 & 26359.00 & 255924.45 & 0.99 & 0.98 & 0.97 \\
32149 & 105990 & 1998 & 137.10 & 0.01 & 22831.00 & 214779.33 & 0.60 & 1.57 & 0.94 \\
17250 & 102274 & 1998 & 2110.90 & -0.07 & 216092.00 & 2003114.19 & 0.98 & 0.95 & 0.93 \\
47810 & 222027 & 1998 & 154.30 & 0.20 & 11275.00 & 138587.40 & 1.37 & 0.90 & 1.23 \\
32208 & 106000 & 1998 & 160.50 & 0.05 & 16550.00 & 151743.18 & 0.97 & 0.95 & 0.92 \\
16966 & 102224 & 1998 & 3032.40 & -0.03 & 251458.00 & 2750097.96 & 1.21 & 0.91 & 1.09 \\
33417 & 106135 & 1998 & 89.90 & -0.01 & 4625.00 & 41615.39 & 1.94 & 0.46 & 0.90 \\
9610 & 101158 & 1998 & 412.90 & 0.23 & 41235.00 & 396790.84 & 1.00 & 0.96 & 0.96 \\
47796 & 221485 & 1998 & 554.90 & 0.28 & 56110.00 & 547624.87 & 0.99 & 0.99 & 0.98 \\
5444 & 100763 & 1998 & 594.90 & 0.13 & 64502.00 & 618019.12 & 0.92 & 1.04 & 0.96 \\
6919 & 100969 & 1998 & 50.70 & 0.03 & 4686.00 & 44737.23 & 1.08 & 0.88 & 0.95 \\
32264 & 106008 & 1998 & 71.40 & 0.10 & 5595.00 & 50363.39 & 1.28 & 0.71 & 0.90 \\
33410 & 106133 & 1998 & 37.50 & 0.04 & 2709.00 & 34560.51 & 1.38 & 0.92 & 1.28 \\
16875 & 102213 & 1998 & 2254.90 & 0.33 & 226019.00 & 1958811.47 & 1.00 & 0.87 & 0.87 \\
33398 & 106129 & 1998 & 614.40 & 0.20 & 64543.00 & 632375.97 & 0.95 & 1.03 & 0.98 \\
5466 & 100764 & 1998 & 525.10 & -0.05 & 49795.00 & 485691.09 & 1.05 & 0.92 & 0.98 \\
57847 & 401145 & 1998 & 36.60 & 0.35 & 3653.00 & 34167.08 & 1.00 & 0.93 & 0.94 \\
5400 & 100760 & 1998 & 817.50 & 0.17 & 68804.00 & 780763.41 & 1.19 & 0.96 & 1.13 \\
14467 & 101861 & 1998 & 794.90 & 0.15 & 76646.00 & 764746.59 & 1.04 & 0.96 & 1.00 \\
5344 & 100754 & 1998 & 1402.20 & 0.10 & 110464.00 & 1231637.49 & 1.27 & 0.88 & 1.11 \\
45879 & 200153 & 1998 & 19.10 & 0.29 & 1948.00 & 17876.87 & 0.98 & 0.94 & 0.92 \\
17104 & 102257 & 1998 & 902.50 & 0.28 & 65528.00 & 786051.78 & 1.38 & 0.87 & 1.20 \\
17081 & 102255 & 1998 & 530.00 & 0.18 & 45240.00 & 540670.54 & 1.17 & 1.02 & 1.20 \\
6938 & 100973 & 1998 & 42.30 & 0.18 & 4228.00 & 42305.63 & 1.00 & 1.00 & 1.00 \\
14446 & 101858 & 1998 & 235.00 & 0.24 & 21620.00 & 207848.18 & 1.09 & 0.88 & 0.96 \\
33444 & 106136 & 1998 & 111.60 & 0.03 & 11117.00 & 97849.05 & 1.00 & 0.88 & 0.88 \\
57900 & 401372 & 1998 & 11.10 & 0.14 & 1115.00 & 9231.47 & 1.00 & 0.83 & 0.83 \\
52239 & 302698 & 1998 & 146.10 & 0.67 & 8644.00 & 120282.10 & 1.69 & 0.82 & 1.39 \\
5367 & 100758 & 1998 & 124.40 & 0.12 & 9189.00 & 91035.77 & 1.35 & 0.73 & 0.99 \\
17038 & 102231 & 1998 & 997.80 & 0.35 & 100774.00 & 952496.92 & 0.99 & 0.95 & 0.95 \\
32236 & 106007 & 1998 & 698.80 & 0.03 & 66814.00 & 605423.39 & 1.05 & 0.87 & 0.91 \\
5353 & 100757 & 1998 & 6.30 & 0.20 & 592.00 & 5227.21 & 1.06 & 0.83 & 0.88 \\
5195 & 100731 & 1998 & 19327.30 & 0.18 & 1600728.00 & 17259652.12 & 1.21 & 0.89 & 1.08 \\
52438 & 302941 & 1998 & 10.40 & 0.20 & 908.00 & 9234.54 & 1.15 & 0.89 & 1.02 \\
16360 & 102130 & 1998 & 987.60 & 0.24 & 98773.00 & 960199.26 & 1.00 & 0.97 & 0.97 \\
32718 & 106052 & 1998 & 152.20 & 0.25 & 15211.00 & 148190.11 & 1.00 & 0.97 & 0.97 \\
15423 & 101990 & 1998 & 144.30 & 0.05 & 13997.00 & 138915.33 & 1.03 & 0.96 & 0.99 \\
6329 & 100849 & 1998 & 78.70 & 0.16 & 7695.00 & 76952.79 & 1.02 & 0.98 & 1.00 \\
9411 & 101134 & 1998 & 99.40 & -0.23 & 9955.00 & 97798.34 & 1.00 & 0.98 & 0.98 \\
15395 & 101989 & 1998 & 262.60 & 0.16 & 26333.00 & 248426.80 & 1.00 & 0.95 & 0.94 \\
47185 & 200342 & 1998 & 5522.60 & 0.00 & 587394.00 & 5685798.34 & 0.94 & 1.03 & 0.97 \\
47469 & 212027 & 1998 & 48.20 & 0.21 & 4906.00 & 46984.76 & 0.98 & 0.97 & 0.96 \\
32733 & 106053 & 1998 & 8.10 & 0.07 & 817.00 & 7824.35 & 0.99 & 0.97 & 0.96 \\
15365 & 101988 & 1998 & 538.00 & 0.09 & 53734.00 & 493338.33 & 1.00 & 0.92 & 0.92 \\
6377 & 100856 & 1998 & 212.30 & 0.17 & 20982.00 & 209890.81 & 1.01 & 0.99 & 1.00 \\
15335 & 101987 & 1998 & 675.60 & 0.37 & 67561.00 & 643986.54 & 1.00 & 0.95 & 0.95 \\
52392 & 302879 & 1998 & 11.30 & 0.32 & 845.00 & 11500.83 & 1.34 & 1.02 & 1.36 \\
15325 & 101984 & 1998 & 325.80 & 0.12 & 32558.00 & 320145.88 & 1.00 & 0.98 & 0.98 \\
32737 & 106057 & 1998 & 100.70 & 0.14 & 8733.00 & 88935.11 & 1.15 & 0.88 & 1.02 \\
14951 & 101925 & 1998 & 407.60 & -0.01 & 35115.00 & 398283.12 & 1.16 & 0.98 & 1.13 \\
15457 & 101992 & 1998 & 912.40 & 0.20 & 91251.00 & 894555.35 & 1.00 & 0.98 & 0.98 \\
14910 & 101922 & 1998 & 184.20 & 0.15 & 15835.00 & 174754.06 & 1.16 & 0.95 & 1.10 \\
32712 & 106051 & 1998 & 65.80 & 0.23 & 5126.00 & 57636.75 & 1.28 & 0.88 & 1.12 \\
9321 & 101131 & 1998 & 1549.60 & 0.23 & 155309.00 & 1417196.77 & 1.00 & 0.91 & 0.91 \\
14873 & 101919 & 1998 & 1200.20 & 0.08 & 95656.00 & 1140154.88 & 1.25 & 0.95 & 1.19 \\
15586 & 102007 & 1998 & 6680.50 & 0.23 & 540802.00 & 6300813.75 & 1.24 & 0.94 & 1.17 \\
15571 & 102005 & 1998 & 1165.50 & 0.29 & 111617.00 & 1202135.03 & 1.04 & 1.03 & 1.08 \\
32990 & 106085 & 1998 & 61.20 & 0.32 & 4539.00 & 38262.45 & 1.35 & 0.63 & 0.84 \\
6229 & 100831 & 1998 & 233.00 & 0.10 & 23890.00 & 227922.68 & 0.98 & 0.98 & 0.95 \\
15527 & 102000 & 1998 & 687.40 & 0.16 & 82428.00 & 656100.78 & 0.83 & 0.95 & 0.80 \\
32912 & 106082 & 1998 & 229.60 & 0.11 & 23019.00 & 194491.35 & 1.00 & 0.85 & 0.84 \\
52347 & 302811 & 1998 & 5.30 & 0.03 & 530.00 & 4573.40 & 1.00 & 0.86 & 0.86 \\
47506 & 212408 & 1998 & 2378.60 & 0.44 & 241223.00 & 2327504.60 & 0.99 & 0.98 & 0.96 \\
6685 & 100910 & 1998 & 150.80 & 0.11 & 13408.00 & 130311.63 & 1.12 & 0.86 & 0.97 \\
32687 & 106049 & 1998 & 376.20 & 0.28 & 37678.00 & 363473.10 & 1.00 & 0.97 & 0.96 \\
15496 & 101999 & 1998 & 2486.10 & -0.03 & 248606.00 & 2465686.87 & 1.00 & 0.99 & 0.99 \\
15477 & 101998 & 1998 & 826.30 & -0.01 & 82632.00 & 753530.55 & 1.00 & 0.91 & 0.91 \\
32699 & 106050 & 1998 & 301.60 & 0.01 & 30823.00 & 302172.28 & 0.98 & 1.00 & 0.98 \\
6668 & 100908 & 1998 & 146.80 & 0.30 & 11684.00 & 136740.01 & 1.26 & 0.93 & 1.17 \\
6256 & 100833 & 1998 & 1049.00 & 0.30 & 104904.00 & 1037371.56 & 1.00 & 0.99 & 0.99 \\
47250 & 200344 & 1998 & 2009.80 & 0.19 & 185061.00 & 1726922.16 & 1.09 & 0.86 & 0.93 \\
15299 & 101982 & 1998 & 360.70 & 0.26 & 22738.00 & 354447.72 & 1.59 & 0.98 & 1.56 \\
32749 & 106060 & 1998 & 2.00 & 0.11 & 168.00 & 1873.79 & 1.19 & 0.94 & 1.12 \\
6503 & 100878 & 1998 & 2367.10 & 0.23 & 209466.00 & 2251392.32 & 1.13 & 0.95 & 1.07 \\
15007 & 101933 & 1998 & 251.50 & -0.04 & 24507.00 & 244052.13 & 1.03 & 0.97 & 1.00 \\
15144 & 101963 & 1998 & 1289.60 & -0.08 & 127888.00 & 1268485.30 & 1.01 & 0.98 & 0.99 \\
32860 & 106070 & 1998 & 440.40 & -0.02 & 63806.00 & 632778.57 & 0.69 & 1.44 & 0.99 \\
32853 & 106069 & 1998 & 103.60 & 0.24 & 10038.00 & 96926.00 & 1.03 & 0.94 & 0.97 \\
6539 & 100889 & 1998 & 42.60 & -0.02 & 4249.00 & 42494.46 & 1.00 & 1.00 & 1.00 \\
47378 & 210681 & 1998 & 38821.30 & 0.32 & 3903151.00 & 34643604.43 & 0.99 & 0.89 & 0.89 \\
15112 & 101958 & 1998 & 797.10 & 0.28 & 79238.00 & 782874.57 & 1.01 & 0.98 & 0.99 \\
32798 & 106064 & 1998 & 467.80 & 0.27 & 46783.00 & 456277.13 & 1.00 & 0.98 & 0.98 \\
15094 & 101956 & 1998 & 3149.40 & 0.39 & 317761.00 & 3040135.35 & 0.99 & 0.97 & 0.96 \\
32848 & 106068 & 1998 & 26.90 & 0.11 & 1375.00 & 12759.98 & 1.96 & 0.47 & 0.93 \\
32834 & 106067 & 1998 & 26.50 & 0.07 & 2920.00 & 28548.87 & 0.91 & 1.08 & 0.98 \\
47409 & 210770 & 1998 & 1582.20 & 0.12 & 178659.00 & 1715250.32 & 0.89 & 1.08 & 0.96 \\
15060 & 101955 & 1998 & 13785.40 & 0.25 & 1048744.00 & 10464593.03 & 1.31 & 0.76 & 1.00 \\
52383 & 302826 & 1998 & 75.00 & 0.15 & 7672.00 & 69613.31 & 0.98 & 0.93 & 0.91 \\
6548 & 100890 & 1998 & 1031.00 & 0.23 & 103000.00 & 972677.25 & 1.00 & 0.94 & 0.94 \\
15616 & 102009 & 1998 & 540.30 & 0.14 & 50265.00 & 456066.90 & 1.07 & 0.84 & 0.91 \\
15173 & 101964 & 1998 & 580.50 & 0.09 & 56348.00 & 563749.44 & 1.03 & 0.97 & 1.00 \\
14996 & 101930 & 1998 & 1323.00 & 0.23 & 128154.00 & 1282495.61 & 1.03 & 0.97 & 1.00 \\
15286 & 101978 & 1998 & 106.30 & 0.27 & 10639.00 & 105371.01 & 1.00 & 0.99 & 0.99 \\
52359 & 302813 & 1998 & 6.80 & 0.06 & 683.00 & 6103.22 & 1.00 & 0.90 & 0.89 \\
6428 & 100868 & 1998 & 205.80 & 0.14 & 16518.00 & 194037.88 & 1.25 & 0.94 & 1.17 \\
47457 & 211485 & 1998 & 556.90 & 0.30 & 55695.00 & 556012.51 & 1.00 & 1.00 & 1.00 \\
32752 & 106061 & 1998 & 143.10 & 0.23 & 11622.00 & 141188.91 & 1.23 & 0.99 & 1.21 \\
15250 & 101972 & 1998 & 956.50 & 0.00 & 95651.00 & 882314.80 & 1.00 & 0.92 & 0.92 \\
15234 & 101970 & 1998 & 69.40 & 0.12 & 6907.00 & 68487.72 & 1.00 & 0.99 & 0.99 \\
6635 & 100906 & 1998 & 1660.50 & 0.15 & 169454.00 & 1650733.70 & 0.98 & 0.99 & 0.97 \\
14983 & 101926 & 1998 & 452.50 & 0.18 & 46078.00 & 455594.58 & 0.98 & 1.01 & 0.99 \\
6454 & 100875 & 1998 & 199.70 & 0.03 & 19959.00 & 193544.14 & 1.00 & 0.97 & 0.97 \\
47334 & 210203 & 1998 & 4696.70 & 0.37 & 472173.00 & 4595334.56 & 0.99 & 0.98 & 0.97 \\
9385 & 101133 & 1998 & 1522.90 & 0.13 & 150555.00 & 1331814.46 & 1.01 & 0.87 & 0.88 \\
15215 & 101968 & 1998 & 87.90 & 0.57 & 8787.00 & 86197.67 & 1.00 & 0.98 & 0.98 \\
9348 & 101132 & 1998 & 105.10 & -0.28 & 10436.00 & 104100.70 & 1.01 & 0.99 & 1.00 \\
32776 & 106062 & 1998 & 8.00 & 0.02 & 710.00 & 6575.19 & 1.13 & 0.82 & 0.93 \\
32870 & 106075 & 1998 & 231.50 & 0.13 & 18770.00 & 202771.06 & 1.23 & 0.88 & 1.08 \\
5782 & 100792 & 1998 & 998.00 & -0.13 & 99940.00 & 960304.61 & 1.00 & 0.96 & 0.96 \\
6195 & 100829 & 1998 & 691.50 & -0.04 & 70359.00 & 664783.21 & 0.98 & 0.96 & 0.94 \\
6699 & 100913 & 1998 & 155.60 & 0.05 & 12579.00 & 135384.59 & 1.24 & 0.87 & 1.08 \\
16149 & 102087 & 1998 & 481.00 & 0.33 & 41887.00 & 379851.60 & 1.15 & 0.79 & 0.91 \\
9494 & 101140 & 1998 & 1220.10 & 0.42 & 123213.00 & 1219525.09 & 0.99 & 1.00 & 0.99 \\
52412 & 302907 & 1998 & 28.30 & 0.00 & 2766.00 & 27660.67 & 1.02 & 0.98 & 1.00 \\
47641 & 216438 & 1998 & 232.50 & 0.31 & 16614.00 & 198606.35 & 1.40 & 0.85 & 1.20 \\
5940 & 100812 & 1998 & 469.40 & 0.16 & 47586.00 & 469478.59 & 0.99 & 1.00 & 0.99 \\
16140 & 102085 & 1998 & 4352.90 & -0.24 & 583402.00 & 3842496.28 & 0.75 & 0.88 & 0.66 \\
16106 & 102080 & 1998 & 956.60 & -0.06 & 85246.00 & 936056.95 & 1.12 & 0.98 & 1.10 \\
9278 & 101127 & 1998 & 252.30 & -0.14 & 32369.00 & 234138.75 & 0.78 & 0.93 & 0.72 \\
33144 & 106097 & 1998 & 7.80 & 0.44 & 702.00 & 6397.28 & 1.11 & 0.82 & 0.91 \\
5973 & 100815 & 1998 & 319.00 & -0.06 & 31862.00 & 313624.28 & 1.00 & 0.98 & 0.98 \\
5981 & 100817 & 1998 & 91.30 & 0.26 & 5108.00 & 40998.11 & 1.79 & 0.45 & 0.80 \\
32559 & 106041 & 1998 & 18.00 & 0.20 & 1802.00 & 18766.61 & 1.00 & 1.04 & 1.04 \\
9482 & 101139 & 1998 & 47.30 & 0.17 & 4734.00 & 44042.74 & 1.00 & 0.93 & 0.93 \\
16168 & 102089 & 1998 & 214.20 & 0.14 & 20502.00 & 188036.47 & 1.04 & 0.88 & 0.92 \\
14739 & 101912 & 1998 & 3687.90 & 0.14 & 344937.00 & 3236487.62 & 1.07 & 0.88 & 0.94 \\
33161 & 106101 & 1998 & 11.90 & 0.03 & 1198.00 & 11450.87 & 0.99 & 0.96 & 0.96 \\
5822 & 100804 & 1998 & 4168.90 & 0.31 & 418580.00 & 3991174.63 & 1.00 & 0.96 & 0.95 \\
16306 & 102124 & 1998 & 2461.30 & 0.30 & 247008.00 & 2447111.34 & 1.00 & 0.99 & 0.99 \\
14705 & 101911 & 1998 & 1350.90 & 0.13 & 128377.00 & 1133943.03 & 1.05 & 0.84 & 0.88 \\
16281 & 102121 & 1998 & 55.70 & 0.28 & 5338.00 & 52610.65 & 1.04 & 0.94 & 0.99 \\
57239 & 400323 & 1998 & 49.10 & 0.06 & 4930.00 & 44063.62 & 1.00 & 0.90 & 0.89 \\
16271 & 102113 & 1998 & 246.70 & 0.25 & 24507.00 & 244678.03 & 1.01 & 0.99 & 1.00 \\
16075 & 102079 & 1998 & 546.40 & 0.04 & 68240.00 & 540191.35 & 0.80 & 0.99 & 0.79 \\
32506 & 106037 & 1998 & 51.80 & -0.04 & 5185.00 & 50487.38 & 1.00 & 0.97 & 0.97 \\
5869 & 100809 & 1998 & 2591.10 & 0.17 & 258055.00 & 2490167.48 & 1.00 & 0.96 & 0.96 \\
16255 & 102105 & 1998 & 197.50 & 0.17 & 19625.00 & 196408.43 & 1.01 & 0.99 & 1.00 \\
47650 & 216504 & 1998 & 73.90 & 0.65 & 7217.00 & 69962.75 & 1.02 & 0.95 & 0.97 \\
32515 & 106038 & 1998 & 221.30 & 0.01 & 21765.00 & 206057.82 & 1.02 & 0.93 & 0.95 \\
5901 & 100811 & 1998 & 1777.50 & 0.13 & 177839.00 & 1518172.36 & 1.00 & 0.85 & 0.85 \\
6767 & 100953 & 1998 & 39.50 & -0.21 & 3949.00 & 40283.52 & 1.00 & 1.02 & 1.02 \\
33121 & 106092 & 1998 & 828.40 & 0.10 & 89594.00 & 703520.44 & 0.92 & 0.85 & 0.79 \\
6010 & 100818 & 1998 & 138.00 & -0.31 & 13908.00 & 137471.90 & 0.99 & 1.00 & 0.99 \\
47620 & 215696 & 1998 & 235.00 & -0.04 & 23504.00 & 221698.96 & 1.00 & 0.94 & 0.94 \\
14829 & 101916 & 1998 & 418.70 & -0.01 & 35553.00 & 412184.93 & 1.18 & 0.98 & 1.16 \\
15787 & 102018 & 1998 & 828.90 & 0.32 & 75884.00 & 804104.75 & 1.09 & 0.97 & 1.06 \\
15756 & 102017 & 1998 & 12049.60 & 0.23 & 919406.00 & 8603772.97 & 1.31 & 0.71 & 0.94 \\
32636 & 106044 & 1998 & 38.80 & -0.02 & 4069.00 & 34557.26 & 0.95 & 0.89 & 0.85 \\
32641 & 106045 & 1998 & 35.30 & 0.26 & 2860.00 & 30969.72 & 1.23 & 0.88 & 1.08 \\
47564 & 212809 & 1998 & 3.60 & 0.09 & 316.00 & 3368.62 & 1.14 & 0.94 & 1.07 \\
6726 & 100925 & 1998 & 117.80 & -0.11 & 11798.00 & 102824.01 & 1.00 & 0.87 & 0.87 \\
6147 & 100825 & 1998 & 142.60 & 0.14 & 14257.00 & 139800.16 & 1.00 & 0.98 & 0.98 \\
15697 & 102015 & 1998 & 424.60 & 0.11 & 35719.00 & 443962.70 & 1.19 & 1.05 & 1.24 \\
47537 & 212658 & 1998 & 3310.20 & 0.22 & 329348.00 & 2941053.51 & 1.01 & 0.89 & 0.89 \\
9446 & 101135 & 1998 & 1567.90 & 0.43 & 156415.00 & 1415363.50 & 1.00 & 0.90 & 0.90 \\
32656 & 106046 & 1998 & 3.00 & -0.06 & 297.00 & 2846.94 & 1.01 & 0.95 & 0.96 \\
15677 & 102013 & 1998 & 1891.60 & 0.27 & 177139.00 & 1552278.19 & 1.07 & 0.82 & 0.88 \\
8 & 100001 & 1998 & 3305.30 & 0.28 & 330410.00 & 3207336.99 & 1.00 & 0.97 & 0.97 \\
15637 & 102010 & 1998 & 12942.60 & 0.24 & 1183922.00 & 12580489.66 & 1.09 & 0.97 & 1.06 \\
6160 & 100827 & 1998 & 327.40 & -0.06 & 32524.00 & 321456.48 & 1.01 & 0.98 & 0.99 \\
32660 & 106047 & 1998 & 6.80 & 0.22 & 681.00 & 6678.84 & 1.00 & 0.98 & 0.98 \\
9459 & 101137 & 1998 & 10.70 & 0.16 & 1061.00 & 10611.49 & 1.01 & 0.99 & 1.00 \\
33094 & 106091 & 1998 & 45.90 & -0.06 & 4539.00 & 44662.19 & 1.01 & 0.97 & 0.98 \\
16024 & 102073 & 1998 & 12451.40 & 0.11 & 1143952.00 & 11297814.27 & 1.09 & 0.91 & 0.99 \\
32587 & 106042 & 1998 & 6.70 & 0.05 & 593.00 & 5549.29 & 1.13 & 0.83 & 0.94 \\
47605 & 215687 & 1998 & 104.50 & 0.01 & 7713.00 & 94380.13 & 1.35 & 0.90 & 1.22 \\
52406 & 302881 & 1998 & 204.40 & 0.27 & 20439.00 & 201619.06 & 1.00 & 0.99 & 0.99 \\
6051 & 100821 & 1998 & 73.20 & 0.10 & 6147.00 & 71044.54 & 1.19 & 0.97 & 1.16 \\
15980 & 102062 & 1998 & 230.90 & 0.03 & 20403.00 & 196533.45 & 1.13 & 0.85 & 0.96 \\
6746 & 100947 & 1998 & 2199.80 & 0.07 & 189095.00 & 1795544.99 & 1.16 & 0.82 & 0.95 \\
15942 & 102061 & 1998 & 17.80 & 0.19 & 1404.00 & 14043.75 & 1.27 & 0.79 & 1.00 \\
14803 & 101914 & 1998 & 78.30 & 0.10 & 7067.00 & 70797.07 & 1.11 & 0.90 & 1.00 \\
15912 & 102059 & 1998 & 533.60 & 0.01 & 48100.00 & 520017.95 & 1.11 & 0.97 & 1.08 \\
6082 & 100822 & 1998 & 17.30 & -0.03 & 1465.00 & 17443.26 & 1.18 & 1.01 & 1.19 \\
15889 & 102052 & 1998 & 346.00 & 0.05 & 44923.00 & 396017.83 & 0.77 & 1.14 & 0.88 \\
33044 & 106088 & 1998 & 33.00 & 0.16 & 3232.00 & 32318.87 & 1.02 & 0.98 & 1.00 \\
33080 & 106090 & 1998 & 138.30 & -0.07 & 13033.00 & 130286.95 & 1.06 & 0.94 & 1.00 \\
31415 & 105882 & 1998 & 289.90 & 0.34 & 28957.00 & 284842.65 & 1.00 & 0.98 & 0.98 \\
32102 & 105983 & 1998 & 218.00 & 0.28 & 19151.00 & 214577.98 & 1.14 & 0.98 & 1.12 \\
9679 & 101165 & 1998 & 1387.60 & 0.28 & 108670.00 & 1270491.79 & 1.28 & 0.92 & 1.17 \\
33919 & 106189 & 1998 & 149.90 & 0.08 & 11548.00 & 142647.75 & 1.30 & 0.95 & 1.24 \\
31643 & 105920 & 1998 & 3073.20 & 0.38 & 305572.00 & 2903106.11 & 1.01 & 0.94 & 0.95 \\
4282 & 100600 & 1998 & 95.90 & -0.01 & 12803.00 & 104405.43 & 0.75 & 1.09 & 0.82 \\
18989 & 102540 & 1998 & 11.80 & -0.04 & 1003.00 & 10882.43 & 1.18 & 0.92 & 1.08 \\
18957 & 102531 & 1998 & 20.70 & 0.14 & 1669.00 & 20613.93 & 1.24 & 1.00 & 1.24 \\
7238 & 101015 & 1998 & 584.50 & 0.01 & 57341.00 & 572731.24 & 1.02 & 0.98 & 1.00 \\
33906 & 106182 & 1998 & 344.50 & -0.11 & 38507.00 & 317228.68 & 0.89 & 0.92 & 0.82 \\
13865 & 101781 & 1998 & 909.20 & -0.05 & 89940.00 & 856729.63 & 1.01 & 0.94 & 0.95 \\
18944 & 102529 & 1998 & 30.70 & -0.16 & 2993.00 & 30597.20 & 1.03 & 1.00 & 1.02 \\
4302 & 100603 & 1998 & 1933.30 & -0.18 & 224667.00 & 2099373.11 & 0.86 & 1.09 & 0.93 \\
48240 & 240056 & 1998 & 223.50 & 0.06 & 24292.00 & 185894.57 & 0.92 & 0.83 & 0.77 \\
18925 & 102528 & 1998 & 177.70 & 0.06 & 17771.00 & 167843.20 & 1.00 & 0.94 & 0.94 \\
18909 & 102527 & 1998 & 319.90 & 0.29 & 31812.00 & 318413.49 & 1.01 & 1.00 & 1.00 \\
13882 & 101785 & 1998 & 1570.60 & 0.41 & 155982.00 & 1514694.40 & 1.01 & 0.96 & 0.97 \\
31671 & 105926 & 1998 & 2.00 & -0.20 & 269.00 & 1966.34 & 0.74 & 0.98 & 0.73 \\
19021 & 102544 & 1998 & 1143.10 & 0.17 & 115080.00 & 1107435.14 & 0.99 & 0.97 & 0.96 \\
19045 & 102545 & 1998 & 226.90 & 0.09 & 23956.00 & 230521.05 & 0.95 & 1.02 & 0.96 \\
33926 & 106192 & 1998 & 871.70 & 0.30 & 74508.00 & 684802.21 & 1.17 & 0.79 & 0.92 \\
19201 & 102563 & 1998 & 325.50 & 0.19 & 32680.00 & 297419.20 & 1.00 & 0.91 & 0.91 \\
31603 & 105916 & 1998 & 151.70 & 0.18 & 18194.00 & 162728.22 & 0.83 & 1.07 & 0.89 \\
40418 & 108118 & 1998 & 27.40 & 0.05 & 2790.00 & 26982.48 & 0.98 & 0.98 & 0.97 \\
19143 & 102551 & 1998 & 393.60 & -0.10 & 37300.00 & 302595.62 & 1.06 & 0.77 & 0.81 \\
31609 & 105917 & 1998 & 23.40 & 0.16 & 2342.00 & 23082.65 & 1.00 & 0.99 & 0.99 \\
31619 & 105918 & 1998 & 394.40 & 0.38 & 38042.00 & 341731.27 & 1.04 & 0.87 & 0.90 \\
44396 & 109300 & 1998 & 896.90 & 0.09 & 75627.00 & 886837.81 & 1.19 & 0.99 & 1.17 \\
7276 & 101018 & 1998 & 19904.10 & 0.35 & 1713344.00 & 18442163.24 & 1.16 & 0.93 & 1.08 \\
19136 & 102550 & 1998 & 59.90 & -0.09 & 4846.00 & 39006.98 & 1.24 & 0.65 & 0.80 \\
9886 & 101200 & 1998 & 41.50 & 0.34 & 4516.00 & 45044.92 & 0.92 & 1.09 & 1.00 \\
19105 & 102549 & 1998 & 258.30 & -0.04 & 27824.00 & 218261.27 & 0.93 & 0.84 & 0.78 \\
13838 & 101769 & 1998 & 2775.70 & 0.14 & 277759.00 & 2743261.59 & 1.00 & 0.99 & 0.99 \\
19077 & 102548 & 1998 & 948.50 & 0.07 & 80838.00 & 754979.94 & 1.17 & 0.80 & 0.93 \\
9114 & 101112 & 1998 & 1747.00 & 0.15 & 160485.00 & 1764552.92 & 1.09 & 1.01 & 1.10 \\
19069 & 102547 & 1998 & 199.50 & 0.08 & 20186.00 & 194154.82 & 0.99 & 0.97 & 0.96 \\
18878 & 102525 & 1998 & 273.00 & 0.18 & 27084.00 & 270283.27 & 1.01 & 0.99 & 1.00 \\
31677 & 105930 & 1998 & 227.80 & 0.83 & 23218.00 & 223944.22 & 0.98 & 0.98 & 0.96 \\
18847 & 102524 & 1998 & 2153.10 & 0.28 & 215290.00 & 2109761.61 & 1.00 & 0.98 & 0.98 \\
18672 & 102501 & 1998 & 345.20 & 0.12 & 34505.00 & 327086.17 & 1.00 & 0.95 & 0.95 \\
33802 & 106172 & 1998 & 56.90 & 0.18 & 5665.00 & 56480.01 & 1.00 & 0.99 & 1.00 \\
48178 & 240040 & 1998 & 305.20 & -0.16 & 40055.00 & 340310.35 & 0.76 & 1.12 & 0.85 \\
33789 & 106170 & 1998 & 145.00 & -0.04 & 14500.00 & 138695.43 & 1.00 & 0.96 & 0.96 \\
4451 & 100633 & 1998 & 797.20 & 0.26 & 78854.00 & 787432.97 & 1.01 & 0.99 & 1.00 \\
7199 & 101013 & 1998 & 29106.10 & 0.24 & 2748381.00 & 24047181.92 & 1.06 & 0.83 & 0.87 \\
13953 & 101789 & 1998 & 878.30 & 0.28 & 89312.00 & 862374.81 & 0.98 & 0.98 & 0.97 \\
4436 & 100625 & 1998 & 1135.90 & 0.19 & 113453.00 & 1083241.19 & 1.00 & 0.95 & 0.95 \\
18653 & 102500 & 1998 & 494.20 & 0.14 & 50713.00 & 477897.45 & 0.97 & 0.97 & 0.94 \\
4476 & 100634 & 1998 & 1753.00 & 0.30 & 175597.00 & 1731073.55 & 1.00 & 0.99 & 0.99 \\
18638 & 102493 & 1998 & 845.30 & 0.28 & 86204.00 & 815648.82 & 0.98 & 0.96 & 0.95 \\
9828 & 101194 & 1998 & 151.50 & 0.11 & 13576.00 & 128914.89 & 1.12 & 0.85 & 0.95 \\
4487 & 100635 & 1998 & 164.20 & -0.12 & 16210.00 & 154890.24 & 1.01 & 0.94 & 0.96 \\
18604 & 102491 & 1998 & 138.60 & -0.04 & 14528.00 & 126411.68 & 0.95 & 0.91 & 0.87 \\
18588 & 102490 & 1998 & 98.60 & 0.27 & 7880.00 & 97282.86 & 1.25 & 0.99 & 1.23 \\
33762 & 106169 & 1998 & 6.20 & -0.10 & 615.00 & 5935.80 & 1.01 & 0.96 & 0.97 \\
13976 & 101794 & 1998 & 352.80 & 0.27 & 28412.00 & 318133.85 & 1.24 & 0.90 & 1.12 \\
31761 & 105935 & 1998 & 684.20 & 0.26 & 57444.00 & 664391.56 & 1.19 & 0.97 & 1.16 \\
4225 & 100590 & 1998 & 105.00 & 0.43 & 10524.00 & 96601.49 & 1.00 & 0.92 & 0.92 \\
52100 & 301571 & 1998 & 115.70 & -0.03 & 11633.00 & 107892.75 & 0.99 & 0.93 & 0.93 \\
13934 & 101788 & 1998 & 809.30 & 0.28 & 81561.00 & 798829.64 & 0.99 & 0.99 & 0.98 \\
18816 & 102523 & 1998 & 607.00 & 0.04 & 62477.00 & 587479.59 & 0.97 & 0.97 & 0.94 \\
33871 & 106179 & 1998 & 68.10 & 0.30 & 6806.00 & 66600.63 & 1.00 & 0.98 & 0.98 \\
9858 & 101198 & 1998 & 222.70 & 0.01 & 17455.00 & 194668.80 & 1.28 & 0.87 & 1.12 \\
4354 & 100611 & 1998 & 1149.70 & 0.22 & 114329.00 & 990411.55 & 1.01 & 0.86 & 0.87 \\
48210 & 240051 & 1998 & 459.90 & 0.21 & 46112.00 & 387366.58 & 1.00 & 0.84 & 0.84 \\
31696 & 105931 & 1998 & 52.20 & 0.07 & 3282.00 & 33875.45 & 1.59 & 0.65 & 1.03 \\
4380 & 100614 & 1998 & 778.70 & 0.14 & 77032.00 & 734431.49 & 1.01 & 0.94 & 0.95 \\
52087 & 301560 & 1998 & 575.70 & -0.04 & 51902.00 & 556185.42 & 1.11 & 0.97 & 1.07 \\
18682 & 102502 & 1998 & 471.20 & -0.03 & 47051.00 & 424426.23 & 1.00 & 0.90 & 0.90 \\
4398 & 100622 & 1998 & 801.00 & -0.06 & 80227.00 & 768506.31 & 1.00 & 0.96 & 0.96 \\
13915 & 101787 & 1998 & 783.80 & -0.06 & 81230.00 & 751719.41 & 0.96 & 0.96 & 0.93 \\
18773 & 102508 & 1998 & 380.30 & 0.34 & 33613.00 & 377146.78 & 1.13 & 0.99 & 1.12 \\
18755 & 102507 & 1998 & 539.20 & 0.37 & 49815.00 & 549740.09 & 1.08 & 1.02 & 1.10 \\
18720 & 102504 & 1998 & 40.90 & 0.11 & 3237.00 & 33207.06 & 1.26 & 0.81 & 1.03 \\
33829 & 106173 & 1998 & 236.80 & -0.02 & 20016.00 & 214803.30 & 1.18 & 0.91 & 1.07 \\
18693 & 102503 & 1998 & 302.80 & 0.01 & 38392.00 & 294611.36 & 0.79 & 0.97 & 0.77 \\
31723 & 105932 & 1998 & 188.70 & 0.22 & 14144.00 & 164252.38 & 1.33 & 0.87 & 1.16 \\
33856 & 106176 & 1998 & 11.10 & 0.33 & 805.00 & 10662.87 & 1.38 & 0.96 & 1.32 \\
18576 & 102489 & 1998 & 342.60 & 0.24 & 26923.00 & 343776.33 & 1.27 & 1.00 & 1.28 \\
13802 & 101764 & 1998 & 642.50 & 0.14 & 64512.00 & 607432.64 & 1.00 & 0.95 & 0.94 \\
19732 & 102650 & 1998 & 13495.00 & 0.09 & 1353479.00 & 12735009.70 & 1.00 & 0.94 & 0.94 \\
34108 & 106208 & 1998 & 49.40 & 0.03 & 5072.00 & 42893.20 & 0.97 & 0.87 & 0.85 \\
4013 & 100538 & 1998 & 1110.80 & -0.11 & 103197.00 & 1019997.56 & 1.08 & 0.92 & 0.99 \\
48345 & 240065 & 1998 & 575.60 & 0.15 & 53070.00 & 525479.03 & 1.08 & 0.91 & 0.99 \\
19698 & 102649 & 1998 & 1337.70 & 0.15 & 108769.00 & 986715.15 & 1.23 & 0.74 & 0.91 \\
34092 & 106207 & 1998 & 6.80 & -0.16 & 657.00 & 6299.70 & 1.04 & 0.93 & 0.96 \\
19668 & 102645 & 1998 & 393.30 & 0.14 & 41549.00 & 385686.02 & 0.95 & 0.98 & 0.93 \\
64 & 100004 & 1998 & 1550.20 & 0.24 & 125066.00 & 1471536.90 & 1.24 & 0.95 & 1.18 \\
13701 & 101758 & 1998 & 301.70 & -0.10 & 50345.00 & 485024.92 & 0.60 & 1.61 & 0.96 \\
19655 & 102641 & 1998 & 555.70 & 0.36 & 49890.00 & 543924.87 & 1.11 & 0.98 & 1.09 \\
19634 & 102639 & 1998 & 126.80 & -0.15 & 13468.00 & 105746.43 & 0.94 & 0.83 & 0.79 \\
31494 & 105900 & 1998 & 13.00 & 0.30 & 1170.00 & 11906.00 & 1.11 & 0.92 & 1.02 \\
19602 & 102636 & 1998 & 683.30 & 0.05 & 74900.00 & 712032.32 & 0.91 & 1.04 & 0.95 \\
4047 & 100543 & 1998 & 703.40 & 0.08 & 74077.00 & 704550.94 & 0.95 & 1.00 & 0.95 \\
19763 & 102651 & 1998 & 3347.50 & 0.28 & 337333.00 & 3324148.43 & 0.99 & 0.99 & 0.99 \\
31464 & 105895 & 1998 & 69.10 & 0.16 & 7171.00 & 65381.99 & 0.96 & 0.95 & 0.91 \\
19797 & 102652 & 1998 & 3563.20 & 0.10 & 356645.00 & 3272361.86 & 1.00 & 0.92 & 0.92 \\
48372 & 240067 & 1998 & 243.50 & 0.07 & 19954.00 & 217729.31 & 1.22 & 0.89 & 1.09 \\
48403 & 240076 & 1998 & 110.10 & 0.23 & 11010.00 & 104559.37 & 1.00 & 0.95 & 0.95 \\
34135 & 106209 & 1998 & 50.20 & -0.14 & 5548.00 & 52198.73 & 0.90 & 1.04 & 0.94 \\
3853 & 100505 & 1998 & 1.70 & -0.09 & 219.00 & 1736.47 & 0.78 & 1.02 & 0.79 \\
3860 & 100506 & 1998 & 10.60 & -0.09 & 1248.00 & 9747.41 & 0.85 & 0.92 & 0.78 \\
31428 & 105883 & 1998 & 36.40 & 0.28 & 3647.00 & 32649.05 & 1.00 & 0.90 & 0.90 \\
3868 & 100507 & 1998 & 19.90 & 0.07 & 2603.00 & 24510.58 & 0.76 & 1.23 & 0.94 \\
31437 & 105886 & 1998 & 55.40 & 0.24 & 5422.00 & 53849.30 & 1.02 & 0.97 & 0.99 \\
3876 & 100508 & 1998 & 9.20 & 0.15 & 984.00 & 8998.26 & 0.93 & 0.98 & 0.91 \\
19592 & 102635 & 1998 & 893.90 & 0.18 & 61664.00 & 624066.53 & 1.45 & 0.70 & 1.01 \\
19920 & 102655 & 1998 & 998.60 & -0.10 & 100182.00 & 913199.49 & 1.00 & 0.91 & 0.91 \\
9956 & 101215 & 1998 & 40.70 & 0.24 & 4836.00 & 40471.31 & 0.84 & 0.99 & 0.84 \\
3916 & 100514 & 1998 & 79.60 & 0.23 & 7498.00 & 69605.03 & 1.06 & 0.87 & 0.93 \\
19876 & 102654 & 1998 & 2064.40 & 0.18 & 207391.00 & 2056194.05 & 1.00 & 1.00 & 0.99 \\
48383 & 240074 & 1998 & 5.50 & -0.17 & 498.00 & 4774.68 & 1.10 & 0.87 & 0.96 \\
9082 & 101111 & 1998 & 1178.50 & 0.12 & 102037.00 & 1236481.40 & 1.15 & 1.05 & 1.21 \\
19838 & 102653 & 1998 & 5041.20 & 0.27 & 504968.00 & 4204610.74 & 1.00 & 0.83 & 0.83 \\
31447 & 105890 & 1998 & 97.40 & 0.08 & 9748.00 & 94953.19 & 1.00 & 0.97 & 0.97 \\
3970 & 100535 & 1998 & 415.30 & 0.07 & 46058.00 & 395516.80 & 0.90 & 0.95 & 0.86 \\
7346 & 101023 & 1998 & 23707.40 & 0.13 & 2427444.00 & 21956575.93 & 0.98 & 0.93 & 0.90 \\
19574 & 102633 & 1998 & 749.80 & -0.07 & 66638.00 & 560173.40 & 1.13 & 0.75 & 0.84 \\
19331 & 102597 & 1998 & 59.40 & 0.49 & 4237.00 & 51600.68 & 1.40 & 0.87 & 1.22 \\
13772 & 101763 & 1998 & 50.00 & -0.09 & 3975.00 & 39140.85 & 1.26 & 0.78 & 0.98 \\
31559 & 105909 & 1998 & 80.30 & -0.01 & 5825.00 & 67927.46 & 1.38 & 0.85 & 1.17 \\
33996 & 106197 & 1998 & 44.50 & 0.01 & 4488.00 & 40510.74 & 0.99 & 0.91 & 0.90 \\
4174 & 100567 & 1998 & 2905.80 & 0.29 & 251028.00 & 2878171.05 & 1.16 & 0.99 & 1.15 \\
19300 & 102588 & 1998 & 887.90 & -0.02 & 121667.00 & 1116639.68 & 0.73 & 1.26 & 0.92 \\
19280 & 102579 & 1998 & 1043.80 & 0.03 & 90755.00 & 947961.67 & 1.15 & 0.91 & 1.04 \\
4142 & 100561 & 1998 & 85.60 & -0.27 & 11478.00 & 72497.27 & 0.75 & 0.85 & 0.63 \\
19268 & 102578 & 1998 & 161.00 & 0.39 & 13239.00 & 150455.08 & 1.22 & 0.93 & 1.14 \\
19252 & 102575 & 1998 & 211.70 & 0.06 & 16020.00 & 183690.77 & 1.32 & 0.87 & 1.15 \\
7312 & 101020 & 1998 & 2602.30 & 0.17 & 236288.00 & 2459795.00 & 1.10 & 0.95 & 1.04 \\
9903 & 101211 & 1998 & 388.80 & 0.25 & 38535.00 & 337182.49 & 1.01 & 0.87 & 0.88 \\
33969 & 106195 & 1998 & 27.40 & 0.05 & 2741.00 & 22870.45 & 1.00 & 0.83 & 0.83 \\
19220 & 102570 & 1998 & 307.00 & 0.30 & 29099.00 & 293640.05 & 1.06 & 0.96 & 1.01 \\
33953 & 106193 & 1998 & 33.00 & 0.15 & 3185.00 & 32494.71 & 1.04 & 0.98 & 1.02 \\
52566 & 303130 & 1998 & 15.00 & 0.03 & 1508.00 & 14869.61 & 0.99 & 0.99 & 0.99 \\
34023 & 106198 & 1998 & 327.30 & -0.14 & 29897.00 & 326019.49 & 1.09 & 1.00 & 1.09 \\
52034 & 301299 & 1998 & 6534.20 & 0.49 & 651217.00 & 6079043.92 & 1.00 & 0.93 & 0.93 \\
13729 & 101759 & 1998 & 90.30 & 0.43 & 8575.00 & 86814.34 & 1.05 & 0.96 & 1.01 \\
19553 & 102624 & 1998 & 1812.30 & 0.24 & 156839.00 & 1777841.60 & 1.16 & 0.98 & 1.13 \\
19507 & 102608 & 1998 & 136.10 & -0.20 & 13745.00 & 135636.95 & 0.99 & 1.00 & 0.99 \\
19365 & 102599 & 1998 & 1939.80 & -0.05 & 194376.00 & 1825147.58 & 1.00 & 0.94 & 0.94 \\
34050 & 106199 & 1998 & 94.30 & -0.17 & 8263.00 & 74752.03 & 1.14 & 0.79 & 0.90 \\
19467 & 102606 & 1998 & 6272.00 & -0.09 & 627208.00 & 5968473.54 & 1.00 & 0.95 & 0.95 \\
31527 & 105905 & 1998 & 26.30 & -0.09 & 2361.00 & 26120.75 & 1.11 & 0.99 & 1.11 \\
13746 & 101762 & 1998 & 4631.80 & 0.01 & 368214.00 & 4555250.76 & 1.26 & 0.98 & 1.24 \\
9921 & 101212 & 1998 & 1413.30 & 0.85 & 140774.00 & 1235705.83 & 1.00 & 0.87 & 0.88 \\
4109 & 100552 & 1998 & 423.80 & -0.03 & 34444.00 & 431331.29 & 1.23 & 1.02 & 1.25 \\
19433 & 102601 & 1998 & 5714.00 & 0.28 & 571392.00 & 5612539.95 & 1.00 & 0.98 & 0.98 \\
19399 & 102600 & 1998 & 978.40 & 0.17 & 98166.00 & 955207.74 & 1.00 & 0.98 & 0.97 \\
4128 & 100559 & 1998 & 66.10 & -0.00 & 7627.00 & 68053.76 & 0.87 & 1.03 & 0.89 \\
19490 & 102607 & 1998 & 905.10 & 0.09 & 90509.00 & 903562.66 & 1.00 & 1.00 & 1.00 \\
18569 & 102486 & 1998 & 38.50 & 0.12 & 3499.00 & 30783.75 & 1.10 & 0.80 & 0.88 \\
31789 & 105936 & 1998 & 117.30 & 0.25 & 9459.00 & 117857.58 & 1.24 & 1.00 & 1.25 \\
32000 & 105973 & 1998 & 12.30 & 0.02 & 1076.00 & 12542.18 & 1.14 & 1.02 & 1.17 \\
45441 & 200060 & 1998 & 373.70 & 0.43 & 29468.00 & 302850.27 & 1.27 & 0.81 & 1.03 \\
45467 & 200061 & 1998 & 96.80 & 0.04 & 9718.00 & 92578.26 & 1.00 & 0.96 & 0.95 \\
45473 & 200065 & 1998 & 3.40 & 0.11 & 329.00 & 3346.61 & 1.03 & 0.98 & 1.02 \\
17618 & 102321 & 1998 & 367.20 & 0.02 & 36719.00 & 356224.84 & 1.00 & 0.97 & 0.97 \\
9724 & 101179 & 1998 & 835.70 & 0.22 & 70884.00 & 836433.21 & 1.18 & 1.00 & 1.18 \\
14226 & 101834 & 1998 & 186.50 & 0.09 & 18631.00 & 183017.38 & 1.00 & 0.98 & 0.98 \\
4996 & 100698 & 1998 & 46.40 & -0.25 & 4668.00 & 41934.21 & 0.99 & 0.90 & 0.90 \\
45504 & 200071 & 1998 & 329.20 & 0.11 & 31001.00 & 310081.75 & 1.06 & 0.94 & 1.00 \\
47999 & 225696 & 1998 & 5.30 & -0.01 & 537.00 & 4955.43 & 0.99 & 0.93 & 0.92 \\
17581 & 102319 & 1998 & 1109.50 & 0.09 & 110950.00 & 1028460.44 & 1.00 & 0.93 & 0.93 \\
7044 & 100992 & 1998 & 716.30 & 0.16 & 71625.00 & 710894.55 & 1.00 & 0.99 & 0.99 \\
33622 & 106157 & 1998 & 34.80 & 0.27 & 3453.00 & 33538.42 & 1.01 & 0.96 & 0.97 \\
14241 & 101835 & 1998 & 1309.40 & 0.28 & 130881.00 & 1242724.48 & 1.00 & 0.95 & 0.95 \\
52536 & 303121 & 1998 & 248.20 & 0.07 & 23832.00 & 234944.50 & 1.04 & 0.95 & 0.99 \\
5015 & 100700 & 1998 & 231.70 & -0.09 & 22828.00 & 220906.97 & 1.01 & 0.95 & 0.97 \\
33642 & 106158 & 1998 & 34.40 & 0.35 & 4386.00 & 40030.45 & 0.78 & 1.16 & 0.91 \\
48008 & 226438 & 1998 & 296.00 & 0.36 & 20644.00 & 256260.51 & 1.43 & 0.87 & 1.24 \\
58070 & 410075 & 1998 & 126.60 & -0.49 & 12725.00 & 118594.79 & 0.99 & 0.94 & 0.93 \\
17654 & 102334 & 1998 & 267.90 & 0.26 & 24745.00 & 259272.72 & 1.08 & 0.97 & 1.05 \\
31970 & 105965 & 1998 & 7.90 & -0.06 & 703.00 & 6716.94 & 1.12 & 0.85 & 0.96 \\
45337 & 200047 & 1998 & 12.20 & 0.15 & 1194.00 & 9764.34 & 1.02 & 0.80 & 0.82 \\
48038 & 227155 & 1998 & 22.60 & 0.09 & 2305.00 & 22088.62 & 0.98 & 0.98 & 0.96 \\
9147 & 101115 & 1998 & 14844.90 & 0.28 & 1131351.00 & 14370060.25 & 1.31 & 0.97 & 1.27 \\
45363 & 200050 & 1998 & 34.30 & -0.14 & 3949.00 & 37391.53 & 0.87 & 1.09 & 0.95 \\
45373 & 200051 & 1998 & 4.00 & 0.27 & 260.00 & 3271.79 & 1.54 & 0.82 & 1.26 \\
17750 & 102356 & 1998 & 2.30 & 0.15 & 231.00 & 2023.50 & 1.00 & 0.88 & 0.88 \\
4937 & 100695 & 1998 & 155.80 & 0.17 & 12546.00 & 153700.95 & 1.24 & 0.99 & 1.23 \\
33666 & 106160 & 1998 & 24.30 & 0.07 & 2413.00 & 23682.22 & 1.01 & 0.97 & 0.98 \\
17706 & 102349 & 1998 & 594.70 & 0.33 & 59448.00 & 573584.67 & 1.00 & 0.96 & 0.96 \\
4963 & 100697 & 1998 & 52.20 & -0.02 & 5177.00 & 51417.46 & 1.01 & 0.99 & 0.99 \\
45415 & 200058 & 1998 & 1204.50 & 0.51 & 84147.00 & 1020844.91 & 1.43 & 0.85 & 1.21 \\
17675 & 102342 & 1998 & 209.00 & 0.33 & 15370.00 & 192962.52 & 1.36 & 0.92 & 1.26 \\
17729 & 102350 & 1998 & 164.40 & 0.35 & 16614.00 & 164059.62 & 0.99 & 1.00 & 0.99 \\
45542 & 200074 & 1998 & 17.20 & 0.09 & 1774.00 & 17182.34 & 0.97 & 1.00 & 0.97 \\
17547 & 102318 & 1998 & 5823.50 & 0.13 & 582347.00 & 5112152.19 & 1.00 & 0.88 & 0.88 \\
32027 & 105974 & 1998 & 44.50 & 0.30 & 3611.00 & 43455.23 & 1.23 & 0.98 & 1.20 \\
5112 & 100724 & 1998 & 81.50 & 0.21 & 6394.00 & 63944.05 & 1.27 & 0.78 & 1.00 \\
32058 & 105978 & 1998 & 122.70 & 0.63 & 11266.00 & 112703.36 & 1.09 & 0.92 & 1.00 \\
5132 & 100726 & 1998 & 6061.60 & 0.31 & 515443.00 & 5243029.12 & 1.18 & 0.86 & 1.02 \\
17382 & 102284 & 1998 & 271.30 & -0.05 & 27254.00 & 269874.20 & 1.00 & 0.99 & 0.99 \\
32067 & 105979 & 1998 & 52.00 & -0.10 & 5000.00 & 46604.60 & 1.04 & 0.90 & 0.93 \\
32070 & 105980 & 1998 & 142.00 & -0.03 & 14214.00 & 136170.51 & 1.00 & 0.96 & 0.96 \\
5155 & 100727 & 1998 & 548.60 & 0.17 & 61615.00 & 587554.67 & 0.89 & 1.07 & 0.95 \\
47918 & 222809 & 1998 & 129.20 & 0.08 & 12705.00 & 121057.76 & 1.02 & 0.94 & 0.95 \\
7015 & 100985 & 1998 & 245.80 & 0.19 & 24498.00 & 216618.66 & 1.00 & 0.88 & 0.88 \\
14327 & 101850 & 1998 & 238.80 & 0.21 & 23881.00 & 235896.21 & 1.00 & 0.99 & 0.99 \\
17308 & 102280 & 1998 & 4211.20 & 0.06 & 446720.00 & 4136670.21 & 0.94 & 0.98 & 0.93 \\
33583 & 106151 & 1998 & 972.90 & -0.02 & 100828.00 & 972683.36 & 0.96 & 1.00 & 0.96 \\
57965 & 410010 & 1998 & 171.00 & 0.21 & 15835.00 & 161743.78 & 1.08 & 0.95 & 1.02 \\
5173 & 100730 & 1998 & 729.30 & 0.08 & 61218.00 & 648008.58 & 1.19 & 0.89 & 1.06 \\
4907 & 100692 & 1998 & 2135.30 & 0.06 & 213522.00 & 1945434.24 & 1.00 & 0.91 & 0.91 \\
47945 & 225413 & 1998 & 23.20 & 0.16 & 2041.00 & 23936.55 & 1.14 & 1.03 & 1.17 \\
17435 & 102306 & 1998 & 12456.90 & 0.14 & 1177238.00 & 11096239.41 & 1.06 & 0.89 & 0.94 \\
5025 & 100701 & 1998 & 88.20 & -0.43 & 9457.00 & 94433.36 & 0.93 & 1.07 & 1.00 \\
47986 & 225687 & 1998 & 396.30 & 0.26 & 31765.00 & 399127.59 & 1.25 & 1.01 & 1.26 \\
5049 & 100710 & 1998 & 428.20 & 0.35 & 35842.00 & 410944.96 & 1.19 & 0.96 & 1.15 \\
33610 & 106156 & 1998 & 6.40 & 0.45 & 426.00 & 5578.14 & 1.50 & 0.87 & 1.31 \\
14274 & 101842 & 1998 & 2817.60 & 0.27 & 282083.00 & 2821494.87 & 1.00 & 1.00 & 1.00 \\
58042 & 410060 & 1998 & 20.70 & -0.03 & 1919.00 & 20491.66 & 1.08 & 0.99 & 1.07 \\
17485 & 102313 & 1998 & 372.70 & 0.12 & 40783.00 & 404621.36 & 0.91 & 1.09 & 0.99 \\
33606 & 106155 & 1998 & 35.70 & 0.07 & 3935.00 & 36819.96 & 0.91 & 1.03 & 0.94 \\
9703 & 101167 & 1998 & 306.60 & -0.06 & 26059.00 & 314337.96 & 1.18 & 1.03 & 1.21 \\
47973 & 225484 & 1998 & 105.50 & 0.11 & 10549.00 & 95959.58 & 1.00 & 0.91 & 0.91 \\
5081 & 100723 & 1998 & 40.90 & 0.18 & 3811.00 & 38177.88 & 1.07 & 0.93 & 1.00 \\
17496 & 102314 & 1998 & 312.10 & 0.03 & 51038.00 & 506762.04 & 0.61 & 1.62 & 0.99 \\
18562 & 102483 & 1998 & 112.30 & -0.08 & 14121.00 & 123898.77 & 0.80 & 1.10 & 0.88 \\
52542 & 303123 & 1998 & 312.10 & 0.06 & 31135.00 & 296923.81 & 1.00 & 0.95 & 0.95 \\
17810 & 102364 & 1998 & 1508.70 & 0.28 & 150989.00 & 1502858.95 & 1.00 & 1.00 & 1.00 \\
31823 & 105945 & 1998 & 16.00 & 0.04 & 1270.00 & 15230.92 & 1.26 & 0.95 & 1.20 \\
4594 & 100642 & 1998 & 1413.00 & 0.19 & 124859.00 & 1389187.98 & 1.13 & 0.98 & 1.11 \\
18353 & 102446 & 1998 & 16.40 & -0.02 & 1547.00 & 15474.77 & 1.06 & 0.94 & 1.00 \\
31829 & 105946 & 1998 & 130.00 & 0.16 & 14390.00 & 119444.91 & 0.90 & 0.92 & 0.83 \\
4640 & 100659 & 1998 & 364.00 & 0.22 & 36385.00 & 332744.63 & 1.00 & 0.91 & 0.91 \\
48102 & 240010 & 1998 & 350.00 & -0.20 & 34410.00 & 335707.91 & 1.02 & 0.96 & 0.98 \\
9797 & 101193 & 1998 & 374.20 & 0.23 & 30065.00 & 296396.54 & 1.24 & 0.79 & 0.99 \\
4674 & 100660 & 1998 & 673.10 & 0.14 & 89339.00 & 853821.87 & 0.75 & 1.27 & 0.96 \\
7144 & 100998 & 1998 & 135.00 & -0.05 & 13429.00 & 123288.65 & 1.01 & 0.91 & 0.92 \\
31840 & 105947 & 1998 & 98.90 & 0.02 & 12561.00 & 101686.49 & 0.79 & 1.03 & 0.81 \\
52552 & 303124 & 1998 & 13.00 & 0.18 & 1265.00 & 12369.99 & 1.03 & 0.95 & 0.98 \\
4701 & 100667 & 1998 & 16.20 & 0.02 & 1620.00 & 14359.02 & 1.00 & 0.89 & 0.89 \\
31851 & 105948 & 1998 & 128.80 & 0.14 & 11413.00 & 105336.49 & 1.13 & 0.82 & 0.92 \\
18225 & 102417 & 1998 & 1502.40 & -0.02 & 129406.00 & 1412877.70 & 1.16 & 0.94 & 1.09 \\
33750 & 106167 & 1998 & 92.60 & -0.13 & 9262.00 & 86331.59 & 1.00 & 0.93 & 0.93 \\
18397 & 102447 & 1998 & 1152.60 & -0.03 & 115367.00 & 1039318.04 & 1.00 & 0.90 & 0.90 \\
31812 & 105943 & 1998 & 7.40 & 0.20 & 1084.00 & 10563.84 & 0.68 & 1.43 & 0.97 \\
18427 & 102452 & 1998 & 142.40 & 0.35 & 12980.00 & 147343.44 & 1.10 & 1.03 & 1.14 \\
18552 & 102482 & 1998 & 532.50 & 0.02 & 47221.00 & 454976.54 & 1.13 & 0.85 & 0.96 \\
31797 & 105938 & 1998 & 28.40 & 0.54 & 2844.00 & 27342.20 & 1.00 & 0.96 & 0.96 \\
4526 & 100637 & 1998 & 969.40 & 0.16 & 96738.00 & 947432.57 & 1.00 & 0.98 & 0.98 \\
18524 & 102470 & 1998 & 1475.30 & 0.17 & 122824.00 & 1431125.10 & 1.20 & 0.97 & 1.17 \\
52128 & 302060 & 1998 & 73.80 & 0.13 & 5436.00 & 70006.74 & 1.36 & 0.95 & 1.29 \\
18511 & 102469 & 1998 & 308.40 & 0.09 & 30961.00 & 306244.92 & 1.00 & 0.99 & 0.99 \\
48139 & 240027 & 1998 & 71.60 & -0.12 & 7943.00 & 67721.28 & 0.90 & 0.95 & 0.85 \\
4560 & 100639 & 1998 & 973.90 & 0.15 & 96535.00 & 899150.48 & 1.01 & 0.92 & 0.93 \\
18492 & 102465 & 1998 & 426.80 & 0.21 & 42563.00 & 422652.49 & 1.00 & 0.99 & 0.99 \\
18470 & 102462 & 1998 & 9.50 & -0.17 & 940.00 & 9405.33 & 1.01 & 0.99 & 1.00 \\
7163 & 101000 & 1998 & 1534.20 & 0.17 & 153199.00 & 1503818.12 & 1.00 & 0.98 & 0.98 \\
31808 & 105942 & 1998 & 1.10 & 0.09 & 118.00 & 1178.27 & 0.93 & 1.07 & 1.00 \\
18453 & 102461 & 1998 & 4022.40 & 0.00 & 332006.00 & 3952421.58 & 1.21 & 0.98 & 1.19 \\
14020 & 101800 & 1998 & 996.80 & 0.25 & 82855.00 & 924422.32 & 1.20 & 0.93 & 1.12 \\
31804 & 105941 & 1998 & 15.30 & 0.33 & 1115.00 & 15069.36 & 1.37 & 0.98 & 1.35 \\
18178 & 102414 & 1998 & 2388.50 & 0.24 & 172069.00 & 1939312.25 & 1.39 & 0.81 & 1.13 \\
52148 & 302206 & 1998 & 119.70 & -0.02 & 11382.00 & 104521.76 & 1.05 & 0.87 & 0.92 \\
31862 & 105949 & 1998 & 281.10 & 0.06 & 28146.00 & 261517.24 & 1.00 & 0.93 & 0.93 \\
31914 & 105960 & 1998 & 59.60 & 0.29 & 4829.00 & 55932.21 & 1.23 & 0.94 & 1.16 \\
17882 & 102371 & 1998 & 562.70 & 0.19 & 56270.00 & 521742.88 & 1.00 & 0.93 & 0.93 \\
31924 & 105961 & 1998 & 17.70 & 0.30 & 1303.00 & 15674.93 & 1.36 & 0.89 & 1.20 \\
14130 & 101805 & 1998 & 924.50 & -0.17 & 79073.00 & 898941.72 & 1.17 & 0.97 & 1.14 \\
17869 & 102367 & 1998 & 810.70 & -0.08 & 80890.00 & 799095.59 & 1.00 & 0.99 & 0.99 \\
33699 & 106163 & 1998 & 116.50 & 0.04 & 11640.00 & 116490.64 & 1.00 & 1.00 & 1.00 \\
45274 & 200011 & 1998 & 68.10 & 0.12 & 6332.00 & 65861.14 & 1.08 & 0.97 & 1.04 \\
45293 & 200015 & 1998 & 6.00 & 0.04 & 442.00 & 3866.72 & 1.36 & 0.64 & 0.87 \\
17841 & 102365 & 1998 & 990.80 & 0.40 & 98346.00 & 963015.35 & 1.01 & 0.97 & 0.98 \\
7081 & 100996 & 1998 & 2184.90 & 0.28 & 219543.00 & 2142248.90 & 1.00 & 0.98 & 0.98 \\
33696 & 106162 & 1998 & 10.40 & 0.08 & 1051.00 & 10509.26 & 0.99 & 1.01 & 1.00 \\
33693 & 106161 & 1998 & 25.30 & 0.09 & 2528.00 & 25290.32 & 1.00 & 1.00 & 1.00 \\
9749 & 101186 & 1998 & 502.70 & 0.18 & 41124.00 & 432236.28 & 1.22 & 0.86 & 1.05 \\
45332 & 200039 & 1998 & 83.10 & 0.33 & 8306.00 & 70376.38 & 1.00 & 0.85 & 0.85 \\
4887 & 100691 & 1998 & 787.30 & 0.25 & 78728.00 & 734752.28 & 1.00 & 0.93 & 0.93 \\
31962 & 105964 & 1998 & 99.90 & 0.33 & 7840.00 & 82886.49 & 1.27 & 0.83 & 1.06 \\
9767 & 101192 & 1998 & 46.00 & 0.12 & 3280.00 & 43550.64 & 1.40 & 0.95 & 1.33 \\
17966 & 102377 & 1998 & 93.60 & -0.03 & 11756.00 & 104045.17 & 0.80 & 1.11 & 0.89 \\
4760 & 100671 & 1998 & 784.30 & 0.08 & 71418.00 & 571480.17 & 1.10 & 0.73 & 0.80 \\
18128 & 102404 & 1998 & 1382.10 & 0.28 & 138662.00 & 1362508.32 & 1.00 & 0.99 & 0.98 \\
18119 & 102399 & 1998 & 11.00 & 0.03 & 1087.00 & 11542.16 & 1.01 & 1.05 & 1.06 \\
33742 & 106165 & 1998 & 19.30 & 0.14 & 1924.00 & 18379.15 & 1.00 & 0.95 & 0.96 \\
48067 & 235413 & 1998 & 45.60 & -0.06 & 3756.00 & 42137.66 & 1.21 & 0.92 & 1.12 \\
18110 & 102398 & 1998 & 13.70 & 0.04 & 1356.00 & 11667.58 & 1.01 & 0.85 & 0.86 \\
33726 & 106164 & 1998 & 16.30 & 0.25 & 1595.00 & 15948.52 & 1.02 & 0.98 & 1.00 \\
4806 & 100682 & 1998 & 93.50 & 0.14 & 8164.00 & 77921.85 & 1.15 & 0.83 & 0.95 \\
17932 & 102376 & 1998 & 21.50 & 0.19 & 1784.00 & 17966.61 & 1.21 & 0.84 & 1.01 \\
18079 & 102396 & 1998 & 311.80 & 0.02 & 19987.00 & 160897.14 & 1.56 & 0.52 & 0.81 \\
18041 & 102387 & 1998 & 54.80 & -0.16 & 5330.00 & 53315.86 & 1.03 & 0.97 & 1.00 \\
18006 & 102386 & 1998 & 162.30 & 0.04 & 16232.00 & 161797.23 & 1.00 & 1.00 & 1.00 \\
31894 & 105954 & 1998 & 3.00 & -0.08 & 326.00 & 2946.89 & 0.92 & 0.98 & 0.90 \\
31899 & 105957 & 1998 & 13.60 & 0.11 & 1128.00 & 11677.81 & 1.21 & 0.86 & 1.04 \\
4839 & 100685 & 1998 & 14.00 & -0.11 & 1430.00 & 13152.43 & 0.98 & 0.94 & 0.92 \\
17992 & 102383 & 1998 & 9.90 & 0.09 & 931.00 & 9787.06 & 1.06 & 0.99 & 1.05 \\
7112 & 100997 & 1998 & 95.60 & 0.18 & 9460.00 & 85397.66 & 1.01 & 0.89 & 0.90 \\
31909 & 105959 & 1998 & 7.90 & -0.03 & 727.00 & 7146.82 & 1.09 & 0.90 & 0.98 \\
52159 & 302545 & 1998 & 28.50 & -0.05 & 2122.00 & 22914.15 & 1.34 & 0.80 & 1.08 \\
24031 & 103255 & 1998 & 158.80 & 0.03 & 18016.00 & 154989.23 & 0.88 & 0.98 & 0.86 \\
25072 & 103429 & 1998 & 1674.00 & 0.60 & 124596.00 & 1452978.08 & 1.34 & 0.87 & 1.17 \\
35276 & 106344 & 1998 & 140.30 & 0.00 & 11280.00 & 121951.38 & 1.24 & 0.87 & 1.08 \\
460 & 100068 & 1998 & 113.50 & 0.17 & 8554.00 & 86022.21 & 1.33 & 0.76 & 1.01 \\
12063 & 101494 & 1998 & 689.10 & 0.26 & 68939.00 & 684920.31 & 1.00 & 0.99 & 0.99 \\
36876 & 106620 & 1998 & 50.00 & 0.07 & 4077.00 & 44044.56 & 1.23 & 0.88 & 1.08 \\
30016 & 105678 & 1998 & 15.00 & -0.03 & 1501.00 & 13753.73 & 1.00 & 0.92 & 0.92 \\
25552 & 103496 & 1998 & 608.50 & 0.40 & 43920.00 & 561325.58 & 1.39 & 0.92 & 1.28 \\
29397 & 105592 & 1998 & 58.00 & 0.10 & 5742.00 & 56057.77 & 1.01 & 0.97 & 0.98 \\
11269 & 101381 & 1998 & 61.30 & 0.15 & 6041.00 & 60426.75 & 1.01 & 0.99 & 1.00 \\
28136 & 105384 & 1998 & 106.50 & 0.21 & 8817.00 & 101045.34 & 1.21 & 0.95 & 1.15 \\
30007 & 105677 & 1998 & 22.40 & 0.16 & 1921.00 & 22213.35 & 1.17 & 0.99 & 1.16 \\
25583 & 103497 & 1998 & 21.80 & 0.05 & 1541.00 & 19489.77 & 1.41 & 0.89 & 1.26 \\
28122 & 105383 & 1998 & 127.20 & 0.30 & 12712.00 & 111192.52 & 1.00 & 0.87 & 0.87 \\
29404 & 105593 & 1998 & 15.10 & 0.04 & 1517.00 & 13474.33 & 1.00 & 0.89 & 0.89 \\
36932 & 106642 & 1998 & 54.20 & -0.06 & 5408.00 & 48586.59 & 1.00 & 0.90 & 0.90 \\
36950 & 106643 & 1998 & 27.40 & 0.23 & 2156.00 & 28148.05 & 1.27 & 1.03 & 1.31 \\
1183 & 100159 & 1998 & 375.00 & -0.08 & 36085.00 & 358681.34 & 1.04 & 0.96 & 0.99 \\
11282 & 101390 & 1998 & 4351.20 & 0.24 & 365880.00 & 4266438.88 & 1.19 & 0.98 & 1.17 \\
29987 & 105665 & 1998 & 117.50 & 0.30 & 11016.00 & 115974.65 & 1.07 & 0.99 & 1.05 \\
28093 & 105382 & 1998 & 62.60 & 0.29 & 4975.00 & 56986.50 & 1.26 & 0.91 & 1.15 \\
1252 & 100167 & 1998 & 508.30 & 0.27 & 41236.00 & 505134.28 & 1.23 & 0.99 & 1.22 \\
28168 & 105390 & 1998 & 436.20 & -0.14 & 41136.00 & 407286.28 & 1.06 & 0.93 & 0.99 \\
11258 & 101380 & 1998 & 310.80 & -0.02 & 30373.00 & 305147.37 & 1.02 & 0.98 & 1.00 \\
25495 & 103494 & 1998 & 418.40 & 0.20 & 30513.00 & 382700.59 & 1.37 & 0.91 & 1.25 \\
35272 & 106341 & 1998 & 61.40 & 0.15 & 4987.00 & 54651.05 & 1.23 & 0.89 & 1.10 \\
25408 & 103483 & 1998 & 665.00 & 0.29 & 54953.00 & 669021.00 & 1.21 & 1.01 & 1.22 \\
28206 & 105393 & 1998 & 21.10 & 0.02 & 2126.00 & 20025.30 & 0.99 & 0.95 & 0.94 \\
12100 & 101503 & 1998 & 261.10 & 0.06 & 26094.00 & 258136.33 & 1.00 & 0.99 & 0.99 \\
27244 & 105259 & 1998 & 124.30 & 0.32 & 12565.00 & 123964.46 & 0.99 & 1.00 & 0.99 \\
35858 & 106419 & 1998 & 1894.50 & 0.03 & 197107.00 & 1929343.16 & 0.96 & 1.02 & 0.98 \\
25432 & 103487 & 1998 & 41.30 & -0.16 & 4521.00 & 39876.71 & 0.91 & 0.97 & 0.88 \\
28074 & 105379 & 1998 & 132.10 & 0.11 & 9992.00 & 114912.93 & 1.32 & 0.87 & 1.15 \\
11247 & 101379 & 1998 & 747.00 & 0.28 & 74023.00 & 734342.35 & 1.01 & 0.98 & 0.99 \\
36866 & 106609 & 1998 & 2.80 & 0.05 & 364.00 & 2995.43 & 0.77 & 1.07 & 0.82 \\
30044 & 105679 & 1998 & 64.20 & 0.54 & 4347.00 & 57604.84 & 1.48 & 0.90 & 1.33 \\
36868 & 106610 & 1998 & 26.10 & 0.03 & 2354.00 & 19481.22 & 1.11 & 0.75 & 0.83 \\
12080 & 101497 & 1998 & 2741.80 & 0.15 & 275464.00 & 2399334.11 & 1.00 & 0.88 & 0.87 \\
35844 & 106418 & 1998 & 542.10 & 0.13 & 33917.00 & 329516.17 & 1.60 & 0.61 & 0.97 \\
1270 & 100171 & 1998 & 1940.60 & 0.06 & 172103.00 & 1924337.03 & 1.13 & 0.99 & 1.12 \\
53458 & 350572 & 1998 & 74.30 & 0.30 & 6459.00 & 77865.25 & 1.15 & 1.05 & 1.21 \\
35318 & 106347 & 1998 & 176.90 & 0.84 & 17938.00 & 167930.28 & 0.99 & 0.95 & 0.94 \\
30072 & 105680 & 1998 & 73.80 & -0.32 & 7382.00 & 71149.41 & 1.00 & 0.96 & 0.96 \\
1151 & 100157 & 1998 & 944.60 & 0.28 & 94304.00 & 922011.77 & 1.00 & 0.98 & 0.98 \\
1101 & 100153 & 1998 & 240.00 & 0.10 & 22431.00 & 206078.51 & 1.07 & 0.86 & 0.92 \\
29941 & 105659 & 1998 & 171.30 & 0.17 & 16871.00 & 169023.25 & 1.02 & 0.99 & 1.00 \\
25825 & 103524 & 1998 & 74228.50 & 0.29 & 5980460.00 & 71724139.42 & 1.24 & 0.97 & 1.20 \\
27981 & 105364 & 1998 & 111.20 & 0.29 & 8468.00 & 106730.70 & 1.31 & 0.96 & 1.26 \\
35784 & 106402 & 1998 & 72.20 & -0.04 & 6230.00 & 68191.67 & 1.16 & 0.94 & 1.09 \\
25858 & 103525 & 1998 & 27288.60 & 0.29 & 2729052.00 & 27197417.63 & 1.00 & 1.00 & 1.00 \\
27975 & 105362 & 1998 & 60.70 & 0.12 & 6040.00 & 59046.88 & 1.00 & 0.97 & 0.98 \\
38983 & 107470 & 1998 & 4.20 & 0.09 & 416.00 & 4001.08 & 1.01 & 0.95 & 0.96 \\
8757 & 101097 & 1998 & 810.60 & 0.41 & 71769.00 & 733519.26 & 1.13 & 0.90 & 1.02 \\
25892 & 103526 & 1998 & 3555.20 & 0.22 & 362354.00 & 3385141.16 & 0.98 & 0.95 & 0.93 \\
27944 & 105354 & 1998 & 230.70 & 0.08 & 18640.00 & 222723.44 & 1.24 & 0.97 & 1.19 \\
1071 & 100150 & 1998 & 22.30 & 0.14 & 2028.00 & 19489.16 & 1.10 & 0.87 & 0.96 \\
35758 & 106401 & 1998 & 470.00 & 0.02 & 40312.00 & 484215.54 & 1.17 & 1.03 & 1.20 \\
53097 & 338393 & 1998 & 23.60 & 0.18 & 1883.00 & 23311.86 & 1.25 & 0.99 & 1.24 \\
37120 & 106682 & 1998 & 14.40 & 0.11 & 1354.00 & 13533.69 & 1.06 & 0.94 & 1.00 \\
25921 & 103529 & 1998 & 5160.50 & 0.24 & 411113.00 & 4754984.72 & 1.26 & 0.92 & 1.16 \\
11954 & 101473 & 1998 & 3385.10 & 0.21 & 319683.00 & 3196889.10 & 1.06 & 0.94 & 1.00 \\
27930 & 105353 & 1998 & 125.10 & 0.10 & 12514.00 & 121179.27 & 1.00 & 0.97 & 0.97 \\
74715 & 601156 & 1998 & 39.50 & -0.06 & 3907.00 & 36352.90 & 1.01 & 0.92 & 0.93 \\
25959 & 103531 & 1998 & 819.90 & -0.21 & 92719.00 & 793746.70 & 0.88 & 0.97 & 0.86 \\
27924 & 105352 & 1998 & 269.40 & 0.18 & 26480.00 & 264812.01 & 1.02 & 0.98 & 1.00 \\
37070 & 106665 & 1998 & 687.70 & 0.02 & 81106.00 & 764422.57 & 0.85 & 1.11 & 0.94 \\
37066 & 106664 & 1998 & 25.40 & 0.19 & 2104.00 & 25795.57 & 1.21 & 1.02 & 1.23 \\
29437 & 105595 & 1998 & 29.70 & 0.22 & 2658.00 & 25885.81 & 1.12 & 0.87 & 0.97 \\
228 & 100019 & 1998 & 3585.00 & 0.06 & 358498.00 & 2994966.07 & 1.00 & 0.84 & 0.84 \\
29411 & 105594 & 1998 & 31.00 & -0.04 & 3093.00 & 29544.96 & 1.00 & 0.95 & 0.96 \\
35331 & 106348 & 1998 & 206.20 & 0.80 & 20925.00 & 183316.31 & 0.99 & 0.89 & 0.88 \\
36976 & 106644 & 1998 & 1.40 & 0.16 & 105.00 & 1170.89 & 1.33 & 0.84 & 1.12 \\
37013 & 106650 & 1998 & 5.40 & 0.05 & 524.00 & 5225.90 & 1.03 & 0.97 & 1.00 \\
25693 & 103514 & 1998 & 2938.50 & 0.17 & 238297.00 & 2465614.68 & 1.23 & 0.84 & 1.03 \\
28042 & 105370 & 1998 & 167.40 & 0.17 & 14012.00 & 155829.95 & 1.19 & 0.93 & 1.11 \\
39143 & 107611 & 1998 & 534.50 & 0.01 & 44529.00 & 513205.67 & 1.20 & 0.96 & 1.15 \\
12014 & 101477 & 1998 & 129.20 & 0.04 & 16297.00 & 149438.75 & 0.79 & 1.16 & 0.92 \\
8207 & 101079 & 1998 & 122.50 & -0.15 & 10990.00 & 109210.52 & 1.11 & 0.89 & 0.99 \\
35828 & 106415 & 1998 & 21.80 & 0.01 & 2116.00 & 19149.71 & 1.03 & 0.88 & 0.90 \\
28071 & 105372 & 1998 & 29.60 & -0.05 & 2974.00 & 29006.36 & 1.00 & 0.98 & 0.98 \\
486 & 100071 & 1998 & 6014.40 & 0.27 & 448476.00 & 4692529.87 & 1.34 & 0.78 & 1.05 \\
37028 & 106652 & 1998 & 87.20 & -0.09 & 6444.00 & 71957.12 & 1.35 & 0.83 & 1.12 \\
25725 & 103520 & 1998 & 6868.50 & 0.42 & 456539.00 & 5622994.97 & 1.50 & 0.82 & 1.23 \\
37058 & 106655 & 1998 & 43.20 & 0.28 & 2843.00 & 35825.95 & 1.52 & 0.83 & 1.26 \\
29953 & 105662 & 1998 & 48.60 & 0.14 & 4831.00 & 48306.25 & 1.01 & 0.99 & 1.00 \\
25757 & 103521 & 1998 & 2420.60 & 0.15 & 206590.00 & 2408800.52 & 1.17 & 1.00 & 1.17 \\
28013 & 105369 & 1998 & 488.60 & 0.29 & 43091.00 & 459495.45 & 1.13 & 0.94 & 1.07 \\
8492 & 101088 & 1998 & 2722.80 & 0.51 & 152527.00 & 2535842.50 & 1.79 & 0.93 & 1.66 \\
35802 & 106413 & 1998 & 24.50 & 0.33 & 2459.00 & 23374.10 & 1.00 & 0.95 & 0.95 \\
11314 & 101393 & 1998 & 508.80 & 0.23 & 61970.00 & 599203.58 & 0.82 & 1.18 & 0.97 \\
25791 & 103523 & 1998 & 3394.30 & 0.08 & 281298.00 & 3207289.93 & 1.21 & 0.94 & 1.14 \\
1136 & 100155 & 1998 & 1464.50 & 0.31 & 108665.00 & 1308308.44 & 1.35 & 0.89 & 1.20 \\
1347 & 100190 & 1998 & 3058.70 & 0.23 & 285993.00 & 2624749.29 & 1.07 & 0.86 & 0.92 \\
28225 & 105394 & 1998 & 46.90 & -0.00 & 5775.00 & 47694.81 & 0.81 & 1.02 & 0.83 \\
36616 & 106569 & 1998 & 6.40 & 0.54 & 644.00 & 6120.47 & 0.99 & 0.96 & 0.95 \\
35245 & 106336 & 1998 & 74.90 & -0.03 & 7425.00 & 72312.45 & 1.01 & 0.97 & 0.97 \\
8677 & 101094 & 1998 & 656.20 & 0.46 & 50095.00 & 606741.49 & 1.31 & 0.92 & 1.21 \\
11156 & 101369 & 1998 & 1864.00 & 0.30 & 139471.00 & 1771937.48 & 1.34 & 0.95 & 1.27 \\
25034 & 103426 & 1998 & 956.50 & 0.39 & 72773.00 & 739846.38 & 1.31 & 0.77 & 1.02 \\
36625 & 106571 & 1998 & 5.70 & 0.30 & 410.00 & 4104.95 & 1.39 & 0.72 & 1.00 \\
28428 & 105424 & 1998 & 1890.40 & 0.39 & 169755.00 & 1729690.96 & 1.11 & 0.91 & 1.02 \\
8096 & 101074 & 1998 & 43.70 & -0.12 & 4285.00 & 41616.43 & 1.02 & 0.95 & 0.97 \\
1565 & 100214 & 1998 & 166.30 & 0.17 & 15108.00 & 144088.73 & 1.10 & 0.87 & 0.95 \\
30144 & 105703 & 1998 & 175.60 & -0.07 & 23375.00 & 158232.24 & 0.75 & 0.90 & 0.68 \\
74773 & 601168 & 1998 & 39.30 & 0.17 & 3935.00 & 38503.52 & 1.00 & 0.98 & 0.98 \\
15043 & 101953 & 1998 & 460.30 & 0.16 & 68094.00 & 644019.35 & 0.68 & 1.40 & 0.95 \\
36651 & 106573 & 1998 & 9.90 & 0.01 & 895.00 & 9441.23 & 1.11 & 0.95 & 1.05 \\
36677 & 106574 & 1998 & 15.80 & -0.02 & 1320.00 & 14261.24 & 1.20 & 0.90 & 1.08 \\
30136 & 105702 & 1998 & 80.60 & 0.12 & 9094.00 & 89369.53 & 0.89 & 1.11 & 0.98 \\
25110 & 103432 & 1998 & 1750.00 & 0.31 & 143358.00 & 1609210.78 & 1.22 & 0.92 & 1.12 \\
28399 & 105421 & 1998 & 31.80 & -0.23 & 2386.00 & 26623.62 & 1.33 & 0.84 & 1.12 \\
25135 & 103436 & 1998 & 16.60 & 0.25 & 1665.00 & 15297.71 & 1.00 & 0.92 & 0.92 \\
36684 & 106576 & 1998 & 5.70 & -0.03 & 493.00 & 5332.30 & 1.16 & 0.94 & 1.08 \\
12188 & 101518 & 1998 & 870.70 & 0.14 & 87070.00 & 845925.91 & 1.00 & 0.97 & 0.97 \\
28457 & 105426 & 1998 & 511.80 & -0.03 & 45264.00 & 457256.75 & 1.13 & 0.89 & 1.01 \\
10828 & 101334 & 1998 & 164.80 & 0.03 & 16291.00 & 137767.47 & 1.01 & 0.84 & 0.85 \\
49289 & 240266 & 1998 & 26.40 & 0.59 & 3083.00 & 27391.18 & 0.86 & 1.04 & 0.89 \\
35183 & 106333 & 1998 & 104.40 & -0.06 & 9767.00 & 106597.05 & 1.07 & 1.02 & 1.09 \\
29301 & 105585 & 1998 & 9.30 & 0.10 & 918.00 & 9043.34 & 1.01 & 0.97 & 0.99 \\
39661 & 107833 & 1998 & 746.60 & 0.41 & 64981.00 & 754081.28 & 1.15 & 1.01 & 1.16 \\
1682 & 100223 & 1998 & 3475.10 & 0.16 & 347117.00 & 3362423.16 & 1.00 & 0.97 & 0.97 \\
35210 & 106334 & 1998 & 281.60 & -0.02 & 24519.00 & 231904.87 & 1.15 & 0.82 & 0.95 \\
28515 & 105432 & 1998 & 8.50 & 0.14 & 692.00 & 8454.87 & 1.23 & 0.99 & 1.22 \\
12243 & 101530 & 1998 & 1545.00 & -0.11 & 158276.00 & 1459588.54 & 0.98 & 0.94 & 0.92 \\
36552 & 106561 & 1998 & 1.90 & 0.00 & 186.00 & 1862.08 & 1.02 & 0.98 & 1.00 \\
24939 & 103395 & 1998 & 186.70 & 0.08 & 17420.00 & 155793.01 & 1.07 & 0.83 & 0.89 \\
36573 & 106563 & 1998 & 30.90 & 0.17 & 3092.00 & 30541.33 & 1.00 & 0.99 & 0.99 \\
1534 & 100213 & 1998 & 230.40 & 0.05 & 23479.00 & 229516.79 & 0.98 & 1.00 & 0.98 \\
28486 & 105427 & 1998 & 47.60 & 0.02 & 4691.00 & 43870.71 & 1.01 & 0.92 & 0.94 \\
35911 & 106427 & 1998 & 5.20 & -0.28 & 520.00 & 5075.17 & 1.00 & 0.98 & 0.98 \\
30172 & 105704 & 1998 & 52.60 & 0.02 & 4737.00 & 47747.41 & 1.11 & 0.91 & 1.01 \\
12228 & 101528 & 1998 & 73.70 & 0.05 & 7427.00 & 70354.77 & 0.99 & 0.95 & 0.95 \\
39634 & 107832 & 1998 & 132.80 & 0.16 & 13523.00 & 123243.71 & 0.98 & 0.93 & 0.91 \\
36583 & 106567 & 1998 & 1.40 & 0.15 & 108.00 & 1324.59 & 1.30 & 0.95 & 1.23 \\
39624 & 107830 & 1998 & 1004.00 & 0.12 & 100891.00 & 915684.98 & 1.00 & 0.91 & 0.91 \\
36590 & 106568 & 1998 & 3.40 & 0.20 & 284.00 & 3148.15 & 1.20 & 0.93 & 1.11 \\
30176 & 105705 & 1998 & 157.90 & 0.52 & 15659.00 & 138198.83 & 1.01 & 0.88 & 0.88 \\
25145 & 103439 & 1998 & 61.50 & 0.01 & 5098.00 & 60639.31 & 1.21 & 0.99 & 1.19 \\
11206 & 101375 & 1998 & 14.00 & -0.03 & 1387.00 & 13886.79 & 1.01 & 0.99 & 1.00 \\
35871 & 106421 & 1998 & 7.00 & -0.02 & 406.00 & 3287.67 & 1.72 & 0.47 & 0.81 \\
36790 & 106595 & 1998 & 3.60 & -0.10 & 332.00 & 3189.79 & 1.08 & 0.89 & 0.96 \\
1417 & 100196 & 1998 & 1430.80 & 0.18 & 141524.00 & 1234887.72 & 1.01 & 0.86 & 0.87 \\
25310 & 103466 & 1998 & 1054.60 & 0.06 & 107502.00 & 1078676.91 & 0.98 & 1.02 & 1.00 \\
28275 & 105400 & 1998 & 11.30 & 0.19 & 1106.00 & 10973.83 & 1.02 & 0.97 & 0.99 \\
35863 & 106420 & 1998 & 35.20 & -0.07 & 3246.00 & 35103.75 & 1.08 & 1.00 & 1.08 \\
11217 & 101376 & 1998 & 464.20 & 0.12 & 45987.00 & 460464.67 & 1.01 & 0.99 & 1.00 \\
25279 & 103464 & 1998 & 1546.70 & 0.32 & 140402.00 & 1617501.48 & 1.10 & 1.05 & 1.15 \\
30089 & 105684 & 1998 & 22.20 & -0.07 & 2157.00 & 21559.53 & 1.03 & 0.97 & 1.00 \\
36821 & 106602 & 1998 & 6.30 & 0.79 & 665.00 & 5925.26 & 0.95 & 0.94 & 0.89 \\
25351 & 103478 & 1998 & 696.10 & 0.10 & 48260.00 & 584064.69 & 1.44 & 0.84 & 1.21 \\
36829 & 106604 & 1998 & 9.20 & 0.06 & 890.00 & 7781.63 & 1.03 & 0.85 & 0.87 \\
12123 & 101511 & 1998 & 104.80 & 0.11 & 10500.00 & 98339.01 & 1.00 & 0.94 & 0.94 \\
30082 & 105682 & 1998 & 151.10 & 0.26 & 14993.00 & 142920.05 & 1.01 & 0.95 & 0.95 \\
28246 & 105399 & 1998 & 32.70 & -0.13 & 3247.00 & 31612.16 & 1.01 & 0.97 & 0.97 \\
1366 & 100192 & 1998 & 87.90 & 0.29 & 7842.00 & 87072.97 & 1.12 & 0.99 & 1.11 \\
28233 & 105397 & 1998 & 90.30 & -0.17 & 8703.00 & 86999.40 & 1.04 & 0.96 & 1.00 \\
29359 & 105589 & 1998 & 36.10 & 0.07 & 3683.00 & 34813.86 & 0.98 & 0.96 & 0.95 \\
50086 & 240392 & 1998 & 135.90 & 0.11 & 13592.00 & 131411.42 & 1.00 & 0.97 & 0.97 \\
25378 & 103481 & 1998 & 46.90 & -0.12 & 3562.00 & 41849.03 & 1.32 & 0.89 & 1.17 \\
65083 & 500660 & 1998 & 37.80 & -0.33 & 4343.00 & 32699.79 & 0.87 & 0.87 & 0.75 \\
29459 & 105597 & 1998 & 25.80 & 0.09 & 2594.00 & 22591.45 & 0.99 & 0.88 & 0.87 \\
30098 & 105686 & 1998 & 105.60 & 0.07 & 9757.00 & 109687.07 & 1.08 & 1.04 & 1.12 \\
65060 & 500659 & 1998 & 14.60 & -0.39 & 1462.00 & 12208.50 & 1.00 & 0.84 & 0.84 \\
28373 & 105420 & 1998 & 106.80 & 0.23 & 6978.00 & 94232.06 & 1.53 & 0.88 & 1.35 \\
36725 & 106581 & 1998 & 35.60 & -0.12 & 3467.00 & 30782.19 & 1.03 & 0.86 & 0.89 \\
39515 & 107716 & 1998 & 178.80 & 0.41 & 18029.00 & 173679.28 & 0.99 & 0.97 & 0.96 \\
30123 & 105701 & 1998 & 482.10 & 0.33 & 48287.00 & 480885.53 & 1.00 & 1.00 & 1.00 \\
1516 & 100209 & 1998 & 5247.00 & 0.34 & 353290.00 & 4845420.38 & 1.49 & 0.92 & 1.37 \\
28347 & 105419 & 1998 & 56.80 & 0.01 & 4268.00 & 38876.66 & 1.33 & 0.68 & 0.91 \\
1478 & 100207 & 1998 & 3917.40 & 0.21 & 385837.00 & 3842983.18 & 1.02 & 0.98 & 1.00 \\
10521 & 101298 & 1998 & 217.10 & -0.07 & 18389.00 & 147323.05 & 1.18 & 0.68 & 0.80 \\
25197 & 103460 & 1998 & 833.30 & 0.22 & 74264.00 & 791648.43 & 1.12 & 0.95 & 1.07 \\
28335 & 105416 & 1998 & 431.40 & -0.15 & 36408.00 & 350954.57 & 1.18 & 0.81 & 0.96 \\
28304 & 105401 & 1998 & 2.00 & 0.38 & 171.00 & 1670.88 & 1.17 & 0.84 & 0.98 \\
30113 & 105700 & 1998 & 343.50 & -0.13 & 34354.00 & 329663.67 & 1.00 & 0.96 & 0.96 \\
193 & 100018 & 1998 & 276.60 & 0.16 & 22091.00 & 207146.27 & 1.25 & 0.75 & 0.94 \\
11194 & 101370 & 1998 & 13.50 & 0.20 & 1160.00 & 12786.87 & 1.16 & 0.95 & 1.10 \\
12157 & 101513 & 1998 & 178.90 & 0.07 & 17753.00 & 175444.54 & 1.01 & 0.98 & 0.99 \\
35889 & 106422 & 1998 & 277.40 & -0.03 & 22108.00 & 258442.53 & 1.25 & 0.93 & 1.17 \\
36768 & 106589 & 1998 & 104.50 & -0.07 & 10247.00 & 100918.88 & 1.02 & 0.97 & 0.98 \\
1447 & 100200 & 1998 & 123.40 & -0.12 & 14738.00 & 110791.01 & 0.84 & 0.90 & 0.75 \\
36770 & 106590 & 1998 & 45.70 & 0.15 & 3451.00 & 40661.15 & 1.32 & 0.89 & 1.18 \\
30105 & 105694 & 1998 & 37.40 & -0.22 & 3292.00 & 31724.34 & 1.14 & 0.85 & 0.96 \\
29327 & 105587 & 1998 & 7.10 & -0.01 & 605.00 & 5413.19 & 1.17 & 0.76 & 0.89 \\
28321 & 105412 & 1998 & 190.00 & 0.16 & 18865.00 & 187349.21 & 1.01 & 0.99 & 0.99 \\
39672 & 107834 & 1998 & 192.70 & 0.22 & 18957.00 & 199883.80 & 1.02 & 1.04 & 1.05 \\
27917 & 105348 & 1998 & 12.20 & 0.19 & 993.00 & 11496.44 & 1.23 & 0.94 & 1.16 \\
10669 & 101307 & 1998 & 404.30 & -0.15 & 47653.00 & 334164.81 & 0.85 & 0.83 & 0.70 \\
26925 & 103628 & 1998 & 1272.70 & 0.04 & 127103.00 & 1230224.09 & 1.00 & 0.97 & 0.97 \\
27464 & 105280 & 1998 & 71.90 & -0.06 & 7291.00 & 71047.62 & 0.99 & 0.99 & 0.97 \\
35658 & 106382 & 1998 & 67.10 & 0.28 & 7609.00 & 75350.26 & 0.88 & 1.12 & 0.99 \\
29752 & 105644 & 1998 & 205.40 & 0.10 & 20566.00 & 193720.96 & 1.00 & 0.94 & 0.94 \\
11712 & 101457 & 1998 & 740.10 & 0.27 & 74214.00 & 631429.83 & 1.00 & 0.85 & 0.85 \\
27458 & 105279 & 1998 & 69.60 & 0.32 & 6960.00 & 68530.20 & 1.00 & 0.98 & 0.98 \\
26960 & 103638 & 1998 & 73.90 & 0.27 & 7366.00 & 73700.95 & 1.00 & 1.00 & 1.00 \\
37511 & 106944 & 1998 & 6.10 & 0.11 & 608.00 & 5935.79 & 1.00 & 0.97 & 0.98 \\
26970 & 103640 & 1998 & 186.90 & 0.24 & 18743.00 & 187197.54 & 1.00 & 1.00 & 1.00 \\
741 & 100093 & 1998 & 594.00 & 0.10 & 59399.00 & 537881.74 & 1.00 & 0.91 & 0.91 \\
37518 & 106947 & 1998 & 45.80 & -0.04 & 4530.00 & 44820.85 & 1.01 & 0.98 & 0.99 \\
35517 & 106370 & 1998 & 9.60 & 0.18 & 954.00 & 8340.11 & 1.01 & 0.87 & 0.87 \\
26978 & 103642 & 1998 & 34.10 & -0.05 & 3474.00 & 33689.61 & 0.98 & 0.99 & 0.97 \\
29730 & 105643 & 1998 & 717.30 & 0.11 & 71869.00 & 666125.28 & 1.00 & 0.93 & 0.93 \\
8316 & 101082 & 1998 & 3032.10 & 0.41 & 222753.00 & 2582825.88 & 1.36 & 0.85 & 1.16 \\
74559 & 601136 & 1998 & 42.20 & 0.13 & 4219.00 & 39554.41 & 1.00 & 0.94 & 0.94 \\
26985 & 103643 & 1998 & 50.70 & 0.20 & 5201.00 & 48975.92 & 0.97 & 0.97 & 0.94 \\
10730 & 101320 & 1998 & 34.80 & -0.26 & 6317.00 & 30117.84 & 0.55 & 0.87 & 0.48 \\
37532 & 106961 & 1998 & 28.20 & 0.16 & 2826.00 & 26696.80 & 1.00 & 0.95 & 0.94 \\
35637 & 106381 & 1998 & 15.70 & -0.09 & 1560.00 & 15021.04 & 1.01 & 0.96 & 0.96 \\
594 & 100079 & 1998 & 2242.00 & 0.34 & 160710.00 & 2073740.87 & 1.40 & 0.92 & 1.29 \\
26912 & 103621 & 1998 & 130.60 & 0.28 & 13020.00 & 124916.56 & 1.00 & 0.96 & 0.96 \\
771 & 100096 & 1998 & 50.60 & 0.50 & 5065.00 & 44162.63 & 1.00 & 0.87 & 0.87 \\
27473 & 105281 & 1998 & 413.20 & -0.16 & 40416.00 & 366337.86 & 1.02 & 0.89 & 0.91 \\
27559 & 105291 & 1998 & 333.10 & 0.13 & 28546.00 & 237224.45 & 1.17 & 0.71 & 0.83 \\
801 & 100097 & 1998 & 8.40 & 0.18 & 841.00 & 7769.88 & 1.00 & 0.92 & 0.92 \\
569 & 100076 & 1998 & 751.80 & 0.26 & 51079.00 & 682316.57 & 1.47 & 0.91 & 1.34 \\
37451 & 106919 & 1998 & 60.60 & 0.14 & 6061.00 & 55563.90 & 1.00 & 0.92 & 0.92 \\
73600 & 600473 & 1998 & 96.50 & 0.41 & 8056.00 & 100513.02 & 1.20 & 1.04 & 1.25 \\
35492 & 106366 & 1998 & 1.20 & 0.14 & 120.00 & 1195.59 & 1.00 & 1.00 & 1.00 \\
26760 & 103606 & 1998 & 13.70 & 0.07 & 1371.00 & 12461.88 & 1.00 & 0.91 & 0.91 \\
27540 & 105287 & 1998 & 149.30 & 0.32 & 14463.00 & 144230.67 & 1.03 & 0.97 & 1.00 \\
27531 & 105286 & 1998 & 194.70 & 0.03 & 13271.00 & 170292.95 & 1.47 & 0.87 & 1.28 \\
26792 & 103607 & 1998 & 491.30 & 0.10 & 50539.00 & 457591.37 & 0.97 & 0.93 & 0.91 \\
11746 & 101460 & 1998 & 6126.80 & 0.02 & 631038.00 & 5082219.80 & 0.97 & 0.83 & 0.81 \\
35544 & 106372 & 1998 & 3.40 & -0.19 & 295.00 & 2894.93 & 1.15 & 0.85 & 0.98 \\
35497 & 106367 & 1998 & 80.30 & 0.06 & 7935.00 & 77926.97 & 1.01 & 0.97 & 0.98 \\
26819 & 103608 & 1998 & 72.70 & 0.18 & 5240.00 & 69328.29 & 1.39 & 0.95 & 1.32 \\
74674 & 601149 & 1998 & 72.40 & 0.04 & 8559.00 & 70527.70 & 0.85 & 0.97 & 0.82 \\
35662 & 106383 & 1998 & 101.20 & 0.29 & 10504.00 & 95454.30 & 0.96 & 0.94 & 0.91 \\
11504 & 101425 & 1998 & 46.60 & 0.29 & 4010.00 & 40131.09 & 1.16 & 0.86 & 1.00 \\
26842 & 103609 & 1998 & 98.00 & -0.06 & 9868.00 & 93263.91 & 0.99 & 0.95 & 0.95 \\
27503 & 105283 & 1998 & 8.80 & 0.09 & 818.00 & 7444.55 & 1.08 & 0.85 & 0.91 \\
29574 & 105616 & 1998 & 9.40 & 0.24 & 929.00 & 9285.14 & 1.01 & 0.99 & 1.00 \\
26858 & 103614 & 1998 & 202.10 & 0.06 & 19978.00 & 188895.31 & 1.01 & 0.93 & 0.95 \\
37476 & 106931 & 1998 & 304.40 & 0.00 & 34533.00 & 330488.19 & 0.88 & 1.09 & 0.96 \\
29762 & 105645 & 1998 & 206.50 & 0.32 & 20854.00 & 183912.13 & 0.99 & 0.89 & 0.88 \\
26890 & 103620 & 1998 & 210.20 & 0.19 & 18500.00 & 199158.48 & 1.14 & 0.95 & 1.08 \\
37485 & 106934 & 1998 & 58.10 & 0.56 & 5606.00 & 49581.49 & 1.04 & 0.85 & 0.88 \\
74587 & 601139 & 1998 & 4262.10 & 0.16 & 407870.00 & 3915613.21 & 1.04 & 0.92 & 0.96 \\
27011 & 103644 & 1998 & 221.50 & 0.00 & 22143.00 & 215434.22 & 1.00 & 0.97 & 0.97 \\
272 & 100030 & 1998 & 98.20 & 0.23 & 18061.00 & 180618.78 & 0.54 & 1.84 & 1.00 \\
629 & 100085 & 1998 & 18726.90 & 0.32 & 1403636.00 & 16961087.67 & 1.33 & 0.91 & 1.21 \\
27306 & 105267 & 1998 & 924.00 & 0.03 & 74339.00 & 755688.58 & 1.24 & 0.82 & 1.02 \\
29615 & 105627 & 1998 & 313.00 & -0.04 & 31138.00 & 311077.56 & 1.01 & 0.99 & 1.00 \\
29656 & 105631 & 1998 & 12.00 & 0.37 & 1189.00 & 11868.69 & 1.01 & 0.99 & 1.00 \\
37740 & 107136 & 1998 & 12.20 & 0.06 & 1293.00 & 11401.24 & 0.94 & 0.93 & 0.88 \\
74626 & 601142 & 1998 & 257.00 & 0.14 & 41372.00 & 397389.25 & 0.62 & 1.55 & 0.96 \\
27189 & 105250 & 1998 & 10.50 & -0.06 & 1115.00 & 8789.46 & 0.94 & 0.84 & 0.79 \\
261 & 100022 & 1998 & 69.20 & 0.00 & 6774.00 & 67754.63 & 1.02 & 0.98 & 1.00 \\
27300 & 105264 & 1998 & 1367.10 & 0.12 & 121507.00 & 1251766.69 & 1.13 & 0.92 & 1.03 \\
35610 & 106380 & 1998 & 28.50 & -0.02 & 2986.00 & 29779.20 & 0.95 & 1.04 & 1.00 \\
27208 & 105253 & 1998 & 38.80 & 0.06 & 4759.00 & 46291.02 & 0.82 & 1.19 & 0.97 \\
655 & 100087 & 1998 & 8278.80 & 0.33 & 598251.00 & 7100758.70 & 1.38 & 0.86 & 1.19 \\
27273 & 105260 & 1998 & 246.40 & -0.06 & 20848.00 & 208188.92 & 1.18 & 0.84 & 1.00 \\
29644 & 105630 & 1998 & 3.30 & 0.27 & 319.00 & 3035.52 & 1.03 & 0.92 & 0.95 \\
27217 & 105256 & 1998 & 23.50 & -0.06 & 3204.00 & 20418.07 & 0.73 & 0.87 & 0.64 \\
37716 & 107135 & 1998 & 27.70 & 0.10 & 2751.00 & 27476.00 & 1.01 & 0.99 & 1.00 \\
37709 & 107004 & 1998 & 302.00 & 0.22 & 23446.00 & 235687.50 & 1.29 & 0.78 & 1.01 \\
35600 & 106379 & 1998 & 61.10 & 0.17 & 4583.00 & 47841.16 & 1.33 & 0.78 & 1.04 \\
27311 & 105268 & 1998 & 508.70 & 0.10 & 49657.00 & 482288.40 & 1.02 & 0.95 & 0.97 \\
53634 & 355027 & 1998 & 178.50 & 0.56 & 34339.00 & 306652.18 & 0.52 & 1.72 & 0.89 \\
29666 & 105632 & 1998 & 5.40 & 0.21 & 526.00 & 5258.65 & 1.03 & 0.97 & 1.00 \\
37539 & 106962 & 1998 & 3.50 & 0.11 & 346.00 & 3169.72 & 1.01 & 0.91 & 0.92 \\
37547 & 106966 & 1998 & 4.30 & -0.00 & 427.00 & 4163.26 & 1.01 & 0.97 & 0.98 \\
27037 & 103645 & 1998 & 307.40 & 0.01 & 30743.00 & 302701.26 & 1.00 & 0.98 & 0.98 \\
27399 & 105276 & 1998 & 774.50 & 0.07 & 77902.00 & 727461.66 & 0.99 & 0.94 & 0.93 \\
35552 & 106375 & 1998 & 11.60 & 0.37 & 1163.00 & 11513.97 & 1.00 & 0.99 & 0.99 \\
27067 & 103647 & 1998 & 14.30 & 0.11 & 1920.00 & 20069.90 & 0.74 & 1.40 & 1.05 \\
27369 & 105275 & 1998 & 124.80 & 0.15 & 12598.00 & 106965.82 & 0.99 & 0.86 & 0.85 \\
707 & 100092 & 1998 & 249.20 & 0.31 & 24925.00 & 217382.18 & 1.00 & 0.87 & 0.87 \\
29701 & 105640 & 1998 & 42.20 & -0.27 & 4840.00 & 44060.18 & 0.87 & 1.04 & 0.91 \\
27095 & 103652 & 1998 & 472.30 & -0.01 & 47706.00 & 415200.93 & 0.99 & 0.88 & 0.87 \\
27358 & 105271 & 1998 & 40.60 & 0.26 & 4293.00 & 40667.49 & 0.95 & 1.00 & 0.95 \\
27114 & 103658 & 1998 & 915.90 & -0.02 & 127302.00 & 902939.13 & 0.72 & 0.99 & 0.71 \\
683 & 100090 & 1998 & 646.60 & 0.44 & 64667.00 & 569748.54 & 1.00 & 0.88 & 0.88 \\
27127 & 105243 & 1998 & 16.80 & -0.13 & 1659.00 & 16437.25 & 1.01 & 0.98 & 0.99 \\
27336 & 105269 & 1998 & 258.00 & -0.07 & 25672.00 & 254659.01 & 1.00 & 0.99 & 0.99 \\
29680 & 105635 & 1998 & 230.90 & 0.12 & 19042.00 & 241681.57 & 1.21 & 1.05 & 1.27 \\
74641 & 601146 & 1998 & 21.20 & 0.07 & 2120.00 & 20434.45 & 1.00 & 0.96 & 0.96 \\
37638 & 106983 & 1998 & 9.40 & 0.19 & 933.00 & 8601.40 & 1.01 & 0.92 & 0.92 \\
11558 & 101430 & 1998 & 151.20 & -0.07 & 9867.00 & 95811.20 & 1.53 & 0.63 & 0.97 \\
35671 & 106386 & 1998 & 544.60 & 0.32 & 59684.00 & 569156.40 & 0.91 & 1.05 & 0.95 \\
26692 & 103600 & 1998 & 538.80 & 0.33 & 54057.00 & 436582.30 & 1.00 & 0.81 & 0.81 \\
37161 & 106706 & 1998 & 7.20 & 0.07 & 715.00 & 6971.68 & 1.01 & 0.97 & 0.98 \\
37175 & 106707 & 1998 & 8.80 & 0.12 & 893.00 & 8689.25 & 0.99 & 0.99 & 0.97 \\
977 & 100113 & 1998 & 1131.20 & -0.03 & 113392.00 & 1125319.91 & 1.00 & 0.99 & 0.99 \\
27822 & 105332 & 1998 & 107.50 & 0.27 & 10784.00 & 107835.16 & 1.00 & 1.00 & 1.00 \\
35407 & 106358 & 1998 & 11.80 & -0.04 & 1188.00 & 11535.47 & 0.99 & 0.98 & 0.97 \\
26212 & 103546 & 1998 & 18187.70 & 0.01 & 1711452.00 & 18364610.16 & 1.06 & 1.01 & 1.07 \\
35733 & 106394 & 1998 & 36.50 & 0.17 & 2813.00 & 33886.37 & 1.30 & 0.93 & 1.20 \\
74683 & 601151 & 1998 & 7.70 & 0.00 & 759.00 & 6301.93 & 1.01 & 0.82 & 0.83 \\
37201 & 106708 & 1998 & 63.70 & 0.07 & 6604.00 & 60832.70 & 0.96 & 0.95 & 0.92 \\
26243 & 103547 & 1998 & 13623.60 & 0.40 & 1036265.00 & 12315838.25 & 1.31 & 0.90 & 1.19 \\
11398 & 101400 & 1998 & 363.10 & -0.13 & 36329.00 & 360859.73 & 1.00 & 0.99 & 0.99 \\
27793 & 105331 & 1998 & 2.90 & -0.05 & 288.00 & 2746.75 & 1.01 & 0.95 & 0.95 \\
35728 & 106392 & 1998 & 111.20 & 0.22 & 9487.00 & 83765.37 & 1.17 & 0.75 & 0.88 \\
522 & 100072 & 1998 & 14984.90 & 0.19 & 1198852.00 & 15299380.61 & 1.25 & 1.02 & 1.28 \\
933 & 100112 & 1998 & 2224.60 & 0.21 & 221868.00 & 2055222.25 & 1.00 & 0.92 & 0.93 \\
35414 & 106359 & 1998 & 322.30 & 0.29 & 28425.00 & 328857.22 & 1.13 & 1.02 & 1.16 \\
11873 & 101464 & 1998 & 350.10 & -0.17 & 48805.00 & 360779.06 & 0.72 & 1.03 & 0.74 \\
27787 & 105327 & 1998 & 64.70 & -0.02 & 6627.00 & 62794.76 & 0.98 & 0.97 & 0.95 \\
35422 & 106360 & 1998 & 150.60 & -0.18 & 15397.00 & 141635.62 & 0.98 & 0.94 & 0.92 \\
26290 & 103564 & 1998 & 1275.30 & 0.05 & 127382.00 & 1246309.39 & 1.00 & 0.98 & 0.98 \\
27779 & 105326 & 1998 & 88.50 & 0.31 & 9057.00 & 85461.18 & 0.98 & 0.97 & 0.94 \\
26308 & 103567 & 1998 & 2221.60 & 0.14 & 297353.00 & 2741294.07 & 0.75 & 1.23 & 0.92 \\
10766 & 101330 & 1998 & 6552.40 & 0.01 & 654415.00 & 5927049.65 & 1.00 & 0.90 & 0.91 \\
26172 & 103545 & 1998 & 31538.20 & 0.16 & 2769729.00 & 29006100.49 & 1.14 & 0.92 & 1.05 \\
53710 & 356500 & 1998 & 431.20 & 0.12 & 41979.00 & 418074.40 & 1.03 & 0.97 & 1.00 \\
27851 & 105333 & 1998 & 26.50 & 0.12 & 2641.00 & 26442.49 & 1.00 & 1.00 & 1.00 \\
11903 & 101465 & 1998 & 86.70 & 0.24 & 8667.00 & 82544.43 & 1.00 & 0.95 & 0.95 \\
8460 & 101087 & 1998 & 550.50 & 0.06 & 61826.00 & 629914.22 & 0.89 & 1.14 & 1.02 \\
29926 & 105658 & 1998 & 410.40 & 0.08 & 38647.00 & 313851.94 & 1.06 & 0.76 & 0.81 \\
11934 & 101466 & 1998 & 61.10 & -0.12 & 8852.00 & 62461.73 & 0.69 & 1.02 & 0.71 \\
26027 & 103535 & 1998 & 1478.80 & 0.28 & 113172.00 & 1414071.85 & 1.31 & 0.96 & 1.25 \\
27898 & 105343 & 1998 & 227.40 & 0.24 & 22308.00 & 219309.99 & 1.02 & 0.96 & 0.98 \\
27893 & 105342 & 1998 & 24.70 & -0.06 & 2449.00 & 22594.71 & 1.01 & 0.91 & 0.92 \\
1021 & 100127 & 1998 & 3494.30 & 0.63 & 348699.00 & 3218845.57 & 1.00 & 0.92 & 0.92 \\
35744 & 106398 & 1998 & 2.50 & 0.14 & 182.00 & 2021.92 & 1.37 & 0.81 & 1.11 \\
27769 & 105322 & 1998 & 33.20 & -0.01 & 3140.00 & 31498.74 & 1.06 & 0.95 & 1.00 \\
35347 & 106353 & 1998 & 31.50 & 0.09 & 3036.00 & 30358.48 & 1.04 & 0.96 & 1.00 \\
26057 & 103536 & 1998 & 1352.40 & 0.29 & 102109.00 & 1234571.23 & 1.32 & 0.91 & 1.21 \\
11370 & 101399 & 1998 & 167.00 & 0.15 & 16396.00 & 165651.51 & 1.02 & 0.99 & 1.01 \\
37151 & 106695 & 1998 & 59.50 & -0.12 & 6195.00 & 58174.00 & 0.96 & 0.98 & 0.94 \\
37154 & 106701 & 1998 & 87.10 & 0.15 & 7821.00 & 81080.36 & 1.11 & 0.93 & 1.04 \\
26104 & 103539 & 1998 & 951.00 & 0.22 & 81299.00 & 736090.99 & 1.17 & 0.77 & 0.91 \\
27873 & 105336 & 1998 & 41.60 & -0.04 & 3848.00 & 35015.96 & 1.08 & 0.84 & 0.91 \\
74708 & 601155 & 1998 & 12.40 & -0.09 & 1217.00 & 11633.51 & 1.02 & 0.94 & 0.96 \\
35381 & 106356 & 1998 & 20.40 & -0.05 & 1833.00 & 18436.54 & 1.11 & 0.90 & 1.01 \\
26138 & 103544 & 1998 & 11753.20 & 0.23 & 1172879.00 & 10942678.26 & 1.00 & 0.93 & 0.93 \\
27859 & 105335 & 1998 & 208.20 & 0.13 & 20802.00 & 193040.27 & 1.00 & 0.93 & 0.93 \\
37123 & 106688 & 1998 & 106.60 & 0.04 & 9160.00 & 75438.12 & 1.16 & 0.71 & 0.82 \\
29863 & 105655 & 1998 & 272.20 & -0.12 & 27005.00 & 250333.38 & 1.01 & 0.92 & 0.93 \\
37225 & 106713 & 1998 & 33.10 & 0.38 & 3322.00 & 32244.95 & 1.00 & 0.97 & 0.97 \\
543 & 100075 & 1998 & 3069.30 & 0.05 & 238480.00 & 2860237.24 & 1.29 & 0.93 & 1.20 \\
11810 & 101462 & 1998 & 1829.50 & -0.06 & 201979.00 & 1744414.72 & 0.91 & 0.95 & 0.86 \\
37310 & 106728 & 1998 & 5.80 & 0.13 & 571.00 & 5702.59 & 1.02 & 0.98 & 1.00 \\
11455 & 101414 & 1998 & 75.80 & 0.18 & 7587.00 & 75852.07 & 1.00 & 1.00 & 1.00 \\
29820 & 105652 & 1998 & 485.20 & 0.34 & 48479.00 & 479363.28 & 1.00 & 0.99 & 0.99 \\
26551 & 103591 & 1998 & 179.90 & 0.02 & 17978.00 & 162000.61 & 1.00 & 0.90 & 0.90 \\
35471 & 106363 & 1998 & 150.00 & 0.44 & 15398.00 & 140652.53 & 0.97 & 0.94 & 0.91 \\
26582 & 103592 & 1998 & 62.50 & 0.25 & 6332.00 & 53944.01 & 0.99 & 0.86 & 0.85 \\
26519 & 103590 & 1998 & 156.60 & 0.11 & 15846.00 & 144799.37 & 0.99 & 0.92 & 0.91 \\
831 & 100098 & 1998 & 7.70 & 0.39 & 773.00 & 6919.57 & 1.00 & 0.90 & 0.90 \\
26616 & 103593 & 1998 & 51773.80 & 0.19 & 5177381.00 & 48094567.86 & 1.00 & 0.93 & 0.93 \\
27604 & 105303 & 1998 & 201.10 & 0.21 & 20535.00 & 200325.24 & 0.98 & 1.00 & 0.98 \\
26648 & 103595 & 1998 & 143.50 & 0.18 & 14350.00 & 134817.14 & 1.00 & 0.94 & 0.94 \\
29545 & 105611 & 1998 & 23.30 & 0.04 & 2330.00 & 22212.35 & 1.00 & 0.95 & 0.95 \\
35486 & 106365 & 1998 & 11.20 & -0.11 & 1122.00 & 10699.92 & 1.00 & 0.96 & 0.95 \\
11779 & 101461 & 1998 & 2187.80 & 0.34 & 179630.00 & 1850972.52 & 1.22 & 0.85 & 1.03 \\
27597 & 105299 & 1998 & 4.20 & -0.13 & 446.00 & 4292.53 & 0.94 & 1.02 & 0.96 \\
37411 & 106869 & 1998 & 133.50 & 0.01 & 13320.00 & 121889.57 & 1.00 & 0.91 & 0.92 \\
8275 & 101081 & 1998 & 597.30 & 0.14 & 50994.00 & 602072.60 & 1.17 & 1.01 & 1.18 \\
27587 & 105295 & 1998 & 412.80 & 0.13 & 41266.00 & 406980.10 & 1.00 & 0.99 & 0.99 \\
29791 & 105647 & 1998 & 154.40 & 0.19 & 15603.00 & 138433.62 & 0.99 & 0.90 & 0.89 \\
35479 & 106364 & 1998 & 4.40 & -0.22 & 438.00 & 4154.52 & 1.00 & 0.94 & 0.95 \\
29533 & 105610 & 1998 & 42.30 & 0.11 & 4327.00 & 41023.19 & 0.98 & 0.97 & 0.95 \\
27688 & 105310 & 1998 & 34.90 & 0.31 & 3169.00 & 29650.70 & 1.10 & 0.85 & 0.94 \\
37244 & 106718 & 1998 & 4.20 & 0.03 & 428.00 & 3835.60 & 0.98 & 0.91 & 0.90 \\
27751 & 105321 & 1998 & 149.10 & 0.16 & 12584.00 & 138048.35 & 1.18 & 0.93 & 1.10 \\
35701 & 106391 & 1998 & 70.40 & 0.04 & 7095.00 & 67937.50 & 0.99 & 0.97 & 0.96 \\
26373 & 103572 & 1998 & 48.80 & 0.12 & 4313.00 & 46493.33 & 1.13 & 0.95 & 1.08 \\
891 & 100101 & 1998 & 349.30 & 0.45 & 34930.00 & 337966.92 & 1.00 & 0.97 & 0.97 \\
37272 & 106725 & 1998 & 11.90 & 0.14 & 1164.00 & 11630.35 & 1.02 & 0.98 & 1.00 \\
26384 & 103573 & 1998 & 379.80 & 0.13 & 37873.00 & 375727.12 & 1.00 & 0.99 & 0.99 \\
27725 & 105320 & 1998 & 103.80 & -0.11 & 10160.00 & 103643.35 & 1.02 & 1.00 & 1.02 \\
11842 & 101463 & 1998 & 1019.60 & 0.21 & 95221.00 & 964905.29 & 1.07 & 0.95 & 1.01 \\
29846 & 105654 & 1998 & 146.40 & 0.15 & 14620.00 & 137846.54 & 1.00 & 0.94 & 0.94 \\
26399 & 103579 & 1998 & 399.50 & 0.45 & 30305.00 & 306213.78 & 1.32 & 0.77 & 1.01 \\
11432 & 101402 & 1998 & 9.10 & 0.07 & 1108.00 & 9577.90 & 0.82 & 1.05 & 0.86 \\
27708 & 105317 & 1998 & 566.10 & 0.16 & 57236.00 & 543078.34 & 0.99 & 0.96 & 0.95 \\
861 & 100099 & 1998 & 38.70 & 0.25 & 3865.00 & 36274.82 & 1.00 & 0.94 & 0.94 \\
10639 & 101302 & 1998 & 525.90 & 0.06 & 52894.00 & 470844.08 & 0.99 & 0.90 & 0.89 \\
29840 & 105653 & 1998 & 6.80 & -0.03 & 597.00 & 6386.36 & 1.14 & 0.94 & 1.07 \\
35449 & 106361 & 1998 & 48.70 & 0.12 & 4678.00 & 46756.68 & 1.04 & 0.96 & 1.00 \\
26482 & 103582 & 1998 & 18.20 & 0.31 & 1332.00 & 17043.71 & 1.37 & 0.94 & 1.28 \\
74677 & 601150 & 1998 & 1.30 & -0.09 & 132.00 & 1213.49 & 0.98 & 0.93 & 0.92 \\
27696 & 105311 & 1998 & 80.30 & 0.49 & 5822.00 & 58241.27 & 1.38 & 0.73 & 1.00 \\
38396 & 107257 & 1998 & 176.40 & 0.29 & 17623.00 & 166928.52 & 1.00 & 0.95 & 0.95 \\
73361 & 600006 & 1998 & 93.20 & -0.01 & 9379.00 & 92297.43 & 0.99 & 0.99 & 0.98 \\
28530 & 105436 & 1998 & 45.50 & 0.12 & 5003.00 & 49776.54 & 0.91 & 1.09 & 0.99 \\
11598 & 101431 & 1998 & 518.10 & 0.06 & 47487.00 & 447759.70 & 1.09 & 0.86 & 0.94 \\
1981 & 100278 & 1998 & 33.90 & 0.35 & 1800.00 & 18584.16 & 1.88 & 0.55 & 1.03 \\
24471 & 103328 & 1998 & 1143.30 & 0.02 & 84881.00 & 1071269.19 & 1.35 & 0.94 & 1.26 \\
10924 & 101354 & 1998 & 1306.10 & 0.26 & 126178.00 & 1239001.54 & 1.04 & 0.95 & 0.98 \\
36446 & 106529 & 1998 & 33.70 & 0.02 & 1822.00 & 15387.16 & 1.85 & 0.46 & 0.84 \\
24571 & 103369 & 1998 & 123.70 & 0.38 & 12335.00 & 120014.05 & 1.00 & 0.97 & 0.97 \\
39770 & 107870 & 1998 & 262.30 & 0.13 & 27645.00 & 250511.53 & 0.95 & 0.96 & 0.91 \\
36296 & 106482 & 1998 & 16.10 & -0.10 & 905.00 & 8937.77 & 1.78 & 0.56 & 0.99 \\
29036 & 105522 & 1998 & 99.00 & 0.07 & 9459.00 & 86524.75 & 1.05 & 0.87 & 0.91 \\
49104 & 240222 & 1998 & 1094.30 & 0.26 & 103150.00 & 1022487.85 & 1.06 & 0.93 & 0.99 \\
30261 & 105721 & 1998 & 165.40 & 0.02 & 16181.00 & 161967.53 & 1.02 & 0.98 & 1.00 \\
29262 & 105574 & 1998 & 61.00 & 0.03 & 6299.00 & 57202.40 & 0.97 & 0.94 & 0.91 \\
29028 & 105520 & 1998 & 36.60 & 0.02 & 3685.00 & 35474.75 & 0.99 & 0.97 & 0.96 \\
39978 & 107967 & 1998 & 208.20 & 0.11 & 17593.00 & 194101.10 & 1.18 & 0.93 & 1.10 \\
28805 & 105478 & 1998 & 33.40 & 0.11 & 3336.00 & 31086.19 & 1.00 & 0.93 & 0.93 \\
24614 & 103372 & 1998 & 1160.30 & 0.10 & 120313.00 & 971344.80 & 0.96 & 0.84 & 0.81 \\
74790 & 601171 & 1998 & 625.30 & 0.28 & 51992.00 & 596251.44 & 1.20 & 0.95 & 1.15 \\
24760 & 103377 & 1998 & 1071.70 & -0.00 & 145091.00 & 1052260.10 & 0.74 & 0.98 & 0.73 \\
40258 & 108083 & 1998 & 17.00 & 0.07 & 1770.00 & 17614.77 & 0.96 & 1.04 & 1.00 \\
35114 & 106321 & 1998 & 2.60 & 0.21 & 176.00 & 2320.45 & 1.48 & 0.89 & 1.32 \\
30233 & 105720 & 1998 & 295.00 & 0.98 & 28106.00 & 281406.47 & 1.05 & 0.95 & 1.00 \\
30365 & 105742 & 1998 & 77.60 & 0.08 & 7768.00 & 71030.57 & 1.00 & 0.92 & 0.91 \\
36467 & 106535 & 1998 & 145.70 & -0.06 & 8405.00 & 76476.42 & 1.73 & 0.52 & 0.91 \\
28957 & 105508 & 1998 & 22.90 & -0.06 & 2293.00 & 22588.66 & 1.00 & 0.99 & 0.99 \\
36356 & 106498 & 1998 & 194.60 & 0.18 & 19391.00 & 177568.09 & 1.00 & 0.91 & 0.92 \\
12306 & 101534 & 1998 & 829.50 & 0.05 & 80394.00 & 778607.61 & 1.03 & 0.94 & 0.97 \\
24143 & 103290 & 1998 & 1600.10 & 0.04 & 161780.00 & 1302305.51 & 0.99 & 0.81 & 0.80 \\
96661 & 611002 & 1998 & 4345.60 & 0.28 & 368568.00 & 4495994.01 & 1.18 & 1.03 & 1.22 \\
40284 & 108084 & 1998 & 67.10 & 0.10 & 6846.00 & 67663.22 & 0.98 & 1.01 & 0.99 \\
36495 & 106541 & 1998 & 16.90 & 0.02 & 1731.00 & 14256.67 & 0.98 & 0.84 & 0.82 \\
2005 & 100280 & 1998 & 55.80 & 0.29 & 5515.00 & 56691.42 & 1.01 & 1.02 & 1.03 \\
35088 & 106320 & 1998 & 1086.50 & 0.08 & 108256.00 & 1047198.57 & 1.00 & 0.96 & 0.97 \\
1885 & 100247 & 1998 & 925.50 & 0.19 & 73288.00 & 862558.87 & 1.26 & 0.93 & 1.18 \\
12516 & 101545 & 1998 & 291.80 & -0.13 & 27518.00 & 266116.33 & 1.06 & 0.91 & 0.97 \\
24449 & 103327 & 1998 & 2012.10 & 0.13 & 170745.00 & 2048302.06 & 1.18 & 1.02 & 1.20 \\
24799 & 103380 & 1998 & 6501.70 & 0.36 & 568671.00 & 6347341.56 & 1.14 & 0.98 & 1.12 \\
388 & 100048 & 1998 & 694.30 & 0.15 & 74208.00 & 611086.49 & 0.94 & 0.88 & 0.82 \\
11086 & 101367 & 1998 & 321.30 & -0.00 & 31490.00 & 308694.52 & 1.02 & 0.96 & 0.98 \\
29273 & 105576 & 1998 & 18.30 & -0.01 & 1998.00 & 16619.78 & 0.92 & 0.91 & 0.83 \\
30393 & 105753 & 1998 & 71.70 & 0.05 & 5645.00 & 52257.60 & 1.27 & 0.73 & 0.93 \\
175 & 100017 & 1998 & 127.90 & 0.13 & 13206.00 & 113843.29 & 0.97 & 0.89 & 0.86 \\
35931 & 106441 & 1998 & 110.30 & 0.04 & 13242.00 & 132424.03 & 0.83 & 1.20 & 1.00 \\
1765 & 100228 & 1998 & 218.60 & 0.12 & 21880.00 & 194074.39 & 1.00 & 0.89 & 0.89 \\
30264 & 105722 & 1998 & 31.40 & 0.26 & 2256.00 & 28739.71 & 1.39 & 0.92 & 1.27 \\
28880 & 105502 & 1998 & 1739.60 & 0.22 & 167286.00 & 1540883.92 & 1.04 & 0.89 & 0.92 \\
39988 & 107968 & 1998 & 75.80 & 0.06 & 6451.00 & 70807.67 & 1.18 & 0.93 & 1.10 \\
40206 & 108073 & 1998 & 99.10 & 0.01 & 11201.00 & 91686.67 & 0.88 & 0.93 & 0.82 \\
1968 & 100263 & 1998 & 4.20 & 0.19 & 334.00 & 3035.18 & 1.26 & 0.72 & 0.91 \\
29146 & 105533 & 1998 & 160.20 & 0.44 & 17354.00 & 165109.95 & 0.92 & 1.03 & 0.95 \\
1928 & 100251 & 1998 & 168.20 & 0.15 & 15574.00 & 152103.96 & 1.08 & 0.90 & 0.98 \\
36147 & 106471 & 1998 & 222.20 & -0.00 & 22017.00 & 214887.33 & 1.01 & 0.97 & 0.98 \\
35141 & 106329 & 1998 & 15.70 & 0.10 & 1513.00 & 15130.20 & 1.04 & 0.96 & 1.00 \\
10460 & 101286 & 1998 & 698.60 & 0.13 & 66646.00 & 555431.85 & 1.05 & 0.80 & 0.83 \\
30385 & 105748 & 1998 & 91.40 & 0.15 & 9096.00 & 86810.00 & 1.00 & 0.95 & 0.95 \\
49072 & 240218 & 1998 & 18.20 & 0.35 & 1571.00 & 13842.66 & 1.16 & 0.76 & 0.88 \\
28686 & 105464 & 1998 & 21.60 & -0.21 & 2321.00 & 19645.13 & 0.93 & 0.91 & 0.85 \\
30292 & 105731 & 1998 & 236.00 & 0.22 & 24840.00 & 211049.39 & 0.95 & 0.89 & 0.85 \\
29180 & 105535 & 1998 & 150.60 & 0.30 & 13181.00 & 145385.92 & 1.14 & 0.97 & 1.10 \\
8027 & 101071 & 1998 & 1784.50 & 0.15 & 189755.00 & 1804231.97 & 0.94 & 1.01 & 0.95 \\
24533 & 103339 & 1998 & 658.20 & 0.12 & 65002.00 & 638483.22 & 1.01 & 0.97 & 0.98 \\
24281 & 103304 & 1998 & 27.20 & -0.11 & 2797.00 & 22646.27 & 0.97 & 0.83 & 0.81 \\
35058 & 106310 & 1998 & 13.20 & -0.01 & 944.00 & 11051.46 & 1.40 & 0.84 & 1.17 \\
288 & 100033 & 1998 & 217.50 & 0.07 & 21585.00 & 215857.36 & 1.01 & 0.99 & 1.00 \\
11022 & 101360 & 1998 & 1593.60 & 0.16 & 137408.00 & 1412471.15 & 1.16 & 0.89 & 1.03 \\
39856 & 107883 & 1998 & 3589.10 & 0.17 & 355494.00 & 3556924.15 & 1.01 & 0.99 & 1.00 \\
320 & 100036 & 1998 & 160.70 & 0.28 & 16047.00 & 150464.37 & 1.00 & 0.94 & 0.94 \\
49237 & 240254 & 1998 & 1501.70 & -0.13 & 131979.00 & 1324486.05 & 1.14 & 0.88 & 1.00 \\
35981 & 106443 & 1998 & 23.40 & 0.05 & 2345.00 & 21933.42 & 1.00 & 0.94 & 0.94 \\
8818 & 101100 & 1998 & 394.10 & -0.50 & 43058.00 & 371134.41 & 0.92 & 0.94 & 0.86 \\
24263 & 103301 & 1998 & 1107.80 & 0.24 & 118708.00 & 1212732.99 & 0.93 & 1.09 & 1.02 \\
28696 & 105465 & 1998 & 40.60 & -0.11 & 4023.00 & 39461.49 & 1.01 & 0.97 & 0.98 \\
36004 & 106445 & 1998 & 98.00 & 0.04 & 6735.00 & 54700.83 & 1.46 & 0.56 & 0.81 \\
39863 & 107892 & 1998 & 246.10 & 0.35 & 24451.00 & 233684.33 & 1.01 & 0.95 & 0.96 \\
40231 & 108082 & 1998 & 9.90 & 0.30 & 934.00 & 9267.52 & 1.06 & 0.94 & 0.99 \\
74793 & 601172 & 1998 & 15.90 & -0.20 & 1933.00 & 17174.17 & 0.82 & 1.08 & 0.89 \\
28677 & 105463 & 1998 & 613.10 & 0.79 & 39043.00 & 509439.07 & 1.57 & 0.83 & 1.30 \\
40121 & 108030 & 1998 & 19.50 & -0.23 & 1900.00 & 16992.27 & 1.03 & 0.87 & 0.89 \\
12504 & 101544 & 1998 & 112.00 & 0.36 & 10906.00 & 109045.10 & 1.03 & 0.97 & 1.00 \\
24501 & 103329 & 1998 & 808.20 & -0.09 & 74126.00 & 820620.80 & 1.09 & 1.02 & 1.11 \\
10971 & 101357 & 1998 & 348.80 & 0.16 & 31205.00 & 298782.59 & 1.12 & 0.86 & 0.96 \\
7976 & 101068 & 1998 & 83552.10 & 0.34 & 6669319.00 & 77859391.82 & 1.25 & 0.93 & 1.17 \\
24723 & 103376 & 1998 & 7100.00 & 0.37 & 601341.00 & 6817468.59 & 1.18 & 0.96 & 1.13 \\
24545 & 103347 & 1998 & 4.10 & 0.18 & 340.00 & 4080.59 & 1.21 & 1.00 & 1.20 \\
36155 & 106474 & 1998 & 37.00 & 0.30 & 3698.00 & 33644.95 & 1.00 & 0.91 & 0.91 \\
35167 & 106330 & 1998 & 76.30 & -0.13 & 6882.00 & 63701.88 & 1.11 & 0.83 & 0.93 \\
40126 & 108037 & 1998 & 3.50 & 0.18 & 287.00 & 3518.58 & 1.22 & 1.01 & 1.23 \\
12388 & 101538 & 1998 & 43.90 & 0.17 & 4463.00 & 43229.69 & 0.98 & 0.98 & 0.97 \\
36165 & 106476 & 1998 & 13.00 & -0.11 & 1265.00 & 12625.36 & 1.03 & 0.97 & 1.00 \\
12324 & 101536 & 1998 & 1309.40 & 0.37 & 132217.00 & 1273008.77 & 0.99 & 0.97 & 0.96 \\
24162 & 103294 & 1998 & 629.40 & 0.11 & 65179.00 & 618527.05 & 0.97 & 0.98 & 0.95 \\
1784 & 100237 & 1998 & 199.70 & 0.19 & 19878.00 & 198513.06 & 1.00 & 0.99 & 1.00 \\
36358 & 106500 & 1998 & 6.20 & 0.23 & 619.00 & 5912.94 & 1.00 & 0.95 & 0.96 \\
36360 & 106502 & 1998 & 37.60 & 0.09 & 3703.00 & 34583.79 & 1.02 & 0.92 & 0.93 \\
24643 & 103373 & 1998 & 86.80 & -0.22 & 8791.00 & 86446.04 & 0.99 & 1.00 & 0.98 \\
40150 & 108051 & 1998 & 435.40 & 0.09 & 36593.00 & 347186.86 & 1.19 & 0.80 & 0.95 \\
53403 & 346113 & 1998 & 400.80 & 0.32 & 31746.00 & 401718.86 & 1.26 & 1.00 & 1.27 \\
24683 & 103375 & 1998 & 920.30 & -0.13 & 109038.00 & 894345.00 & 0.84 & 0.97 & 0.82 \\
39929 & 107958 & 1998 & 7.30 & 0.03 & 546.00 & 5335.27 & 1.34 & 0.73 & 0.98 \\
28994 & 105511 & 1998 & 41.40 & -0.21 & 4134.00 & 37689.47 & 1.00 & 0.91 & 0.91 \\
35955 & 106442 & 1998 & 1423.90 & 0.29 & 91122.00 & 1265564.67 & 1.56 & 0.89 & 1.39 \\
36457 & 106530 & 1998 & 70.10 & -0.03 & 7066.00 & 70815.17 & 0.99 & 1.01 & 1.00 \\
8066 & 101073 & 1998 & 5476.40 & 0.40 & 450719.00 & 5328297.88 & 1.22 & 0.97 & 1.18 \\
30224 & 105718 & 1998 & 45.90 & 0.16 & 5386.00 & 46354.02 & 0.85 & 1.01 & 0.86 \\
24880 & 103383 & 1998 & 959.80 & -0.14 & 154608.00 & 1324464.80 & 0.62 & 1.38 & 0.86 \\
30371 & 105746 & 1998 & 186.80 & 0.23 & 18564.00 & 179672.08 & 1.01 & 0.96 & 0.97 \\
28938 & 105507 & 1998 & 991.20 & 0.28 & 72976.00 & 796637.18 & 1.36 & 0.80 & 1.09 \\
10860 & 101340 & 1998 & 12712.90 & 0.17 & 1271291.00 & 10573465.27 & 1.00 & 0.83 & 0.83 \\
36082 & 106461 & 1998 & 30.40 & 0.04 & 3063.00 & 29912.17 & 0.99 & 0.98 & 0.98 \\
39704 & 107836 & 1998 & 1.20 & 0.04 & 120.00 & 1173.37 & 1.00 & 0.98 & 0.98 \\
40042 & 108018 & 1998 & 224.30 & 0.12 & 22402.00 & 213469.91 & 1.00 & 0.95 & 0.95 \\
40340 & 108112 & 1998 & 31.50 & -0.18 & 3170.00 & 31191.23 & 0.99 & 0.99 & 0.98 \\
35073 & 106318 & 1998 & 85.40 & 0.19 & 7022.00 & 74006.38 & 1.22 & 0.87 & 1.05 \\
2033 & 100286 & 1998 & 64.90 & -0.01 & 6323.00 & 61531.56 & 1.03 & 0.95 & 0.97 \\
10982 & 101358 & 1998 & 584.90 & 0.24 & 58423.00 & 584902.39 & 1.00 & 1.00 & 1.00 \\
35921 & 106434 & 1998 & 349.30 & -0.02 & 28533.00 & 258237.13 & 1.22 & 0.74 & 0.91 \\
39706 & 107837 & 1998 & 15.70 & -0.12 & 1559.00 & 13459.28 & 1.01 & 0.86 & 0.86 \\
35134 & 106326 & 1998 & 2.20 & -0.38 & 230.00 & 1938.28 & 0.96 & 0.88 & 0.84 \\
24084 & 103264 & 1998 & 872.90 & 0.22 & 74114.00 & 860013.65 & 1.18 & 0.99 & 1.16 \\
36127 & 106467 & 1998 & 74.10 & 0.21 & 4673.00 & 48609.01 & 1.59 & 0.66 & 1.04 \\
36411 & 106524 & 1998 & 16.10 & -0.21 & 1690.00 & 16451.85 & 0.95 & 1.02 & 0.97 \\
28567 & 105444 & 1998 & 9.00 & 0.30 & 1081.00 & 11249.82 & 0.83 & 1.25 & 1.04 \\
49145 & 240234 & 1998 & 80.00 & -0.12 & 7611.00 & 71246.84 & 1.05 & 0.89 & 0.94 \\
36533 & 106560 & 1998 & 24.50 & 0.29 & 2459.00 & 23319.24 & 1.00 & 0.95 & 0.95 \\
35064 & 106317 & 1998 & 95.60 & 0.33 & 7253.00 & 88764.82 & 1.32 & 0.93 & 1.22 \\
415 & 100055 & 1998 & 9164.50 & 0.14 & 832072.00 & 8251026.98 & 1.10 & 0.90 & 0.99 \\
40034 & 108013 & 1998 & 260.00 & 0.32 & 25977.00 & 253822.63 & 1.00 & 0.98 & 0.98 \\
29126 & 105531 & 1998 & 29.50 & 0.19 & 2730.00 & 26653.70 & 1.08 & 0.90 & 0.98 \\
35913 & 106428 & 1998 & 79.40 & 0.07 & 7888.00 & 72124.87 & 1.01 & 0.91 & 0.91 \\
28863 & 105489 & 1998 & 3.50 & -0.09 & 392.00 & 3385.07 & 0.89 & 0.97 & 0.86 \\
337 & 100040 & 1998 & 1092.30 & 0.40 & 82450.00 & 977375.29 & 1.32 & 0.89 & 1.19 \\
11120 & 101368 & 1998 & 797.60 & 0.23 & 67833.00 & 753041.86 & 1.18 & 0.94 & 1.11 \\
12545 & 101553 & 1998 & 593.60 & 0.32 & 44262.00 & 547907.04 & 1.34 & 0.92 & 1.24 \\
36432 & 106527 & 1998 & 3.00 & 0.11 & 300.00 & 2992.06 & 1.00 & 1.00 & 1.00 \\
40019 & 108009 & 1998 & 3.40 & 0.18 & 271.00 & 3510.36 & 1.25 & 1.03 & 1.30 \\
24901 & 103394 & 1998 & 46.40 & -0.04 & 4264.00 & 37182.74 & 1.09 & 0.80 & 0.87 \\
24403 & 103319 & 1998 & 357.90 & 0.26 & 35769.00 & 352066.33 & 1.00 & 0.98 & 0.98 \\
8008 & 101069 & 1998 & 7149.90 & 0.28 & 602205.00 & 6534840.65 & 1.19 & 0.91 & 1.09 \\
30203 & 105708 & 1998 & 147.40 & 0.23 & 14743.00 & 136323.11 & 1.00 & 0.92 & 0.92 \\
24053 & 103259 & 1998 & 2147.60 & 0.05 & 207381.00 & 2083953.80 & 1.04 & 0.97 & 1.00 \\
28852 & 105487 & 1998 & 110.20 & 0.03 & 11448.00 & 102178.18 & 0.96 & 0.93 & 0.89 \\
39678 & 107835 & 1998 & 461.80 & 0.03 & 49888.00 & 434059.59 & 0.93 & 0.94 & 0.87 \\
35983 & 106444 & 1998 & 135.40 & 0.32 & 13489.00 & 129840.57 & 1.00 & 0.96 & 0.96 \\
24591 & 103370 & 1998 & 189.00 & 0.09 & 14839.00 & 182721.35 & 1.27 & 0.97 & 1.23 \\
28745 & 105475 & 1998 & 539.10 & 0.03 & 60742.00 & 535135.81 & 0.89 & 0.99 & 0.88 \\
24369 & 103318 & 1998 & 1460.50 & 0.08 & 145909.00 & 1359020.83 & 1.00 & 0.93 & 0.93 \\
28537 & 105437 & 1998 & 2953.50 & 0.12 & 291453.00 & 2915878.19 & 1.01 & 0.99 & 1.00 \\
36437 & 106528 & 1998 & 3.90 & 0.18 & 388.00 & 3598.27 & 1.01 & 0.92 & 0.93 \\
28560 & 105443 & 1998 & 170.10 & 0.41 & 13241.00 & 141289.06 & 1.28 & 0.83 & 1.07 \\
30414 & 105754 & 1998 & 29.90 & -0.09 & 2736.00 & 23031.03 & 1.09 & 0.77 & 0.84 \\
28723 & 105472 & 1998 & 158.00 & 0.20 & 14466.00 & 144718.06 & 1.09 & 0.92 & 1.00 \\
35180 & 106332 & 1998 & 32.50 & -0.12 & 3255.00 & 28412.90 & 1.00 & 0.87 & 0.87 \\
24112 & 103267 & 1998 & 222.20 & 0.38 & 21676.00 & 214152.50 & 1.03 & 0.96 & 0.99 \\
28870 & 105498 & 1998 & 163.30 & 0.25 & 13317.00 & 159570.67 & 1.23 & 0.98 & 1.20 \\
10476 & 101287 & 1998 & 533.30 & 0.06 & 51605.00 & 430767.73 & 1.03 & 0.81 & 0.83 \\
1745 & 100227 & 1998 & 202.70 & 0.18 & 20304.00 & 198391.10 & 1.00 & 0.98 & 0.98 \\
36525 & 106556 & 1998 & 27.00 & 0.40 & 1700.00 & 24473.12 & 1.59 & 0.91 & 1.44 \\
40180 & 108071 & 1998 & 24.40 & 0.39 & 2470.00 & 24561.40 & 0.99 & 1.01 & 0.99 \\
39755 & 107860 & 1998 & 11.60 & -0.03 & 999.00 & 10181.14 & 1.16 & 0.88 & 1.02 \\
39918 & 107938 & 1998 & 135.20 & 0.15 & 11208.00 & 119495.36 & 1.21 & 0.88 & 1.07 \\
29174 & 105534 & 1998 & 265.80 & 0.11 & 36817.00 & 261329.14 & 0.72 & 0.98 & 0.71 \\
36073 & 106458 & 1998 & 20.70 & 0.21 & 1993.00 & 19941.35 & 1.04 & 0.96 & 1.00 \\
24341 & 103315 & 1998 & 59.40 & -0.00 & 5623.00 & 56232.89 & 1.06 & 0.95 & 1.00 \\
12408 & 101539 & 1998 & 884.20 & 0.14 & 88469.00 & 843205.59 & 1.00 & 0.95 & 0.95 \\
36507 & 106545 & 1998 & 9.50 & 0.02 & 689.00 & 8854.30 & 1.38 & 0.93 & 1.29 \\
29065 & 105523 & 1998 & 156.90 & 0.18 & 12728.00 & 148464.00 & 1.23 & 0.95 & 1.17 \\
49254 & 240256 & 1998 & 127.60 & 0.26 & 11118.00 & 100168.74 & 1.15 & 0.79 & 0.90 \\
40288 & 108087 & 1998 & 15.60 & 0.52 & 1649.00 & 16487.59 & 0.95 & 1.06 & 1.00 \\
39758 & 107863 & 1998 & 681.50 & -0.09 & 60753.00 & 607497.70 & 1.12 & 0.89 & 1.00 \\
28715 & 105471 & 1998 & 38.70 & 0.07 & 3767.00 & 37682.90 & 1.03 & 0.97 & 1.00 \\
35027 & 106307 & 1998 & 11.40 & 0.25 & 1190.00 & 11889.23 & 0.96 & 1.04 & 1.00 \\
24559 & 103366 & 1998 & 196.50 & 0.22 & 18290.00 & 181602.73 & 1.07 & 0.92 & 0.99 \\
28612 & 105450 & 1998 & 13.10 & 0.08 & 1386.00 & 12118.75 & 0.95 & 0.93 & 0.87 \\
28774 & 105476 & 1998 & 64.20 & -0.16 & 6093.00 & 58323.68 & 1.05 & 0.91 & 0.96 \\
40314 & 108109 & 1998 & 21.30 & -0.05 & 1617.00 & 16267.44 & 1.32 & 0.76 & 1.01 \\
29222 & 105545 & 1998 & 893.70 & 0.25 & 78958.00 & 889739.82 & 1.13 & 1.00 & 1.13 \\
11054 & 101364 & 1998 & 87.20 & 0.13 & 8717.00 & 76517.81 & 1.00 & 0.88 & 0.88 \\
24096 & 103266 & 1998 & 345.50 & 0.20 & 26262.00 & 345055.28 & 1.32 & 1.00 & 1.31 \\
30349 & 105740 & 1998 & 776.30 & -0.13 & 75292.00 & 604310.93 & 1.03 & 0.78 & 0.80 \\
28583 & 105448 & 1998 & 60.40 & -0.23 & 5609.00 & 51611.77 & 1.08 & 0.85 & 0.92 \\
24232 & 103299 & 1998 & 523.50 & 0.02 & 70047.00 & 603904.75 & 0.75 & 1.15 & 0.86 \\
24419 & 103326 & 1998 & 2347.20 & 0.38 & 178564.00 & 2393746.00 & 1.31 & 1.02 & 1.34 \\
24839 & 103381 & 1998 & 30226.30 & 0.35 & 2607917.00 & 29787403.40 & 1.16 & 0.99 & 1.14 \\
49265 & 240261 & 1998 & 271.20 & 0.08 & 27092.00 & 253865.11 & 1.00 & 0.94 & 0.94 \\
10956 & 101356 & 1998 & 375.20 & -0.11 & 37395.00 & 373997.29 & 1.00 & 1.00 & 1.00 \\
36528 & 106557 & 1998 & 41.70 & 0.50 & 2567.00 & 37132.23 & 1.62 & 0.89 & 1.45 \\
49061 & 240212 & 1998 & 4988.80 & 0.27 & 366111.00 & 4479748.01 & 1.36 & 0.90 & 1.22 \\
30321 & 105737 & 1998 & 51.50 & 0.29 & 5159.00 & 50152.04 & 1.00 & 0.97 & 0.97 \\
40070 & 108021 & 1998 & 316.60 & 0.22 & 32700.00 & 291645.87 & 0.97 & 0.92 & 0.89 \\
12357 & 101537 & 1998 & 958.10 & 0.17 & 95613.00 & 900409.56 & 1.00 & 0.94 & 0.94 \\
12442 & 101541 & 1998 & 170.40 & 0.34 & 16540.00 & 165373.67 & 1.03 & 0.97 & 1.00 \\
40006 & 107994 & 1998 & 178.10 & 0.24 & 11008.00 & 176715.42 & 1.62 & 0.99 & 1.61 \\
17751 & 102356 & 1999 & 3.50 & 0.41 & 254.00 & 2875.22 & 1.38 & 0.82 & 1.13 \\
37367 & 106740 & 1999 & 305.20 & 0.11 & 30661.00 & 297438.89 & 1.00 & 0.97 & 0.97 \\
45379 & 200055 & 1999 & 72.70 & 0.03 & 6315.00 & 68815.56 & 1.15 & 0.95 & 1.09 \\
28923 & 105506 & 1999 & 183.20 & 0.18 & 18325.00 & 178820.60 & 1.00 & 0.98 & 0.98 \\
17655 & 102334 & 1999 & 312.10 & 0.17 & 29854.00 & 309060.48 & 1.05 & 0.99 & 1.04 \\
4938 & 100695 & 1999 & 187.60 & 0.22 & 18730.00 & 176572.52 & 1.00 & 0.94 & 0.94 \\
17707 & 102349 & 1999 & 1270.40 & 0.24 & 126526.00 & 1227527.10 & 1.00 & 0.97 & 0.97 \\
37375 & 106742 & 1999 & 102.30 & 0.14 & 10219.00 & 102204.85 & 1.00 & 1.00 & 1.00 \\
17730 & 102350 & 1999 & 389.80 & 0.55 & 38985.00 & 386104.03 & 1.00 & 0.99 & 0.99 \\
32738 & 106057 & 1999 & 243.70 & 0.84 & 24373.00 & 223975.32 & 1.00 & 0.92 & 0.92 \\
15326 & 101984 & 1999 & 337.50 & 0.23 & 33978.00 & 335079.71 & 0.99 & 0.99 & 0.99 \\
45416 & 200058 & 1999 & 3256.70 & 0.55 & 325973.00 & 3014655.26 & 1.00 & 0.93 & 0.92 \\
37312 & 106729 & 1999 & 494.60 & 0.20 & 43068.00 & 424604.11 & 1.15 & 0.86 & 0.99 \\
33643 & 106158 & 1999 & 150.70 & 0.36 & 15362.00 & 143284.36 & 0.98 & 0.95 & 0.93 \\
27663 & 105309 & 1999 & 514.20 & 0.42 & 41547.00 & 462734.46 & 1.24 & 0.90 & 1.11 \\
27634 & 105306 & 1999 & 150.60 & 0.91 & 15115.00 & 146500.36 & 1.00 & 0.97 & 0.97 \\
37373 & 106741 & 1999 & 8.50 & 0.02 & 1492.00 & 6871.96 & 0.57 & 0.81 & 0.46 \\
37337 & 106730 & 1999 & 15.70 & 0.17 & 1598.00 & 14046.15 & 0.98 & 0.89 & 0.88 \\
33667 & 106160 & 1999 & 21.20 & 0.06 & 2130.00 & 20714.90 & 1.00 & 0.98 & 0.97 \\
17676 & 102342 & 1999 & 356.50 & 0.29 & 29394.00 & 349991.91 & 1.21 & 0.98 & 1.19 \\
4964 & 100697 & 1999 & 34.90 & -0.02 & 3462.00 & 33577.16 & 1.01 & 0.96 & 0.97 \\
27605 & 105303 & 1999 & 131.10 & 0.11 & 13232.00 & 130340.74 & 0.99 & 0.99 & 0.99 \\
544 & 100075 & 1999 & 3493.80 & 0.45 & 349373.00 & 3412547.70 & 1.00 & 0.98 & 0.98 \\
36264 & 106480 & 1999 & 268.70 & 0.33 & 15927.00 & 156560.50 & 1.69 & 0.58 & 0.98 \\
57240 & 400323 & 1999 & 156.40 & 1.40 & 15615.00 & 127767.78 & 1.00 & 0.82 & 0.82 \\
58071 & 410075 & 1999 & 117.80 & 0.15 & 12068.00 & 117759.00 & 0.98 & 1.00 & 0.98 \\
15424 & 101990 & 1999 & 116.80 & 0.06 & 12442.00 & 103241.08 & 0.94 & 0.88 & 0.83 \\
5050 & 100710 & 1999 & 540.70 & 0.23 & 54059.00 & 516331.35 & 1.00 & 0.95 & 0.96 \\
36007 & 106447 & 1999 & 33.30 & 1.43 & 3398.00 & 32693.67 & 0.98 & 0.98 & 0.96 \\
37226 & 106713 & 1999 & 36.50 & 0.21 & 3651.00 & 34563.65 & 1.00 & 0.95 & 0.95 \\
32043 & 105977 & 1999 & 1306.90 & 0.27 & 130690.00 & 1227450.59 & 1.00 & 0.94 & 0.94 \\
5026 & 100701 & 1999 & 50.70 & 0.08 & 6201.00 & 62022.10 & 0.82 & 1.22 & 1.00 \\
37245 & 106718 & 1999 & 4.50 & 0.02 & 444.00 & 4122.66 & 1.01 & 0.92 & 0.93 \\
32034 & 105976 & 1999 & 13.40 & 0.31 & 1262.00 & 11770.82 & 1.06 & 0.88 & 0.93 \\
27752 & 105321 & 1999 & 270.20 & 0.17 & 25516.00 & 253644.83 & 1.06 & 0.94 & 0.99 \\
28871 & 105498 & 1999 & 211.20 & 0.32 & 20643.00 & 203624.82 & 1.02 & 0.96 & 0.99 \\
6330 & 100849 & 1999 & 107.70 & 0.42 & 9791.00 & 104981.49 & 1.10 & 0.97 & 1.07 \\
37246 & 106722 & 1999 & 2.20 & 0.02 & 219.00 & 2185.90 & 1.00 & 0.99 & 1.00 \\
37247 & 106724 & 1999 & 20.00 & -0.01 & 1619.00 & 15646.89 & 1.24 & 0.78 & 0.97 \\
17505 & 102317 & 1999 & 19.00 & -0.13 & 1464.00 & 14043.78 & 1.30 & 0.74 & 0.96 \\
27770 & 105322 & 1999 & 27.50 & -0.00 & 2695.00 & 26957.47 & 1.02 & 0.98 & 1.00 \\
17497 & 102314 & 1999 & 450.20 & 0.90 & 44628.00 & 446280.41 & 1.01 & 0.99 & 1.00 \\
32719 & 106052 & 1999 & 190.10 & 0.16 & 19079.00 & 186325.48 & 1.00 & 0.98 & 0.98 \\
17473 & 102312 & 1999 & 203.20 & 0.99 & 20589.00 & 202551.53 & 0.99 & 1.00 & 0.98 \\
5082 & 100723 & 1999 & 31.10 & 0.03 & 3106.00 & 30622.75 & 1.00 & 0.98 & 0.99 \\
36005 & 106445 & 1999 & 85.40 & -0.03 & 8542.00 & 83946.86 & 1.00 & 0.98 & 0.98 \\
37202 & 106708 & 1999 & 340.80 & 0.24 & 36311.00 & 329287.08 & 0.94 & 0.97 & 0.91 \\
37215 & 106710 & 1999 & 61.20 & 0.07 & 6977.00 & 67240.91 & 0.88 & 1.10 & 0.96 \\
27794 & 105331 & 1999 & 2.50 & 0.20 & 247.00 & 2472.04 & 1.01 & 0.99 & 1.00 \\
6301 & 100847 & 1999 & 3.00 & 0.04 & 297.00 & 2481.83 & 1.01 & 0.83 & 0.84 \\
17548 & 102318 & 1999 & 5154.00 & 0.02 & 475261.00 & 4529908.69 & 1.08 & 0.88 & 0.95 \\
15458 & 101992 & 1999 & 728.60 & 0.29 & 72860.00 & 700222.72 & 1.00 & 0.96 & 0.96 \\
32713 & 106051 & 1999 & 85.50 & 0.10 & 8501.00 & 81649.83 & 1.01 & 0.95 & 0.96 \\
17486 & 102313 & 1999 & 668.70 & 0.98 & 67878.00 & 668334.84 & 0.99 & 1.00 & 0.98 \\
33607 & 106155 & 1999 & 24.30 & -0.06 & 2426.00 & 24103.57 & 1.00 & 0.99 & 0.99 \\
58043 & 410060 & 1999 & 25.50 & 0.47 & 2560.00 & 22891.70 & 1.00 & 0.90 & 0.89 \\
35702 & 106391 & 1999 & 74.80 & 0.28 & 7446.00 & 70691.00 & 1.00 & 0.95 & 0.95 \\
523 & 100072 & 1999 & 19265.80 & 0.44 & 1926572.00 & 18314521.25 & 1.00 & 0.95 & 0.95 \\
27780 & 105326 & 1999 & 186.70 & 0.24 & 21358.00 & 203167.73 & 0.87 & 1.09 & 0.95 \\
27788 & 105327 & 1999 & 93.90 & 0.13 & 9378.00 & 88753.84 & 1.00 & 0.95 & 0.95 \\
35699 & 106388 & 1999 & 734.50 & 0.22 & 75037.00 & 716407.96 & 0.98 & 0.98 & 0.95 \\
15366 & 101988 & 1999 & 499.20 & 0.02 & 53529.00 & 455515.08 & 0.93 & 0.91 & 0.85 \\
29523 & 105606 & 1999 & 14.50 & 1.03 & 1374.00 & 13735.21 & 1.06 & 0.95 & 1.00 \\
17619 & 102321 & 1999 & 279.00 & -0.04 & 27526.00 & 275299.96 & 1.01 & 0.99 & 1.00 \\
45474 & 200065 & 1999 & 4.00 & 0.05 & 398.00 & 3791.55 & 1.01 & 0.95 & 0.95 \\
29527 & 105607 & 1999 & 9.60 & 0.39 & 795.00 & 7782.83 & 1.21 & 0.81 & 0.98 \\
45468 & 200061 & 1999 & 79.20 & 0.07 & 7919.00 & 78441.40 & 1.00 & 0.99 & 0.99 \\
45442 & 200060 & 1999 & 1048.40 & 0.53 & 103828.00 & 979154.83 & 1.01 & 0.93 & 0.94 \\
32734 & 106053 & 1999 & 8.00 & 0.02 & 797.00 & 6827.06 & 1.00 & 0.85 & 0.86 \\
32001 & 105973 & 1999 & 11.70 & 0.18 & 1166.00 & 11553.60 & 1.00 & 0.99 & 0.99 \\
29181 & 105535 & 1999 & 174.40 & 0.67 & 17466.00 & 172274.74 & 1.00 & 0.99 & 0.99 \\
29534 & 105610 & 1999 & 520.80 & 0.90 & 52339.00 & 509943.82 & 1.00 & 0.98 & 0.97 \\
6378 & 100856 & 1999 & 197.70 & 0.04 & 18186.00 & 187247.61 & 1.09 & 0.95 & 1.03 \\
15336 & 101987 & 1999 & 1774.40 & 0.68 & 177438.00 & 1650594.85 & 1.00 & 0.93 & 0.93 \\
37311 & 106728 & 1999 & 5.20 & 0.13 & 527.00 & 4993.82 & 0.99 & 0.96 & 0.95 \\
27697 & 105311 & 1999 & 238.80 & 0.69 & 14200.00 & 196923.87 & 1.68 & 0.82 & 1.39 \\
47186 & 200342 & 1999 & 4383.10 & 0.09 & 469346.00 & 4214425.90 & 0.93 & 0.96 & 0.90 \\
74678 & 601150 & 1999 & 1.80 & 0.69 & 169.00 & 1663.78 & 1.07 & 0.92 & 0.98 \\
4997 & 100698 & 1999 & 34.90 & 0.29 & 3513.00 & 34747.83 & 0.99 & 1.00 & 0.99 \\
37273 & 106725 & 1999 & 29.50 & 0.56 & 2861.00 & 28622.13 & 1.03 & 0.97 & 1.00 \\
5016 & 100700 & 1999 & 116.40 & 0.11 & 11908.00 & 113585.91 & 0.98 & 0.98 & 0.95 \\
27726 & 105320 & 1999 & 85.60 & 0.08 & 8500.00 & 81857.08 & 1.01 & 0.96 & 0.96 \\
15396 & 101989 & 1999 & 424.70 & 0.61 & 45492.00 & 381299.31 & 0.93 & 0.90 & 0.84 \\
33611 & 106156 & 1999 & 35.00 & 0.56 & 3493.00 & 34926.15 & 1.00 & 1.00 & 1.00 \\
29512 & 105603 & 1999 & 3.50 & 0.42 & 345.00 & 3108.69 & 1.01 & 0.89 & 0.90 \\
29197 & 105536 & 1999 & 624.10 & 1.61 & 62005.00 & 572664.03 & 1.01 & 0.92 & 0.92 \\
28881 & 105502 & 1999 & 2316.40 & 0.20 & 231819.00 & 1959403.43 & 1.00 & 0.85 & 0.85 \\
37285 & 106726 & 1999 & 2263.90 & 0.20 & 225934.00 & 2216232.42 & 1.00 & 0.98 & 0.98 \\
27709 & 105317 & 1999 & 981.90 & 0.20 & 103462.00 & 966957.38 & 0.95 & 0.98 & 0.93 \\
33623 & 106157 & 1999 & 332.60 & 1.25 & 33338.00 & 319326.12 & 1.00 & 0.96 & 0.96 \\
45505 & 200071 & 1999 & 542.00 & 0.55 & 51376.00 & 510683.24 & 1.05 & 0.94 & 0.99 \\
29518 & 105604 & 1999 & 43.20 & 0.08 & 4316.00 & 42419.24 & 1.00 & 0.98 & 0.98 \\
36290 & 106481 & 1999 & 25.70 & 0.10 & 2394.00 & 23933.76 & 1.07 & 0.93 & 1.00 \\
36297 & 106482 & 1999 & 25.00 & -0.00 & 2415.00 & 24208.22 & 1.04 & 0.97 & 1.00 \\
17582 & 102319 & 1999 & 941.00 & 0.01 & 90259.00 & 899144.38 & 1.04 & 0.96 & 1.00 \\
31936 & 105963 & 1999 & 1082.80 & 0.30 & 88752.00 & 930321.22 & 1.22 & 0.86 & 1.05 \\
45364 & 200050 & 1999 & 30.50 & -0.07 & 3022.00 & 28810.70 & 1.01 & 0.94 & 0.95 \\
4702 & 100667 & 1999 & 15.80 & 0.07 & 1542.00 & 15021.48 & 1.02 & 0.95 & 0.97 \\
74645 & 601147 & 1999 & 91.30 & 0.10 & 9130.00 & 81175.97 & 1.00 & 0.89 & 0.89 \\
74815 & 601183 & 1999 & 12.50 & 0.07 & 1246.00 & 11100.84 & 1.00 & 0.89 & 0.89 \\
32854 & 106069 & 1999 & 67.50 & 0.24 & 6762.00 & 62133.54 & 1.00 & 0.92 & 0.92 \\
27359 & 105271 & 1999 & 61.40 & 0.19 & 6179.00 & 61097.69 & 0.99 & 1.00 & 0.99 \\
29066 & 105523 & 1999 & 182.10 & 0.16 & 16459.00 & 178554.25 & 1.11 & 0.98 & 1.08 \\
4723 & 100669 & 1999 & 186.60 & 0.74 & 18660.00 & 173193.03 & 1.00 & 0.93 & 0.93 \\
4675 & 100660 & 1999 & 969.40 & 0.18 & 146920.00 & 1454042.86 & 0.66 & 1.50 & 0.99 \\
37617 & 106975 & 1999 & 31.50 & 0.13 & 2871.00 & 30063.19 & 1.10 & 0.95 & 1.05 \\
15113 & 101958 & 1999 & 1254.00 & 0.20 & 125333.00 & 1239958.90 & 1.00 & 0.99 & 0.99 \\
27337 & 105269 & 1999 & 291.50 & 0.08 & 28072.00 & 260936.61 & 1.04 & 0.90 & 0.93 \\
18320 & 102425 & 1999 & 3044.90 & 0.46 & 232893.00 & 2751224.05 & 1.31 & 0.90 & 1.18 \\
4641 & 100659 & 1999 & 489.80 & 0.42 & 49009.00 & 489961.79 & 1.00 & 1.00 & 1.00 \\
47379 & 210681 & 1999 & 46927.00 & 0.25 & 4704171.00 & 40590787.44 & 1.00 & 0.86 & 0.86 \\
36128 & 106467 & 1999 & 205.50 & 0.22 & 15770.00 & 159921.98 & 1.30 & 0.78 & 1.01 \\
27370 & 105275 & 1999 & 109.10 & 0.11 & 10968.00 & 106839.49 & 0.99 & 0.98 & 0.97 \\
35638 & 106381 & 1999 & 8.40 & -0.01 & 863.00 & 8529.40 & 0.97 & 1.02 & 0.99 \\
37533 & 106961 & 1999 & 114.00 & 0.25 & 11384.00 & 113551.05 & 1.00 & 1.00 & 1.00 \\
18120 & 102399 & 1999 & 11.40 & 0.21 & 1144.00 & 10686.26 & 1.00 & 0.94 & 0.93 \\
18129 & 102404 & 1999 & 3615.40 & 0.20 & 360987.00 & 3557496.00 & 1.00 & 0.98 & 0.99 \\
4761 & 100671 & 1999 & 448.90 & -0.05 & 61392.00 & 563579.94 & 0.73 & 1.26 & 0.92 \\
29037 & 105522 & 1999 & 104.50 & 0.07 & 10475.00 & 102590.55 & 1.00 & 0.98 & 0.98 \\
31863 & 105949 & 1999 & 371.50 & 0.23 & 37399.00 & 335145.06 & 0.99 & 0.90 & 0.90 \\
36137 & 106470 & 1999 & 617.20 & 0.14 & 58219.00 & 582464.39 & 1.06 & 0.94 & 1.00 \\
37540 & 106962 & 1999 & 4.90 & 0.29 & 482.00 & 4470.98 & 1.02 & 0.91 & 0.93 \\
18179 & 102414 & 1999 & 4468.30 & 0.42 & 318332.00 & 3935158.30 & 1.40 & 0.88 & 1.24 \\
4739 & 100670 & 1999 & 112.30 & 0.11 & 11124.00 & 108669.18 & 1.01 & 0.97 & 0.98 \\
6504 & 100878 & 1999 & 2758.00 & 0.20 & 271948.00 & 2711553.12 & 1.01 & 0.98 & 1.00 \\
37548 & 106968 & 1999 & 248.80 & 0.81 & 17160.00 & 241373.93 & 1.45 & 0.97 & 1.41 \\
27400 & 105276 & 1999 & 851.40 & 0.09 & 84964.00 & 785452.54 & 1.00 & 0.92 & 0.92 \\
15145 & 101963 & 1999 & 1130.40 & 0.08 & 114406.00 & 1109566.26 & 0.99 & 0.98 & 0.97 \\
37573 & 106969 & 1999 & 16.70 & -0.01 & 1500.00 & 16118.58 & 1.11 & 0.97 & 1.07 \\
74642 & 601146 & 1999 & 22.50 & 0.36 & 1919.00 & 17935.92 & 1.17 & 0.80 & 0.93 \\
29076 & 105525 & 1999 & 249.00 & 0.02 & 25719.00 & 218189.29 & 0.97 & 0.88 & 0.85 \\
18471 & 102462 & 1999 & 7.50 & -0.16 & 746.00 & 7444.67 & 1.01 & 0.99 & 1.00 \\
36078 & 106459 & 1999 & 1.20 & 0.00 & 123.00 & 1183.20 & 0.98 & 0.99 & 0.96 \\
18493 & 102465 & 1999 & 527.20 & 0.37 & 52314.00 & 507276.54 & 1.01 & 0.96 & 0.97 \\
4561 & 100639 & 1999 & 1385.20 & 0.37 & 110288.00 & 1325197.81 & 1.26 & 0.96 & 1.20 \\
47410 & 210770 & 1999 & 1576.60 & 0.05 & 177942.00 & 1803627.48 & 0.89 & 1.14 & 1.01 \\
37652 & 106985 & 1999 & 60.80 & 0.06 & 6286.00 & 58410.69 & 0.97 & 0.96 & 0.93 \\
37658 & 106992 & 1999 & 177.10 & 0.05 & 17547.00 & 173862.21 & 1.01 & 0.98 & 0.99 \\
58780 & 410217 & 1999 & 2.40 & 0.07 & 227.00 & 2112.06 & 1.06 & 0.88 & 0.93 \\
31798 & 105938 & 1999 & 86.20 & 1.60 & 8638.00 & 82370.47 & 1.00 & 0.96 & 0.95 \\
15061 & 101955 & 1999 & 18832.80 & 0.17 & 1388002.00 & 15134037.38 & 1.36 & 0.80 & 1.09 \\
18512 & 102469 & 1999 & 202.90 & -0.02 & 20401.00 & 201306.91 & 0.99 & 0.99 & 0.99 \\
630 & 100085 & 1999 & 27878.40 & 0.29 & 2787842.00 & 25115087.86 & 1.00 & 0.90 & 0.90 \\
37669 & 106993 & 1999 & 36.20 & 0.01 & 3673.00 & 36018.76 & 0.99 & 0.99 & 0.98 \\
18525 & 102470 & 1999 & 1322.80 & 0.08 & 148037.00 & 1423204.19 & 0.89 & 1.08 & 0.96 \\
27245 & 105259 & 1999 & 424.90 & 0.17 & 42458.00 & 420222.75 & 1.00 & 0.99 & 0.99 \\
18544 & 102474 & 1999 & 210.70 & 0.47 & 15399.00 & 203584.81 & 1.37 & 0.97 & 1.32 \\
4527 & 100637 & 1999 & 918.30 & 0.36 & 92726.00 & 890050.04 & 0.99 & 0.97 & 0.96 \\
35601 & 106379 & 1999 & 171.00 & 0.33 & 17067.00 & 155161.81 & 1.00 & 0.91 & 0.91 \\
35611 & 106380 & 1999 & 43.70 & 0.12 & 4370.00 & 43633.70 & 1.00 & 1.00 & 1.00 \\
15174 & 101964 & 1999 & 535.10 & 0.07 & 53457.00 & 534620.86 & 1.00 & 1.00 & 1.00 \\
32835 & 106067 & 1999 & 134.20 & 0.32 & 13415.00 & 123876.73 & 1.00 & 0.92 & 0.92 \\
6540 & 100889 & 1999 & 17.40 & -0.11 & 1748.00 & 16695.97 & 1.00 & 0.96 & 0.96 \\
18354 & 102446 & 1999 & 16.80 & 0.17 & 1679.00 & 15675.36 & 1.00 & 0.93 & 0.93 \\
4595 & 100642 & 1999 & 1605.80 & 0.17 & 152040.00 & 1634958.43 & 1.06 & 1.02 & 1.08 \\
37637 & 106980 & 1999 & 191.60 & 0.04 & 19144.00 & 180525.54 & 1.00 & 0.94 & 0.94 \\
37639 & 106983 & 1999 & 36.10 & 0.16 & 3610.00 & 32860.12 & 1.00 & 0.91 & 0.91 \\
37645 & 106984 & 1999 & 12.50 & 0.02 & 1244.00 & 12435.33 & 1.00 & 0.99 & 1.00 \\
31813 & 105943 & 1999 & 39.50 & 0.09 & 3957.00 & 38880.93 & 1.00 & 0.98 & 0.98 \\
27312 & 105268 & 1999 & 521.10 & -0.00 & 51758.00 & 495238.12 & 1.01 & 0.95 & 0.96 \\
18398 & 102447 & 1999 & 1740.30 & 0.15 & 173722.00 & 1596528.65 & 1.00 & 0.92 & 0.92 \\
6549 & 100890 & 1999 & 1200.00 & 0.20 & 119870.00 & 1192141.64 & 1.00 & 0.99 & 0.99 \\
15095 & 101956 & 1999 & 4097.30 & 0.24 & 411353.00 & 4019598.53 & 1.00 & 0.98 & 0.98 \\
18428 & 102452 & 1999 & 186.70 & 0.44 & 13783.00 & 169610.94 & 1.35 & 0.91 & 1.23 \\
29104 & 105527 & 1999 & 5.00 & 0.11 & 493.00 & 4613.52 & 1.01 & 0.92 & 0.94 \\
32807 & 106066 & 1999 & 715.30 & 0.90 & 71530.00 & 624912.17 & 1.00 & 0.87 & 0.87 \\
32849 & 106068 & 1999 & 23.70 & 0.10 & 2362.00 & 22396.38 & 1.00 & 0.94 & 0.95 \\
18454 & 102461 & 1999 & 4491.20 & 0.92 & 449467.00 & 4035095.18 & 1.00 & 0.90 & 0.90 \\
31809 & 105942 & 1999 & 2.50 & 0.72 & 237.00 & 2372.98 & 1.05 & 0.95 & 1.00 \\
29616 & 105627 & 1999 & 211.60 & -0.12 & 21188.00 & 202926.12 & 1.00 & 0.96 & 0.96 \\
31805 & 105941 & 1999 & 26.10 & 0.35 & 1821.00 & 22144.92 & 1.43 & 0.85 & 1.22 \\
45374 & 200051 & 1999 & 4.00 & 0.01 & 380.00 & 3799.94 & 1.05 & 0.95 & 1.00 \\
33727 & 106164 & 1999 & 43.40 & 1.09 & 4440.00 & 43550.53 & 0.98 & 1.00 & 0.98 \\
28958 & 105508 & 1999 & 32.50 & 0.39 & 3252.00 & 32343.26 & 1.00 & 1.00 & 0.99 \\
6429 & 100868 & 1999 & 161.30 & -0.00 & 16130.00 & 157547.45 & 1.00 & 0.98 & 0.98 \\
27541 & 105287 & 1999 & 493.20 & 1.91 & 49126.00 & 491248.10 & 1.00 & 1.00 & 1.00 \\
33694 & 106161 & 1999 & 10.30 & -0.07 & 1028.00 & 8347.54 & 1.00 & 0.81 & 0.81 \\
36166 & 106476 & 1999 & 8.30 & 0.09 & 790.00 & 7666.93 & 1.05 & 0.92 & 0.97 \\
570 & 100076 & 1999 & 1079.10 & 0.41 & 107910.00 & 1014338.71 & 1.00 & 0.94 & 0.94 \\
17842 & 102365 & 1999 & 1201.10 & 0.25 & 126187.00 & 1186581.57 & 0.95 & 0.99 & 0.94 \\
321 & 100036 & 1999 & 145.80 & 0.13 & 14586.00 & 143044.19 & 1.00 & 0.98 & 0.98 \\
33697 & 106162 & 1999 & 14.60 & 0.45 & 1452.00 & 13980.42 & 1.01 & 0.96 & 0.96 \\
4888 & 100691 & 1999 & 1294.00 & 0.48 & 129446.00 & 1259815.87 & 1.00 & 0.97 & 0.97 \\
45294 & 200015 & 1999 & 10.20 & 0.46 & 1022.00 & 9392.82 & 1.00 & 0.92 & 0.92 \\
74675 & 601149 & 1999 & 38.40 & -0.08 & 3845.00 & 37890.05 & 1.00 & 0.99 & 0.99 \\
15251 & 101972 & 1999 & 1544.80 & 0.54 & 140293.00 & 1403132.05 & 1.10 & 0.91 & 1.00 \\
33700 & 106163 & 1999 & 157.30 & 0.38 & 14803.00 & 159635.70 & 1.06 & 1.01 & 1.08 \\
32753 & 106061 & 1999 & 187.70 & 0.58 & 18964.00 & 177565.36 & 0.99 & 0.95 & 0.94 \\
32750 & 106060 & 1999 & 2.50 & 0.16 & 231.00 & 2348.46 & 1.08 & 0.94 & 1.02 \\
32913 & 106082 & 1999 & 328.60 & 0.02 & 29574.00 & 256577.73 & 1.11 & 0.78 & 0.87 \\
31971 & 105965 & 1999 & 11.00 & 0.29 & 851.00 & 10291.26 & 1.29 & 0.94 & 1.21 \\
27598 & 105299 & 1999 & 3.50 & 0.01 & 378.00 & 3074.80 & 0.93 & 0.88 & 0.81 \\
29175 & 105534 & 1999 & 358.20 & 0.83 & 35823.00 & 338139.42 & 1.00 & 0.94 & 0.94 \\
17782 & 102357 & 1999 & 1935.90 & 0.08 & 193313.00 & 1898185.50 & 1.00 & 0.98 & 0.98 \\
47251 & 200344 & 1999 & 2211.20 & 0.02 & 217162.00 & 2050697.77 & 1.02 & 0.93 & 0.94 \\
15300 & 101982 & 1999 & 464.10 & 0.18 & 51044.00 & 409632.31 & 0.91 & 0.88 & 0.80 \\
37412 & 106869 & 1999 & 128.00 & 0.11 & 12849.00 & 126154.88 & 1.00 & 0.99 & 0.98 \\
27588 & 105295 & 1999 & 473.20 & 0.16 & 47315.00 & 462547.22 & 1.00 & 0.98 & 0.98 \\
4908 & 100692 & 1999 & 2307.40 & 0.41 & 230925.00 & 2171295.05 & 1.00 & 0.94 & 0.94 \\
29546 & 105611 & 1999 & 24.80 & 0.22 & 2482.00 & 22825.19 & 1.00 & 0.92 & 0.92 \\
35672 & 106386 & 1999 & 1420.60 & 0.18 & 161009.00 & 1547449.07 & 0.88 & 1.09 & 0.96 \\
37420 & 106896 & 1999 & 542.60 & 0.62 & 54223.00 & 524510.26 & 1.00 & 0.97 & 0.97 \\
31963 & 105964 & 1999 & 192.10 & 0.33 & 19229.00 & 162697.04 & 1.00 & 0.85 & 0.85 \\
37426 & 106914 & 1999 & 673.30 & 0.24 & 46810.00 & 406467.77 & 1.44 & 0.60 & 0.87 \\
15287 & 101978 & 1999 & 117.60 & 0.18 & 12106.00 & 119832.85 & 0.97 & 1.02 & 0.99 \\
28939 & 105507 & 1999 & 964.20 & 0.21 & 96225.00 & 938555.51 & 1.00 & 0.97 & 0.98 \\
17811 & 102364 & 1999 & 1776.70 & 0.21 & 180446.00 & 1791994.26 & 0.98 & 1.01 & 0.99 \\
45333 & 200039 & 1999 & 45.80 & 0.06 & 4586.00 & 44111.37 & 1.00 & 0.96 & 0.96 \\
35663 & 106383 & 1999 & 121.80 & 0.24 & 12200.00 & 115160.34 & 1.00 & 0.95 & 0.94 \\
17870 & 102367 & 1999 & 722.80 & 0.09 & 76331.00 & 716552.26 & 0.95 & 0.99 & 0.94 \\
28984 & 105510 & 1999 & 50.50 & 0.56 & 3707.00 & 52917.68 & 1.36 & 1.05 & 1.43 \\
31925 & 105961 & 1999 & 35.00 & 0.45 & 3500.00 & 30626.81 & 1.00 & 0.88 & 0.88 \\
6455 & 100875 & 1999 & 250.30 & 0.03 & 24996.00 & 241266.44 & 1.00 & 0.96 & 0.97 \\
17993 & 102383 & 1999 & 9.60 & 0.01 & 1038.00 & 9116.60 & 0.92 & 0.95 & 0.88 \\
27459 & 105279 & 1999 & 65.10 & 0.12 & 6959.00 & 71430.17 & 0.94 & 1.10 & 1.03 \\
31895 & 105954 & 1999 & 3.40 & 0.03 & 338.00 & 3181.21 & 1.01 & 0.94 & 0.94 \\
36156 & 106474 & 1999 & 90.98 & 0.20 & 9089.00 & 87578.20 & 1.00 & 0.96 & 0.96 \\
36148 & 106471 & 1999 & 174.30 & -0.05 & 17437.00 & 174387.85 & 1.00 & 1.00 & 1.00 \\
15216 & 101968 & 1999 & 247.90 & 1.03 & 24009.00 & 238152.95 & 1.03 & 0.96 & 0.99 \\
18007 & 102386 & 1999 & 113.40 & -0.02 & 11349.00 & 110256.52 & 1.00 & 0.97 & 0.97 \\
31888 & 105951 & 1999 & 335.90 & 1.18 & 33637.00 & 331796.93 & 1.00 & 0.99 & 0.99 \\
37512 & 106944 & 1999 & 6.20 & 0.19 & 621.00 & 5734.36 & 1.00 & 0.92 & 0.92 \\
18042 & 102387 & 1999 & 35.10 & -0.13 & 3562.00 & 33629.47 & 0.99 & 0.96 & 0.94 \\
37519 & 106947 & 1999 & 34.90 & 0.05 & 3667.00 & 33882.70 & 0.95 & 0.97 & 0.92 \\
36048 & 106451 & 1999 & 282.50 & 0.46 & 25855.00 & 276532.04 & 1.09 & 0.98 & 1.07 \\
18080 & 102396 & 1999 & 983.20 & 0.15 & 98314.00 & 927572.46 & 1.00 & 0.94 & 0.94 \\
47335 & 210203 & 1999 & 7947.10 & 0.46 & 765704.00 & 7660014.10 & 1.04 & 0.96 & 1.00 \\
4807 & 100682 & 1999 & 81.30 & -0.07 & 7706.00 & 75472.97 & 1.06 & 0.93 & 0.98 \\
37521 & 106948 & 1999 & 43.40 & 0.24 & 4406.00 & 39540.18 & 0.99 & 0.91 & 0.90 \\
29029 & 105520 & 1999 & 22.30 & -0.02 & 2894.00 & 26171.21 & 0.77 & 1.17 & 0.90 \\
27429 & 105278 & 1999 & 1165.80 & 0.78 & 114998.00 & 1006217.45 & 1.01 & 0.86 & 0.87 \\
32777 & 106062 & 1999 & 7.10 & 0.02 & 789.00 & 6989.78 & 0.90 & 0.98 & 0.89 \\
27465 & 105280 & 1999 & 60.50 & 0.05 & 5808.00 & 61262.96 & 1.04 & 1.01 & 1.05 \\
595 & 100079 & 1999 & 3363.80 & 0.51 & 336623.00 & 3217824.69 & 1.00 & 0.96 & 0.96 \\
36033 & 106449 & 1999 & 395.60 & 1.53 & 39245.00 & 330868.59 & 1.01 & 0.84 & 0.84 \\
27504 & 105283 & 1999 & 9.50 & 0.18 & 895.00 & 9325.09 & 1.06 & 0.98 & 1.04 \\
31915 & 105960 & 1999 & 95.20 & 0.39 & 9512.00 & 92346.61 & 1.00 & 0.97 & 0.97 \\
32871 & 106075 & 1999 & 379.30 & 0.45 & 38084.00 & 344524.79 & 1.00 & 0.91 & 0.90 \\
15235 & 101970 & 1999 & 42.40 & 0.04 & 4095.00 & 40947.44 & 1.04 & 0.97 & 1.00 \\
17883 & 102371 & 1999 & 394.50 & -0.03 & 39050.00 & 376881.51 & 1.01 & 0.96 & 0.97 \\
45275 & 200011 & 1999 & 114.60 & 0.50 & 11485.00 & 113525.40 & 1.00 & 0.99 & 0.99 \\
29575 & 105616 & 1999 & 22.00 & 0.99 & 1116.00 & 20418.72 & 1.97 & 0.93 & 1.83 \\
37477 & 106931 & 1999 & 692.50 & 1.48 & 81527.00 & 808898.73 & 0.85 & 1.17 & 0.99 \\
17903 & 102372 & 1999 & 3356.00 & -0.10 & 475955.00 & 4044483.39 & 0.71 & 1.21 & 0.85 \\
37486 & 106934 & 1999 & 1693.80 & 1.58 & 174769.00 & 1655634.92 & 0.97 & 0.98 & 0.95 \\
27474 & 105281 & 1999 & 316.30 & -0.08 & 31220.00 & 267289.37 & 1.01 & 0.85 & 0.86 \\
28995 & 105511 & 1999 & 26.00 & -0.03 & 2458.00 & 24009.69 & 1.06 & 0.92 & 0.98 \\
17933 & 102376 & 1999 & 42.20 & 0.67 & 4222.00 & 38501.77 & 1.00 & 0.91 & 0.91 \\
45195 & 109437 & 1999 & 130.90 & 0.06 & 18552.00 & 178380.30 & 0.71 & 1.36 & 0.96 \\
31910 & 105959 & 1999 & 7.10 & 0.01 & 768.00 & 7208.90 & 0.92 & 1.02 & 0.94 \\
35659 & 106382 & 1999 & 115.30 & 0.20 & 11529.00 & 114292.37 & 1.00 & 0.99 & 0.99 \\
17967 & 102377 & 1999 & 105.30 & 0.53 & 10513.00 & 103978.06 & 1.00 & 0.99 & 0.99 \\
4840 & 100685 & 1999 & 13.20 & 0.03 & 1316.00 & 12110.74 & 1.00 & 0.92 & 0.92 \\
27908 & 105346 & 1999 & 732.40 & -0.00 & 80406.00 & 832335.62 & 0.91 & 1.14 & 1.04 \\
35729 & 106392 & 1999 & 154.80 & 0.38 & 11281.00 & 144880.24 & 1.37 & 0.94 & 1.28 \\
36787 & 106594 & 1999 & 4.60 & 0.39 & 363.00 & 4186.44 & 1.27 & 0.91 & 1.15 \\
5622 & 100775 & 1999 & 804.10 & 0.19 & 80271.00 & 782274.30 & 1.00 & 0.97 & 0.97 \\
6011 & 100818 & 1999 & 153.40 & 0.36 & 15100.00 & 144936.41 & 1.02 & 0.94 & 0.96 \\
35864 & 106420 & 1999 & 44.20 & 0.27 & 4423.00 & 42683.62 & 1.00 & 0.97 & 0.97 \\
33309 & 106114 & 1999 & 71.20 & 0.12 & 7210.00 & 68652.76 & 0.99 & 0.96 & 0.95 \\
36468 & 106535 & 1999 & 161.90 & 0.16 & 13558.00 & 147154.48 & 1.19 & 0.91 & 1.09 \\
28276 & 105400 & 1999 & 19.20 & 1.21 & 1815.00 & 17790.49 & 1.06 & 0.93 & 0.98 \\
33095 & 106091 & 1999 & 78.70 & 0.74 & 7804.00 & 74513.74 & 1.01 & 0.95 & 0.95 \\
16724 & 102182 & 1999 & 47.50 & 0.11 & 4744.00 & 46273.66 & 1.00 & 0.97 & 0.98 \\
28649 & 105458 & 1999 & 592.90 & 0.19 & 58836.00 & 557155.17 & 1.01 & 0.94 & 0.95 \\
6021 & 100820 & 1999 & 709.50 & 0.34 & 70686.00 & 707027.25 & 1.00 & 1.00 & 1.00 \\
36815 & 106597 & 1999 & 186.80 & 0.27 & 17580.00 & 188115.36 & 1.06 & 1.01 & 1.07 \\
16025 & 102073 & 1999 & 11698.50 & 0.10 & 1176922.00 & 11419262.21 & 0.99 & 0.98 & 0.97 \\
16741 & 102183 & 1999 & 91.90 & 0.96 & 9197.00 & 86747.56 & 1.00 & 0.94 & 0.94 \\
36822 & 106602 & 1999 & 270.90 & 1.48 & 28050.00 & 270199.34 & 0.97 & 1.00 & 0.96 \\
28634 & 105457 & 1999 & 1596.70 & 0.33 & 160237.00 & 1569088.49 & 1.00 & 0.98 & 0.98 \\
5647 & 100784 & 1999 & 4624.50 & 0.20 & 464168.00 & 4252425.88 & 1.00 & 0.92 & 0.92 \\
16690 & 102178 & 1999 & 1222.50 & 0.39 & 122389.00 & 1194153.20 & 1.00 & 0.98 & 0.98 \\
33122 & 106092 & 1999 & 918.90 & 0.14 & 91856.00 & 850009.02 & 1.00 & 0.93 & 0.93 \\
29274 & 105576 & 1999 & 14.90 & -0.04 & 1476.00 & 14791.19 & 1.01 & 0.99 & 1.00 \\
35890 & 106422 & 1999 & 178.50 & -0.02 & 25089.00 & 220362.41 & 0.71 & 1.23 & 0.88 \\
16647 & 102173 & 1999 & 87.70 & 0.07 & 8821.00 & 86991.21 & 0.99 & 0.99 & 0.99 \\
35930 & 106435 & 1999 & 350.50 & 0.10 & 35055.00 & 344422.49 & 1.00 & 0.98 & 0.98 \\
16657 & 102175 & 1999 & 1010.80 & 0.25 & 101058.00 & 982100.58 & 1.00 & 0.97 & 0.97 \\
46275 & 200207 & 1999 & 13.40 & 0.01 & 1205.00 & 11551.50 & 1.11 & 0.86 & 0.96 \\
46266 & 200205 & 1999 & 76.30 & 0.45 & 7181.00 & 58781.44 & 1.06 & 0.77 & 0.82 \\
36769 & 106589 & 1999 & 42.00 & 0.11 & 4039.00 & 39918.82 & 1.04 & 0.95 & 0.99 \\
33277 & 106110 & 1999 & 271.10 & 0.24 & 27120.00 & 267625.53 & 1.00 & 0.99 & 0.99 \\
36830 & 106604 & 1999 & 9.40 & -0.03 & 940.00 & 9131.81 & 1.00 & 0.97 & 0.97 \\
5677 & 100785 & 1999 & 1625.10 & 0.23 & 167222.00 & 1635169.75 & 0.97 & 1.01 & 0.98 \\
16076 & 102079 & 1999 & 580.50 & 0.48 & 58451.00 & 478734.65 & 0.99 & 0.82 & 0.82 \\
36771 & 106590 & 1999 & 34.90 & -0.04 & 3308.00 & 33084.42 & 1.06 & 0.95 & 1.00 \\
29263 & 105574 & 1999 & 66.80 & -0.05 & 6494.00 & 64896.82 & 1.03 & 0.97 & 1.00 \\
28305 & 105401 & 1999 & 4.30 & 0.49 & 478.00 & 4003.62 & 0.90 & 0.93 & 0.84 \\
33282 & 106113 & 1999 & 538.70 & 0.20 & 55366.00 & 522819.29 & 0.97 & 0.97 & 0.94 \\
32560 & 106041 & 1999 & 42.10 & 0.80 & 4214.00 & 43573.29 & 1.00 & 1.03 & 1.03 \\
29328 & 105587 & 1999 & 7.70 & 0.60 & 761.00 & 7563.48 & 1.01 & 0.98 & 0.99 \\
35872 & 106421 & 1999 & 7.60 & -0.03 & 749.00 & 7448.11 & 1.01 & 0.98 & 0.99 \\
5982 & 100817 & 1999 & 182.10 & 0.18 & 18189.00 & 177640.79 & 1.00 & 0.98 & 0.98 \\
33336 & 106116 & 1999 & 9.90 & -0.18 & 643.00 & 6431.35 & 1.54 & 0.65 & 1.00 \\
33072 & 106089 & 1999 & 92.00 & 0.36 & 7591.00 & 76141.46 & 1.21 & 0.83 & 1.00 \\
35956 & 106442 & 1999 & 3443.80 & 0.30 & 265250.00 & 3067766.72 & 1.30 & 0.89 & 1.16 \\
57840 & 401082 & 1999 & 5.30 & -0.11 & 535.00 & 4741.25 & 0.99 & 0.89 & 0.89 \\
35845 & 106418 & 1999 & 727.30 & 0.58 & 72684.00 & 708478.69 & 1.00 & 0.97 & 0.97 \\
6083 & 100822 & 1999 & 16.00 & 0.15 & 1691.00 & 16916.11 & 0.95 & 1.06 & 1.00 \\
16846 & 102197 & 1999 & 108.80 & 0.12 & 8666.00 & 83500.87 & 1.26 & 0.77 & 0.96 \\
5505 & 100769 & 1999 & 2798.70 & 0.13 & 272485.00 & 2674779.53 & 1.03 & 0.96 & 0.98 \\
36865 & 106606 & 1999 & 7.50 & -0.01 & 889.00 & 9312.80 & 0.84 & 1.24 & 1.05 \\
28687 & 105464 & 1999 & 16.40 & -0.04 & 1639.00 & 13899.71 & 1.00 & 0.85 & 0.85 \\
74794 & 601172 & 1999 & 1.20 & -0.06 & 94.00 & 939.75 & 1.28 & 0.78 & 1.00 \\
33395 & 106128 & 1999 & 162.60 & 0.33 & 30948.00 & 309638.25 & 0.53 & 1.90 & 1.00 \\
57848 & 401145 & 1999 & 96.60 & 1.22 & 9353.00 & 93567.84 & 1.03 & 0.97 & 1.00 \\
35841 & 106417 & 1999 & 118.80 & 0.06 & 12187.00 & 120000.44 & 0.97 & 1.01 & 0.98 \\
57853 & 401189 & 1999 & 115.70 & 0.33 & 8711.00 & 110387.44 & 1.33 & 0.95 & 1.27 \\
28678 & 105463 & 1999 & 3453.30 & 1.39 & 350728.00 & 3243752.48 & 0.98 & 0.94 & 0.92 \\
28198 & 105391 & 1999 & 16.90 & 0.31 & 1698.00 & 16311.10 & 1.00 & 0.97 & 0.96 \\
35859 & 106419 & 1999 & 2718.60 & 0.17 & 271839.00 & 2648913.44 & 1.00 & 0.97 & 0.97 \\
5536 & 100771 & 1999 & 136.40 & -0.02 & 13376.00 & 132663.11 & 1.02 & 0.97 & 0.99 \\
29354 & 105588 & 1999 & 14.10 & 0.17 & 1413.00 & 12678.03 & 1.00 & 0.90 & 0.90 \\
28234 & 105397 & 1999 & 63.70 & -0.05 & 6182.00 & 61963.32 & 1.03 & 0.97 & 1.00 \\
6052 & 100821 & 1999 & 75.10 & -0.07 & 7534.00 & 75335.46 & 1.00 & 1.00 & 1.00 \\
15981 & 102062 & 1999 & 138.20 & 0.13 & 13143.00 & 131439.49 & 1.05 & 0.95 & 1.00 \\
33347 & 106123 & 1999 & 930.80 & 0.22 & 76588.00 & 773082.01 & 1.22 & 0.83 & 1.01 \\
32331 & 106011 & 1999 & 1833.70 & 0.26 & 182588.00 & 1742831.95 & 1.00 & 0.95 & 0.95 \\
28226 & 105394 & 1999 & 38.60 & 0.28 & 4001.00 & 37307.07 & 0.96 & 0.97 & 0.93 \\
29360 & 105589 & 1999 & 195.00 & 1.07 & 20720.00 & 185270.34 & 0.94 & 0.95 & 0.89 \\
16774 & 102191 & 1999 & 67.70 & 0.06 & 6765.00 & 63058.58 & 1.00 & 0.93 & 0.93 \\
28247 & 105399 & 1999 & 42.30 & 0.40 & 3898.00 & 38868.83 & 1.09 & 0.92 & 1.00 \\
15943 & 102061 & 1999 & 34.90 & 0.38 & 2892.00 & 28919.05 & 1.21 & 0.83 & 1.00 \\
74791 & 601171 & 1999 & 847.70 & 0.20 & 88231.00 & 750156.47 & 0.96 & 0.88 & 0.85 \\
33354 & 106124 & 1999 & 59.90 & 0.32 & 6044.00 & 58353.10 & 0.99 & 0.97 & 0.97 \\
33081 & 106090 & 1999 & 91.00 & -0.05 & 9056.00 & 91225.81 & 1.00 & 1.00 & 1.01 \\
15913 & 102059 & 1999 & 551.50 & 0.21 & 55319.00 & 520690.83 & 1.00 & 0.94 & 0.94 \\
28207 & 105393 & 1999 & 24.50 & 0.30 & 2456.00 & 22562.69 & 1.00 & 0.92 & 0.92 \\
33380 & 106127 & 1999 & 143.80 & -0.05 & 14363.00 & 141623.56 & 1.00 & 0.98 & 0.99 \\
32306 & 106010 & 1999 & 430.60 & 0.04 & 42839.00 & 373441.53 & 1.01 & 0.87 & 0.87 \\
16817 & 102193 & 1999 & 765.90 & 0.11 & 72446.00 & 724844.79 & 1.06 & 0.95 & 1.00 \\
32588 & 106042 & 1999 & 6.60 & -0.07 & 649.00 & 6126.28 & 1.02 & 0.93 & 0.94 \\
28322 & 105412 & 1999 & 216.90 & 0.23 & 21797.00 & 201204.35 & 1.00 & 0.93 & 0.92 \\
16626 & 102166 & 1999 & 164.70 & 0.56 & 16493.00 & 161520.80 & 1.00 & 0.98 & 0.98 \\
28336 & 105416 & 1999 & 525.90 & 0.33 & 45588.00 & 438440.12 & 1.15 & 0.83 & 0.96 \\
33215 & 106103 & 1999 & 29.80 & 0.31 & 2968.00 & 29701.32 & 1.00 & 1.00 & 1.00 \\
36577 & 106566 & 1999 & 2.60 & 1.13 & 261.00 & 2520.17 & 1.00 & 0.97 & 0.97 \\
16413 & 102134 & 1999 & 113.30 & -0.00 & 11341.00 & 111860.47 & 1.00 & 0.99 & 0.99 \\
32472 & 106033 & 1999 & 464.80 & 0.08 & 47607.00 & 473900.19 & 0.98 & 1.02 & 1.00 \\
35918 & 106429 & 1999 & 58.70 & -0.01 & 5490.00 & 54188.18 & 1.07 & 0.92 & 0.99 \\
5902 & 100811 & 1999 & 1628.70 & 0.17 & 164614.00 & 1596109.64 & 0.99 & 0.98 & 0.97 \\
36584 & 106567 & 1999 & 2.70 & 0.47 & 181.00 & 2195.56 & 1.49 & 0.81 & 1.21 \\
16441 & 102145 & 1999 & 158.30 & 0.12 & 16117.00 & 158437.16 & 0.98 & 1.00 & 0.98 \\
36591 & 106568 & 1999 & 3.90 & 0.14 & 331.00 & 3225.40 & 1.18 & 0.83 & 0.97 \\
16463 & 102150 & 1999 & 135.70 & 0.52 & 13943.00 & 129800.07 & 0.97 & 0.96 & 0.93 \\
16200 & 102090 & 1999 & 4792.50 & 0.36 & 479580.00 & 4120070.10 & 1.00 & 0.86 & 0.86 \\
28458 & 105426 & 1999 & 618.90 & -0.07 & 57692.00 & 533001.12 & 1.07 & 0.86 & 0.92 \\
33162 & 106101 & 1999 & 12.20 & 0.01 & 1198.00 & 11912.54 & 1.02 & 0.98 & 0.99 \\
36617 & 106569 & 1999 & 78.50 & 1.51 & 7850.00 & 74707.58 & 1.00 & 0.95 & 0.95 \\
33226 & 106104 & 1999 & 9.30 & 0.02 & 918.00 & 9200.74 & 1.01 & 0.99 & 1.00 \\
35922 & 106434 & 1999 & 758.70 & 0.09 & 76179.00 & 721769.52 & 1.00 & 0.95 & 0.95 \\
16494 & 102151 & 1999 & 19.50 & 0.53 & 2068.00 & 17659.54 & 0.94 & 0.91 & 0.85 \\
5751 & 100791 & 1999 & 3798.40 & -0.16 & 431864.00 & 3650051.21 & 0.88 & 0.96 & 0.85 \\
32516 & 106038 & 1999 & 211.70 & 0.04 & 21086.00 & 199260.76 & 1.00 & 0.94 & 0.94 \\
16404 & 102133 & 1999 & 67.50 & 0.45 & 6988.00 & 67166.02 & 0.97 & 1.00 & 0.96 \\
28538 & 105437 & 1999 & 2724.00 & -0.01 & 290065.00 & 2800575.22 & 0.94 & 1.03 & 0.97 \\
16396 & 102132 & 1999 & 97.70 & 0.24 & 9788.00 & 96750.75 & 1.00 & 0.99 & 0.99 \\
16272 & 102113 & 1999 & 334.00 & 0.39 & 33069.00 & 330917.43 & 1.01 & 0.99 & 1.00 \\
32507 & 106037 & 1999 & 46.20 & 0.00 & 4828.00 & 46521.17 & 0.96 & 1.01 & 0.96 \\
16282 & 102121 & 1999 & 84.30 & 0.32 & 7149.00 & 70804.16 & 1.18 & 0.84 & 0.99 \\
32500 & 106036 & 1999 & 102.00 & 1.67 & 10122.00 & 104696.58 & 1.01 & 1.03 & 1.03 \\
416 & 100055 & 1999 & 10463.20 & 0.16 & 989026.00 & 9075471.10 & 1.06 & 0.87 & 0.92 \\
36534 & 106560 & 1999 & 31.20 & 0.21 & 2697.00 & 29341.42 & 1.16 & 0.94 & 1.09 \\
28516 & 105432 & 1999 & 11.50 & 0.08 & 1145.00 & 11314.32 & 1.00 & 0.98 & 0.99 \\
5870 & 100809 & 1999 & 2834.90 & 0.24 & 283895.00 & 2765071.60 & 1.00 & 0.98 & 0.97 \\
16307 & 102124 & 1999 & 2435.50 & -0.01 & 244240.00 & 2422347.06 & 1.00 & 0.99 & 0.99 \\
36622 & 106570 & 1999 & 161.40 & 0.82 & 16139.00 & 149770.50 & 1.00 & 0.93 & 0.93 \\
5823 & 100804 & 1999 & 4750.90 & 0.10 & 475703.00 & 4575899.55 & 1.00 & 0.96 & 0.96 \\
16256 & 102105 & 1999 & 149.40 & -0.03 & 14539.00 & 145402.06 & 1.03 & 0.97 & 1.00 \\
36553 & 106561 & 1999 & 1.60 & -0.03 & 153.00 & 1530.51 & 1.05 & 0.96 & 1.00 \\
16361 & 102130 & 1999 & 1096.80 & 0.16 & 109728.00 & 1072275.58 & 1.00 & 0.98 & 0.98 \\
5783 & 100792 & 1999 & 612.10 & 0.00 & 60799.00 & 595595.97 & 1.01 & 0.97 & 0.98 \\
33188 & 106102 & 1999 & 8.10 & 0.97 & 459.00 & 4666.21 & 1.76 & 0.58 & 1.02 \\
16241 & 102104 & 1999 & 251.90 & -0.03 & 24873.00 & 217247.11 & 1.01 & 0.86 & 0.87 \\
36574 & 106563 & 1999 & 25.90 & 0.06 & 2391.00 & 21316.93 & 1.08 & 0.82 & 0.89 \\
28487 & 105427 & 1999 & 102.90 & 1.09 & 10295.00 & 98229.32 & 1.00 & 0.95 & 0.95 \\
29302 & 105585 & 1999 & 9.50 & 0.07 & 937.00 & 9247.55 & 1.01 & 0.97 & 0.99 \\
33145 & 106097 & 1999 & 15.80 & 0.45 & 1582.00 & 15453.00 & 1.00 & 0.98 & 0.98 \\
16169 & 102089 & 1999 & 211.40 & 0.02 & 21096.00 & 190135.07 & 1.00 & 0.90 & 0.90 \\
36626 & 106571 & 1999 & 12.50 & 0.27 & 1261.00 & 10105.95 & 0.99 & 0.81 & 0.80 \\
16141 & 102085 & 1999 & 3214.40 & 0.27 & 321763.00 & 2589313.03 & 1.00 & 0.81 & 0.80 \\
36687 & 106577 & 1999 & 1748.20 & 0.53 & 101018.00 & 1415817.81 & 1.73 & 0.81 & 1.40 \\
16606 & 102160 & 1999 & 3.40 & 0.97 & 251.00 & 2896.36 & 1.35 & 0.85 & 1.15 \\
16107 & 102080 & 1999 & 1080.20 & 0.33 & 93540.00 & 924220.50 & 1.15 & 0.86 & 0.99 \\
33246 & 106108 & 1999 & 48.70 & 0.12 & 4862.00 & 41771.49 & 1.00 & 0.86 & 0.86 \\
28374 & 105420 & 1999 & 148.80 & 0.34 & 11650.00 & 141092.36 & 1.28 & 0.95 & 1.21 \\
32544 & 106039 & 1999 & 102.80 & 0.03 & 10286.00 & 93590.06 & 1.00 & 0.91 & 0.91 \\
36726 & 106581 & 1999 & 26.90 & -0.04 & 3467.00 & 25293.05 & 0.78 & 0.94 & 0.73 \\
36685 & 106576 & 1999 & 3.50 & 0.02 & 339.00 & 3227.63 & 1.03 & 0.92 & 0.95 \\
33267 & 106109 & 1999 & 28.20 & 0.19 & 2826.00 & 24230.56 & 1.00 & 0.86 & 0.86 \\
389 & 100048 & 1999 & 658.30 & 0.17 & 67215.00 & 605691.67 & 0.98 & 0.92 & 0.90 \\
36729 & 106583 & 1999 & 27.70 & 0.86 & 2813.00 & 28124.88 & 0.98 & 1.02 & 1.00 \\
28348 & 105419 & 1999 & 81.50 & 0.01 & 7874.00 & 71477.03 & 1.04 & 0.88 & 0.91 \\
16614 & 102163 & 1999 & 572.60 & 0.21 & 56618.00 & 534147.57 & 1.01 & 0.93 & 0.94 \\
32403 & 106023 & 1999 & 137.10 & 1.27 & 13726.00 & 126539.40 & 1.00 & 0.92 & 0.92 \\
28613 & 105450 & 1999 & 18.20 & 0.20 & 1864.00 & 18642.46 & 0.98 & 1.02 & 1.00 \\
36496 & 106541 & 1999 & 20.00 & 0.30 & 1657.00 & 18847.34 & 1.21 & 0.94 & 1.14 \\
5974 & 100815 & 1999 & 395.70 & 0.09 & 39580.00 & 386741.39 & 1.00 & 0.98 & 0.98 \\
36743 & 106584 & 1999 & 10.40 & 0.08 & 977.00 & 8612.51 & 1.06 & 0.83 & 0.88 \\
57739 & 401015 & 1999 & 7335.80 & 0.16 & 1445752.00 & 14402825.67 & 0.51 & 1.96 & 1.00 \\
33045 & 106088 & 1999 & 46.40 & 0.04 & 4537.00 & 45376.37 & 1.02 & 0.98 & 1.00 \\
36683 & 106575 & 1999 & 5.20 & 0.09 & 511.00 & 4854.51 & 1.02 & 0.93 & 0.95 \\
5941 & 100812 & 1999 & 506.60 & 0.23 & 50797.00 & 507425.81 & 1.00 & 1.00 & 1.00 \\
28429 & 105424 & 1999 & 3482.30 & 0.34 & 265549.00 & 3179036.99 & 1.31 & 0.91 & 1.20 \\
16516 & 102152 & 1999 & 312.80 & 0.13 & 31256.00 & 302166.13 & 1.00 & 0.97 & 0.97 \\
33231 & 106106 & 1999 & 3.70 & -0.08 & 372.00 & 3517.59 & 0.99 & 0.95 & 0.95 \\
35896 & 106424 & 1999 & 34.80 & 0.22 & 3468.00 & 33402.87 & 1.00 & 0.96 & 0.96 \\
74774 & 601168 & 1999 & 60.60 & 0.26 & 6057.00 & 59078.08 & 1.00 & 0.97 & 0.98 \\
16150 & 102087 & 1999 & 517.50 & 0.09 & 51089.00 & 428685.99 & 1.01 & 0.83 & 0.84 \\
16539 & 102154 & 1999 & 227.40 & -0.07 & 22731.00 & 190176.20 & 1.00 & 0.84 & 0.84 \\
5727 & 100790 & 1999 & 272.90 & -0.11 & 26241.00 & 262312.49 & 1.04 & 0.96 & 1.00 \\
16560 & 102156 & 1999 & 11.80 & -0.02 & 1180.00 & 10192.90 & 1.00 & 0.86 & 0.86 \\
28584 & 105448 & 1999 & 45.70 & -0.03 & 4573.00 & 41002.53 & 1.00 & 0.90 & 0.90 \\
33236 & 106107 & 1999 & 83.00 & 0.13 & 8304.00 & 74750.30 & 1.00 & 0.90 & 0.90 \\
46300 & 200210 & 1999 & 3.20 & 0.10 & 315.00 & 2674.88 & 1.02 & 0.84 & 0.85 \\
32444 & 106028 & 1999 & 53.40 & 0.54 & 5568.00 & 55275.95 & 0.96 & 1.04 & 0.99 \\
36678 & 106574 & 1999 & 19.60 & 0.18 & 2006.00 & 17587.49 & 0.98 & 0.90 & 0.88 \\
28400 & 105421 & 1999 & 41.60 & 0.13 & 3987.00 & 36725.01 & 1.04 & 0.88 & 0.92 \\
16590 & 102157 & 1999 & 2.50 & 0.27 & 207.00 & 2185.53 & 1.21 & 0.87 & 1.06 \\
36652 & 106573 & 1999 & 10.40 & -0.03 & 1052.00 & 9833.69 & 0.99 & 0.95 & 0.93 \\
17436 & 102306 & 1999 & 16171.10 & 0.17 & 1639172.00 & 14613856.86 & 0.99 & 0.90 & 0.89 \\
35829 & 106415 & 1999 & 24.80 & 0.38 & 2419.00 & 21058.08 & 1.03 & 0.85 & 0.87 \\
33399 & 106129 & 1999 & 456.10 & 0.12 & 45905.00 & 458394.53 & 0.99 & 1.01 & 1.00 \\
27931 & 105353 & 1999 & 83.20 & 0.00 & 8238.00 & 82363.53 & 1.01 & 0.99 & 1.00 \\
5247 & 100741 & 1999 & 218.90 & 0.27 & 21420.00 & 214177.67 & 1.02 & 0.98 & 1.00 \\
74716 & 601156 & 1999 & 82.50 & 0.43 & 8096.00 & 76178.68 & 1.02 & 0.92 & 0.94 \\
27925 & 105352 & 1999 & 316.40 & 0.23 & 30547.00 & 305635.88 & 1.04 & 0.97 & 1.00 \\
32130 & 105984 & 1999 & 210.30 & 0.24 & 21150.00 & 209217.79 & 0.99 & 0.99 & 0.99 \\
15572 & 102005 & 1999 & 1409.00 & 0.33 & 140879.00 & 1246476.65 & 1.00 & 0.88 & 0.88 \\
17251 & 102274 & 1999 & 1620.20 & 0.04 & 161925.00 & 1516889.77 & 1.00 & 0.94 & 0.94 \\
27918 & 105348 & 1999 & 15.70 & 0.27 & 1231.00 & 15005.56 & 1.28 & 0.96 & 1.22 \\
5214 & 100736 & 1999 & 321.50 & 0.15 & 32163.00 & 295221.12 & 1.00 & 0.92 & 0.92 \\
36349 & 106487 & 1999 & 22.00 & 0.06 & 2109.00 & 21075.00 & 1.04 & 0.96 & 1.00 \\
29460 & 105597 & 1999 & 85.90 & 0.29 & 8571.00 & 74790.83 & 1.00 & 0.87 & 0.87 \\
33561 & 106149 & 1999 & 790.30 & 1.03 & 79027.00 & 765597.14 & 1.00 & 0.97 & 0.97 \\
35753 & 106400 & 1999 & 84.00 & 0.20 & 8255.00 & 77009.46 & 1.02 & 0.92 & 0.93 \\
37121 & 106682 & 1999 & 24.70 & 0.33 & 2411.00 & 24125.39 & 1.02 & 0.98 & 1.00 \\
6230 & 100831 & 1999 & 193.60 & 0.07 & 19397.00 & 190994.26 & 1.00 & 0.99 & 0.98 \\
28531 & 105436 & 1999 & 42.20 & 0.10 & 4037.00 & 43714.20 & 1.05 & 1.04 & 1.08 \\
15587 & 102007 & 1999 & 7757.10 & 0.24 & 725842.00 & 7602844.60 & 1.07 & 0.98 & 1.05 \\
32661 & 106047 & 1999 & 14.10 & 1.13 & 1409.00 & 13894.74 & 1.00 & 0.99 & 0.99 \\
17224 & 102271 & 1999 & 2036.30 & 0.15 & 200220.00 & 2072025.37 & 1.02 & 1.02 & 1.03 \\
28806 & 105478 & 1999 & 30.70 & 0.15 & 3070.00 & 30041.21 & 1.00 & 0.98 & 0.98 \\
27982 & 105364 & 1999 & 137.00 & 0.06 & 13727.00 & 136278.50 & 1.00 & 0.99 & 0.99 \\
29438 & 105595 & 1999 & 42.40 & 0.25 & 4297.00 & 41138.78 & 0.99 & 0.97 & 0.96 \\
17190 & 102270 & 1999 & 1582.40 & 0.23 & 148763.00 & 1563182.27 & 1.06 & 0.99 & 1.05 \\
27976 & 105362 & 1999 & 56.30 & 0.16 & 5630.00 & 55721.05 & 1.00 & 0.99 & 0.99 \\
33536 & 106147 & 1999 & 59.80 & 0.52 & 5986.00 & 58808.76 & 1.00 & 0.98 & 0.98 \\
36361 & 106502 & 1999 & 24.50 & -0.06 & 2353.00 & 23530.25 & 1.04 & 0.96 & 1.00 \\
5269 & 100745 & 1999 & 2766.90 & 0.11 & 276151.00 & 2527201.08 & 1.00 & 0.91 & 0.92 \\
36359 & 106500 & 1999 & 7.90 & 0.16 & 789.00 & 7739.84 & 1.00 & 0.98 & 0.98 \\
33573 & 106150 & 1999 & 121.30 & 0.48 & 11972.00 & 102659.68 & 1.01 & 0.85 & 0.86 \\
37085 & 106675 & 1999 & 24.20 & 0.07 & 2424.00 & 21750.44 & 1.00 & 0.90 & 0.90 \\
37095 & 106678 & 1999 & 12.80 & 0.36 & 1297.00 & 11323.54 & 0.99 & 0.88 & 0.87 \\
6196 & 100829 & 1999 & 676.10 & 0.37 & 67960.00 & 653796.44 & 0.99 & 0.97 & 0.96 \\
27950 & 105358 & 1999 & 620.40 & 0.45 & 61704.00 & 575374.98 & 1.01 & 0.93 & 0.93 \\
32150 & 105990 & 1999 & 202.00 & 0.48 & 32850.00 & 320104.98 & 0.61 & 1.58 & 0.97 \\
32144 & 105987 & 1999 & 276.00 & 0.10 & 28035.00 & 278674.56 & 0.98 & 1.01 & 0.99 \\
35759 & 106401 & 1999 & 1089.00 & 0.71 & 108682.00 & 910184.81 & 1.00 & 0.84 & 0.84 \\
27945 & 105354 & 1999 & 349.50 & 0.42 & 25575.00 & 360276.80 & 1.37 & 1.03 & 1.41 \\
29223 & 105545 & 1999 & 1420.60 & 0.52 & 141722.00 & 1332545.36 & 1.00 & 0.94 & 0.94 \\
57932 & 410003 & 1999 & 1530.80 & 0.13 & 153333.00 & 1500802.49 & 1.00 & 0.98 & 0.98 \\
33550 & 106148 & 1999 & 32.40 & 0.11 & 3231.00 & 31724.66 & 1.00 & 0.98 & 0.98 \\
37076 & 106667 & 1999 & 10.80 & 0.04 & 895.00 & 8785.31 & 1.21 & 0.81 & 0.98 \\
27899 & 105343 & 1999 & 93.10 & 0.15 & 9235.00 & 90068.36 & 1.01 & 0.97 & 0.98 \\
27860 & 105335 & 1999 & 285.20 & 0.06 & 28502.00 & 281354.95 & 1.00 & 0.99 & 0.99 \\
15497 & 101999 & 1999 & 1127.80 & -0.02 & 112788.00 & 1090942.22 & 1.00 & 0.97 & 0.97 \\
338 & 100040 & 1999 & 2725.30 & 0.50 & 221841.00 & 2487502.38 & 1.23 & 0.91 & 1.12 \\
57966 & 410010 & 1999 & 176.90 & 0.03 & 17025.00 & 175788.97 & 1.04 & 0.99 & 1.03 \\
27852 & 105333 & 1999 & 19.90 & -0.04 & 1980.00 & 19803.05 & 1.01 & 1.00 & 1.00 \\
15478 & 101998 & 1999 & 1099.90 & 0.26 & 109987.00 & 1027805.07 & 1.00 & 0.93 & 0.93 \\
17383 & 102284 & 1999 & 206.60 & 0.01 & 20653.00 & 206574.81 & 1.00 & 1.00 & 1.00 \\
32059 & 105978 & 1999 & 264.00 & 1.28 & 25226.00 & 252267.40 & 1.05 & 0.96 & 1.00 \\
5133 & 100726 & 1999 & 11239.40 & 0.47 & 1123985.00 & 9386442.57 & 1.00 & 0.84 & 0.84 \\
35734 & 106394 & 1999 & 68.50 & 0.16 & 6850.00 & 66669.86 & 1.00 & 0.97 & 0.97 \\
37162 & 106706 & 1999 & 9.30 & 0.10 & 923.00 & 9035.24 & 1.01 & 0.97 & 0.98 \\
32700 & 106050 & 1999 & 438.70 & 0.40 & 43850.00 & 425616.11 & 1.00 & 0.97 & 0.97 \\
5113 & 100724 & 1999 & 75.60 & 0.15 & 7557.00 & 70098.23 & 1.00 & 0.93 & 0.93 \\
37176 & 106707 & 1999 & 9.40 & 0.15 & 931.00 & 9306.94 & 1.01 & 0.99 & 1.00 \\
32940 & 106083 & 1999 & 731.30 & 0.01 & 67929.00 & 645447.71 & 1.08 & 0.88 & 0.95 \\
28853 & 105487 & 1999 & 186.70 & 0.72 & 19038.00 & 173431.88 & 0.98 & 0.93 & 0.91 \\
35984 & 106444 & 1999 & 172.80 & 0.20 & 16304.00 & 169342.59 & 1.06 & 0.98 & 1.04 \\
37155 & 106701 & 1999 & 103.00 & 0.13 & 10274.00 & 102111.06 & 1.00 & 0.99 & 0.99 \\
5156 & 100727 & 1999 & 597.00 & 0.31 & 59743.00 & 564857.45 & 1.00 & 0.95 & 0.95 \\
32071 & 105980 & 1999 & 176.40 & 0.13 & 17719.00 & 163909.16 & 1.00 & 0.93 & 0.93 \\
33598 & 106152 & 1999 & 556.10 & 0.27 & 57099.00 & 556434.77 & 0.97 & 1.00 & 0.97 \\
32103 & 105983 & 1999 & 361.30 & 0.28 & 28531.00 & 298445.52 & 1.27 & 0.83 & 1.05 \\
5196 & 100731 & 1999 & 17434.10 & -0.01 & 1742359.00 & 17050147.24 & 1.00 & 0.98 & 0.98 \\
35745 & 106398 & 1999 & 3.40 & 0.14 & 283.00 & 2949.60 & 1.20 & 0.87 & 1.04 \\
27894 & 105342 & 1999 & 17.30 & -0.03 & 1732.00 & 16158.24 & 1.00 & 0.93 & 0.93 \\
15528 & 102000 & 1999 & 901.60 & 0.50 & 90199.00 & 863410.51 & 1.00 & 0.96 & 0.96 \\
30 & 100003 & 1999 & 724.90 & 0.18 & 71765.00 & 669994.90 & 1.01 & 0.92 & 0.93 \\
74709 & 601155 & 1999 & 13.30 & -0.05 & 1296.00 & 11241.93 & 1.03 & 0.85 & 0.87 \\
36333 & 106485 & 1999 & 173.40 & 0.66 & 16358.00 & 163598.64 & 1.06 & 0.94 & 1.00 \\
17291 & 102278 & 1999 & 291.00 & -0.07 & 38154.00 & 333504.77 & 0.76 & 1.15 & 0.87 \\
37124 & 106688 & 1999 & 103.00 & 0.06 & 10312.00 & 86464.85 & 1.00 & 0.84 & 0.84 \\
37127 & 106690 & 1999 & 54.00 & -0.00 & 6323.00 & 53069.62 & 0.85 & 0.98 & 0.84 \\
17309 & 102280 & 1999 & 3488.50 & 0.08 & 348971.00 & 3311025.82 & 1.00 & 0.95 & 0.95 \\
5174 & 100730 & 1999 & 1276.90 & 1.00 & 127706.00 & 1224031.00 & 1.00 & 0.96 & 0.96 \\
33584 & 106151 & 1999 & 1050.50 & 0.11 & 106082.00 & 1042119.80 & 0.99 & 0.99 & 0.98 \\
37152 & 106695 & 1999 & 40.70 & -0.14 & 4059.00 & 37638.10 & 1.00 & 0.92 & 0.93 \\
36330 & 106484 & 1999 & 4.70 & 0.02 & 407.00 & 4068.52 & 1.15 & 0.87 & 1.00 \\
27874 & 105336 & 1999 & 13.80 & -0.07 & 1343.00 & 12474.80 & 1.03 & 0.90 & 0.93 \\
36323 & 106483 & 1999 & 3.00 & 0.09 & 293.00 & 2925.53 & 1.02 & 0.98 & 1.00 \\
17339 & 102282 & 1999 & 146.80 & 0.07 & 14689.00 & 141937.72 & 1.00 & 0.97 & 0.97 \\
32688 & 106049 & 1999 & 572.40 & 0.20 & 57375.00 & 531853.15 & 1.00 & 0.93 & 0.93 \\
6257 & 100833 & 1999 & 1397.30 & 0.30 & 140645.00 & 1402309.57 & 0.99 & 1.00 & 1.00 \\
37072 & 106666 & 1999 & 5.50 & 0.24 & 548.00 & 4962.54 & 1.00 & 0.90 & 0.91 \\
5290 & 100746 & 1999 & 1439.60 & 0.11 & 143897.00 & 1431897.15 & 1.00 & 0.99 & 1.00 \\
57919 & 402013 & 1999 & 126.90 & 0.28 & 12608.00 & 102854.34 & 1.01 & 0.81 & 0.82 \\
33418 & 106135 & 1999 & 53.20 & 0.04 & 5490.00 & 52617.92 & 0.97 & 0.99 & 0.96 \\
36921 & 106640 & 1999 & 70.50 & 0.38 & 7068.00 & 68009.52 & 1.00 & 0.96 & 0.96 \\
16967 & 102224 & 1999 & 3967.80 & 0.23 & 396779.00 & 3949905.93 & 1.00 & 1.00 & 1.00 \\
28123 & 105383 & 1999 & 164.90 & 0.18 & 16486.00 & 148237.12 & 1.00 & 0.90 & 0.90 \\
17003 & 102230 & 1999 & 43.50 & 0.20 & 4525.00 & 42398.14 & 0.96 & 0.97 & 0.94 \\
5401 & 100760 & 1999 & 1023.00 & 0.64 & 102524.00 & 984883.59 & 1.00 & 0.96 & 0.96 \\
28716 & 105471 & 1999 & 30.00 & 0.66 & 2913.00 & 29134.65 & 1.03 & 0.97 & 1.00 \\
36933 & 106642 & 1999 & 143.10 & 0.15 & 14311.00 & 127057.53 & 1.00 & 0.89 & 0.89 \\
33445 & 106136 & 1999 & 144.50 & 0.35 & 15583.00 & 129468.71 & 0.93 & 0.90 & 0.83 \\
15757 & 102017 & 1999 & 15592.00 & 0.07 & 1557627.00 & 14418353.45 & 1.00 & 0.92 & 0.93 \\
28724 & 105472 & 1999 & 399.30 & 1.26 & 37071.00 & 370819.31 & 1.08 & 0.93 & 1.00 \\
17039 & 102231 & 1999 & 1317.10 & 0.30 & 128810.00 & 1265493.63 & 1.02 & 0.96 & 0.98 \\
36438 & 106528 & 1999 & 14.80 & 1.11 & 1487.00 & 12597.17 & 1.00 & 0.85 & 0.85 \\
5368 & 100758 & 1999 & 133.90 & -0.02 & 13387.00 & 132158.72 & 1.00 & 0.99 & 0.99 \\
28094 & 105382 & 1999 & 174.20 & 0.40 & 17402.00 & 158964.19 & 1.00 & 0.91 & 0.91 \\
28075 & 105379 & 1999 & 335.20 & 0.21 & 27778.00 & 302664.26 & 1.21 & 0.90 & 1.09 \\
57874 & 401355 & 1999 & 17.40 & 0.40 & 1745.00 & 15657.17 & 1.00 & 0.90 & 0.90 \\
29398 & 105592 & 1999 & 296.30 & 0.41 & 29458.00 & 287282.09 & 1.01 & 0.97 & 0.98 \\
28137 & 105384 & 1999 & 87.70 & -0.03 & 11482.00 & 100876.44 & 0.76 & 1.15 & 0.88 \\
6114 & 100823 & 1999 & 30.80 & -0.12 & 3490.00 & 34901.70 & 0.88 & 1.13 & 1.00 \\
36871 & 106619 & 1999 & 454.20 & 1.67 & 42617.00 & 418198.67 & 1.07 & 0.92 & 0.98 \\
16876 & 102213 & 1999 & 3224.50 & 0.07 & 322449.00 & 2912776.28 & 1.00 & 0.90 & 0.90 \\
28169 & 105390 & 1999 & 340.30 & 0.06 & 31759.00 & 317315.27 & 1.07 & 0.93 & 1.00 \\
5467 & 100764 & 1999 & 316.40 & -0.06 & 31551.00 & 303099.75 & 1.00 & 0.96 & 0.96 \\
57901 & 401372 & 1999 & 9.40 & 0.07 & 925.00 & 8461.62 & 1.02 & 0.90 & 0.91 \\
461 & 100068 & 1999 & 137.20 & 0.22 & 13716.00 & 131292.63 & 1.00 & 0.96 & 0.96 \\
29236 & 105561 & 1999 & 30.60 & 0.04 & 3054.00 & 30313.37 & 1.00 & 0.99 & 0.99 \\
36877 & 106620 & 1999 & 48.50 & 0.11 & 4638.00 & 44071.43 & 1.05 & 0.91 & 0.95 \\
36447 & 106529 & 1999 & 67.60 & 0.43 & 6766.00 & 64905.02 & 1.00 & 0.96 & 0.96 \\
15788 & 102018 & 1999 & 1107.20 & 0.35 & 95727.00 & 1060673.67 & 1.16 & 0.96 & 1.11 \\
5445 & 100763 & 1999 & 1060.60 & 0.71 & 106043.00 & 990288.93 & 1.00 & 0.93 & 0.93 \\
33411 & 106133 & 1999 & 174.20 & 1.25 & 15521.00 & 154590.77 & 1.12 & 0.89 & 1.00 \\
5354 & 100757 & 1999 & 9.30 & 0.67 & 935.00 & 8898.13 & 0.99 & 0.96 & 0.95 \\
28072 & 105372 & 1999 & 12.60 & 0.22 & 1263.00 & 10443.71 & 1.00 & 0.83 & 0.83 \\
17074 & 102241 & 1999 & 272.10 & 0.39 & 25091.00 & 247361.76 & 1.08 & 0.91 & 0.99 \\
36392 & 106523 & 1999 & 12.30 & 0.21 & 1020.00 & 8374.06 & 1.21 & 0.68 & 0.82 \\
37059 & 106655 & 1999 & 216.30 & 0.70 & 21624.00 & 199332.18 & 1.00 & 0.92 & 0.92 \\
5324 & 100753 & 1999 & 1865.10 & -0.04 & 185565.00 & 1533576.98 & 1.01 & 0.82 & 0.83 \\
28014 & 105369 & 1999 & 523.40 & 0.15 & 46343.00 & 493467.91 & 1.13 & 0.94 & 1.06 \\
33504 & 106143 & 1999 & 212.70 & 0.64 & 14034.00 & 183421.49 & 1.52 & 0.86 & 1.31 \\
36389 & 106521 & 1999 & 1.90 & -0.11 & 189.00 & 1893.14 & 1.01 & 1.00 & 1.00 \\
32657 & 106046 & 1999 & 2.40 & -0.08 & 265.00 & 2112.59 & 0.91 & 0.88 & 0.80 \\
17137 & 102258 & 1999 & 913.80 & 0.23 & 91371.00 & 905816.72 & 1.00 & 0.99 & 0.99 \\
35803 & 106413 & 1999 & 97.90 & 0.58 & 9793.00 & 93033.09 & 1.00 & 0.95 & 0.95 \\
33531 & 106144 & 1999 & 126.50 & 1.11 & 12850.00 & 124542.49 & 0.98 & 0.98 & 0.97 \\
45869 & 200151 & 1999 & 35.60 & 0.19 & 3556.00 & 30019.69 & 1.00 & 0.84 & 0.84 \\
32181 & 105999 & 1999 & 6.00 & 0.00 & 620.00 & 6366.19 & 0.97 & 1.06 & 1.03 \\
17147 & 102259 & 1999 & 285.90 & 0.43 & 22135.00 & 280159.94 & 1.29 & 0.98 & 1.27 \\
35785 & 106402 & 1999 & 41.70 & 0.04 & 4124.00 & 41237.61 & 1.01 & 0.99 & 1.00 \\
57912 & 402003 & 1999 & 115.80 & -0.01 & 11605.00 & 104379.16 & 1.00 & 0.90 & 0.90 \\
487 & 100071 & 1999 & 8580.30 & 0.33 & 858027.00 & 7822913.63 & 1.00 & 0.91 & 0.91 \\
36363 & 106519 & 1999 & 10.20 & 0.48 & 1017.00 & 9882.90 & 1.00 & 0.97 & 0.97 \\
32177 & 105997 & 1999 & 22.90 & -0.05 & 2331.00 & 23179.17 & 0.98 & 1.01 & 0.99 \\
37067 & 106664 & 1999 & 29.50 & 0.14 & 2742.00 & 29316.06 & 1.08 & 0.99 & 1.07 \\
15638 & 102010 & 1999 & 19755.50 & 0.50 & 1504395.00 & 18511275.13 & 1.31 & 0.94 & 1.23 \\
37071 & 106665 & 1999 & 544.40 & 0.17 & 51719.00 & 499505.82 & 1.05 & 0.92 & 0.97 \\
17161 & 102261 & 1999 & 1775.80 & 0.01 & 177574.00 & 1741931.20 & 1.00 & 0.98 & 0.98 \\
15678 & 102013 & 1999 & 3853.50 & 0.47 & 266220.00 & 2593242.85 & 1.45 & 0.67 & 0.97 \\
28775 & 105476 & 1999 & 51.90 & -0.04 & 4886.00 & 46865.23 & 1.06 & 0.90 & 0.96 \\
29405 & 105593 & 1999 & 25.30 & 0.18 & 2524.00 & 23814.49 & 1.00 & 0.94 & 0.94 \\
32642 & 106045 & 1999 & 46.20 & 0.21 & 4618.00 & 43818.83 & 1.00 & 0.95 & 0.95 \\
28746 & 105475 & 1999 & 469.90 & 0.14 & 46451.00 & 445790.22 & 1.01 & 0.95 & 0.96 \\
36433 & 106527 & 1999 & 11.40 & 0.25 & 1132.00 & 11280.46 & 1.01 & 0.99 & 1.00 \\
6148 & 100825 & 1999 & 115.60 & 0.07 & 11529.00 & 115104.89 & 1.00 & 1.00 & 1.00 \\
37014 & 106650 & 1999 & 7.40 & -0.05 & 745.00 & 6764.14 & 0.99 & 0.91 & 0.91 \\
28043 & 105370 & 1999 & 181.00 & 0.26 & 15064.00 & 170033.23 & 1.20 & 0.94 & 1.13 \\
9 & 100001 & 1999 & 4365.90 & 0.22 & 436508.00 & 4010600.50 & 1.00 & 0.92 & 0.92 \\
37029 & 106652 & 1999 & 56.80 & -0.01 & 8032.00 & 83017.91 & 0.71 & 1.46 & 1.03 \\
45887 & 200155 & 1999 & 2.20 & 0.01 & 243.00 & 2379.51 & 0.91 & 1.08 & 0.98 \\
36412 & 106524 & 1999 & 11.60 & -0.04 & 1597.00 & 15843.11 & 0.73 & 1.37 & 0.99 \\
5345 & 100754 & 1999 & 1206.60 & -0.04 & 120382.00 & 1181760.90 & 1.00 & 0.98 & 0.98 \\
29412 & 105594 & 1999 & 37.30 & -0.04 & 3727.00 & 35679.11 & 1.00 & 0.96 & 0.96 \\
17105 & 102257 & 1999 & 1126.20 & 0.20 & 100709.00 & 1135041.78 & 1.12 & 1.01 & 1.13 \\
6161 & 100827 & 1999 & 304.60 & 0.21 & 30974.00 & 309725.10 & 0.98 & 1.02 & 1.00 \\
15698 & 102015 & 1999 & 746.30 & 0.84 & 47712.00 & 667901.78 & 1.56 & 0.89 & 1.40 \\
45880 & 200153 & 1999 & 38.80 & 0.08 & 3887.00 & 37450.37 & 1.00 & 0.97 & 0.96 \\
32209 & 106000 & 1999 & 290.40 & 0.59 & 28670.00 & 270783.10 & 1.01 & 0.93 & 0.94 \\
45888 & 200156 & 1999 & 42.00 & 0.08 & 3558.00 & 40180.09 & 1.18 & 0.96 & 1.13 \\
24032 & 103255 & 1999 & 186.60 & 0.29 & 18658.00 & 182872.12 & 1.00 & 0.98 & 0.98 \\
38011 & 107187 & 1999 & 263.20 & 1.73 & 26081.00 & 238888.18 & 1.01 & 0.91 & 0.92 \\
18553 & 102482 & 1999 & 306.40 & -0.10 & 31557.00 & 309350.60 & 0.97 & 1.01 & 0.98 \\
30173 & 105704 & 1999 & 16.80 & 0.24 & 1719.00 & 16248.00 & 0.98 & 0.97 & 0.95 \\
22216 & 102996 & 1999 & 454.90 & 0.50 & 46178.00 & 442166.37 & 0.99 & 0.97 & 0.96 \\
41882 & 108867 & 1999 & 732.70 & 0.16 & 73275.00 & 661161.59 & 1.00 & 0.90 & 0.90 \\
2817 & 100360 & 1999 & 1162.30 & 1.04 & 116289.00 & 1081334.86 & 1.00 & 0.93 & 0.93 \\
30177 & 105705 & 1999 & 354.60 & 1.26 & 35310.00 & 347549.73 & 1.00 & 0.98 & 0.98 \\
39635 & 107832 & 1999 & 143.70 & 0.25 & 14414.00 & 138179.13 & 1.00 & 0.96 & 0.96 \\
30910 & 105811 & 1999 & 5.20 & 0.07 & 533.00 & 5205.36 & 0.98 & 1.00 & 0.98 \\
24940 & 103395 & 1999 & 165.40 & 0.13 & 16541.00 & 139205.25 & 1.00 & 0.84 & 0.84 \\
30906 & 105809 & 1999 & 2.80 & -0.12 & 372.00 & 3190.49 & 0.75 & 1.14 & 0.86 \\
2828 & 100362 & 1999 & 39.30 & -0.05 & 3925.00 & 38968.38 & 1.00 & 0.99 & 0.99 \\
35211 & 106334 & 1999 & 307.70 & 0.00 & 30781.00 & 289996.49 & 1.00 & 0.94 & 0.94 \\
41857 & 108866 & 1999 & 130.00 & 0.11 & 13624.00 & 123025.96 & 0.95 & 0.95 & 0.90 \\
41851 & 108861 & 1999 & 79.50 & 0.04 & 8023.00 & 80244.18 & 0.99 & 1.01 & 1.00 \\
2795 & 100358 & 1999 & 1816.30 & 0.52 & 181633.00 & 1763851.64 & 1.00 & 0.97 & 0.97 \\
1683 & 100223 & 1999 & 4021.30 & 0.06 & 401115.00 & 3864215.05 & 1.00 & 0.96 & 0.96 \\
30899 & 105807 & 1999 & 31.50 & 0.01 & 2958.00 & 31431.42 & 1.06 & 1.00 & 1.06 \\
35184 & 106333 & 1999 & 59.00 & -0.04 & 5811.00 & 58113.87 & 1.02 & 0.98 & 1.00 \\
39662 & 107833 & 1999 & 1014.00 & 0.37 & 101414.00 & 990956.81 & 1.00 & 0.98 & 0.98 \\
22275 & 102999 & 1999 & 424.30 & 0.13 & 42357.00 & 422951.13 & 1.00 & 1.00 & 1.00 \\
41837 & 108860 & 1999 & 63.90 & 0.21 & 6468.00 & 61806.70 & 0.99 & 0.97 & 0.96 \\
22296 & 103005 & 1999 & 129.40 & 0.15 & 12303.00 & 122804.03 & 1.05 & 0.95 & 1.00 \\
41830 & 108859 & 1999 & 36.30 & 0.33 & 3828.00 & 37192.80 & 0.95 & 1.02 & 0.97 \\
41823 & 108858 & 1999 & 96.90 & 0.06 & 9554.00 & 90448.97 & 1.01 & 0.93 & 0.95 \\
30882 & 105806 & 1999 & 242.10 & 1.01 & 14231.00 & 210093.22 & 1.70 & 0.87 & 1.48 \\
22244 & 102997 & 1999 & 10581.70 & 0.74 & 726925.00 & 10117867.88 & 1.46 & 0.96 & 1.39 \\
41888 & 108868 & 1999 & 66.90 & 0.05 & 8637.00 & 80741.61 & 0.77 & 1.21 & 0.93 \\
22185 & 102994 & 1999 & 106.60 & 0.17 & 10692.00 & 107031.90 & 1.00 & 1.00 & 1.00 \\
34541 & 106251 & 1999 & 55.10 & 0.07 & 5455.00 & 53851.43 & 1.01 & 0.98 & 0.99 \\
42069 & 108918 & 1999 & 10.10 & 0.07 & 773.00 & 8454.57 & 1.31 & 0.84 & 1.09 \\
2880 & 100368 & 1999 & 376.20 & 0.26 & 43039.00 & 318616.22 & 0.87 & 0.85 & 0.74 \\
30145 & 105703 & 1999 & 104.90 & -0.04 & 10490.00 & 100874.84 & 1.00 & 0.96 & 0.96 \\
35246 & 106336 & 1999 & 21.40 & -0.22 & 3374.00 & 29521.70 & 0.63 & 1.38 & 0.87 \\
39577 & 107726 & 1999 & 837.20 & 0.14 & 77449.00 & 766661.51 & 1.08 & 0.92 & 0.99 \\
25035 & 103426 & 1999 & 1761.60 & 0.64 & 175759.00 & 1577157.14 & 1.00 & 0.90 & 0.90 \\
22074 & 102989 & 1999 & 2861.90 & 0.21 & 244874.00 & 2727543.09 & 1.17 & 0.95 & 1.11 \\
118 & 100009 & 1999 & 337.60 & 0.48 & 33870.00 & 337003.70 & 1.00 & 1.00 & 0.99 \\
30933 & 105838 & 1999 & 12.00 & 0.11 & 1204.00 & 11690.28 & 1.00 & 0.97 & 0.97 \\
22107 & 102990 & 1999 & 2937.90 & 0.05 & 244426.00 & 2455454.87 & 1.20 & 0.84 & 1.00 \\
41973 & 108886 & 1999 & 107.10 & 0.76 & 10741.00 & 101753.33 & 1.00 & 0.95 & 0.95 \\
41970 & 108876 & 1999 & 45.70 & 0.04 & 4687.00 & 43847.81 & 0.98 & 0.96 & 0.94 \\
1595 & 100217 & 1999 & 98.40 & 0.08 & 9530.00 & 96063.50 & 1.03 & 0.98 & 1.01 \\
22141 & 102993 & 1999 & 3703.20 & 0.20 & 367954.00 & 3536345.66 & 1.01 & 0.95 & 0.96 \\
2861 & 100366 & 1999 & 2.20 & 0.42 & 202.00 & 2067.33 & 1.09 & 0.94 & 1.02 \\
35219 & 106335 & 1999 & 669.70 & 1.20 & 66968.00 & 601727.49 & 1.00 & 0.90 & 0.90 \\
2852 & 100365 & 1999 & 744.50 & 0.15 & 76185.00 & 727309.40 & 0.98 & 0.98 & 0.95 \\
39625 & 107830 & 1999 & 798.10 & 0.03 & 78900.00 & 709167.12 & 1.01 & 0.89 & 0.90 \\
24992 & 103406 & 1999 & 2251.30 & 0.18 & 226489.00 & 2024903.74 & 0.99 & 0.90 & 0.89 \\
39673 & 107834 & 1999 & 237.50 & 0.17 & 23767.00 & 236023.72 & 1.00 & 0.99 & 0.99 \\
22332 & 103007 & 1999 & 1654.90 & 0.03 & 162799.00 & 1627894.45 & 1.02 & 0.98 & 1.00 \\
41782 & 108856 & 1999 & 6.10 & 0.02 & 568.00 & 5817.07 & 1.07 & 0.95 & 1.02 \\
41755 & 108853 & 1999 & 127.70 & 0.01 & 12620.00 & 108968.43 & 1.01 & 0.85 & 0.86 \\
36074 & 106458 & 1999 & 52.70 & 0.18 & 5122.00 & 51255.50 & 1.03 & 0.97 & 1.00 \\
34588 & 106257 & 1999 & 185.90 & 0.42 & 18582.00 & 163644.80 & 1.00 & 0.88 & 0.88 \\
2688 & 100352 & 1999 & 2464.80 & 0.19 & 246066.00 & 2403953.32 & 1.00 & 0.98 & 0.98 \\
1746 & 100227 & 1999 & 235.70 & 0.24 & 20176.00 & 237991.35 & 1.17 & 1.01 & 1.18 \\
39756 & 107860 & 1999 & 14.70 & 0.44 & 1173.00 & 14150.40 & 1.25 & 0.96 & 1.21 \\
22556 & 103019 & 1999 & 559.00 & 0.18 & 55421.00 & 495903.35 & 1.01 & 0.89 & 0.89 \\
34615 & 106258 & 1999 & 44.40 & 0.14 & 4498.00 & 43727.37 & 0.99 & 0.98 & 0.97 \\
22568 & 103021 & 1999 & 40.80 & 0.05 & 4153.00 & 41069.14 & 0.98 & 1.01 & 0.99 \\
41619 & 108782 & 1999 & 258.60 & 0.07 & 26063.00 & 253733.50 & 0.99 & 0.98 & 0.97 \\
41605 & 108780 & 1999 & 78.70 & 0.07 & 6822.00 & 71788.93 & 1.15 & 0.91 & 1.05 \\
2668 & 100351 & 1999 & 199.00 & 0.23 & 19822.00 & 194966.51 & 1.00 & 0.98 & 0.98 \\
176 & 100017 & 1999 & 174.00 & 0.07 & 17532.00 & 170873.53 & 0.99 & 0.98 & 0.97 \\
39759 & 107863 & 1999 & 640.90 & 0.00 & 65813.00 & 615903.65 & 0.97 & 0.96 & 0.94 \\
24800 & 103380 & 1999 & 7829.00 & 0.22 & 707781.00 & 8129978.00 & 1.11 & 1.04 & 1.15 \\
22599 & 103024 & 1999 & 367.60 & 0.11 & 36615.00 & 341201.97 & 1.00 & 0.93 & 0.93 \\
41595 & 108777 & 1999 & 24.00 & 0.18 & 2400.00 & 23997.12 & 1.00 & 1.00 & 1.00 \\
2649 & 100350 & 1999 & 230.70 & 0.35 & 23873.00 & 227918.36 & 0.97 & 0.99 & 0.95 \\
34619 & 106261 & 1999 & 406.70 & 0.97 & 40680.00 & 397700.07 & 1.00 & 0.98 & 0.98 \\
41563 & 108773 & 1999 & 10.90 & 0.02 & 1149.00 & 9969.05 & 0.95 & 0.91 & 0.87 \\
41548 & 108765 & 1999 & 627.90 & 0.08 & 63202.00 & 613506.41 & 0.99 & 0.98 & 0.97 \\
1766 & 100228 & 1999 & 197.40 & -0.01 & 20690.00 & 180778.01 & 0.95 & 0.92 & 0.87 \\
2630 & 100348 & 1999 & 235.20 & 0.20 & 24512.00 & 224280.24 & 0.96 & 0.95 & 0.91 \\
39766 & 107868 & 1999 & 72.80 & 0.23 & 7331.00 & 72989.69 & 0.99 & 1.00 & 1.00 \\
34630 & 106262 & 1999 & 607.80 & 0.37 & 60797.00 & 574020.30 & 1.00 & 0.94 & 0.94 \\
30137 & 105702 & 1999 & 90.10 & 0.46 & 8803.00 & 87768.49 & 1.02 & 0.97 & 1.00 \\
22527 & 103017 & 1999 & 2986.60 & 0.21 & 267455.00 & 2932829.78 & 1.12 & 0.98 & 1.10 \\
41642 & 108826 & 1999 & 48.70 & 0.09 & 4959.00 & 44109.76 & 0.98 & 0.91 & 0.89 \\
2751 & 100357 & 1999 & 324.70 & 0.11 & 32394.00 & 322196.38 & 1.00 & 0.99 & 0.99 \\
24902 & 103394 & 1999 & 37.10 & 0.01 & 3699.00 & 35322.24 & 1.00 & 0.95 & 0.95 \\
30204 & 105708 & 1999 & 620.50 & 1.44 & 62186.00 & 583403.31 & 1.00 & 0.94 & 0.94 \\
1702 & 100226 & 1999 & 9651.80 & -0.00 & 684772.00 & 6872276.43 & 1.41 & 0.71 & 1.00 \\
22376 & 103008 & 1999 & 186.20 & 0.24 & 17869.00 & 178744.13 & 1.04 & 0.96 & 1.00 \\
41749 & 108852 & 1999 & 41.80 & 0.20 & 4316.00 & 34810.66 & 0.97 & 0.83 & 0.81 \\
39679 & 107835 & 1999 & 365.50 & -0.04 & 36570.00 & 343904.48 & 1.00 & 0.94 & 0.94 \\
24881 & 103383 & 1999 & 1812.70 & 1.19 & 113685.00 & 1727085.33 & 1.59 & 0.95 & 1.52 \\
22412 & 103011 & 1999 & 93.40 & -0.09 & 9279.00 & 89501.89 & 1.01 & 0.96 & 0.96 \\
2720 & 100355 & 1999 & 7202.10 & 0.18 & 713559.00 & 7000125.94 & 1.01 & 0.97 & 0.98 \\
30210 & 105716 & 1999 & 1474.60 & 2.91 & 138213.00 & 1171474.19 & 1.07 & 0.79 & 0.85 \\
34555 & 106255 & 1999 & 1148.90 & 0.36 & 86808.00 & 1091726.99 & 1.32 & 0.95 & 1.26 \\
41724 & 108849 & 1999 & 213.20 & 0.08 & 21496.00 & 212582.69 & 0.99 & 1.00 & 0.99 \\
41700 & 108840 & 1999 & 44.80 & -0.01 & 4608.00 & 43182.93 & 0.97 & 0.96 & 0.94 \\
39705 & 107836 & 1999 & 1.70 & 0.14 & 174.00 & 1742.76 & 0.98 & 1.03 & 1.00 \\
34581 & 106256 & 1999 & 67.00 & 0.19 & 5396.00 & 55502.44 & 1.24 & 0.83 & 1.03 \\
30854 & 105805 & 1999 & 21.40 & 0.19 & 2130.00 & 20841.34 & 1.00 & 0.97 & 0.98 \\
39707 & 107837 & 1999 & 13.90 & 0.01 & 1383.00 & 12234.24 & 1.01 & 0.88 & 0.88 \\
39728 & 107858 & 1999 & 11.50 & 0.13 & 1327.00 & 9454.92 & 0.87 & 0.82 & 0.71 \\
24840 & 103381 & 1999 & 43755.20 & 0.41 & 3620648.00 & 43083441.58 & 1.21 & 0.98 & 1.19 \\
41667 & 108827 & 1999 & 825.40 & 0.16 & 81139.00 & 719205.36 & 1.02 & 0.87 & 0.89 \\
35181 & 106332 & 1999 & 12.00 & 0.02 & 1241.00 & 12407.41 & 0.97 & 1.03 & 1.00 \\
25073 & 103429 & 1999 & 3773.90 & 1.03 & 380061.00 & 3664806.20 & 0.99 & 0.97 & 0.96 \\
39561 & 107722 & 1999 & 22.90 & 0.25 & 2272.00 & 22760.05 & 1.01 & 0.99 & 1.00 \\
21595 & 102895 & 1999 & 2133.30 & 0.09 & 220718.00 & 2207254.65 & 0.97 & 1.03 & 1.00 \\
3140 & 100411 & 1999 & 5500.00 & 0.54 & 551497.00 & 5266212.67 & 1.00 & 0.96 & 0.95 \\
25352 & 103478 & 1999 & 964.40 & 0.42 & 96740.00 & 951335.28 & 1.00 & 0.99 & 0.98 \\
34523 & 106249 & 1999 & 117.60 & 0.20 & 11719.00 & 113777.37 & 1.00 & 0.97 & 0.97 \\
21609 & 102897 & 1999 & 298.10 & 0.01 & 30436.00 & 268869.54 & 0.98 & 0.90 & 0.88 \\
31039 & 105852 & 1999 & 49.20 & 0.08 & 4918.00 & 45731.34 & 1.00 & 0.93 & 0.93 \\
30083 & 105682 & 1999 & 260.10 & 0.26 & 25367.00 & 248062.20 & 1.03 & 0.95 & 0.98 \\
21621 & 102901 & 1999 & 99.70 & 0.54 & 10714.00 & 107136.55 & 0.93 & 1.07 & 1.00 \\
1367 & 100192 & 1999 & 94.20 & 0.12 & 9493.00 & 91638.67 & 0.99 & 0.97 & 0.97 \\
39366 & 107673 & 1999 & 30.20 & 0.06 & 2990.00 & 29600.97 & 1.01 & 0.98 & 0.99 \\
42316 & 108951 & 1999 & 102.90 & -0.02 & 10351.00 & 95104.36 & 0.99 & 0.92 & 0.92 \\
3105 & 100409 & 1999 & 398.00 & 0.49 & 39873.00 & 376584.06 & 1.00 & 0.95 & 0.94 \\
39373 & 107677 & 1999 & 79.40 & 0.10 & 7939.00 & 74306.69 & 1.00 & 0.94 & 0.94 \\
39389 & 107680 & 1999 & 467.00 & 0.06 & 49613.00 & 434232.89 & 0.94 & 0.93 & 0.88 \\
25311 & 103466 & 1999 & 1154.20 & 0.14 & 113170.00 & 1115662.62 & 1.02 & 0.97 & 0.99 \\
31034 & 105851 & 1999 & 50.90 & 0.38 & 4253.00 & 48255.90 & 1.20 & 0.95 & 1.13 \\
30090 & 105684 & 1999 & 15.30 & -0.01 & 1470.00 & 14710.68 & 1.04 & 0.96 & 1.00 \\
1418 & 100196 & 1999 & 1557.80 & 0.12 & 154356.00 & 1382248.58 & 1.01 & 0.89 & 0.90 \\
21676 & 102939 & 1999 & 7798.60 & 0.21 & 713460.00 & 7683876.41 & 1.09 & 0.99 & 1.08 \\
42267 & 108947 & 1999 & 1493.20 & 0.29 & 148153.00 & 1425173.15 & 1.01 & 0.95 & 0.96 \\
42258 & 108946 & 1999 & 6.90 & 0.17 & 757.00 & 6596.82 & 0.91 & 0.96 & 0.87 \\
31029 & 105849 & 1999 & 51.80 & 0.32 & 4260.00 & 46189.65 & 1.22 & 0.89 & 1.08 \\
39416 & 107692 & 1999 & 31.20 & 0.29 & 3205.00 & 27840.08 & 0.97 & 0.89 & 0.87 \\
21706 & 102940 & 1999 & 1095.30 & 0.24 & 90393.00 & 1053593.85 & 1.21 & 0.96 & 1.17 \\
42233 & 108944 & 1999 & 147.30 & 0.17 & 14709.00 & 131117.17 & 1.00 & 0.89 & 0.89 \\
42208 & 108943 & 1999 & 145.30 & 0.13 & 14557.00 & 141071.40 & 1.00 & 0.97 & 0.97 \\
42202 & 108939 & 1999 & 110.20 & 0.14 & 11047.00 & 109936.31 & 1.00 & 1.00 & 1.00 \\
30078 & 105681 & 1999 & 302.60 & 0.70 & 30128.00 & 258999.68 & 1.00 & 0.86 & 0.86 \\
3083 & 100408 & 1999 & 276.60 & 0.26 & 27654.00 & 274107.19 & 1.00 & 0.99 & 0.99 \\
42332 & 108952 & 1999 & 186.40 & 0.03 & 18113.00 & 170828.57 & 1.03 & 0.92 & 0.94 \\
34427 & 106239 & 1999 & 33.30 & 1.98 & 3603.00 & 31535.87 & 0.92 & 0.95 & 0.88 \\
39324 & 107654 & 1999 & 32.30 & 0.05 & 3220.00 & 31697.17 & 1.00 & 0.98 & 0.98 \\
39326 & 107664 & 1999 & 2.90 & 0.35 & 281.00 & 2810.85 & 1.03 & 0.97 & 1.00 \\
25433 & 103487 & 1999 & 52.50 & 0.28 & 5254.00 & 49620.22 & 1.00 & 0.95 & 0.94 \\
34443 & 106240 & 1999 & 141.60 & 0.36 & 14146.00 & 132419.59 & 1.00 & 0.94 & 0.94 \\
31082 & 105857 & 1999 & 95.80 & -0.01 & 9579.00 & 92181.16 & 1.00 & 0.96 & 0.96 \\
21393 & 102861 & 1999 & 278.00 & 0.48 & 27694.00 & 276461.88 & 1.00 & 0.99 & 1.00 \\
3212 & 100415 & 1999 & 453.70 & -0.04 & 52054.00 & 485332.55 & 0.87 & 1.07 & 0.93 \\
35277 & 106344 & 1999 & 152.80 & 0.19 & 13200.00 & 130572.17 & 1.16 & 0.85 & 0.99 \\
35273 & 106341 & 1999 & 60.20 & 0.43 & 6301.00 & 57481.28 & 0.96 & 0.95 & 0.91 \\
21421 & 102871 & 1999 & 113.00 & 0.14 & 13533.00 & 138096.33 & 0.83 & 1.22 & 1.02 \\
39329 & 107666 & 1999 & 79.20 & 1.64 & 7921.00 & 76849.87 & 1.00 & 0.97 & 0.97 \\
34470 & 106244 & 1999 & 5.30 & 0.63 & 529.00 & 4898.51 & 1.00 & 0.92 & 0.93 \\
42432 & 108966 & 1999 & 699.80 & 0.04 & 57776.00 & 550208.80 & 1.21 & 0.79 & 0.95 \\
31067 & 105854 & 1999 & 751.50 & 0.54 & 75153.00 & 706248.25 & 1.00 & 0.94 & 0.94 \\
25409 & 103483 & 1999 & 892.10 & 0.23 & 89236.00 & 863846.71 & 1.00 & 0.97 & 0.97 \\
21452 & 102872 & 1999 & 567.80 & 0.11 & 59869.00 & 564963.08 & 0.95 & 1.00 & 0.94 \\
21496 & 102875 & 1999 & 25.70 & -0.12 & 2598.00 & 25984.67 & 0.99 & 1.01 & 1.00 \\
3182 & 100413 & 1999 & 70.00 & 0.38 & 6827.00 & 60171.25 & 1.03 & 0.86 & 0.88 \\
21507 & 102876 & 1999 & 36.70 & -0.07 & 3681.00 & 36824.90 & 1.00 & 1.00 & 1.00 \\
1348 & 100190 & 1999 & 3757.50 & 0.26 & 375262.00 & 3393776.31 & 1.00 & 0.90 & 0.90 \\
30073 & 105680 & 1999 & 35.70 & 0.10 & 3611.00 & 35432.70 & 0.99 & 0.99 & 0.98 \\
39333 & 107670 & 1999 & 130.20 & 0.02 & 12547.00 & 125453.70 & 1.04 & 0.96 & 1.00 \\
25379 & 103481 & 1999 & 63.30 & 0.30 & 6337.00 & 55926.73 & 1.00 & 0.88 & 0.88 \\
21531 & 102893 & 1999 & 37.40 & 0.00 & 5392.00 & 45357.45 & 0.69 & 1.21 & 0.84 \\
42369 & 108955 & 1999 & 44.70 & 0.20 & 4831.00 & 46494.28 & 0.93 & 1.04 & 0.96 \\
34496 & 106248 & 1999 & 123.60 & 0.11 & 12573.00 & 119539.64 & 0.98 & 0.97 & 0.95 \\
42357 & 108953 & 1999 & 445.30 & 0.24 & 30984.00 & 369422.79 & 1.44 & 0.83 & 1.19 \\
39435 & 107693 & 1999 & 10.20 & 0.05 & 986.00 & 9718.37 & 1.03 & 0.95 & 0.99 \\
96674 & 611003 & 1999 & 1.10 & 0.05 & 116.00 & 1081.13 & 0.95 & 0.98 & 0.93 \\
30114 & 105700 & 1999 & 237.50 & 0.12 & 23749.00 & 225404.91 & 1.00 & 0.95 & 0.95 \\
21891 & 102969 & 1999 & 950.70 & 0.21 & 98531.00 & 996527.34 & 0.96 & 1.05 & 1.01 \\
2993 & 100395 & 1999 & 663.70 & 0.17 & 66340.00 & 639971.39 & 1.00 & 0.96 & 0.96 \\
1517 & 100209 & 1999 & 13313.20 & 1.03 & 1358284.00 & 12323144.87 & 0.98 & 0.93 & 0.91 \\
39508 & 107712 & 1999 & 16.60 & 0.15 & 1657.00 & 16793.39 & 1.00 & 1.01 & 1.01 \\
34532 & 106250 & 1999 & 66.40 & 0.25 & 6653.00 & 58836.66 & 1.00 & 0.89 & 0.88 \\
21908 & 102979 & 1999 & 82.10 & 0.02 & 8027.00 & 79136.94 & 1.02 & 0.96 & 0.99 \\
42126 & 108925 & 1999 & 1224.00 & 0.20 & 121793.00 & 1090752.06 & 1.00 & 0.89 & 0.90 \\
42114 & 108924 & 1999 & 72.40 & 0.04 & 7184.00 & 69129.51 & 1.01 & 0.95 & 0.96 \\
63123 & 500483 & 1999 & 630.80 & 0.17 & 63150.00 & 562257.82 & 1.00 & 0.89 & 0.89 \\
2953 & 100389 & 1999 & 695.70 & 0.16 & 69485.00 & 691443.75 & 1.00 & 0.99 & 1.00 \\
39516 & 107716 & 1999 & 334.00 & 0.32 & 36426.00 & 346211.95 & 0.92 & 1.04 & 0.95 \\
25146 & 103439 & 1999 & 70.50 & 0.32 & 7054.00 & 62831.40 & 1.00 & 0.89 & 0.89 \\
30124 & 105701 & 1999 & 997.70 & 0.32 & 99730.00 & 998450.29 & 1.00 & 1.00 & 1.00 \\
21956 & 102981 & 1999 & 154.30 & 0.59 & 14623.00 & 146263.30 & 1.06 & 0.95 & 1.00 \\
42094 & 108919 & 1999 & 362.80 & 0.23 & 36285.00 & 350486.05 & 1.00 & 0.97 & 0.97 \\
63177 & 500486 & 1999 & 327.80 & 0.24 & 35629.00 & 333624.28 & 0.92 & 1.02 & 0.94 \\
1535 & 100213 & 1999 & 183.90 & -0.04 & 17906.00 & 176818.39 & 1.03 & 0.96 & 0.99 \\
25136 & 103436 & 1999 & 19.20 & 0.20 & 1925.00 & 18178.65 & 1.00 & 0.95 & 0.94 \\
2914 & 100379 & 1999 & 627.20 & -0.07 & 62609.00 & 619815.67 & 1.00 & 0.99 & 0.99 \\
39547 & 107720 & 1999 & 98.80 & 0.20 & 7957.00 & 65366.29 & 1.24 & 0.66 & 0.82 \\
99 & 100006 & 1999 & 6198.20 & 0.07 & 619575.00 & 5447057.75 & 1.00 & 0.88 & 0.88 \\
25111 & 103432 & 1999 & 2559.80 & 0.44 & 183509.00 & 2231520.05 & 1.39 & 0.87 & 1.22 \\
22001 & 102984 & 1999 & 52.40 & 0.03 & 10477.00 & 98510.82 & 0.50 & 1.88 & 0.94 \\
30951 & 105841 & 1999 & 27.30 & 0.60 & 2745.00 & 27307.15 & 0.99 & 1.00 & 0.99 \\
2889 & 100369 & 1999 & 570.90 & 0.20 & 62488.00 & 516771.99 & 0.91 & 0.91 & 0.83 \\
1479 & 100207 & 1999 & 4011.60 & 0.13 & 398200.00 & 3629400.38 & 1.01 & 0.90 & 0.91 \\
25280 & 103464 & 1999 & 1867.00 & 0.22 & 170116.00 & 1881784.11 & 1.10 & 1.01 & 1.11 \\
21883 & 102964 & 1999 & 470.60 & 0.04 & 46077.00 & 460751.16 & 1.02 & 0.98 & 1.00 \\
21750 & 102949 & 1999 & 2338.70 & 0.05 & 226597.00 & 2257580.72 & 1.03 & 0.97 & 1.00 \\
31021 & 105848 & 1999 & 64.70 & 0.14 & 6196.00 & 56045.44 & 1.04 & 0.87 & 0.90 \\
65061 & 500659 & 1999 & 15.20 & 0.15 & 1530.00 & 14751.45 & 0.99 & 0.97 & 0.96 \\
30099 & 105686 & 1999 & 133.70 & 0.98 & 13364.00 & 131620.82 & 1.00 & 0.98 & 0.98 \\
194 & 100018 & 1999 & 493.50 & 0.27 & 48155.00 & 471394.78 & 1.02 & 0.96 & 0.98 \\
21768 & 102951 & 1999 & 6762.10 & 0.06 & 628650.00 & 6054726.54 & 1.08 & 0.90 & 0.96 \\
63003 & 500466 & 1999 & 381.30 & 0.14 & 38229.00 & 368355.46 & 1.00 & 0.97 & 0.96 \\
3053 & 100401 & 1999 & 369.40 & 0.04 & 36480.00 & 330963.41 & 1.01 & 0.90 & 0.91 \\
1448 & 100200 & 1999 & 123.50 & -0.04 & 12130.00 & 116530.22 & 1.02 & 0.94 & 0.96 \\
39454 & 107694 & 1999 & 6.50 & 0.00 & 639.00 & 6251.82 & 1.02 & 0.96 & 0.98 \\
3044 & 100400 & 1999 & 1331.90 & 0.35 & 121175.00 & 1212522.46 & 1.10 & 0.91 & 1.00 \\
30106 & 105694 & 1999 & 30.80 & 0.13 & 2990.00 & 29367.86 & 1.03 & 0.95 & 0.98 \\
21812 & 102952 & 1999 & 837.00 & 0.35 & 81025.00 & 809372.39 & 1.03 & 0.97 & 1.00 \\
30997 & 105846 & 1999 & 937.80 & 0.38 & 59489.00 & 759083.56 & 1.58 & 0.81 & 1.28 \\
39475 & 107702 & 1999 & 354.80 & 0.08 & 28512.00 & 309849.78 & 1.24 & 0.87 & 1.09 \\
39503 & 107711 & 1999 & 1.90 & 0.05 & 102.00 & 1019.45 & 1.86 & 0.54 & 1.00 \\
21853 & 102957 & 1999 & 1154.80 & 0.35 & 110189.00 & 1102257.42 & 1.05 & 0.95 & 1.00 \\
42133 & 108926 & 1999 & 33.60 & -0.01 & 3298.00 & 31917.97 & 1.02 & 0.95 & 0.97 \\
25198 & 103460 & 1999 & 1657.80 & 0.91 & 102971.00 & 1430285.30 & 1.61 & 0.86 & 1.39 \\
30979 & 105843 & 1999 & 16.20 & 0.13 & 1621.00 & 16090.29 & 1.00 & 0.99 & 0.99 \\
41530 & 108764 & 1999 & 507.80 & 0.21 & 48431.00 & 490660.91 & 1.05 & 0.97 & 1.01 \\
30798 & 105803 & 1999 & 868.60 & 0.38 & 59702.00 & 708224.87 & 1.45 & 0.82 & 1.19 \\
23519 & 103183 & 1999 & 1651.20 & 0.37 & 164541.00 & 1533499.75 & 1.00 & 0.93 & 0.93 \\
23545 & 103184 & 1999 & 548.00 & 0.11 & 54761.00 & 444608.89 & 1.00 & 0.81 & 0.81 \\
34959 & 106297 & 1999 & 43.00 & 0.29 & 4303.00 & 44012.71 & 1.00 & 1.02 & 1.02 \\
23552 & 103186 & 1999 & 1301.20 & 0.31 & 128000.00 & 1253442.29 & 1.02 & 0.96 & 0.98 \\
2135 & 100292 & 1999 & 205.20 & -0.06 & 20531.00 & 203290.14 & 1.00 & 0.99 & 0.99 \\
40161 & 108070 & 1999 & 160.60 & -0.04 & 17323.00 & 157365.99 & 0.93 & 0.98 & 0.91 \\
24233 & 103299 & 1999 & 582.80 & 0.64 & 58720.00 & 552425.57 & 0.99 & 0.95 & 0.94 \\
30372 & 105746 & 1999 & 190.10 & 0.15 & 18992.00 & 183345.61 & 1.00 & 0.96 & 0.97 \\
35059 & 106310 & 1999 & 21.10 & 0.17 & 2684.00 & 26839.06 & 0.79 & 1.27 & 1.00 \\
23571 & 103193 & 1999 & 43.90 & -0.09 & 4361.00 & 41415.03 & 1.01 & 0.94 & 0.95 \\
40901 & 108163 & 1999 & 250.50 & 0.29 & 23538.00 & 189095.15 & 1.06 & 0.75 & 0.80 \\
40181 & 108071 & 1999 & 54.40 & 0.68 & 5578.00 & 55799.32 & 0.98 & 1.03 & 1.00 \\
40899 & 108162 & 1999 & 122.40 & 0.06 & 11058.00 & 112243.95 & 1.11 & 0.92 & 1.02 \\
34970 & 106298 & 1999 & 19.60 & 0.43 & 2162.00 & 19435.20 & 0.91 & 0.99 & 0.90 \\
40207 & 108073 & 1999 & 50.90 & 0.01 & 4947.00 & 45373.57 & 1.03 & 0.89 & 0.92 \\
23598 & 103199 & 1999 & 6.10 & 0.04 & 573.00 & 5729.81 & 1.06 & 0.94 & 1.00 \\
30496 & 105762 & 1999 & 361.20 & 0.62 & 26535.00 & 364006.83 & 1.36 & 1.01 & 1.37 \\
24198 & 103296 & 1999 & 2884.40 & 0.40 & 288946.00 & 2846782.25 & 1.00 & 0.99 & 0.99 \\
23601 & 103202 & 1999 & 68.70 & 0.22 & 6747.00 & 67486.21 & 1.02 & 0.98 & 1.00 \\
40890 & 108161 & 1999 & 375.20 & 0.07 & 37384.00 & 347587.44 & 1.00 & 0.93 & 0.93 \\
23625 & 103204 & 1999 & 216.70 & 0.01 & 21366.00 & 212522.53 & 1.01 & 0.98 & 0.99 \\
40840 & 108158 & 1999 & 33.50 & 0.09 & 3352.00 & 32639.56 & 1.00 & 0.97 & 0.97 \\
2006 & 100280 & 1999 & 70.90 & 0.21 & 6762.00 & 69989.66 & 1.05 & 0.99 & 1.04 \\
96662 & 611002 & 1999 & 5800.60 & 0.29 & 519414.00 & 5555194.89 & 1.12 & 0.96 & 1.07 \\
40221 & 108074 & 1999 & 13.70 & 0.09 & 1364.00 & 13641.12 & 1.00 & 1.00 & 1.00 \\
23656 & 103205 & 1999 & 36.10 & -0.02 & 3576.00 & 35005.94 & 1.01 & 0.97 & 0.98 \\
40830 & 108156 & 1999 & 27.20 & 0.05 & 2453.00 & 24361.03 & 1.11 & 0.90 & 0.99 \\
40820 & 108155 & 1999 & 69.60 & 0.06 & 6997.00 & 67754.80 & 0.99 & 0.97 & 0.97 \\
34932 & 106294 & 1999 & 52.80 & -0.03 & 5663.00 & 54560.59 & 0.93 & 1.03 & 0.96 \\
40043 & 108018 & 1999 & 159.00 & 0.02 & 16014.00 & 145067.53 & 0.99 & 0.91 & 0.91 \\
23412 & 103175 & 1999 & 1534.30 & -0.03 & 153243.00 & 1446147.94 & 1.00 & 0.94 & 0.94 \\
30350 & 105740 & 1999 & 641.30 & -0.10 & 63637.00 & 551129.04 & 1.01 & 0.86 & 0.87 \\
30591 & 105775 & 1999 & 2678.10 & -0.01 & 355573.00 & 2724094.94 & 0.75 & 1.02 & 0.77 \\
40071 & 108021 & 1999 & 302.50 & 0.08 & 30157.00 & 272826.28 & 1.00 & 0.90 & 0.90 \\
24342 & 103315 & 1999 & 60.30 & 0.13 & 6081.00 & 59168.20 & 0.99 & 0.98 & 0.97 \\
34884 & 106284 & 1999 & 376.30 & 0.72 & 22142.00 & 343632.99 & 1.70 & 0.91 & 1.55 \\
24316 & 103308 & 1999 & 8386.20 & 0.17 & 842369.00 & 7385349.60 & 1.00 & 0.88 & 0.88 \\
23446 & 103177 & 1999 & 256.90 & 0.16 & 25306.00 & 224666.60 & 1.02 & 0.87 & 0.89 \\
2246 & 100299 & 1999 & 42.40 & -0.21 & 4243.00 & 34771.71 & 1.00 & 0.82 & 0.82 \\
30355 & 105741 & 1999 & 95.60 & 0.35 & 9560.00 & 83202.67 & 1.00 & 0.87 & 0.87 \\
34899 & 106292 & 1999 & 11.90 & 0.17 & 1199.00 & 11440.34 & 0.99 & 0.96 & 0.95 \\
30562 & 105770 & 1999 & 27.50 & 0.23 & 2750.00 & 22789.14 & 1.00 & 0.83 & 0.83 \\
2232 & 100298 & 1999 & 838.80 & 0.04 & 83626.00 & 813596.17 & 1.00 & 0.97 & 0.97 \\
1969 & 100263 & 1999 & 7.10 & 0.24 & 708.00 & 6504.36 & 1.00 & 0.92 & 0.92 \\
2220 & 100296 & 1999 & 18.50 & 0.28 & 1812.00 & 19054.57 & 1.02 & 1.03 & 1.05 \\
30552 & 105769 & 1999 & 91.80 & 0.49 & 5692.00 & 56975.25 & 1.61 & 0.62 & 1.00 \\
40122 & 108030 & 1999 & 25.14 & -0.06 & 2772.00 & 20459.28 & 0.91 & 0.81 & 0.74 \\
23477 & 103179 & 1999 & 534.30 & 0.40 & 54431.00 & 516043.58 & 0.98 & 0.97 & 0.95 \\
40127 & 108037 & 1999 & 6.40 & 0.63 & 419.00 & 5959.16 & 1.53 & 0.93 & 1.42 \\
40151 & 108051 & 1999 & 475.50 & 0.13 & 47472.00 & 394779.32 & 1.00 & 0.83 & 0.83 \\
34925 & 106293 & 1999 & 81.30 & 0.34 & 10675.00 & 100480.36 & 0.76 & 1.24 & 0.94 \\
23498 & 103182 & 1999 & 388.80 & 0.11 & 38863.00 & 372373.95 & 1.00 & 0.96 & 0.96 \\
2199 & 100295 & 1999 & 18.30 & 0.19 & 1855.00 & 16653.19 & 0.99 & 0.91 & 0.90 \\
24264 & 103301 & 1999 & 1670.20 & 0.65 & 167857.00 & 1572445.25 & 1.00 & 0.94 & 0.94 \\
2166 & 100293 & 1999 & 206.30 & 0.46 & 21060.00 & 195297.80 & 0.98 & 0.95 & 0.93 \\
30366 & 105742 & 1999 & 81.90 & 0.66 & 8200.00 & 71759.25 & 1.00 & 0.88 & 0.88 \\
30524 & 105763 & 1999 & 189.90 & -0.03 & 20577.00 & 178560.54 & 0.92 & 0.94 & 0.87 \\
23382 & 103174 & 1999 & 2084.40 & 0.09 & 208268.00 & 1934774.87 & 1.00 & 0.93 & 0.93 \\
40815 & 108154 & 1999 & 62.40 & 0.05 & 6186.00 & 54280.38 & 1.01 & 0.87 & 0.88 \\
23689 & 103208 & 1999 & 1733.10 & 0.40 & 174474.00 & 1543743.13 & 0.99 & 0.89 & 0.88 \\
23876 & 103226 & 1999 & 201.30 & 0.44 & 19591.00 & 195981.99 & 1.03 & 0.97 & 1.00 \\
40289 & 108087 & 1999 & 240.30 & 1.68 & 21862.00 & 214392.31 & 1.10 & 0.89 & 0.98 \\
2060 & 100287 & 1999 & 60.00 & 0.23 & 6029.00 & 56300.93 & 1.00 & 0.94 & 0.93 \\
24113 & 103267 & 1999 & 336.40 & 0.43 & 32359.00 & 323559.76 & 1.04 & 0.96 & 1.00 \\
23894 & 103228 & 1999 & 125.70 & 0.88 & 6949.00 & 103092.93 & 1.81 & 0.82 & 1.48 \\
40315 & 108109 & 1999 & 25.40 & 0.01 & 2453.00 & 23601.92 & 1.04 & 0.93 & 0.96 \\
24097 & 103266 & 1999 & 828.60 & 1.33 & 81590.00 & 772506.61 & 1.02 & 0.93 & 0.95 \\
23910 & 103232 & 1999 & 285.90 & 0.33 & 20172.00 & 240612.18 & 1.42 & 0.84 & 1.19 \\
40537 & 108137 & 1999 & 1.30 & 0.30 & 200.00 & 1802.83 & 0.65 & 1.39 & 0.90 \\
23930 & 103242 & 1999 & 78.20 & 0.06 & 7832.00 & 79799.57 & 1.00 & 1.02 & 1.02 \\
40530 & 108136 & 1999 & 13.40 & 0.11 & 924.00 & 11238.44 & 1.45 & 0.84 & 1.22 \\
40505 & 108134 & 1999 & 54.20 & 0.34 & 5442.00 & 48171.55 & 1.00 & 0.89 & 0.89 \\
30415 & 105754 & 1999 & 33.50 & -0.08 & 3165.00 & 31640.43 & 1.06 & 0.94 & 1.00 \\
30430 & 105758 & 1999 & 95.20 & -0.11 & 9512.00 & 92131.14 & 1.00 & 0.97 & 0.97 \\
24085 & 103264 & 1999 & 1206.70 & 0.40 & 120601.00 & 1150806.94 & 1.00 & 0.95 & 0.95 \\
158 & 100016 & 1999 & 383.10 & 0.26 & 38267.00 & 375536.77 & 1.00 & 0.98 & 0.98 \\
23960 & 103251 & 1999 & 515.50 & 0.55 & 51557.00 & 497436.18 & 1.00 & 0.96 & 0.96 \\
40479 & 108122 & 1999 & 24.70 & -0.01 & 2455.00 & 23618.88 & 1.01 & 0.96 & 0.96 \\
40341 & 108112 & 1999 & 32.50 & 0.02 & 3266.00 & 32221.39 & 1.00 & 0.99 & 0.99 \\
23981 & 103252 & 1999 & 262.10 & -0.04 & 26183.00 & 256797.10 & 1.00 & 0.98 & 0.98 \\
40453 & 108121 & 1999 & 214.20 & 0.02 & 21429.00 & 215350.00 & 1.00 & 1.01 & 1.00 \\
24054 & 103259 & 1999 & 1736.00 & -0.01 & 188541.00 & 1790290.61 & 0.92 & 1.03 & 0.95 \\
24002 & 103253 & 1999 & 126.90 & 0.01 & 12674.00 & 123781.94 & 1.00 & 0.98 & 0.98 \\
40427 & 108119 & 1999 & 199.40 & 0.23 & 20004.00 & 189588.33 & 1.00 & 0.95 & 0.95 \\
2034 & 100286 & 1999 & 57.60 & 0.01 & 5708.00 & 52762.29 & 1.01 & 0.92 & 0.92 \\
40419 & 108118 & 1999 & 32.30 & 0.15 & 3129.00 & 31295.92 & 1.03 & 0.97 & 1.00 \\
30440 & 105760 & 1999 & 461.80 & 0.22 & 37845.00 & 432646.40 & 1.22 & 0.94 & 1.14 \\
40232 & 108082 & 1999 & 7.20 & 0.07 & 700.00 & 6995.53 & 1.03 & 0.97 & 1.00 \\
40285 & 108084 & 1999 & 22.40 & 0.20 & 2396.00 & 23984.50 & 0.93 & 1.07 & 1.00 \\
23858 & 103224 & 1999 & 372.20 & 0.84 & 20959.00 & 345082.82 & 1.78 & 0.93 & 1.65 \\
40790 & 108153 & 1999 & 3.70 & 0.01 & 374.00 & 3682.86 & 0.99 & 1.00 & 0.98 \\
40765 & 108149 & 1999 & 3.30 & 0.18 & 449.00 & 4501.98 & 0.73 & 1.36 & 1.00 \\
24163 & 103294 & 1999 & 546.40 & 0.09 & 65100.00 & 549173.25 & 0.84 & 1.01 & 0.84 \\
23723 & 103209 & 1999 & 187.40 & 0.05 & 17317.00 & 168778.62 & 1.08 & 0.90 & 0.97 \\
40736 & 108147 & 1999 & 24.30 & 0.20 & 2589.00 & 22374.77 & 0.94 & 0.92 & 0.86 \\
2098 & 100291 & 1999 & 313.40 & 0.13 & 17636.00 & 168154.41 & 1.78 & 0.54 & 0.95 \\
30386 & 105748 & 1999 & 58.40 & -0.02 & 5899.00 & 56625.81 & 0.99 & 0.97 & 0.96 \\
2095 & 100290 & 1999 & 349.00 & 0.13 & 44167.00 & 436526.41 & 0.79 & 1.25 & 0.99 \\
34988 & 106303 & 1999 & 10.60 & 0.15 & 922.00 & 10654.40 & 1.15 & 1.01 & 1.16 \\
40259 & 108083 & 1999 & 6.40 & -0.02 & 562.00 & 5510.60 & 1.14 & 0.86 & 0.98 \\
23759 & 103212 & 1999 & 2537.20 & 0.33 & 254019.00 & 2212614.11 & 1.00 & 0.87 & 0.87 \\
40686 & 108145 & 1999 & 30.70 & 0.45 & 2448.00 & 24682.88 & 1.25 & 0.80 & 1.01 \\
30468 & 105761 & 1999 & 353.30 & -0.04 & 37654.00 & 333444.42 & 0.94 & 0.94 & 0.89 \\
30394 & 105753 & 1999 & 70.10 & 0.01 & 6843.00 & 68436.06 & 1.02 & 0.98 & 1.00 \\
23795 & 103213 & 1999 & 1663.90 & 0.60 & 163329.00 & 1560191.58 & 1.02 & 0.94 & 0.96 \\
40636 & 108143 & 1999 & 8.40 & -0.19 & 731.00 & 7104.34 & 1.15 & 0.85 & 0.97 \\
35032 & 106309 & 1999 & 63.50 & -0.06 & 6373.00 & 54076.38 & 1.00 & 0.85 & 0.85 \\
34992 & 106305 & 1999 & 33.60 & 0.20 & 3442.00 & 34627.58 & 0.98 & 1.03 & 1.01 \\
35028 & 106307 & 1999 & 26.10 & 0.87 & 2702.00 & 27027.52 & 0.97 & 1.04 & 1.00 \\
23825 & 103214 & 1999 & 3353.70 & 0.75 & 337472.00 & 3175331.69 & 0.99 & 0.95 & 0.94 \\
40586 & 108141 & 1999 & 134.10 & -0.08 & 17546.00 & 159245.05 & 0.76 & 1.19 & 0.91 \\
40577 & 108140 & 1999 & 62.80 & 0.13 & 6131.00 & 61291.82 & 1.02 & 0.98 & 1.00 \\
35019 & 106306 & 1999 & 27.80 & 0.15 & 2787.00 & 27869.15 & 1.00 & 1.00 & 1.00 \\
34864 & 106283 & 1999 & 195.30 & 0.06 & 19126.00 & 191288.75 & 1.02 & 0.98 & 1.00 \\
40035 & 108013 & 1999 & 339.10 & 0.31 & 33972.00 & 332832.25 & 1.00 & 0.98 & 0.98 \\
30761 & 105794 & 1999 & 18.70 & 0.42 & 1898.00 & 16204.89 & 0.99 & 0.87 & 0.85 \\
22891 & 103084 & 1999 & 222.90 & -0.13 & 22292.00 & 216734.33 & 1.00 & 0.97 & 0.97 \\
39864 & 107892 & 1999 & 930.10 & 0.93 & 96543.00 & 847906.81 & 0.96 & 0.91 & 0.88 \\
24644 & 103373 & 1999 & 108.10 & 0.63 & 10911.00 & 98808.64 & 0.99 & 0.91 & 0.91 \\
2476 & 100333 & 1999 & 74.00 & -0.06 & 7146.00 & 71472.97 & 1.04 & 0.97 & 1.00 \\
24615 & 103372 & 1999 & 1025.50 & 0.02 & 107853.00 & 894756.40 & 0.95 & 0.87 & 0.83 \\
142 & 100010 & 1999 & 808.70 & 0.04 & 80903.00 & 758164.00 & 1.00 & 0.94 & 0.94 \\
30751 & 105793 & 1999 & 6174.40 & 1.00 & 610427.00 & 5149774.68 & 1.01 & 0.83 & 0.84 \\
30262 & 105721 & 1999 & 125.60 & -0.01 & 10427.00 & 104247.48 & 1.20 & 0.83 & 1.00 \\
22947 & 103090 & 1999 & 1310.60 & -0.00 & 131106.00 & 1311110.65 & 1.00 & 1.00 & 1.00 \\
41322 & 108728 & 1999 & 5.30 & 0.10 & 519.00 & 5133.75 & 1.02 & 0.97 & 0.99 \\
39893 & 107928 & 1999 & 3313.40 & 0.23 & 331407.00 & 3277265.75 & 1.00 & 0.99 & 0.99 \\
22962 & 103099 & 1999 & 924.90 & 0.20 & 92398.00 & 845943.00 & 1.00 & 0.91 & 0.92 \\
34675 & 106270 & 1999 & 5.90 & 0.15 & 582.00 & 5716.13 & 1.01 & 0.97 & 0.98 \\
22981 & 103100 & 1999 & 610.90 & 0.11 & 61092.00 & 606330.74 & 1.00 & 0.99 & 0.99 \\
41311 & 108726 & 1999 & 18.00 & 0.07 & 1691.00 & 18331.61 & 1.06 & 1.02 & 1.08 \\
30730 & 105791 & 1999 & 1.30 & 0.32 & 107.00 & 1288.26 & 1.21 & 0.99 & 1.20 \\
24592 & 103370 & 1999 & 379.40 & 0.85 & 38522.00 & 371200.36 & 0.98 & 0.98 & 0.96 \\
30265 & 105722 & 1999 & 43.50 & 0.36 & 3224.00 & 40056.23 & 1.35 & 0.92 & 1.24 \\
22999 & 103101 & 1999 & 210.90 & -0.02 & 21100.00 & 207810.29 & 1.00 & 0.99 & 0.98 \\
41301 & 108723 & 1999 & 76.80 & 0.20 & 7017.00 & 70142.41 & 1.09 & 0.91 & 1.00 \\
41276 & 108719 & 1999 & 86.60 & 0.17 & 4787.00 & 71601.92 & 1.81 & 0.83 & 1.50 \\
2440 & 100330 & 1999 & 6146.80 & 0.16 & 613333.00 & 6072198.65 & 1.00 & 0.99 & 0.99 \\
35135 & 106326 & 1999 & 2.70 & 0.26 & 264.00 & 2639.90 & 1.02 & 0.98 & 1.00 \\
1857 & 100245 & 1999 & 721.60 & 0.34 & 60557.00 & 740453.85 & 1.19 & 1.03 & 1.22 \\
23030 & 103103 & 1999 & 446.40 & 0.18 & 44769.00 & 445801.73 & 1.00 & 1.00 & 1.00 \\
39919 & 107938 & 1999 & 142.80 & 0.43 & 14265.00 & 140019.21 & 1.00 & 0.98 & 0.98 \\
34659 & 106268 & 1999 & 162.50 & 0.09 & 16029.00 & 154812.74 & 1.01 & 0.95 & 0.97 \\
24572 & 103369 & 1999 & 203.30 & 0.54 & 20376.00 & 199323.74 & 1.00 & 0.98 & 0.98 \\
39857 & 107883 & 1999 & 2855.80 & -0.03 & 300662.00 & 2817555.75 & 0.95 & 0.99 & 0.94 \\
41370 & 108736 & 1999 & 7.90 & 0.61 & 792.00 & 7517.48 & 1.00 & 0.95 & 0.95 \\
24761 & 103377 & 1999 & 1350.70 & 0.51 & 95906.00 & 1085144.50 & 1.41 & 0.80 & 1.13 \\
2611 & 100347 & 1999 & 1556.20 & 0.20 & 163108.00 & 1433660.93 & 0.95 & 0.92 & 0.88 \\
41483 & 108761 & 1999 & 1.60 & 0.03 & 159.00 & 1530.16 & 1.01 & 0.96 & 0.96 \\
30234 & 105720 & 1999 & 7202.90 & 1.75 & 740112.00 & 6756452.29 & 0.97 & 0.94 & 0.91 \\
39797 & 107874 & 1999 & 61.90 & -0.00 & 6347.00 & 60034.91 & 0.98 & 0.97 & 0.95 \\
39803 & 107875 & 1999 & 194.20 & 0.12 & 23109.00 & 167002.88 & 0.84 & 0.86 & 0.72 \\
24724 & 103376 & 1999 & 9649.30 & 0.35 & 738317.00 & 9051188.98 & 1.31 & 0.94 & 1.23 \\
35168 & 106330 & 1999 & 72.50 & 0.22 & 7245.00 & 69923.77 & 1.00 & 0.96 & 0.97 \\
35142 & 106329 & 1999 & 28.40 & 0.53 & 2670.00 & 26690.42 & 1.06 & 0.94 & 1.00 \\
30783 & 105798 & 1999 & 1827.50 & 1.29 & 182991.00 & 1646462.91 & 1.00 & 0.90 & 0.90 \\
2539 & 100343 & 1999 & 435.60 & -0.03 & 43636.00 & 421496.69 & 1.00 & 0.97 & 0.97 \\
1785 & 100237 & 1999 & 174.60 & 0.12 & 17476.00 & 173130.26 & 1.00 & 0.99 & 0.99 \\
22793 & 103065 & 1999 & 510.50 & 0.71 & 50875.00 & 497300.33 & 1.00 & 0.97 & 0.98 \\
41434 & 108759 & 1999 & 5.40 & 0.07 & 517.00 & 5164.79 & 1.04 & 0.96 & 1.00 \\
41422 & 108752 & 1999 & 5.80 & 0.27 & 580.00 & 5763.13 & 1.00 & 0.99 & 0.99 \\
41397 & 108749 & 1999 & 1.50 & 0.08 & 141.00 & 1406.69 & 1.06 & 0.94 & 1.00 \\
34653 & 106267 & 1999 & 10.70 & 0.05 & 1094.00 & 10942.04 & 0.98 & 1.02 & 1.00 \\
22828 & 103067 & 1999 & 164.00 & 0.33 & 16394.00 & 159789.40 & 1.00 & 0.97 & 0.97 \\
41381 & 108742 & 1999 & 2.30 & 0.17 & 226.00 & 2185.06 & 1.02 & 0.95 & 0.97 \\
1823 & 100244 & 1999 & 300.20 & 1.11 & 30172.00 & 285466.72 & 0.99 & 0.95 & 0.95 \\
22860 & 103073 & 1999 & 898.00 & -0.09 & 86322.00 & 829685.64 & 1.04 & 0.92 & 0.96 \\
2505 & 100336 & 1999 & 29.00 & 0.05 & 2975.00 & 27920.91 & 0.97 & 0.96 & 0.94 \\
24370 & 103318 & 1999 & 1895.60 & 0.24 & 189705.00 & 1703449.84 & 1.00 & 0.90 & 0.90 \\
34708 & 106272 & 1999 & 1689.50 & 0.20 & 169068.00 & 1654861.06 & 1.00 & 0.98 & 0.98 \\
34791 & 106277 & 1999 & 28.80 & 0.04 & 2806.00 & 28067.92 & 1.03 & 0.97 & 1.00 \\
24450 & 103327 & 1999 & 1889.40 & 0.06 & 195256.00 & 1932137.39 & 0.97 & 1.02 & 0.99 \\
23249 & 103152 & 1999 & 2944.60 & 0.19 & 278597.00 & 2858854.24 & 1.06 & 0.97 & 1.03 \\
23281 & 103154 & 1999 & 986.10 & 0.32 & 91214.00 & 912190.29 & 1.08 & 0.93 & 1.00 \\
30651 & 105781 & 1999 & 174.50 & 0.11 & 17481.00 & 171678.12 & 1.00 & 0.98 & 0.98 \\
23296 & 103158 & 1999 & 3419.80 & 0.19 & 332613.00 & 3326606.55 & 1.03 & 0.97 & 1.00 \\
2333 & 100319 & 1999 & 116.00 & -0.07 & 11612.00 & 110876.81 & 1.00 & 0.96 & 0.95 \\
39999 & 107992 & 1999 & 63.20 & 0.34 & 6369.00 & 62314.47 & 0.99 & 0.99 & 0.98 \\
23308 & 103160 & 1999 & 133.30 & 0.52 & 12785.00 & 122432.61 & 1.04 & 0.92 & 0.96 \\
30642 & 105780 & 1999 & 259.80 & 0.22 & 25723.00 & 243229.33 & 1.01 & 0.94 & 0.95 \\
40007 & 107994 & 1999 & 1355.60 & 1.90 & 137977.00 & 1150210.08 & 0.98 & 0.85 & 0.83 \\
2313 & 100315 & 1999 & 500.30 & 0.13 & 50988.00 & 490082.72 & 0.98 & 0.98 & 0.96 \\
34814 & 106278 & 1999 & 138.50 & 0.19 & 13810.00 & 127689.65 & 1.00 & 0.92 & 0.92 \\
30322 & 105737 & 1999 & 13.30 & 0.19 & 1525.00 & 13029.20 & 0.87 & 0.98 & 0.85 \\
35074 & 106318 & 1999 & 143.90 & 0.57 & 14397.00 & 139632.40 & 1.00 & 0.97 & 0.97 \\
34826 & 106281 & 1999 & 23.70 & 0.19 & 2411.00 & 21604.22 & 0.98 & 0.91 & 0.90 \\
1934 & 100259 & 1999 & 178.20 & -0.04 & 12123.00 & 121280.68 & 1.47 & 0.68 & 1.00 \\
24404 & 103319 & 1999 & 401.70 & 0.18 & 40155.00 & 392566.99 & 1.00 & 0.98 & 0.98 \\
23337 & 103165 & 1999 & 3.20 & 0.55 & 315.00 & 2954.07 & 1.02 & 0.92 & 0.94 \\
40020 & 108009 & 1999 & 14.60 & 0.50 & 1221.00 & 14910.50 & 1.20 & 1.02 & 1.22 \\
23350 & 103166 & 1999 & 6.20 & 0.42 & 612.00 & 6116.44 & 1.01 & 0.99 & 1.00 \\
35065 & 106317 & 1999 & 391.80 & 0.24 & 37937.00 & 379512.31 & 1.03 & 0.97 & 1.00 \\
2272 & 100305 & 1999 & 102.20 & 0.40 & 10283.00 & 102878.58 & 0.99 & 1.01 & 1.00 \\
34837 & 106282 & 1999 & 449.00 & 0.26 & 44801.00 & 415552.11 & 1.00 & 0.93 & 0.93 \\
39989 & 107968 & 1999 & 91.50 & 0.14 & 9166.00 & 86669.59 & 1.00 & 0.95 & 0.95 \\
30709 & 105788 & 1999 & 11.20 & 0.09 & 1109.00 & 11071.15 & 1.01 & 0.99 & 1.00 \\
35089 & 106320 & 1999 & 1578.90 & 0.46 & 124225.00 & 1373760.55 & 1.27 & 0.87 & 1.11 \\
23216 & 103145 & 1999 & 116.40 & -0.00 & 17961.00 & 184621.44 & 0.65 & 1.59 & 1.03 \\
24560 & 103366 & 1999 & 250.20 & 0.10 & 24703.00 & 243039.01 & 1.01 & 0.97 & 0.98 \\
41236 & 108690 & 1999 & 71.80 & -0.10 & 7401.00 & 65744.97 & 0.97 & 0.92 & 0.89 \\
2417 & 100323 & 1999 & 5283.00 & 0.25 & 528289.00 & 4886734.31 & 1.00 & 0.92 & 0.93 \\
24546 & 103347 & 1999 & 7.30 & 0.73 & 515.00 & 7357.65 & 1.42 & 1.01 & 1.43 \\
30288 & 105724 & 1999 & 39.60 & 0.18 & 3961.00 & 37584.39 & 1.00 & 0.95 & 0.95 \\
23073 & 103110 & 1999 & 1183.90 & 0.00 & 114397.00 & 1123976.85 & 1.03 & 0.95 & 0.98 \\
41209 & 108673 & 1999 & 67.30 & 0.08 & 6725.00 & 66269.44 & 1.00 & 0.98 & 0.99 \\
23101 & 103122 & 1999 & 790.80 & 0.72 & 79080.00 & 761726.52 & 1.00 & 0.96 & 0.96 \\
41184 & 108670 & 1999 & 570.70 & 0.25 & 57936.00 & 561267.94 & 0.99 & 0.98 & 0.97 \\
34737 & 106275 & 1999 & 27.70 & 0.23 & 2686.00 & 26853.09 & 1.03 & 0.97 & 1.00 \\
2397 & 100322 & 1999 & 823.40 & 0.01 & 81757.00 & 790590.06 & 1.01 & 0.96 & 0.97 \\
30293 & 105731 & 1999 & 819.70 & 0.37 & 79259.00 & 699800.39 & 1.03 & 0.85 & 0.88 \\
34764 & 106276 & 1999 & 69.80 & 1.11 & 6840.00 & 61251.90 & 1.02 & 0.88 & 0.90 \\
39930 & 107958 & 1999 & 10.10 & -0.01 & 1057.00 & 10290.48 & 0.96 & 1.02 & 0.97 \\
23141 & 103134 & 1999 & 769.60 & 0.19 & 79278.00 & 746064.84 & 0.97 & 0.97 & 0.94 \\
30682 & 105783 & 1999 & 875.40 & -0.04 & 97596.00 & 867352.96 & 0.90 & 0.99 & 0.89 \\
35115 & 106321 & 1999 & 5.10 & 0.28 & 370.00 & 4227.17 & 1.38 & 0.83 & 1.14 \\
39955 & 107960 & 1999 & 12.00 & 0.06 & 1172.00 & 11719.81 & 1.02 & 0.98 & 1.00 \\
24502 & 103329 & 1999 & 725.00 & -0.04 & 72144.00 & 702174.72 & 1.00 & 0.97 & 0.97 \\
23157 & 103136 & 1999 & 502.50 & 0.21 & 50201.00 & 493404.49 & 1.00 & 0.98 & 0.98 \\
39971 & 107964 & 1999 & 296.90 & 0.08 & 20108.00 & 205042.70 & 1.48 & 0.69 & 1.02 \\
23168 & 103138 & 1999 & 2852.20 & 0.21 & 482753.00 & 4647822.95 & 0.59 & 1.63 & 0.96 \\
30677 & 105782 & 1999 & 46.10 & 0.02 & 4598.00 & 45531.81 & 1.00 & 0.99 & 0.99 \\
39979 & 107967 & 1999 & 186.20 & 0.20 & 18626.00 & 181079.98 & 1.00 & 0.97 & 0.97 \\
24472 & 103328 & 1999 & 1506.20 & 0.50 & 110968.00 & 1490988.43 & 1.36 & 0.99 & 1.34 \\
2365 & 100320 & 1999 & 63.30 & 0.06 & 6382.00 & 63799.21 & 0.99 & 1.01 & 1.00 \\
31790 & 105936 & 1999 & 146.90 & 0.20 & 12044.00 & 127551.17 & 1.22 & 0.87 & 1.06 \\
21360 & 102854 & 1999 & 673.90 & 0.28 & 67648.00 & 664524.82 & 1.00 & 0.99 & 0.98 \\
38113 & 107202 & 1999 & 9.90 & 0.09 & 996.00 & 9689.16 & 0.99 & 0.98 & 0.97 \\
19400 & 102600 & 1999 & 984.90 & 0.00 & 98497.00 & 956893.95 & 1.00 & 0.97 & 0.97 \\
4110 & 100552 & 1999 & 343.50 & 0.49 & 34303.00 & 337488.90 & 1.00 & 0.98 & 0.98 \\
31528 & 105905 & 1999 & 28.10 & 0.18 & 2505.00 & 29396.96 & 1.12 & 1.05 & 1.17 \\
19434 & 102601 & 1999 & 7399.50 & 0.15 & 739948.00 & 7298053.58 & 1.00 & 0.99 & 0.99 \\
26693 & 103600 & 1999 & 913.00 & 0.37 & 90892.00 & 804478.20 & 1.00 & 0.88 & 0.89 \\
38184 & 107215 & 1999 & 137.40 & 0.04 & 16921.00 & 147668.73 & 0.81 & 1.07 & 0.87 \\
38209 & 107222 & 1999 & 68.30 & 0.05 & 6578.00 & 63508.47 & 1.04 & 0.93 & 0.97 \\
19468 & 102606 & 1999 & 5101.50 & 0.01 & 510151.00 & 4986444.70 & 1.00 & 0.98 & 0.98 \\
38230 & 107224 & 1999 & 17.40 & 0.50 & 1716.00 & 15887.33 & 1.01 & 0.91 & 0.93 \\
802 & 100097 & 1999 & 193.30 & 0.89 & 19331.00 & 166753.27 & 1.00 & 0.86 & 0.86 \\
26663 & 103597 & 1999 & 2679.80 & 0.21 & 236164.00 & 1955216.87 & 1.13 & 0.73 & 0.83 \\
19508 & 102608 & 1999 & 80.90 & -0.05 & 8092.00 & 77707.58 & 1.00 & 0.96 & 0.96 \\
34051 & 106199 & 1999 & 148.50 & -0.05 & 12125.00 & 115877.62 & 1.22 & 0.78 & 0.96 \\
29792 & 105647 & 1999 & 167.50 & 0.33 & 16772.00 & 150084.83 & 1.00 & 0.90 & 0.89 \\
19554 & 102624 & 1999 & 2047.20 & 0.20 & 204719.00 & 2005206.11 & 1.00 & 0.98 & 0.98 \\
4062 & 100544 & 1999 & 371.80 & 0.03 & 37183.00 & 365705.82 & 1.00 & 0.98 & 0.98 \\
26649 & 103595 & 1999 & 183.40 & 0.38 & 17745.00 & 177445.44 & 1.03 & 0.97 & 1.00 \\
35480 & 106364 & 1999 & 5.90 & 0.20 & 592.00 & 5758.07 & 1.00 & 0.98 & 0.97 \\
38240 & 107226 & 1999 & 5.90 & 0.11 & 620.00 & 5792.02 & 0.95 & 0.98 & 0.93 \\
38265 & 107227 & 1999 & 36.20 & 0.25 & 3666.00 & 33832.30 & 0.99 & 0.93 & 0.92 \\
26617 & 103593 & 1999 & 53135.70 & 0.01 & 4907497.00 & 51492319.79 & 1.08 & 0.97 & 1.05 \\
34059 & 106200 & 1999 & 744.70 & 0.37 & 56188.00 & 676805.78 & 1.33 & 0.91 & 1.20 \\
35472 & 106363 & 1999 & 573.60 & 0.39 & 57354.00 & 569860.75 & 1.00 & 0.99 & 0.99 \\
19491 & 102607 & 1999 & 965.20 & 0.30 & 96525.00 & 914135.33 & 1.00 & 0.95 & 0.95 \\
34024 & 106198 & 1999 & 308.80 & -0.03 & 30680.00 & 305396.91 & 1.01 & 0.99 & 1.00 \\
4129 & 100559 & 1999 & 85.00 & 0.30 & 8496.00 & 84159.46 & 1.00 & 0.99 & 0.99 \\
19366 & 102599 & 1999 & 1547.90 & 0.12 & 154798.00 & 1467253.48 & 1.00 & 0.95 & 0.95 \\
38009 & 107186 & 1999 & 3.00 & 0.01 & 353.00 & 3524.82 & 0.85 & 1.17 & 1.00 \\
19221 & 102570 & 1999 & 578.80 & 0.83 & 57889.00 & 525176.17 & 1.00 & 0.91 & 0.91 \\
40393 & 108117 & 1999 & 200.80 & 0.98 & 20079.00 & 184130.16 & 1.00 & 0.92 & 0.92 \\
26843 & 103609 & 1999 & 68.90 & -0.08 & 7546.00 & 65806.69 & 0.91 & 0.96 & 0.87 \\
33970 & 106195 & 1999 & 9.00 & -0.22 & 896.00 & 7680.65 & 1.00 & 0.85 & 0.86 \\
38015 & 107192 & 1999 & 178.20 & 0.26 & 14624.00 & 138848.49 & 1.22 & 0.78 & 0.95 \\
38034 & 107193 & 1999 & 13.50 & 0.97 & 1345.00 & 13446.14 & 1.00 & 1.00 & 1.00 \\
26820 & 103608 & 1999 & 60.40 & 0.05 & 5986.00 & 56914.99 & 1.01 & 0.94 & 0.95 \\
19253 & 102575 & 1999 & 359.90 & 0.89 & 25954.00 & 347583.04 & 1.39 & 0.97 & 1.34 \\
35498 & 106367 & 1999 & 144.30 & 0.21 & 14789.00 & 141046.26 & 0.98 & 0.98 & 0.95 \\
19269 & 102578 & 1999 & 213.60 & 0.35 & 20801.00 & 208089.00 & 1.03 & 0.97 & 1.00 \\
19281 & 102579 & 1999 & 953.10 & 0.48 & 95278.00 & 920512.72 & 1.00 & 0.97 & 0.97 \\
38038 & 107196 & 1999 & 6.10 & 0.15 & 608.00 & 5964.47 & 1.00 & 0.98 & 0.98 \\
38063 & 107198 & 1999 & 46.80 & 0.08 & 4594.00 & 45938.83 & 1.02 & 0.98 & 1.00 \\
19301 & 102588 & 1999 & 466.10 & -0.19 & 46222.00 & 460546.64 & 1.01 & 0.99 & 1.00 \\
4175 & 100567 & 1999 & 3330.20 & 0.29 & 333027.00 & 3238857.61 & 1.00 & 0.97 & 0.97 \\
26793 & 103607 & 1999 & 300.00 & 0.59 & 35415.00 & 296012.31 & 0.85 & 0.99 & 0.84 \\
31560 & 105909 & 1999 & 66.20 & -0.02 & 6462.00 & 64866.47 & 1.02 & 0.98 & 1.00 \\
35493 & 106366 & 1999 & 12.00 & 0.98 & 1154.00 & 11538.61 & 1.04 & 0.96 & 1.00 \\
38079 & 107199 & 1999 & 40.40 & 0.06 & 4004.00 & 40049.60 & 1.01 & 0.99 & 1.00 \\
33997 & 106197 & 1999 & 48.50 & 0.06 & 4805.00 & 46232.29 & 1.01 & 0.95 & 0.96 \\
26761 & 103606 & 1999 & 15.80 & 0.46 & 1603.00 & 15208.13 & 0.99 & 0.96 & 0.95 \\
73601 & 600473 & 1999 & 96.60 & 0.12 & 10297.00 & 102177.19 & 0.94 & 1.06 & 0.99 \\
35487 & 106365 & 1999 & 14.60 & 0.03 & 1465.00 & 14331.93 & 1.00 & 0.98 & 0.98 \\
19575 & 102633 & 1999 & 616.20 & -0.05 & 61625.00 & 592382.61 & 1.00 & 0.96 & 0.96 \\
31501 & 105903 & 1999 & 24.00 & 0.33 & 2417.00 & 23949.30 & 0.99 & 1.00 & 0.99 \\
832 & 100098 & 1999 & 36.80 & 0.42 & 3675.00 & 30848.28 & 1.00 & 0.84 & 0.84 \\
19593 & 102635 & 1999 & 812.50 & 0.02 & 73129.00 & 659768.61 & 1.11 & 0.81 & 0.90 \\
38397 & 107257 & 1999 & 375.50 & 0.10 & 37537.00 & 358283.80 & 1.00 & 0.95 & 0.95 \\
31448 & 105890 & 1999 & 73.40 & 0.85 & 7346.00 & 69957.78 & 1.00 & 0.95 & 0.95 \\
3947 & 100521 & 1999 & 251.20 & 0.23 & 25544.00 & 251478.73 & 0.98 & 1.00 & 0.98 \\
26483 & 103582 & 1999 & 22.10 & 0.63 & 2207.00 & 19454.70 & 1.00 & 0.88 & 0.88 \\
19839 & 102653 & 1999 & 5672.30 & 0.22 & 566730.00 & 4764014.34 & 1.00 & 0.84 & 0.84 \\
19877 & 102654 & 1999 & 2063.50 & 0.13 & 206844.00 & 1823222.51 & 1.00 & 0.88 & 0.88 \\
3917 & 100514 & 1999 & 91.50 & 0.23 & 9154.00 & 91549.63 & 1.00 & 1.00 & 1.00 \\
862 & 100099 & 1999 & 91.10 & 0.52 & 9109.00 & 84734.20 & 1.00 & 0.93 & 0.93 \\
38422 & 107258 & 1999 & 137.00 & 0.13 & 13070.00 & 130704.47 & 1.05 & 0.95 & 1.00 \\
38431 & 107259 & 1999 & 35.20 & 0.04 & 3196.00 & 31958.41 & 1.10 & 0.91 & 1.00 \\
26432 & 103580 & 1999 & 556.40 & 0.13 & 56561.00 & 494876.84 & 0.98 & 0.89 & 0.87 \\
29841 & 105653 & 1999 & 7.30 & 0.30 & 732.00 & 7321.25 & 1.00 & 1.00 & 1.00 \\
31438 & 105886 & 1999 & 65.50 & 0.31 & 6554.00 & 63467.88 & 1.00 & 0.97 & 0.97 \\
38456 & 107260 & 1999 & 2.60 & -0.03 & 255.00 & 2410.08 & 1.02 & 0.93 & 0.95 \\
19921 & 102655 & 1999 & 1082.30 & 0.21 & 108409.00 & 956710.35 & 1.00 & 0.88 & 0.88 \\
3877 & 100508 & 1999 & 7.30 & -0.02 & 659.00 & 6249.64 & 1.11 & 0.86 & 0.95 \\
38481 & 107263 & 1999 & 780.80 & 0.11 & 78063.00 & 755773.79 & 1.00 & 0.97 & 0.97 \\
26400 & 103579 & 1999 & 730.10 & 0.62 & 74748.00 & 634098.86 & 0.98 & 0.87 & 0.85 \\
3869 & 100507 & 1999 & 17.70 & 0.17 & 1570.00 & 15242.98 & 1.13 & 0.86 & 0.97 \\
31429 & 105883 & 1999 & 51.00 & 0.20 & 6128.00 & 41376.03 & 0.83 & 0.81 & 0.68 \\
3861 & 100506 & 1999 & 8.30 & 0.17 & 847.00 & 8307.47 & 0.98 & 1.00 & 0.98 \\
38505 & 107266 & 1999 & 309.80 & 0.01 & 30091.00 & 304844.36 & 1.03 & 0.98 & 1.01 \\
29847 & 105654 & 1999 & 158.20 & 0.46 & 15217.00 & 152234.59 & 1.04 & 0.96 & 1.00 \\
35423 & 106360 & 1999 & 118.60 & -0.03 & 13336.00 & 129455.36 & 0.89 & 1.09 & 0.97 \\
892 & 100101 & 1999 & 1174.20 & 0.73 & 117423.00 & 1092003.80 & 1.00 & 0.93 & 0.93 \\
19949 & 102659 & 1999 & 6021.70 & 0.33 & 602357.00 & 5621673.13 & 1.00 & 0.93 & 0.93 \\
3971 & 100535 & 1999 & 309.80 & -0.12 & 30964.00 & 287680.04 & 1.00 & 0.93 & 0.93 \\
33954 & 106193 & 1999 & 40.50 & 0.35 & 3083.00 & 35934.50 & 1.31 & 0.89 & 1.17 \\
19798 & 102652 & 1999 & 3439.50 & 0.04 & 344486.00 & 3266648.10 & 1.00 & 0.95 & 0.95 \\
19764 & 102651 & 1999 & 5049.10 & 0.21 & 505890.00 & 4994571.11 & 1.00 & 0.99 & 0.99 \\
4048 & 100543 & 1999 & 632.50 & 0.08 & 63609.00 & 631902.86 & 0.99 & 1.00 & 0.99 \\
38271 & 107234 & 1999 & 17.90 & 0.05 & 1767.00 & 17669.30 & 1.01 & 0.99 & 1.00 \\
26583 & 103592 & 1999 & 43.00 & 0.17 & 4404.00 & 38407.99 & 0.98 & 0.89 & 0.87 \\
19603 & 102636 & 1999 & 488.10 & 0.02 & 48343.00 & 474380.58 & 1.01 & 0.97 & 0.98 \\
31495 & 105900 & 1999 & 40.30 & 0.08 & 3979.00 & 37642.34 & 1.01 & 0.93 & 0.95 \\
19635 & 102639 & 1999 & 95.30 & -0.01 & 9527.00 & 87864.07 & 1.00 & 0.92 & 0.92 \\
38291 & 107242 & 1999 & 90.30 & 0.01 & 8877.00 & 85253.76 & 1.02 & 0.94 & 0.96 \\
38316 & 107243 & 1999 & 760.50 & 0.11 & 64408.00 & 619047.81 & 1.18 & 0.81 & 0.96 \\
38341 & 107244 & 1999 & 461.10 & 0.19 & 42430.00 & 438582.33 & 1.09 & 0.95 & 1.03 \\
26552 & 103591 & 1999 & 211.30 & 0.19 & 21257.00 & 196653.21 & 0.99 & 0.93 & 0.93 \\
19656 & 102641 & 1999 & 818.80 & 0.45 & 81846.00 & 793203.26 & 1.00 & 0.97 & 0.97 \\
73390 & 600012 & 1999 & 603.90 & 2.12 & 61623.00 & 493290.56 & 0.98 & 0.82 & 0.80 \\
19669 & 102645 & 1999 & 382.60 & 0.04 & 38021.00 & 362730.87 & 1.01 & 0.95 & 0.95 \\
65 & 100004 & 1999 & 2238.40 & 0.35 & 223978.00 & 2147706.16 & 1.00 & 0.96 & 0.96 \\
38366 & 107246 & 1999 & 347.10 & 0.37 & 35354.00 & 284072.47 & 0.98 & 0.82 & 0.80 \\
38372 & 107248 & 1999 & 275.10 & 0.04 & 35058.00 & 249110.65 & 0.78 & 0.91 & 0.71 \\
29821 & 105652 & 1999 & 821.60 & 0.28 & 81449.00 & 809389.67 & 1.01 & 0.99 & 0.99 \\
19699 & 102649 & 1999 & 1028.00 & -0.03 & 102814.00 & 854973.51 & 1.00 & 0.83 & 0.83 \\
4014 & 100538 & 1999 & 691.30 & -0.07 & 79098.00 & 700259.24 & 0.87 & 1.01 & 0.89 \\
34093 & 106207 & 1999 & 20.20 & 1.33 & 1440.00 & 18874.91 & 1.40 & 0.93 & 1.31 \\
35450 & 106361 & 1999 & 101.20 & 0.48 & 9538.00 & 95382.80 & 1.06 & 0.94 & 1.00 \\
34109 & 106208 & 1999 & 23.30 & -0.17 & 3064.00 & 24613.48 & 0.76 & 1.06 & 0.80 \\
19733 & 102650 & 1999 & 12321.20 & 0.04 & 1240760.00 & 11602549.10 & 0.99 & 0.94 & 0.94 \\
31465 & 105895 & 1999 & 304.70 & 1.50 & 30414.00 & 257955.81 & 1.00 & 0.85 & 0.85 \\
73362 & 600006 & 1999 & 103.30 & 0.57 & 10338.00 & 102768.60 & 1.00 & 0.99 & 0.99 \\
26859 & 103614 & 1999 & 171.80 & 0.15 & 16707.00 & 167035.29 & 1.03 & 0.97 & 1.00 \\
38005 & 107185 & 1999 & 52.70 & 0.11 & 5277.00 & 51762.88 & 1.00 & 0.98 & 0.98 \\
37995 & 107181 & 1999 & 53.80 & 0.05 & 4833.00 & 48765.64 & 1.11 & 0.91 & 1.01 \\
37773 & 107143 & 1999 & 4.10 & 0.06 & 408.00 & 3870.71 & 1.00 & 0.94 & 0.95 \\
33803 & 106172 & 1999 & 76.20 & 0.31 & 6077.00 & 71488.27 & 1.25 & 0.94 & 1.18 \\
31724 & 105932 & 1999 & 212.60 & 0.08 & 21492.00 & 193522.13 & 0.99 & 0.91 & 0.90 \\
37776 & 107144 & 1999 & 214.80 & 0.35 & 25508.00 & 203327.05 & 0.84 & 0.95 & 0.80 \\
37824 & 107147 & 1999 & 106.20 & 0.12 & 8061.00 & 79764.16 & 1.32 & 0.75 & 0.99 \\
27131 & 105246 & 1999 & 1572.00 & 0.14 & 183813.00 & 1511512.48 & 0.86 & 0.96 & 0.82 \\
59032 & 410401 & 1999 & 80.20 & 0.25 & 8038.00 & 71744.76 & 1.00 & 0.89 & 0.89 \\
18694 & 102503 & 1999 & 363.80 & 0.48 & 45511.00 & 357759.35 & 0.80 & 0.98 & 0.79 \\
27128 & 105243 & 1999 & 8.50 & -0.08 & 839.00 & 8380.08 & 1.01 & 0.99 & 1.00 \\
18721 & 102504 & 1999 & 45.60 & 0.23 & 4600.00 & 38344.45 & 0.99 & 0.84 & 0.83 \\
33830 & 106173 & 1999 & 187.00 & 0.05 & 20185.00 & 211585.84 & 0.93 & 1.13 & 1.05 \\
18756 & 102507 & 1999 & 784.50 & 0.49 & 78443.00 & 763641.38 & 1.00 & 0.97 & 0.97 \\
29681 & 105635 & 1999 & 307.20 & 0.51 & 30643.00 & 302863.46 & 1.00 & 0.99 & 0.99 \\
35578 & 106376 & 1999 & 8.60 & 0.25 & 689.00 & 8703.64 & 1.25 & 1.01 & 1.26 \\
18774 & 102508 & 1999 & 530.60 & 0.32 & 53059.00 & 515868.51 & 1.00 & 0.97 & 0.97 \\
4399 & 100622 & 1999 & 539.10 & 0.04 & 53928.00 & 507791.42 & 1.00 & 0.94 & 0.94 \\
684 & 100090 & 1999 & 1156.50 & 0.33 & 115651.00 & 1102902.16 & 1.00 & 0.95 & 0.95 \\
4381 & 100614 & 1999 & 774.40 & 0.09 & 89684.00 & 744929.27 & 0.86 & 0.96 & 0.83 \\
35553 & 106375 & 1999 & 30.60 & 0.38 & 3060.00 & 27661.16 & 1.00 & 0.90 & 0.90 \\
74613 & 601140 & 1999 & 11.30 & 0.05 & 1114.00 & 10981.95 & 1.01 & 0.97 & 0.99 \\
29702 & 105640 & 1999 & 22.40 & 0.04 & 2145.00 & 19523.51 & 1.04 & 0.87 & 0.91 \\
59107 & 410433 & 1999 & 301.70 & 0.05 & 27065.00 & 271861.02 & 1.11 & 0.90 & 1.00 \\
33872 & 106179 & 1999 & 91.10 & 0.25 & 9106.00 & 90475.32 & 1.00 & 0.99 & 0.99 \\
37862 & 107152 & 1999 & 98.40 & 0.02 & 9474.00 & 102284.29 & 1.04 & 1.04 & 1.08 \\
37710 & 107004 & 1999 & 284.90 & -0.11 & 28514.00 & 274582.66 & 1.00 & 0.96 & 0.96 \\
18563 & 102483 & 1999 & 71.50 & 0.02 & 8872.00 & 87948.24 & 0.81 & 1.23 & 0.99 \\
37717 & 107135 & 1999 & 25.00 & 0.12 & 2509.00 & 25001.41 & 1.00 & 1.00 & 1.00 \\
27218 & 105256 & 1999 & 19.40 & 0.11 & 1928.00 & 19234.73 & 1.01 & 0.99 & 1.00 \\
18570 & 102486 & 1999 & 31.70 & 0.03 & 3192.00 & 30803.81 & 0.99 & 0.97 & 0.97 \\
33751 & 106167 & 1999 & 99.30 & 0.35 & 9932.00 & 88827.45 & 1.00 & 0.89 & 0.89 \\
18577 & 102489 & 1999 & 242.60 & 0.14 & 24115.00 & 241226.18 & 1.01 & 0.99 & 1.00 \\
27209 & 105253 & 1999 & 29.60 & 0.12 & 2909.00 & 29092.24 & 1.02 & 0.98 & 1.00 \\
18589 & 102490 & 1999 & 105.10 & 0.15 & 10497.00 & 104968.64 & 1.00 & 1.00 & 1.00 \\
4488 & 100635 & 1999 & 241.70 & -0.02 & 24266.00 & 228388.75 & 1.00 & 0.94 & 0.94 \\
262 & 100022 & 1999 & 41.30 & -0.05 & 4078.00 & 42205.77 & 1.01 & 1.02 & 1.03 \\
18605 & 102491 & 1999 & 241.40 & 0.26 & 24146.00 & 235009.05 & 1.00 & 0.97 & 0.97 \\
656 & 100087 & 1999 & 11361.30 & 0.34 & 1136134.00 & 10512594.37 & 1.00 & 0.93 & 0.93 \\
27200 & 105252 & 1999 & 37.30 & 0.56 & 3805.00 & 38063.14 & 0.98 & 1.02 & 1.00 \\
33763 & 106169 & 1999 & 6.90 & 0.17 & 697.00 & 6471.12 & 0.99 & 0.94 & 0.93 \\
31762 & 105935 & 1999 & 735.60 & 0.18 & 73564.00 & 727931.63 & 1.00 & 0.99 & 0.99 \\
18639 & 102493 & 1999 & 2345.30 & 0.13 & 233654.00 & 2314859.88 & 1.00 & 0.99 & 0.99 \\
4477 & 100634 & 1999 & 2160.70 & 0.26 & 217730.00 & 2082519.78 & 0.99 & 0.96 & 0.96 \\
27190 & 105250 & 1999 & 9.40 & 0.05 & 1040.00 & 10713.07 & 0.90 & 1.14 & 1.03 \\
74627 & 601142 & 1999 & 430.70 & 0.97 & 43562.00 & 420647.77 & 0.99 & 0.98 & 0.97 \\
4452 & 100633 & 1999 & 959.90 & 0.26 & 96338.00 & 931228.16 & 1.00 & 0.97 & 0.97 \\
37741 & 107136 & 1999 & 42.20 & 0.29 & 4216.00 & 37957.53 & 1.00 & 0.90 & 0.90 \\
29657 & 105631 & 1999 & 14.30 & 0.58 & 1079.00 & 14090.40 & 1.33 & 0.99 & 1.31 \\
37743 & 107137 & 1999 & 15.50 & 0.04 & 1531.00 & 15317.05 & 1.01 & 0.99 & 1.00 \\
33790 & 106170 & 1999 & 116.50 & 0.00 & 11630.00 & 114188.77 & 1.00 & 0.98 & 0.98 \\
37748 & 107141 & 1999 & 23.50 & 0.35 & 2346.00 & 22976.05 & 1.00 & 0.98 & 0.98 \\
18673 & 102501 & 1999 & 165.00 & -0.01 & 28393.00 & 238257.23 & 0.58 & 1.44 & 0.84 \\
34136 & 106209 & 1999 & 26.60 & -0.05 & 3259.00 & 29303.92 & 0.82 & 1.10 & 0.90 \\
18817 & 102523 & 1999 & 500.60 & 0.06 & 54996.00 & 552401.83 & 0.91 & 1.10 & 1.00 \\
27068 & 103647 & 1999 & 18.70 & 0.27 & 2190.00 & 16581.16 & 0.85 & 0.89 & 0.76 \\
19062 & 102546 & 1999 & 70.70 & 0.17 & 7067.00 & 69799.20 & 1.00 & 0.99 & 0.99 \\
19070 & 102547 & 1999 & 156.10 & -0.02 & 15395.00 & 153980.69 & 1.01 & 0.99 & 1.00 \\
742 & 100093 & 1999 & 574.50 & 0.32 & 57449.00 & 499068.57 & 1.00 & 0.87 & 0.87 \\
19078 & 102548 & 1999 & 1025.80 & 0.12 & 104451.00 & 1014879.33 & 0.98 & 0.99 & 0.97 \\
26961 & 103638 & 1999 & 76.50 & 0.14 & 7645.00 & 75923.35 & 1.00 & 0.99 & 0.99 \\
19106 & 102549 & 1999 & 218.90 & 0.05 & 21807.00 & 212044.44 & 1.00 & 0.97 & 0.97 \\
37956 & 107174 & 1999 & 36.90 & 0.10 & 3662.00 & 34405.60 & 1.01 & 0.93 & 0.94 \\
19137 & 102550 & 1999 & 67.90 & 0.06 & 6776.00 & 67446.60 & 1.00 & 0.99 & 1.00 \\
33927 & 106192 & 1999 & 2311.50 & 0.25 & 145273.00 & 1482372.72 & 1.59 & 0.64 & 1.02 \\
31620 & 105918 & 1999 & 2575.40 & 1.47 & 256156.00 & 2376056.12 & 1.01 & 0.92 & 0.93 \\
26926 & 103628 & 1999 & 1160.20 & 0.10 & 113007.00 & 1047141.04 & 1.03 & 0.90 & 0.93 \\
31610 & 105917 & 1999 & 18.10 & 0.00 & 2016.00 & 20144.75 & 0.90 & 1.11 & 1.00 \\
26913 & 103621 & 1999 & 198.30 & 0.27 & 19769.00 & 197722.28 & 1.00 & 1.00 & 1.00 \\
19144 & 102551 & 1999 & 319.90 & -0.07 & 31724.00 & 315448.71 & 1.01 & 0.99 & 0.99 \\
74437 & 601001 & 1999 & 9.30 & 0.06 & 1049.00 & 9071.80 & 0.89 & 0.98 & 0.86 \\
19178 & 102559 & 1999 & 105.70 & 0.01 & 10479.00 & 91427.32 & 1.01 & 0.86 & 0.87 \\
31604 & 105916 & 1999 & 230.90 & 0.50 & 18043.00 & 221671.68 & 1.28 & 0.96 & 1.23 \\
29753 & 105644 & 1999 & 222.10 & 0.15 & 22325.00 & 207951.16 & 0.99 & 0.94 & 0.93 \\
772 & 100096 & 1999 & 89.60 & 0.35 & 8960.00 & 81331.40 & 1.00 & 0.91 & 0.91 \\
19202 & 102563 & 1999 & 412.40 & 0.53 & 29692.00 & 424531.72 & 1.39 & 1.03 & 1.43 \\
4226 & 100590 & 1999 & 232.40 & 0.68 & 24755.00 & 226631.67 & 0.94 & 0.98 & 0.92 \\
26891 & 103620 & 1999 & 268.60 & 0.10 & 26655.00 & 257224.79 & 1.01 & 0.96 & 0.97 \\
4212 & 100575 & 1999 & 15.60 & 0.31 & 1145.00 & 13075.78 & 1.36 & 0.84 & 1.14 \\
37983 & 107179 & 1999 & 896.50 & 0.20 & 93052.00 & 857155.14 & 0.96 & 0.96 & 0.92 \\
29731 & 105643 & 1999 & 959.50 & 0.16 & 97031.00 & 902121.93 & 0.99 & 0.94 & 0.93 \\
31678 & 105930 & 1999 & 1809.90 & 1.35 & 180201.00 & 1759167.52 & 1.00 & 0.97 & 0.98 \\
19046 & 102545 & 1999 & 213.00 & 0.04 & 21241.00 & 208934.32 & 1.00 & 0.98 & 0.98 \\
19022 & 102544 & 1999 & 957.50 & 0.22 & 95143.00 & 947922.60 & 1.01 & 0.99 & 1.00 \\
18848 & 102524 & 1999 & 3056.70 & 0.20 & 305506.00 & 2995149.80 & 1.00 & 0.98 & 0.98 \\
708 & 100092 & 1999 & 386.50 & 0.35 & 38649.00 & 356477.81 & 1.00 & 0.92 & 0.92 \\
18879 & 102525 & 1999 & 323.00 & 0.14 & 33243.00 & 297445.20 & 0.97 & 0.92 & 0.89 \\
44397 & 109300 & 1999 & 878.90 & 0.12 & 97641.00 & 908726.57 & 0.90 & 1.03 & 0.93 \\
37879 & 107153 & 1999 & 33.70 & 0.09 & 3378.00 & 32976.61 & 1.00 & 0.98 & 0.98 \\
31672 & 105926 & 1999 & 1.50 & -0.08 & 269.00 & 1439.42 & 0.56 & 0.96 & 0.54 \\
27038 & 103645 & 1999 & 109.40 & -0.17 & 10659.00 & 105481.21 & 1.03 & 0.96 & 0.99 \\
18910 & 102527 & 1999 & 370.80 & 0.16 & 37658.00 & 376279.23 & 0.98 & 1.01 & 1.00 \\
4303 & 100603 & 1999 & 1465.00 & 0.03 & 146499.00 & 1240861.34 & 1.00 & 0.85 & 0.85 \\
37925 & 107162 & 1999 & 13.50 & 0.21 & 1348.00 & 13202.38 & 1.00 & 0.98 & 0.98 \\
37929 & 107167 & 1999 & 9.20 & 0.09 & 1213.00 & 8904.47 & 0.76 & 0.97 & 0.73 \\
18926 & 102528 & 1999 & 126.10 & 0.14 & 12633.00 & 122090.46 & 1.00 & 0.97 & 0.97 \\
27012 & 103644 & 1999 & 74.30 & 0.06 & 11782.00 & 123093.24 & 0.63 & 1.66 & 1.04 \\
35545 & 106372 & 1999 & 7.50 & 0.38 & 701.00 & 7005.40 & 1.07 & 0.93 & 1.00 \\
74588 & 601139 & 1999 & 4068.20 & 0.03 & 437928.00 & 4046345.98 & 0.93 & 0.99 & 0.92 \\
33881 & 106180 & 1999 & 61.40 & 0.35 & 5002.00 & 55782.20 & 1.23 & 0.91 & 1.12 \\
37933 & 107169 & 1999 & 2.30 & -0.08 & 220.00 & 2194.72 & 1.05 & 0.95 & 1.00 \\
37934 & 107171 & 1999 & 56.70 & 0.07 & 5649.00 & 56174.28 & 1.00 & 0.99 & 0.99 \\
33907 & 106182 & 1999 & 249.50 & 0.05 & 24946.00 & 226915.88 & 1.00 & 0.91 & 0.91 \\
37938 & 107173 & 1999 & 71.50 & -0.04 & 7081.00 & 67485.21 & 1.01 & 0.94 & 0.95 \\
26986 & 103643 & 1999 & 144.40 & 0.24 & 10799.00 & 117271.41 & 1.34 & 0.81 & 1.09 \\
18958 & 102531 & 1999 & 24.50 & 0.32 & 2432.00 & 24315.10 & 1.01 & 0.99 & 1.00 \\
31644 & 105920 & 1999 & 6080.90 & 0.41 & 446997.00 & 5571913.32 & 1.36 & 0.92 & 1.25 \\
18990 & 102540 & 1999 & 8.50 & -0.08 & 944.00 & 8312.12 & 0.90 & 0.98 & 0.88 \\
4283 & 100600 & 1999 & 52.90 & 0.04 & 6092.00 & 58982.16 & 0.87 & 1.11 & 0.97 \\
35518 & 106370 & 1999 & 65.00 & 1.25 & 6493.00 & 54254.55 & 1.00 & 0.83 & 0.84 \\
74560 & 601136 & 1999 & 55.10 & 0.28 & 5550.00 & 53313.55 & 0.99 & 0.97 & 0.96 \\
33920 & 106189 & 1999 & 340.30 & 0.50 & 33948.00 & 314377.10 & 1.00 & 0.92 & 0.93 \\
31416 & 105882 & 1999 & 323.80 & 0.33 & 32400.00 & 320079.15 & 1.00 & 0.99 & 0.99 \\
26374 & 103572 & 1999 & 50.40 & 0.00 & 5066.00 & 50655.65 & 0.99 & 1.01 & 1.00 \\
19979 & 102660 & 1999 & 6709.00 & 0.20 & 673979.00 & 6438085.30 & 1.00 & 0.96 & 0.96 \\
39118 & 107607 & 1999 & 44.30 & -0.04 & 4437.00 & 42338.00 & 1.00 & 0.96 & 0.95 \\
39128 & 107608 & 1999 & 31.70 & 0.01 & 3159.00 & 29641.01 & 1.00 & 0.94 & 0.94 \\
25726 & 103520 & 1999 & 15684.60 & 0.29 & 1581021.00 & 15690374.79 & 0.99 & 1.00 & 0.99 \\
20912 & 102799 & 1999 & 404.60 & 0.12 & 40270.00 & 402042.48 & 1.00 & 0.99 & 1.00 \\
34329 & 106223 & 1999 & 88.40 & -0.04 & 10398.00 & 100269.96 & 0.85 & 1.13 & 0.96 \\
29954 & 105662 & 1999 & 121.90 & 0.16 & 12350.00 & 122450.00 & 0.99 & 1.00 & 0.99 \\
34349 & 106224 & 1999 & 17.60 & 0.08 & 1607.00 & 16461.39 & 1.10 & 0.94 & 1.02 \\
20926 & 102802 & 1999 & 497.00 & 0.05 & 49690.00 & 485884.37 & 1.00 & 0.98 & 0.98 \\
31188 & 105866 & 1999 & 8079.10 & 0.21 & 643317.00 & 6910861.50 & 1.26 & 0.86 & 1.07 \\
39138 & 107609 & 1999 & 143.30 & 0.22 & 13110.00 & 129820.25 & 1.09 & 0.91 & 0.99 \\
20938 & 102812 & 1999 & 31.41 & 0.07 & 4880.00 & 47618.75 & 0.64 & 1.52 & 0.98 \\
20947 & 102813 & 1999 & 548.20 & -0.01 & 58249.00 & 533041.07 & 0.94 & 0.97 & 0.92 \\
29965 & 105663 & 1999 & 1.60 & -0.15 & 116.00 & 1005.09 & 1.38 & 0.63 & 0.87 \\
1137 & 100155 & 1999 & 2455.20 & 0.22 & 245751.00 & 2415167.65 & 1.00 & 0.98 & 0.98 \\
20977 & 102814 & 1999 & 122.40 & -0.09 & 14287.00 & 140075.88 & 0.86 & 1.14 & 0.98 \\
20993 & 102818 & 1999 & 80.50 & 0.24 & 6766.00 & 81786.28 & 1.19 & 1.02 & 1.21 \\
39144 & 107611 & 1999 & 625.60 & 0.19 & 57330.00 & 557877.82 & 1.09 & 0.89 & 0.97 \\
25694 & 103514 & 1999 & 2874.10 & -0.00 & 325722.00 & 2653516.94 & 0.88 & 0.92 & 0.81 \\
21004 & 102821 & 1999 & 234.90 & 0.80 & 16559.00 & 243917.33 & 1.42 & 1.04 & 1.47 \\
29970 & 105664 & 1999 & 274.80 & 0.19 & 25792.00 & 218862.97 & 1.07 & 0.80 & 0.85 \\
35332 & 106348 & 1999 & 344.90 & 0.94 & 34506.00 & 327984.71 & 1.00 & 0.95 & 0.95 \\
34363 & 106226 & 1999 & 74.90 & -0.03 & 7347.00 & 73321.81 & 1.02 & 0.98 & 1.00 \\
1152 & 100157 & 1999 & 1534.40 & 0.20 & 152470.00 & 1504701.27 & 1.01 & 0.98 & 0.99 \\
21015 & 102823 & 1999 & 37.00 & 0.01 & 3619.00 & 36622.92 & 1.02 & 0.99 & 1.01 \\
39093 & 107605 & 1999 & 17.00 & 0.10 & 1686.00 & 16061.15 & 1.01 & 0.94 & 0.95 \\
21023 & 102824 & 1999 & 90.90 & 0.11 & 9319.00 & 89195.61 & 0.98 & 0.98 & 0.96 \\
20884 & 102798 & 1999 & 1357.40 & 0.60 & 135556.00 & 1320481.01 & 1.00 & 0.97 & 0.97 \\
3446 & 100435 & 1999 & 5.00 & -0.06 & 508.00 & 4647.16 & 0.98 & 0.93 & 0.91 \\
38984 & 107470 & 1999 & 9.30 & 0.54 & 931.00 & 8690.11 & 1.00 & 0.93 & 0.93 \\
34319 & 106222 & 1999 & 21.80 & 0.17 & 2187.00 & 21051.89 & 1.00 & 0.97 & 0.96 \\
31227 & 105869 & 1999 & 118.80 & 0.08 & 11788.00 & 117924.92 & 1.01 & 0.99 & 1.00 \\
39015 & 107564 & 1999 & 132.50 & -0.00 & 13250.00 & 106661.88 & 1.00 & 0.80 & 0.80 \\
20709 & 102784 & 1999 & 40561.20 & 0.56 & 2846672.00 & 34871350.10 & 1.42 & 0.86 & 1.22 \\
39017 & 107565 & 1999 & 2.30 & 0.02 & 233.00 & 2338.58 & 0.99 & 1.02 & 1.00 \\
25826 & 103524 & 1999 & 104652.30 & 0.21 & 10465233.00 & 104561515.90 & 1.00 & 1.00 & 1.00 \\
20728 & 102788 & 1999 & 225.60 & 0.05 & 24054.00 & 246170.44 & 0.94 & 1.09 & 1.02 \\
3524 & 100453 & 1999 & 198.30 & 0.32 & 19821.00 & 186442.00 & 1.00 & 0.94 & 0.94 \\
29942 & 105659 & 1999 & 238.60 & 0.12 & 23818.00 & 238209.25 & 1.00 & 1.00 & 1.00 \\
20767 & 102789 & 1999 & 1438.20 & -0.00 & 171369.00 & 1690949.06 & 0.84 & 1.18 & 0.99 \\
3495 & 100441 & 1999 & 800.40 & 0.03 & 80505.00 & 783217.66 & 0.99 & 0.98 & 0.97 \\
39021 & 107566 & 1999 & 2.10 & 0.02 & 209.00 & 1961.03 & 1.00 & 0.93 & 0.94 \\
39052 & 107579 & 1999 & 47.40 & 0.02 & 4767.00 & 44652.16 & 0.99 & 0.94 & 0.94 \\
31221 & 105868 & 1999 & 111.50 & 0.59 & 11152.00 & 109829.45 & 1.00 & 0.99 & 0.98 \\
25792 & 103523 & 1999 & 5090.50 & 0.36 & 509054.00 & 4944653.76 & 1.00 & 0.97 & 0.97 \\
20806 & 102795 & 1999 & 249.10 & -0.02 & 25206.00 & 241645.56 & 0.99 & 0.97 & 0.96 \\
3461 & 100439 & 1999 & 41.60 & 0.31 & 4842.00 & 47898.98 & 0.86 & 1.15 & 0.99 \\
1102 & 100153 & 1999 & 204.90 & -0.01 & 20682.00 & 187972.41 & 0.99 & 0.92 & 0.91 \\
39054 & 107580 & 1999 & 42.40 & 0.04 & 4210.00 & 41628.88 & 1.01 & 0.98 & 0.99 \\
31216 & 105867 & 1999 & 5.60 & 0.77 & 552.00 & 5496.76 & 1.01 & 0.98 & 1.00 \\
39056 & 107598 & 1999 & 20.40 & 0.04 & 1885.00 & 20398.93 & 1.08 & 1.00 & 1.08 \\
25758 & 103521 & 1999 & 3837.60 & 0.13 & 383756.00 & 3834276.06 & 1.00 & 1.00 & 1.00 \\
96689 & 611006 & 1999 & 249.30 & 0.94 & 24749.00 & 247507.28 & 1.01 & 0.99 & 1.00 \\
20670 & 102783 & 1999 & 1469.80 & -0.01 & 158290.00 & 1348422.51 & 0.93 & 0.92 & 0.85 \\
61923 & 500308 & 1999 & 147.80 & 0.22 & 14761.00 & 143270.56 & 1.00 & 0.97 & 0.97 \\
39194 & 107618 & 1999 & 3362.30 & 0.13 & 336170.00 & 3200058.46 & 1.00 & 0.95 & 0.95 \\
39282 & 107648 & 1999 & 1080.60 & 0.29 & 100412.00 & 985170.12 & 1.08 & 0.91 & 0.98 \\
21194 & 102837 & 1999 & 899.70 & -0.01 & 90144.00 & 857364.89 & 1.00 & 0.95 & 0.95 \\
42621 & 108987 & 1999 & 105.40 & 0.04 & 7964.00 & 81320.52 & 1.32 & 0.77 & 1.02 \\
34394 & 106231 & 1999 & 112.00 & 0.00 & 11427.00 & 114174.89 & 0.98 & 1.02 & 1.00 \\
30017 & 105678 & 1999 & 9.70 & -0.05 & 915.00 & 8771.32 & 1.06 & 0.90 & 0.96 \\
35319 & 106347 & 1999 & 237.30 & 0.54 & 23732.00 & 237293.51 & 1.00 & 1.00 & 1.00 \\
1253 & 100167 & 1999 & 587.40 & 0.22 & 53335.00 & 570643.63 & 1.10 & 0.97 & 1.07 \\
39298 & 107650 & 1999 & 261.80 & 0.36 & 26123.00 & 231974.85 & 1.00 & 0.89 & 0.89 \\
39304 & 107652 & 1999 & 103.00 & 0.24 & 10444.00 & 93924.47 & 0.99 & 0.91 & 0.90 \\
21252 & 102843 & 1999 & 439.60 & 0.18 & 43861.00 & 437582.53 & 1.00 & 1.00 & 1.00 \\
34421 & 106236 & 1999 & 91.10 & 0.04 & 9033.00 & 77067.98 & 1.01 & 0.85 & 0.85 \\
25496 & 103494 & 1999 & 494.60 & 0.27 & 47517.00 & 475340.03 & 1.04 & 0.96 & 1.00 \\
21276 & 102844 & 1999 & 957.50 & 0.27 & 95824.00 & 952795.81 & 1.00 & 1.00 & 0.99 \\
21301 & 102846 & 1999 & 71.20 & -0.01 & 8383.00 & 85874.40 & 0.85 & 1.21 & 1.02 \\
21310 & 102847 & 1999 & 37.70 & 0.07 & 3770.00 & 37156.62 & 1.00 & 0.99 & 0.99 \\
34424 & 106238 & 1999 & 5.40 & 0.07 & 532.00 & 5323.19 & 1.02 & 0.99 & 1.00 \\
1271 & 100171 & 1999 & 1278.20 & -0.05 & 123117.00 & 1219687.01 & 1.04 & 0.95 & 0.99 \\
21329 & 102852 & 1999 & 3577.10 & 1.83 & 355568.00 & 3452081.66 & 1.01 & 0.97 & 0.97 \\
3240 & 100417 & 1999 & 10.60 & 0.08 & 930.00 & 10091.69 & 1.14 & 0.95 & 1.09 \\
39169 & 107616 & 1999 & 16.50 & 0.01 & 1334.00 & 13017.35 & 1.24 & 0.79 & 0.98 \\
21162 & 102835 & 1999 & 125.50 & 0.31 & 12354.00 & 111508.90 & 1.02 & 0.89 & 0.90 \\
39274 & 107636 & 1999 & 160.40 & 10.20 & 15743.00 & 146471.92 & 1.02 & 0.91 & 0.93 \\
61928 & 500309 & 1999 & 71.30 & 0.14 & 7133.00 & 69601.44 & 1.00 & 0.98 & 0.98 \\
34367 & 106230 & 1999 & 137.80 & -0.03 & 14673.00 & 146738.63 & 0.94 & 1.06 & 1.00 \\
31160 & 105865 & 1999 & 65.00 & 0.08 & 6588.00 & 65120.66 & 0.99 & 1.00 & 0.99 \\
25628 & 103498 & 1999 & 419.70 & 0.62 & 42103.00 & 412510.59 & 1.00 & 0.98 & 0.98 \\
21058 & 102825 & 1999 & 34.10 & 0.00 & 3399.00 & 33984.32 & 1.00 & 1.00 & 1.00 \\
29988 & 105665 & 1999 & 134.80 & 0.22 & 13938.00 & 142465.61 & 0.97 & 1.06 & 1.02 \\
21065 & 102827 & 1999 & 156.90 & 0.11 & 15682.00 & 156050.52 & 1.00 & 0.99 & 1.00 \\
1184 & 100159 & 1999 & 257.70 & 0.02 & 25340.00 & 253403.26 & 1.02 & 0.98 & 1.00 \\
21088 & 102828 & 1999 & 199.30 & 0.13 & 19929.00 & 197273.06 & 1.00 & 0.99 & 0.99 \\
30000 & 105676 & 1999 & 334.00 & -0.08 & 44370.00 & 425734.81 & 0.75 & 1.27 & 0.96 \\
31153 & 105864 & 1999 & 38.60 & 0.37 & 3899.00 & 35405.17 & 0.99 & 0.92 & 0.91 \\
1222 & 100166 & 1999 & 7187.60 & 0.22 & 604780.00 & 6059996.43 & 1.19 & 0.84 & 1.00 \\
21097 & 102829 & 1999 & 56.50 & 0.16 & 5654.00 & 52054.66 & 1.00 & 0.92 & 0.92 \\
61933 & 500310 & 1999 & 7.20 & 0.10 & 530.00 & 6504.95 & 1.36 & 0.90 & 1.23 \\
39236 & 107626 & 1999 & 235.00 & 0.09 & 23506.00 & 225993.85 & 1.00 & 0.96 & 0.96 \\
21108 & 102832 & 1999 & 589.70 & 0.51 & 58968.00 & 579507.87 & 1.00 & 0.98 & 0.98 \\
61958 & 500312 & 1999 & 4.60 & -0.02 & 460.00 & 4375.56 & 1.00 & 0.95 & 0.95 \\
31149 & 105862 & 1999 & 63.40 & 0.04 & 6250.00 & 62627.41 & 1.01 & 0.99 & 1.00 \\
21140 & 102833 & 1999 & 122.20 & 0.72 & 12221.00 & 118860.29 & 1.00 & 0.97 & 0.97 \\
3330 & 100424 & 1999 & 34.00 & 0.50 & 3047.00 & 33844.84 & 1.12 & 1.00 & 1.11 \\
39253 & 107627 & 1999 & 487.00 & -0.03 & 48669.00 & 462416.12 & 1.00 & 0.95 & 0.95 \\
39270 & 107630 & 1999 & 83.30 & 0.07 & 8295.00 & 67467.70 & 1.00 & 0.81 & 0.81 \\
31137 & 105861 & 1999 & 223.30 & -0.12 & 21649.00 & 216507.45 & 1.03 & 0.97 & 1.00 \\
25553 & 103496 & 1999 & 762.40 & 0.31 & 74200.00 & 742396.40 & 1.03 & 0.97 & 1.00 \\
25859 & 103525 & 1999 & 35778.60 & 0.21 & 3282395.00 & 34598668.16 & 1.09 & 0.97 & 1.05 \\
34202 & 106212 & 1999 & 82.10 & 0.24 & 8076.00 & 80416.85 & 1.02 & 0.98 & 1.00 \\
31362 & 105880 & 1999 & 85.80 & 0.28 & 8677.00 & 85463.24 & 0.99 & 1.00 & 0.98 \\
38629 & 107302 & 1999 & 466.20 & 0.18 & 46614.00 & 464722.15 & 1.00 & 1.00 & 1.00 \\
38654 & 107303 & 1999 & 52.40 & 0.07 & 5333.00 & 52007.81 & 0.98 & 0.99 & 0.98 \\
26213 & 103546 & 1999 & 16699.80 & 0.12 & 1877201.00 & 16763946.65 & 0.89 & 1.00 & 0.89 \\
20202 & 102688 & 1999 & 122.00 & 0.02 & 12162.00 & 116897.22 & 1.00 & 0.96 & 0.96 \\
61287 & 500027 & 1999 & 226.50 & 0.56 & 22613.00 & 206477.07 & 1.00 & 0.91 & 0.91 \\
34220 & 106213 & 1999 & 23.80 & 0.77 & 1733.00 & 20847.16 & 1.37 & 0.88 & 1.20 \\
38662 & 107306 & 1999 & 162.40 & 0.15 & 16378.00 & 152212.98 & 0.99 & 0.94 & 0.93 \\
61315 & 500028 & 1999 & 22.80 & 0.22 & 2274.00 & 22628.79 & 1.00 & 0.99 & 1.00 \\
3753 & 100480 & 1999 & 163.90 & 0.21 & 18770.00 & 140119.56 & 0.87 & 0.85 & 0.75 \\
31344 & 105879 & 1999 & 612.40 & 0.38 & 61012.00 & 577204.10 & 1.00 & 0.94 & 0.95 \\
26173 & 103545 & 1999 & 26901.80 & -0.06 & 3062328.00 & 26406491.51 & 0.88 & 0.98 & 0.86 \\
29893 & 105656 & 1999 & 175.20 & 0.19 & 17514.00 & 167542.18 & 1.00 & 0.96 & 0.96 \\
978 & 100113 & 1999 & 1351.90 & 0.33 & 135332.00 & 1312891.24 & 1.00 & 0.97 & 0.97 \\
61329 & 500037 & 1999 & 8683.70 & 0.25 & 867507.00 & 7750391.53 & 1.00 & 0.89 & 0.89 \\
38710 & 107309 & 1999 & 259.20 & 0.04 & 14361.00 & 143632.79 & 1.80 & 0.55 & 1.00 \\
20254 & 102696 & 1999 & 561.30 & -0.16 & 56214.00 & 555932.50 & 1.00 & 0.99 & 0.99 \\
35382 & 106356 & 1999 & 17.10 & -0.13 & 1740.00 & 16404.53 & 0.98 & 0.96 & 0.94 \\
61371 & 500047 & 1999 & 3.00 & 0.11 & 296.00 & 2912.91 & 1.01 & 0.97 & 0.98 \\
3723 & 100475 & 1999 & 227.10 & 0.22 & 22718.00 & 217235.01 & 1.00 & 0.96 & 0.96 \\
26139 & 103544 & 1999 & 22139.60 & 0.24 & 2213959.00 & 21517447.92 & 1.00 & 0.97 & 0.97 \\
38745 & 107318 & 1999 & 5.90 & -0.05 & 839.00 & 7658.40 & 0.70 & 1.30 & 0.91 \\
38746 & 107319 & 1999 & 156.56 & 0.09 & 15770.00 & 152904.24 & 0.99 & 0.98 & 0.97 \\
61397 & 500048 & 1999 & 72.40 & 0.26 & 6275.00 & 65990.03 & 1.15 & 0.91 & 1.05 \\
34232 & 106214 & 1999 & 46.00 & 0.07 & 4283.00 & 42836.37 & 1.07 & 0.93 & 1.00 \\
35408 & 106358 & 1999 & 11.60 & 0.08 & 1124.00 & 11676.29 & 1.03 & 1.01 & 1.04 \\
38750 & 107320 & 1999 & 308.02 & 0.16 & 30549.00 & 279468.26 & 1.01 & 0.91 & 0.91 \\
3783 & 100481 & 1999 & 62.50 & 0.15 & 6956.00 & 55594.23 & 0.90 & 0.89 & 0.80 \\
26244 & 103547 & 1999 & 19068.00 & 0.22 & 1602327.00 & 18297371.25 & 1.19 & 0.96 & 1.14 \\
20014 & 102663 & 1999 & 12270.60 & 1.01 & 1233693.00 & 11229333.00 & 0.99 & 0.92 & 0.91 \\
38538 & 107279 & 1999 & 7.10 & 0.04 & 526.00 & 4358.16 & 1.35 & 0.61 & 0.83 \\
38539 & 107281 & 1999 & 13.10 & 0.11 & 1309.00 & 12852.49 & 1.00 & 0.98 & 0.98 \\
20048 & 102664 & 1999 & 1262.20 & 0.01 & 126252.00 & 1142843.91 & 1.00 & 0.91 & 0.91 \\
26341 & 103570 & 1999 & 21.00 & 0.05 & 2165.00 & 18780.31 & 0.97 & 0.89 & 0.87 \\
38564 & 107282 & 1999 & 50.20 & 0.24 & 5035.00 & 47609.25 & 1.00 & 0.95 & 0.95 \\
34163 & 106210 & 1999 & 14.70 & 0.20 & 1570.00 & 15758.23 & 0.94 & 1.07 & 1.00 \\
20109 & 102667 & 1999 & 17911.90 & 0.41 & 1793349.00 & 14829330.38 & 1.00 & 0.83 & 0.83 \\
3815 & 100485 & 1999 & 511.40 & 0.05 & 51451.00 & 495544.79 & 0.99 & 0.97 & 0.96 \\
38572 & 107290 & 1999 & 477.50 & 0.06 & 39568.00 & 441206.87 & 1.21 & 0.92 & 1.12 \\
26309 & 103567 & 1999 & 1922.70 & 0.08 & 261558.00 & 2583511.67 & 0.74 & 1.34 & 0.99 \\
31389 & 105881 & 1999 & 225.60 & -0.01 & 28502.00 & 259721.33 & 0.79 & 1.15 & 0.91 \\
29864 & 105655 & 1999 & 324.40 & -0.02 & 32144.00 & 302117.73 & 1.01 & 0.93 & 0.94 \\
20142 & 102669 & 1999 & 46.80 & 0.16 & 4622.00 & 46097.28 & 1.01 & 0.98 & 1.00 \\
38586 & 107294 & 1999 & 152.00 & 0.27 & 11972.00 & 109891.23 & 1.27 & 0.72 & 0.92 \\
26291 & 103564 & 1999 & 956.00 & 0.13 & 95812.00 & 929630.28 & 1.00 & 0.97 & 0.97 \\
35415 & 106359 & 1999 & 413.10 & 0.26 & 38430.00 & 414576.10 & 1.07 & 1.00 & 1.08 \\
34190 & 106211 & 1999 & 89.60 & 0.03 & 8958.00 & 84290.30 & 1.00 & 0.94 & 0.94 \\
934 & 100112 & 1999 & 4157.90 & 0.11 & 415935.00 & 3878582.20 & 1.00 & 0.93 & 0.93 \\
20174 & 102673 & 1999 & 459.40 & 0.11 & 45776.00 & 444917.67 & 1.00 & 0.97 & 0.97 \\
38619 & 107300 & 1999 & 92.90 & 0.05 & 7316.00 & 81353.49 & 1.27 & 0.88 & 1.11 \\
20187 & 102676 & 1999 & 301.30 & 0.11 & 29981.00 & 297132.53 & 1.00 & 0.99 & 0.99 \\
34292 & 106221 & 1999 & 64.90 & -0.03 & 7557.00 & 73169.26 & 0.86 & 1.13 & 0.97 \\
38758 & 107322 & 1999 & 10.10 & 0.04 & 1033.00 & 10326.47 & 0.98 & 1.02 & 1.00 \\
3621 & 100463 & 1999 & 29.50 & 0.78 & 2832.00 & 28318.82 & 1.04 & 0.96 & 1.00 \\
38864 & 107338 & 1999 & 7.20 & 0.08 & 497.00 & 5148.80 & 1.45 & 0.72 & 1.04 \\
31278 & 105874 & 1999 & 54.80 & 0.35 & 5482.00 & 53538.94 & 1.00 & 0.98 & 0.98 \\
20521 & 102761 & 1999 & 36509.60 & 0.01 & 4128920.00 & 36272439.43 & 0.88 & 0.99 & 0.88 \\
25960 & 103531 & 1999 & 1018.20 & 0.21 & 111153.00 & 990058.84 & 0.92 & 0.97 & 0.89 \\
20558 & 102767 & 1999 & 13576.00 & 0.30 & 1014095.00 & 12189030.50 & 1.34 & 0.90 & 1.20 \\
3589 & 100460 & 1999 & 91.00 & 0.30 & 10177.00 & 81257.81 & 0.89 & 0.89 & 0.80 \\
38879 & 107348 & 1999 & 71.80 & 0.38 & 6023.00 & 73768.00 & 1.19 & 1.03 & 1.22 \\
38880 & 107350 & 1999 & 768.00 & 0.01 & 43363.00 & 434721.53 & 1.77 & 0.57 & 1.00 \\
25922 & 103529 & 1999 & 8816.40 & 0.26 & 883418.00 & 8652245.68 & 1.00 & 0.98 & 0.98 \\
3577 & 100457 & 1999 & 315.20 & 0.34 & 32839.00 & 293920.33 & 0.96 & 0.93 & 0.90 \\
35344 & 106352 & 1999 & 83.00 & -0.08 & 8300.00 & 81185.24 & 1.00 & 0.98 & 0.98 \\
34286 & 106220 & 1999 & 540.70 & 0.87 & 30639.00 & 440407.11 & 1.76 & 0.81 & 1.44 \\
31267 & 105873 & 1999 & 13.90 & 0.22 & 1326.00 & 13990.34 & 1.05 & 1.01 & 1.06 \\
20598 & 102774 & 1999 & 8686.30 & 0.53 & 623391.00 & 8876380.27 & 1.39 & 1.02 & 1.42 \\
20618 & 102775 & 1999 & 1787.40 & 0.01 & 190525.00 & 1687983.75 & 0.94 & 0.94 & 0.89 \\
3564 & 100456 & 1999 & 3.60 & 0.34 & 365.00 & 3534.96 & 0.99 & 0.98 & 0.97 \\
1072 & 100150 & 1999 & 22.80 & 0.09 & 2242.00 & 22960.32 & 1.02 & 1.01 & 1.02 \\
38921 & 107354 & 1999 & 33.60 & 0.12 & 3427.00 & 31660.15 & 0.98 & 0.94 & 0.92 \\
38946 & 107358 & 1999 & 4.50 & 0.04 & 439.00 & 4390.64 & 1.03 & 0.98 & 1.00 \\
31254 & 105871 & 1999 & 9.40 & -0.08 & 1138.00 & 9907.58 & 0.83 & 1.05 & 0.87 \\
3555 & 100455 & 1999 & 9.80 & 0.11 & 977.00 & 8422.15 & 1.00 & 0.86 & 0.86 \\
25893 & 103526 & 1999 & 5066.30 & 0.34 & 515064.00 & 4625959.73 & 0.98 & 0.91 & 0.90 \\
38971 & 107361 & 1999 & 43.90 & 0.02 & 4377.00 & 42600.13 & 1.00 & 0.97 & 0.97 \\
38974 & 107362 & 1999 & 87.50 & 0.03 & 8747.00 & 86448.40 & 1.00 & 0.99 & 0.99 \\
61566 & 500096 & 1999 & 30.70 & -0.00 & 4619.00 & 41013.35 & 0.66 & 1.34 & 0.89 \\
38754 & 107321 & 1999 & 22.20 & 0.03 & 2277.00 & 22773.47 & 0.97 & 1.03 & 1.00 \\
20499 & 102760 & 1999 & 3411.00 & 0.79 & 341585.00 & 3056127.84 & 1.00 & 0.90 & 0.89 \\
31292 & 105875 & 1999 & 112.10 & 0.45 & 11224.00 & 109023.29 & 1.00 & 0.97 & 0.97 \\
26105 & 103539 & 1999 & 840.70 & -0.03 & 99380.00 & 834354.35 & 0.85 & 0.99 & 0.84 \\
38781 & 107323 & 1999 & 28.40 & 0.10 & 2887.00 & 27925.74 & 0.98 & 0.98 & 0.97 \\
20301 & 102715 & 1999 & 4984.60 & 0.12 & 499170.00 & 4817356.81 & 1.00 & 0.97 & 0.97 \\
34259 & 106216 & 1999 & 531.40 & 0.92 & 32186.00 & 427969.44 & 1.65 & 0.81 & 1.33 \\
38806 & 107328 & 1999 & 28.90 & -0.05 & 3066.00 & 26950.63 & 0.94 & 0.93 & 0.88 \\
20335 & 102716 & 1999 & 1806.90 & 0.11 & 191406.00 & 1868300.24 & 0.94 & 1.03 & 0.98 \\
3680 & 100468 & 1999 & 334.60 & -0.04 & 33454.00 & 329208.66 & 1.00 & 0.98 & 0.98 \\
38818 & 107329 & 1999 & 43.50 & 0.34 & 4418.00 & 42856.55 & 0.98 & 0.99 & 0.97 \\
31316 & 105878 & 1999 & 388.30 & 0.35 & 39334.00 & 337793.28 & 0.99 & 0.87 & 0.86 \\
26058 & 103536 & 1999 & 2348.20 & 0.21 & 236229.00 & 2349532.43 & 0.99 & 1.00 & 0.99 \\
35348 & 106353 & 1999 & 29.10 & 0.26 & 2787.00 & 27871.99 & 1.04 & 0.96 & 1.00 \\
61479 & 500082 & 1999 & 58.10 & 0.08 & 5705.00 & 56205.90 & 1.02 & 0.97 & 0.99 \\
61489 & 500083 & 1999 & 5.90 & 0.05 & 600.00 & 5306.67 & 0.98 & 0.90 & 0.88 \\
31308 & 105877 & 1999 & 16.30 & 0.14 & 1602.00 & 14189.38 & 1.02 & 0.87 & 0.89 \\
20386 & 102733 & 1999 & 4038.80 & 0.04 & 380162.00 & 3809220.29 & 1.06 & 0.94 & 1.00 \\
38822 & 107331 & 1999 & 2.60 & 0.18 & 260.00 & 2573.01 & 1.00 & 0.99 & 0.99 \\
20418 & 102737 & 1999 & 2209.60 & 0.21 & 188375.00 & 2067134.27 & 1.17 & 0.94 & 1.10 \\
31301 & 105876 & 1999 & 131.80 & 0.72 & 13189.00 & 122393.70 & 1.00 & 0.93 & 0.93 \\
26028 & 103535 & 1999 & 2543.50 & 0.34 & 259337.00 & 2541677.14 & 0.98 & 1.00 & 0.98 \\
20448 & 102744 & 1999 & 2578.90 & 0.17 & 238817.00 & 2570180.32 & 1.08 & 1.00 & 1.08 \\
1022 & 100127 & 1999 & 9179.90 & 0.71 & 913002.00 & 8284077.14 & 1.01 & 0.90 & 0.91 \\
29922 & 105657 & 1999 & 17.30 & -0.01 & 1685.00 & 16416.33 & 1.03 & 0.95 & 0.97 \\
61555 & 500094 & 1999 & 9.30 & 0.01 & 929.00 & 8467.60 & 1.00 & 0.91 & 0.91 \\
26014 & 103533 & 1999 & 10420.90 & 0.07 & 1065365.00 & 8492009.49 & 0.98 & 0.81 & 0.80 \\
20470 & 102757 & 1999 & 32313.30 & 0.69 & 3220758.00 & 31575229.68 & 1.00 & 0.98 & 0.98 \\
43077 & 109059 & 1999 & 50.90 & 0.36 & 4954.00 & 50078.25 & 1.03 & 0.98 & 1.01 \\
29927 & 105658 & 1999 & 301.60 & 0.19 & 29604.00 & 295759.62 & 1.02 & 0.98 & 1.00 \\
29127 & 105531 & 1999 & 78.40 & 1.50 & 7851.00 & 76014.42 & 1.00 & 0.97 & 0.97 \\
52384 & 302826 & 1999 & 142.10 & 0.37 & 14339.00 & 132630.62 & 0.99 & 0.93 & 0.92 \\
14131 & 101805 & 1999 & 1199.50 & 0.75 & 119728.00 & 1153432.26 & 1.00 & 0.96 & 0.96 \\
7429 & 101039 & 1999 & 6329.10 & 0.25 & 529242.00 & 5688983.76 & 1.20 & 0.90 & 1.07 \\
13034 & 101622 & 1999 & 2461.30 & 0.73 & 249505.00 & 2206842.49 & 0.99 & 0.90 & 0.88 \\
12307 & 101534 & 1999 & 1068.90 & -0.00 & 109030.00 & 1024306.37 & 0.98 & 0.96 & 0.94 \\
11371 & 101399 & 1999 & 164.20 & 0.23 & 16653.00 & 164019.16 & 0.99 & 1.00 & 0.98 \\
8317 & 101082 & 1999 & 2427.40 & -0.03 & 286556.00 & 2617247.30 & 0.85 & 1.08 & 0.91 \\
12764 & 101593 & 1999 & 32.30 & 0.06 & 3224.00 & 29591.76 & 1.00 & 0.92 & 0.92 \\
11364 & 101398 & 1999 & 221.70 & 0.24 & 18237.00 & 179827.38 & 1.22 & 0.81 & 0.99 \\
9680 & 101165 & 1999 & 1640.30 & 0.20 & 166017.00 & 1636802.30 & 0.99 & 1.00 & 0.99 \\
9182 & 101116 & 1999 & 2885.00 & 0.27 & 263547.00 & 2607546.86 & 1.09 & 0.90 & 0.99 \\
10004 & 101252 & 1999 & 72.50 & 0.48 & 7004.00 & 70042.18 & 1.04 & 0.97 & 1.00 \\
48446 & 240085 & 1999 & 174.70 & 0.30 & 17433.00 & 173124.50 & 1.00 & 0.99 & 0.99 \\
14403 & 101854 & 1999 & 13564.70 & 0.27 & 1357008.00 & 12500153.52 & 1.00 & 0.92 & 0.92 \\
12754 & 101592 & 1999 & 207.80 & 0.09 & 20802.00 & 175199.80 & 1.00 & 0.84 & 0.84 \\
13281 & 101717 & 1999 & 97.90 & 0.17 & 10110.00 & 92631.35 & 0.97 & 0.95 & 0.92 \\
47565 & 212809 & 1999 & 3.60 & 0.06 & 367.00 & 3621.33 & 0.98 & 1.01 & 0.99 \\
10210 & 101274 & 1999 & 250.10 & 0.17 & 25016.00 & 242875.36 & 1.00 & 0.97 & 0.97 \\
14740 & 101912 & 1999 & 3630.70 & 0.02 & 362598.00 & 3314599.23 & 1.00 & 0.91 & 0.91 \\
13155 & 101681 & 1999 & 876.00 & 0.82 & 89568.00 & 782115.21 & 0.98 & 0.89 & 0.87 \\
10957 & 101356 & 1999 & 469.10 & 0.39 & 46854.00 & 455879.74 & 1.00 & 0.97 & 0.97 \\
48927 & 240153 & 1999 & 87.10 & 0.04 & 7153.00 & 72224.64 & 1.22 & 0.83 & 1.01 \\
48140 & 240027 & 1999 & 65.50 & -0.03 & 6442.00 & 62897.35 & 1.02 & 0.96 & 0.98 \\
13352 & 101729 & 1999 & 372.80 & 0.31 & 34170.00 & 309683.76 & 1.09 & 0.83 & 0.91 \\
11399 & 101400 & 1999 & 295.80 & 0.02 & 29478.00 & 292055.05 & 1.00 & 0.99 & 0.99 \\
8758 & 101097 & 1999 & 1595.20 & 0.67 & 96622.00 & 1290695.62 & 1.65 & 0.81 & 1.34 \\
10670 & 101307 & 1999 & 384.00 & -0.06 & 38406.00 & 371222.32 & 1.00 & 0.97 & 0.97 \\
12124 & 101511 & 1999 & 215.60 & 0.95 & 19249.00 & 202333.44 & 1.12 & 0.94 & 1.05 \\
13584 & 101744 & 1999 & 2876.10 & 1.41 & 286798.00 & 2686910.79 & 1.00 & 0.93 & 0.94 \\
12272 & 101531 & 1999 & 87.10 & 0.02 & 8873.00 & 74883.37 & 0.98 & 0.86 & 0.84 \\
52537 & 303121 & 1999 & 294.40 & 0.19 & 27611.00 & 281026.69 & 1.07 & 0.95 & 1.02 \\
10925 & 101354 & 1999 & 1746.20 & 0.26 & 174668.00 & 1674718.02 & 1.00 & 0.96 & 0.96 \\
47887 & 222658 & 1999 & 265.70 & 0.25 & 26527.00 & 259749.26 & 1.00 & 0.98 & 0.98 \\
48265 & 240058 & 1999 & 38.80 & 0.11 & 4088.00 & 38042.91 & 0.95 & 0.98 & 0.93 \\
9904 & 101211 & 1999 & 464.90 & 0.07 & 46082.00 & 449348.81 & 1.01 & 0.97 & 0.98 \\
49266 & 240261 & 1999 & 141.10 & -0.09 & 14121.00 & 137958.95 & 1.00 & 0.98 & 0.98 \\
8967 & 101107 & 1999 & 1533.20 & 0.41 & 151973.00 & 1410006.85 & 1.01 & 0.92 & 0.93 \\
52518 & 302997 & 1999 & 58.80 & 1.44 & 6302.00 & 53972.79 & 0.93 & 0.92 & 0.86 \\
10799 & 101331 & 1999 & 149.90 & 0.17 & 14979.00 & 147122.38 & 1.00 & 0.98 & 0.98 \\
8678 & 101094 & 1999 & 1412.90 & 0.93 & 76496.00 & 1180516.48 & 1.85 & 0.84 & 1.54 \\
49440 & 240300 & 1999 & 54.70 & 0.23 & 5523.00 & 53337.23 & 0.99 & 0.98 & 0.97 \\
14706 & 101911 & 1999 & 1431.10 & 0.04 & 142618.00 & 1416977.49 & 1.00 & 0.99 & 0.99 \\
49255 & 240256 & 1999 & 175.30 & 0.13 & 9503.00 & 89996.62 & 1.84 & 0.51 & 0.95 \\
47470 & 212027 & 1999 & 101.80 & 0.72 & 10303.00 & 101862.46 & 0.99 & 1.00 & 0.99 \\
9704 & 101167 & 1999 & 140.80 & -0.02 & 13674.00 & 132341.55 & 1.03 & 0.94 & 0.97 \\
10196 & 101268 & 1999 & 3325.90 & 0.39 & 303471.00 & 3225353.04 & 1.10 & 0.97 & 1.06 \\
8067 & 101073 & 1999 & 11705.40 & 0.73 & 731071.00 & 9661738.04 & 1.60 & 0.83 & 1.32 \\
12782 & 101595 & 1999 & 3888.70 & 0.58 & 310975.00 & 3349541.95 & 1.25 & 0.86 & 1.08 \\
9661 & 101161 & 1999 & 1495.10 & 0.08 & 145885.00 & 1459258.77 & 1.02 & 0.98 & 1.00 \\
6910 & 100968 & 1999 & 398.20 & 0.06 & 39879.00 & 329647.33 & 1.00 & 0.83 & 0.83 \\
9630 & 101160 & 1999 & 308.90 & -0.06 & 30342.00 & 295774.84 & 1.02 & 0.96 & 0.97 \\
13616 & 101748 & 1999 & 26.60 & 0.17 & 2657.00 & 24468.10 & 1.00 & 0.92 & 0.92 \\
53487 & 351048 & 1999 & 115.70 & 0.38 & 11391.00 & 106344.19 & 1.02 & 0.92 & 0.93 \\
13071 & 101626 & 1999 & 2087.30 & 0.08 & 211217.00 & 1759607.41 & 0.99 & 0.84 & 0.83 \\
12101 & 101503 & 1999 & 341.50 & 0.57 & 24148.00 & 312159.19 & 1.41 & 0.91 & 1.29 \\
52478 & 302964 & 1999 & 5.40 & 0.32 & 532.00 & 5041.21 & 1.02 & 0.93 & 0.95 \\
10983 & 101358 & 1999 & 591.10 & 0.07 & 59446.00 & 589795.48 & 0.99 & 1.00 & 0.99 \\
9922 & 101212 & 1999 & 2826.40 & 0.45 & 282161.00 & 2514274.90 & 1.00 & 0.89 & 0.89 \\
48255 & 240057 & 1999 & 82.50 & -0.01 & 8251.00 & 78431.55 & 1.00 & 0.95 & 0.95 \\
52240 & 302698 & 1999 & 396.20 & 0.47 & 40416.00 & 383733.42 & 0.98 & 0.97 & 0.95 \\
14672 & 101908 & 1999 & 13.30 & 0.33 & 1324.00 & 12944.71 & 1.00 & 0.97 & 0.98 \\
6790 & 100954 & 1999 & 1764.50 & 0.14 & 175152.00 & 1709490.16 & 1.01 & 0.97 & 0.98 \\
47766 & 221210 & 1999 & 85.60 & 0.08 & 5123.00 & 79769.30 & 1.67 & 0.93 & 1.56 \\
53711 & 356500 & 1999 & 485.20 & 0.19 & 47191.00 & 471886.02 & 1.03 & 0.97 & 1.00 \\
6758 & 100950 & 1999 & 199.00 & 0.00 & 19944.00 & 188840.69 & 1.00 & 0.95 & 0.95 \\
9611 & 101158 & 1999 & 752.30 & 0.28 & 75204.00 & 734320.05 & 1.00 & 0.98 & 0.98 \\
14772 & 101913 & 1999 & 38.80 & 0.42 & 3888.00 & 38367.36 & 1.00 & 0.99 & 0.99 \\
13381 & 101736 & 1999 & 72.10 & 0.17 & 7207.00 & 69553.23 & 1.00 & 0.96 & 0.97 \\
48068 & 235413 & 1999 & 56.10 & 0.26 & 4855.00 & 55495.18 & 1.16 & 0.99 & 1.14 \\
10257 & 101276 & 1999 & 538.77 & 0.14 & 53877.00 & 494017.88 & 1.00 & 0.92 & 0.92 \\
52266 & 302731 & 1999 & 45.20 & 0.28 & 4599.00 & 45098.14 & 0.98 & 1.00 & 0.98 \\
7493 & 101042 & 1999 & 12623.90 & 0.09 & 1430032.00 & 13597489.18 & 0.88 & 1.08 & 0.95 \\
12358 & 101537 & 1999 & 1118.90 & -0.00 & 111557.00 & 1057595.39 & 1.00 & 0.95 & 0.95 \\
12325 & 101536 & 1999 & 1794.50 & 0.28 & 181810.00 & 1723429.36 & 0.99 & 0.96 & 0.95 \\
13265 & 101714 & 1999 & 22.80 & 0.13 & 1970.00 & 22787.86 & 1.16 & 1.00 & 1.16 \\
11904 & 101465 & 1999 & 109.40 & 0.18 & 10987.00 & 106925.33 & 1.00 & 0.98 & 0.97 \\
48434 & 240083 & 1999 & 382.10 & 0.20 & 38753.00 & 364624.37 & 0.99 & 0.95 & 0.94 \\
52035 & 301299 & 1999 & 17838.00 & 0.29 & 1776709.00 & 17477170.59 & 1.00 & 0.98 & 0.98 \\
47987 & 225687 & 1999 & 1949.90 & 2.44 & 193811.00 & 1938565.35 & 1.01 & 0.99 & 1.00 \\
8461 & 101087 & 1999 & 468.60 & 0.31 & 41721.00 & 448295.68 & 1.12 & 0.96 & 1.07 \\
52188 & 302676 & 1999 & 2.40 & 0.31 & 179.00 & 1761.53 & 1.34 & 0.73 & 0.98 \\
6768 & 100953 & 1999 & 22.40 & -0.18 & 2464.00 & 24647.89 & 0.91 & 1.10 & 1.00 \\
14512 & 101871 & 1999 & 454.70 & 0.14 & 45466.00 & 453146.01 & 1.00 & 1.00 & 1.00 \\
10046 & 101256 & 1999 & 9.40 & 0.23 & 947.00 & 9369.45 & 0.99 & 1.00 & 0.99 \\
52214 & 302677 & 1999 & 5.90 & 0.30 & 595.00 & 5722.59 & 0.99 & 0.97 & 0.96 \\
49238 & 240254 & 1999 & 993.30 & 0.02 & 99330.00 & 957918.58 & 1.00 & 0.96 & 0.96 \\
52590 & 303140 & 1999 & 96.80 & 0.16 & 9302.00 & 87486.09 & 1.04 & 0.90 & 0.94 \\
54841 & 400018 & 1999 & 93.60 & 0.16 & 9494.00 & 91644.34 & 0.99 & 0.98 & 0.97 \\
8629 & 101092 & 1999 & 332.30 & -0.12 & 47296.00 & 409111.72 & 0.70 & 1.23 & 0.87 \\
10972 & 101357 & 1999 & 324.90 & 0.05 & 32451.00 & 294653.81 & 1.00 & 0.91 & 0.91 \\
10229 & 101275 & 1999 & 870.30 & 0.23 & 86972.00 & 837090.74 & 1.00 & 0.96 & 0.96 \\
13272 & 101716 & 1999 & 41.70 & 0.17 & 4195.00 & 38133.56 & 0.99 & 0.91 & 0.91 \\
14952 & 101925 & 1999 & 390.50 & 0.04 & 38629.00 & 374469.92 & 1.01 & 0.96 & 0.97 \\
9035 & 101109 & 1999 & 50.50 & -0.30 & 5446.00 & 54286.94 & 0.93 & 1.07 & 1.00 \\
7313 & 101020 & 1999 & 3686.40 & 0.46 & 293891.00 & 2841984.58 & 1.25 & 0.77 & 0.97 \\
8396 & 101085 & 1999 & 626.20 & 0.26 & 58830.00 & 650223.25 & 1.06 & 1.04 & 1.11 \\
9725 & 101179 & 1999 & 999.00 & 0.22 & 99623.00 & 982266.07 & 1.00 & 0.98 & 0.99 \\
47475 & 212351 & 1999 & 204.70 & 0.20 & 21298.00 & 202125.11 & 0.96 & 0.99 & 0.95 \\
9829 & 101194 & 1999 & 187.70 & 0.15 & 16478.00 & 167246.47 & 1.14 & 0.89 & 1.01 \\
7239 & 101015 & 1999 & 534.60 & 0.19 & 55952.00 & 572059.52 & 0.96 & 1.07 & 1.02 \\
13866 & 101781 & 1999 & 890.40 & 0.32 & 89249.00 & 862356.64 & 1.00 & 0.97 & 0.97 \\
12837 & 101602 & 1999 & 2797.50 & 0.09 & 290363.00 & 2579936.16 & 0.96 & 0.92 & 0.89 \\
53635 & 355027 & 1999 & 682.70 & 0.94 & 123605.00 & 1032769.94 & 0.55 & 1.51 & 0.84 \\
10861 & 101340 & 1999 & 16603.10 & 0.25 & 1612787.00 & 13783930.23 & 1.03 & 0.83 & 0.85 \\
52129 & 302060 & 1999 & 319.30 & 0.29 & 31898.00 & 307654.54 & 1.00 & 0.96 & 0.96 \\
49316 & 240269 & 1999 & 131.20 & -0.05 & 12123.00 & 121213.44 & 1.08 & 0.92 & 1.00 \\
14227 & 101834 & 1999 & 227.60 & 0.10 & 22782.00 & 225753.53 & 1.00 & 0.99 & 0.99 \\
9115 & 101112 & 1999 & 1801.10 & 0.26 & 185530.00 & 1826735.11 & 0.97 & 1.01 & 0.98 \\
9887 & 101200 & 1999 & 67.50 & 0.47 & 6822.00 & 68275.24 & 0.99 & 1.01 & 1.00 \\
12189 & 101518 & 1999 & 601.30 & -0.02 & 59960.00 & 588571.42 & 1.00 & 0.98 & 0.98 \\
13552 & 101743 & 1999 & 36909.40 & 0.65 & 3689461.00 & 31650477.16 & 1.00 & 0.86 & 0.86 \\
14468 & 101861 & 1999 & 791.90 & 0.11 & 79443.00 & 685777.13 & 1.00 & 0.87 & 0.86 \\
52137 & 302067 & 1999 & 16.50 & 0.25 & 1642.00 & 16419.07 & 1.00 & 1.00 & 1.00 \\
10128 & 101262 & 1999 & 19.50 & 0.10 & 1958.00 & 19005.38 & 1.00 & 0.97 & 0.97 \\
13803 & 101764 & 1999 & 1067.70 & 0.25 & 107214.00 & 988368.99 & 1.00 & 0.93 & 0.92 \\
48592 & 240112 & 1999 & 26.10 & 0.86 & 2604.00 & 25433.33 & 1.00 & 0.97 & 0.98 \\
11456 & 101414 & 1999 & 48.40 & 0.05 & 4820.00 & 48101.89 & 1.00 & 0.99 & 1.00 \\
54886 & 400020 & 1999 & 13.60 & 0.27 & 1368.00 & 12627.98 & 0.99 & 0.93 & 0.92 \\
11505 & 101425 & 1999 & 59.10 & 0.15 & 5250.00 & 50179.56 & 1.13 & 0.85 & 0.96 \\
13977 & 101794 & 1999 & 1181.20 & 0.44 & 117150.00 & 1099536.99 & 1.01 & 0.93 & 0.94 \\
6669 & 100908 & 1999 & 227.50 & 0.39 & 22661.00 & 208189.29 & 1.00 & 0.92 & 0.92 \\
7082 & 100996 & 1999 & 3113.70 & 0.20 & 332274.00 & 2892265.18 & 0.94 & 0.93 & 0.87 \\
54899 & 400021 & 1999 & 24.00 & 0.27 & 2404.00 & 23247.60 & 1.00 & 0.97 & 0.97 \\
54823 & 400017 & 1999 & 44.60 & 0.23 & 4533.00 & 43555.59 & 0.98 & 0.98 & 0.96 \\
11483 & 101422 & 1999 & 80.90 & 0.53 & 5482.00 & 72266.12 & 1.48 & 0.89 & 1.32 \\
47507 & 212408 & 1999 & 4724.50 & 0.27 & 472380.00 & 4538981.79 & 1.00 & 0.96 & 0.96 \\
12158 & 101513 & 1999 & 155.30 & 0.08 & 14851.00 & 148524.18 & 1.05 & 0.96 & 1.00 \\
11647 & 101455 & 1999 & 27406.80 & 0.45 & 2536220.00 & 22166786.91 & 1.08 & 0.81 & 0.87 \\
13730 & 101759 & 1999 & 101.50 & 0.20 & 8813.00 & 88305.45 & 1.15 & 0.87 & 1.00 \\
52088 & 301560 & 1999 & 478.20 & 0.15 & 44481.00 & 487688.45 & 1.08 & 1.02 & 1.10 \\
7589 & 101047 & 1999 & 354.80 & 0.10 & 35468.00 & 346210.47 & 1.00 & 0.98 & 0.98 \\
51907 & 300102 & 1999 & 127.70 & 0.11 & 12788.00 & 124862.86 & 1.00 & 0.98 & 0.98 \\
54787 & 400014 & 1999 & 6.30 & 0.29 & 628.00 & 6096.87 & 1.00 & 0.97 & 0.97 \\
12824 & 101601 & 1999 & 888.00 & -0.00 & 91090.00 & 842945.61 & 0.97 & 0.95 & 0.93 \\
48009 & 226438 & 1999 & 695.20 & 0.33 & 69471.00 & 659595.60 & 1.00 & 0.95 & 0.95 \\
48877 & 240149 & 1999 & 11.40 & -0.00 & 1187.00 & 11880.18 & 0.96 & 1.04 & 1.00 \\
48492 & 240090 & 1999 & 33.10 & 0.10 & 3307.00 & 32254.62 & 1.00 & 0.97 & 0.98 \\
13315 & 101723 & 1999 & 40.80 & 0.30 & 4086.00 & 35739.06 & 1.00 & 0.88 & 0.87 \\
52101 & 301571 & 1999 & 285.60 & 1.37 & 28727.00 & 265409.95 & 0.99 & 0.93 & 0.92 \\
8998 & 101108 & 1999 & 612.20 & 0.14 & 59194.00 & 606345.91 & 1.03 & 0.99 & 1.02 \\
48000 & 225696 & 1999 & 15.20 & 0.99 & 1512.00 & 14123.53 & 1.01 & 0.93 & 0.93 \\
10640 & 101302 & 1999 & 363.10 & -0.11 & 36310.00 & 355721.29 & 1.00 & 0.98 & 0.98 \\
48325 & 240062 & 1999 & 461.40 & 0.06 & 46133.00 & 428901.37 & 1.00 & 0.93 & 0.93 \\
10076 & 101258 & 1999 & 1974.10 & 0.31 & 170504.00 & 1740397.71 & 1.16 & 0.88 & 1.02 \\
49337 & 240284 & 1999 & 75.30 & 0.09 & 7312.00 & 73129.03 & 1.03 & 0.97 & 1.00 \\
14911 & 101922 & 1999 & 237.90 & 0.20 & 21642.00 & 213847.77 & 1.10 & 0.90 & 0.99 \\
11955 & 101473 & 1999 & 3845.00 & 0.30 & 384249.00 & 3718208.62 & 1.00 & 0.97 & 0.97 \\
7462 & 101040 & 1999 & 3614.00 & 0.22 & 337075.00 & 3586016.68 & 1.07 & 0.99 & 1.06 \\
53542 & 351713 & 1999 & 205.00 & 0.25 & 14151.00 & 173471.31 & 1.45 & 0.85 & 1.23 \\
13482 & 101741 & 1999 & 7492.60 & 0.51 & 747222.00 & 6021654.89 & 1.00 & 0.80 & 0.81 \\
13773 & 101763 & 1999 & 53.40 & 0.06 & 5377.00 & 52647.77 & 0.99 & 0.99 & 0.98 \\
12244 & 101530 & 1999 & 1195.10 & 0.25 & 123150.00 & 1110648.47 & 0.97 & 0.93 & 0.90 \\
6686 & 100910 & 1999 & 165.60 & 0.07 & 16564.00 & 161971.36 & 1.00 & 0.98 & 0.98 \\
9750 & 101186 & 1999 & 656.80 & 0.24 & 55237.00 & 621665.54 & 1.19 & 0.95 & 1.13 \\
7391 & 101038 & 1999 & 7222.60 & 0.05 & 694241.00 & 6340413.45 & 1.04 & 0.88 & 0.91 \\
52543 & 303123 & 1999 & 244.20 & 0.04 & 24387.00 & 238300.57 & 1.00 & 0.98 & 0.98 \\
47685 & 220681 & 1999 & 1374.10 & 0.43 & 137474.00 & 1250137.54 & 1.00 & 0.91 & 0.91 \\
10166 & 101264 & 1999 & 807.60 & 0.95 & 71995.00 & 749500.25 & 1.12 & 0.93 & 1.04 \\
49290 & 240266 & 1999 & 447.90 & 0.96 & 45506.00 & 432371.88 & 0.98 & 0.97 & 0.95 \\
9148 & 101115 & 1999 & 19791.70 & 0.21 & 1843747.00 & 19795457.16 & 1.07 & 1.00 & 1.07 \\
7045 & 100992 & 1999 & 1347.90 & 0.70 & 133221.00 & 1309154.66 & 1.01 & 0.97 & 0.98 \\
14242 & 101835 & 1999 & 1915.60 & 0.21 & 191384.00 & 1895427.90 & 1.00 & 0.99 & 0.99 \\
52567 & 303130 & 1999 & 21.00 & 0.12 & 2136.00 & 20926.07 & 0.98 & 1.00 & 0.98 \\
6920 & 100969 & 1999 & 61.80 & 0.20 & 6062.00 & 58455.97 & 1.02 & 0.95 & 0.96 \\
7113 & 100997 & 1999 & 127.40 & 0.14 & 13459.00 & 111431.14 & 0.95 & 0.87 & 0.83 \\
13457 & 101740 & 1999 & 42330.20 & 0.47 & 4219544.00 & 38525785.18 & 1.00 & 0.91 & 0.91 \\
53098 & 338393 & 1999 & 35.20 & 0.34 & 3521.00 & 34937.07 & 1.00 & 0.99 & 0.99 \\
48669 & 240117 & 1999 & 304.30 & -0.00 & 31942.00 & 280688.67 & 0.95 & 0.92 & 0.88 \\
9768 & 101192 & 1999 & 81.50 & 0.40 & 5715.00 & 66587.85 & 1.43 & 0.82 & 1.17 \\
13048 & 101623 & 1999 & 2568.40 & 0.10 & 255787.00 & 2444906.26 & 1.00 & 0.95 & 0.96 \\
12229 & 101528 & 1999 & 73.70 & 0.18 & 7696.00 & 74997.90 & 0.96 & 1.02 & 0.97 \\
9798 & 101193 & 1999 & 554.60 & 0.23 & 45794.00 & 513327.98 & 1.21 & 0.93 & 1.12 \\
14387 & 101853 & 1999 & 458.20 & -0.06 & 46472.00 & 403542.93 & 0.99 & 0.88 & 0.87 \\
52553 & 303124 & 1999 & 17.10 & 0.50 & 1660.00 & 16605.30 & 1.03 & 0.97 & 1.00 \\
13747 & 101762 & 1999 & 3481.10 & 0.26 & 345944.00 & 3084044.05 & 1.01 & 0.89 & 0.89 \\
12803 & 101600 & 1999 & 4589.60 & 0.20 & 472179.00 & 4444278.64 & 0.97 & 0.97 & 0.94 \\
6978 & 100978 & 1999 & 90.80 & 0.05 & 9079.00 & 89016.78 & 1.00 & 0.98 & 0.98 \\
8097 & 101074 & 1999 & 52.30 & 0.26 & 3647.00 & 47387.57 & 1.43 & 0.91 & 1.30 \\
7769 & 101057 & 1999 & 6646.30 & 0.05 & 766222.00 & 6298971.58 & 0.87 & 0.95 & 0.82 \\
52149 & 302206 & 1999 & 244.10 & 0.14 & 15805.00 & 133922.31 & 1.54 & 0.55 & 0.85 \\
13124 & 101668 & 1999 & 115.00 & 0.03 & 11497.00 & 109880.57 & 1.00 & 0.96 & 0.96 \\
6985 & 100980 & 1999 & 90.90 & 0.05 & 9118.00 & 89867.00 & 1.00 & 0.99 & 0.99 \\
12914 & 101606 & 1999 & 9013.00 & 0.39 & 727918.00 & 8080332.72 & 1.24 & 0.90 & 1.11 \\
47797 & 221485 & 1999 & 1087.80 & 0.12 & 97802.00 & 1004608.05 & 1.11 & 0.92 & 1.03 \\
6727 & 100925 & 1999 & 45.90 & 0.07 & 4785.00 & 47836.77 & 0.96 & 1.04 & 1.00 \\
47674 & 217585 & 1999 & 291.30 & 0.06 & 24096.00 & 251442.72 & 1.21 & 0.86 & 1.04 \\
47642 & 216438 & 1999 & 541.80 & 0.24 & 54085.00 & 492297.79 & 1.00 & 0.91 & 0.91 \\
52160 & 302545 & 1999 & 52.70 & 0.48 & 5161.00 & 48961.31 & 1.02 & 0.93 & 0.95 \\
47861 & 222408 & 1999 & 463.30 & 0.15 & 46281.00 & 451409.01 & 1.00 & 0.97 & 0.98 \\
12220 & 101523 & 1999 & 592.40 & 0.02 & 63355.00 & 642474.62 & 0.94 & 1.08 & 1.01 \\
11433 & 101402 & 1999 & 8.80 & 0.09 & 879.00 & 7411.37 & 1.00 & 0.84 & 0.84 \\
48523 & 240103 & 1999 & 45.70 & 0.06 & 4412.00 & 43015.62 & 1.04 & 0.94 & 0.97 \\
10145 & 101263 & 1999 & 538.60 & 0.15 & 49436.00 & 530818.95 & 1.09 & 0.99 & 1.07 \\
48471 & 240087 & 1999 & 60.30 & 0.04 & 6026.00 & 58096.73 & 1.00 & 0.96 & 0.96 \\
14830 & 101916 & 1999 & 395.50 & 0.00 & 39604.00 & 375355.20 & 1.00 & 0.95 & 0.95 \\
8028 & 101071 & 1999 & 2286.20 & 0.40 & 182948.00 & 2029513.40 & 1.25 & 0.89 & 1.11 \\
14353 & 101851 & 1999 & 2693.20 & 0.36 & 269302.00 & 2263612.58 & 1.00 & 0.84 & 0.84 \\
10477 & 101287 & 1999 & 439.00 & -0.02 & 43899.00 & 357344.84 & 1.00 & 0.81 & 0.81 \\
47836 & 222351 & 1999 & 239.80 & 0.03 & 25842.00 & 220141.35 & 0.93 & 0.92 & 0.85 \\
49062 & 240212 & 1999 & 5030.90 & -0.05 & 519659.00 & 4615796.68 & 0.97 & 0.92 & 0.89 \\
13954 & 101789 & 1999 & 880.70 & 0.19 & 91272.00 & 845414.44 & 0.96 & 0.96 & 0.93 \\
10461 & 101286 & 1999 & 1073.60 & 0.52 & 107366.00 & 946288.18 & 1.00 & 0.88 & 0.88 \\
48608 & 240114 & 1999 & 425.30 & 0.25 & 41940.00 & 419527.69 & 1.01 & 0.99 & 1.00 \\
7145 & 100998 & 1999 & 117.40 & 0.18 & 13870.00 & 100064.63 & 0.85 & 0.85 & 0.72 \\
14196 & 101820 & 1999 & 570.20 & 0.17 & 73037.00 & 707443.78 & 0.78 & 1.24 & 0.97 \\
9460 & 101137 & 1999 & 13.30 & 0.10 & 1354.00 & 12735.57 & 0.98 & 0.96 & 0.94 \\
48997 & 240198 & 1999 & 2341.90 & 0.03 & 234198.00 & 2297337.03 & 1.00 & 0.98 & 0.98 \\
47946 & 225413 & 1999 & 20.50 & -0.03 & 2017.00 & 19518.46 & 1.02 & 0.95 & 0.97 \\
47919 & 222809 & 1999 & 64.50 & 0.02 & 6049.00 & 60238.41 & 1.07 & 0.93 & 1.00 \\
8819 & 101100 & 1999 & 618.70 & 0.73 & 79733.00 & 749057.85 & 0.78 & 1.21 & 0.94 \\
51951 & 300673 & 1999 & 31.50 & 0.06 & 3135.00 & 29789.40 & 1.00 & 0.95 & 0.95 \\
12546 & 101553 & 1999 & 840.80 & 0.22 & 84590.00 & 878499.38 & 0.99 & 1.04 & 1.04 \\
14628 & 101903 & 1999 & 402.90 & 0.11 & 35263.00 & 320806.86 & 1.14 & 0.80 & 0.91 \\
9447 & 101135 & 1999 & 1518.50 & 0.01 & 151769.00 & 1420085.07 & 1.00 & 0.94 & 0.94 \\
6812 & 100958 & 1999 & 5.20 & 0.26 & 562.00 & 5617.77 & 0.93 & 1.08 & 1.00 \\
52393 & 302879 & 1999 & 29.90 & 0.27 & 2129.00 & 24715.48 & 1.40 & 0.83 & 1.16 \\
9300 & 101128 & 1999 & 25.00 & 0.18 & 2562.00 & 26387.91 & 0.98 & 1.06 & 1.03 \\
6834 & 100962 & 1999 & 3136.20 & 0.37 & 254472.00 & 2809546.88 & 1.23 & 0.90 & 1.10 \\
48933 & 240154 & 1999 & 33.80 & 0.03 & 3144.00 & 32155.77 & 1.08 & 0.95 & 1.02 \\
11218 & 101376 & 1999 & 922.10 & 0.13 & 92632.00 & 926459.84 & 1.00 & 1.00 & 1.00 \\
48684 & 240121 & 1999 & 1813.00 & 0.50 & 186043.00 & 1592336.43 & 0.97 & 0.88 & 0.86 \\
48714 & 240130 & 1999 & 2874.30 & 0.88 & 287691.00 & 2727075.25 & 1.00 & 0.95 & 0.95 \\
52339 & 302780 & 1999 & 158.70 & 0.03 & 14890.00 & 135413.01 & 1.07 & 0.85 & 0.91 \\
6700 & 100913 & 1999 & 268.50 & 0.16 & 26961.00 & 244226.47 & 1.00 & 0.91 & 0.91 \\
9495 & 101140 & 1999 & 1637.43 & 0.38 & 163977.00 & 1623860.46 & 1.00 & 0.99 & 0.99 \\
6878 & 100967 & 1999 & 373.00 & -0.05 & 37129.00 & 325764.80 & 1.00 & 0.87 & 0.88 \\
14021 & 101800 & 1999 & 1202.20 & 0.37 & 120215.00 & 1120686.47 & 1.00 & 0.93 & 0.93 \\
9279 & 101127 & 1999 & 255.70 & 0.20 & 22743.00 & 236270.09 & 1.12 & 0.92 & 1.04 \\
14554 & 101876 & 1999 & 255.70 & 0.16 & 25567.00 & 245660.25 & 1.00 & 0.96 & 0.96 \\
15008 & 101933 & 1999 & 273.60 & 0.27 & 26425.00 & 262131.62 & 1.04 & 0.96 & 0.99 \\
12064 & 101494 & 1999 & 855.50 & 0.20 & 73519.00 & 852480.40 & 1.16 & 1.00 & 1.16 \\
9483 & 101139 & 1999 & 43.50 & 0.01 & 4356.00 & 43369.27 & 1.00 & 1.00 & 1.00 \\
7832 & 101062 & 1999 & 724.40 & -0.01 & 92741.00 & 750653.33 & 0.78 & 1.04 & 0.81 \\
7977 & 101068 & 1999 & 105938.50 & 0.24 & 9228720.00 & 97321452.06 & 1.15 & 0.92 & 1.05 \\
12505 & 101544 & 1999 & 212.70 & 0.65 & 20984.00 & 209904.07 & 1.01 & 0.99 & 1.00 \\
10321 & 101279 & 1999 & 43.12 & 0.25 & 4312.00 & 40561.84 & 1.00 & 0.94 & 0.94 \\
12015 & 101477 & 1999 & 172.90 & 0.71 & 17139.00 & 171389.49 & 1.01 & 0.99 & 1.00 \\
10702 & 101312 & 1999 & 10646.80 & 0.97 & 1064674.00 & 9598219.61 & 1.00 & 0.90 & 0.90 \\
48582 & 240111 & 1999 & 601.10 & 0.24 & 62624.00 & 565803.04 & 0.96 & 0.94 & 0.90 \\
7347 & 101023 & 1999 & 22828.60 & 0.07 & 2446520.00 & 21510499.20 & 0.93 & 0.94 & 0.88 \\
14328 & 101850 & 1999 & 328.00 & 0.05 & 32272.00 & 322818.17 & 1.02 & 0.98 & 1.00 \\
12517 & 101545 & 1999 & 189.90 & 0.07 & 18739.00 & 182366.38 & 1.01 & 0.96 & 0.97 \\
52407 & 302881 & 1999 & 204.60 & 0.16 & 20477.00 & 203668.99 & 1.00 & 1.00 & 0.99 \\
50087 & 240392 & 1999 & 279.70 & 0.20 & 29399.00 & 261226.34 & 0.95 & 0.93 & 0.89 \\
11843 & 101463 & 1999 & 1369.60 & 0.23 & 121365.00 & 1350846.15 & 1.13 & 0.99 & 1.11 \\
7016 & 100985 & 1999 & 489.80 & 0.43 & 49841.00 & 448937.69 & 0.98 & 0.92 & 0.90 \\
6870 & 100966 & 1999 & 15.00 & -0.09 & 1550.00 & 15048.85 & 0.97 & 1.00 & 0.97 \\
9322 & 101131 & 1999 & 4609.83 & 0.72 & 462779.00 & 4313160.10 & 1.00 & 0.94 & 0.93 \\
49073 & 240218 & 1999 & 277.90 & 1.83 & 28023.00 & 225440.70 & 0.99 & 0.81 & 0.80 \\
47606 & 215687 & 1999 & 70.70 & 0.19 & 7022.00 & 63263.29 & 1.01 & 0.89 & 0.90 \\
11121 & 101368 & 1999 & 1205.30 & 0.32 & 120466.00 & 1113360.25 & 1.00 & 0.92 & 0.92 \\
10382 & 101284 & 1999 & 1955.80 & 0.94 & 195586.00 & 1746582.20 & 1.00 & 0.89 & 0.89 \\
7907 & 101065 & 1999 & 1622.80 & 0.23 & 116166.00 & 1413194.19 & 1.40 & 0.87 & 1.22 \\
12595 & 101557 & 1999 & 87.40 & 0.03 & 8988.00 & 87864.36 & 0.97 & 1.01 & 0.98 \\
9386 & 101133 & 1999 & 1123.24 & 0.01 & 112149.00 & 1067458.51 & 1.00 & 0.95 & 0.95 \\
9349 & 101132 & 1999 & 170.85 & 0.42 & 17158.00 & 171359.04 & 1.00 & 1.00 & 1.00 \\
13221 & 101708 & 1999 & 633.60 & 0.09 & 61805.00 & 566880.31 & 1.03 & 0.89 & 0.92 \\
48561 & 240107 & 1999 & 17.90 & 0.50 & 1735.00 & 16271.45 & 1.03 & 0.91 & 0.94 \\
54813 & 400015 & 1999 & 21.90 & 1.86 & 2155.00 & 21092.62 & 1.02 & 0.96 & 0.98 \\
11157 & 101369 & 1999 & 3158.50 & 0.46 & 315021.00 & 3103338.90 & 1.00 & 0.98 & 0.99 \\
15044 & 101953 & 1999 & 468.50 & 0.12 & 48222.00 & 464195.12 & 0.97 & 0.99 & 0.96 \\
10731 & 101320 & 1999 & 38.70 & 0.36 & 3868.00 & 35840.37 & 1.00 & 0.93 & 0.93 \\
13413 & 101738 & 1999 & 4434.50 & 1.22 & 441464.00 & 4294370.72 & 1.00 & 0.97 & 0.97 \\
7639 & 101050 & 1999 & 609.40 & 0.46 & 61543.00 & 551490.02 & 0.99 & 0.90 & 0.90 \\
12609 & 101560 & 1999 & 34.40 & 0.08 & 3176.00 & 32767.66 & 1.08 & 0.95 & 1.03 \\
48211 & 240051 & 1999 & 550.10 & 0.12 & 55063.00 & 545242.13 & 1.00 & 0.99 & 0.99 \\
7557 & 101045 & 1999 & 17579.90 & 0.04 & 1957261.00 & 18403659.17 & 0.90 & 1.05 & 0.94 \\
49010 & 240199 & 1999 & 2522.80 & 0.71 & 252273.00 & 2305236.55 & 1.00 & 0.91 & 0.91 \\
52360 & 302813 & 1999 & 10.80 & 0.13 & 1097.00 & 10223.77 & 0.98 & 0.95 & 0.93 \\
12881 & 101603 & 1999 & 1698.70 & 0.33 & 172777.00 & 1696644.06 & 0.98 & 1.00 & 0.98 \\
7200 & 101013 & 1999 & 28432.30 & 0.04 & 2873455.00 & 24294927.77 & 0.99 & 0.85 & 0.85 \\
13839 & 101769 & 1999 & 3249.30 & 0.36 & 325303.00 & 3112888.87 & 1.00 & 0.96 & 0.96 \\
10352 & 101283 & 1999 & 1136.10 & -0.02 & 113610.00 & 1036363.77 & 1.00 & 0.91 & 0.91 \\
12639 & 101561 & 1999 & 163.10 & 0.96 & 16345.00 & 156724.78 & 1.00 & 0.96 & 0.96 \\
52348 & 302811 & 1999 & 10.80 & 0.32 & 1085.00 & 10275.15 & 1.00 & 0.95 & 0.95 \\
48902 & 240152 & 1999 & 219.90 & 0.02 & 21913.00 & 203420.52 & 1.00 & 0.93 & 0.93 \\
12570 & 101554 & 1999 & 531.20 & 0.01 & 71717.00 & 606781.75 & 0.74 & 1.14 & 0.85 \\
11207 & 101375 & 1999 & 18.00 & 0.29 & 1773.00 & 17689.87 & 1.02 & 0.98 & 1.00 \\
14302 & 101843 & 1999 & 500.30 & 0.04 & 48820.00 & 488236.16 & 1.02 & 0.98 & 1.00 \\
8529 & 101089 & 1999 & 46.20 & -0.06 & 5638.00 & 52687.00 & 0.82 & 1.14 & 0.93 \\
52616 & 303175 & 1999 & 1238.50 & 0.32 & 90295.00 & 1004684.19 & 1.37 & 0.81 & 1.11 \\
48103 & 240010 & 1999 & 182.00 & -0.07 & 17622.00 & 174169.46 & 1.03 & 0.96 & 0.99 \\
48346 & 240065 & 1999 & 612.50 & 0.07 & 61261.00 & 574764.18 & 1.00 & 0.94 & 0.94 \\
8276 & 101081 & 1999 & 435.10 & 0.02 & 47854.00 & 461547.07 & 0.91 & 1.06 & 0.96 \\
11780 & 101461 & 1999 & 3285.60 & 0.27 & 232585.00 & 2713864.58 & 1.41 & 0.83 & 1.17 \\
48373 & 240067 & 1999 & 263.60 & -0.01 & 26425.00 & 232620.82 & 1.00 & 0.88 & 0.88 \\
9412 & 101134 & 1999 & 95.99 & -0.09 & 9596.00 & 77478.56 & 1.00 & 0.81 & 0.81 \\
12999 & 101618 & 1999 & 468.70 & -0.02 & 47023.00 & 464984.71 & 1.00 & 0.99 & 0.99 \\
11195 & 101370 & 1999 & 13.30 & 0.16 & 1328.00 & 12538.29 & 1.00 & 0.94 & 0.94 \\
53443 & 350408 & 1999 & 9.30 & 0.17 & 914.00 & 8736.36 & 1.02 & 0.94 & 0.96 \\
11811 & 101462 & 1999 & 1036.60 & 0.06 & 163961.00 & 1583464.99 & 0.63 & 1.53 & 0.97 \\
48179 & 240040 & 1999 & 328.90 & 0.34 & 35463.00 & 340594.63 & 0.93 & 1.04 & 0.96 \\
8566 & 101090 & 1999 & 3205.40 & 0.81 & 228797.00 & 2802854.89 & 1.40 & 0.87 & 1.23 \\
11087 & 101367 & 1999 & 338.20 & 0.14 & 33856.00 & 322408.20 & 1.00 & 0.95 & 0.95 \\
9859 & 101198 & 1999 & 250.30 & 0.27 & 20635.00 & 246553.19 & 1.21 & 0.99 & 1.19 \\
11559 & 101430 & 1999 & 139.40 & -0.01 & 14006.00 & 138560.13 & 1.00 & 0.99 & 0.99 \\
52294 & 302760 & 1999 & 20.70 & 0.28 & 2053.00 & 18792.22 & 1.01 & 0.91 & 0.92 \\
7736 & 101056 & 1999 & 48528.30 & 0.10 & 4842727.00 & 48098659.82 & 1.00 & 0.99 & 0.99 \\
49146 & 240234 & 1999 & 59.40 & -0.06 & 5874.00 & 53764.02 & 1.01 & 0.91 & 0.92 \\
51994 & 300695 & 1999 & 86.30 & 0.04 & 8646.00 & 83492.28 & 1.00 & 0.97 & 0.97 \\
55095 & 400061 & 1999 & 69.60 & 0.03 & 6537.00 & 62238.93 & 1.06 & 0.89 & 0.95 \\
47621 & 215696 & 1999 & 333.40 & 0.72 & 21466.00 & 320308.29 & 1.55 & 0.96 & 1.49 \\
52320 & 302763 & 1999 & 23.80 & 0.27 & 2382.00 & 22279.66 & 1.00 & 0.94 & 0.94 \\
6636 & 100906 & 1999 & 1771.80 & 0.26 & 177214.00 & 1750022.99 & 1.00 & 0.99 & 0.99 \\
12704 & 101588 & 1999 & 65.70 & 0.43 & 6584.00 & 61111.93 & 1.00 & 0.93 & 0.93 \\
14984 & 101926 & 1999 & 542.50 & 0.07 & 55722.00 & 524581.30 & 0.97 & 0.97 & 0.94 \\
14874 & 101919 & 1999 & 2916.20 & 0.38 & 293083.00 & 2750493.69 & 1.00 & 0.94 & 0.94 \\
51943 & 300657 & 1999 & 59.60 & 0.42 & 4823.00 & 48309.81 & 1.24 & 0.81 & 1.00 \\
10292 & 101278 & 1999 & 148.28 & 0.02 & 13893.00 & 138893.81 & 1.07 & 0.94 & 1.00 \\
47974 & 225484 & 1999 & 98.50 & 0.15 & 9877.00 & 97691.39 & 1.00 & 0.99 & 0.99 \\
13181 & 101703 & 1999 & 56095.20 & 0.17 & 4776173.00 & 46064787.43 & 1.17 & 0.82 & 0.96 \\
8009 & 101069 & 1999 & 10437.80 & 0.21 & 994403.00 & 10043399.11 & 1.05 & 0.96 & 1.01 \\
10507 & 101294 & 1999 & 56.40 & 0.54 & 5639.00 & 46518.75 & 1.00 & 0.82 & 0.82 \\
12443 & 101541 & 1999 & 312.10 & 0.40 & 29418.00 & 294195.42 & 1.06 & 0.94 & 1.00 \\
14447 & 101858 & 1999 & 349.10 & 0.23 & 34874.00 & 304285.61 & 1.00 & 0.87 & 0.87 \\
11874 & 101464 & 1999 & 221.80 & -0.02 & 28589.00 & 260221.44 & 0.78 & 1.17 & 0.91 \\
13883 & 101785 & 1999 & 4307.10 & 1.10 & 383426.00 & 3754286.15 & 1.12 & 0.87 & 0.98 \\
6939 & 100973 & 1999 & 58.10 & 0.04 & 5794.00 & 57227.24 & 1.00 & 0.98 & 0.99 \\
14275 & 101842 & 1999 & 3523.40 & 0.18 & 352391.00 & 3492250.41 & 1.00 & 0.99 & 0.99 \\
47744 & 221051 & 1999 & 7268.50 & 0.47 & 524223.00 & 6156233.10 & 1.39 & 0.85 & 1.17 \\
12389 & 101538 & 1999 & 112.60 & 0.62 & 11260.00 & 111349.79 & 1.00 & 0.99 & 0.99 \\
52287 & 302732 & 1999 & 466.40 & 1.09 & 46714.00 & 440341.74 & 1.00 & 0.94 & 0.94 \\
6990 & 100981 & 1999 & 87.60 & 0.39 & 8769.00 & 86652.67 & 1.00 & 0.99 & 0.99 \\
12719 & 101590 & 1999 & 9.40 & 0.20 & 954.00 & 8682.14 & 0.99 & 0.92 & 0.91 \\
11315 & 101393 & 1999 & 783.30 & 0.19 & 80319.00 & 751980.55 & 0.98 & 0.96 & 0.94 \\
11023 & 101360 & 1999 & 1993.90 & 0.28 & 172755.00 & 1779670.64 & 1.15 & 0.89 & 1.03 \\
7664 & 101054 & 1999 & 16114.30 & 0.24 & 1609997.00 & 13812330.62 & 1.00 & 0.86 & 0.86 \\
13935 & 101788 & 1999 & 923.60 & 0.20 & 93774.00 & 908983.29 & 0.98 & 0.98 & 0.97 \\
12949 & 101616 & 1999 & 28704.90 & 0.31 & 2395828.00 & 25382112.09 & 1.20 & 0.88 & 1.06 \\
14658 & 101906 & 1999 & 31.00 & 0.18 & 3211.00 & 26631.21 & 0.97 & 0.86 & 0.83 \\
13168 & 101698 & 1999 & 324.64 & 0.01 & 32395.00 & 319585.67 & 1.00 & 0.98 & 0.99 \\
9582 & 101151 & 1999 & 212.15 & 0.20 & 21183.00 & 197261.06 & 1.00 & 0.93 & 0.93 \\
48039 & 227155 & 1999 & 46.00 & 0.42 & 4736.00 & 44461.89 & 0.97 & 0.97 & 0.94 \\
13329 & 101728 & 1999 & 198.20 & 0.62 & 19680.00 & 196864.20 & 1.01 & 0.99 & 1.00 \\
7164 & 101000 & 1999 & 1684.00 & 0.06 & 167460.00 & 1644749.05 & 1.01 & 0.98 & 0.98 \\
12409 & 101539 & 1999 & 1231.90 & 0.11 & 122900.00 & 1205187.06 & 1.00 & 0.98 & 0.98 \\
52471 & 302944 & 1999 & 37.00 & 0.16 & 3696.00 & 34161.05 & 1.00 & 0.92 & 0.92 \\
12081 & 101497 & 1999 & 2695.70 & 0.06 & 284599.00 & 2470424.85 & 0.95 & 0.92 & 0.87 \\
12712 & 101589 & 1999 & 38.00 & 0.27 & 3835.00 & 35222.11 & 0.99 & 0.93 & 0.92 \\
9544 & 101149 & 1999 & 5654.74 & 1.45 & 565303.00 & 5216201.35 & 1.00 & 0.92 & 0.92 \\
11283 & 101390 & 1999 & 4932.40 & 0.20 & 493039.00 & 4849920.85 & 1.00 & 0.98 & 0.98 \\
9217 & 101119 & 1999 & 40.90 & 0.06 & 4082.00 & 40856.35 & 1.00 & 1.00 & 1.00 \\
6747 & 100947 & 1999 & 2187.20 & 0.15 & 219061.00 & 2141359.09 & 1.00 & 0.98 & 0.98 \\
9957 & 101215 & 1999 & 52.90 & 0.21 & 6531.00 & 60490.51 & 0.81 & 1.14 & 0.93 \\
12670 & 101562 & 1999 & 375.70 & -0.03 & 40663.00 & 385983.06 & 0.92 & 1.03 & 0.95 \\
14648 & 101905 & 1999 & 18.30 & 0.13 & 1808.00 & 14750.25 & 1.01 & 0.81 & 0.82 \\
13013 & 101621 & 1999 & 5089.90 & 0.16 & 491120.00 & 4357722.49 & 1.04 & 0.86 & 0.89 \\
55057 & 400050 & 1999 & 186.70 & -0.03 & 19572.00 & 181436.13 & 0.95 & 0.97 & 0.93 \\
8208 & 101079 & 1999 & 193.80 & 0.34 & 24164.00 & 214379.04 & 0.80 & 1.11 & 0.89 \\
48643 & 240116 & 1999 & 211.70 & 1.76 & 22620.00 & 214143.64 & 0.94 & 1.01 & 0.95 \\
11259 & 101380 & 1999 & 273.20 & 0.12 & 27244.00 & 259946.23 & 1.00 & 0.95 & 0.95 \\
53459 & 350572 & 1999 & 94.70 & 0.42 & 9334.00 & 93372.06 & 1.01 & 0.99 & 1.00 \\
9508 & 101141 & 1999 & 4712.36 & 0.13 & 472504.00 & 4030495.42 & 1.00 & 0.86 & 0.85 \\
14997 & 101930 & 1999 & 1646.00 & 0.20 & 156626.00 & 1566769.87 & 1.05 & 0.95 & 1.00 \\
12968 & 101617 & 1999 & 58.40 & -0.00 & 6011.00 & 47254.84 & 0.97 & 0.81 & 0.79 \\
11248 & 101379 & 1999 & 1047.10 & 0.20 & 104638.00 & 1035163.55 & 1.00 & 0.99 & 0.99 \\
48983 & 240197 & 1999 & 1801.60 & 0.36 & 180162.00 & 1734374.57 & 1.00 & 0.96 & 0.96 \\
48404 & 240076 & 1999 & 174.60 & 0.28 & 17463.00 & 169381.72 & 1.00 & 0.97 & 0.97 \\
48725 & 240134 & 1999 & 21.90 & -0.06 & 2186.00 & 21289.74 & 1.00 & 0.97 & 0.97 \\
54595 & 377010 & 1999 & 3.60 & 0.39 & 277.00 & 3033.28 & 1.30 & 0.84 & 1.10 \\
13916 & 101787 & 1999 & 772.70 & 0.28 & 79014.00 & 748154.71 & 0.98 & 0.97 & 0.95 \\
52413 & 302907 & 1999 & 32.20 & 0.25 & 3011.00 & 30115.25 & 1.07 & 0.94 & 1.00 \\
11599 & 101431 & 1999 & 476.90 & 0.17 & 47707.00 & 472654.29 & 1.00 & 0.99 & 0.99 \\
14804 & 101914 & 1999 & 64.60 & -0.01 & 6485.00 & 63550.91 & 1.00 & 0.98 & 0.98 \\
52439 & 302941 & 1999 & 20.40 & 0.14 & 1572.00 & 16743.90 & 1.30 & 0.82 & 1.07 \\
47538 & 212658 & 1999 & 5243.50 & 0.24 & 522919.00 & 4817323.07 & 1.00 & 0.92 & 0.92 \\
13513 & 101742 & 1999 & 13543.10 & 0.96 & 1325191.00 & 11854487.89 & 1.02 & 0.88 & 0.89 \\
7277 & 101018 & 1999 & 27876.30 & 0.24 & 2371022.00 & 25727156.93 & 1.18 & 0.92 & 1.09 \\
11747 & 101460 & 1999 & 4404.90 & 0.08 & 440642.00 & 4183847.27 & 1.00 & 0.95 & 0.95 \\
49395 & 240291 & 1999 & 52.80 & 0.96 & 5251.00 & 51969.08 & 1.01 & 0.98 & 0.99 \\
49105 & 240222 & 1999 & 1182.20 & 0.10 & 110885.00 & 1109153.67 & 1.07 & 0.94 & 1.00 \\
11270 & 101381 & 1999 & 49.90 & -0.09 & 4888.00 & 48790.30 & 1.02 & 0.98 & 1.00 \\
48601 & 240113 & 1999 & 62.90 & 0.24 & 6001.00 & 60024.45 & 1.05 & 0.95 & 1.00 \\
11055 & 101364 & 1999 & 88.90 & 0.09 & 9002.00 & 81617.94 & 0.99 & 0.92 & 0.91 \\
9083 & 101111 & 1999 & 514.70 & -0.10 & 65714.00 & 619506.31 & 0.78 & 1.20 & 0.94 \\
47837 & 222351 & 2000 & 154.40 & -0.28 & 16395.00 & 145950.36 & 0.94 & 0.95 & 0.89 \\
37774 & 107143 & 2000 & 4.00 & 0.22 & 615.00 & 5702.92 & 0.65 & 1.43 & 0.93 \\
14404 & 101854 & 2000 & 11774.60 & -0.07 & 1177529.00 & 10644778.03 & 1.00 & 0.90 & 0.90 \\
27201 & 105252 & 2000 & 30.10 & -0.16 & 3014.00 & 29298.08 & 1.00 & 0.97 & 0.97 \\
45870 & 200151 & 2000 & 29.30 & 0.00 & 2876.00 & 27296.50 & 1.02 & 0.93 & 0.95 \\
44398 & 109300 & 2000 & 769.40 & 0.01 & 76838.00 & 754464.55 & 1.00 & 0.98 & 0.98 \\
27210 & 105253 & 2000 & 31.20 & 0.15 & 3117.00 & 30423.34 & 1.00 & 0.98 & 0.98 \\
37711 & 107004 & 2000 & 144.30 & -0.14 & 14846.00 & 141282.68 & 0.97 & 0.98 & 0.95 \\
13072 & 101626 & 2000 & 2673.10 & 0.17 & 268682.00 & 2465840.55 & 0.99 & 0.92 & 0.92 \\
37777 & 107144 & 2000 & 193.60 & -0.29 & 22780.00 & 188576.39 & 0.85 & 0.97 & 0.83 \\
41943 & 108871 & 2000 & 30.40 & -0.06 & 3044.00 & 29005.60 & 1.00 & 0.95 & 0.95 \\
27191 & 105250 & 2000 & 10.00 & 0.16 & 1000.00 & 8837.26 & 1.00 & 0.88 & 0.88 \\
27246 & 105259 & 2000 & 410.10 & -0.09 & 46312.00 & 460141.10 & 0.89 & 1.12 & 0.99 \\
2853 & 100365 & 2000 & 667.90 & -0.03 & 67548.00 & 661837.28 & 0.99 & 0.99 & 0.98 \\
57920 & 402013 & 2000 & 110.70 & -0.29 & 17785.00 & 103745.14 & 0.62 & 0.94 & 0.58 \\
37681 & 106995 & 2000 & 1385.80 & -0.18 & 132413.00 & 1272012.91 & 1.05 & 0.92 & 0.96 \\
37670 & 106993 & 2000 & 41.20 & 0.26 & 3768.00 & 40776.04 & 1.09 & 0.99 & 1.08 \\
4304 & 100603 & 2000 & 1312.20 & 0.27 & 129098.00 & 1290860.58 & 1.02 & 0.98 & 1.00 \\
74628 & 601142 & 2000 & 483.80 & -0.05 & 54579.00 & 467347.67 & 0.89 & 0.97 & 0.86 \\
37742 & 107136 & 2000 & 111.10 & -0.04 & 11184.00 & 111866.53 & 0.99 & 1.01 & 1.00 \\
17148 & 102259 & 2000 & 346.30 & 0.16 & 35042.00 & 321392.64 & 0.99 & 0.93 & 0.92 \\
5325 & 100753 & 2000 & 1498.60 & 0.09 & 152779.00 & 1302276.02 & 0.98 & 0.87 & 0.85 \\
8359 & 101084 & 2000 & 4287.30 & -0.32 & 805016.00 & 3877087.06 & 0.53 & 0.90 & 0.48 \\
37744 & 107137 & 2000 & 85.50 & -0.13 & 8201.00 & 82003.32 & 1.04 & 0.96 & 1.00 \\
41919 & 108870 & 2000 & 22.70 & 0.01 & 2265.00 & 22597.43 & 1.00 & 1.00 & 1.00 \\
17138 & 102258 & 2000 & 723.30 & -0.12 & 72365.00 & 705447.39 & 1.00 & 0.98 & 0.97 \\
18911 & 102527 & 2000 & 262.40 & -0.13 & 32582.00 & 297082.53 & 0.81 & 1.13 & 0.91 \\
27219 & 105256 & 2000 & 30.30 & 0.28 & 3215.00 & 30104.05 & 0.94 & 0.99 & 0.94 \\
37718 & 107135 & 2000 & 36.50 & 0.13 & 3628.00 & 35901.26 & 1.01 & 0.98 & 0.99 \\
22186 & 102994 & 2000 & 96.90 & -0.01 & 8891.00 & 78304.51 & 1.09 & 0.81 & 0.88 \\
37749 & 107141 & 2000 & 49.20 & 0.34 & 4916.00 & 47153.51 & 1.00 & 0.96 & 0.96 \\
11600 & 101431 & 2000 & 401.10 & 0.08 & 41574.00 & 388844.48 & 0.96 & 0.97 & 0.94 \\
41946 & 108874 & 2000 & 35.50 & -0.10 & 3552.00 & 34132.41 & 1.00 & 0.96 & 0.96 \\
13884 & 101785 & 2000 & 3861.30 & -0.25 & 537493.00 & 3429577.68 & 0.72 & 0.89 & 0.64 \\
631 & 100085 & 2000 & 25028.30 & 1.54 & 2776886.00 & 23436218.17 & 0.90 & 0.94 & 0.84 \\
47812 & 222027 & 2000 & 1686.40 & 0.07 & 168633.00 & 1584508.62 & 1.00 & 0.94 & 0.94 \\
17162 & 102261 & 2000 & 1137.40 & 0.04 & 113383.00 & 1109643.54 & 1.00 & 0.98 & 0.98 \\
657 & 100087 & 2000 & 9442.30 & -0.21 & 1192317.00 & 7604605.21 & 0.79 & 0.81 & 0.64 \\
53653 & 355965 & 2000 & 382.40 & 0.09 & 38236.00 & 377471.79 & 1.00 & 0.99 & 0.99 \\
27132 & 105246 & 2000 & 2092.20 & 0.26 & 177565.00 & 1783727.32 & 1.18 & 0.85 & 1.00 \\
49632 & 240327 & 2000 & 153.40 & -0.06 & 18223.00 & 148902.17 & 0.84 & 0.97 & 0.82 \\
26762 & 103606 & 2000 & 19.30 & 0.04 & 1963.00 & 18093.69 & 0.98 & 0.94 & 0.92 \\
54596 & 377010 & 2000 & 5.30 & 0.09 & 498.00 & 4976.85 & 1.06 & 0.94 & 1.00 \\
38080 & 107199 & 2000 & 29.80 & -0.04 & 2947.00 & 29469.62 & 1.01 & 0.99 & 1.00 \\
38064 & 107198 & 2000 & 96.60 & 0.18 & 8997.00 & 90002.06 & 1.07 & 0.93 & 1.00 \\
73602 & 600473 & 2000 & 65.20 & -0.12 & 8251.00 & 69127.03 & 0.79 & 1.06 & 0.84 \\
13014 & 101621 & 2000 & 4391.80 & -0.08 & 466847.00 & 3813374.05 & 0.94 & 0.87 & 0.82 \\
26794 & 103607 & 2000 & 175.70 & -0.18 & 18772.00 & 171158.27 & 0.94 & 0.97 & 0.91 \\
2752 & 100357 & 2000 & 300.50 & 0.04 & 29981.00 & 297603.98 & 1.00 & 0.99 & 0.99 \\
38039 & 107196 & 2000 & 52.40 & 0.25 & 5223.00 & 48596.38 & 1.00 & 0.93 & 0.93 \\
17310 & 102280 & 2000 & 2611.80 & 0.00 & 261902.00 & 2521240.17 & 1.00 & 0.97 & 0.96 \\
14329 & 101850 & 2000 & 295.00 & 0.01 & 29719.00 & 284236.70 & 0.99 & 0.96 & 0.96 \\
22377 & 103008 & 2000 & 257.20 & -0.05 & 25290.00 & 252870.98 & 1.02 & 0.98 & 1.00 \\
38035 & 107193 & 2000 & 12.20 & -0.28 & 1419.00 & 11318.31 & 0.86 & 0.93 & 0.80 \\
26821 & 103608 & 2000 & 52.60 & 0.01 & 5257.00 & 49259.98 & 1.00 & 0.94 & 0.94 \\
41756 & 108853 & 2000 & 402.50 & 0.31 & 40149.00 & 346061.58 & 1.00 & 0.86 & 0.86 \\
59108 & 410433 & 2000 & 510.60 & 0.13 & 51368.00 & 500518.76 & 0.99 & 0.98 & 0.97 \\
38016 & 107192 & 2000 & 376.10 & 0.21 & 19242.00 & 168596.45 & 1.95 & 0.45 & 0.88 \\
5197 & 100731 & 2000 & 13161.90 & 0.06 & 1315869.00 & 12964113.50 & 1.00 & 0.98 & 0.99 \\
26844 & 103609 & 2000 & 32.30 & 0.20 & 3292.00 & 29941.91 & 0.98 & 0.93 & 0.91 \\
38012 & 107187 & 2000 & 107.80 & -0.05 & 10772.00 & 104725.50 & 1.00 & 0.97 & 0.97 \\
41783 & 108856 & 2000 & 19.20 & 0.22 & 1787.00 & 17872.73 & 1.07 & 0.93 & 1.00 \\
6991 & 100981 & 2000 & 91.70 & -0.14 & 9172.00 & 90035.66 & 1.00 & 0.98 & 0.98 \\
38006 & 107185 & 2000 & 43.70 & -0.26 & 4397.00 & 43425.55 & 0.99 & 0.99 & 0.99 \\
26860 & 103614 & 2000 & 234.20 & 0.12 & 22777.00 & 227717.23 & 1.03 & 0.97 & 1.00 \\
803 & 100097 & 2000 & 244.60 & -0.26 & 24461.00 & 223725.86 & 1.00 & 0.91 & 0.91 \\
41750 & 108852 & 2000 & 86.80 & -0.15 & 8061.00 & 75611.43 & 1.08 & 0.87 & 0.94 \\
38272 & 107234 & 2000 & 14.10 & -0.02 & 1408.00 & 13686.43 & 1.00 & 0.97 & 0.97 \\
833 & 100098 & 2000 & 187.10 & 0.17 & 18715.00 & 150208.23 & 1.00 & 0.80 & 0.80 \\
26618 & 103593 & 2000 & 47413.80 & 0.09 & 4552360.00 & 44963856.56 & 1.04 & 0.95 & 0.99 \\
41725 & 108849 & 2000 & 413.40 & 0.13 & 33336.00 & 331726.51 & 1.24 & 0.80 & 1.00 \\
57967 & 410010 & 2000 & 184.40 & 0.14 & 16395.00 & 170159.14 & 1.12 & 0.92 & 1.04 \\
38266 & 107227 & 2000 & 111.60 & 0.00 & 11218.00 & 111388.46 & 0.99 & 1.00 & 0.99 \\
38241 & 107226 & 2000 & 18.50 & 0.23 & 1917.00 & 15843.08 & 0.97 & 0.86 & 0.83 \\
5157 & 100727 & 2000 & 561.60 & 0.01 & 56139.00 & 544288.15 & 1.00 & 0.97 & 0.97 \\
38231 & 107224 & 2000 & 101.50 & 0.08 & 8620.00 & 79289.57 & 1.18 & 0.78 & 0.92 \\
11781 & 101461 & 2000 & 3423.40 & -0.08 & 304815.00 & 2969495.98 & 1.12 & 0.87 & 0.97 \\
47920 & 222809 & 2000 & 36.60 & 0.19 & 3342.00 & 33421.67 & 1.10 & 0.91 & 1.00 \\
26650 & 103595 & 2000 & 173.90 & -0.04 & 17008.00 & 166312.82 & 1.02 & 0.96 & 0.98 \\
38114 & 107202 & 2000 & 13.10 & 0.15 & 1306.00 & 12581.51 & 1.00 & 0.96 & 0.96 \\
8277 & 101081 & 2000 & 647.70 & 0.29 & 54919.00 & 539226.66 & 1.18 & 0.83 & 0.98 \\
13000 & 101618 & 2000 & 286.40 & 0.05 & 28501.00 & 281443.00 & 1.00 & 0.98 & 0.99 \\
18818 & 102523 & 2000 & 498.90 & 0.18 & 50613.00 & 479858.78 & 0.99 & 0.96 & 0.95 \\
38210 & 107222 & 2000 & 66.50 & 0.12 & 7455.00 & 70537.44 & 0.89 & 1.06 & 0.95 \\
2721 & 100355 & 2000 & 6763.50 & -0.11 & 674863.00 & 6603006.62 & 1.00 & 0.98 & 0.98 \\
38185 & 107215 & 2000 & 205.20 & 0.32 & 15708.00 & 167430.26 & 1.31 & 0.82 & 1.07 \\
26694 & 103600 & 2000 & 468.70 & -0.17 & 45786.00 & 416329.87 & 1.02 & 0.89 & 0.91 \\
22413 & 103011 & 2000 & 76.60 & -0.03 & 7616.00 & 67620.01 & 1.01 & 0.88 & 0.89 \\
38156 & 107209 & 2000 & 157.90 & -0.11 & 15855.00 & 137946.11 & 1.00 & 0.87 & 0.87 \\
63349 & 500500 & 2000 & 167.10 & -0.18 & 16973.00 & 165244.96 & 0.98 & 0.99 & 0.97 \\
38130 & 107204 & 2000 & 516.10 & -0.72 & 53763.00 & 457736.28 & 0.96 & 0.89 & 0.85 \\
17340 & 102282 & 2000 & 128.90 & -0.01 & 13136.00 & 121834.26 & 0.98 & 0.95 & 0.93 \\
7017 & 100985 & 2000 & 945.50 & -0.04 & 101346.00 & 758462.42 & 0.93 & 0.80 & 0.75 \\
5175 & 100730 & 2000 & 990.80 & -0.23 & 99229.00 & 886268.17 & 1.00 & 0.89 & 0.89 \\
37996 & 107181 & 2000 & 69.30 & 0.17 & 6972.00 & 63724.09 & 0.99 & 0.92 & 0.91 \\
18849 & 102524 & 2000 & 2910.80 & -0.10 & 291525.00 & 2836515.24 & 1.00 & 0.97 & 0.97 \\
22333 & 103007 & 2000 & 1489.90 & 0.10 & 144304.00 & 1419908.38 & 1.03 & 0.95 & 0.98 \\
27013 & 103644 & 2000 & 51.00 & 0.11 & 5171.00 & 51199.52 & 0.99 & 1.00 & 0.99 \\
17225 & 102271 & 2000 & 1549.20 & -0.05 & 154440.00 & 1543392.37 & 1.00 & 1.00 & 1.00 \\
8318 & 101082 & 2000 & 3123.60 & 0.37 & 276085.00 & 2801989.90 & 1.13 & 0.90 & 1.01 \\
22245 & 102997 & 2000 & 6537.20 & -0.38 & 679119.00 & 6283965.79 & 0.96 & 0.96 & 0.93 \\
37926 & 107162 & 2000 & 9.60 & -0.09 & 965.00 & 9489.03 & 0.99 & 0.99 & 0.98 \\
5270 & 100745 & 2000 & 2695.60 & 0.07 & 268787.00 & 2367903.26 & 1.00 & 0.88 & 0.88 \\
13049 & 101623 & 2000 & 2006.40 & 0.00 & 201690.00 & 1947757.68 & 0.99 & 0.97 & 0.97 \\
27039 & 103645 & 2000 & 70.40 & 0.13 & 7051.00 & 65515.29 & 1.00 & 0.93 & 0.93 \\
2809 & 100359 & 2000 & 543.30 & -0.11 & 51283.00 & 439489.21 & 1.06 & 0.81 & 0.86 \\
18880 & 102525 & 2000 & 492.70 & 0.17 & 49075.00 & 480427.45 & 1.00 & 0.98 & 0.98 \\
14388 & 101853 & 2000 & 321.60 & -0.04 & 31874.00 & 281942.13 & 1.01 & 0.88 & 0.88 \\
37880 & 107153 & 2000 & 34.40 & -0.14 & 3441.00 & 29331.40 & 1.00 & 0.85 & 0.85 \\
45807 & 200142 & 2000 & 139.90 & -0.13 & 13199.00 & 130272.37 & 1.06 & 0.93 & 0.99 \\
2818 & 100360 & 2000 & 1165.70 & -0.08 & 106481.00 & 1076907.55 & 1.09 & 0.92 & 1.01 \\
37863 & 107152 & 2000 & 88.90 & -0.02 & 8893.00 & 80452.21 & 1.00 & 0.90 & 0.90 \\
41883 & 108867 & 2000 & 1502.60 & -0.18 & 182716.00 & 1299475.59 & 0.82 & 0.86 & 0.71 \\
27097 & 103652 & 2000 & 141.40 & 0.01 & 14245.00 & 128524.02 & 0.99 & 0.91 & 0.90 \\
37825 & 107147 & 2000 & 127.90 & -0.25 & 12678.00 & 133751.24 & 1.01 & 1.05 & 1.05 \\
74614 & 601140 & 2000 & 14.40 & 0.06 & 1437.00 & 13353.25 & 1.00 & 0.93 & 0.93 \\
22217 & 102996 & 2000 & 731.40 & 0.07 & 85189.00 & 764667.34 & 0.86 & 1.05 & 0.90 \\
685 & 100090 & 2000 & 1169.90 & -0.16 & 116987.00 & 1093081.26 & 1.00 & 0.93 & 0.93 \\
17191 & 102270 & 2000 & 1306.10 & -0.06 & 130229.00 & 1254014.84 & 1.00 & 0.96 & 0.96 \\
27116 & 103658 & 2000 & 249.30 & -0.04 & 25008.00 & 224821.79 & 1.00 & 0.90 & 0.90 \\
48878 & 240149 & 2000 & 14.90 & 0.17 & 1451.00 & 14506.65 & 1.03 & 0.97 & 1.00 \\
11648 & 101455 & 2000 & 37705.90 & 0.07 & 3202937.00 & 29809392.10 & 1.18 & 0.79 & 0.93 \\
709 & 100092 & 2000 & 703.50 & 2.37 & 70351.00 & 581683.10 & 1.00 & 0.83 & 0.83 \\
74589 & 601139 & 2000 & 3453.00 & 0.09 & 325655.00 & 3297993.52 & 1.06 & 0.96 & 1.01 \\
37930 & 107167 & 2000 & 7.50 & -0.07 & 767.00 & 7527.59 & 0.98 & 1.00 & 0.98 \\
41858 & 108866 & 2000 & 83.10 & -0.32 & 8504.00 & 80622.25 & 0.98 & 0.97 & 0.95 \\
14354 & 101851 & 2000 & 2389.90 & -0.12 & 238824.00 & 2017458.86 & 1.00 & 0.84 & 0.84 \\
37984 & 107179 & 2000 & 745.10 & -0.10 & 91026.00 & 793965.06 & 0.82 & 1.07 & 0.87 \\
5215 & 100736 & 2000 & 791.50 & 0.58 & 77896.00 & 679839.23 & 1.02 & 0.86 & 0.87 \\
26892 & 103620 & 2000 & 257.90 & 0.08 & 25931.00 & 252131.60 & 0.99 & 0.98 & 0.97 \\
773 & 100096 & 2000 & 113.20 & -0.14 & 11316.00 & 110474.54 & 1.00 & 0.98 & 0.98 \\
74438 & 601001 & 2000 & 8.10 & 0.17 & 1516.00 & 13337.86 & 0.53 & 1.65 & 0.88 \\
26914 & 103621 & 2000 & 220.20 & -0.04 & 22443.00 & 218019.30 & 0.98 & 0.99 & 0.97 \\
41831 & 108859 & 2000 & 75.90 & -0.33 & 7765.00 & 77633.83 & 0.98 & 1.02 & 1.00 \\
26927 & 103628 & 2000 & 1108.10 & 0.20 & 111049.00 & 1072101.60 & 1.00 & 0.97 & 0.97 \\
22297 & 103005 & 2000 & 200.20 & 0.38 & 18629.00 & 186327.95 & 1.07 & 0.93 & 1.00 \\
37962 & 107175 & 2000 & 1691.80 & -0.19 & 194793.00 & 1510222.14 & 0.87 & 0.89 & 0.78 \\
41838 & 108860 & 2000 & 131.20 & -0.20 & 13003.00 & 129861.26 & 1.01 & 0.99 & 1.00 \\
37957 & 107174 & 2000 & 42.60 & 0.11 & 4288.00 & 41852.71 & 0.99 & 0.98 & 0.98 \\
11714 & 101457 & 2000 & 551.00 & -0.08 & 55099.00 & 507851.98 & 1.00 & 0.92 & 0.92 \\
26962 & 103638 & 2000 & 29.80 & -0.07 & 2983.00 & 29829.86 & 1.00 & 1.00 & 1.00 \\
37935 & 107171 & 2000 & 52.70 & -0.10 & 5684.00 & 50881.38 & 0.93 & 0.97 & 0.90 \\
41852 & 108861 & 2000 & 90.50 & -0.09 & 9074.00 & 89368.02 & 1.00 & 0.99 & 0.98 \\
37939 & 107173 & 2000 & 164.20 & 0.29 & 13170.00 & 131893.83 & 1.25 & 0.80 & 1.00 \\
2796 & 100358 & 2000 & 1389.80 & -0.19 & 128574.00 & 1325827.46 & 1.08 & 0.95 & 1.03 \\
5248 & 100741 & 2000 & 259.50 & 0.13 & 26004.00 & 243861.46 & 1.00 & 0.94 & 0.94 \\
47888 & 222658 & 2000 & 308.30 & 0.08 & 30735.00 & 302895.20 & 1.00 & 0.98 & 0.99 \\
5291 & 100746 & 2000 & 1192.80 & 0.10 & 119212.00 & 1192099.23 & 1.00 & 1.00 & 1.00 \\
26987 & 103643 & 2000 & 135.70 & -0.12 & 13797.00 & 137965.80 & 0.98 & 1.02 & 1.00 \\
74561 & 601136 & 2000 & 58.50 & 0.02 & 5613.00 & 53321.29 & 1.04 & 0.91 & 0.95 \\
57933 & 410003 & 2000 & 1048.50 & -0.08 & 100198.00 & 1002028.21 & 1.05 & 0.96 & 1.00 \\
7610 & 101048 & 2000 & 14703.30 & -0.12 & 1474694.00 & 13705667.52 & 1.00 & 0.93 & 0.93 \\
17252 & 102274 & 2000 & 2188.60 & 0.33 & 221303.00 & 2067744.27 & 0.99 & 0.94 & 0.93 \\
743 & 100093 & 2000 & 375.70 & -0.29 & 37569.00 & 327984.46 & 1.00 & 0.87 & 0.87 \\
22276 & 102999 & 2000 & 266.70 & -0.15 & 26960.00 & 253100.49 & 0.99 & 0.95 & 0.94 \\
13035 & 101622 & 2000 & 2657.40 & -0.19 & 282203.00 & 2341828.73 & 0.94 & 0.88 & 0.83 \\
22142 & 102993 & 2000 & 5703.30 & 0.09 & 417269.00 & 4051470.65 & 1.37 & 0.71 & 0.97 \\
16414 & 102134 & 2000 & 93.70 & -0.03 & 9184.00 & 85439.49 & 1.02 & 0.91 & 0.93 \\
37659 & 106992 & 2000 & 207.90 & 0.03 & 16939.00 & 160000.37 & 1.23 & 0.77 & 0.94 \\
28044 & 105370 & 2000 & 159.80 & -0.06 & 19450.00 & 159004.48 & 0.82 & 1.00 & 0.82 \\
37030 & 106652 & 2000 & 21.90 & -0.14 & 2191.00 & 21379.81 & 1.00 & 0.98 & 0.98 \\
16818 & 102193 & 2000 & 509.50 & -0.07 & 50971.00 & 492584.68 & 1.00 & 0.97 & 0.97 \\
13169 & 101698 & 2000 & 252.00 & -0.06 & 25673.00 & 239246.70 & 0.98 & 0.95 & 0.93 \\
37060 & 106655 & 2000 & 222.20 & -0.26 & 22164.00 & 205265.26 & 1.00 & 0.92 & 0.93 \\
63004 & 500466 & 2000 & 456.20 & -0.01 & 46746.00 & 418942.13 & 0.98 & 0.92 & 0.90 \\
28015 & 105369 & 2000 & 341.60 & -0.11 & 39740.00 & 355374.72 & 0.86 & 1.04 & 0.89 \\
5537 & 100771 & 2000 & 219.60 & 0.28 & 21773.00 & 216530.89 & 1.01 & 0.99 & 0.99 \\
488 & 100071 & 2000 & 8158.10 & -0.15 & 920450.00 & 7157599.05 & 0.89 & 0.88 & 0.78 \\
47745 & 221051 & 2000 & 8474.20 & 0.02 & 816431.00 & 8236538.40 & 1.04 & 0.97 & 1.01 \\
37068 & 106664 & 2000 & 24.40 & -0.18 & 2807.00 & 23835.20 & 0.87 & 0.98 & 0.85 \\
37073 & 106666 & 2000 & 38.20 & 3.02 & 4412.00 & 35050.81 & 0.87 & 0.92 & 0.79 \\
42156 & 108931 & 2000 & 38.30 & 0.00 & 3800.00 & 33320.84 & 1.01 & 0.87 & 0.88 \\
11316 & 101393 & 2000 & 800.80 & -0.01 & 79833.00 & 774402.70 & 1.00 & 0.97 & 0.97 \\
37077 & 106667 & 2000 & 27.50 & 0.31 & 2460.00 & 22281.51 & 1.12 & 0.81 & 0.91 \\
37015 & 106650 & 2000 & 11.50 & 0.15 & 1132.00 & 10815.67 & 1.02 & 0.94 & 0.96 \\
28076 & 105379 & 2000 & 272.60 & -0.01 & 26005.00 & 257283.26 & 1.05 & 0.94 & 0.99 \\
11271 & 101381 & 2000 & 78.40 & 0.59 & 7821.00 & 76537.41 & 1.00 & 0.98 & 0.98 \\
16789 & 102192 & 2000 & 266.00 & 0.17 & 26666.00 & 261844.65 & 1.00 & 0.98 & 0.98 \\
3054 & 100401 & 2000 & 303.20 & 0.12 & 30300.00 & 292485.65 & 1.00 & 0.96 & 0.97 \\
36922 & 106640 & 2000 & 102.50 & -0.13 & 7068.00 & 59065.29 & 1.45 & 0.58 & 0.84 \\
28124 & 105383 & 2000 & 150.70 & -0.10 & 15056.00 & 136057.62 & 1.00 & 0.90 & 0.90 \\
13182 & 101703 & 2000 & 56212.00 & 0.04 & 4581140.00 & 43617629.27 & 1.23 & 0.78 & 0.95 \\
36934 & 106642 & 2000 & 334.70 & 0.21 & 33466.00 & 305610.91 & 1.00 & 0.91 & 0.91 \\
36978 & 106644 & 2000 & 5.00 & -0.25 & 626.00 & 4153.61 & 0.80 & 0.83 & 0.66 \\
8494 & 101088 & 2000 & 6350.20 & -0.18 & 1215371.00 & 5900057.33 & 0.52 & 0.93 & 0.49 \\
3045 & 100400 & 2000 & 1181.20 & -0.17 & 109742.00 & 1073841.08 & 1.08 & 0.91 & 0.98 \\
28095 & 105382 & 2000 & 268.10 & -0.08 & 27041.00 & 235073.40 & 0.99 & 0.88 & 0.87 \\
11284 & 101390 & 2000 & 4505.80 & 0.00 & 450571.00 & 4328432.67 & 1.00 & 0.96 & 0.96 \\
19047 & 102545 & 2000 & 226.80 & 0.30 & 16248.00 & 195000.94 & 1.40 & 0.86 & 1.20 \\
55096 & 400061 & 2000 & 64.30 & 0.19 & 6066.00 & 57458.96 & 1.06 & 0.89 & 0.95 \\
21813 & 102952 & 2000 & 647.30 & -0.16 & 77606.00 & 740987.19 & 0.83 & 1.14 & 0.95 \\
28138 & 105384 & 2000 & 67.00 & -0.16 & 6690.00 & 63334.43 & 1.00 & 0.95 & 0.95 \\
27983 & 105364 & 2000 & 166.50 & 0.10 & 16653.00 & 159599.57 & 1.00 & 0.96 & 0.96 \\
37122 & 106682 & 2000 & 20.20 & -0.12 & 1944.00 & 19445.06 & 1.04 & 0.96 & 1.00 \\
13156 & 101681 & 2000 & 1601.90 & -0.13 & 170845.00 & 1374846.80 & 0.94 & 0.86 & 0.80 \\
27909 & 105346 & 2000 & 814.10 & 0.22 & 57966.00 & 740859.70 & 1.40 & 0.91 & 1.28 \\
16877 & 102213 & 2000 & 1923.50 & -0.10 & 192358.00 & 1735761.51 & 1.00 & 0.90 & 0.90 \\
27900 & 105343 & 2000 & 71.50 & -0.02 & 7121.00 & 70031.83 & 1.00 & 0.98 & 0.98 \\
53536 & 351589 & 2000 & 12.90 & -0.09 & 1364.00 & 11588.33 & 0.95 & 0.90 & 0.85 \\
21884 & 102964 & 2000 & 495.40 & 0.16 & 41034.00 & 429917.03 & 1.21 & 0.87 & 1.05 \\
8462 & 101087 & 2000 & 635.50 & 0.28 & 79349.00 & 553705.04 & 0.80 & 0.87 & 0.70 \\
74710 & 601155 & 2000 & 20.80 & 0.29 & 2076.00 & 20280.50 & 1.00 & 0.98 & 0.98 \\
57854 & 401189 & 2000 & 110.70 & -0.17 & 12983.00 & 105090.32 & 0.85 & 0.95 & 0.81 \\
6911 & 100968 & 2000 & 237.00 & 0.02 & 23717.00 & 236588.55 & 1.00 & 1.00 & 1.00 \\
11365 & 101398 & 2000 & 130.40 & -0.34 & 8652.00 & 86528.02 & 1.51 & 0.66 & 1.00 \\
37125 & 106688 & 2000 & 95.40 & 0.00 & 8225.00 & 69645.39 & 1.16 & 0.73 & 0.85 \\
4284 & 100600 & 2000 & 45.20 & 0.11 & 4518.00 & 40424.28 & 1.00 & 0.89 & 0.89 \\
11372 & 101399 & 2000 & 161.00 & -0.07 & 17081.00 & 156849.00 & 0.94 & 0.97 & 0.92 \\
21892 & 102969 & 2000 & 867.10 & -0.10 & 91295.00 & 886124.89 & 0.95 & 1.02 & 0.97 \\
57849 & 401145 & 2000 & 71.70 & -0.43 & 6735.00 & 66946.23 & 1.06 & 0.93 & 0.99 \\
14513 & 101871 & 2000 & 326.40 & -0.05 & 38833.00 & 397106.37 & 0.84 & 1.22 & 1.02 \\
27977 & 105362 & 2000 & 49.30 & -0.04 & 4930.00 & 48156.21 & 1.00 & 0.98 & 0.98 \\
47767 & 221210 & 2000 & 100.50 & -0.03 & 10115.00 & 96919.73 & 0.99 & 0.96 & 0.96 \\
16847 & 102197 & 2000 & 138.00 & 0.16 & 13802.00 & 131854.25 & 1.00 & 0.96 & 0.96 \\
37086 & 106675 & 2000 & 34.90 & 0.18 & 3494.00 & 28000.59 & 1.00 & 0.80 & 0.80 \\
27951 & 105358 & 2000 & 500.50 & -0.05 & 54594.00 & 518275.25 & 0.92 & 1.04 & 0.95 \\
27919 & 105348 & 2000 & 15.50 & -0.07 & 1391.00 & 13912.11 & 1.11 & 0.90 & 1.00 \\
74717 & 601156 & 2000 & 95.90 & -0.14 & 9468.00 & 89734.31 & 1.01 & 0.94 & 0.95 \\
5506 & 100769 & 2000 & 2885.40 & 0.09 & 288273.00 & 2823473.28 & 1.00 & 0.98 & 0.98 \\
27932 & 105353 & 2000 & 67.40 & 0.06 & 6107.00 & 61066.66 & 1.10 & 0.91 & 1.00 \\
53512 & 351459 & 2000 & 791.50 & -0.05 & 88837.00 & 670570.32 & 0.89 & 0.85 & 0.75 \\
27926 & 105352 & 2000 & 263.00 & -0.14 & 26299.00 & 254230.51 & 1.00 & 0.97 & 0.97 \\
19023 & 102544 & 2000 & 859.90 & 0.13 & 79760.00 & 723620.89 & 1.08 & 0.84 & 0.91 \\
42134 & 108926 & 2000 & 86.90 & 0.14 & 6609.00 & 66421.66 & 1.31 & 0.76 & 1.01 \\
42149 & 108930 & 2000 & 85.80 & -0.05 & 8343.00 & 80796.71 & 1.03 & 0.94 & 0.97 \\
5565 & 100772 & 2000 & 614.50 & 0.02 & 80825.00 & 797640.12 & 0.76 & 1.30 & 0.99 \\
36904 & 106627 & 2000 & 621.20 & -0.16 & 63981.00 & 582589.21 & 0.97 & 0.94 & 0.91 \\
11196 & 101370 & 2000 & 11.10 & -0.10 & 1155.00 & 9508.06 & 0.96 & 0.86 & 0.82 \\
36772 & 106590 & 2000 & 34.70 & 0.09 & 3486.00 & 33118.93 & 1.00 & 0.95 & 0.95 \\
5648 & 100784 & 2000 & 5728.50 & 0.16 & 570539.00 & 5171600.16 & 1.00 & 0.90 & 0.91 \\
28306 & 105401 & 2000 & 7.80 & -0.16 & 1213.00 & 10967.74 & 0.64 & 1.41 & 0.90 \\
42234 & 108944 & 2000 & 28.90 & -0.13 & 2956.00 & 24844.45 & 0.98 & 0.86 & 0.84 \\
28323 & 105412 & 2000 & 209.40 & -0.04 & 22902.00 & 199278.41 & 0.91 & 0.95 & 0.87 \\
11208 & 101375 & 2000 & 15.80 & -0.08 & 1538.00 & 15384.27 & 1.03 & 0.97 & 1.00 \\
5623 & 100775 & 2000 & 683.30 & -0.02 & 68281.00 & 670732.90 & 1.00 & 0.98 & 0.98 \\
53444 & 350408 & 2000 & 14.20 & 0.09 & 1409.00 & 13091.41 & 1.01 & 0.92 & 0.93 \\
19079 & 102548 & 2000 & 1244.10 & 0.28 & 108223.00 & 1027268.48 & 1.15 & 0.83 & 0.95 \\
42209 & 108943 & 2000 & 386.90 & 0.10 & 28433.00 & 263083.53 & 1.36 & 0.68 & 0.93 \\
11219 & 101376 & 2000 & 1491.80 & 0.07 & 149105.00 & 1436417.62 & 1.00 & 0.96 & 0.96 \\
36788 & 106594 & 2000 & 4.00 & -0.11 & 557.00 & 4009.85 & 0.72 & 1.00 & 0.72 \\
13840 & 101769 & 2000 & 2420.80 & -0.20 & 298249.00 & 2305107.50 & 0.81 & 0.95 & 0.77 \\
21707 & 102940 & 2000 & 1579.80 & 0.04 & 156511.00 & 1529105.20 & 1.01 & 0.97 & 0.98 \\
28337 & 105416 & 2000 & 1421.70 & 0.37 & 121469.00 & 1207462.09 & 1.17 & 0.85 & 0.99 \\
28401 & 105421 & 2000 & 49.00 & 0.21 & 4165.00 & 39007.18 & 1.18 & 0.80 & 0.94 \\
7558 & 101045 & 2000 & 13914.90 & 0.07 & 1351071.00 & 11639978.42 & 1.03 & 0.84 & 0.86 \\
21677 & 102939 & 2000 & 7096.60 & -0.08 & 753371.00 & 6803031.04 & 0.94 & 0.96 & 0.90 \\
46267 & 200205 & 2000 & 86.30 & 0.29 & 7366.00 & 74040.69 & 1.17 & 0.86 & 1.01 \\
36688 & 106577 & 2000 & 2312.40 & -0.18 & 234201.00 & 2166669.37 & 0.99 & 0.94 & 0.93 \\
36714 & 106580 & 2000 & 97.00 & 0.12 & 9804.00 & 96612.91 & 0.99 & 1.00 & 0.99 \\
28375 & 105420 & 2000 & 110.30 & -0.22 & 14922.00 & 108942.35 & 0.74 & 0.99 & 0.73 \\
42268 & 108947 & 2000 & 1742.60 & -0.37 & 273032.00 & 1633206.63 & 0.64 & 0.94 & 0.60 \\
36730 & 106583 & 2000 & 36.90 & -0.10 & 3912.00 & 39125.47 & 0.94 & 1.06 & 1.00 \\
28349 & 105419 & 2000 & 88.70 & 0.11 & 7944.00 & 85132.29 & 1.12 & 0.96 & 1.07 \\
53438 & 349609 & 2000 & 25.40 & -0.18 & 2516.00 & 24518.09 & 1.01 & 0.97 & 0.97 \\
16691 & 102178 & 2000 & 1237.60 & -0.12 & 124466.00 & 1211713.94 & 0.99 & 0.98 & 0.97 \\
36727 & 106581 & 2000 & 35.10 & 0.26 & 3710.00 & 36329.19 & 0.95 & 1.04 & 0.98 \\
16725 & 102182 & 2000 & 71.30 & 0.18 & 7213.00 & 68782.80 & 0.99 & 0.96 & 0.95 \\
36823 & 106602 & 2000 & 325.40 & -0.33 & 35063.00 & 348168.17 & 0.93 & 1.07 & 0.99 \\
50063 & 240391 & 2000 & 138.20 & -0.04 & 22722.00 & 133071.78 & 0.61 & 0.96 & 0.59 \\
74756 & 601160 & 2000 & 68.10 & -0.18 & 6830.00 & 66803.85 & 1.00 & 0.98 & 0.98 \\
28199 & 105391 & 2000 & 19.30 & 0.09 & 2020.00 & 16294.66 & 0.96 & 0.84 & 0.81 \\
19063 & 102546 & 2000 & 49.30 & -0.10 & 4830.00 & 42671.22 & 1.02 & 0.87 & 0.88 \\
11249 & 101379 & 2000 & 955.50 & -0.09 & 96757.00 & 926581.48 & 0.99 & 0.97 & 0.96 \\
6879 & 100967 & 2000 & 367.30 & 0.19 & 29864.00 & 308097.01 & 1.23 & 0.84 & 1.03 \\
28170 & 105390 & 2000 & 463.50 & 0.20 & 46616.00 & 430843.05 & 0.99 & 0.93 & 0.92 \\
11260 & 101380 & 2000 & 275.10 & 0.11 & 28258.00 & 267653.28 & 0.97 & 0.97 & 0.95 \\
462 & 100068 & 2000 & 88.40 & -0.02 & 12770.00 & 116567.87 & 0.69 & 1.32 & 0.91 \\
16775 & 102191 & 2000 & 63.80 & 0.06 & 6378.00 & 58693.55 & 1.00 & 0.92 & 0.92 \\
36878 & 106620 & 2000 & 247.80 & 0.60 & 24633.00 & 240311.62 & 1.01 & 0.97 & 0.98 \\
53460 & 350572 & 2000 & 74.60 & -0.17 & 9441.00 & 76003.06 & 0.79 & 1.02 & 0.81 \\
21769 & 102951 & 2000 & 5198.50 & -0.02 & 603156.00 & 5597161.75 & 0.86 & 1.08 & 0.93 \\
36816 & 106597 & 2000 & 153.70 & -0.05 & 18056.00 & 176765.12 & 0.85 & 1.15 & 0.98 \\
14555 & 101876 & 2000 & 226.90 & -0.03 & 26789.00 & 214883.80 & 0.85 & 0.95 & 0.80 \\
21751 & 102949 & 2000 & 2351.40 & 0.15 & 208447.00 & 2315817.00 & 1.13 & 0.98 & 1.11 \\
42203 & 108939 & 2000 & 135.40 & -0.18 & 13469.00 & 134687.90 & 1.01 & 0.99 & 1.00 \\
19071 & 102547 & 2000 & 44.30 & 0.03 & 6961.00 & 63494.52 & 0.64 & 1.43 & 0.91 \\
3084 & 100408 & 2000 & 250.40 & -0.02 & 25468.00 & 205320.02 & 0.98 & 0.82 & 0.81 \\
28208 & 105393 & 2000 & 129.50 & 0.24 & 13092.00 & 121754.68 & 0.99 & 0.94 & 0.93 \\
28235 & 105397 & 2000 & 63.20 & 0.21 & 6412.00 & 62850.94 & 0.99 & 0.99 & 0.98 \\
16742 & 102183 & 2000 & 180.00 & -0.07 & 18251.00 & 169187.68 & 0.99 & 0.94 & 0.93 \\
50088 & 240392 & 2000 & 263.60 & -0.00 & 26143.00 & 258073.38 & 1.01 & 0.98 & 0.99 \\
28227 & 105394 & 2000 & 27.30 & -0.29 & 3875.00 & 24582.51 & 0.70 & 0.90 & 0.63 \\
6871 & 100966 & 2000 & 8.30 & -0.16 & 959.00 & 8917.09 & 0.87 & 1.07 & 0.93 \\
5604 & 100773 & 2000 & 1378.00 & -0.04 & 138427.00 & 1373643.66 & 1.00 & 1.00 & 0.99 \\
57902 & 401372 & 2000 & 7.70 & -0.07 & 925.00 & 7874.20 & 0.83 & 1.02 & 0.85 \\
37153 & 106695 & 2000 & 43.60 & 0.17 & 4278.00 & 40403.20 & 1.02 & 0.93 & 0.94 \\
37156 & 106701 & 2000 & 85.20 & -0.11 & 8517.00 & 84196.03 & 1.00 & 0.99 & 0.99 \\
42025 & 108910 & 2000 & 42.60 & -0.07 & 4213.00 & 40729.70 & 1.01 & 0.96 & 0.97 \\
27466 & 105280 & 2000 & 60.40 & 0.33 & 5309.00 & 56790.03 & 1.14 & 0.94 & 1.07 \\
49781 & 240360 & 2000 & 295.90 & 0.01 & 25492.00 & 219767.93 & 1.16 & 0.74 & 0.86 \\
596 & 100079 & 2000 & 2835.30 & -0.14 & 372155.00 & 2762726.98 & 0.76 & 0.97 & 0.74 \\
37513 & 106944 & 2000 & 11.70 & 0.08 & 1161.00 & 11220.32 & 1.01 & 0.96 & 0.97 \\
27475 & 105281 & 2000 & 347.20 & 0.25 & 30477.00 & 283326.37 & 1.14 & 0.82 & 0.93 \\
13867 & 101781 & 2000 & 783.30 & -0.11 & 107375.00 & 684554.69 & 0.73 & 0.87 & 0.64 \\
27430 & 105278 & 2000 & 1668.00 & -0.23 & 166279.00 & 1595018.03 & 1.00 & 0.96 & 0.96 \\
8397 & 101085 & 2000 & 539.20 & -0.14 & 76417.00 & 532119.61 & 0.71 & 0.99 & 0.70 \\
5369 & 100758 & 2000 & 83.10 & 0.13 & 8311.00 & 80805.09 & 1.00 & 0.97 & 0.97 \\
37534 & 106961 & 2000 & 159.70 & -0.05 & 16456.00 & 152383.91 & 0.97 & 0.95 & 0.93 \\
48827 & 240144 & 2000 & 13.20 & 0.12 & 1310.00 & 11528.74 & 1.01 & 0.87 & 0.88 \\
37522 & 106948 & 2000 & 85.80 & 0.22 & 10035.00 & 78061.86 & 0.86 & 0.91 & 0.78 \\
37541 & 106962 & 2000 & 5.80 & -0.09 & 586.00 & 5296.83 & 0.99 & 0.91 & 0.90 \\
17040 & 102231 & 2000 & 1078.60 & -0.17 & 115042.00 & 1085711.72 & 0.94 & 1.01 & 0.94 \\
37487 & 106934 & 2000 & 2039.40 & -0.28 & 205255.00 & 1993730.63 & 0.99 & 0.98 & 0.97 \\
37453 & 106919 & 2000 & 3.00 & 0.14 & 423.00 & 4107.37 & 0.71 & 1.37 & 0.97 \\
2881 & 100368 & 2000 & 406.70 & -0.05 & 45004.00 & 359476.56 & 0.90 & 0.88 & 0.80 \\
27542 & 105287 & 2000 & 378.00 & -0.46 & 37870.00 & 374054.30 & 1.00 & 0.99 & 0.99 \\
7590 & 101047 & 2000 & 328.20 & -0.01 & 34468.00 & 326768.71 & 0.95 & 1.00 & 0.95 \\
571 & 100076 & 2000 & 1264.50 & -0.02 & 158490.00 & 1026145.31 & 0.80 & 0.81 & 0.65 \\
18959 & 102531 & 2000 & 23.40 & 0.01 & 2500.00 & 23684.25 & 0.94 & 1.01 & 0.95 \\
7240 & 101015 & 2000 & 570.90 & 0.11 & 53732.00 & 564543.93 & 1.06 & 0.99 & 1.05 \\
42035 & 108912 & 2000 & 86.10 & -0.25 & 8318.00 & 83165.49 & 1.04 & 0.97 & 1.00 \\
27533 & 105286 & 2000 & 448.60 & -0.06 & 46072.00 & 421706.49 & 0.97 & 0.94 & 0.92 \\
14469 & 101861 & 2000 & 885.00 & 0.15 & 84205.00 & 792771.49 & 1.05 & 0.90 & 0.94 \\
57885 & 401363 & 2000 & 11.70 & -0.09 & 1189.00 & 11888.29 & 0.98 & 1.02 & 1.00 \\
22047 & 102988 & 2000 & 40.20 & 0.05 & 4044.00 & 40440.29 & 0.99 & 1.01 & 1.00 \\
27505 & 105283 & 2000 & 8.60 & -0.08 & 917.00 & 8202.30 & 0.94 & 0.95 & 0.89 \\
11506 & 101425 & 2000 & 34.20 & -0.19 & 5250.00 & 45217.34 & 0.65 & 1.32 & 0.86 \\
37478 & 106931 & 2000 & 821.70 & -0.01 & 99075.00 & 957957.96 & 0.83 & 1.17 & 0.97 \\
5402 & 100760 & 2000 & 895.30 & -0.10 & 88625.00 & 852472.95 & 1.01 & 0.95 & 0.96 \\
11484 & 101422 & 2000 & 81.80 & -0.20 & 8414.00 & 77337.66 & 0.97 & 0.95 & 0.92 \\
27401 & 105276 & 2000 & 989.50 & 0.16 & 96538.00 & 897947.22 & 1.02 & 0.91 & 0.93 \\
18927 & 102528 & 2000 & 98.60 & 0.04 & 10583.00 & 97775.73 & 0.93 & 0.99 & 0.92 \\
27338 & 105269 & 2000 & 475.70 & 0.33 & 48620.00 & 463343.84 & 0.98 & 0.97 & 0.95 \\
5346 & 100754 & 2000 & 860.80 & 0.05 & 86116.00 & 846102.75 & 1.00 & 0.98 & 0.98 \\
74643 & 601146 & 2000 & 18.30 & -0.12 & 1889.00 & 18263.30 & 0.97 & 1.00 & 0.97 \\
37640 & 106983 & 2000 & 188.70 & 0.19 & 18774.00 & 183945.24 & 1.01 & 0.97 & 0.98 \\
22108 & 102990 & 2000 & 3624.40 & 0.06 & 362046.00 & 3568454.68 & 1.00 & 0.98 & 0.99 \\
41983 & 108889 & 2000 & 26.40 & -0.41 & 2664.00 & 25919.98 & 0.99 & 0.98 & 0.97 \\
37646 & 106984 & 2000 & 10.30 & 0.13 & 1034.00 & 10155.13 & 1.00 & 0.99 & 0.98 \\
27313 & 105268 & 2000 & 755.30 & 0.11 & 75291.00 & 748846.83 & 1.00 & 0.99 & 0.99 \\
41974 & 108886 & 2000 & 82.90 & -0.10 & 8315.00 & 74008.02 & 1.00 & 0.89 & 0.89 \\
41971 & 108876 & 2000 & 25.20 & 0.06 & 2520.00 & 24570.25 & 1.00 & 0.98 & 0.98 \\
45881 & 200153 & 2000 & 27.80 & -0.02 & 2821.00 & 26336.61 & 0.99 & 0.95 & 0.93 \\
37653 & 106985 & 2000 & 61.60 & -0.08 & 6335.00 & 59939.94 & 0.97 & 0.97 & 0.95 \\
17106 & 102257 & 2000 & 2381.30 & -0.09 & 240171.00 & 2323725.76 & 0.99 & 0.98 & 0.97 \\
37619 & 106978 & 2000 & 302.70 & -0.19 & 30595.00 & 304924.31 & 0.99 & 1.01 & 1.00 \\
45889 & 200156 & 2000 & 48.10 & -0.13 & 5162.00 & 44709.24 & 0.93 & 0.93 & 0.87 \\
22075 & 102989 & 2000 & 2949.90 & -0.09 & 297074.00 & 2892263.94 & 0.99 & 0.98 & 0.97 \\
5355 & 100757 & 2000 & 5.10 & -0.10 & 514.00 & 4520.68 & 0.99 & 0.89 & 0.88 \\
37574 & 106969 & 2000 & 203.50 & 0.21 & 21121.00 & 212002.46 & 0.96 & 1.04 & 1.00 \\
41996 & 108901 & 2000 & 270.00 & 0.04 & 32318.00 & 266132.61 & 0.84 & 0.99 & 0.82 \\
37599 & 106972 & 2000 & 31.60 & -0.05 & 5249.00 & 51351.55 & 0.60 & 1.63 & 0.98 \\
14448 & 101858 & 2000 & 653.90 & 0.19 & 49074.00 & 418217.90 & 1.33 & 0.64 & 0.85 \\
41993 & 108900 & 2000 & 19.10 & -0.29 & 2781.00 & 25784.55 & 0.69 & 1.35 & 0.93 \\
37616 & 106974 & 2000 & 611.40 & -0.06 & 62142.00 & 529968.12 & 0.98 & 0.87 & 0.85 \\
6940 & 100973 & 2000 & 51.60 & -0.02 & 5794.00 & 54815.35 & 0.89 & 1.06 & 0.95 \\
17083 & 102255 & 2000 & 223.70 & -0.21 & 22239.00 & 220960.43 & 1.01 & 0.99 & 0.99 \\
53579 & 354336 & 2000 & 31.50 & 0.02 & 3063.00 & 30062.52 & 1.03 & 0.95 & 0.98 \\
74646 & 601147 & 2000 & 94.70 & 0.09 & 9447.00 & 79972.68 & 1.00 & 0.84 & 0.85 \\
27360 & 105271 & 2000 & 55.00 & -0.11 & 5508.00 & 51722.92 & 1.00 & 0.94 & 0.94 \\
11560 & 101430 & 2000 & 134.50 & 0.23 & 13456.00 & 128074.02 & 1.00 & 0.95 & 0.95 \\
27371 & 105275 & 2000 & 93.40 & -0.06 & 9267.00 & 91959.35 & 1.01 & 0.98 & 0.99 \\
27561 & 105291 & 2000 & 427.90 & 0.22 & 34059.00 & 300376.41 & 1.26 & 0.70 & 0.88 \\
37427 & 106914 & 2000 & 407.70 & -0.20 & 54209.00 & 370047.21 & 0.75 & 0.91 & 0.68 \\
524 & 100072 & 2000 & 17673.60 & -0.11 & 2257939.00 & 16394838.21 & 0.78 & 0.93 & 0.73 \\
21957 & 102981 & 2000 & 233.30 & 0.15 & 23524.00 & 230811.79 & 0.99 & 0.99 & 0.98 \\
27781 & 105326 & 2000 & 155.50 & -0.12 & 22213.00 & 220709.96 & 0.70 & 1.42 & 0.99 \\
42110 & 108923 & 2000 & 43.10 & -0.15 & 4148.00 & 36489.44 & 1.04 & 0.85 & 0.88 \\
27771 & 105322 & 2000 & 26.90 & 0.13 & 2596.00 & 25964.79 & 1.04 & 0.97 & 1.00 \\
37227 & 106713 & 2000 & 44.80 & -0.01 & 4483.00 & 42871.12 & 1.00 & 0.96 & 0.96 \\
42103 & 108920 & 2000 & 92.00 & -0.24 & 9126.00 & 91187.49 & 1.01 & 0.99 & 1.00 \\
27789 & 105327 & 2000 & 108.60 & 0.13 & 10973.00 & 105117.30 & 0.99 & 0.97 & 0.96 \\
27753 & 105321 & 2000 & 332.00 & 0.15 & 31888.00 & 318890.99 & 1.04 & 0.96 & 1.00 \\
42095 & 108919 & 2000 & 546.30 & -0.45 & 55090.00 & 517709.31 & 0.99 & 0.95 & 0.94 \\
37274 & 106725 & 2000 & 24.80 & -0.25 & 2232.00 & 22315.87 & 1.11 & 0.90 & 1.00 \\
18991 & 102540 & 2000 & 8.50 & 0.06 & 984.00 & 9964.15 & 0.86 & 1.17 & 1.01 \\
27727 & 105320 & 2000 & 98.70 & 0.14 & 9908.00 & 92677.92 & 1.00 & 0.94 & 0.94 \\
63178 & 500486 & 2000 & 351.30 & 0.20 & 29608.00 & 338753.20 & 1.19 & 0.96 & 1.14 \\
11434 & 101402 & 2000 & 9.10 & 0.18 & 821.00 & 7296.57 & 1.11 & 0.80 & 0.89 \\
37286 & 106726 & 2000 & 2297.80 & -0.10 & 229625.00 & 2253666.62 & 1.00 & 0.98 & 0.98 \\
5446 & 100763 & 2000 & 1481.80 & -0.07 & 147388.00 & 1357430.09 & 1.01 & 0.92 & 0.92 \\
27795 & 105331 & 2000 & 3.30 & -0.01 & 318.00 & 2552.65 & 1.04 & 0.77 & 0.80 \\
2994 & 100395 & 2000 & 720.90 & 0.11 & 71954.00 & 686656.23 & 1.00 & 0.95 & 0.95 \\
27861 & 105335 & 2000 & 273.90 & 0.01 & 25987.00 & 231922.83 & 1.05 & 0.85 & 0.89 \\
21909 & 102979 & 2000 & 65.90 & -0.08 & 6432.00 & 63313.56 & 1.02 & 0.96 & 0.98 \\
27853 & 105333 & 2000 & 11.90 & 0.03 & 1125.00 & 11252.94 & 1.06 & 0.95 & 1.00 \\
74685 & 601151 & 2000 & 28.70 & -0.04 & 2869.00 & 27309.93 & 1.00 & 0.95 & 0.95 \\
42127 & 108925 & 2000 & 1573.70 & -0.05 & 154641.00 & 1447177.84 & 1.02 & 0.92 & 0.94 \\
37163 & 106706 & 2000 & 9.80 & 0.07 & 978.00 & 9405.78 & 1.00 & 0.96 & 0.96 \\
2954 & 100389 & 2000 & 601.00 & -0.11 & 60048.00 & 597791.30 & 1.00 & 0.99 & 1.00 \\
37177 & 106707 & 2000 & 16.40 & 0.19 & 1642.00 & 16233.68 & 1.00 & 0.99 & 0.99 \\
42115 & 108924 & 2000 & 105.70 & 0.30 & 10254.00 & 102539.83 & 1.03 & 0.97 & 1.00 \\
27824 & 105332 & 2000 & 189.50 & -0.17 & 21982.00 & 186121.09 & 0.86 & 0.98 & 0.85 \\
63124 & 500483 & 2000 & 855.70 & -0.09 & 102163.00 & 768569.02 & 0.84 & 0.90 & 0.75 \\
49925 & 240376 & 2000 & 1.70 & 0.00 & 167.00 & 1635.47 & 1.02 & 0.96 & 0.98 \\
37203 & 106708 & 2000 & 422.00 & -0.10 & 42672.00 & 407499.35 & 0.99 & 0.97 & 0.95 \\
11400 & 101400 & 2000 & 258.90 & 0.15 & 26367.00 & 264310.78 & 0.98 & 1.02 & 1.00 \\
37216 & 106710 & 2000 & 76.10 & 0.19 & 7201.00 & 64761.38 & 1.06 & 0.85 & 0.90 \\
5468 & 100764 & 2000 & 268.50 & 0.17 & 26930.00 & 247519.58 & 1.00 & 0.92 & 0.92 \\
6921 & 100969 & 2000 & 56.40 & -0.10 & 6658.00 & 56622.68 & 0.85 & 1.00 & 0.85 \\
13125 & 101668 & 2000 & 117.30 & 0.04 & 11733.00 & 109807.05 & 1.00 & 0.94 & 0.94 \\
37376 & 106742 & 2000 & 118.30 & 0.19 & 11878.00 & 117637.47 & 1.00 & 0.99 & 0.99 \\
22016 & 102987 & 2000 & 1247.90 & -0.12 & 125083.00 & 1194250.14 & 1.00 & 0.96 & 0.95 \\
27606 & 105303 & 2000 & 144.50 & 0.15 & 15589.00 & 129917.67 & 0.93 & 0.90 & 0.83 \\
27599 & 105299 & 2000 & 3.20 & -0.08 & 380.00 & 3010.50 & 0.84 & 0.94 & 0.79 \\
27635 & 105306 & 2000 & 280.10 & -0.18 & 28006.00 & 278288.50 & 1.00 & 0.99 & 0.99 \\
57880 & 401360 & 2000 & 56.50 & -0.31 & 5644.00 & 51365.77 & 1.00 & 0.91 & 0.91 \\
42070 & 108918 & 2000 & 17.80 & 0.12 & 1530.00 & 15014.91 & 1.16 & 0.84 & 0.98 \\
37413 & 106869 & 2000 & 119.10 & 0.08 & 11638.00 & 116817.08 & 1.02 & 0.98 & 1.00 \\
53543 & 351713 & 2000 & 281.60 & -0.03 & 33194.00 & 273246.45 & 0.85 & 0.97 & 0.82 \\
37421 & 106896 & 2000 & 424.50 & -0.22 & 40802.00 & 410981.27 & 1.04 & 0.97 & 1.01 \\
38280 & 107235 & 2000 & 89.30 & 0.35 & 5417.00 & 47530.11 & 1.65 & 0.53 & 0.88 \\
17004 & 102230 & 2000 & 33.20 & -0.18 & 3316.00 & 29976.16 & 1.00 & 0.90 & 0.90 \\
42062 & 108915 & 2000 & 42.20 & -0.29 & 4228.00 & 36250.96 & 1.00 & 0.86 & 0.86 \\
27589 & 105295 & 2000 & 507.40 & 0.03 & 51205.00 & 478848.00 & 0.99 & 0.94 & 0.94 \\
47798 & 221485 & 2000 & 903.60 & -0.08 & 90813.00 & 886057.58 & 1.00 & 0.98 & 0.98 \\
2890 & 100369 & 2000 & 484.80 & -0.09 & 50592.00 & 463042.92 & 0.96 & 0.96 & 0.92 \\
22002 & 102984 & 2000 & 31.60 & 0.03 & 3153.00 & 27765.61 & 1.00 & 0.88 & 0.88 \\
27710 & 105317 & 2000 & 1506.80 & 0.12 & 151830.00 & 1416457.41 & 0.99 & 0.94 & 0.93 \\
21990 & 102983 & 2000 & 33.10 & -0.15 & 3343.00 & 30536.73 & 0.99 & 0.92 & 0.91 \\
74679 & 601150 & 2000 & 1.50 & -0.03 & 210.00 & 1524.06 & 0.71 & 1.02 & 0.73 \\
57875 & 401355 & 2000 & 169.40 & 0.96 & 11815.00 & 143951.18 & 1.43 & 0.85 & 1.22 \\
27698 & 105311 & 2000 & 285.10 & -0.33 & 26920.00 & 250387.63 & 1.06 & 0.88 & 0.93 \\
74676 & 601149 & 2000 & 36.20 & 0.06 & 3108.00 & 28903.85 & 1.16 & 0.80 & 0.93 \\
27690 & 105310 & 2000 & 358.20 & -0.38 & 35170.00 & 341840.24 & 1.02 & 0.95 & 0.97 \\
37368 & 106740 & 2000 & 313.90 & -0.04 & 33930.00 & 303247.83 & 0.93 & 0.97 & 0.89 \\
16968 & 102224 & 2000 & 4066.20 & -0.03 & 472498.00 & 4176792.69 & 0.86 & 1.03 & 0.88 \\
37313 & 106729 & 2000 & 1229.50 & -0.08 & 124214.00 & 1211592.67 & 0.99 & 0.99 & 0.98 \\
2915 & 100379 & 2000 & 526.90 & -0.01 & 52424.00 & 518095.29 & 1.01 & 0.98 & 0.99 \\
11457 & 101414 & 2000 & 13.00 & 0.02 & 2093.00 & 21568.87 & 0.62 & 1.66 & 1.03 \\
48803 & 240143 & 2000 & 291.40 & -0.26 & 38595.00 & 251694.43 & 0.76 & 0.86 & 0.65 \\
27664 & 105309 & 2000 & 494.60 & 0.01 & 53944.00 & 490297.31 & 0.92 & 0.99 & 0.91 \\
545 & 100075 & 2000 & 4372.30 & 3.10 & 509699.00 & 4162986.19 & 0.86 & 0.95 & 0.82 \\
37338 & 106730 & 2000 & 20.60 & 0.16 & 2049.00 & 19585.67 & 1.01 & 0.95 & 0.96 \\
8429 & 101086 & 2000 & 1677.40 & -0.30 & 323963.00 & 1563678.62 & 0.52 & 0.93 & 0.48 \\
41701 & 108840 & 2000 & 39.80 & 0.10 & 3997.00 & 41214.22 & 1.00 & 1.04 & 1.03 \\
63522 & 500513 & 2000 & 140.90 & 0.06 & 13724.00 & 134155.90 & 1.03 & 0.95 & 0.98 \\
26584 & 103592 & 2000 & 54.50 & 0.26 & 5451.00 & 49620.54 & 1.00 & 0.91 & 0.91 \\
1747 & 100227 & 2000 & 204.40 & -0.09 & 23367.00 & 166188.94 & 0.87 & 0.81 & 0.71 \\
23297 & 103158 & 2000 & 3320.40 & -0.06 & 305729.00 & 3025636.61 & 1.09 & 0.91 & 0.99 \\
49256 & 240256 & 2000 & 110.50 & -0.04 & 6594.00 & 62426.16 & 1.68 & 0.56 & 0.95 \\
39757 & 107860 & 2000 & 20.50 & 0.22 & 1825.00 & 16801.95 & 1.12 & 0.82 & 0.92 \\
39760 & 107863 & 2000 & 599.80 & 0.14 & 59877.00 & 571345.13 & 1.00 & 0.95 & 0.95 \\
2334 & 100319 & 2000 & 103.70 & 0.12 & 10356.00 & 99979.36 & 1.00 & 0.96 & 0.97 \\
24801 & 103380 & 2000 & 6735.80 & -0.09 & 775605.00 & 6628687.99 & 0.87 & 0.98 & 0.85 \\
18526 & 102470 & 2000 & 889.10 & -0.07 & 124427.00 & 1026214.19 & 0.71 & 1.15 & 0.82 \\
23309 & 103160 & 2000 & 130.20 & -0.15 & 12147.00 & 121463.57 & 1.07 & 0.93 & 1.00 \\
1767 & 100228 & 2000 & 171.30 & 0.08 & 17226.00 & 143973.98 & 0.99 & 0.84 & 0.84 \\
12308 & 101534 & 2000 & 1278.70 & 0.28 & 124835.00 & 1209226.14 & 1.02 & 0.95 & 0.97 \\
39767 & 107868 & 2000 & 58.60 & -0.08 & 5855.00 & 58138.67 & 1.00 & 0.99 & 0.99 \\
39772 & 107870 & 2000 & 620.90 & -0.18 & 87730.00 & 555769.63 & 0.71 & 0.90 & 0.63 \\
2314 & 100315 & 2000 & 415.70 & -0.08 & 42201.00 & 421905.24 & 0.99 & 1.01 & 1.00 \\
18043 & 102387 & 2000 & 38.60 & 0.28 & 3688.00 & 38451.11 & 1.05 & 1.00 & 1.04 \\
24762 & 103377 & 2000 & 1871.80 & 0.17 & 168197.00 & 1370442.06 & 1.11 & 0.73 & 0.81 \\
7801 & 101061 & 2000 & 3572.40 & -0.06 & 342067.00 & 3542968.75 & 1.04 & 0.99 & 1.04 \\
8068 & 101073 & 2000 & 10979.40 & -0.27 & 1471307.00 & 9041693.11 & 0.75 & 0.82 & 0.61 \\
39798 & 107874 & 2000 & 112.00 & 0.23 & 11479.00 & 107076.19 & 0.98 & 0.96 & 0.93 \\
23282 & 103154 & 2000 & 815.90 & -0.08 & 75684.00 & 756830.52 & 1.08 & 0.93 & 1.00 \\
39729 & 107858 & 2000 & 12.90 & 0.11 & 1469.00 & 11193.86 & 0.88 & 0.87 & 0.76 \\
18545 & 102474 & 2000 & 155.70 & -0.22 & 25503.00 & 207657.37 & 0.61 & 1.33 & 0.81 \\
12783 & 101595 & 2000 & 3226.30 & -0.29 & 296646.00 & 2871122.59 & 1.09 & 0.89 & 0.97 \\
24903 & 103394 & 2000 & 31.40 & 0.22 & 3699.00 & 38282.05 & 0.85 & 1.22 & 1.03 \\
64011 & 500577 & 2000 & 60.20 & 0.06 & 5065.00 & 52094.58 & 1.19 & 0.87 & 1.03 \\
7114 & 100997 & 2000 & 222.50 & 0.17 & 22265.00 & 203516.23 & 1.00 & 0.91 & 0.91 \\
17994 & 102383 & 2000 & 8.10 & -0.05 & 849.00 & 7572.32 & 0.95 & 0.93 & 0.89 \\
39680 & 107835 & 2000 & 364.30 & 0.12 & 36352.00 & 361979.24 & 1.00 & 0.99 & 1.00 \\
23250 & 103152 & 2000 & 2882.50 & -0.05 & 287422.00 & 2799722.13 & 1.00 & 0.97 & 0.97 \\
49267 & 240261 & 2000 & 63.30 & 0.05 & 6188.00 & 63323.75 & 1.02 & 1.00 & 1.02 \\
1703 & 100226 & 2000 & 8515.40 & 0.09 & 658565.00 & 6303009.94 & 1.29 & 0.74 & 0.96 \\
12273 & 101531 & 2000 & 44.40 & -0.09 & 6961.00 & 56424.44 & 0.64 & 1.27 & 0.81 \\
39708 & 107837 & 2000 & 15.20 & 0.30 & 1541.00 & 12979.77 & 0.99 & 0.85 & 0.84 \\
18008 & 102386 & 2000 & 48.70 & 0.09 & 5251.00 & 54383.97 & 0.93 & 1.12 & 1.04 \\
24841 & 103381 & 2000 & 34939.20 & -0.20 & 4371768.00 & 34708240.58 & 0.80 & 0.99 & 0.79 \\
39674 & 107834 & 2000 & 219.10 & -0.12 & 21921.00 & 217854.30 & 1.00 & 0.99 & 0.99 \\
39804 & 107875 & 2000 & 306.70 & 0.06 & 34380.00 & 256828.79 & 0.89 & 0.84 & 0.75 \\
24645 & 103373 & 2000 & 94.40 & -0.05 & 9112.00 & 85110.53 & 1.04 & 0.90 & 0.93 \\
49212 & 240250 & 2000 & 279.90 & 0.30 & 27841.00 & 250433.94 & 1.01 & 0.89 & 0.90 \\
41008 & 108176 & 2000 & 23.00 & -0.18 & 2300.00 & 22997.49 & 1.00 & 1.00 & 1.00 \\
40999 & 108174 & 2000 & 25.40 & -0.22 & 3602.00 & 25837.62 & 0.71 & 1.02 & 0.72 \\
24616 & 103372 & 2000 & 955.60 & 0.14 & 92135.00 & 810282.35 & 1.04 & 0.85 & 0.88 \\
39894 & 107928 & 2000 & 2618.30 & -0.07 & 263957.00 & 2625609.63 & 0.99 & 1.00 & 0.99 \\
18494 & 102465 & 2000 & 489.10 & -0.03 & 56323.00 & 475095.09 & 0.87 & 0.97 & 0.84 \\
24593 & 103370 & 2000 & 367.80 & -0.14 & 36407.00 & 362933.75 & 1.01 & 0.99 & 1.00 \\
12359 & 101537 & 2000 & 739.30 & 0.07 & 73869.00 & 695579.80 & 1.00 & 0.94 & 0.94 \\
1858 & 100245 & 2000 & 633.50 & -0.05 & 64326.00 & 575954.74 & 0.98 & 0.91 & 0.90 \\
39920 & 107938 & 2000 & 131.90 & -0.02 & 13113.00 & 120241.51 & 1.01 & 0.91 & 0.92 \\
23413 & 103175 & 2000 & 892.00 & 0.10 & 88838.00 & 823367.65 & 1.00 & 0.92 & 0.93 \\
24561 & 103366 & 2000 & 209.20 & -0.06 & 19313.00 & 183922.80 & 1.08 & 0.88 & 0.95 \\
4808 & 100682 & 2000 & 88.80 & 0.22 & 7615.00 & 76750.92 & 1.17 & 0.86 & 1.01 \\
40980 & 108170 & 2000 & 1203.50 & -0.37 & 121389.00 & 1151747.80 & 0.99 & 0.96 & 0.95 \\
58781 & 410217 & 2000 & 2.70 & 0.14 & 257.00 & 2293.47 & 1.05 & 0.85 & 0.89 \\
23383 & 103174 & 2000 & 1398.30 & 0.01 & 139915.00 & 1388978.81 & 1.00 & 0.99 & 0.99 \\
39810 & 107877 & 2000 & 21.60 & -0.20 & 2164.00 & 21196.29 & 1.00 & 0.98 & 0.98 \\
18513 & 102469 & 2000 & 130.30 & 0.03 & 13043.00 & 123163.79 & 1.00 & 0.95 & 0.94 \\
48141 & 240027 & 2000 & 101.50 & 0.32 & 10206.00 & 97541.58 & 0.99 & 0.96 & 0.96 \\
24725 & 103376 & 2000 & 8484.60 & -0.15 & 1040938.00 & 7447269.63 & 0.82 & 0.88 & 0.72 \\
12326 & 101536 & 2000 & 2280.60 & 0.05 & 227299.00 & 2266898.06 & 1.00 & 0.99 & 1.00 \\
1786 & 100237 & 2000 & 121.30 & -0.07 & 12140.00 & 121342.37 & 1.00 & 1.00 & 1.00 \\
23351 & 103166 & 2000 & 4.10 & -0.29 & 638.00 & 4289.42 & 0.64 & 1.05 & 0.67 \\
24685 & 103375 & 2000 & 1520.70 & -0.16 & 197072.00 & 1429468.69 & 0.77 & 0.94 & 0.73 \\
64739 & 500625 & 2000 & 1254.00 & -0.06 & 114846.00 & 1099872.08 & 1.09 & 0.88 & 0.96 \\
4562 & 100639 & 2000 & 1630.40 & -0.03 & 160927.00 & 1589608.20 & 1.01 & 0.97 & 0.99 \\
49239 & 240254 & 2000 & 805.90 & 0.16 & 73750.00 & 767596.38 & 1.09 & 0.95 & 1.04 \\
1824 & 100244 & 2000 & 278.00 & -0.14 & 27819.00 & 241224.94 & 1.00 & 0.87 & 0.87 \\
39858 & 107883 & 2000 & 2107.70 & 0.03 & 230704.00 & 2093564.11 & 0.91 & 0.99 & 0.91 \\
39865 & 107892 & 2000 & 797.90 & -0.25 & 82630.00 & 680326.92 & 0.97 & 0.85 & 0.82 \\
4528 & 100637 & 2000 & 1095.50 & 0.14 & 101054.00 & 1003529.41 & 1.08 & 0.92 & 0.99 \\
39663 & 107833 & 2000 & 686.90 & -0.25 & 68696.00 & 683860.19 & 1.00 & 1.00 & 1.00 \\
1684 & 100223 & 2000 & 3629.80 & 0.02 & 353902.00 & 3380123.44 & 1.03 & 0.93 & 0.96 \\
41210 & 108673 & 2000 & 83.70 & 0.12 & 8477.00 & 82870.26 & 0.99 & 0.99 & 0.98 \\
25199 & 103460 & 2000 & 1587.80 & -0.20 & 197685.00 & 1515085.11 & 0.80 & 0.95 & 0.77 \\
49338 & 240284 & 2000 & 69.90 & -0.02 & 7878.00 & 74540.22 & 0.89 & 1.07 & 0.95 \\
1480 & 100207 & 2000 & 3106.90 & -0.13 & 295380.00 & 2908722.64 & 1.05 & 0.94 & 0.98 \\
23102 & 103122 & 2000 & 704.30 & -0.32 & 71961.00 & 693660.33 & 0.98 & 0.98 & 0.96 \\
45276 & 200011 & 2000 & 136.10 & 0.07 & 14262.00 & 128871.44 & 0.95 & 0.95 & 0.90 \\
1518 & 100209 & 2000 & 15726.30 & -0.12 & 1409996.00 & 14009115.21 & 1.12 & 0.89 & 0.99 \\
18590 & 102490 & 2000 & 95.50 & -0.06 & 9808.00 & 92121.73 & 0.97 & 0.96 & 0.94 \\
39504 & 107711 & 2000 & 13.20 & 0.01 & 1324.00 & 13062.66 & 1.00 & 0.99 & 0.99 \\
39509 & 107712 & 2000 & 17.50 & 0.01 & 1749.00 & 16317.70 & 1.00 & 0.93 & 0.93 \\
41185 & 108670 & 2000 & 575.00 & -0.30 & 58498.00 & 547307.91 & 0.98 & 0.95 & 0.94 \\
49317 & 240269 & 2000 & 135.20 & 0.29 & 12794.00 & 127918.67 & 1.06 & 0.95 & 1.00 \\
17904 & 102372 & 2000 & 2204.40 & 0.12 & 201132.00 & 2108275.76 & 1.10 & 0.96 & 1.05 \\
39517 & 107716 & 2000 & 215.00 & -0.24 & 25824.00 & 258165.68 & 0.83 & 1.20 & 1.00 \\
17884 & 102371 & 2000 & 329.40 & 0.18 & 32046.00 & 280427.31 & 1.03 & 0.85 & 0.88 \\
25147 & 103439 & 2000 & 76.20 & 0.08 & 8159.00 & 70785.22 & 0.93 & 0.93 & 0.87 \\
39476 & 107702 & 2000 & 457.00 & -0.11 & 48077.00 & 432351.86 & 0.95 & 0.95 & 0.90 \\
39473 & 107701 & 2000 & 73.30 & -0.14 & 7382.00 & 64881.59 & 0.99 & 0.89 & 0.88 \\
41265 & 108710 & 2000 & 1701.70 & 2.74 & 170913.00 & 1478637.06 & 1.00 & 0.87 & 0.87 \\
18606 & 102491 & 2000 & 345.20 & 0.08 & 33815.00 & 332631.71 & 1.02 & 0.96 & 0.98 \\
65062 & 500659 & 2000 & 13.80 & 0.11 & 1530.00 & 16017.71 & 0.90 & 1.16 & 1.05 \\
41237 & 108690 & 2000 & 34.80 & -0.38 & 3866.00 & 35018.10 & 0.90 & 1.01 & 0.91 \\
39455 & 107694 & 2000 & 5.60 & 0.04 & 564.00 & 5490.39 & 0.99 & 0.98 & 0.97 \\
2418 & 100323 & 2000 & 6380.60 & -0.10 & 658510.00 & 5523232.30 & 0.97 & 0.87 & 0.84 \\
4489 & 100635 & 2000 & 437.50 & 0.29 & 41826.00 & 418268.52 & 1.05 & 0.96 & 1.00 \\
14132 & 101805 & 2000 & 911.10 & -0.11 & 90719.00 & 890701.44 & 1.00 & 0.98 & 0.98 \\
17871 & 102367 & 2000 & 753.70 & 0.12 & 75698.00 & 667453.72 & 1.00 & 0.89 & 0.88 \\
12159 & 101513 & 2000 & 142.30 & 0.07 & 13627.00 & 133438.70 & 1.04 & 0.94 & 0.98 \\
1449 & 100200 & 2000 & 203.20 & 0.28 & 19829.00 & 193557.88 & 1.02 & 0.95 & 0.98 \\
12825 & 101601 & 2000 & 377.60 & -0.11 & 38372.00 & 367642.40 & 0.98 & 0.97 & 0.96 \\
39470 & 107699 & 2000 & 34.80 & -0.07 & 3459.00 & 29582.45 & 1.01 & 0.85 & 0.86 \\
23074 & 103110 & 2000 & 1051.70 & 0.20 & 111062.00 & 1037999.29 & 0.95 & 0.99 & 0.93 \\
1536 & 100213 & 2000 & 317.80 & 0.56 & 31778.00 & 292314.15 & 1.00 & 0.92 & 0.92 \\
25137 & 103436 & 2000 & 16.10 & -0.08 & 1609.00 & 15656.16 & 1.00 & 0.97 & 0.97 \\
12190 & 101518 & 2000 & 362.50 & 0.08 & 35626.00 & 353977.56 & 1.02 & 0.98 & 0.99 \\
18564 & 102483 & 2000 & 65.90 & 0.22 & 7025.00 & 73377.24 & 0.94 & 1.11 & 1.04 \\
12221 & 101523 & 2000 & 467.40 & -0.11 & 46767.00 & 451983.84 & 1.00 & 0.97 & 0.97 \\
39626 & 107830 & 2000 & 579.70 & -0.02 & 57982.00 & 554995.99 & 1.00 & 0.96 & 0.96 \\
24993 & 103406 & 2000 & 2586.00 & 0.20 & 258085.00 & 2494714.17 & 1.00 & 0.96 & 0.97 \\
12230 & 101528 & 2000 & 84.60 & 0.16 & 8890.00 & 81496.34 & 0.95 & 0.96 & 0.92 \\
18554 & 102482 & 2000 & 269.50 & 0.23 & 28397.00 & 288015.37 & 0.95 & 1.07 & 1.01 \\
39636 & 107832 & 2000 & 164.20 & 0.15 & 16769.00 & 159970.59 & 0.98 & 0.97 & 0.95 \\
2366 & 100320 & 2000 & 55.20 & 0.17 & 5497.00 & 54936.33 & 1.00 & 1.00 & 1.00 \\
24941 & 103395 & 2000 & 132.60 & 0.00 & 16541.00 & 138853.27 & 0.80 & 1.05 & 0.84 \\
12245 & 101530 & 2000 & 895.60 & 0.02 & 100899.00 & 924561.46 & 0.89 & 1.03 & 0.92 \\
23169 & 103138 & 2000 & 1967.40 & -0.09 & 390792.00 & 3361092.00 & 0.50 & 1.71 & 0.86 \\
45196 & 109437 & 2000 & 119.90 & -0.12 & 22029.00 & 220254.33 & 0.54 & 1.84 & 1.00 \\
25036 & 103426 & 2000 & 1677.40 & -0.15 & 194644.00 & 1523664.89 & 0.86 & 0.91 & 0.78 \\
2398 & 100322 & 2000 & 515.80 & 0.09 & 51449.00 & 499497.22 & 1.00 & 0.97 & 0.97 \\
39523 & 107719 & 2000 & 171.50 & -0.08 & 13525.00 & 116219.69 & 1.27 & 0.68 & 0.86 \\
12804 & 101600 & 2000 & 4076.20 & -0.12 & 446509.00 & 3672787.22 & 0.91 & 0.90 & 0.82 \\
25112 & 103432 & 2000 & 2614.90 & -0.01 & 305730.00 & 2886856.96 & 0.86 & 1.10 & 0.94 \\
18578 & 102489 & 2000 & 164.10 & -0.09 & 16749.00 & 167404.26 & 0.98 & 1.02 & 1.00 \\
23142 & 103134 & 2000 & 617.10 & -0.10 & 61812.00 & 594436.74 & 1.00 & 0.96 & 0.96 \\
13978 & 101794 & 2000 & 2041.80 & -0.14 & 202901.00 & 1884086.32 & 1.01 & 0.92 & 0.93 \\
39562 & 107722 & 2000 & 97.60 & 0.04 & 9774.00 & 97743.12 & 1.00 & 1.00 & 1.00 \\
49291 & 240266 & 2000 & 621.20 & -0.19 & 70745.00 & 497648.44 & 0.88 & 0.80 & 0.70 \\
8098 & 101074 & 2000 & 446.20 & 0.71 & 26844.00 & 268048.18 & 1.66 & 0.60 & 1.00 \\
1567 & 100214 & 2000 & 306.60 & 0.20 & 28717.00 & 253350.90 & 1.07 & 0.83 & 0.88 \\
23158 & 103136 & 2000 & 352.90 & -0.14 & 35005.00 & 349566.66 & 1.01 & 0.99 & 1.00 \\
39578 & 107726 & 2000 & 1000.70 & 0.01 & 100305.00 & 988763.42 & 1.00 & 0.99 & 0.99 \\
7770 & 101057 & 2000 & 6833.60 & 0.19 & 605666.00 & 5543347.96 & 1.13 & 0.81 & 0.92 \\
1887 & 100247 & 2000 & 1270.20 & 0.03 & 130227.00 & 1131944.30 & 0.98 & 0.89 & 0.87 \\
18112 & 102398 & 2000 & 10.30 & -0.06 & 1261.00 & 8616.18 & 0.82 & 0.84 & 0.68 \\
40368 & 108115 & 2000 & 1758.40 & -0.12 & 175076.00 & 1715294.19 & 1.00 & 0.98 & 0.98 \\
23657 & 103205 & 2000 & 39.60 & 0.24 & 3933.00 & 36731.84 & 1.01 & 0.93 & 0.93 \\
24055 & 103259 & 2000 & 1766.60 & 0.18 & 139957.00 & 1551087.86 & 1.26 & 0.88 & 1.11 \\
2035 & 100286 & 2000 & 55.00 & -0.01 & 5546.00 & 52653.04 & 0.99 & 0.96 & 0.95 \\
12547 & 101553 & 2000 & 573.40 & -0.25 & 57146.00 & 561600.55 & 1.00 & 0.98 & 0.98 \\
40394 & 108117 & 2000 & 1342.00 & 0.47 & 134196.00 & 1186775.23 & 1.00 & 0.88 & 0.88 \\
40831 & 108156 & 2000 & 24.80 & -0.11 & 2288.00 & 22668.12 & 1.08 & 0.91 & 0.99 \\
24033 & 103255 & 2000 & 183.10 & 0.02 & 20685.00 & 178588.41 & 0.89 & 0.98 & 0.86 \\
40821 & 108155 & 2000 & 91.10 & -0.01 & 9173.00 & 90820.81 & 0.99 & 1.00 & 0.99 \\
40420 & 108118 & 2000 & 51.80 & 0.02 & 4613.00 & 45327.54 & 1.12 & 0.88 & 0.98 \\
40428 & 108119 & 2000 & 206.10 & -0.10 & 22480.00 & 192332.00 & 0.92 & 0.93 & 0.86 \\
40816 & 108154 & 2000 & 57.50 & -0.09 & 6434.00 & 54908.88 & 0.89 & 0.95 & 0.85 \\
12640 & 101561 & 2000 & 108.20 & -0.22 & 11036.00 & 106146.83 & 0.98 & 0.98 & 0.96 \\
24003 & 103253 & 2000 & 152.50 & 0.24 & 12884.00 & 140456.01 & 1.18 & 0.92 & 1.09 \\
18399 & 102447 & 2000 & 2270.40 & 0.17 & 230597.00 & 2102562.67 & 0.98 & 0.93 & 0.91 \\
7146 & 100998 & 2000 & 92.30 & -0.05 & 9724.00 & 87818.96 & 0.95 & 0.95 & 0.90 \\
23982 & 103252 & 2000 & 268.80 & 0.08 & 22592.00 & 254192.47 & 1.19 & 0.95 & 1.13 \\
40342 & 108112 & 2000 & 23.90 & 0.11 & 2406.00 & 23907.98 & 0.99 & 1.00 & 0.99 \\
24086 & 103264 & 2000 & 1215.00 & -0.12 & 121861.00 & 1189916.86 & 1.00 & 0.98 & 0.98 \\
12506 & 101544 & 2000 & 278.90 & -0.12 & 28099.00 & 280921.92 & 0.99 & 1.01 & 1.00 \\
40260 & 108083 & 2000 & 110.80 & 0.50 & 12348.00 & 107440.43 & 0.90 & 0.97 & 0.87 \\
40891 & 108161 & 2000 & 407.00 & -0.11 & 40718.00 & 388914.11 & 1.00 & 0.96 & 0.96 \\
24145 & 103290 & 2000 & 1939.90 & 0.21 & 178054.00 & 1434276.77 & 1.09 & 0.74 & 0.81 \\
4724 & 100669 & 2000 & 380.10 & -0.08 & 38042.00 & 370674.91 & 1.00 & 0.98 & 0.97 \\
40874 & 108160 & 2000 & 27.80 & -0.08 & 2784.00 & 27300.58 & 1.00 & 0.98 & 0.98 \\
12518 & 101545 & 2000 & 160.60 & 0.05 & 14618.00 & 146041.44 & 1.10 & 0.91 & 1.00 \\
40286 & 108084 & 2000 & 16.90 & -0.10 & 1694.00 & 16936.68 & 1.00 & 1.00 & 1.00 \\
23626 & 103204 & 2000 & 297.90 & 0.27 & 29830.00 & 277472.21 & 1.00 & 0.93 & 0.93 \\
24114 & 103267 & 2000 & 719.70 & -0.06 & 70082.00 & 700527.97 & 1.03 & 0.97 & 1.00 \\
40316 & 108109 & 2000 & 27.10 & 0.04 & 2542.00 & 24475.64 & 1.07 & 0.90 & 0.96 \\
40850 & 108159 & 2000 & 25.50 & 0.03 & 2433.00 & 23965.24 & 1.05 & 0.94 & 0.99 \\
24098 & 103266 & 2000 & 691.80 & -0.38 & 68553.00 & 635632.90 & 1.01 & 0.92 & 0.93 \\
49063 & 240212 & 2000 & 7430.90 & 0.39 & 595235.00 & 6181881.41 & 1.25 & 0.83 & 1.04 \\
4703 & 100667 & 2000 & 15.60 & 0.15 & 1560.00 & 13602.50 & 1.00 & 0.87 & 0.87 \\
40841 & 108158 & 2000 & 27.70 & -0.11 & 2779.00 & 27103.52 & 1.00 & 0.98 & 0.98 \\
40838 & 108157 & 2000 & 14.70 & -0.59 & 1528.00 & 15261.86 & 0.96 & 1.04 & 1.00 \\
40480 & 108122 & 2000 & 53.70 & 0.32 & 5322.00 & 51555.54 & 1.01 & 0.96 & 0.97 \\
4676 & 100660 & 2000 & 844.10 & 0.01 & 157936.00 & 1486955.50 & 0.53 & 1.76 & 0.94 \\
23690 & 103208 & 2000 & 1719.00 & -0.01 & 170511.00 & 1462567.85 & 1.01 & 0.85 & 0.86 \\
4642 & 100659 & 2000 & 542.70 & 0.06 & 74396.00 & 608832.82 & 0.73 & 1.12 & 0.82 \\
23859 & 103224 & 2000 & 279.10 & -0.19 & 39127.00 & 257131.08 & 0.71 & 0.92 & 0.66 \\
12596 & 101557 & 2000 & 62.90 & 0.01 & 7829.00 & 70931.98 & 0.80 & 1.13 & 0.91 \\
2099 & 100291 & 2000 & 340.40 & 0.18 & 34926.00 & 298381.25 & 0.97 & 0.88 & 0.85 \\
40574 & 108139 & 2000 & 94.40 & -0.09 & 14231.00 & 135954.38 & 0.66 & 1.44 & 0.96 \\
40578 & 108140 & 2000 & 135.90 & 0.11 & 13456.00 & 134547.17 & 1.01 & 0.99 & 1.00 \\
18355 & 102446 & 2000 & 17.40 & 0.09 & 1747.00 & 15532.48 & 1.00 & 0.89 & 0.89 \\
23760 & 103212 & 2000 & 3339.80 & 0.08 & 330221.00 & 2789655.47 & 1.01 & 0.84 & 0.84 \\
23826 & 103214 & 2000 & 3433.30 & -0.09 & 345348.00 & 3200300.88 & 0.99 & 0.93 & 0.93 \\
40687 & 108145 & 2000 & 91.40 & 0.41 & 8464.00 & 79164.45 & 1.08 & 0.87 & 0.94 \\
23796 & 103213 & 2000 & 1574.40 & -0.13 & 159259.00 & 1512889.60 & 0.99 & 0.96 & 0.95 \\
12610 & 101560 & 2000 & 22.90 & -0.29 & 2344.00 & 22555.20 & 0.98 & 0.98 & 0.96 \\
7908 & 101065 & 2000 & 4431.20 & 0.33 & 487543.00 & 4078933.51 & 0.91 & 0.92 & 0.84 \\
23877 & 103226 & 2000 & 187.30 & -0.15 & 21666.00 & 151816.04 & 0.86 & 0.81 & 0.70 \\
40737 & 108147 & 2000 & 23.90 & -0.15 & 2801.00 & 20941.36 & 0.85 & 0.88 & 0.75 \\
23961 & 103251 & 2000 & 512.70 & -0.03 & 59811.00 & 473385.50 & 0.86 & 0.92 & 0.79 \\
12571 & 101554 & 2000 & 292.80 & -0.00 & 29297.00 & 289003.94 & 1.00 & 0.99 & 0.99 \\
40791 & 108153 & 2000 & 5.40 & 0.13 & 546.00 & 5262.28 & 0.99 & 0.97 & 0.96 \\
40506 & 108134 & 2000 & 86.30 & -0.16 & 10520.00 & 76976.56 & 0.82 & 0.89 & 0.73 \\
40766 & 108149 & 2000 & 26.30 & 0.22 & 4836.00 & 47557.07 & 0.54 & 1.81 & 0.98 \\
40531 & 108136 & 2000 & 20.30 & 0.07 & 2007.00 & 20078.41 & 1.01 & 0.99 & 1.00 \\
48104 & 240010 & 2000 & 172.40 & 0.26 & 17192.00 & 168346.35 & 1.00 & 0.98 & 0.98 \\
4596 & 100642 & 2000 & 1505.50 & -0.10 & 153483.00 & 1457521.85 & 0.98 & 0.97 & 0.95 \\
48998 & 240198 & 2000 & 2301.10 & 0.09 & 223320.00 & 2332074.71 & 1.03 & 1.01 & 1.04 \\
23931 & 103242 & 2000 & 85.10 & 0.26 & 8611.00 & 84906.61 & 0.99 & 1.00 & 0.99 \\
40538 & 108137 & 2000 & 1.40 & 0.31 & 120.00 & 1270.73 & 1.17 & 0.91 & 1.06 \\
23724 & 103209 & 2000 & 165.40 & 0.04 & 16040.00 & 131596.11 & 1.03 & 0.80 & 0.82 \\
23911 & 103232 & 2000 & 312.50 & -0.11 & 34813.00 & 283584.28 & 0.90 & 0.91 & 0.81 \\
2061 & 100287 & 2000 & 67.90 & 0.02 & 6758.00 & 63995.72 & 1.00 & 0.94 & 0.95 \\
40741 & 108148 & 2000 & 157.20 & 0.31 & 22298.00 & 221335.37 & 0.70 & 1.41 & 0.99 \\
15045 & 101953 & 2000 & 402.60 & 0.01 & 42046.00 & 428333.84 & 0.96 & 1.06 & 1.02 \\
1419 & 100196 & 2000 & 2089.80 & 0.18 & 187851.00 & 1839622.66 & 1.11 & 0.88 & 0.98 \\
23602 & 103202 & 2000 & 67.30 & -0.08 & 5836.00 & 56618.34 & 1.15 & 0.84 & 0.97 \\
40233 & 108082 & 2000 & 12.90 & 0.12 & 1292.00 & 12326.92 & 1.00 & 0.96 & 0.95 \\
24451 & 103327 & 2000 & 1881.40 & 0.06 & 192690.00 & 1941796.48 & 0.98 & 1.03 & 1.01 \\
12410 & 101539 & 2000 & 1699.90 & 0.05 & 169971.00 & 1699645.20 & 1.00 & 1.00 & 1.00 \\
40000 & 107992 & 2000 & 54.40 & -0.13 & 5495.00 & 51819.30 & 0.99 & 0.95 & 0.94 \\
2233 & 100298 & 2000 & 600.30 & 0.05 & 59747.00 & 597126.54 & 1.00 & 0.99 & 1.00 \\
2221 & 100296 & 2000 & 11.40 & -0.18 & 1466.00 & 13631.35 & 0.78 & 1.20 & 0.93 \\
23478 & 103179 & 2000 & 469.20 & -0.11 & 44884.00 & 405377.35 & 1.05 & 0.86 & 0.90 \\
49147 & 240234 & 2000 & 60.30 & 0.20 & 5963.00 & 55246.73 & 1.01 & 0.92 & 0.93 \\
1935 & 100259 & 2000 & 170.70 & 0.14 & 16685.00 & 166651.84 & 1.02 & 0.98 & 1.00 \\
24405 & 103319 & 2000 & 347.10 & -0.07 & 34693.00 & 329203.56 & 1.00 & 0.95 & 0.95 \\
2200 & 100295 & 2000 & 18.90 & 0.05 & 2010.00 & 15292.11 & 0.94 & 0.81 & 0.76 \\
4762 & 100671 & 2000 & 254.20 & -0.02 & 39115.00 & 379771.47 & 0.65 & 1.49 & 0.97 \\
40036 & 108013 & 2000 & 306.60 & -0.13 & 31270.00 & 307342.71 & 0.98 & 1.00 & 0.98 \\
23499 & 103182 & 2000 & 306.40 & 0.07 & 30924.00 & 290811.65 & 0.99 & 0.95 & 0.94 \\
7165 & 101000 & 2000 & 1413.20 & 0.03 & 140707.00 & 1382482.55 & 1.00 & 0.98 & 0.98 \\
24371 & 103318 & 2000 & 1821.80 & -0.03 & 194652.00 & 1881478.25 & 0.94 & 1.03 & 0.97 \\
18472 & 102462 & 2000 & 6.90 & 0.11 & 701.00 & 6994.68 & 0.98 & 1.01 & 1.00 \\
24473 & 103328 & 2000 & 1116.90 & -0.16 & 156952.00 & 1142361.50 & 0.71 & 1.02 & 0.73 \\
24547 & 103347 & 2000 & 4.50 & -0.36 & 450.00 & 4489.22 & 1.00 & 1.00 & 1.00 \\
8029 & 101071 & 2000 & 7011.30 & 0.32 & 660043.00 & 5338663.53 & 1.06 & 0.76 & 0.81 \\
40951 & 108166 & 2000 & 247.60 & -0.21 & 24553.00 & 245407.59 & 1.01 & 0.99 & 1.00 \\
24535 & 103339 & 2000 & 1719.80 & -0.22 & 173428.00 & 1666321.81 & 0.99 & 0.97 & 0.96 \\
48069 & 235413 & 2000 & 61.80 & 0.11 & 6146.00 & 53821.45 & 1.01 & 0.87 & 0.88 \\
18121 & 102399 & 2000 & 8.90 & -0.12 & 1238.00 & 9038.49 & 0.72 & 1.02 & 0.73 \\
39931 & 107958 & 2000 & 10.50 & 0.15 & 954.00 & 10959.68 & 1.10 & 1.04 & 1.15 \\
7833 & 101062 & 2000 & 1037.40 & 0.31 & 92430.00 & 863038.27 & 1.12 & 0.83 & 0.93 \\
18130 & 102404 & 2000 & 3309.40 & -0.10 & 331310.00 & 3282007.84 & 1.00 & 0.99 & 0.99 \\
24503 & 103329 & 2000 & 881.70 & 0.16 & 84810.00 & 881897.95 & 1.04 & 1.00 & 1.04 \\
12390 & 101538 & 2000 & 185.00 & -0.19 & 18242.00 & 179713.04 & 1.01 & 0.97 & 0.99 \\
23447 & 103177 & 2000 & 194.20 & -0.03 & 19053.00 & 175390.52 & 1.02 & 0.90 & 0.92 \\
39956 & 107960 & 2000 & 30.20 & 0.15 & 3024.00 & 29490.56 & 1.00 & 0.98 & 0.98 \\
40949 & 108165 & 2000 & 14.50 & -0.14 & 1318.00 & 11679.49 & 1.10 & 0.81 & 0.89 \\
39980 & 107967 & 2000 & 186.30 & -0.06 & 18828.00 & 164238.62 & 0.99 & 0.88 & 0.87 \\
39990 & 107968 & 2000 & 93.70 & 0.18 & 9252.00 & 85622.60 & 1.01 & 0.91 & 0.93 \\
2167 & 100293 & 2000 & 241.00 & -0.05 & 24100.00 & 223898.16 & 1.00 & 0.93 & 0.93 \\
8010 & 101069 & 2000 & 9572.20 & -0.09 & 1030369.00 & 9309676.01 & 0.93 & 0.97 & 0.90 \\
40044 & 108018 & 2000 & 337.00 & 0.37 & 32938.00 & 280820.65 & 1.02 & 0.83 & 0.85 \\
1983 & 100278 & 2000 & 119.20 & -0.09 & 11909.00 & 117655.39 & 1.00 & 0.99 & 0.99 \\
23572 & 103193 & 2000 & 47.30 & 0.22 & 4707.00 & 45436.95 & 1.00 & 0.96 & 0.97 \\
40162 & 108070 & 2000 & 156.60 & 0.15 & 15250.00 & 152514.60 & 1.03 & 0.97 & 1.00 \\
24234 & 103299 & 2000 & 520.20 & -0.14 & 63094.00 & 474970.26 & 0.82 & 0.91 & 0.75 \\
40182 & 108071 & 2000 & 125.30 & -0.03 & 14156.00 & 141545.42 & 0.89 & 1.13 & 1.00 \\
40900 & 108162 & 2000 & 225.90 & -0.03 & 23140.00 & 215660.53 & 0.98 & 0.95 & 0.93 \\
18227 & 102417 & 2000 & 1576.40 & -0.14 & 187573.00 & 1411346.36 & 0.84 & 0.90 & 0.75 \\
7978 & 101068 & 2000 & 93066.90 & -0.09 & 10409142.00 & 88355237.45 & 0.89 & 0.95 & 0.85 \\
48984 & 240197 & 2000 & 1340.00 & -0.26 & 183037.00 & 1322774.21 & 0.73 & 0.99 & 0.72 \\
24199 & 103296 & 2000 & 2466.60 & -0.14 & 291544.00 & 2453830.23 & 0.85 & 0.99 & 0.84 \\
2007 & 100280 & 2000 & 48.20 & -0.06 & 4907.00 & 47638.83 & 0.98 & 0.99 & 0.97 \\
40222 & 108074 & 2000 & 11.40 & -0.22 & 1137.00 & 11381.59 & 1.00 & 1.00 & 1.00 \\
12478 & 101542 & 2000 & 64.10 & 0.01 & 6045.00 & 54426.21 & 1.06 & 0.85 & 0.90 \\
4740 & 100670 & 2000 & 106.10 & 0.03 & 10448.00 & 102899.96 & 1.02 & 0.97 & 0.98 \\
18180 & 102414 & 2000 & 5919.40 & -0.02 & 666555.00 & 4834153.96 & 0.89 & 0.82 & 0.73 \\
12444 & 101541 & 2000 & 335.20 & -0.09 & 32918.00 & 329118.18 & 1.02 & 0.98 & 1.00 \\
40072 & 108021 & 2000 & 971.90 & 0.52 & 97188.00 & 908057.69 & 1.00 & 0.93 & 0.93 \\
24343 & 103315 & 2000 & 59.10 & 0.10 & 5913.00 & 57337.55 & 1.00 & 0.97 & 0.97 \\
40096 & 108029 & 2000 & 602.10 & -0.05 & 67713.00 & 630731.54 & 0.89 & 1.05 & 0.93 \\
14022 & 101800 & 2000 & 895.30 & -0.16 & 90292.00 & 826878.11 & 0.99 & 0.92 & 0.92 \\
24317 & 103308 & 2000 & 8517.90 & 0.09 & 902328.00 & 8247316.58 & 0.94 & 0.97 & 0.91 \\
1970 & 100263 & 2000 & 6.80 & -0.04 & 633.00 & 5881.90 & 1.07 & 0.86 & 0.93 \\
40123 & 108030 & 2000 & 41.70 & 0.24 & 3985.00 & 36860.03 & 1.05 & 0.88 & 0.92 \\
23546 & 103184 & 2000 & 500.70 & 0.20 & 48987.00 & 459911.97 & 1.02 & 0.92 & 0.94 \\
23553 & 103186 & 2000 & 1267.60 & -0.11 & 140526.00 & 1140718.68 & 0.90 & 0.90 & 0.81 \\
40152 & 108051 & 2000 & 517.00 & 0.14 & 52373.00 & 481791.30 & 0.99 & 0.93 & 0.92 \\
24283 & 103304 & 2000 & 31.60 & 0.11 & 2785.00 & 24080.69 & 1.13 & 0.76 & 0.86 \\
2136 & 100292 & 2000 & 285.50 & 0.29 & 28288.00 & 256709.82 & 1.01 & 0.90 & 0.91 \\
49106 & 240222 & 2000 & 886.80 & -0.13 & 88846.00 & 867337.86 & 1.00 & 0.98 & 0.98 \\
17384 & 102284 & 2000 & 221.80 & 0.27 & 22082.00 & 218041.45 & 1.00 & 0.98 & 0.99 \\
4889 & 100691 & 2000 & 1533.50 & -0.05 & 153155.00 & 1427321.59 & 1.00 & 0.93 & 0.93 \\
5051 & 100710 & 2000 & 512.50 & 0.02 & 56316.00 & 539053.94 & 0.91 & 1.05 & 0.96 \\
38751 & 107320 & 2000 & 288.50 & -0.27 & 28963.00 & 285583.31 & 1.00 & 0.99 & 0.99 \\
63525 & 500514 & 2000 & 148.70 & 0.06 & 15200.00 & 148878.72 & 0.98 & 1.00 & 0.98 \\
38755 & 107321 & 2000 & 17.20 & -0.05 & 1722.00 & 16793.43 & 1.00 & 0.98 & 0.98 \\
38759 & 107322 & 2000 & 11.00 & 0.05 & 1109.00 & 10438.76 & 0.99 & 0.95 & 0.94 \\
5027 & 100701 & 2000 & 41.60 & 0.17 & 4168.00 & 39101.09 & 1.00 & 0.94 & 0.94 \\
2650 & 100350 & 2000 & 257.00 & -0.04 & 25410.00 & 253034.80 & 1.01 & 0.98 & 1.00 \\
47988 & 225687 & 2000 & 1768.10 & -0.44 & 176256.00 & 1589207.71 & 1.00 & 0.90 & 0.90 \\
26106 & 103539 & 2000 & 954.40 & 0.19 & 79498.00 & 867525.49 & 1.20 & 0.91 & 1.09 \\
22637 & 103027 & 2000 & 4789.80 & 0.19 & 458824.00 & 3915352.00 & 1.04 & 0.82 & 0.85 \\
5017 & 100700 & 2000 & 90.10 & 0.15 & 8870.00 & 87869.47 & 1.02 & 0.98 & 0.99 \\
49446 & 240301 & 2000 & 11.20 & -0.15 & 1099.00 & 9958.01 & 1.02 & 0.89 & 0.91 \\
17549 & 102318 & 2000 & 4088.30 & -0.01 & 361092.00 & 3499608.79 & 1.13 & 0.86 & 0.97 \\
38782 & 107323 & 2000 & 56.10 & 0.15 & 5559.00 & 53475.29 & 1.01 & 0.95 & 0.96 \\
38807 & 107328 & 2000 & 32.80 & 0.15 & 3341.00 & 31084.64 & 0.98 & 0.95 & 0.93 \\
38819 & 107329 & 2000 & 50.40 & -0.23 & 5413.00 & 46811.37 & 0.93 & 0.93 & 0.86 \\
38747 & 107319 & 2000 & 180.90 & -0.08 & 18520.00 & 185200.87 & 0.98 & 1.02 & 1.00 \\
41556 & 108766 & 2000 & 598.80 & 0.10 & 55553.00 & 508485.01 & 1.08 & 0.85 & 0.92 \\
3106 & 100409 & 2000 & 387.80 & 0.07 & 39551.00 & 374344.47 & 0.98 & 0.97 & 0.95 \\
26140 & 103544 & 2000 & 21127.70 & -0.11 & 2459261.00 & 20954833.97 & 0.86 & 0.99 & 0.85 \\
17506 & 102317 & 2000 & 18.90 & 0.16 & 1773.00 & 17733.54 & 1.07 & 0.94 & 1.00 \\
38686 & 107308 & 2000 & 1014.00 & 0.25 & 101922.00 & 964105.06 & 0.99 & 0.95 & 0.95 \\
7665 & 101054 & 2000 & 15082.30 & -0.11 & 1631105.00 & 14880620.40 & 0.92 & 0.99 & 0.91 \\
59049 & 410418 & 2000 & 1202.80 & -0.06 & 146136.00 & 1282519.46 & 0.82 & 1.07 & 0.88 \\
26174 & 103545 & 2000 & 25082.80 & 0.17 & 2293462.00 & 24009564.31 & 1.09 & 0.96 & 1.05 \\
22600 & 103024 & 2000 & 451.50 & 0.16 & 44783.00 & 399882.69 & 1.01 & 0.89 & 0.89 \\
49470 & 240302 & 2000 & 5.30 & -0.02 & 500.00 & 4437.52 & 1.06 & 0.84 & 0.89 \\
979 & 100113 & 2000 & 1027.70 & -0.16 & 103476.00 & 987345.13 & 0.99 & 0.96 & 0.95 \\
38711 & 107309 & 2000 & 211.10 & -0.12 & 25202.00 & 227150.98 & 0.84 & 1.08 & 0.90 \\
41596 & 108777 & 2000 & 28.70 & 0.04 & 2873.00 & 29721.17 & 1.00 & 1.04 & 1.03 \\
45575 & 200078 & 2000 & 2.70 & -0.13 & 280.00 & 2799.75 & 0.96 & 1.04 & 1.00 \\
38719 & 107310 & 2000 & 67.00 & -0.06 & 7576.00 & 61807.53 & 0.88 & 0.92 & 0.82 \\
53712 & 356500 & 2000 & 437.70 & -0.04 & 43257.00 & 426921.35 & 1.01 & 0.98 & 0.99 \\
38723 & 107316 & 2000 & 536.10 & 0.11 & 53603.00 & 533672.42 & 1.00 & 1.00 & 1.00 \\
41571 & 108776 & 2000 & 173.30 & 0.18 & 15340.00 & 138323.10 & 1.13 & 0.80 & 0.90 \\
47975 & 225484 & 2000 & 83.40 & -0.05 & 8212.00 & 79089.60 & 1.02 & 0.95 & 0.96 \\
11905 & 101465 & 2000 & 110.30 & -0.04 & 11692.00 & 105048.44 & 0.94 & 0.95 & 0.90 \\
38663 & 107306 & 2000 & 241.60 & 0.05 & 25193.00 & 241466.11 & 0.96 & 1.00 & 0.96 \\
45544 & 200074 & 2000 & 70.00 & -0.05 & 8465.00 & 65056.85 & 0.83 & 0.93 & 0.77 \\
41549 & 108765 & 2000 & 1266.00 & -0.13 & 131912.00 & 1248362.16 & 0.96 & 0.99 & 0.95 \\
45506 & 200071 & 2000 & 524.70 & 0.03 & 54479.00 & 505249.38 & 0.96 & 0.96 & 0.93 \\
2612 & 100347 & 2000 & 1109.90 & -0.10 & 115491.00 & 1058755.05 & 0.96 & 0.95 & 0.92 \\
25961 & 103531 & 2000 & 1351.10 & 0.36 & 104070.00 & 1245543.52 & 1.30 & 0.92 & 1.20 \\
41484 & 108761 & 2000 & 5.00 & 0.45 & 484.00 & 4840.92 & 1.03 & 0.97 & 1.00 \\
12915 & 101606 & 2000 & 8464.40 & -0.19 & 864797.00 & 8201595.75 & 0.98 & 0.97 & 0.95 \\
38875 & 107339 & 2000 & 130.70 & -0.18 & 14058.00 & 115386.55 & 0.93 & 0.88 & 0.82 \\
38881 & 107350 & 2000 & 964.90 & 0.11 & 86365.00 & 790462.20 & 1.12 & 0.82 & 0.92 \\
63648 & 500539 & 2000 & 9.20 & -0.13 & 915.00 & 9152.41 & 1.01 & 0.99 & 1.00 \\
25923 & 103529 & 2000 & 8967.60 & -0.09 & 934792.00 & 8892557.51 & 0.96 & 0.99 & 0.95 \\
1073 & 100150 & 2000 & 23.60 & 0.04 & 2109.00 & 22012.82 & 1.12 & 0.93 & 1.04 \\
38922 & 107354 & 2000 & 44.50 & 0.31 & 4497.00 & 39809.50 & 0.99 & 0.89 & 0.89 \\
4998 & 100698 & 2000 & 37.90 & 0.39 & 3676.00 & 34959.42 & 1.03 & 0.92 & 0.95 \\
26059 & 103536 & 2000 & 2414.80 & -0.09 & 254763.00 & 2313177.48 & 0.95 & 0.96 & 0.91 \\
49441 & 240300 & 2000 & 104.50 & -0.15 & 12367.00 & 92338.69 & 0.84 & 0.88 & 0.75 \\
38823 & 107331 & 2000 & 4.10 & 0.16 & 405.00 & 3761.24 & 1.01 & 0.92 & 0.93 \\
38844 & 107336 & 2000 & 125.20 & 0.16 & 12514.00 & 100976.89 & 1.00 & 0.81 & 0.81 \\
45535 & 200073 & 2000 & 736.60 & 0.22 & 70091.00 & 581084.60 & 1.05 & 0.79 & 0.83 \\
2631 & 100348 & 2000 & 262.30 & -0.04 & 27356.00 & 250688.85 & 0.96 & 0.96 & 0.92 \\
26029 & 103535 & 2000 & 2853.60 & -0.10 & 302577.00 & 2636892.33 & 0.94 & 0.92 & 0.87 \\
22681 & 103028 & 2000 & 5010.20 & 0.00 & 560883.00 & 4302264.17 & 0.89 & 0.86 & 0.77 \\
49432 & 240297 & 2000 & 735.70 & 0.07 & 77369.00 & 682355.48 & 0.95 & 0.93 & 0.88 \\
1023 & 100127 & 2000 & 9086.60 & -0.25 & 921211.00 & 7594179.89 & 0.99 & 0.84 & 0.82 \\
38857 & 107337 & 2000 & 20.40 & -0.10 & 2034.00 & 19643.63 & 1.00 & 0.96 & 0.97 \\
41531 & 108764 & 2000 & 309.40 & -0.10 & 31659.00 & 309183.34 & 0.98 & 1.00 & 0.98 \\
14243 & 101835 & 2000 & 1823.60 & -0.09 & 182241.00 & 1810138.97 & 1.00 & 0.99 & 0.99 \\
58044 & 410060 & 2000 & 20.90 & -0.21 & 2073.00 & 20234.67 & 1.01 & 0.97 & 0.98 \\
26015 & 103533 & 2000 & 18466.80 & 0.61 & 1307701.00 & 13435777.62 & 1.41 & 0.73 & 1.03 \\
7046 & 100992 & 2000 & 1503.00 & -0.20 & 152883.00 & 1444807.61 & 0.98 & 0.96 & 0.95 \\
17583 & 102319 & 2000 & 690.10 & -0.02 & 65493.00 & 653882.19 & 1.05 & 0.95 & 1.00 \\
38865 & 107338 & 2000 & 26.10 & -0.09 & 2719.00 & 25226.90 & 0.96 & 0.97 & 0.93 \\
26214 & 103546 & 2000 & 16614.90 & 0.16 & 1602648.00 & 15854997.92 & 1.04 & 0.95 & 0.99 \\
12950 & 101616 & 2000 & 27447.60 & -0.09 & 2752079.00 & 25680696.50 & 1.00 & 0.94 & 0.93 \\
38655 & 107303 & 2000 & 62.60 & -0.21 & 6951.00 & 56966.65 & 0.90 & 0.91 & 0.82 \\
14303 & 101843 & 2000 & 250.10 & -0.03 & 32470.00 & 315220.97 & 0.77 & 1.26 & 0.97 \\
38389 & 107253 & 2000 & 89.40 & -0.01 & 8935.00 & 77880.75 & 1.00 & 0.87 & 0.87 \\
5114 & 100724 & 2000 & 61.10 & -0.08 & 6094.00 & 57558.65 & 1.00 & 0.94 & 0.94 \\
48903 & 240152 & 2000 & 170.80 & -0.04 & 17496.00 & 150721.47 & 0.98 & 0.88 & 0.86 \\
26484 & 103582 & 2000 & 22.60 & 0.03 & 2204.00 & 21521.19 & 1.03 & 0.95 & 0.98 \\
22501 & 103016 & 2000 & 274.10 & 0.04 & 36118.00 & 342425.86 & 0.76 & 1.25 & 0.95 \\
863 & 100099 & 2000 & 172.00 & -0.23 & 17196.00 & 148054.72 & 1.00 & 0.86 & 0.86 \\
38398 & 107257 & 2000 & 378.20 & -0.02 & 38011.00 & 358299.05 & 0.99 & 0.95 & 0.94 \\
38423 & 107258 & 2000 & 137.40 & -0.07 & 13802.00 & 137836.81 & 1.00 & 1.00 & 1.00 \\
41668 & 108827 & 2000 & 1058.70 & -0.16 & 106372.00 & 928293.36 & 1.00 & 0.88 & 0.87 \\
38432 & 107259 & 2000 & 96.80 & 0.37 & 8717.00 & 87180.80 & 1.11 & 0.90 & 1.00 \\
4400 & 100622 & 2000 & 425.50 & 0.19 & 43545.00 & 413839.74 & 0.98 & 0.97 & 0.95 \\
47947 & 225413 & 2000 & 21.90 & 0.34 & 2185.00 & 21380.89 & 1.00 & 0.98 & 0.98 \\
26433 & 103580 & 2000 & 533.40 & 0.09 & 57344.00 & 534116.66 & 0.93 & 1.00 & 0.93 \\
17437 & 102306 & 2000 & 21790.90 & 0.10 & 2193826.00 & 19000561.41 & 0.99 & 0.87 & 0.87 \\
38457 & 107260 & 2000 & 3.20 & 0.16 & 299.00 & 2989.33 & 1.07 & 0.93 & 1.00 \\
41643 & 108826 & 2000 & 152.50 & 0.25 & 15504.00 & 132866.32 & 0.98 & 0.87 & 0.86 \\
38382 & 107250 & 2000 & 12.60 & -0.24 & 1340.00 & 13403.51 & 0.94 & 1.06 & 1.00 \\
73363 & 600006 & 2000 & 106.40 & 0.04 & 12138.00 & 92099.17 & 0.88 & 0.87 & 0.76 \\
4382 & 100614 & 2000 & 757.80 & 0.15 & 75586.00 & 712026.14 & 1.00 & 0.94 & 0.94 \\
48212 & 240051 & 2000 & 382.50 & -0.00 & 35803.00 & 352072.63 & 1.07 & 0.92 & 0.98 \\
38317 & 107243 & 2000 & 1314.60 & 0.17 & 95364.00 & 869689.18 & 1.38 & 0.66 & 0.91 \\
4356 & 100611 & 2000 & 1146.00 & 0.18 & 115358.00 & 1116618.01 & 0.99 & 0.97 & 0.97 \\
26553 & 103591 & 2000 & 408.10 & 0.37 & 40813.00 & 345732.95 & 1.00 & 0.85 & 0.85 \\
7640 & 101050 & 2000 & 453.40 & -0.05 & 54843.00 & 558318.52 & 0.83 & 1.23 & 1.02 \\
22468 & 103014 & 2000 & 335.00 & -0.14 & 42018.00 & 347265.74 & 0.80 & 1.04 & 0.83 \\
38342 & 107244 & 2000 & 787.70 & -0.28 & 78692.00 & 769344.39 & 1.00 & 0.98 & 0.98 \\
41676 & 108839 & 2000 & 1432.90 & -0.11 & 142066.00 & 1384170.41 & 1.01 & 0.97 & 0.97 \\
38373 & 107248 & 2000 & 309.70 & 0.32 & 23896.00 & 226192.20 & 1.30 & 0.73 & 0.95 \\
63474 & 500511 & 2000 & 1278.70 & -0.05 & 159809.00 & 1380008.98 & 0.80 & 1.08 & 0.86 \\
22486 & 103015 & 2000 & 470.60 & -0.06 & 52239.00 & 429427.65 & 0.90 & 0.91 & 0.82 \\
26521 & 103590 & 2000 & 240.30 & 0.17 & 24028.00 & 219736.97 & 1.00 & 0.91 & 0.91 \\
11812 & 101462 & 2000 & 665.90 & -0.07 & 78728.00 & 713983.49 & 0.85 & 1.07 & 0.91 \\
5134 & 100726 & 2000 & 11470.50 & -0.10 & 1152804.00 & 9555810.94 & 1.00 & 0.83 & 0.83 \\
22528 & 103017 & 2000 & 2623.70 & -0.09 & 293995.00 & 2829268.28 & 0.89 & 1.08 & 0.96 \\
11844 & 101463 & 2000 & 1231.60 & -0.10 & 128398.00 & 1187915.22 & 0.96 & 0.96 & 0.93 \\
22569 & 103021 & 2000 & 41.50 & 0.11 & 4121.00 & 38572.52 & 1.01 & 0.93 & 0.94 \\
26292 & 103564 & 2000 & 822.40 & 0.14 & 82390.00 & 803236.93 & 1.00 & 0.98 & 0.97 \\
11875 & 101464 & 2000 & 225.00 & 0.32 & 21921.00 & 219467.09 & 1.03 & 0.98 & 1.00 \\
935 & 100112 & 2000 & 6093.90 & 0.13 & 603656.00 & 5545361.98 & 1.01 & 0.91 & 0.92 \\
41620 & 108782 & 2000 & 588.80 & 0.04 & 60533.00 & 570692.95 & 0.97 & 0.97 & 0.94 \\
17487 & 102313 & 2000 & 638.50 & -0.14 & 64583.00 & 631247.90 & 0.99 & 0.99 & 0.98 \\
38620 & 107300 & 2000 & 85.10 & -0.20 & 9149.00 & 71832.06 & 0.93 & 0.84 & 0.79 \\
41606 & 108780 & 2000 & 105.60 & -0.09 & 10122.00 & 90730.66 & 1.04 & 0.86 & 0.90 \\
26245 & 103547 & 2000 & 16232.10 & -0.20 & 1873582.00 & 16871629.17 & 0.87 & 1.04 & 0.90 \\
2669 & 100351 & 2000 & 112.40 & -0.07 & 11247.00 & 107931.98 & 1.00 & 0.96 & 0.96 \\
49485 & 240304 & 2000 & 133.00 & 0.45 & 13183.00 & 117745.51 & 1.01 & 0.89 & 0.89 \\
14276 & 101842 & 2000 & 3183.10 & -0.09 & 317327.00 & 3166774.46 & 1.00 & 0.99 & 1.00 \\
38630 & 107302 & 2000 & 492.70 & -0.07 & 49276.00 & 487923.43 & 1.00 & 0.99 & 0.99 \\
18757 & 102507 & 2000 & 539.80 & -0.28 & 53976.00 & 506340.00 & 1.00 & 0.94 & 0.94 \\
17498 & 102314 & 2000 & 450.00 & -0.06 & 51443.00 & 429346.34 & 0.87 & 0.95 & 0.83 \\
17474 & 102312 & 2000 & 180.60 & -0.14 & 24780.00 & 163925.83 & 0.73 & 0.91 & 0.66 \\
26310 & 103567 & 2000 & 1451.30 & 0.00 & 207838.00 & 1979024.00 & 0.70 & 1.36 & 0.95 \\
5083 & 100723 & 2000 & 28.90 & 0.09 & 2894.00 & 27570.86 & 1.00 & 0.95 & 0.95 \\
38573 & 107290 & 2000 & 548.40 & -0.09 & 39568.00 & 405630.45 & 1.39 & 0.74 & 1.03 \\
38506 & 107266 & 2000 & 323.20 & 0.04 & 25202.00 & 281136.25 & 1.28 & 0.87 & 1.12 \\
41639 & 108805 & 2000 & 13.40 & 0.64 & 1002.00 & 12555.46 & 1.34 & 0.94 & 1.25 \\
893 & 100101 & 2000 & 1333.30 & -0.28 & 133322.00 & 1199609.61 & 1.00 & 0.90 & 0.90 \\
26375 & 103572 & 2000 & 43.40 & -0.01 & 4344.00 & 42346.19 & 1.00 & 0.98 & 0.97 \\
18775 & 102508 & 2000 & 418.90 & -0.19 & 41889.00 & 412720.69 & 1.00 & 0.99 & 0.99 \\
38536 & 107277 & 2000 & 15.50 & 0.29 & 1405.00 & 15935.59 & 1.10 & 1.03 & 1.13 \\
7703 & 101055 & 2000 & 30397.40 & -0.06 & 2916942.00 & 29558333.52 & 1.04 & 0.97 & 1.01 \\
13917 & 101787 & 2000 & 830.00 & -0.00 & 85324.00 & 804696.24 & 0.97 & 0.97 & 0.94 \\
58033 & 410055 & 2000 & 85.50 & -0.26 & 8549.00 & 85457.39 & 1.00 & 1.00 & 1.00 \\
26342 & 103570 & 2000 & 18.30 & -0.02 & 1402.00 & 14354.14 & 1.31 & 0.78 & 1.02 \\
38565 & 107282 & 2000 & 50.30 & -0.07 & 5077.00 & 43755.57 & 0.99 & 0.87 & 0.86 \\
2689 & 100352 & 2000 & 2264.40 & -0.11 & 251189.00 & 2211256.78 & 0.90 & 0.98 & 0.88 \\
38569 & 107285 & 2000 & 4.80 & 0.06 & 309.00 & 3027.26 & 1.55 & 0.63 & 0.98 \\
63498 & 500512 & 2000 & 736.80 & 0.08 & 86045.00 & 750450.40 & 0.86 & 1.02 & 0.87 \\
49493 & 240305 & 2000 & 71.30 & -0.30 & 7093.00 & 69466.80 & 1.01 & 0.97 & 0.98 \\
38540 & 107281 & 2000 & 19.70 & -0.05 & 1975.00 & 19061.08 & 1.00 & 0.97 & 0.97 \\
14228 & 101834 & 2000 & 231.50 & 0.08 & 23265.00 & 230102.01 & 1.00 & 0.99 & 0.99 \\
38947 & 107358 & 2000 & 8.00 & 0.31 & 831.00 & 8059.05 & 0.96 & 1.01 & 0.97 \\
45365 & 200050 & 2000 & 45.30 & 0.28 & 4572.00 & 42990.78 & 0.99 & 0.95 & 0.94 \\
4478 & 100634 & 2000 & 1919.20 & -0.08 & 188372.00 & 1883247.00 & 1.02 & 0.98 & 1.00 \\
39299 & 107650 & 2000 & 220.60 & -0.51 & 22254.00 & 208640.31 & 0.99 & 0.95 & 0.94 \\
48934 & 240154 & 2000 & 47.10 & 0.21 & 3919.00 & 40666.39 & 1.20 & 0.86 & 1.04 \\
25497 & 103494 & 2000 & 386.40 & -0.15 & 45855.00 & 398901.55 & 0.84 & 1.03 & 0.87 \\
45339 & 200047 & 2000 & 28.50 & -0.16 & 2814.00 & 25315.79 & 1.01 & 0.89 & 0.90 \\
18640 & 102493 & 2000 & 2104.40 & -0.05 & 210124.00 & 2046194.72 & 1.00 & 0.97 & 0.97 \\
1272 & 100171 & 2000 & 780.60 & 0.11 & 68381.00 & 670292.06 & 1.14 & 0.86 & 0.98 \\
39305 & 107652 & 2000 & 200.20 & -0.31 & 20221.00 & 182800.56 & 0.99 & 0.91 & 0.90 \\
39311 & 107653 & 2000 & 43.00 & -0.07 & 4285.00 & 40751.90 & 1.00 & 0.95 & 0.95 \\
17783 & 102357 & 2000 & 1782.80 & 0.01 & 178152.00 & 1770271.82 & 1.00 & 0.99 & 0.99 \\
12082 & 101497 & 2000 & 2197.10 & 0.04 & 219396.00 & 1795093.94 & 1.00 & 0.82 & 0.82 \\
22963 & 103099 & 2000 & 806.80 & -0.10 & 80838.00 & 796987.37 & 1.00 & 0.99 & 0.99 \\
48040 & 227155 & 2000 & 46.80 & 0.22 & 5039.00 & 49228.25 & 0.93 & 1.05 & 0.98 \\
39283 & 107648 & 2000 & 1657.30 & -0.26 & 165192.00 & 1651939.90 & 1.00 & 1.00 & 1.00 \\
1254 & 100167 & 2000 & 502.40 & -0.07 & 49929.00 & 484793.55 & 1.01 & 0.96 & 0.97 \\
12065 & 101494 & 2000 & 549.10 & -0.31 & 55007.00 & 544479.92 & 1.00 & 0.99 & 0.99 \\
22892 & 103084 & 2000 & 286.60 & 0.24 & 23673.00 & 252544.33 & 1.21 & 0.88 & 1.07 \\
45380 & 200055 & 2000 & 78.40 & -0.01 & 7843.00 & 74974.66 & 1.00 & 0.96 & 0.96 \\
25585 & 103497 & 2000 & 49.60 & 0.06 & 5236.00 & 44192.96 & 0.95 & 0.89 & 0.84 \\
39254 & 107627 & 2000 & 525.00 & 0.18 & 53265.00 & 436051.22 & 0.99 & 0.83 & 0.82 \\
45375 & 200051 & 2000 & 3.20 & -0.01 & 315.00 & 2829.33 & 1.02 & 0.88 & 0.90 \\
39271 & 107630 & 2000 & 67.10 & -0.21 & 9398.00 & 88581.70 & 0.71 & 1.32 & 0.94 \\
4939 & 100695 & 2000 & 189.20 & 0.06 & 18904.00 & 175466.33 & 1.00 & 0.93 & 0.93 \\
39273 & 107632 & 2000 & 64.30 & 0.03 & 6722.00 & 61689.09 & 0.96 & 0.96 & 0.92 \\
39275 & 107636 & 2000 & 62.50 & -0.57 & 9363.00 & 52266.64 & 0.67 & 0.84 & 0.56 \\
13955 & 101789 & 2000 & 672.80 & -0.11 & 69348.00 & 649107.13 & 0.97 & 0.96 & 0.94 \\
17752 & 102356 & 2000 & 4.10 & 0.07 & 324.00 & 3399.48 & 1.27 & 0.83 & 1.05 \\
25554 & 103496 & 2000 & 601.60 & -0.14 & 73045.00 & 631967.48 & 0.82 & 1.05 & 0.87 \\
41348 & 108732 & 2000 & 463.30 & -0.53 & 44267.00 & 362963.89 & 1.05 & 0.78 & 0.82 \\
39279 & 107641 & 2000 & 451.50 & -0.11 & 43487.00 & 434844.68 & 1.04 & 0.96 & 1.00 \\
41323 & 108728 & 2000 & 16.30 & 0.07 & 1621.00 & 16147.88 & 1.01 & 0.99 & 1.00 \\
2477 & 100333 & 2000 & 40.90 & -0.02 & 4052.00 & 40522.54 & 1.01 & 0.99 & 1.00 \\
39325 & 107654 & 2000 & 37.10 & 0.06 & 4233.00 & 39766.56 & 0.88 & 1.07 & 0.94 \\
41312 & 108726 & 2000 & 17.30 & 0.16 & 1724.00 & 17236.90 & 1.00 & 1.00 & 1.00 \\
25434 & 103487 & 2000 & 95.50 & 0.42 & 9719.00 & 92104.22 & 0.98 & 0.96 & 0.95 \\
25353 & 103478 & 2000 & 1730.70 & 0.07 & 172753.00 & 1672318.45 & 1.00 & 0.97 & 0.97 \\
12125 & 101511 & 2000 & 226.10 & -0.13 & 20776.00 & 205600.61 & 1.09 & 0.91 & 0.99 \\
1368 & 100192 & 2000 & 94.80 & 0.09 & 9281.00 & 92811.51 & 1.02 & 0.98 & 1.00 \\
39367 & 107673 & 2000 & 40.40 & 0.17 & 4028.00 & 36785.79 & 1.00 & 0.91 & 0.91 \\
2441 & 100330 & 2000 & 4257.10 & 0.00 & 481293.00 & 4748879.99 & 0.88 & 1.12 & 0.99 \\
39374 & 107677 & 2000 & 123.60 & 0.11 & 12147.00 & 114388.29 & 1.02 & 0.93 & 0.94 \\
23031 & 103103 & 2000 & 525.80 & -0.11 & 52579.00 & 524028.76 & 1.00 & 1.00 & 1.00 \\
25312 & 103466 & 2000 & 1119.60 & 0.03 & 110017.00 & 1066830.58 & 1.02 & 0.95 & 0.97 \\
7083 & 100996 & 2000 & 2776.50 & -0.10 & 281406.00 & 2748442.82 & 0.99 & 0.99 & 0.98 \\
17843 & 102365 & 2000 & 1082.50 & -0.12 & 108500.00 & 1071798.30 & 1.00 & 0.99 & 0.99 \\
39390 & 107680 & 2000 & 388.50 & -0.11 & 40890.00 & 362084.72 & 0.95 & 0.93 & 0.89 \\
12838 & 101602 & 2000 & 2089.10 & -0.02 & 211792.00 & 1955599.65 & 0.99 & 0.94 & 0.92 \\
39417 & 107692 & 2000 & 33.30 & 0.15 & 4009.00 & 31977.78 & 0.83 & 0.96 & 0.80 \\
41277 & 108719 & 2000 & 155.00 & -0.01 & 14371.00 & 132243.62 & 1.08 & 0.85 & 0.92 \\
39346 & 107671 & 2000 & 38.50 & 0.08 & 3840.00 & 36334.27 & 1.00 & 0.94 & 0.95 \\
7737 & 101056 & 2000 & 39963.10 & -0.02 & 4160627.00 & 38518216.96 & 0.96 & 0.96 & 0.93 \\
65133 & 500664 & 2000 & 1992.30 & 0.07 & 187090.00 & 1805777.84 & 1.06 & 0.91 & 0.97 \\
4909 & 100692 & 2000 & 2024.50 & -0.12 & 201754.00 & 1898421.55 & 1.00 & 0.94 & 0.94 \\
22982 & 103100 & 2000 & 557.20 & 0.04 & 55704.00 & 546576.11 & 1.00 & 0.98 & 0.98 \\
25410 & 103483 & 2000 & 864.60 & -0.08 & 86339.00 & 853124.47 & 1.00 & 0.99 & 0.99 \\
12102 & 101503 & 2000 & 361.20 & -0.14 & 36432.00 & 360545.07 & 0.99 & 1.00 & 0.99 \\
39237 & 107626 & 2000 & 376.30 & 0.04 & 42709.00 & 361450.96 & 0.88 & 0.96 & 0.85 \\
14164 & 101819 & 2000 & 227.00 & 0.22 & 21222.00 & 172380.77 & 1.07 & 0.76 & 0.81 \\
1349 & 100190 & 2000 & 3374.70 & -0.15 & 327435.00 & 3236831.04 & 1.03 & 0.96 & 0.99 \\
17812 & 102364 & 2000 & 1456.30 & -0.09 & 145601.00 & 1419917.16 & 1.00 & 0.98 & 0.98 \\
23000 & 103101 & 2000 & 215.30 & 0.25 & 21336.00 & 210139.14 & 1.01 & 0.98 & 0.98 \\
39334 & 107670 & 2000 & 185.10 & 0.18 & 20286.00 & 173901.86 & 0.91 & 0.94 & 0.86 \\
45334 & 200039 & 2000 & 107.00 & 0.39 & 10718.00 & 100828.40 & 1.00 & 0.94 & 0.94 \\
41302 & 108723 & 2000 & 85.50 & -0.10 & 8875.00 & 88758.29 & 0.96 & 1.04 & 1.00 \\
25380 & 103481 & 2000 & 96.10 & -0.01 & 9572.00 & 87529.62 & 1.00 & 0.91 & 0.91 \\
25281 & 103464 & 2000 & 1473.40 & -0.19 & 187191.00 & 1498457.09 & 0.79 & 1.02 & 0.80 \\
2506 & 100336 & 2000 & 27.40 & 0.11 & 2800.00 & 28775.64 & 0.98 & 1.05 & 1.03 \\
58072 & 410075 & 2000 & 180.10 & 0.84 & 17582.00 & 173982.90 & 1.02 & 0.97 & 0.99 \\
18695 & 102503 & 2000 & 360.80 & 0.02 & 38588.00 & 354632.68 & 0.94 & 0.98 & 0.92 \\
2540 & 100343 & 2000 & 402.80 & 0.06 & 40293.00 & 389188.72 & 1.00 & 0.97 & 0.97 \\
39022 & 107566 & 2000 & 2.50 & 0.18 & 246.00 & 2310.59 & 1.02 & 0.92 & 0.94 \\
22794 & 103065 & 2000 & 473.60 & -0.04 & 48146.00 & 455972.82 & 0.98 & 0.96 & 0.95 \\
17656 & 102334 & 2000 & 289.10 & -0.09 & 29141.00 & 269294.81 & 0.99 & 0.93 & 0.92 \\
39028 & 107573 & 2000 & 446.30 & -0.06 & 44839.00 & 420819.65 & 1.00 & 0.94 & 0.94 \\
39053 & 107579 & 2000 & 35.20 & -0.02 & 5547.00 & 50477.44 & 0.63 & 1.43 & 0.91 \\
59033 & 410401 & 2000 & 110.90 & 0.39 & 11022.00 & 99889.39 & 1.01 & 0.90 & 0.91 \\
25793 & 103523 & 2000 & 4834.90 & -0.15 & 644800.00 & 4797678.66 & 0.75 & 0.99 & 0.74 \\
8209 & 101079 & 2000 & 272.90 & 0.32 & 24677.00 & 231210.46 & 1.11 & 0.85 & 0.94 \\
41423 & 108752 & 2000 & 14.00 & 0.06 & 1397.00 & 13856.00 & 1.00 & 0.99 & 0.99 \\
1103 & 100153 & 2000 & 195.40 & 0.03 & 19244.00 & 204071.48 & 1.02 & 1.04 & 1.06 \\
39055 & 107580 & 2000 & 37.20 & -0.12 & 4635.00 & 45466.70 & 0.80 & 1.22 & 0.98 \\
39057 & 107598 & 2000 & 26.20 & 0.06 & 2213.00 & 19999.79 & 1.18 & 0.76 & 0.90 \\
48010 & 226438 & 2000 & 786.20 & -0.20 & 78570.00 & 773323.71 & 1.00 & 0.98 & 0.98 \\
4965 & 100697 & 2000 & 40.30 & 0.27 & 3971.00 & 37706.98 & 1.01 & 0.94 & 0.95 \\
25827 & 103524 & 2000 & 88240.50 & -0.09 & 10552674.00 & 93718024.61 & 0.84 & 1.06 & 0.89 \\
39016 & 107564 & 2000 & 112.80 & 0.11 & 11266.00 & 106713.43 & 1.00 & 0.95 & 0.95 \\
17620 & 102321 & 2000 & 191.40 & -0.09 & 17998.00 & 179983.37 & 1.06 & 0.94 & 1.00 \\
25894 & 103526 & 2000 & 4661.00 & -0.13 & 585087.00 & 4523766.22 & 0.80 & 0.97 & 0.77 \\
38972 & 107361 & 2000 & 45.10 & 0.26 & 4499.00 & 43347.33 & 1.00 & 0.96 & 0.96 \\
41459 & 108760 & 2000 & 103.20 & -0.06 & 10628.00 & 106252.91 & 0.97 & 1.03 & 1.00 \\
45475 & 200065 & 2000 & 4.90 & 0.09 & 494.00 & 4870.84 & 0.99 & 0.99 & 0.99 \\
38975 & 107362 & 2000 & 94.70 & 0.07 & 9468.00 & 90421.87 & 1.00 & 0.95 & 0.96 \\
41435 & 108759 & 2000 & 26.60 & 0.12 & 2545.00 & 25457.48 & 1.05 & 0.96 & 1.00 \\
49422 & 240296 & 2000 & 693.80 & 0.19 & 71587.00 & 691022.35 & 0.97 & 1.00 & 0.97 \\
45469 & 200061 & 2000 & 66.70 & 0.10 & 6668.00 & 62923.12 & 1.00 & 0.94 & 0.94 \\
48001 & 225696 & 2000 & 45.30 & -0.09 & 4508.00 & 41187.23 & 1.00 & 0.91 & 0.91 \\
25860 & 103525 & 2000 & 37170.70 & -0.09 & 3699089.00 & 33389795.58 & 1.00 & 0.90 & 0.90 \\
38985 & 107470 & 2000 & 7.40 & -0.15 & 741.00 & 7127.28 & 1.00 & 0.96 & 0.96 \\
45443 & 200060 & 2000 & 830.20 & -0.16 & 83533.00 & 821589.29 & 0.99 & 0.99 & 0.98 \\
38991 & 107563 & 2000 & 946.90 & -0.10 & 95507.00 & 836195.49 & 0.99 & 0.88 & 0.88 \\
49414 & 240295 & 2000 & 1434.70 & -0.06 & 160705.00 & 1421245.39 & 0.89 & 0.99 & 0.88 \\
39018 & 107565 & 2000 & 1.90 & -0.09 & 189.00 & 1895.27 & 1.01 & 1.00 & 1.00 \\
41398 & 108749 & 2000 & 1.60 & 0.18 & 148.00 & 1563.61 & 1.08 & 0.98 & 1.06 \\
48017 & 226504 & 2000 & 22.50 & -0.21 & 3519.00 & 22510.37 & 0.64 & 1.00 & 0.64 \\
17708 & 102349 & 2000 & 1137.40 & -0.07 & 113709.00 & 1089438.39 & 1.00 & 0.96 & 0.96 \\
1153 & 100157 & 2000 & 1347.70 & -0.10 & 134887.00 & 1309450.37 & 1.00 & 0.97 & 0.97 \\
41382 & 108742 & 2000 & 31.80 & -0.08 & 3191.00 & 30182.69 & 1.00 & 0.95 & 0.95 \\
48180 & 240040 & 2000 & 389.50 & 0.26 & 32044.00 & 364294.71 & 1.22 & 0.94 & 1.14 \\
39170 & 107616 & 2000 & 54.20 & 0.11 & 5645.00 & 54228.72 & 0.96 & 1.00 & 0.96 \\
39195 & 107618 & 2000 & 3373.50 & -0.03 & 361254.00 & 3183745.60 & 0.93 & 0.94 & 0.88 \\
4453 & 100633 & 2000 & 895.10 & -0.06 & 88002.00 & 879818.21 & 1.02 & 0.98 & 1.00 \\
22861 & 103073 & 2000 & 812.90 & 0.20 & 76191.00 & 725579.69 & 1.07 & 0.89 & 0.95 \\
25629 & 103498 & 2000 & 388.30 & -0.07 & 45978.00 & 356659.08 & 0.84 & 0.92 & 0.78 \\
41371 & 108736 & 2000 & 34.10 & -0.06 & 3410.00 & 31427.01 & 1.00 & 0.92 & 0.92 \\
1185 & 100159 & 2000 & 145.90 & -0.10 & 14452.00 & 144108.78 & 1.01 & 0.99 & 1.00 \\
12882 & 101603 & 2000 & 1881.00 & -0.06 & 192990.00 & 1848982.59 & 0.97 & 0.98 & 0.96 \\
1223 & 100166 & 2000 & 7654.90 & 0.07 & 580574.00 & 5805596.67 & 1.32 & 0.76 & 1.00 \\
17731 & 102350 & 2000 & 535.10 & -0.04 & 53626.00 & 536255.18 & 1.00 & 1.00 & 1.00 \\
41389 & 108745 & 2000 & 3.50 & 0.26 & 515.00 & 5146.10 & 0.68 & 1.47 & 1.00 \\
14197 & 101820 & 2000 & 379.30 & -0.18 & 45869.00 & 451294.41 & 0.83 & 1.19 & 0.98 \\
25695 & 103514 & 2000 & 2561.40 & 0.14 & 243826.00 & 2362584.75 & 1.05 & 0.92 & 0.97 \\
39145 & 107611 & 2000 & 744.90 & 0.07 & 55560.00 & 530382.93 & 1.34 & 0.71 & 0.95 \\
48928 & 240153 & 2000 & 142.00 & 0.06 & 12970.00 & 127977.48 & 1.09 & 0.90 & 0.99 \\
25759 & 103521 & 2000 & 4531.70 & 0.06 & 404426.00 & 4242255.36 & 1.12 & 0.94 & 1.05 \\
7201 & 101013 & 2000 & 16133.80 & -0.18 & 1655306.00 & 15152568.84 & 0.97 & 0.94 & 0.92 \\
49406 & 240293 & 2000 & 1414.20 & -0.05 & 160405.00 & 1371663.19 & 0.88 & 0.97 & 0.86 \\
39119 & 107607 & 2000 & 45.40 & 0.07 & 4116.00 & 41844.82 & 1.10 & 0.92 & 1.02 \\
13936 & 101788 & 2000 & 899.60 & -0.09 & 91288.00 & 886407.83 & 0.99 & 0.99 & 0.97 \\
39228 & 107623 & 2000 & 7.80 & -0.23 & 782.00 & 7479.02 & 1.00 & 0.96 & 0.96 \\
39129 & 107608 & 2000 & 39.20 & 0.16 & 3532.00 & 35066.57 & 1.11 & 0.89 & 0.99 \\
45417 & 200058 & 2000 & 2325.20 & -0.18 & 232680.00 & 2200772.04 & 1.00 & 0.95 & 0.95 \\
4438 & 100625 & 2000 & 1279.10 & 0.16 & 127720.00 & 1189268.97 & 1.00 & 0.93 & 0.93 \\
22829 & 103067 & 2000 & 149.30 & -0.14 & 14794.00 & 144524.22 & 1.01 & 0.97 & 0.98 \\
1138 & 100155 & 2000 & 2086.50 & -0.08 & 210226.00 & 2075451.16 & 0.99 & 0.99 & 0.99 \\
36679 & 106574 & 2000 & 20.60 & 0.06 & 2161.00 & 19097.31 & 0.95 & 0.93 & 0.88 \\
40662 & 108144 & 2000 & 550.40 & 0.28 & 74963.00 & 620298.01 & 0.73 & 1.13 & 0.83 \\
30799 & 105803 & 2000 & 2330.90 & 0.31 & 213985.00 & 1844129.72 & 1.09 & 0.79 & 0.86 \\
30678 & 105782 & 2000 & 45.40 & 0.16 & 4755.00 & 48542.19 & 0.95 & 1.07 & 1.02 \\
33479 & 106140 & 2000 & 1561.20 & -0.01 & 156128.00 & 1506566.26 & 1.00 & 0.97 & 0.96 \\
13617 & 101748 & 2000 & 40.10 & 0.20 & 3871.00 & 36454.14 & 1.04 & 0.91 & 0.94 \\
34792 & 106277 & 2000 & 78.70 & 0.43 & 8016.00 & 71138.36 & 0.98 & 0.90 & 0.89 \\
6084 & 100822 & 2000 & 14.30 & 0.01 & 1438.00 & 14380.24 & 0.99 & 1.01 & 1.00 \\
10293 & 101278 & 2000 & 71.80 & -0.11 & 4848.00 & 48496.66 & 1.48 & 0.68 & 1.00 \\
48602 & 240113 & 2000 & 64.60 & 0.02 & 6231.00 & 61952.92 & 1.04 & 0.96 & 0.99 \\
43933 & 109233 & 2000 & 146.90 & -0.07 & 16066.00 & 144682.26 & 0.91 & 0.98 & 0.90 \\
30643 & 105780 & 2000 & 341.10 & -0.01 & 34402.00 & 337055.59 & 0.99 & 0.99 & 0.98 \\
52479 & 302964 & 2000 & 18.70 & 0.31 & 2210.00 & 21903.37 & 0.85 & 1.17 & 0.99 \\
20807 & 102795 & 2000 & 252.30 & 0.16 & 27156.00 & 253121.33 & 0.93 & 1.00 & 0.93 \\
14805 & 101914 & 2000 & 42.20 & 0.07 & 4239.00 & 40448.28 & 1.00 & 0.96 & 0.95 \\
15914 & 102059 & 2000 & 621.70 & 0.13 & 62167.00 & 583507.44 & 1.00 & 0.94 & 0.94 \\
30652 & 105781 & 2000 & 213.70 & 0.18 & 18245.00 & 184167.87 & 1.17 & 0.86 & 1.01 \\
34815 & 106278 & 2000 & 125.50 & 0.03 & 12535.00 & 125102.28 & 1.00 & 1.00 & 1.00 \\
30683 & 105783 & 2000 & 1094.10 & 0.26 & 106560.00 & 1017085.00 & 1.03 & 0.93 & 0.95 \\
34676 & 106270 & 2000 & 29.20 & 0.24 & 3578.00 & 26848.16 & 0.82 & 0.92 & 0.75 \\
3496 & 100441 & 2000 & 626.20 & -0.09 & 64892.00 & 608950.17 & 0.96 & 0.97 & 0.94 \\
14953 & 101925 & 2000 & 715.00 & 0.53 & 68608.00 & 686115.66 & 1.04 & 0.96 & 1.00 \\
54862 & 400019 & 2000 & 334.50 & 0.11 & 32215.00 & 330909.14 & 1.04 & 0.99 & 1.03 \\
30731 & 105791 & 2000 & 2.20 & 0.14 & 164.00 & 1566.20 & 1.34 & 0.71 & 0.96 \\
42899 & 109031 & 2000 & 13.00 & -0.08 & 1132.00 & 9959.84 & 1.15 & 0.77 & 0.88 \\
34709 & 106272 & 2000 & 1725.00 & -0.10 & 169068.00 & 1507751.50 & 1.02 & 0.87 & 0.89 \\
47252 & 200344 & 2000 & 3137.10 & 0.31 & 289711.00 & 2897254.17 & 1.08 & 0.92 & 1.00 \\
30710 & 105788 & 2000 & 10.70 & -0.04 & 1040.00 & 10248.04 & 1.03 & 0.96 & 0.99 \\
34738 & 106275 & 2000 & 11.00 & -0.24 & 1110.00 & 10567.31 & 0.99 & 0.96 & 0.95 \\
34765 & 106276 & 2000 & 36.20 & 0.14 & 3616.00 & 33388.69 & 1.00 & 0.92 & 0.92 \\
32210 & 106000 & 2000 & 519.00 & 0.06 & 47500.00 & 474402.63 & 1.09 & 0.91 & 1.00 \\
33446 & 106136 & 2000 & 115.90 & -0.20 & 11876.00 & 103322.23 & 0.98 & 0.89 & 0.87 \\
34827 & 106281 & 2000 & 24.20 & 0.17 & 2423.00 & 19577.47 & 1.00 & 0.81 & 0.81 \\
15301 & 101982 & 2000 & 437.10 & -0.09 & 45088.00 & 419183.94 & 0.97 & 0.96 & 0.93 \\
30553 & 105769 & 2000 & 94.90 & -0.27 & 8771.00 & 82397.79 & 1.08 & 0.87 & 0.94 \\
43309 & 109093 & 2000 & 22.40 & -0.02 & 2116.00 & 18096.83 & 1.06 & 0.81 & 0.86 \\
6748 & 100947 & 2000 & 1379.10 & -0.14 & 137851.00 & 1329854.52 & 1.00 & 0.96 & 0.96 \\
34926 & 106293 & 2000 & 68.60 & -0.20 & 9421.00 & 87178.42 & 0.73 & 1.27 & 0.93 \\
15944 & 102061 & 2000 & 285.80 & 0.37 & 22545.00 & 225461.43 & 1.27 & 0.79 & 1.00 \\
34933 & 106294 & 2000 & 71.50 & 0.32 & 7137.00 & 69852.34 & 1.00 & 0.98 & 0.98 \\
33419 & 106135 & 2000 & 46.60 & 0.19 & 4473.00 & 44729.47 & 1.04 & 0.96 & 1.00 \\
48292 & 240060 & 2000 & 331.70 & -0.24 & 32941.00 & 269435.37 & 1.01 & 0.81 & 0.82 \\
30525 & 105763 & 2000 & 238.20 & 0.12 & 21269.00 & 228920.56 & 1.12 & 0.96 & 1.08 \\
96663 & 611002 & 2000 & 5258.20 & -0.06 & 591128.00 & 5232613.23 & 0.89 & 1.00 & 0.89 \\
9612 & 101158 & 2000 & 582.50 & -0.22 & 58394.00 & 572113.13 & 1.00 & 0.98 & 0.98 \\
13703 & 101758 & 2000 & 381.30 & -0.10 & 66899.00 & 669084.60 & 0.57 & 1.75 & 1.00 \\
48609 & 240114 & 2000 & 506.00 & -0.06 & 49496.00 & 494974.76 & 1.02 & 0.98 & 1.00 \\
30563 & 105770 & 2000 & 24.80 & 0.04 & 2495.00 & 20748.92 & 0.99 & 0.84 & 0.83 \\
3462 & 100439 & 2000 & 29.00 & -0.21 & 3167.00 & 31406.61 & 0.92 & 1.08 & 0.99 \\
34900 & 106292 & 2000 & 12.30 & 0.15 & 1288.00 & 12006.85 & 0.95 & 0.98 & 0.93 \\
42865 & 109028 & 2000 & 197.20 & 0.09 & 20551.00 & 198120.05 & 0.96 & 1.00 & 0.96 \\
30616 & 105779 & 2000 & 2198.00 & -0.03 & 204195.00 & 1872291.83 & 1.08 & 0.85 & 0.92 \\
52852 & 330794 & 2000 & 10.40 & 0.01 & 981.00 & 9806.31 & 1.06 & 0.94 & 1.00 \\
34838 & 106282 & 2000 & 468.60 & 0.03 & 46987.00 & 443750.17 & 1.00 & 0.95 & 0.94 \\
42863 & 109026 & 2000 & 25.30 & -0.20 & 2461.00 & 24606.04 & 1.03 & 0.97 & 1.00 \\
34865 & 106283 & 2000 & 203.60 & 0.16 & 21488.00 & 195583.52 & 0.95 & 0.96 & 0.91 \\
52241 & 302698 & 2000 & 389.20 & -0.20 & 38257.00 & 373616.04 & 1.02 & 0.96 & 0.98 \\
19401 & 102600 & 2000 & 846.50 & 0.13 & 84645.00 & 802545.06 & 1.00 & 0.95 & 0.95 \\
30592 & 105775 & 2000 & 2956.70 & 0.30 & 295463.00 & 2625769.58 & 1.00 & 0.89 & 0.89 \\
32238 & 106007 & 2000 & 2793.70 & 0.08 & 285164.00 & 2640172.05 & 0.98 & 0.95 & 0.93 \\
10322 & 101279 & 2000 & 40.10 & 0.10 & 3710.00 & 36674.79 & 1.08 & 0.91 & 0.99 \\
34885 & 106284 & 2000 & 389.00 & -0.16 & 46776.00 & 342454.80 & 0.83 & 0.88 & 0.73 \\
6409 & 100864 & 2000 & 574.10 & 0.06 & 57089.00 & 486908.31 & 1.01 & 0.85 & 0.85 \\
143 & 100010 & 2000 & 638.20 & 0.09 & 63741.00 & 603598.32 & 1.00 & 0.95 & 0.95 \\
4130 & 100559 & 2000 & 67.80 & -0.20 & 6782.00 & 66407.14 & 1.00 & 0.98 & 0.98 \\
30752 & 105793 & 2000 & 6165.00 & -0.22 & 764581.00 & 5110606.06 & 0.81 & 0.83 & 0.67 \\
9631 & 101160 & 2000 & 335.80 & 0.21 & 33615.00 & 318177.83 & 1.00 & 0.95 & 0.95 \\
42964 & 109044 & 2000 & 78.60 & 0.14 & 6601.00 & 66811.67 & 1.19 & 0.85 & 1.01 \\
15337 & 101987 & 2000 & 2731.90 & -0.07 & 269412.00 & 2624390.83 & 1.01 & 0.96 & 0.97 \\
3816 & 100485 & 2000 & 506.40 & 0.17 & 54760.00 & 501562.16 & 0.92 & 0.99 & 0.92 \\
33551 & 106148 & 2000 & 143.50 & 0.24 & 13916.00 & 132186.51 & 1.03 & 0.92 & 0.95 \\
30916 & 105836 & 2000 & 685.20 & -0.41 & 109046.00 & 664091.54 & 0.63 & 0.97 & 0.61 \\
119 & 100009 & 2000 & 347.40 & -0.11 & 35635.00 & 317965.81 & 0.97 & 0.92 & 0.89 \\
42955 & 109042 & 2000 & 91.70 & 0.28 & 6005.00 & 58681.70 & 1.53 & 0.64 & 0.98 \\
52189 & 302676 & 2000 & 3.00 & 0.01 & 237.00 & 2055.94 & 1.27 & 0.69 & 0.87 \\
30900 & 105807 & 2000 & 30.00 & -0.02 & 3168.00 & 28900.75 & 0.95 & 0.96 & 0.91 \\
34542 & 106251 & 2000 & 50.90 & -0.08 & 5198.00 & 49769.43 & 0.98 & 0.98 & 0.96 \\
7430 & 101039 & 2000 & 5411.40 & -0.14 & 632511.00 & 5183205.75 & 0.86 & 0.96 & 0.82 \\
20710 & 102784 & 2000 & 35816.10 & -0.17 & 4024138.00 & 29572861.13 & 0.89 & 0.83 & 0.73 \\
6115 & 100823 & 2000 & 24.10 & 0.13 & 2340.00 & 23398.41 & 1.03 & 0.97 & 1.00 \\
8968 & 101107 & 2000 & 1242.10 & -0.07 & 153764.00 & 1334532.07 & 0.81 & 1.07 & 0.87 \\
32151 & 105990 & 2000 & 249.40 & 0.04 & 35534.00 & 340235.54 & 0.70 & 1.36 & 0.96 \\
34533 & 106250 & 2000 & 68.80 & -0.01 & 6880.00 & 61002.35 & 1.00 & 0.89 & 0.89 \\
48435 & 240083 & 2000 & 191.50 & 0.04 & 17436.00 & 173507.85 & 1.10 & 0.91 & 1.00 \\
30960 & 105842 & 2000 & 567.10 & -0.07 & 57407.00 & 547525.30 & 0.99 & 0.97 & 0.95 \\
9662 & 101161 & 2000 & 1364.70 & -0.00 & 139138.00 & 1314756.57 & 0.98 & 0.96 & 0.94 \\
30934 & 105838 & 2000 & 15.60 & 0.17 & 1562.00 & 14904.76 & 1.00 & 0.96 & 0.95 \\
13458 & 101740 & 2000 & 39174.00 & -0.15 & 4648335.00 & 35951059.30 & 0.84 & 0.92 & 0.77 \\
32145 & 105987 & 2000 & 209.50 & -0.13 & 28113.00 & 250966.19 & 0.75 & 1.20 & 0.89 \\
30946 & 105840 & 2000 & 30.70 & 0.06 & 3088.00 & 29810.93 & 0.99 & 0.97 & 0.97 \\
20671 & 102783 & 2000 & 1519.40 & 0.17 & 139083.00 & 1338022.45 & 1.09 & 0.88 & 0.96 \\
30940 & 105839 & 2000 & 69.20 & -0.18 & 6922.00 & 67685.77 & 1.00 & 0.98 & 0.98 \\
19636 & 102639 & 2000 & 98.90 & 0.25 & 9498.00 & 92005.96 & 1.04 & 0.93 & 0.97 \\
30952 & 105841 & 2000 & 21.50 & -0.13 & 2170.00 & 21000.59 & 0.99 & 0.98 & 0.97 \\
43262 & 109088 & 2000 & 2940.00 & -0.01 & 314215.00 & 2839249.22 & 0.94 & 0.97 & 0.90 \\
4111 & 100552 & 2000 & 207.80 & -0.17 & 20580.00 & 205150.56 & 1.01 & 0.99 & 1.00 \\
43302 & 109092 & 2000 & 40.30 & 0.09 & 2946.00 & 32543.55 & 1.37 & 0.81 & 1.10 \\
34631 & 106262 & 2000 & 574.40 & -0.02 & 57455.00 & 530559.41 & 1.00 & 0.92 & 0.92 \\
33532 & 106144 & 2000 & 274.00 & -0.51 & 29129.00 & 230364.31 & 0.94 & 0.84 & 0.79 \\
34620 & 106261 & 2000 & 420.20 & -0.02 & 42052.00 & 388884.47 & 1.00 & 0.93 & 0.92 \\
33505 & 106143 & 2000 & 225.60 & -0.13 & 23096.00 & 213944.11 & 0.98 & 0.95 & 0.93 \\
42931 & 109037 & 2000 & 275.50 & 0.00 & 27554.00 & 264981.14 & 1.00 & 0.96 & 0.96 \\
30784 & 105798 & 2000 & 2086.70 & -0.15 & 209541.00 & 2095387.75 & 1.00 & 1.00 & 1.00 \\
34654 & 106267 & 2000 & 9.40 & 0.01 & 879.00 & 9102.86 & 1.07 & 0.97 & 1.04 \\
10258 & 101276 & 2000 & 384.10 & -0.03 & 37535.00 & 354801.08 & 1.02 & 0.92 & 0.95 \\
3525 & 100453 & 2000 & 202.00 & 0.02 & 20437.00 & 177319.14 & 0.99 & 0.88 & 0.87 \\
34660 & 106268 & 2000 & 153.00 & 0.10 & 14702.00 & 131121.03 & 1.04 & 0.86 & 0.89 \\
46276 & 200207 & 2000 & 15.40 & 0.06 & 1491.00 & 14948.14 & 1.03 & 0.97 & 1.00 \\
30827 & 105804 & 2000 & 634.90 & -0.15 & 61687.00 & 564446.88 & 1.03 & 0.89 & 0.92 \\
32182 & 105999 & 2000 & 7.00 & 0.36 & 701.00 & 6739.16 & 1.00 & 0.96 & 0.96 \\
20729 & 102788 & 2000 & 185.60 & 0.15 & 19849.00 & 207546.88 & 0.94 & 1.12 & 1.05 \\
33537 & 106147 & 2000 & 85.90 & 0.12 & 8565.00 & 80437.16 & 1.00 & 0.94 & 0.94 \\
10211 & 101274 & 2000 & 275.50 & 0.06 & 29584.00 & 258002.09 & 0.93 & 0.94 & 0.87 \\
15327 & 101984 & 2000 & 272.60 & -0.11 & 28265.00 & 261893.51 & 0.96 & 0.96 & 0.93 \\
34556 & 106255 & 2000 & 966.20 & -0.23 & 117984.00 & 979377.68 & 0.82 & 1.01 & 0.83 \\
34582 & 106256 & 2000 & 93.50 & 0.04 & 9350.00 & 91541.56 & 1.00 & 0.98 & 0.98 \\
32178 & 105997 & 2000 & 26.70 & 0.26 & 2910.00 & 28718.81 & 0.92 & 1.08 & 0.99 \\
19435 & 102601 & 2000 & 7600.30 & -0.00 & 760027.00 & 7319518.99 & 1.00 & 0.96 & 0.96 \\
30855 & 105805 & 2000 & 20.50 & -0.08 & 2284.00 & 19563.91 & 0.90 & 0.95 & 0.86 \\
34589 & 106257 & 2000 & 247.30 & 0.22 & 25040.00 & 209708.12 & 0.99 & 0.85 & 0.84 \\
34616 & 106258 & 2000 & 55.70 & -0.14 & 5668.00 & 54262.99 & 0.98 & 0.97 & 0.96 \\
10230 & 101275 & 2000 & 1187.70 & -0.04 & 105552.00 & 970649.49 & 1.13 & 0.82 & 0.92 \\
30762 & 105794 & 2000 & 24.70 & 0.11 & 2480.00 & 22661.85 & 1.00 & 0.92 & 0.91 \\
10353 & 101283 & 2000 & 1112.80 & 0.20 & 111282.00 & 987041.00 & 1.00 & 0.89 & 0.89 \\
33310 & 106114 & 2000 & 71.00 & 0.03 & 7092.00 & 65815.04 & 1.00 & 0.93 & 0.93 \\
6022 & 100820 & 2000 & 624.00 & -0.15 & 62263.00 & 622639.72 & 1.00 & 1.00 & 1.00 \\
35247 & 106336 & 2000 & 19.90 & 0.18 & 3374.00 & 32896.20 & 0.59 & 1.65 & 0.97 \\
42747 & 109015 & 2000 & 119.50 & -0.08 & 13368.00 & 116448.23 & 0.89 & 0.97 & 0.87 \\
52295 & 302760 & 2000 & 62.90 & 0.29 & 6287.00 & 56441.62 & 1.00 & 0.90 & 0.90 \\
42746 & 109014 & 2000 & 9.90 & -0.24 & 641.00 & 5475.34 & 1.54 & 0.55 & 0.85 \\
30146 & 105703 & 2000 & 94.90 & 0.12 & 9485.00 & 91470.74 & 1.00 & 0.96 & 0.96 \\
6012 & 100818 & 2000 & 110.60 & -0.23 & 11068.00 & 92958.91 & 1.00 & 0.84 & 0.84 \\
30138 & 105702 & 2000 & 87.20 & 0.03 & 9309.00 & 74654.05 & 0.94 & 0.86 & 0.80 \\
20994 & 102818 & 2000 & 58.70 & -0.12 & 5571.00 & 55711.63 & 1.05 & 0.95 & 1.00 \\
30125 & 105701 & 2000 & 1145.30 & -0.10 & 114963.00 & 1137379.13 & 1.00 & 0.99 & 0.99 \\
9799 & 101193 & 2000 & 578.80 & -0.04 & 59731.00 & 519175.46 & 0.97 & 0.90 & 0.87 \\
30115 & 105700 & 2000 & 239.50 & 0.14 & 23947.00 & 229763.89 & 1.00 & 0.96 & 0.96 \\
19670 & 102645 & 2000 & 397.40 & 0.11 & 41113.00 & 351052.81 & 0.97 & 0.88 & 0.85 \\
15252 & 101972 & 2000 & 2125.30 & -0.03 & 212530.00 & 1937293.58 & 1.00 & 0.91 & 0.91 \\
30226 & 105718 & 2000 & 204.30 & -0.23 & 20878.00 & 164994.89 & 0.98 & 0.81 & 0.79 \\
10478 & 101287 & 2000 & 444.00 & 0.18 & 44394.00 & 367952.60 & 1.00 & 0.83 & 0.83 \\
32360 & 106014 & 2000 & 116.80 & 0.03 & 11696.00 & 101671.10 & 1.00 & 0.87 & 0.87 \\
42762 & 109016 & 2000 & 14.90 & 0.10 & 1323.00 & 13391.45 & 1.13 & 0.90 & 1.01 \\
30205 & 105708 & 2000 & 442.90 & -0.39 & 82505.00 & 425066.57 & 0.54 & 0.96 & 0.52 \\
35212 & 106334 & 2000 & 264.50 & 0.13 & 28266.00 & 291560.05 & 0.94 & 1.10 & 1.03 \\
20978 & 102814 & 2000 & 127.90 & 0.28 & 12870.00 & 121565.92 & 0.99 & 0.95 & 0.94 \\
177 & 100017 & 2000 & 174.70 & 0.02 & 17558.00 & 171661.76 & 0.99 & 0.98 & 0.98 \\
3396 & 100431 & 2000 & 306.90 & 0.10 & 29175.00 & 271665.04 & 1.05 & 0.89 & 0.93 \\
20015 & 102663 & 2000 & 10550.00 & -0.34 & 976518.00 & 9201417.00 & 1.08 & 0.87 & 0.94 \\
30107 & 105694 & 2000 & 24.80 & -0.08 & 2756.00 & 22832.86 & 0.90 & 0.92 & 0.83 \\
19333 & 102597 & 2000 & 867.10 & -0.13 & 86457.00 & 848603.56 & 1.00 & 0.98 & 0.98 \\
52321 & 302763 & 2000 & 38.20 & 0.09 & 3810.00 & 36485.09 & 1.00 & 0.96 & 0.96 \\
6637 & 100906 & 2000 & 2128.60 & 0.12 & 219364.00 & 2117883.56 & 0.97 & 0.99 & 0.97 \\
14985 & 101926 & 2000 & 485.70 & -0.03 & 50021.00 & 499805.76 & 0.97 & 1.03 & 1.00 \\
30046 & 105679 & 2000 & 503.00 & -0.17 & 66577.00 & 488681.72 & 0.76 & 0.97 & 0.73 \\
35274 & 106341 & 2000 & 52.20 & -0.14 & 5350.00 & 51872.95 & 0.98 & 0.99 & 0.97 \\
6456 & 100875 & 2000 & 157.60 & -0.09 & 15933.00 & 159334.07 & 0.99 & 1.01 & 1.00 \\
35320 & 106347 & 2000 & 174.60 & -0.19 & 26574.00 & 172276.71 & 0.66 & 0.99 & 0.65 \\
4144 & 100561 & 2000 & 65.90 & 0.14 & 7282.00 & 71846.94 & 0.90 & 1.09 & 0.99 \\
30018 & 105678 & 2000 & 9.90 & 0.23 & 923.00 & 8843.80 & 1.07 & 0.89 & 0.96 \\
35305 & 106345 & 2000 & 116.40 & 0.46 & 11253.00 & 112567.50 & 1.03 & 0.97 & 1.00 \\
30074 & 105680 & 2000 & 31.70 & 0.19 & 3250.00 & 30516.64 & 0.98 & 0.96 & 0.94 \\
15236 & 101970 & 2000 & 57.40 & 0.10 & 5737.00 & 56823.47 & 1.00 & 0.99 & 0.99 \\
21005 & 102821 & 2000 & 191.90 & -0.05 & 18409.00 & 184072.75 & 1.04 & 0.96 & 1.00 \\
30100 & 105686 & 2000 & 123.50 & -0.21 & 12315.00 & 110476.71 & 1.00 & 0.89 & 0.90 \\
195 & 100018 & 2000 & 298.90 & -0.18 & 29780.00 & 284661.17 & 1.00 & 0.95 & 0.96 \\
14773 & 101913 & 2000 & 34.50 & -0.16 & 3497.00 & 33766.72 & 0.99 & 0.98 & 0.97 \\
16077 & 102079 & 2000 & 460.20 & -0.03 & 45738.00 & 384283.35 & 1.01 & 0.84 & 0.84 \\
33283 & 106113 & 2000 & 470.90 & -0.10 & 48606.00 & 454485.37 & 0.97 & 0.97 & 0.94 \\
13382 & 101736 & 2000 & 70.40 & -0.02 & 7189.00 & 68616.89 & 0.98 & 0.97 & 0.95 \\
30091 & 105684 & 2000 & 36.70 & 0.46 & 3752.00 & 35375.92 & 0.98 & 0.96 & 0.94 \\
30084 & 105682 & 2000 & 308.50 & -0.05 & 30862.00 & 293784.15 & 1.00 & 0.95 & 0.95 \\
5983 & 100817 & 2000 & 305.70 & -0.10 & 30551.00 & 303717.80 & 1.00 & 0.99 & 0.99 \\
6759 & 100950 & 2000 & 237.00 & 0.19 & 23924.00 & 229141.42 & 0.99 & 0.97 & 0.96 \\
33278 & 106110 & 2000 & 233.40 & -0.08 & 23343.00 & 233010.25 & 1.00 & 1.00 & 1.00 \\
47622 & 215696 & 2000 & 227.10 & -0.16 & 29022.00 & 230408.36 & 0.78 & 1.01 & 0.79 \\
5975 & 100815 & 2000 & 434.40 & 0.07 & 41564.00 & 407673.80 & 1.05 & 0.94 & 0.98 \\
33337 & 106116 & 2000 & 7.00 & 0.00 & 416.00 & 4161.14 & 1.68 & 0.59 & 1.00 \\
30235 & 105720 & 2000 & 3664.50 & -0.54 & 365174.00 & 3436333.03 & 1.00 & 0.94 & 0.94 \\
20948 & 102813 & 2000 & 599.40 & 0.22 & 59868.00 & 531969.74 & 1.00 & 0.89 & 0.89 \\
35169 & 106330 & 2000 & 95.60 & 0.10 & 9568.00 & 85022.47 & 1.00 & 0.89 & 0.89 \\
32294 & 106009 & 2000 & 2160.40 & -0.05 & 215330.00 & 2103300.09 & 1.00 & 0.97 & 0.98 \\
47459 & 211485 & 2000 & 250.50 & -0.07 & 25073.00 & 245128.89 & 1.00 & 0.98 & 0.98 \\
35029 & 106307 & 2000 & 25.70 & -0.20 & 3076.00 & 30754.29 & 0.84 & 1.20 & 1.00 \\
10383 & 101284 & 2000 & 2658.20 & -0.08 & 265825.00 & 2451593.33 & 1.00 & 0.92 & 0.92 \\
35033 & 106309 & 2000 & 119.70 & 0.27 & 11947.00 & 102831.84 & 1.00 & 0.86 & 0.86 \\
30395 & 105753 & 2000 & 55.60 & 0.09 & 5403.00 & 54029.67 & 1.03 & 0.97 & 1.00 \\
52288 & 302732 & 2000 & 770.40 & -0.10 & 78840.00 & 732230.95 & 0.98 & 0.95 & 0.93 \\
30387 & 105748 & 2000 & 51.80 & 0.13 & 5057.00 & 48645.61 & 1.02 & 0.94 & 0.96 \\
159 & 100016 & 2000 & 276.70 & -0.24 & 27582.00 & 274561.23 & 1.00 & 0.99 & 1.00 \\
15982 & 102062 & 2000 & 152.30 & 0.28 & 13768.00 & 137684.75 & 1.11 & 0.90 & 1.00 \\
30431 & 105758 & 2000 & 97.00 & -0.17 & 12220.00 & 95232.38 & 0.79 & 0.98 & 0.78 \\
42820 & 109021 & 2000 & 14.90 & 0.02 & 1585.00 & 13136.85 & 0.94 & 0.88 & 0.83 \\
30441 & 105760 & 2000 & 342.40 & -0.31 & 43640.00 & 368513.61 & 0.78 & 1.08 & 0.84 \\
3447 & 100435 & 2000 & 3.30 & -0.26 & 342.00 & 3249.60 & 0.96 & 0.98 & 0.95 \\
30497 & 105762 & 2000 & 251.60 & -0.40 & 31659.00 & 246259.17 & 0.79 & 0.98 & 0.78 \\
20857 & 102797 & 2000 & 35.90 & 0.09 & 3680.00 & 34859.49 & 0.98 & 0.97 & 0.95 \\
15288 & 101978 & 2000 & 92.70 & -0.11 & 9967.00 & 97527.24 & 0.93 & 1.05 & 0.98 \\
32266 & 106008 & 2000 & 342.20 & -0.13 & 35552.00 & 314937.33 & 0.96 & 0.92 & 0.89 \\
34989 & 106303 & 2000 & 7.80 & -0.21 & 774.00 & 7419.01 & 1.01 & 0.95 & 0.96 \\
30469 & 105761 & 2000 & 382.00 & 0.02 & 33529.00 & 348893.88 & 1.14 & 0.91 & 1.04 \\
19657 & 102641 & 2000 & 692.70 & -0.18 & 72330.00 & 660949.62 & 0.96 & 0.95 & 0.91 \\
34993 & 106305 & 2000 & 54.60 & 0.27 & 5707.00 & 56685.39 & 0.96 & 1.04 & 0.99 \\
33400 & 106129 & 2000 & 339.00 & -0.02 & 34634.00 & 345031.18 & 0.98 & 1.02 & 1.00 \\
35020 & 106306 & 2000 & 28.70 & -0.13 & 2873.00 & 27619.11 & 1.00 & 0.96 & 0.96 \\
52267 & 302731 & 2000 & 249.00 & 0.26 & 24765.00 & 225843.09 & 1.01 & 0.91 & 0.91 \\
32307 & 106010 & 2000 & 548.70 & 0.10 & 55644.00 & 527870.06 & 0.99 & 0.96 & 0.95 \\
30980 & 105843 & 2000 & 10.90 & -0.01 & 1090.00 & 10301.53 & 1.00 & 0.95 & 0.95 \\
42797 & 109019 & 2000 & 56.00 & 0.12 & 4868.00 & 51406.55 & 1.15 & 0.92 & 1.06 \\
13414 & 101738 & 2000 & 6267.00 & -0.31 & 811682.00 & 6028398.00 & 0.77 & 0.96 & 0.74 \\
35136 & 106326 & 2000 & 2.60 & -0.23 & 264.00 & 2206.85 & 0.98 & 0.85 & 0.84 \\
20927 & 102802 & 2000 & 291.40 & 0.09 & 29324.00 & 280999.15 & 0.99 & 0.96 & 0.96 \\
33348 & 106123 & 2000 & 756.60 & -0.10 & 82895.00 & 684585.46 & 0.91 & 0.90 & 0.83 \\
30266 & 105722 & 2000 & 34.60 & -0.19 & 3757.00 & 31099.33 & 0.92 & 0.90 & 0.83 \\
9583 & 101151 & 2000 & 260.70 & 0.04 & 25944.00 & 231143.28 & 1.00 & 0.89 & 0.89 \\
20049 & 102664 & 2000 & 1362.10 & 0.20 & 134235.00 & 1221731.67 & 1.01 & 0.90 & 0.91 \\
8820 & 101100 & 2000 & 803.70 & 0.50 & 87632.00 & 788052.14 & 0.92 & 0.98 & 0.90 \\
47607 & 215687 & 2000 & 241.30 & 0.33 & 23880.00 & 204578.18 & 1.01 & 0.85 & 0.86 \\
10462 & 101286 & 2000 & 1304.50 & 0.00 & 130453.00 & 1159264.23 & 1.00 & 0.89 & 0.89 \\
16026 & 102073 & 2000 & 9384.30 & 0.03 & 944976.00 & 8521592.70 & 0.99 & 0.91 & 0.90 \\
35143 & 106329 & 2000 & 31.10 & -0.09 & 2928.00 & 29276.68 & 1.06 & 0.94 & 1.00 \\
52472 & 302944 & 2000 & 16.00 & 0.10 & 1372.00 & 13512.86 & 1.17 & 0.84 & 0.98 \\
20939 & 102812 & 2000 & 16.10 & -0.15 & 1421.00 & 14212.78 & 1.13 & 0.88 & 1.00 \\
32332 & 106011 & 2000 & 1550.20 & -0.36 & 154144.00 & 1511209.49 & 1.01 & 0.97 & 0.98 \\
6053 & 100821 & 2000 & 89.90 & 0.17 & 8938.00 & 89363.86 & 1.01 & 0.99 & 1.00 \\
30289 & 105724 & 2000 & 37.50 & -0.02 & 3724.00 & 36445.82 & 1.01 & 0.97 & 0.98 \\
30294 & 105731 & 2000 & 1613.40 & 0.04 & 164203.00 & 1471994.57 & 0.98 & 0.91 & 0.90 \\
35060 & 106310 & 2000 & 22.70 & -0.10 & 3050.00 & 26975.15 & 0.74 & 1.19 & 0.88 \\
30356 & 105741 & 2000 & 152.50 & -0.04 & 17388.00 & 162730.54 & 0.88 & 1.07 & 0.94 \\
30351 & 105740 & 2000 & 622.40 & 0.15 & 62591.00 & 554229.80 & 0.99 & 0.89 & 0.89 \\
33381 & 106127 & 2000 & 121.80 & 0.12 & 11834.00 & 120156.70 & 1.03 & 0.99 & 1.02 \\
35066 & 106317 & 2000 & 311.90 & -0.16 & 31202.00 & 301016.98 & 1.00 & 0.97 & 0.96 \\
35075 & 106318 & 2000 & 156.00 & 0.05 & 15471.00 & 152337.43 & 1.01 & 0.98 & 0.98 \\
33355 & 106124 & 2000 & 297.40 & 0.20 & 30057.00 & 285054.58 & 0.99 & 0.96 & 0.95 \\
54842 & 400018 & 2000 & 154.50 & 0.00 & 14329.00 & 127605.88 & 1.08 & 0.83 & 0.89 \\
19367 & 102599 & 2000 & 1356.10 & 0.17 & 135617.00 & 1285762.96 & 1.00 & 0.95 & 0.95 \\
30323 & 105737 & 2000 & 6.90 & -0.10 & 672.00 & 6576.31 & 1.03 & 0.95 & 0.98 \\
35090 & 106320 & 2000 & 1371.10 & -0.17 & 141345.00 & 1271903.41 & 0.97 & 0.93 & 0.90 \\
35116 & 106321 & 2000 & 6.00 & -0.04 & 645.00 & 5607.64 & 0.93 & 0.93 & 0.87 \\
30373 & 105746 & 2000 & 250.90 & -0.02 & 25121.00 & 225691.33 & 1.00 & 0.90 & 0.90 \\
32131 & 105984 & 2000 & 190.80 & -0.12 & 19075.00 & 189636.20 & 1.00 & 0.99 & 0.99 \\
20143 & 102669 & 2000 & 45.70 & -0.07 & 4979.00 & 46056.11 & 0.92 & 1.01 & 0.93 \\
30987 & 105845 & 2000 & 4.30 & -0.22 & 487.00 & 4868.61 & 0.88 & 1.13 & 1.00 \\
9905 & 101211 & 2000 & 361.70 & -0.01 & 36185.00 & 347591.93 & 1.00 & 0.96 & 0.96 \\
31937 & 105963 & 2000 & 888.00 & -0.15 & 97726.00 & 837078.05 & 0.91 & 0.94 & 0.86 \\
19604 & 102636 & 2000 & 545.80 & 0.25 & 53870.00 & 492664.58 & 1.01 & 0.90 & 0.91 \\
52036 & 301299 & 2000 & 17148.50 & -0.20 & 1715934.00 & 17071719.22 & 1.00 & 1.00 & 0.99 \\
31926 & 105961 & 2000 & 118.50 & 0.13 & 11854.00 & 106551.69 & 1.00 & 0.90 & 0.90 \\
33998 & 106197 & 2000 & 74.30 & 0.19 & 7409.00 & 67884.74 & 1.00 & 0.91 & 0.92 \\
34025 & 106198 & 2000 & 243.20 & 0.17 & 22254.00 & 233400.41 & 1.09 & 0.96 & 1.05 \\
33701 & 106163 & 2000 & 156.30 & 0.05 & 15517.00 & 154845.83 & 1.01 & 0.99 & 1.00 \\
14875 & 101919 & 2000 & 2256.00 & -0.18 & 220451.00 & 2204732.10 & 1.02 & 0.98 & 1.00 \\
7392 & 101038 & 2000 & 5126.00 & -0.20 & 623002.00 & 5097780.73 & 0.82 & 0.99 & 0.82 \\
15588 & 102007 & 2000 & 7179.50 & -0.05 & 831994.00 & 6534642.06 & 0.86 & 0.91 & 0.79 \\
52591 & 303140 & 2000 & 321.10 & -0.04 & 31720.00 & 317235.07 & 1.01 & 0.99 & 1.00 \\
15425 & 101990 & 2000 & 116.30 & 0.17 & 11639.00 & 108725.22 & 1.00 & 0.93 & 0.93 \\
15573 & 102005 & 2000 & 1195.50 & -0.12 & 119553.00 & 1063021.50 & 1.00 & 0.89 & 0.89 \\
33971 & 106195 & 2000 & 7.60 & 0.07 & 732.00 & 6109.30 & 1.04 & 0.80 & 0.83 \\
31621 & 105918 & 2000 & 2805.50 & -0.25 & 279615.00 & 2423803.87 & 1.00 & 0.86 & 0.87 \\
9888 & 101200 & 2000 & 56.40 & -0.19 & 5364.00 & 53628.22 & 1.05 & 0.95 & 1.00 \\
13731 & 101759 & 2000 & 69.90 & -0.16 & 6169.00 & 61678.61 & 1.13 & 0.88 & 1.00 \\
31611 & 105917 & 2000 & 27.00 & 0.21 & 2721.00 & 24790.28 & 0.99 & 0.92 & 0.91 \\
61490 & 500083 & 2000 & 8.50 & 0.26 & 738.00 & 7727.85 & 1.15 & 0.91 & 1.05 \\
20387 & 102733 & 2000 & 4809.00 & 0.40 & 443900.00 & 4872587.59 & 1.08 & 1.01 & 1.10 \\
31916 & 105960 & 2000 & 301.70 & -0.14 & 30170.00 & 292416.32 & 1.00 & 0.97 & 0.97 \\
6687 & 100910 & 2000 & 143.70 & 0.05 & 14392.00 & 133752.96 & 1.00 & 0.93 & 0.93 \\
52568 & 303130 & 2000 & 27.20 & 0.07 & 2372.00 & 22776.23 & 1.15 & 0.84 & 0.96 \\
33955 & 106193 & 2000 & 44.60 & -0.06 & 4340.00 & 42805.27 & 1.03 & 0.96 & 0.99 \\
48524 & 240103 & 2000 & 64.10 & 0.23 & 4943.00 & 52885.58 & 1.30 & 0.83 & 1.07 \\
3754 & 100480 & 2000 & 149.20 & -0.09 & 16035.00 & 138801.02 & 0.93 & 0.93 & 0.87 \\
6231 & 100831 & 2000 & 176.40 & 0.05 & 17249.00 & 168812.72 & 1.02 & 0.96 & 0.98 \\
31529 & 105905 & 2000 & 33.50 & 0.31 & 3353.00 & 33188.07 & 1.00 & 0.99 & 0.99 \\
20419 & 102737 & 2000 & 2570.10 & -0.09 & 265462.00 & 2319923.66 & 0.97 & 0.90 & 0.87 \\
9923 & 101212 & 2000 & 1735.20 & -0.35 & 174116.00 & 1715717.60 & 1.00 & 0.99 & 0.99 \\
31466 & 105895 & 2000 & 523.00 & -0.27 & 69505.00 & 519340.53 & 0.75 & 0.99 & 0.75 \\
66 & 100004 & 2000 & 2115.40 & -0.10 & 245512.00 & 2133152.23 & 0.86 & 1.01 & 0.87 \\
34110 & 106208 & 2000 & 39.40 & 0.47 & 2673.00 & 32746.51 & 1.47 & 0.83 & 1.23 \\
15618 & 102009 & 2000 & 687.40 & 0.05 & 68914.00 & 672617.18 & 1.00 & 0.98 & 0.98 \\
33644 & 106158 & 2000 & 215.70 & -0.06 & 21183.00 & 206946.44 & 1.02 & 0.96 & 0.98 \\
9149 & 101115 & 2000 & 17844.20 & -0.09 & 2100777.00 & 18167429.67 & 0.85 & 1.02 & 0.86 \\
31449 & 105890 & 2000 & 56.60 & -0.21 & 9339.00 & 49111.25 & 0.61 & 0.87 & 0.53 \\
9084 & 101111 & 2000 & 495.80 & 0.19 & 44643.00 & 457500.67 & 1.11 & 0.92 & 1.02 \\
61556 & 500094 & 2000 & 21.20 & 0.27 & 2107.00 & 20088.49 & 1.01 & 0.95 & 0.95 \\
9958 & 101215 & 2000 & 44.10 & -0.10 & 6657.00 & 55703.70 & 0.66 & 1.26 & 0.84 \\
31439 & 105886 & 2000 & 63.80 & -0.06 & 6554.00 & 55983.09 & 0.97 & 0.88 & 0.85 \\
20471 & 102757 & 2000 & 26232.60 & -0.26 & 3717278.00 & 25131051.70 & 0.71 & 0.96 & 0.68 \\
33668 & 106160 & 2000 & 19.00 & 0.09 & 1947.00 & 18498.15 & 0.98 & 0.97 & 0.95 \\
51995 & 300695 & 2000 & 108.60 & 0.22 & 10973.00 & 100396.03 & 0.99 & 0.92 & 0.91 \\
20449 & 102744 & 2000 & 2037.50 & -0.06 & 200802.00 & 2048211.59 & 1.01 & 1.01 & 1.02 \\
9751 & 101186 & 2000 & 676.60 & -0.06 & 72781.00 & 580174.29 & 0.93 & 0.86 & 0.80 \\
34052 & 106199 & 2000 & 208.60 & 0.25 & 19486.00 & 197001.41 & 1.07 & 0.94 & 1.01 \\
52544 & 303123 & 2000 & 119.70 & -0.07 & 11281.00 & 112492.30 & 1.06 & 0.94 & 1.00 \\
43147 & 109067 & 2000 & 126.20 & -0.03 & 9627.00 & 80132.92 & 1.31 & 0.63 & 0.83 \\
6331 & 100849 & 2000 & 104.50 & -0.11 & 11986.00 & 87066.25 & 0.87 & 0.83 & 0.73 \\
31502 & 105903 & 2000 & 30.10 & 0.19 & 2674.00 & 24206.75 & 1.13 & 0.80 & 0.91 \\
31964 & 105964 & 2000 & 192.20 & -0.08 & 19095.00 & 153604.12 & 1.01 & 0.80 & 0.80 \\
34068 & 106203 & 2000 & 97.50 & -0.05 & 10422.00 & 96770.61 & 0.94 & 0.99 & 0.93 \\
31496 & 105900 & 2000 & 14.70 & -0.04 & 1568.00 & 14353.64 & 0.94 & 0.98 & 0.92 \\
31972 & 105965 & 2000 & 8.10 & -0.29 & 702.00 & 7017.93 & 1.15 & 0.87 & 1.00 \\
61288 & 500027 & 2000 & 197.90 & -0.06 & 23394.00 & 191690.66 & 0.85 & 0.97 & 0.82 \\
43099 & 109064 & 2000 & 114.40 & -0.24 & 16110.00 & 102704.72 & 0.71 & 0.90 & 0.64 \\
54915 & 400025 & 2000 & 220.50 & -0.02 & 26677.00 & 199364.45 & 0.83 & 0.90 & 0.75 \\
34094 & 106207 & 2000 & 19.00 & -0.16 & 2496.00 & 22670.68 & 0.76 & 1.19 & 0.91 \\
19555 & 102624 & 2000 & 1625.20 & -0.15 & 205355.00 & 1760865.96 & 0.79 & 1.08 & 0.86 \\
33928 & 106192 & 2000 & 2646.20 & -0.13 & 260882.00 & 2472028.53 & 1.01 & 0.93 & 0.95 \\
61480 & 500082 & 2000 & 83.90 & 0.13 & 7497.00 & 70124.65 & 1.12 & 0.84 & 0.94 \\
52130 & 302060 & 2000 & 386.90 & -0.18 & 38185.00 & 380561.04 & 1.01 & 0.98 & 1.00 \\
20255 & 102696 & 2000 & 373.10 & 0.04 & 37341.00 & 370939.46 & 1.00 & 0.99 & 0.99 \\
31799 & 105938 & 2000 & 108.30 & -0.23 & 11016.00 & 107069.73 & 0.98 & 0.99 & 0.97 \\
61398 & 500048 & 2000 & 62.20 & -0.05 & 6492.00 & 60294.54 & 0.96 & 0.97 & 0.93 \\
19594 & 102635 & 2000 & 918.90 & 0.02 & 92761.00 & 817386.21 & 0.99 & 0.89 & 0.88 \\
31791 & 105936 & 2000 & 132.80 & -0.08 & 13566.00 & 126635.48 & 0.98 & 0.95 & 0.93 \\
31864 & 105949 & 2000 & 220.50 & -0.09 & 21054.00 & 205996.21 & 1.05 & 0.93 & 0.98 \\
33744 & 106165 & 2000 & 184.70 & -0.08 & 17889.00 & 175908.40 & 1.03 & 0.95 & 0.98 \\
33764 & 106169 & 2000 & 20.70 & 0.44 & 1550.00 & 14765.68 & 1.34 & 0.71 & 0.95 \\
15498 & 101999 & 2000 & 1361.50 & 0.23 & 149138.00 & 1334261.37 & 0.91 & 0.98 & 0.89 \\
31763 & 105935 & 2000 & 438.20 & -0.11 & 43984.00 & 426704.13 & 1.00 & 0.97 & 0.97 \\
9830 & 101194 & 2000 & 210.90 & -0.02 & 21823.00 & 194070.63 & 0.97 & 0.92 & 0.89 \\
33752 & 106167 & 2000 & 152.30 & 0.15 & 15235.00 & 145379.70 & 1.00 & 0.95 & 0.95 \\
31806 & 105941 & 2000 & 25.70 & -0.22 & 2566.00 & 23630.42 & 1.00 & 0.92 & 0.92 \\
31810 & 105942 & 2000 & 2.40 & -0.26 & 265.00 & 2093.00 & 0.91 & 0.87 & 0.79 \\
6302 & 100847 & 2000 & 3.20 & 0.10 & 270.00 & 2302.80 & 1.19 & 0.72 & 0.85 \\
61372 & 500047 & 2000 & 2.60 & -0.08 & 257.00 & 2444.27 & 1.01 & 0.94 & 0.95 \\
6670 & 100908 & 2000 & 275.40 & -0.14 & 27743.00 & 263933.84 & 0.99 & 0.96 & 0.95 \\
31831 & 105946 & 2000 & 109.40 & 0.02 & 13222.00 & 109571.60 & 0.83 & 1.00 & 0.83 \\
3724 & 100475 & 2000 & 302.70 & 0.11 & 30250.00 & 291096.73 & 1.00 & 0.96 & 0.96 \\
31825 & 105945 & 2000 & 16.10 & -0.02 & 1581.00 & 15813.83 & 1.02 & 0.98 & 1.00 \\
48326 & 240062 & 2000 & 727.60 & 0.52 & 61898.00 & 656956.61 & 1.18 & 0.90 & 1.06 \\
31853 & 105948 & 2000 & 40.60 & 0.12 & 4406.00 & 39834.82 & 0.92 & 0.98 & 0.90 \\
15459 & 101992 & 2000 & 588.10 & -0.09 & 58852.00 & 571562.66 & 1.00 & 0.97 & 0.97 \\
13553 & 101743 & 2000 & 24253.00 & -0.32 & 3103924.00 & 20951264.77 & 0.78 & 0.86 & 0.67 \\
31814 & 105943 & 2000 & 49.20 & 0.08 & 3019.00 & 31671.80 & 1.63 & 0.64 & 1.05 \\
15479 & 101998 & 2000 & 1301.00 & 0.03 & 131896.00 & 1212422.54 & 0.99 & 0.93 & 0.92 \\
52138 & 302067 & 2000 & 17.20 & 0.09 & 1729.00 & 16605.01 & 0.99 & 0.97 & 0.96 \\
14912 & 101922 & 2000 & 280.00 & 0.09 & 26498.00 & 264985.29 & 1.06 & 0.95 & 1.00 \\
47476 & 212351 & 2000 & 363.70 & -0.22 & 35733.00 & 357226.81 & 1.02 & 0.98 & 1.00 \\
31752 & 105933 & 2000 & 2727.70 & -0.11 & 312887.00 & 2426569.66 & 0.87 & 0.89 & 0.78 \\
33728 & 106164 & 2000 & 75.50 & 0.06 & 7483.00 & 73954.36 & 1.01 & 0.98 & 0.99 \\
52161 & 302545 & 2000 & 91.90 & 0.00 & 9425.00 & 87188.60 & 0.98 & 0.95 & 0.93 \\
33873 & 106179 & 2000 & 89.90 & -0.10 & 8994.00 & 83602.78 & 1.00 & 0.93 & 0.93 \\
4063 & 100544 & 2000 & 500.00 & 0.29 & 47965.00 & 515962.22 & 1.04 & 1.03 & 1.08 \\
31679 & 105930 & 2000 & 1195.10 & -0.41 & 124436.00 & 1091905.63 & 0.96 & 0.91 & 0.88 \\
31896 & 105954 & 2000 & 3.10 & 0.01 & 317.00 & 3004.63 & 0.98 & 0.97 & 0.95 \\
9860 & 101198 & 2000 & 267.60 & -0.01 & 29975.00 & 217923.18 & 0.89 & 0.81 & 0.73 \\
31673 & 105926 & 2000 & 1.40 & 0.00 & 154.00 & 1398.24 & 0.91 & 1.00 & 0.91 \\
33908 & 106182 & 2000 & 266.20 & 0.26 & 26679.00 & 218603.93 & 1.00 & 0.82 & 0.82 \\
31901 & 105957 & 2000 & 18.70 & -0.17 & 2127.00 & 16980.18 & 0.88 & 0.91 & 0.80 \\
3681 & 100468 & 2000 & 339.80 & 0.19 & 34184.00 & 317958.92 & 0.99 & 0.94 & 0.93 \\
31645 & 105920 & 2000 & 6016.80 & -0.18 & 717875.00 & 5641819.54 & 0.84 & 0.94 & 0.79 \\
48472 & 240087 & 2000 & 61.30 & -0.01 & 6262.00 & 58527.35 & 0.98 & 0.95 & 0.93 \\
33921 & 106189 & 2000 & 630.20 & -0.03 & 63954.00 & 582260.35 & 0.99 & 0.92 & 0.91 \\
33882 & 106180 & 2000 & 67.70 & -0.06 & 7326.00 & 65076.84 & 0.92 & 0.96 & 0.89 \\
15529 & 102000 & 2000 & 873.80 & -0.12 & 95878.00 & 847562.80 & 0.91 & 0.97 & 0.88 \\
9116 & 101112 & 2000 & 1465.90 & -0.07 & 159447.00 & 1428167.86 & 0.92 & 0.97 & 0.90 \\
48493 & 240090 & 2000 & 34.00 & 0.10 & 3403.00 & 33146.34 & 1.00 & 0.97 & 0.97 \\
33791 & 106170 & 2000 & 150.70 & 0.34 & 15071.00 & 142721.99 & 1.00 & 0.95 & 0.95 \\
31889 & 105951 & 2000 & 378.00 & -0.24 & 38606.00 & 377038.57 & 0.98 & 1.00 & 0.98 \\
52102 & 301571 & 2000 & 330.20 & 0.11 & 33709.00 & 300322.68 & 0.98 & 0.91 & 0.89 \\
33804 & 106172 & 2000 & 98.50 & 0.13 & 9981.00 & 81254.11 & 0.99 & 0.82 & 0.81 \\
31725 & 105932 & 2000 & 217.90 & 0.02 & 19792.00 & 176325.15 & 1.10 & 0.81 & 0.89 \\
61422 & 500064 & 2000 & 9.50 & 0.10 & 936.00 & 7642.57 & 1.01 & 0.80 & 0.82 \\
33831 & 106173 & 2000 & 249.40 & 0.10 & 25263.00 & 245206.69 & 0.99 & 0.98 & 0.97 \\
52089 & 301560 & 2000 & 451.00 & 0.05 & 47817.00 & 443802.90 & 0.94 & 0.98 & 0.93 \\
4049 & 100543 & 2000 & 523.90 & -0.04 & 58279.00 & 547973.61 & 0.90 & 1.05 & 0.94 \\
48318 & 240061 & 2000 & 246.20 & 0.26 & 19191.00 & 168319.10 & 1.28 & 0.68 & 0.88 \\
6258 & 100833 & 2000 & 1280.10 & -0.13 & 127439.00 & 1271143.47 & 1.00 & 0.99 & 1.00 \\
47508 & 212408 & 2000 & 3668.10 & -0.24 & 366549.00 & 3562235.81 & 1.00 & 0.97 & 0.97 \\
31698 & 105931 & 2000 & 615.20 & 0.29 & 40431.00 & 373449.49 & 1.52 & 0.61 & 0.92 \\
61330 & 500037 & 2000 & 6529.50 & -0.19 & 631320.00 & 5343673.71 & 1.03 & 0.82 & 0.85 \\
32404 & 106023 & 2000 & 261.10 & 0.04 & 27468.00 & 239686.76 & 0.95 & 0.92 & 0.87 \\
6197 & 100829 & 2000 & 699.50 & 0.09 & 70627.00 & 651325.89 & 0.99 & 0.93 & 0.92 \\
32072 & 105980 & 2000 & 274.90 & 0.18 & 27948.00 & 261660.79 & 0.98 & 0.95 & 0.94 \\
31154 & 105864 & 2000 & 32.70 & -0.22 & 3278.00 & 27314.17 & 1.00 & 0.84 & 0.83 \\
43025 & 109052 & 2000 & 21.90 & 0.01 & 2024.00 & 18381.29 & 1.08 & 0.84 & 0.91 \\
15367 & 101988 & 2000 & 448.90 & 0.10 & 46133.00 & 438223.74 & 0.97 & 0.98 & 0.95 \\
48447 & 240085 & 2000 & 271.50 & 0.11 & 26988.00 & 260469.83 & 1.01 & 0.96 & 0.97 \\
10077 & 101258 & 2000 & 3058.90 & 0.16 & 275387.00 & 2471854.40 & 1.11 & 0.81 & 0.90 \\
34395 & 106231 & 2000 & 110.80 & 0.26 & 11603.00 & 115425.45 & 0.95 & 1.04 & 0.99 \\
33599 & 106152 & 2000 & 721.60 & -0.16 & 73538.00 & 696915.47 & 0.98 & 0.97 & 0.95 \\
51908 & 300102 & 2000 & 75.70 & -0.20 & 9849.00 & 81633.91 & 0.77 & 1.08 & 0.83 \\
33585 & 106151 & 2000 & 1064.20 & 0.10 & 108222.00 & 1060522.91 & 0.98 & 1.00 & 0.98 \\
20619 & 102775 & 2000 & 1696.00 & 0.13 & 163207.00 & 1650871.10 & 1.04 & 0.97 & 1.01 \\
31110 & 105860 & 2000 & 5754.40 & -0.27 & 719160.00 & 5272185.34 & 0.80 & 0.92 & 0.73 \\
8999 & 101108 & 2000 & 688.60 & 0.06 & 67811.00 & 599453.59 & 1.02 & 0.87 & 0.88 \\
3565 & 100456 & 2000 & 3.40 & -0.05 & 365.00 & 3350.09 & 0.93 & 0.99 & 0.92 \\
34428 & 106239 & 2000 & 145.60 & -0.06 & 14601.00 & 144338.59 & 1.00 & 0.99 & 0.99 \\
31161 & 105865 & 2000 & 90.80 & 0.21 & 9328.00 & 89244.74 & 0.97 & 0.98 & 0.96 \\
48583 & 240111 & 2000 & 3992.00 & 0.33 & 422892.00 & 3529094.40 & 0.94 & 0.88 & 0.83 \\
10047 & 101256 & 2000 & 8.10 & -0.05 & 860.00 & 7827.59 & 0.94 & 0.97 & 0.91 \\
15724 & 102016 & 2000 & 17314.80 & -0.07 & 1923465.00 & 15171962.46 & 0.90 & 0.88 & 0.79 \\
34330 & 106223 & 2000 & 84.50 & 0.24 & 8449.00 & 84208.75 & 1.00 & 1.00 & 1.00 \\
3578 & 100457 & 2000 & 337.10 & -0.13 & 33733.00 & 321947.57 & 1.00 & 0.96 & 0.95 \\
20599 & 102774 & 2000 & 10361.60 & -0.04 & 868953.00 & 8112388.37 & 1.19 & 0.78 & 0.93 \\
34350 & 106224 & 2000 & 26.70 & 0.17 & 2620.00 & 23359.88 & 1.02 & 0.87 & 0.89 \\
52650 & 305590 & 2000 & 27.10 & 0.16 & 2830.00 & 26516.17 & 0.96 & 0.98 & 0.94 \\
20175 & 102673 & 2000 & 488.30 & 0.14 & 49731.00 & 464176.34 & 0.98 & 0.95 & 0.93 \\
54887 & 400020 & 2000 & 37.70 & -0.03 & 3751.00 & 35777.08 & 1.01 & 0.95 & 0.95 \\
34364 & 106226 & 2000 & 52.00 & 0.07 & 5203.00 & 50089.65 & 1.00 & 0.96 & 0.96 \\
34368 & 106230 & 2000 & 172.60 & 0.28 & 21214.00 & 195526.23 & 0.81 & 1.13 & 0.92 \\
31189 & 105866 & 2000 & 7845.50 & -0.09 & 842053.00 & 7756219.35 & 0.93 & 0.99 & 0.92 \\
15758 & 102017 & 2000 & 14134.40 & -0.01 & 1465160.00 & 12632090.77 & 0.96 & 0.89 & 0.86 \\
61259 & 500025 & 2000 & 1.20 & 0.04 & 106.00 & 1032.46 & 1.13 & 0.86 & 0.97 \\
34444 & 106240 & 2000 & 175.50 & -0.03 & 18591.00 & 145781.86 & 0.94 & 0.83 & 0.78 \\
3556 & 100455 & 2000 & 6.70 & -0.15 & 977.00 & 7971.01 & 0.69 & 1.19 & 0.82 \\
31022 & 105848 & 2000 & 69.20 & -0.01 & 6926.00 & 67416.41 & 1.00 & 0.97 & 0.97 \\
31014 & 105847 & 2000 & 33.60 & 0.17 & 3380.00 & 30789.79 & 0.99 & 0.92 & 0.91 \\
34524 & 106249 & 2000 & 93.40 & -0.10 & 9293.00 & 88688.77 & 1.01 & 0.95 & 0.95 \\
14831 & 101916 & 2000 & 339.20 & 0.23 & 34054.00 & 321455.57 & 1.00 & 0.95 & 0.94 \\
31030 & 105849 & 2000 & 53.90 & -0.13 & 5298.00 & 45334.94 & 1.02 & 0.84 & 0.86 \\
15789 & 102018 & 2000 & 1020.80 & -0.08 & 119611.00 & 972293.81 & 0.85 & 0.95 & 0.81 \\
48593 & 240112 & 2000 & 40.90 & -0.07 & 4198.00 & 39102.96 & 0.97 & 0.96 & 0.93 \\
30998 & 105846 & 2000 & 2600.70 & 0.03 & 260180.00 & 2601869.20 & 1.00 & 1.00 & 1.00 \\
10146 & 101263 & 2000 & 387.90 & -0.20 & 48197.00 & 398512.29 & 0.80 & 1.03 & 0.83 \\
20655 & 102777 & 2000 & 6064.10 & -0.51 & 1055424.00 & 5221034.80 & 0.57 & 0.86 & 0.49 \\
33562 & 106149 & 2000 & 774.90 & -0.22 & 77461.00 & 708093.11 & 1.00 & 0.91 & 0.91 \\
13748 & 101762 & 2000 & 3375.00 & 0.11 & 345944.00 & 3023940.62 & 0.98 & 0.90 & 0.87 \\
10129 & 101262 & 2000 & 18.40 & 0.07 & 1837.00 & 17752.49 & 1.00 & 0.96 & 0.97 \\
43021 & 109051 & 2000 & 104.20 & -0.17 & 12288.00 & 120220.72 & 0.85 & 1.15 & 0.98 \\
9681 & 101165 & 2000 & 1583.60 & -0.10 & 158374.00 & 1548921.04 & 1.00 & 0.98 & 0.98 \\
31 & 100003 & 2000 & 958.40 & 0.20 & 96843.00 & 851675.73 & 0.99 & 0.89 & 0.88 \\
31083 & 105857 & 2000 & 110.60 & 0.22 & 11064.00 & 106094.57 & 1.00 & 0.96 & 0.96 \\
34471 & 106244 & 2000 & 4.20 & -0.19 & 427.00 & 3705.39 & 0.98 & 0.88 & 0.87 \\
9183 & 101116 & 2000 & 2125.20 & -0.20 & 303933.00 & 2159429.72 & 0.70 & 1.02 & 0.71 \\
19469 & 102606 & 2000 & 4706.70 & 0.15 & 470675.00 & 4570511.53 & 1.00 & 0.97 & 0.97 \\
43002 & 109048 & 2000 & 342.60 & -0.07 & 37078.00 & 303579.10 & 0.92 & 0.89 & 0.82 \\
31068 & 105854 & 2000 & 842.30 & -0.02 & 83611.00 & 796176.90 & 1.01 & 0.95 & 0.95 \\
34497 & 106248 & 2000 & 151.20 & 0.17 & 13063.00 & 114688.96 & 1.16 & 0.76 & 0.88 \\
32104 & 105983 & 2000 & 325.20 & -0.08 & 32875.00 & 315318.84 & 0.99 & 0.97 & 0.96 \\
20154 & 102671 & 2000 & 284.50 & -0.13 & 32123.00 & 279858.87 & 0.89 & 0.98 & 0.87 \\
31040 & 105852 & 2000 & 49.40 & 0.18 & 4942.00 & 45423.46 & 1.00 & 0.92 & 0.92 \\
47566 & 212809 & 2000 & 2.40 & 0.22 & 242.00 & 2170.45 & 0.99 & 0.90 & 0.90 \\
20188 & 102676 & 2000 & 258.60 & -0.11 & 26724.00 & 239412.05 & 0.97 & 0.93 & 0.90 \\
31222 & 105868 & 2000 & 147.60 & -0.11 & 14767.00 & 141889.83 & 1.00 & 0.96 & 0.96 \\
13585 & 101744 & 2000 & 1487.00 & -0.46 & 191871.00 & 1361195.23 & 0.77 & 0.92 & 0.71 \\
34221 & 106213 & 2000 & 32.70 & 0.03 & 4130.00 & 28633.81 & 0.79 & 0.88 & 0.69 \\
3622 & 100463 & 2000 & 143.00 & 0.28 & 12815.00 & 128156.36 & 1.12 & 0.90 & 1.00 \\
31345 & 105879 & 2000 & 1417.00 & 0.09 & 141608.00 & 1310017.24 & 1.00 & 0.92 & 0.93 \\
48562 & 240107 & 2000 & 45.00 & 0.79 & 4755.00 & 42019.99 & 0.95 & 0.93 & 0.88 \\
31363 & 105880 & 2000 & 288.30 & 0.27 & 29119.00 & 283994.34 & 0.99 & 0.99 & 0.98 \\
33612 & 106156 & 2000 & 18.90 & -0.24 & 3604.00 & 36033.16 & 0.52 & 1.91 & 1.00 \\
32029 & 105974 & 2000 & 63.00 & -0.19 & 6055.00 & 60548.73 & 1.04 & 0.96 & 1.00 \\
34233 & 106214 & 2000 & 74.60 & 0.22 & 7182.00 & 71823.11 & 1.04 & 0.96 & 1.00 \\
19509 & 102608 & 2000 & 86.20 & 0.29 & 8617.00 & 79706.57 & 1.00 & 0.92 & 0.92 \\
10005 & 101252 & 2000 & 97.00 & -0.01 & 8979.00 & 89799.17 & 1.08 & 0.93 & 1.00 \\
10001 & 101251 & 2000 & 43.30 & 0.04 & 4327.00 & 35746.19 & 1.00 & 0.83 & 0.83 \\
52538 & 303121 & 2000 & 270.40 & -0.03 & 27025.00 & 262854.41 & 1.00 & 0.97 & 0.97 \\
61567 & 500096 & 2000 & 49.30 & 0.35 & 4364.00 & 45873.32 & 1.13 & 0.93 & 1.05 \\
34203 & 106212 & 2000 & 97.10 & 0.09 & 10206.00 & 93638.17 & 0.95 & 0.96 & 0.92 \\
32002 & 105973 & 2000 & 10.10 & -0.05 & 1168.00 & 10573.88 & 0.86 & 1.05 & 0.91 \\
20203 & 102688 & 2000 & 136.20 & 0.17 & 13632.00 & 115046.99 & 1.00 & 0.84 & 0.84 \\
31417 & 105882 & 2000 & 244.40 & -0.14 & 33838.00 & 241713.34 & 0.72 & 0.99 & 0.71 \\
43083 & 109062 & 2000 & 18.80 & -0.16 & 1844.00 & 18434.36 & 1.02 & 0.98 & 1.00 \\
6701 & 100913 & 2000 & 346.70 & 0.06 & 34743.00 & 320829.18 & 1.00 & 0.93 & 0.92 \\
15639 & 102010 & 2000 & 15467.60 & -0.26 & 2065665.00 & 15307236.13 & 0.75 & 0.99 & 0.74 \\
33624 & 106157 & 2000 & 937.00 & -0.08 & 92879.00 & 915531.02 & 1.01 & 0.98 & 0.99 \\
34164 & 106210 & 2000 & 8.80 & -0.09 & 1067.00 & 9816.43 & 0.82 & 1.12 & 0.92 \\
15397 & 101989 & 2000 & 404.20 & -0.11 & 41508.00 & 366712.94 & 0.97 & 0.91 & 0.88 \\
20500 & 102760 & 2000 & 3129.10 & -0.33 & 440383.00 & 2766096.44 & 0.71 & 0.88 & 0.63 \\
31390 & 105881 & 2000 & 268.40 & 0.31 & 19275.00 & 261374.44 & 1.39 & 0.97 & 1.36 \\
34191 & 106211 & 2000 & 86.20 & 0.14 & 8613.00 & 79984.84 & 1.00 & 0.93 & 0.93 \\
52617 & 303175 & 2000 & 1001.20 & -0.09 & 112847.00 & 1069899.39 & 0.89 & 1.07 & 0.95 \\
32035 & 105976 & 2000 & 14.50 & -0.28 & 1810.00 & 12026.33 & 0.80 & 0.83 & 0.66 \\
31317 & 105878 & 2000 & 502.10 & 0.13 & 50267.00 & 436956.19 & 1.00 & 0.87 & 0.87 \\
31309 & 105877 & 2000 & 13.90 & -0.10 & 1585.00 & 13734.16 & 0.88 & 0.99 & 0.87 \\
31255 & 105871 & 2000 & 9.70 & 0.26 & 881.00 & 9116.72 & 1.10 & 0.94 & 1.03 \\
52640 & 305586 & 2000 & 79.90 & -0.13 & 11033.00 & 65602.91 & 0.72 & 0.82 & 0.59 \\
33608 & 106155 & 2000 & 13.00 & -0.10 & 1289.00 & 12538.24 & 1.01 & 0.96 & 0.97 \\
47471 & 212027 & 2000 & 142.20 & 0.11 & 14156.00 & 134684.76 & 1.00 & 0.95 & 0.95 \\
34293 & 106221 & 2000 & 99.60 & 0.16 & 9841.00 & 95566.84 & 1.01 & 0.96 & 0.97 \\
43029 & 109053 & 2000 & 834.00 & -0.16 & 83552.00 & 819286.71 & 1.00 & 0.98 & 0.98 \\
43031 & 109056 & 2000 & 201.10 & 0.22 & 14711.00 & 143437.43 & 1.37 & 0.71 & 0.98 \\
6162 & 100827 & 2000 & 296.30 & 0.05 & 29462.00 & 290802.39 & 1.01 & 0.98 & 0.99 \\
34320 & 106222 & 2000 & 142.30 & -0.05 & 14144.00 & 133650.04 & 1.01 & 0.94 & 0.94 \\
19492 & 102607 & 2000 & 910.10 & -0.07 & 91019.00 & 835524.61 & 1.00 & 0.92 & 0.92 \\
32060 & 105978 & 2000 & 212.90 & -0.33 & 19799.00 & 193929.03 & 1.08 & 0.91 & 0.98 \\
31228 & 105869 & 2000 & 88.40 & 0.04 & 8844.00 & 87576.19 & 1.00 & 0.99 & 0.99 \\
6149 & 100825 & 2000 & 95.70 & 0.05 & 9539.00 & 92535.09 & 1.00 & 0.97 & 0.97 \\
47187 & 200342 & 2000 & 3522.10 & 0.08 & 320493.00 & 3332864.57 & 1.10 & 0.95 & 1.04 \\
15699 & 102015 & 2000 & 642.50 & -0.22 & 64510.00 & 604031.56 & 1.00 & 0.94 & 0.94 \\
47539 & 212658 & 2000 & 6591.20 & 0.07 & 638069.00 & 6135481.06 & 1.03 & 0.93 & 0.96 \\
61613 & 500107 & 2000 & 549.50 & -0.27 & 55592.00 & 522267.14 & 0.99 & 0.95 & 0.94 \\
3784 & 100481 & 2000 & 73.80 & 0.19 & 8294.00 & 63859.58 & 0.89 & 0.87 & 0.77 \\
31302 & 105876 & 2000 & 104.90 & -0.07 & 10492.00 & 92232.35 & 1.00 & 0.88 & 0.88 \\
51952 & 300673 & 2000 & 62.50 & 0.29 & 6402.00 & 58931.84 & 0.98 & 0.94 & 0.92 \\
96690 & 611006 & 2000 & 209.50 & -0.25 & 27634.00 & 170893.21 & 0.76 & 0.82 & 0.62 \\
52519 & 302997 & 2000 & 154.30 & 0.19 & 15572.00 & 142430.23 & 0.99 & 0.92 & 0.91 \\
31293 & 105875 & 2000 & 78.20 & -0.23 & 7803.00 & 73713.81 & 1.00 & 0.94 & 0.94 \\
20559 & 102767 & 2000 & 12764.70 & -0.19 & 1329962.00 & 11796294.86 & 0.96 & 0.92 & 0.89 \\
31279 & 105874 & 2000 & 37.00 & 0.04 & 3732.00 & 35407.05 & 0.99 & 0.96 & 0.95 \\
51944 & 300657 & 2000 & 50.40 & -0.20 & 4669.00 & 52899.12 & 1.08 & 1.05 & 1.13 \\
32044 & 105977 & 2000 & 1398.00 & 0.09 & 144493.00 & 1298139.67 & 0.97 & 0.93 & 0.90 \\
31268 & 105873 & 2000 & 12.50 & -0.05 & 1278.00 & 13239.27 & 0.98 & 1.06 & 1.04 \\
34287 & 106220 & 2000 & 855.90 & -0.13 & 103162.00 & 780584.56 & 0.83 & 0.91 & 0.76 \\
43052 & 109058 & 2000 & 29.80 & 0.06 & 2937.00 & 29013.56 & 1.01 & 0.97 & 0.99 \\
21024 & 102824 & 2000 & 68.90 & 0.03 & 6684.00 & 66841.10 & 1.03 & 0.97 & 1.00 \\
52554 & 303124 & 2000 & 38.00 & 0.06 & 2427.00 & 23044.75 & 1.57 & 0.61 & 0.95 \\
30009 & 105677 & 2000 & 65.10 & 0.03 & 6419.00 & 64196.84 & 1.01 & 0.99 & 1.00 \\
28985 & 105510 & 2000 & 40.00 & -0.23 & 6009.00 & 39726.44 & 0.67 & 0.99 & 0.66 \\
36167 & 106476 & 2000 & 6.90 & 0.14 & 675.00 & 6246.66 & 1.02 & 0.91 & 0.93 \\
36193 & 106477 & 2000 & 1028.20 & -0.11 & 120482.00 & 994002.38 & 0.85 & 0.97 & 0.83 \\
28996 & 105511 & 2000 & 28.10 & 0.30 & 2816.00 & 27291.73 & 1.00 & 0.97 & 0.97 \\
322 & 100036 & 2000 & 109.30 & -0.02 & 10936.00 & 107905.84 & 1.00 & 0.99 & 0.99 \\
21394 & 102861 & 2000 & 208.40 & -0.21 & 27737.00 & 209399.38 & 0.75 & 1.00 & 0.75 \\
9323 & 101131 & 2000 & 8280.20 & -0.08 & 826360.00 & 7581773.73 & 1.00 & 0.92 & 0.92 \\
28940 & 105507 & 2000 & 909.80 & -0.09 & 90761.00 & 855669.20 & 1.00 & 0.94 & 0.94 \\
10958 & 101356 & 2000 & 516.60 & -0.06 & 51640.00 & 504856.41 & 1.00 & 0.98 & 0.98 \\
14673 & 101908 & 2000 & 29.10 & 0.08 & 2492.00 & 24920.47 & 1.17 & 0.86 & 1.00 \\
28959 & 105508 & 2000 & 30.80 & 0.14 & 3069.00 & 30533.45 & 1.00 & 0.99 & 0.99 \\
16464 & 102150 & 2000 & 103.40 & -0.16 & 10470.00 & 96830.30 & 0.99 & 0.94 & 0.92 \\
13282 & 101717 & 2000 & 94.60 & -0.00 & 9597.00 & 90481.90 & 0.99 & 0.96 & 0.94 \\
9448 & 101135 & 2000 & 2168.40 & 0.26 & 226795.00 & 1988220.59 & 0.96 & 0.92 & 0.88 \\
3241 & 100417 & 2000 & 11.30 & 0.00 & 1236.00 & 10816.63 & 0.91 & 0.96 & 0.88 \\
29038 & 105522 & 2000 & 99.00 & 0.15 & 9881.00 & 97670.31 & 1.00 & 0.99 & 0.99 \\
10926 & 101354 & 2000 & 1798.00 & -0.02 & 181055.00 & 1752336.83 & 0.99 & 0.97 & 0.97 \\
52394 & 302879 & 2000 & 31.90 & -0.14 & 3192.00 & 27671.19 & 1.00 & 0.87 & 0.87 \\
29030 & 105520 & 2000 & 17.10 & 0.08 & 1719.00 & 15690.83 & 0.99 & 0.92 & 0.91 \\
16442 & 102145 & 2000 & 176.70 & -0.01 & 17881.00 & 170822.95 & 0.99 & 0.97 & 0.96 \\
21361 & 102854 & 2000 & 482.30 & -0.11 & 63386.00 & 565941.46 & 0.76 & 1.17 & 0.89 \\
15114 & 101958 & 2000 & 1136.60 & -0.10 & 113387.00 & 1108476.43 & 1.00 & 0.98 & 0.98 \\
46393 & 200233 & 2000 & 4.60 & -0.13 & 463.00 & 4561.16 & 0.99 & 0.99 & 0.99 \\
36157 & 106474 & 2000 & 112.40 & -0.10 & 12598.00 & 109729.98 & 0.89 & 0.98 & 0.87 \\
32662 & 106047 & 2000 & 12.30 & -0.13 & 1637.00 & 12132.08 & 0.75 & 0.99 & 0.74 \\
42474 & 108970 & 2000 & 95.20 & -0.18 & 9571.00 & 95669.69 & 0.99 & 1.00 & 1.00 \\
36149 & 106471 & 2000 & 201.30 & 0.23 & 16873.00 & 199628.00 & 1.19 & 0.99 & 1.18 \\
48385 & 240074 & 2000 & 21.90 & 0.30 & 2053.00 & 19167.83 & 1.07 & 0.88 & 0.93 \\
32968 & 106084 & 2000 & 2406.00 & -0.37 & 357176.00 & 2036859.43 & 0.67 & 0.85 & 0.57 \\
10984 & 101358 & 2000 & 394.10 & -0.03 & 39404.00 & 390191.96 & 1.00 & 0.99 & 0.99 \\
32701 & 106050 & 2000 & 478.30 & 0.07 & 47364.00 & 469619.71 & 1.01 & 0.98 & 0.99 \\
6791 & 100954 & 2000 & 1254.90 & -0.06 & 125167.00 & 1200706.63 & 1.00 & 0.96 & 0.96 \\
16495 & 102151 & 2000 & 16.30 & -0.13 & 1623.00 & 15530.71 & 1.00 & 0.95 & 0.96 \\
28854 & 105487 & 2000 & 195.70 & -0.15 & 20864.00 & 177609.13 & 0.94 & 0.91 & 0.85 \\
339 & 100040 & 2000 & 5502.80 & 0.04 & 548353.00 & 5394133.81 & 1.00 & 0.98 & 0.98 \\
28872 & 105498 & 2000 & 148.30 & -0.10 & 14842.00 & 147907.28 & 1.00 & 1.00 & 1.00 \\
36324 & 106483 & 2000 & 5.50 & 0.07 & 532.00 & 5317.45 & 1.03 & 0.97 & 1.00 \\
42451 & 108968 & 2000 & 10.30 & -0.03 & 1062.00 & 9969.93 & 0.97 & 0.97 & 0.94 \\
36331 & 106484 & 2000 & 3.80 & 0.01 & 366.00 & 3657.19 & 1.04 & 0.96 & 1.00 \\
36334 & 106485 & 2000 & 172.60 & -0.27 & 17211.00 & 172191.07 & 1.00 & 1.00 & 1.00 \\
15096 & 101956 & 2000 & 3523.40 & -0.12 & 354148.00 & 3489669.63 & 0.99 & 0.99 & 0.99 \\
42448 & 108967 & 2000 & 100.60 & -0.17 & 10190.00 & 96900.34 & 0.99 & 0.96 & 0.95 \\
36350 & 106487 & 2000 & 19.40 & 0.12 & 1934.00 & 19364.32 & 1.00 & 1.00 & 1.00 \\
4015 & 100538 & 2000 & 567.00 & 0.09 & 62011.00 & 572662.15 & 0.91 & 1.01 & 0.92 \\
21422 & 102871 & 2000 & 141.50 & 0.16 & 14190.00 & 128208.13 & 1.00 & 0.91 & 0.90 \\
28882 & 105502 & 2000 & 2106.50 & -0.02 & 210245.00 & 1778992.23 & 1.00 & 0.84 & 0.85 \\
28924 & 105506 & 2000 & 238.40 & -0.05 & 23713.00 & 237128.89 & 1.01 & 0.99 & 1.00 \\
36265 & 106480 & 2000 & 477.60 & -0.16 & 56107.00 & 407102.04 & 0.85 & 0.85 & 0.73 \\
3213 & 100415 & 2000 & 138.10 & 0.03 & 14577.00 & 141481.15 & 0.95 & 1.02 & 0.97 \\
6550 & 100890 & 2000 & 1190.10 & 0.01 & 119085.00 & 1180583.84 & 1.00 & 0.99 & 0.99 \\
13273 & 101716 & 2000 & 40.90 & 0.05 & 4326.00 & 36903.10 & 0.95 & 0.90 & 0.85 \\
36291 & 106481 & 2000 & 44.30 & 0.19 & 4194.00 & 41939.34 & 1.06 & 0.95 & 1.00 \\
19840 & 102653 & 2000 & 5428.30 & 0.00 & 533630.00 & 4840455.98 & 1.02 & 0.89 & 0.91 \\
36298 & 106482 & 2000 & 58.30 & 0.31 & 5026.00 & 49202.48 & 1.16 & 0.84 & 0.98 \\
7527 & 101043 & 2000 & 15220.40 & -0.27 & 1464422.00 & 15105971.11 & 1.04 & 0.99 & 1.03 \\
32689 & 106049 & 2000 & 558.60 & -0.09 & 59857.00 & 543656.81 & 0.93 & 0.97 & 0.91 \\
13266 & 101714 & 2000 & 16.30 & -0.15 & 1574.00 & 15742.34 & 1.04 & 0.97 & 1.00 \\
10973 & 101357 & 2000 & 285.40 & 0.06 & 28455.00 & 269276.07 & 1.00 & 0.94 & 0.95 \\
36138 & 106470 & 2000 & 665.20 & -0.01 & 65510.00 & 622595.57 & 1.02 & 0.94 & 0.95 \\
42510 & 108972 & 2000 & 49.70 & -0.31 & 5053.00 & 49888.72 & 0.98 & 1.00 & 0.99 \\
36129 & 106467 & 2000 & 225.40 & -0.16 & 25034.00 & 213845.07 & 0.90 & 0.95 & 0.85 \\
33073 & 106089 & 2000 & 140.00 & -0.12 & 11041.00 & 142789.57 & 1.27 & 1.02 & 1.29 \\
29224 & 105545 & 2000 & 1436.00 & -0.18 & 170163.00 & 1401138.10 & 0.84 & 0.98 & 0.82 \\
35985 & 106444 & 2000 & 138.30 & 0.01 & 13850.00 & 137309.76 & 1.00 & 0.99 & 0.99 \\
36008 & 106447 & 2000 & 279.40 & -0.16 & 27012.00 & 270129.07 & 1.03 & 0.97 & 1.00 \\
13804 & 101764 & 2000 & 1093.40 & -0.09 & 118370.00 & 1019596.10 & 0.92 & 0.93 & 0.86 \\
29198 & 105536 & 2000 & 1158.50 & -0.25 & 115625.00 & 1015978.06 & 1.00 & 0.88 & 0.88 \\
21302 & 102846 & 2000 & 98.70 & -0.05 & 15043.00 & 148023.99 & 0.66 & 1.50 & 0.98 \\
10 & 100001 & 2000 & 4391.80 & -0.01 & 439173.00 & 3946151.98 & 1.00 & 0.90 & 0.90 \\
29182 & 105535 & 2000 & 413.70 & -0.00 & 41207.00 & 384658.72 & 1.00 & 0.93 & 0.93 \\
50632 & 240432 & 2000 & 16.90 & 0.33 & 1425.00 & 13542.53 & 1.19 & 0.80 & 0.95 \\
29176 & 105534 & 2000 & 261.70 & -0.19 & 26168.00 & 252951.44 & 1.00 & 0.97 & 0.97 \\
36034 & 106449 & 2000 & 517.20 & -0.24 & 92749.00 & 520143.65 & 0.56 & 1.01 & 0.56 \\
4213 & 100575 & 2000 & 43.80 & -0.06 & 4818.00 & 44901.28 & 0.91 & 1.03 & 0.93 \\
36044 & 106450 & 2000 & 32.00 & -0.01 & 3280.00 & 30401.09 & 0.98 & 0.95 & 0.93 \\
290 & 100033 & 2000 & 573.50 & -0.07 & 69624.00 & 585548.76 & 0.82 & 1.02 & 0.84 \\
10862 & 101340 & 2000 & 19871.70 & 0.04 & 1791594.00 & 15990825.84 & 1.11 & 0.80 & 0.89 \\
15009 & 101933 & 2000 & 248.10 & -0.08 & 23756.00 & 237576.38 & 1.04 & 0.96 & 1.00 \\
29277 & 105581 & 2000 & 185.50 & 0.19 & 17925.00 & 171466.12 & 1.03 & 0.92 & 0.96 \\
42561 & 108979 & 2000 & 111.20 & -0.00 & 11790.00 & 108289.56 & 0.94 & 0.97 & 0.92 \\
19222 & 102570 & 2000 & 540.80 & -0.18 & 54077.00 & 467482.38 & 1.00 & 0.86 & 0.86 \\
42542 & 108977 & 2000 & 10.40 & -0.13 & 1038.00 & 10305.22 & 1.00 & 0.99 & 0.99 \\
29264 & 105574 & 2000 & 56.30 & 0.08 & 5643.00 & 53604.22 & 1.00 & 0.95 & 0.95 \\
35957 & 106442 & 2000 & 3743.40 & -0.09 & 412680.00 & 3642516.46 & 0.91 & 0.97 & 0.88 \\
55032 & 400049 & 2000 & 41.10 & -0.38 & 7766.00 & 38531.93 & 0.53 & 0.94 & 0.50 \\
33082 & 106090 & 2000 & 55.70 & 0.15 & 5584.00 & 52950.34 & 1.00 & 0.95 & 0.95 \\
52340 & 302780 & 2000 & 119.70 & 0.12 & 11931.00 & 100219.25 & 1.00 & 0.84 & 0.84 \\
16362 & 102130 & 2000 & 670.20 & -0.31 & 67665.00 & 638614.53 & 0.99 & 0.95 & 0.94 \\
35933 & 106441 & 2000 & 78.90 & 0.54 & 4915.00 & 43182.08 & 1.61 & 0.55 & 0.88 \\
21311 & 102847 & 2000 & 35.70 & -0.03 & 3658.00 & 32573.70 & 0.98 & 0.91 & 0.89 \\
36049 & 106451 & 2000 & 440.30 & -0.11 & 42828.00 & 417055.28 & 1.03 & 0.95 & 0.97 \\
16108 & 102080 & 2000 & 1425.60 & 0.15 & 140960.00 & 1195102.49 & 1.01 & 0.84 & 0.85 \\
19878 & 102654 & 2000 & 1518.80 & -0.11 & 154187.00 & 1443897.13 & 0.99 & 0.95 & 0.94 \\
36098 & 106463 & 2000 & 1.20 & 0.01 & 104.00 & 956.53 & 1.15 & 0.80 & 0.92 \\
47686 & 220681 & 2000 & 1451.10 & -0.08 & 163111.00 & 1287116.17 & 0.89 & 0.89 & 0.79 \\
32638 & 106044 & 2000 & 12.60 & 0.00 & 1356.00 & 11374.20 & 0.93 & 0.90 & 0.84 \\
36103 & 106464 & 2000 & 521.10 & -0.17 & 62833.00 & 479781.82 & 0.83 & 0.92 & 0.76 \\
42530 & 108976 & 2000 & 8.70 & -0.09 & 859.00 & 8172.14 & 1.01 & 0.94 & 0.95 \\
4227 & 100590 & 2000 & 385.90 & -0.00 & 38612.00 & 368514.65 & 1.00 & 0.95 & 0.95 \\
43672 & 109189 & 2000 & 125.50 & 0.15 & 6822.00 & 67271.04 & 1.84 & 0.54 & 0.99 \\
42527 & 108975 & 2000 & 61.90 & -0.33 & 6400.00 & 57223.05 & 0.97 & 0.92 & 0.89 \\
32643 & 106045 & 2000 & 38.10 & -0.16 & 4087.00 & 40872.25 & 0.93 & 1.07 & 1.00 \\
42511 & 108973 & 2000 & 104.30 & -0.04 & 10405.00 & 96731.68 & 1.00 & 0.93 & 0.93 \\
29067 & 105523 & 2000 & 156.50 & -0.06 & 15546.00 & 154457.56 & 1.01 & 0.99 & 0.99 \\
47380 & 210681 & 2000 & 40284.80 & -0.07 & 4463670.00 & 34319179.31 & 0.90 & 0.85 & 0.77 \\
74816 & 601183 & 2000 & 19.80 & 0.00 & 1688.00 & 15053.90 & 1.17 & 0.76 & 0.89 \\
29077 & 105525 & 2000 & 276.90 & 0.19 & 28253.00 & 259682.35 & 0.98 & 0.94 & 0.92 \\
29105 & 105527 & 2000 & 5.30 & 0.15 & 478.00 & 4830.02 & 1.11 & 0.91 & 1.01 \\
33019 & 106086 & 2000 & 56.40 & -0.21 & 8529.00 & 58921.77 & 0.66 & 1.04 & 0.69 \\
21330 & 102852 & 2000 & 3868.70 & -0.25 & 553053.00 & 3389219.28 & 0.70 & 0.88 & 0.61 \\
5784 & 100792 & 2000 & 366.60 & -0.03 & 47769.00 & 458026.72 & 0.77 & 1.25 & 0.96 \\
29148 & 105533 & 2000 & 598.60 & -0.13 & 59857.00 & 522718.55 & 1.00 & 0.87 & 0.87 \\
33046 & 106088 & 2000 & 42.40 & 0.25 & 4148.00 & 41482.15 & 1.02 & 0.98 & 1.00 \\
47675 & 217585 & 2000 & 344.70 & -0.09 & 38147.00 & 356482.44 & 0.90 & 1.03 & 0.93 \\
36075 & 106458 & 2000 & 32.20 & -0.19 & 3231.00 & 32298.60 & 1.00 & 1.00 & 1.00 \\
16397 & 102132 & 2000 & 69.30 & -0.23 & 6891.00 & 67511.21 & 1.01 & 0.97 & 0.98 \\
29128 & 105531 & 2000 & 158.30 & -0.07 & 16889.00 & 151611.89 & 0.94 & 0.96 & 0.90 \\
9461 & 101137 & 2000 & 50.70 & 0.10 & 5253.00 & 41222.33 & 0.97 & 0.81 & 0.78 \\
36079 & 106459 & 2000 & 2.00 & 0.20 & 125.00 & 1256.19 & 1.60 & 0.63 & 1.00 \\
48670 & 240117 & 2000 & 406.00 & 0.26 & 36753.00 & 344829.55 & 1.10 & 0.85 & 0.94 \\
16405 & 102133 & 2000 & 57.60 & -0.14 & 5788.00 & 54359.44 & 1.00 & 0.94 & 0.94 \\
46341 & 200224 & 2000 & 41.10 & 0.14 & 3806.00 & 40105.20 & 1.08 & 0.98 & 1.05 \\
35923 & 106434 & 2000 & 842.90 & 0.05 & 84280.00 & 767483.72 & 1.00 & 0.91 & 0.91 \\
48374 & 240067 & 2000 & 505.50 & 0.38 & 50372.00 & 487714.97 & 1.00 & 0.96 & 0.97 \\
42433 & 108966 & 2000 & 874.20 & -0.13 & 87660.00 & 876600.74 & 1.00 & 1.00 & 1.00 \\
28539 & 105437 & 2000 & 2376.00 & 0.06 & 231613.00 & 2261003.67 & 1.03 & 0.95 & 0.98 \\
14629 & 101903 & 2000 & 264.60 & -0.15 & 36773.00 & 254636.61 & 0.72 & 0.96 & 0.69 \\
6835 & 100962 & 2000 & 4487.90 & 0.09 & 407925.00 & 3663962.22 & 1.10 & 0.82 & 0.90 \\
16615 & 102163 & 2000 & 328.60 & -0.11 & 41505.00 & 384985.93 & 0.79 & 1.17 & 0.93 \\
11122 & 101368 & 2000 & 1157.90 & -0.08 & 116017.00 & 1104347.12 & 1.00 & 0.95 & 0.95 \\
21596 & 102895 & 2000 & 1453.30 & -0.11 & 172124.00 & 1553778.20 & 0.84 & 1.07 & 0.90 \\
57740 & 401015 & 2000 & 8359.30 & 0.03 & 1621097.00 & 15054859.24 & 0.52 & 1.80 & 0.93 \\
28532 & 105436 & 2000 & 17.30 & -0.11 & 1733.00 & 16756.78 & 1.00 & 0.97 & 0.97 \\
417 & 100055 & 2000 & 11905.00 & 0.15 & 1153451.00 & 10954708.18 & 1.03 & 0.92 & 0.95 \\
36535 & 106560 & 2000 & 118.60 & -0.08 & 12077.00 & 116233.78 & 0.98 & 0.98 & 0.96 \\
32855 & 106069 & 2000 & 56.30 & -0.04 & 5600.00 & 51509.28 & 1.01 & 0.91 & 0.92 \\
16627 & 102166 & 2000 & 97.30 & -0.25 & 13091.00 & 101873.18 & 0.74 & 1.05 & 0.78 \\
43572 & 109147 & 2000 & 42.40 & -0.58 & 3931.00 & 39313.85 & 1.08 & 0.93 & 1.00 \\
36527 & 106556 & 2000 & 110.10 & -0.40 & 13036.00 & 100919.92 & 0.84 & 0.92 & 0.77 \\
14639 & 101904 & 2000 & 32.50 & -0.19 & 4509.00 & 30944.36 & 0.72 & 0.95 & 0.69 \\
9387 & 101133 & 2000 & 1600.40 & 0.27 & 159168.00 & 1433836.38 & 1.01 & 0.90 & 0.90 \\
28585 & 105448 & 2000 & 60.90 & 0.29 & 5552.00 & 54045.36 & 1.10 & 0.89 & 0.97 \\
9350 & 101132 & 2000 & 497.50 & 0.58 & 49874.00 & 466691.27 & 1.00 & 0.94 & 0.94 \\
3141 & 100411 & 2000 & 5771.40 & 0.07 & 592857.00 & 5386457.90 & 0.97 & 0.93 & 0.91 \\
36530 & 106557 & 2000 & 202.30 & -0.41 & 22712.00 & 196840.01 & 0.89 & 0.97 & 0.87 \\
28569 & 105444 & 2000 & 4.40 & 0.20 & 491.00 & 4468.17 & 0.90 & 1.02 & 0.91 \\
15062 & 101955 & 2000 & 17054.80 & -0.08 & 1771500.00 & 16965140.18 & 0.96 & 0.99 & 0.96 \\
42333 & 108952 & 2000 & 475.90 & 0.75 & 43239.00 & 439973.94 & 1.10 & 0.92 & 1.02 \\
42358 & 108953 & 2000 & 415.70 & -0.32 & 56695.00 & 369107.76 & 0.73 & 0.89 & 0.65 \\
28517 & 105432 & 2000 & 11.50 & 0.03 & 1150.00 & 11231.68 & 1.00 & 0.98 & 0.98 \\
36554 & 106561 & 2000 & 2.90 & 0.27 & 274.00 & 2742.78 & 1.06 & 0.95 & 1.00 \\
21637 & 102937 & 2000 & 55.90 & 0.31 & 4698.00 & 58596.77 & 1.19 & 1.05 & 1.25 \\
36623 & 106570 & 2000 & 62.30 & -0.33 & 5598.00 & 61545.96 & 1.11 & 0.99 & 1.10 \\
5678 & 100785 & 2000 & 1763.20 & -0.15 & 179977.00 & 1816134.63 & 0.98 & 1.03 & 1.01 \\
19765 & 102651 & 2000 & 4044.90 & -0.09 & 450593.00 & 4081407.52 & 0.90 & 1.01 & 0.91 \\
11158 & 101369 & 2000 & 3835.40 & -0.11 & 382952.00 & 3779306.99 & 1.00 & 0.99 & 0.99 \\
32850 & 106068 & 2000 & 24.60 & 0.01 & 2362.00 & 25291.48 & 1.04 & 1.03 & 1.07 \\
36618 & 106569 & 2000 & 74.50 & -0.25 & 6743.00 & 72067.85 & 1.10 & 0.97 & 1.07 \\
16658 & 102175 & 2000 & 792.60 & -0.11 & 98415.00 & 797742.81 & 0.81 & 1.01 & 0.81 \\
36627 & 106571 & 2000 & 27.40 & 0.03 & 2737.00 & 22190.14 & 1.00 & 0.81 & 0.81 \\
28430 & 105424 & 2000 & 4453.30 & -0.19 & 445984.00 & 4333411.70 & 1.00 & 0.97 & 0.97 \\
32836 & 106067 & 2000 & 851.90 & 0.20 & 53446.00 & 499625.18 & 1.59 & 0.59 & 0.93 \\
42292 & 108950 & 2000 & 28.80 & 0.12 & 2878.00 & 25044.05 & 1.00 & 0.87 & 0.87 \\
36653 & 106573 & 2000 & 10.30 & 0.07 & 1064.00 & 10848.93 & 0.97 & 1.05 & 1.02 \\
43623 & 109158 & 2000 & 33.80 & -0.07 & 2050.00 & 18678.07 & 1.65 & 0.55 & 0.91 \\
21610 & 102897 & 2000 & 218.90 & -0.07 & 22231.00 & 212893.29 & 0.98 & 0.97 & 0.96 \\
16648 & 102173 & 2000 & 58.80 & -0.08 & 5875.00 & 55358.33 & 1.00 & 0.94 & 0.94 \\
32800 & 106064 & 2000 & 242.80 & -0.17 & 31688.00 & 264777.13 & 0.77 & 1.09 & 0.84 \\
21622 & 102901 & 2000 & 121.90 & -0.16 & 15658.00 & 120292.24 & 0.78 & 0.99 & 0.77 \\
28488 & 105427 & 2000 & 264.30 & 0.06 & 27097.00 & 238352.80 & 0.98 & 0.90 & 0.88 \\
36575 & 106563 & 2000 & 3.60 & 0.10 & 362.00 & 3600.07 & 0.99 & 1.00 & 0.99 \\
36578 & 106566 & 2000 & 2.90 & -0.17 & 445.00 & 2873.03 & 0.65 & 0.99 & 0.65 \\
36585 & 106567 & 2000 & 3.30 & -0.08 & 316.00 & 3158.49 & 1.04 & 0.96 & 1.00 \\
32808 & 106066 & 2000 & 839.90 & -0.01 & 104558.00 & 723181.52 & 0.80 & 0.86 & 0.69 \\
36592 & 106568 & 2000 & 4.40 & 0.35 & 385.00 & 3847.62 & 1.14 & 0.87 & 1.00 \\
19107 & 102549 & 2000 & 229.30 & 0.18 & 23809.00 & 183605.57 & 0.96 & 0.80 & 0.77 \\
28459 & 105426 & 2000 & 915.30 & 0.23 & 93098.00 & 856754.40 & 0.98 & 0.94 & 0.92 \\
74775 & 601168 & 2000 & 78.50 & -0.15 & 7857.00 & 72311.70 & 1.00 & 0.92 & 0.92 \\
13222 & 101708 & 2000 & 487.30 & -0.08 & 48767.00 & 479734.15 & 1.00 & 0.98 & 0.98 \\
19799 & 102652 & 2000 & 2989.20 & 0.09 & 289900.00 & 2478115.26 & 1.03 & 0.83 & 0.85 \\
36509 & 106545 & 2000 & 58.20 & -0.42 & 6005.00 & 60054.87 & 0.97 & 1.03 & 1.00 \\
390 & 100048 & 2000 & 521.30 & -0.12 & 52057.00 & 492592.55 & 1.00 & 0.94 & 0.95 \\
32739 & 106057 & 2000 & 406.70 & -0.15 & 40671.00 & 359739.43 & 1.00 & 0.88 & 0.88 \\
36413 & 106524 & 2000 & 9.20 & 0.13 & 931.00 & 9421.92 & 0.99 & 1.02 & 1.01 \\
46333 & 200223 & 2000 & 10.60 & -0.35 & 1000.00 & 10003.92 & 1.06 & 0.94 & 1.00 \\
46330 & 200221 & 2000 & 4.00 & -0.14 & 402.00 & 3882.71 & 1.00 & 0.97 & 0.97 \\
42416 & 108964 & 2000 & 913.30 & -0.06 & 91064.00 & 894169.61 & 1.00 & 0.98 & 0.98 \\
28747 & 105475 & 2000 & 510.90 & 0.14 & 51103.00 & 502346.66 & 1.00 & 0.98 & 0.98 \\
42429 & 108965 & 2000 & 101.90 & -0.05 & 12586.00 & 82454.95 & 0.81 & 0.81 & 0.66 \\
36434 & 106527 & 2000 & 15.20 & -0.10 & 1531.00 & 14736.91 & 0.99 & 0.97 & 0.96 \\
36439 & 106528 & 2000 & 906.40 & 0.82 & 90738.00 & 843695.15 & 1.00 & 0.93 & 0.93 \\
28725 & 105472 & 2000 & 359.90 & -0.38 & 33295.00 & 332849.32 & 1.08 & 0.92 & 1.00 \\
3183 & 100413 & 2000 & 71.40 & -0.14 & 7110.00 & 59479.33 & 1.00 & 0.83 & 0.84 \\
16540 & 102154 & 2000 & 218.70 & 0.01 & 21776.00 & 193667.60 & 1.00 & 0.89 & 0.89 \\
28717 & 105471 & 2000 & 4.80 & -0.21 & 457.00 & 4574.01 & 1.05 & 0.95 & 1.00 \\
36393 & 106523 & 2000 & 28.90 & -0.01 & 1815.00 & 15861.13 & 1.59 & 0.55 & 0.87 \\
28776 & 105476 & 2000 & 62.50 & 0.32 & 6215.00 & 60090.39 & 1.01 & 0.96 & 0.97 \\
5752 & 100791 & 2000 & 3701.80 & 0.41 & 333978.00 & 3342485.94 & 1.11 & 0.90 & 1.00 \\
28807 & 105478 & 2000 & 37.20 & 0.21 & 4076.00 & 38620.67 & 0.91 & 1.04 & 0.95 \\
32941 & 106083 & 2000 & 1388.40 & 0.36 & 106698.00 & 1140747.11 & 1.30 & 0.82 & 1.07 \\
28800 & 105477 & 2000 & 6.90 & -0.07 & 673.00 & 6725.80 & 1.03 & 0.97 & 1.00 \\
36364 & 106519 & 2000 & 16.10 & 0.06 & 1612.00 & 15781.15 & 1.00 & 0.98 & 0.98 \\
32714 & 106051 & 2000 & 70.50 & -0.03 & 7049.00 & 68900.90 & 1.00 & 0.98 & 0.98 \\
3972 & 100535 & 2000 & 418.40 & 0.37 & 39259.00 & 370228.05 & 1.07 & 0.88 & 0.94 \\
21453 & 102872 & 2000 & 674.20 & 0.17 & 67572.00 & 601199.67 & 1.00 & 0.89 & 0.89 \\
16517 & 102152 & 2000 & 308.40 & 0.04 & 30870.00 & 287223.48 & 1.00 & 0.93 & 0.93 \\
11024 & 101360 & 2000 & 1908.60 & -0.02 & 191803.00 & 1749527.71 & 1.00 & 0.92 & 0.91 \\
48347 & 240065 & 2000 & 925.40 & 0.03 & 105689.00 & 1013079.01 & 0.88 & 1.09 & 0.96 \\
36390 & 106521 & 2000 & 1.80 & 0.02 & 170.00 & 1569.79 & 1.06 & 0.87 & 0.92 \\
32720 & 106052 & 2000 & 170.10 & -0.08 & 17012.00 & 161190.96 & 1.00 & 0.95 & 0.95 \\
14659 & 101906 & 2000 & 25.40 & -0.02 & 2931.00 & 25622.69 & 0.87 & 1.01 & 0.87 \\
47411 & 210770 & 2000 & 1577.70 & 0.10 & 173822.00 & 1733554.97 & 0.91 & 1.10 & 1.00 \\
21508 & 102876 & 2000 & 29.20 & -0.02 & 4679.00 & 46662.74 & 0.62 & 1.60 & 1.00 \\
28704 & 105469 & 2000 & 7.00 & -0.03 & 862.00 & 6788.36 & 0.81 & 0.97 & 0.79 \\
46321 & 200213 & 2000 & 16.10 & -0.20 & 2447.00 & 13383.49 & 0.66 & 0.83 & 0.55 \\
28650 & 105458 & 2000 & 530.30 & 0.02 & 52540.00 & 490225.94 & 1.01 & 0.92 & 0.93 \\
36469 & 106535 & 2000 & 155.80 & 0.03 & 16283.00 & 148170.33 & 0.96 & 0.95 & 0.91 \\
47436 & 211051 & 2000 & 528.60 & 0.01 & 52132.00 & 486397.27 & 1.01 & 0.92 & 0.93 \\
28635 & 105457 & 2000 & 2256.00 & -0.04 & 225245.00 & 2247942.45 & 1.00 & 1.00 & 1.00 \\
53405 & 346113 & 2000 & 396.40 & -0.21 & 59697.00 & 497697.02 & 0.66 & 1.26 & 0.83 \\
42370 & 108955 & 2000 & 55.90 & -0.38 & 5596.00 & 55921.19 & 1.00 & 1.00 & 1.00 \\
11088 & 101367 & 2000 & 374.40 & 0.11 & 37466.00 & 349672.00 & 1.00 & 0.93 & 0.93 \\
46301 & 200210 & 2000 & 7.00 & -0.04 & 732.00 & 6316.68 & 0.96 & 0.90 & 0.86 \\
36497 & 106541 & 2000 & 37.20 & 0.11 & 3931.00 & 39308.51 & 0.95 & 1.06 & 1.00 \\
48676 & 240118 & 2000 & 53.30 & 0.17 & 5551.00 & 51917.60 & 0.96 & 0.97 & 0.94 \\
32872 & 106075 & 2000 & 504.30 & -0.12 & 55506.00 & 456574.87 & 0.91 & 0.91 & 0.82 \\
55058 & 400050 & 2000 & 196.20 & 0.17 & 22154.00 & 200038.60 & 0.89 & 1.02 & 0.90 \\
36463 & 106533 & 2000 & 19.50 & -0.17 & 3482.00 & 31309.91 & 0.56 & 1.61 & 0.90 \\
36448 & 106529 & 2000 & 152.10 & 0.24 & 15214.00 & 146302.79 & 1.00 & 0.96 & 0.96 \\
7278 & 101018 & 2000 & 26646.70 & -0.11 & 2915437.00 & 24990101.52 & 0.91 & 0.94 & 0.86 \\
28698 & 105465 & 2000 & 57.00 & 0.01 & 8401.00 & 77632.90 & 0.68 & 1.36 & 0.92 \\
11056 & 101364 & 2000 & 93.80 & 0.11 & 9471.00 & 82762.33 & 0.99 & 0.88 & 0.87 \\
74795 & 601172 & 2000 & 2.00 & 0.24 & 172.00 & 1715.03 & 1.16 & 0.86 & 1.00 \\
21532 & 102893 & 2000 & 39.80 & -0.84 & 3006.00 & 30576.62 & 1.32 & 0.77 & 1.02 \\
52361 & 302813 & 2000 & 15.30 & 0.20 & 2214.00 & 21710.50 & 0.69 & 1.42 & 0.98 \\
28688 & 105464 & 2000 & 18.80 & 0.29 & 1888.00 & 15572.12 & 1.00 & 0.83 & 0.82 \\
5728 & 100790 & 2000 & 213.70 & 0.15 & 19120.00 & 200789.86 & 1.12 & 0.94 & 1.05 \\
19734 & 102650 & 2000 & 10188.50 & 0.13 & 1031434.00 & 9651547.28 & 0.99 & 0.95 & 0.94 \\
28679 & 105463 & 2000 & 3805.80 & -0.29 & 550915.00 & 3358915.13 & 0.69 & 0.88 & 0.61 \\
74792 & 601171 & 2000 & 815.90 & -0.09 & 85032.00 & 801503.27 & 0.96 & 0.98 & 0.94 \\
32754 & 106061 & 2000 & 236.70 & -0.03 & 23522.00 & 235217.23 & 1.01 & 0.99 & 1.00 \\
16561 & 102156 & 2000 & 10.00 & -0.21 & 1085.00 & 8835.98 & 0.92 & 0.88 & 0.81 \\
6593 & 100900 & 2000 & 58.40 & 0.18 & 5063.00 & 41009.39 & 1.15 & 0.70 & 0.81 \\
21277 & 102844 & 2000 & 876.00 & -0.04 & 87663.00 & 861165.44 & 1.00 & 0.98 & 0.98 \\
43476 & 109126 & 2000 & 165.30 & -0.13 & 16203.00 & 162033.23 & 1.02 & 0.98 & 1.00 \\
4176 & 100567 & 2000 & 2476.40 & -0.18 & 247642.00 & 2292997.74 & 1.00 & 0.93 & 0.93 \\
19950 & 102659 & 2000 & 6122.30 & -0.02 & 626194.00 & 5899788.38 & 0.98 & 0.96 & 0.94 \\
51091 & 240482 & 2000 & 1.40 & 0.09 & 105.00 & 1080.00 & 1.33 & 0.77 & 1.03 \\
14998 & 101930 & 2000 & 1464.10 & -0.10 & 142268.00 & 1422645.73 & 1.03 & 0.97 & 1.00 \\
32508 & 106037 & 2000 & 36.40 & 0.02 & 3642.00 & 35634.18 & 1.00 & 0.98 & 0.98 \\
29576 & 105616 & 2000 & 26.80 & -0.14 & 2124.00 & 20716.06 & 1.26 & 0.77 & 0.98 \\
21141 & 102833 & 2000 & 73.50 & -0.36 & 7381.00 & 70496.13 & 1.00 & 0.96 & 0.96 \\
29865 & 105655 & 2000 & 604.80 & 0.30 & 60421.00 & 581895.18 & 1.00 & 0.96 & 0.96 \\
16242 & 102104 & 2000 & 220.20 & 0.25 & 21432.00 & 211266.80 & 1.03 & 0.96 & 0.99 \\
53099 & 338393 & 2000 & 40.20 & 0.05 & 3521.00 & 42131.35 & 1.14 & 1.05 & 1.20 \\
9509 & 101141 & 2000 & 4316.00 & 0.05 & 430094.00 & 3644174.03 & 1.00 & 0.84 & 0.85 \\
19270 & 102578 & 2000 & 135.60 & -0.26 & 13740.00 & 137351.99 & 0.99 & 1.01 & 1.00 \\
21066 & 102827 & 2000 & 133.90 & -0.01 & 13371.00 & 129434.36 & 1.00 & 0.97 & 0.97 \\
47652 & 216504 & 2000 & 106.30 & -0.48 & 13888.00 & 100074.40 & 0.77 & 0.94 & 0.72 \\
32445 & 106028 & 2000 & 268.00 & 0.23 & 26560.00 & 213338.11 & 1.01 & 0.80 & 0.80 \\
10732 & 101320 & 2000 & 70.50 & 0.35 & 7049.00 & 61830.72 & 1.00 & 0.88 & 0.88 \\
35639 & 106381 & 2000 & 5.00 & 0.32 & 471.00 & 4705.78 & 1.06 & 0.94 & 1.00 \\
274 & 100030 & 2000 & 718.40 & -0.19 & 75994.00 & 604198.63 & 0.95 & 0.84 & 0.80 \\
29605 & 105623 & 2000 & 29.40 & 0.08 & 4611.00 & 44515.06 & 0.64 & 1.51 & 0.97 \\
35424 & 106360 & 2000 & 110.90 & 0.27 & 9790.00 & 104161.77 & 1.13 & 0.94 & 1.06 \\
61959 & 500312 & 2000 & 2.50 & 0.33 & 247.00 & 2337.12 & 1.01 & 0.93 & 0.95 \\
29617 & 105627 & 2000 & 250.00 & 0.29 & 25206.00 & 248012.64 & 0.99 & 0.99 & 0.98 \\
54814 & 400015 & 2000 & 11.50 & -0.37 & 1171.00 & 10765.64 & 0.98 & 0.94 & 0.92 \\
15146 & 101963 & 2000 & 932.90 & 0.11 & 94328.00 & 832539.00 & 0.99 & 0.89 & 0.88 \\
74909 & 601197 & 2000 & 52.60 & -0.17 & 4753.00 & 47524.63 & 1.11 & 0.90 & 1.00 \\
35612 & 106380 & 2000 & 61.10 & 0.11 & 6101.00 & 60924.44 & 1.00 & 1.00 & 1.00 \\
35673 & 106386 & 2000 & 1391.60 & -0.11 & 152696.00 & 1526855.12 & 0.91 & 1.10 & 1.00 \\
7463 & 101040 & 2000 & 3393.20 & -0.03 & 363896.00 & 3280948.05 & 0.93 & 0.97 & 0.90 \\
10768 & 101330 & 2000 & 3295.80 & 0.21 & 327242.00 & 3068557.29 & 1.01 & 0.93 & 0.94 \\
15175 & 101964 & 2000 & 581.70 & 0.26 & 56595.00 & 560846.79 & 1.03 & 0.96 & 0.99 \\
47643 & 216438 & 2000 & 664.70 & -0.07 & 66409.00 & 647998.81 & 1.00 & 0.97 & 0.98 \\
35703 & 106391 & 2000 & 85.10 & -0.08 & 8590.00 & 83280.34 & 0.99 & 0.98 & 0.97 \\
47660 & 216749 & 2000 & 36.90 & -0.05 & 3957.00 & 35333.25 & 0.93 & 0.96 & 0.89 \\
29513 & 105603 & 2000 & 2.40 & -0.24 & 342.00 & 2516.75 & 0.70 & 1.05 & 0.74 \\
35409 & 106358 & 2000 & 8.20 & 0.18 & 854.00 & 7799.57 & 0.96 & 0.95 & 0.91 \\
5871 & 100809 & 2000 & 2241.20 & -0.14 & 223500.00 & 2085577.14 & 1.00 & 0.93 & 0.93 \\
35488 & 106365 & 2000 & 15.00 & 0.21 & 1529.00 & 14727.49 & 0.98 & 0.98 & 0.96 \\
21163 & 102835 & 2000 & 167.30 & 0.11 & 15567.00 & 132869.24 & 1.07 & 0.79 & 0.85 \\
35416 & 106359 & 2000 & 416.80 & 0.00 & 41666.00 & 414550.19 & 1.00 & 0.99 & 0.99 \\
6488 & 100876 & 2000 & 5.10 & 0.04 & 313.00 & 5063.30 & 1.63 & 0.99 & 1.62 \\
13330 & 101728 & 2000 & 146.30 & -0.26 & 14534.00 & 142401.52 & 1.01 & 0.97 & 0.98 \\
16151 & 102087 & 2000 & 838.90 & 0.39 & 84043.00 & 750325.91 & 1.00 & 0.89 & 0.89 \\
29524 & 105606 & 2000 & 11.20 & -0.45 & 1344.00 & 13311.01 & 0.83 & 1.19 & 0.99 \\
32517 & 106038 & 2000 & 282.30 & 0.23 & 28188.00 & 256752.51 & 1.00 & 0.91 & 0.91 \\
16257 & 102105 & 2000 & 120.00 & 0.13 & 11748.00 & 117280.93 & 1.02 & 0.98 & 1.00 \\
29528 & 105607 & 2000 & 11.40 & -0.56 & 1155.00 & 10690.65 & 0.99 & 0.94 & 0.93 \\
52414 & 302907 & 2000 & 72.20 & 0.38 & 6747.00 & 67470.04 & 1.07 & 0.93 & 1.00 \\
29535 & 105610 & 2000 & 581.70 & -0.11 & 55331.00 & 553232.87 & 1.05 & 0.95 & 1.00 \\
48256 & 240057 & 2000 & 225.40 & 0.68 & 22547.00 & 203591.08 & 1.00 & 0.90 & 0.90 \\
29547 & 105611 & 2000 & 51.10 & 0.27 & 5046.00 & 46175.31 & 1.01 & 0.90 & 0.92 \\
48405 & 240076 & 2000 & 122.20 & -0.24 & 12229.00 & 121287.01 & 1.00 & 0.99 & 0.99 \\
29894 & 105656 & 2000 & 139.20 & -0.01 & 13916.00 & 130438.06 & 1.00 & 0.94 & 0.94 \\
42707 & 109006 & 2000 & 91.70 & -0.21 & 9171.00 & 90172.78 & 1.00 & 0.98 & 0.98 \\
29848 & 105654 & 2000 & 145.00 & -0.12 & 14526.00 & 144300.40 & 1.00 & 1.00 & 0.99 \\
29822 & 105652 & 2000 & 703.10 & -0.15 & 69593.00 & 695532.92 & 1.01 & 0.99 & 1.00 \\
61934 & 500310 & 2000 & 8.10 & -0.06 & 866.00 & 6679.21 & 0.94 & 0.82 & 0.77 \\
35519 & 106370 & 2000 & 61.80 & -0.27 & 6098.00 & 52102.70 & 1.01 & 0.84 & 0.85 \\
16201 & 102090 & 2000 & 5703.90 & -0.07 & 556625.00 & 4940189.68 & 1.02 & 0.87 & 0.89 \\
29732 & 105643 & 2000 & 1050.20 & 0.08 & 105132.00 & 1004769.23 & 1.00 & 0.96 & 0.96 \\
35473 & 106363 & 2000 & 500.60 & -0.19 & 53036.00 & 490777.14 & 0.94 & 0.98 & 0.93 \\
21098 & 102829 & 2000 & 74.20 & 0.06 & 7999.00 & 70550.44 & 0.93 & 0.95 & 0.88 \\
47336 & 210203 & 2000 & 7786.00 & -0.17 & 760564.00 & 7605522.56 & 1.02 & 0.98 & 1.00 \\
35511 & 106369 & 2000 & 272.10 & -0.45 & 41859.00 & 277388.26 & 0.65 & 1.02 & 0.66 \\
35481 & 106364 & 2000 & 14.60 & 0.44 & 1464.00 & 13171.91 & 1.00 & 0.90 & 0.90 \\
10671 & 101307 & 2000 & 464.80 & 0.30 & 46471.00 & 411291.94 & 1.00 & 0.88 & 0.89 \\
32473 & 106033 & 2000 & 621.00 & 0.33 & 59737.00 & 596958.32 & 1.04 & 0.96 & 1.00 \\
29754 & 105644 & 2000 & 147.90 & 0.08 & 14863.00 & 140048.32 & 1.00 & 0.95 & 0.94 \\
13353 & 101729 & 2000 & 270.40 & -0.23 & 26113.00 & 261131.00 & 1.04 & 0.97 & 1.00 \\
52385 & 302826 & 2000 & 170.80 & -0.14 & 17514.00 & 163034.51 & 0.98 & 0.95 & 0.93 \\
29764 & 105645 & 2000 & 5056.20 & 0.18 & 585270.00 & 4277433.02 & 0.86 & 0.85 & 0.73 \\
54824 & 400017 & 2000 & 98.60 & 0.03 & 7696.00 & 67485.76 & 1.28 & 0.68 & 0.88 \\
35499 & 106367 & 2000 & 116.30 & -0.09 & 11647.00 & 115093.36 & 1.00 & 0.99 & 0.99 \\
21089 & 102828 & 2000 & 231.40 & -0.05 & 23596.00 & 224753.11 & 0.98 & 0.97 & 0.95 \\
46515 & 200250 & 2000 & 67.10 & -0.06 & 6727.00 & 66336.83 & 1.00 & 0.99 & 0.99 \\
263 & 100022 & 2000 & 24.00 & -0.16 & 2408.00 & 21173.79 & 1.00 & 0.88 & 0.88 \\
29646 & 105630 & 2000 & 12.30 & -0.14 & 1478.00 & 10390.35 & 0.83 & 0.84 & 0.70 \\
10703 & 101312 & 2000 & 12850.80 & -0.10 & 1285077.00 & 11455667.36 & 1.00 & 0.89 & 0.89 \\
5903 & 100811 & 2000 & 1381.40 & -0.04 & 135754.00 & 1345040.00 & 1.02 & 0.97 & 0.99 \\
29842 & 105653 & 2000 & 6.10 & -0.04 & 614.00 & 6037.87 & 0.99 & 0.99 & 0.98 \\
29658 & 105631 & 2000 & 11.50 & -0.24 & 1394.00 & 13328.07 & 0.82 & 1.16 & 0.96 \\
29668 & 105632 & 2000 & 67.60 & -0.01 & 7404.00 & 72326.18 & 0.91 & 1.07 & 0.98 \\
19282 & 102579 & 2000 & 872.90 & -0.25 & 87153.00 & 770212.82 & 1.00 & 0.88 & 0.88 \\
6769 & 100953 & 2000 & 21.00 & 0.22 & 2139.00 & 21264.60 & 0.98 & 1.01 & 0.99 \\
48337 & 240063 & 2000 & 2090.80 & -0.39 & 354312.00 & 1831412.49 & 0.59 & 0.88 & 0.52 \\
52440 & 302941 & 2000 & 17.70 & -0.14 & 1925.00 & 17759.88 & 0.92 & 1.00 & 0.92 \\
35579 & 106376 & 2000 & 13.60 & 0.07 & 1461.00 & 13261.46 & 0.93 & 0.98 & 0.91 \\
35451 & 106361 & 2000 & 186.10 & 0.12 & 18031.00 & 180316.87 & 1.03 & 0.97 & 1.00 \\
29682 & 105635 & 2000 & 262.90 & -0.00 & 26243.00 & 253752.19 & 1.00 & 0.97 & 0.97 \\
21109 & 102832 & 2000 & 397.70 & -0.32 & 40037.00 & 389663.51 & 0.99 & 0.98 & 0.97 \\
14741 & 101912 & 2000 & 3756.90 & 0.21 & 371758.00 & 3305355.35 & 1.01 & 0.88 & 0.89 \\
35554 & 106375 & 2000 & 37.90 & -0.12 & 3784.00 & 35341.82 & 1.00 & 0.93 & 0.93 \\
33237 & 106107 & 2000 & 57.70 & -0.07 & 5774.00 & 51162.99 & 1.00 & 0.89 & 0.89 \\
16170 & 102089 & 2000 & 171.30 & 0.11 & 17044.00 & 140923.43 & 1.01 & 0.82 & 0.83 \\
10641 & 101302 & 2000 & 318.90 & 0.19 & 31887.00 & 306725.43 & 1.00 & 0.96 & 0.96 \\
42653 & 108992 & 2000 & 2.60 & -0.04 & 333.00 & 3116.13 & 0.78 & 1.20 & 0.94 \\
29793 & 105647 & 2000 & 243.40 & 0.38 & 24339.00 & 211042.42 & 1.00 & 0.87 & 0.87 \\
10800 & 101331 & 2000 & 113.40 & -0.08 & 11329.00 & 108667.77 & 1.00 & 0.96 & 0.96 \\
19922 & 102655 & 2000 & 1194.60 & 0.20 & 117283.00 & 1108983.54 & 1.02 & 0.93 & 0.95 \\
35830 & 106415 & 2000 & 638.30 & 0.86 & 63888.00 & 524776.18 & 1.00 & 0.82 & 0.82 \\
230 & 100019 & 2000 & 3986.30 & 0.10 & 402621.00 & 3506235.58 & 0.99 & 0.88 & 0.87 \\
19980 & 102660 & 2000 & 7118.70 & -0.02 & 725992.00 & 7019708.55 & 0.98 & 0.99 & 0.97 \\
32561 & 106041 & 2000 & 76.50 & 0.07 & 7869.00 & 67425.61 & 0.97 & 0.88 & 0.86 \\
29399 & 105592 & 2000 & 509.50 & -0.10 & 50990.00 & 506909.21 & 1.00 & 0.99 & 0.99 \\
3918 & 100514 & 2000 & 77.20 & -0.09 & 8542.00 & 80189.38 & 0.90 & 1.04 & 0.94 \\
52408 & 302881 & 2000 & 171.70 & -0.11 & 17090.00 & 169962.63 & 1.00 & 0.99 & 0.99 \\
35345 & 106352 & 2000 & 41.20 & -0.14 & 4030.00 & 40988.04 & 1.02 & 0.99 & 1.02 \\
32589 & 106042 & 2000 & 6.80 & 0.17 & 536.00 & 5277.28 & 1.27 & 0.78 & 0.98 \\
29989 & 105665 & 2000 & 112.50 & -0.11 & 12800.00 & 114202.87 & 0.88 & 1.02 & 0.89 \\
42631 & 108988 & 2000 & 687.90 & -0.03 & 78255.00 & 713085.91 & 0.88 & 1.04 & 0.91 \\
9280 & 101127 & 2000 & 172.10 & -0.23 & 25193.00 & 165871.80 & 0.68 & 0.96 & 0.66 \\
16308 & 102124 & 2000 & 2033.50 & -0.03 & 203287.00 & 1991950.55 & 1.00 & 0.98 & 0.98 \\
9545 & 101149 & 2000 & 6211.20 & -0.23 & 619150.00 & 5371366.46 & 1.00 & 0.86 & 0.87 \\
29406 & 105593 & 2000 & 50.00 & 0.14 & 5040.00 & 44553.63 & 0.99 & 0.89 & 0.88 \\
29303 & 105585 & 2000 & 12.80 & 0.10 & 1137.00 & 10263.61 & 1.13 & 0.80 & 0.90 \\
42742 & 109011 & 2000 & 70.60 & 0.24 & 7076.00 & 66658.76 & 1.00 & 0.94 & 0.94 \\
10830 & 101334 & 2000 & 628.50 & -0.11 & 63180.00 & 503550.40 & 0.99 & 0.80 & 0.80 \\
6505 & 100878 & 2000 & 2793.20 & 0.06 & 279122.00 & 2695988.64 & 1.00 & 0.97 & 0.97 \\
29928 & 105658 & 2000 & 197.10 & 0.15 & 22518.00 & 205150.42 & 0.88 & 1.04 & 0.91 \\
10606 & 101300 & 2000 & 1723.30 & 0.41 & 172334.00 & 1388070.15 & 1.00 & 0.81 & 0.81 \\
35897 & 106424 & 2000 & 157.80 & 0.26 & 15592.00 & 141570.54 & 1.01 & 0.90 & 0.91 \\
35873 & 106421 & 2000 & 21.80 & 0.29 & 2176.00 & 21306.85 & 1.00 & 0.98 & 0.98 \\
42622 & 108987 & 2000 & 140.20 & -0.08 & 13815.00 & 122692.94 & 1.01 & 0.88 & 0.89 \\
42714 & 109009 & 2000 & 87.30 & -0.14 & 10017.00 & 81173.00 & 0.87 & 0.93 & 0.81 \\
35865 & 106420 & 2000 & 50.90 & 0.09 & 5715.00 & 43199.55 & 0.89 & 0.85 & 0.76 \\
33268 & 106109 & 2000 & 23.90 & 0.10 & 2391.00 & 21052.72 & 1.00 & 0.88 & 0.88 \\
35860 & 106419 & 2000 & 2165.30 & -0.11 & 216504.00 & 2087302.85 & 1.00 & 0.96 & 0.96 \\
29355 & 105588 & 2000 & 12.50 & -0.06 & 1238.00 & 10474.61 & 1.01 & 0.84 & 0.85 \\
43423 & 109117 & 2000 & 500.90 & -0.01 & 51534.00 & 450371.90 & 0.97 & 0.90 & 0.87 \\
29361 & 105589 & 2000 & 370.20 & -0.21 & 39192.00 & 342110.70 & 0.94 & 0.92 & 0.87 \\
29971 & 105664 & 2000 & 172.50 & -0.23 & 17295.00 & 153953.78 & 1.00 & 0.89 & 0.89 \\
33096 & 106091 & 2000 & 110.60 & -0.14 & 11186.00 & 110712.06 & 0.99 & 1.00 & 0.99 \\
29329 & 105587 & 2000 & 19.90 & 0.07 & 2007.00 & 19119.53 & 0.99 & 0.96 & 0.95 \\
7348 & 101023 & 2000 & 22352.90 & 0.09 & 2045409.00 & 20361732.09 & 1.09 & 0.91 & 1.00 \\
61978 & 500315 & 2000 & 21.40 & 0.17 & 1953.00 & 18895.38 & 1.10 & 0.88 & 0.97 \\
29955 & 105662 & 2000 & 131.30 & -0.02 & 12792.00 & 127921.05 & 1.03 & 0.97 & 1.00 \\
21253 & 102843 & 2000 & 444.90 & 0.07 & 44054.00 & 427716.31 & 1.01 & 0.96 & 0.97 \\
35846 & 106418 & 2000 & 783.70 & -0.08 & 79434.00 & 750342.77 & 0.99 & 0.96 & 0.94 \\
29943 & 105659 & 2000 & 244.60 & -0.06 & 24066.00 & 240626.04 & 1.02 & 0.98 & 1.00 \\
33123 & 106092 & 2000 & 858.60 & 0.07 & 85855.00 & 864696.97 & 1.00 & 1.01 & 1.01 \\
61924 & 500308 & 2000 & 182.90 & -0.10 & 19832.00 & 181243.17 & 0.92 & 0.99 & 0.91 \\
35842 & 106417 & 2000 & 89.60 & -0.10 & 11277.00 & 101187.14 & 0.79 & 1.13 & 0.90 \\
42597 & 108985 & 2000 & 1112.10 & -0.09 & 113103.00 & 956917.44 & 0.98 & 0.86 & 0.85 \\
8679 & 101094 & 2000 & 1042.20 & -0.20 & 150234.00 & 868595.64 & 0.69 & 0.83 & 0.58 \\
32545 & 106039 & 2000 & 164.10 & 0.21 & 16406.00 & 147653.82 & 1.00 & 0.90 & 0.90 \\
9484 & 101139 & 2000 & 36.10 & 0.09 & 3587.00 & 35151.32 & 1.01 & 0.97 & 0.98 \\
35746 & 106398 & 2000 & 5.50 & 0.12 & 549.00 & 5444.14 & 1.00 & 0.99 & 0.99 \\
35915 & 106428 & 2000 & 336.00 & -0.15 & 33366.00 & 290109.19 & 1.01 & 0.86 & 0.87 \\
14707 & 101911 & 2000 & 1477.20 & 0.13 & 150139.00 & 1342790.40 & 0.98 & 0.91 & 0.89 \\
61974 & 500313 & 2000 & 25.80 & -0.23 & 2222.00 & 19906.50 & 1.16 & 0.77 & 0.90 \\
35383 & 106356 & 2000 & 13.20 & 0.16 & 1585.00 & 14327.95 & 0.83 & 1.09 & 0.90 \\
29487 & 105598 & 2000 & 313.00 & -0.07 & 32147.00 & 308395.41 & 0.97 & 0.99 & 0.96 \\
9496 & 101140 & 2000 & 1628.20 & -0.24 & 171807.00 & 1717120.67 & 0.95 & 1.05 & 1.00 \\
15217 & 101968 & 2000 & 287.80 & -0.22 & 28777.00 & 279761.45 & 1.00 & 0.97 & 0.97 \\
33146 & 106097 & 2000 & 64.50 & -0.09 & 6466.00 & 63097.27 & 1.00 & 0.98 & 0.98 \\
10567 & 101299 & 2000 & 2605.10 & 0.07 & 260522.00 & 2303110.89 & 1.00 & 0.88 & 0.88 \\
35735 & 106394 & 2000 & 95.60 & 0.21 & 9561.00 & 94401.32 & 1.00 & 0.99 & 0.99 \\
9218 & 101119 & 2000 & 50.10 & 0.19 & 5006.00 & 47093.11 & 1.00 & 0.94 & 0.94 \\
35919 & 106429 & 2000 & 42.80 & 0.04 & 4121.00 & 35772.39 & 1.04 & 0.84 & 0.87 \\
61929 & 500309 & 2000 & 85.60 & 0.12 & 9092.00 & 88049.38 & 0.94 & 1.03 & 0.97 \\
16142 & 102085 & 2000 & 2122.10 & -0.19 & 212893.00 & 1962659.99 & 1.00 & 0.92 & 0.92 \\
42587 & 108984 & 2000 & 33.80 & 0.16 & 3512.00 & 32056.38 & 0.96 & 0.95 & 0.91 \\
35730 & 106392 & 2000 & 168.00 & -0.04 & 19353.00 & 162728.53 & 0.87 & 0.97 & 0.84 \\
16273 & 102113 & 2000 & 267.30 & -0.15 & 26637.00 & 266317.31 & 1.00 & 1.00 & 1.00 \\
3870 & 100507 & 2000 & 13.50 & -0.00 & 1571.00 & 14931.92 & 0.86 & 1.11 & 0.95 \\
5942 & 100812 & 2000 & 377.50 & -0.17 & 37386.00 & 360896.26 & 1.01 & 0.96 & 0.97 \\
54788 & 400014 & 2000 & 253.60 & 0.73 & 25409.00 & 231152.01 & 1.00 & 0.91 & 0.91 \\
19254 & 102575 & 2000 & 257.70 & -0.17 & 24322.00 & 243259.81 & 1.06 & 0.94 & 1.00 \\
3878 & 100508 & 2000 & 5.80 & 0.07 & 574.00 & 5559.07 & 1.01 & 0.96 & 0.97 \\
19700 & 102649 & 2000 & 750.90 & 0.10 & 73403.00 & 658360.38 & 1.02 & 0.88 & 0.90 \\
13774 & 101763 & 2000 & 55.10 & 0.13 & 5542.00 & 54381.05 & 0.99 & 0.99 & 0.98 \\
21195 & 102837 & 2000 & 1245.30 & 0.32 & 124068.00 & 1174033.86 & 1.00 & 0.94 & 0.95 \\
29439 & 105595 & 2000 & 29.90 & -0.06 & 2981.00 & 27800.08 & 1.00 & 0.93 & 0.93 \\
7494 & 101042 & 2000 & 11781.40 & 0.12 & 1046083.00 & 10742279.39 & 1.13 & 0.91 & 1.03 \\
30001 & 105676 & 2000 & 199.00 & 0.04 & 24956.00 & 220394.21 & 0.80 & 1.11 & 0.88 \\
7314 & 101020 & 2000 & 2707.30 & -0.18 & 302675.00 & 2539155.80 & 0.89 & 0.94 & 0.84 \\
33247 & 106108 & 2000 & 46.90 & 0.17 & 4641.00 & 44477.33 & 1.01 & 0.95 & 0.96 \\
35349 & 106353 & 2000 & 41.90 & 0.19 & 3842.00 & 38419.25 & 1.09 & 0.92 & 1.00 \\
35754 & 106400 & 2000 & 361.00 & -0.09 & 36647.00 & 338212.12 & 0.99 & 0.94 & 0.92 \\
29461 & 105597 & 2000 & 131.50 & -0.03 & 13150.00 & 127037.29 & 1.00 & 0.97 & 0.97 \\
5824 & 100804 & 2000 & 3853.80 & 0.02 & 390298.00 & 3553483.59 & 0.99 & 0.92 & 0.91 \\
35804 & 106413 & 2000 & 772.60 & 0.27 & 77243.00 & 723606.96 & 1.00 & 0.94 & 0.94 \\
2168 & 100293 & 2001 & 213.80 & -0.20 & 21381.00 & 211758.40 & 1.00 & 0.99 & 0.99 \\
61317 & 500028 & 2001 & 21.80 & 0.01 & 3283.00 & 27244.87 & 0.66 & 1.25 & 0.83 \\
63948 & 500571 & 2001 & 164.90 & -0.31 & 19536.00 & 168186.75 & 0.84 & 1.02 & 0.86 \\
40688 & 108145 & 2001 & 34.50 & -0.32 & 3366.00 & 33655.86 & 1.02 & 0.98 & 1.00 \\
43415 & 109112 & 2001 & 44.60 & -0.37 & 4466.00 & 42784.64 & 1.00 & 0.96 & 0.96 \\
2367 & 100320 & 2001 & 54.80 & 0.03 & 5468.00 & 53882.75 & 1.00 & 0.98 & 0.99 \\
43371 & 109104 & 2001 & 43.00 & -0.24 & 4343.00 & 41540.49 & 0.99 & 0.97 & 0.96 \\
43392 & 109111 & 2001 & 40.10 & -0.24 & 4745.00 & 42455.81 & 0.85 & 1.06 & 0.89 \\
3755 & 100480 & 2001 & 146.50 & -0.12 & 16998.00 & 123554.42 & 0.86 & 0.84 & 0.73 \\
19800 & 102652 & 2001 & 2394.40 & -0.02 & 251020.00 & 2088730.05 & 0.95 & 0.87 & 0.83 \\
41084 & 108192 & 2001 & 48.60 & -0.06 & 8309.00 & 80249.09 & 0.58 & 1.65 & 0.97 \\
43596 & 109148 & 2001 & 23.10 & -0.16 & 2126.00 & 18188.33 & 1.09 & 0.79 & 0.86 \\
23218 & 103145 & 2001 & 76.80 & -0.08 & 12405.00 & 128710.49 & 0.62 & 1.68 & 1.04 \\
48473 & 240087 & 2001 & 52.30 & -0.10 & 5253.00 & 51514.40 & 1.00 & 0.98 & 0.98 \\
43384 & 109110 & 2001 & 19.00 & 0.04 & 1900.00 & 18767.85 & 1.00 & 0.99 & 0.99 \\
20016 & 102663 & 2001 & 5763.50 & -0.41 & 704559.00 & 4703474.00 & 0.82 & 0.82 & 0.67 \\
2201 & 100295 & 2001 & 17.30 & -0.08 & 1804.00 & 14125.67 & 0.96 & 0.82 & 0.78 \\
43598 & 109151 & 2001 & 155.20 & -0.19 & 14202.00 & 136936.91 & 1.09 & 0.88 & 0.96 \\
43573 & 109147 & 2001 & 20.30 & -0.56 & 1755.00 & 14232.19 & 1.16 & 0.70 & 0.81 \\
49012 & 240199 & 2001 & 991.20 & -0.35 & 99130.00 & 968067.89 & 1.00 & 0.98 & 0.98 \\
41111 & 108202 & 2001 & 42.10 & -0.03 & 2746.00 & 27359.64 & 1.53 & 0.65 & 1.00 \\
54203 & 364917 & 2001 & 33.00 & 0.04 & 2511.00 & 21380.78 & 1.31 & 0.65 & 0.85 \\
23188 & 103144 & 2001 & 22.50 & -0.15 & 1665.00 & 18531.93 & 1.35 & 0.82 & 1.11 \\
2100 & 100291 & 2001 & 457.30 & 0.12 & 45601.00 & 429375.18 & 1.00 & 0.94 & 0.94 \\
2222 & 100296 & 2001 & 9.70 & -0.11 & 909.00 & 10044.11 & 1.07 & 1.04 & 1.10 \\
20256 & 102696 & 2001 & 353.40 & 0.04 & 36135.00 & 333491.86 & 0.98 & 0.94 & 0.92 \\
19766 & 102651 & 2001 & 3472.70 & -0.12 & 367331.00 & 3376313.09 & 0.95 & 0.97 & 0.92 \\
23761 & 103212 & 2001 & 3753.20 & -0.05 & 378362.00 & 3163668.07 & 0.99 & 0.84 & 0.84 \\
41090 & 108194 & 2001 & 24.50 & -0.06 & 2078.00 & 22851.16 & 1.18 & 0.93 & 1.10 \\
40713 & 108146 & 2001 & 80.20 & -0.29 & 7901.00 & 79014.31 & 1.02 & 0.99 & 1.00 \\
43424 & 109117 & 2001 & 258.00 & -0.28 & 27172.00 & 245630.16 & 0.95 & 0.95 & 0.90 \\
48406 & 240076 & 2001 & 100.90 & -0.13 & 9966.00 & 91774.00 & 1.01 & 0.91 & 0.92 \\
23159 & 103136 & 2001 & 373.20 & -0.19 & 37257.00 & 368684.46 & 1.00 & 0.99 & 0.99 \\
43376 & 109108 & 2001 & 25.50 & -0.03 & 2555.00 & 25225.22 & 1.00 & 0.99 & 0.99 \\
41093 & 108197 & 2001 & 11.10 & 0.03 & 1103.00 & 10840.54 & 1.01 & 0.98 & 0.98 \\
41099 & 108200 & 2001 & 141.00 & 0.04 & 13279.00 & 132786.28 & 1.06 & 0.94 & 1.00 \\
19981 & 102660 & 2001 & 6711.40 & -0.14 & 701285.00 & 6484599.10 & 0.96 & 0.97 & 0.92 \\
43263 & 109088 & 2001 & 2079.20 & -0.12 & 196060.00 & 1987064.22 & 1.06 & 0.96 & 1.01 \\
43444 & 109122 & 2001 & 21.60 & -0.22 & 1881.00 & 21152.33 & 1.15 & 0.98 & 1.12 \\
43559 & 109144 & 2001 & 43.70 & -0.34 & 3173.00 & 31711.45 & 1.38 & 0.73 & 1.00 \\
20111 & 102667 & 2001 & 18887.70 & -0.13 & 2031595.00 & 16682466.20 & 0.93 & 0.88 & 0.82 \\
40990 & 108172 & 2001 & 135.00 & 0.02 & 13572.00 & 121672.19 & 0.99 & 0.90 & 0.90 \\
64235 & 500593 & 2001 & 297.80 & -0.01 & 21367.00 & 221999.16 & 1.39 & 0.75 & 1.04 \\
20155 & 102671 & 2001 & 250.80 & -0.15 & 25296.00 & 248522.78 & 0.99 & 0.99 & 0.98 \\
23658 & 103205 & 2001 & 44.30 & 0.11 & 4352.00 & 41314.82 & 1.02 & 0.93 & 0.95 \\
2261 & 100303 & 2001 & 365.50 & -0.14 & 40292.00 & 305910.52 & 0.91 & 0.84 & 0.76 \\
2315 & 100315 & 2001 & 292.00 & -0.20 & 29196.00 & 285701.84 & 1.00 & 0.98 & 0.98 \\
40832 & 108156 & 2001 & 22.30 & -0.14 & 1943.00 & 19223.17 & 1.15 & 0.86 & 0.99 \\
61260 & 500025 & 2001 & 1.80 & 0.17 & 141.00 & 1386.45 & 1.28 & 0.77 & 0.98 \\
13618 & 101748 & 2001 & 118.30 & 0.00 & 12130.00 & 112353.18 & 0.98 & 0.95 & 0.93 \\
23310 & 103160 & 2001 & 119.70 & 0.01 & 11721.00 & 117213.76 & 1.02 & 0.98 & 1.00 \\
48448 & 240085 & 2001 & 217.00 & -0.15 & 19599.00 & 224129.65 & 1.11 & 1.03 & 1.14 \\
23414 & 103175 & 2001 & 796.70 & -0.01 & 79639.00 & 735661.03 & 1.00 & 0.92 & 0.92 \\
40822 & 108155 & 2001 & 76.70 & -0.16 & 8454.00 & 79939.77 & 0.91 & 1.04 & 0.95 \\
40981 & 108170 & 2001 & 572.20 & -0.46 & 57903.00 & 541688.25 & 0.99 & 0.95 & 0.94 \\
43502 & 109129 & 2001 & 114.60 & -0.12 & 11398.00 & 110411.96 & 1.01 & 0.96 & 0.97 \\
12641 & 101561 & 2001 & 98.50 & 0.03 & 8642.00 & 94851.64 & 1.14 & 0.96 & 1.10 \\
40817 & 108154 & 2001 & 50.10 & -0.09 & 4957.00 & 44532.11 & 1.01 & 0.89 & 0.90 \\
20176 & 102673 & 2001 & 451.70 & -0.05 & 47502.00 & 421252.87 & 0.95 & 0.93 & 0.89 \\
48985 & 240197 & 2001 & 761.70 & -0.35 & 76174.00 & 748163.60 & 1.00 & 0.98 & 0.98 \\
2335 & 100319 & 2001 & 130.90 & 0.29 & 13164.00 & 124071.83 & 0.99 & 0.95 & 0.94 \\
43531 & 109142 & 2001 & 219.30 & -0.18 & 21677.00 & 216773.89 & 1.01 & 0.99 & 1.00 \\
23298 & 103158 & 2001 & 2425.00 & -0.19 & 232974.00 & 2329716.62 & 1.04 & 0.96 & 1.00 \\
40892 & 108161 & 2001 & 320.00 & -0.12 & 36059.00 & 328126.50 & 0.89 & 1.03 & 0.91 \\
19879 & 102654 & 2001 & 1183.30 & -0.12 & 135158.00 & 1149817.12 & 0.88 & 0.97 & 0.85 \\
64258 & 500594 & 2001 & 560.30 & 0.01 & 53598.00 & 574658.13 & 1.05 & 1.03 & 1.07 \\
7802 & 101061 & 2001 & 2939.40 & -0.18 & 284278.00 & 2792112.49 & 1.03 & 0.95 & 0.98 \\
19923 & 102655 & 2001 & 1070.20 & -0.04 & 113879.00 & 966612.76 & 0.94 & 0.90 & 0.85 \\
23627 & 103204 & 2001 & 234.40 & -0.11 & 24249.00 & 199294.77 & 0.97 & 0.85 & 0.82 \\
64212 & 500592 & 2001 & 525.20 & -0.10 & 42196.00 & 436108.99 & 1.24 & 0.83 & 1.03 \\
2274 & 100305 & 2001 & 97.90 & -0.29 & 11611.00 & 84133.63 & 0.84 & 0.86 & 0.72 \\
40851 & 108159 & 2001 & 42.50 & -0.12 & 4255.00 & 40421.01 & 1.00 & 0.95 & 0.95 \\
3919 & 100514 & 2001 & 64.00 & -0.13 & 6403.00 & 63731.11 & 1.00 & 1.00 & 1.00 \\
41009 & 108176 & 2001 & 22.30 & -0.43 & 2233.00 & 21356.40 & 1.00 & 0.96 & 0.96 \\
3817 & 100485 & 2001 & 721.50 & -0.14 & 73082.00 & 701871.07 & 0.99 & 0.97 & 0.96 \\
43286 & 109090 & 2001 & 35.50 & -0.03 & 2973.00 & 30338.76 & 1.19 & 0.85 & 1.02 \\
23352 & 103166 & 2001 & 1.80 & -0.50 & 245.00 & 2108.53 & 0.73 & 1.17 & 0.86 \\
40842 & 108158 & 2001 & 20.90 & -0.24 & 2108.00 & 21081.30 & 0.99 & 1.01 & 1.00 \\
43450 & 109124 & 2001 & 144.60 & -0.11 & 14557.00 & 135888.50 & 0.99 & 0.94 & 0.93 \\
43473 & 109125 & 2001 & 294.80 & -0.31 & 29623.00 & 287440.45 & 1.00 & 0.98 & 0.97 \\
40875 & 108160 & 2001 & 24.90 & -0.14 & 2487.00 & 24661.61 & 1.00 & 0.99 & 0.99 \\
23384 & 103174 & 2001 & 1484.80 & -0.07 & 148610.00 & 1464836.87 & 1.00 & 0.99 & 0.99 \\
43303 & 109092 & 2001 & 49.80 & 0.00 & 4849.00 & 48430.47 & 1.03 & 0.97 & 1.00 \\
40839 & 108157 & 2001 & 5.20 & -0.52 & 373.00 & 3746.66 & 1.39 & 0.72 & 1.00 \\
7349 & 101023 & 2001 & 21831.50 & -0.05 & 2318552.00 & 20343387.99 & 0.94 & 0.93 & 0.88 \\
41002 & 108175 & 2001 & 65.10 & 0.01 & 5337.00 & 57881.68 & 1.22 & 0.89 & 1.08 \\
41000 & 108174 & 2001 & 23.50 & -0.19 & 2035.00 & 20920.44 & 1.15 & 0.89 & 1.03 \\
20144 & 102669 & 2001 & 27.40 & -0.27 & 2863.00 & 25768.31 & 0.96 & 0.94 & 0.90 \\
43477 & 109126 & 2001 & 126.40 & -0.16 & 12673.00 & 124915.63 & 1.00 & 0.99 & 0.99 \\
23603 & 103202 & 2001 & 57.90 & -0.09 & 5723.00 & 51632.34 & 1.01 & 0.89 & 0.90 \\
43479 & 109127 & 2001 & 24.60 & -0.11 & 1985.00 & 20979.34 & 1.24 & 0.85 & 1.06 \\
48436 & 240083 & 2001 & 365.40 & -0.14 & 35307.00 & 353063.11 & 1.03 & 0.97 & 1.00 \\
43566 & 109145 & 2001 & 45.20 & -0.34 & 5296.00 & 48627.22 & 0.85 & 1.08 & 0.92 \\
48386 & 240074 & 2001 & 25.10 & -0.11 & 2573.00 & 22355.62 & 0.98 & 0.89 & 0.87 \\
43326 & 109095 & 2001 & 34.70 & 0.09 & 3479.00 & 31936.50 & 1.00 & 0.92 & 0.92 \\
3973 & 100535 & 2001 & 322.60 & -0.03 & 34381.00 & 338555.33 & 0.94 & 1.05 & 0.98 \\
20204 & 102688 & 2001 & 94.10 & -0.29 & 9406.00 & 78291.17 & 1.00 & 0.83 & 0.83 \\
43334 & 109099 & 2001 & 29.20 & -0.24 & 2153.00 & 21233.64 & 1.36 & 0.73 & 0.99 \\
48999 & 240198 & 2001 & 1618.70 & -0.10 & 159746.00 & 1597460.46 & 1.01 & 0.99 & 1.00 \\
43357 & 109100 & 2001 & 5.70 & -0.24 & 492.00 & 5209.43 & 1.16 & 0.91 & 1.06 \\
43254 & 109087 & 2001 & 326.10 & -0.39 & 33698.00 & 285962.94 & 0.97 & 0.88 & 0.85 \\
23448 & 103177 & 2001 & 218.30 & 0.02 & 16848.00 & 186829.42 & 1.30 & 0.86 & 1.11 \\
43231 & 109086 & 2001 & 204.70 & 0.02 & 20058.00 & 196234.67 & 1.02 & 0.96 & 0.98 \\
23725 & 103209 & 2001 & 158.30 & 0.04 & 15832.00 & 140478.27 & 1.00 & 0.89 & 0.89 \\
12672 & 101562 & 2001 & 320.00 & 0.06 & 33739.00 & 320030.76 & 0.95 & 1.00 & 0.95 \\
20050 & 102664 & 2001 & 2037.30 & 0.22 & 184637.00 & 1636904.17 & 1.10 & 0.80 & 0.89 \\
43435 & 109119 & 2001 & 98.40 & -0.25 & 12665.00 & 109245.05 & 0.78 & 1.11 & 0.86 \\
61289 & 500027 & 2001 & 156.30 & -0.14 & 17729.00 & 131843.94 & 0.88 & 0.84 & 0.74 \\
40742 & 108148 & 2001 & 61.50 & -0.13 & 6031.00 & 60311.44 & 1.02 & 0.98 & 1.00 \\
43554 & 109143 & 2001 & 701.20 & -0.26 & 79932.00 & 779887.82 & 0.88 & 1.11 & 0.98 \\
12784 & 101595 & 2001 & 2102.30 & -0.29 & 170178.00 & 1979144.89 & 1.24 & 0.94 & 1.16 \\
2234 & 100298 & 2001 & 650.80 & 0.04 & 69262.00 & 630883.27 & 0.94 & 0.97 & 0.91 \\
40738 & 108147 & 2001 & 18.70 & -0.15 & 1866.00 & 18473.23 & 1.00 & 0.99 & 0.99 \\
23554 & 103186 & 2001 & 998.20 & -0.22 & 99653.00 & 949900.12 & 1.00 & 0.95 & 0.95 \\
43368 & 109102 & 2001 & 19.60 & -0.23 & 1957.00 & 18885.81 & 1.00 & 0.96 & 0.97 \\
64012 & 500577 & 2001 & 169.90 & 0.31 & 12023.00 & 121006.52 & 1.41 & 0.71 & 1.01 \\
40957 & 108168 & 2001 & 227.60 & 0.06 & 15902.00 & 184203.96 & 1.43 & 0.81 & 1.16 \\
7834 & 101062 & 2001 & 1244.80 & 0.09 & 116888.00 & 1043068.02 & 1.06 & 0.84 & 0.89 \\
19841 & 102653 & 2001 & 4684.80 & -0.10 & 496065.00 & 4001478.91 & 0.94 & 0.85 & 0.81 \\
20189 & 102676 & 2001 & 199.50 & -0.19 & 20093.00 & 188655.39 & 0.99 & 0.95 & 0.94 \\
64074 & 500586 & 2001 & 639.10 & -0.04 & 42764.00 & 437552.66 & 1.49 & 0.68 & 1.02 \\
23283 & 103154 & 2001 & 518.80 & -0.25 & 49984.00 & 499771.45 & 1.04 & 0.96 & 1.00 \\
40952 & 108166 & 2001 & 158.40 & -0.40 & 15600.00 & 156371.84 & 1.02 & 0.99 & 1.00 \\
3785 & 100481 & 2001 & 89.80 & 0.00 & 10384.00 & 74809.80 & 0.86 & 0.83 & 0.72 \\
23691 & 103208 & 2001 & 1760.40 & -0.02 & 176243.00 & 1479788.58 & 1.00 & 0.84 & 0.84 \\
48375 & 240067 & 2001 & 406.50 & -0.12 & 40688.00 & 390754.04 & 1.00 & 0.96 & 0.96 \\
43310 & 109093 & 2001 & 21.60 & -0.21 & 2083.00 & 20829.43 & 1.04 & 0.96 & 1.00 \\
41038 & 108183 & 2001 & 87.50 & -0.17 & 8729.00 & 87289.25 & 1.00 & 1.00 & 1.00 \\
40792 & 108153 & 2001 & 9.90 & 0.21 & 1138.00 & 10492.06 & 0.87 & 1.06 & 0.92 \\
13586 & 101744 & 2001 & 1175.00 & -0.12 & 103395.00 & 1106935.89 & 1.14 & 0.94 & 1.07 \\
19951 & 102659 & 2001 & 5582.70 & -0.03 & 591884.00 & 5519639.23 & 0.94 & 0.99 & 0.93 \\
41057 & 108185 & 2001 & 3.40 & -0.19 & 345.00 & 3432.64 & 0.99 & 1.01 & 0.99 \\
41059 & 108186 & 2001 & 16.30 & -0.30 & 1636.00 & 16358.78 & 1.00 & 1.00 & 1.00 \\
40767 & 108149 & 2001 & 26.10 & -0.13 & 2486.00 & 24117.80 & 1.05 & 0.92 & 0.97 \\
43314 & 109094 & 2001 & 301.20 & -0.13 & 30095.00 & 288502.67 & 1.00 & 0.96 & 0.96 \\
40950 & 108165 & 2001 & 8.00 & -0.36 & 804.00 & 7845.63 & 1.00 & 0.98 & 0.98 \\
23573 & 103193 & 2001 & 49.20 & 0.10 & 4528.00 & 41308.40 & 1.09 & 0.84 & 0.91 \\
41063 & 108188 & 2001 & 688.40 & -0.15 & 61924.00 & 536985.07 & 1.11 & 0.78 & 0.87 \\
23251 & 103152 & 2001 & 2557.20 & -0.12 & 253175.00 & 2477710.96 & 1.01 & 0.97 & 0.98 \\
20078 & 102665 & 2001 & 16.50 & 0.02 & 1650.00 & 14854.41 & 1.00 & 0.90 & 0.90 \\
54916 & 400025 & 2001 & 148.80 & -0.30 & 15515.00 & 135611.42 & 0.96 & 0.91 & 0.87 \\
23479 & 103179 & 2001 & 295.00 & -0.21 & 29514.00 & 288107.30 & 1.00 & 0.98 & 0.98 \\
3107 & 100409 & 2001 & 257.40 & -0.23 & 25898.00 & 241870.36 & 0.99 & 0.94 & 0.93 \\
3725 & 100475 & 2001 & 395.40 & 0.00 & 39867.00 & 376920.18 & 0.99 & 0.95 & 0.95 \\
41984 & 108889 & 2001 & 19.10 & -0.44 & 1869.00 & 18087.27 & 1.02 & 0.95 & 0.97 \\
13331 & 101728 & 2001 & 75.90 & -0.34 & 7040.00 & 70394.60 & 1.08 & 0.93 & 1.00 \\
54789 & 400014 & 2001 & 214.40 & -0.14 & 23967.00 & 225987.00 & 0.89 & 1.05 & 0.94 \\
22109 & 102990 & 2001 & 4326.10 & -0.03 & 401575.00 & 3962161.00 & 1.08 & 0.92 & 0.99 \\
48854 & 240148 & 2001 & 277.10 & -0.12 & 21603.00 & 208082.41 & 1.28 & 0.75 & 0.96 \\
21142 & 102833 & 2001 & 35.80 & -0.31 & 3592.00 & 32718.66 & 1.00 & 0.91 & 0.91 \\
41975 & 108886 & 2001 & 73.80 & -0.18 & 7390.00 & 72642.74 & 1.00 & 0.98 & 0.98 \\
61960 & 500312 & 2001 & 10.50 & 0.09 & 1028.00 & 10183.00 & 1.02 & 0.97 & 0.99 \\
22143 & 102993 & 2001 & 5562.80 & -0.11 & 508659.00 & 5177632.39 & 1.09 & 0.93 & 1.02 \\
41947 & 108874 & 2001 & 38.40 & -0.09 & 3844.00 & 37905.84 & 1.00 & 0.99 & 0.99 \\
42678 & 108993 & 2001 & 34.00 & 0.15 & 1979.00 & 19356.05 & 1.72 & 0.57 & 0.98 \\
41944 & 108871 & 2001 & 29.70 & -0.12 & 2966.00 & 27562.51 & 1.00 & 0.93 & 0.93 \\
42693 & 108994 & 2001 & 370.80 & -0.27 & 34159.00 & 358538.62 & 1.09 & 0.97 & 1.05 \\
41920 & 108870 & 2001 & 23.90 & -0.04 & 2207.00 & 21668.15 & 1.08 & 0.91 & 0.98 \\
42700 & 108995 & 2001 & 9.30 & -0.53 & 1022.00 & 10225.12 & 0.91 & 1.10 & 1.00 \\
21110 & 102832 & 2001 & 196.10 & -0.29 & 20165.00 & 201654.39 & 0.97 & 1.03 & 1.00 \\
13073 & 101626 & 2001 & 2081.00 & -0.23 & 208135.00 & 1890355.04 & 1.00 & 0.91 & 0.91 \\
54815 & 400015 & 2001 & 5.60 & -0.45 & 492.00 & 5553.57 & 1.14 & 0.99 & 1.13 \\
61935 & 500310 & 2001 & 5.70 & -0.21 & 539.00 & 5367.48 & 1.06 & 0.94 & 1.00 \\
2854 & 100365 & 2001 & 480.60 & -0.18 & 47436.00 & 470903.21 & 1.01 & 0.98 & 0.99 \\
21164 & 102835 & 2001 & 204.30 & -0.08 & 15956.00 & 159751.83 & 1.28 & 0.78 & 1.00 \\
41990 & 108890 & 2001 & 3.40 & -0.31 & 329.00 & 3287.26 & 1.03 & 0.97 & 1.00 \\
41994 & 108900 & 2001 & 14.00 & -0.44 & 1599.00 & 14662.42 & 0.88 & 1.05 & 0.92 \\
42039 & 108914 & 2001 & 16.20 & 0.10 & 1640.00 & 14703.89 & 0.99 & 0.91 & 0.90 \\
42598 & 108985 & 2001 & 1713.60 & 0.02 & 132756.00 & 1468689.03 & 1.29 & 0.86 & 1.11 \\
2882 & 100368 & 2001 & 301.70 & -0.22 & 30428.00 & 278416.02 & 0.99 & 0.92 & 0.91 \\
7591 & 101047 & 2001 & 279.80 & -0.10 & 30112.00 & 278907.70 & 0.93 & 1.00 & 0.93 \\
61979 & 500315 & 2001 & 39.90 & 0.19 & 3922.00 & 38226.82 & 1.02 & 0.96 & 0.97 \\
61975 & 500313 & 2001 & 36.10 & -0.20 & 2284.00 & 21737.59 & 1.58 & 0.60 & 0.95 \\
22048 & 102988 & 2001 & 55.50 & 0.01 & 5240.00 & 52179.40 & 1.06 & 0.94 & 1.00 \\
42036 & 108912 & 2001 & 53.00 & -0.27 & 5345.00 & 53446.67 & 0.99 & 1.01 & 1.00 \\
42623 & 108987 & 2001 & 117.30 & -0.11 & 13274.00 & 122460.71 & 0.88 & 1.04 & 0.92 \\
42026 & 108910 & 2001 & 37.10 & -0.22 & 3726.00 & 37261.25 & 1.00 & 1.00 & 1.00 \\
42024 & 108908 & 2001 & 68.20 & -0.61 & 9588.00 & 62704.50 & 0.71 & 0.92 & 0.65 \\
42632 & 108988 & 2001 & 496.50 & -0.12 & 48392.00 & 498955.41 & 1.03 & 1.00 & 1.03 \\
21196 & 102837 & 2001 & 1184.80 & -0.10 & 106268.00 & 1104143.02 & 1.11 & 0.93 & 1.04 \\
7495 & 101042 & 2001 & 15210.30 & -0.04 & 1436958.00 & 12564615.44 & 1.06 & 0.83 & 0.87 \\
48828 & 240144 & 2001 & 9.70 & -0.15 & 967.00 & 7899.02 & 1.00 & 0.81 & 0.82 \\
22076 & 102989 & 2001 & 1983.40 & -0.12 & 265376.00 & 1908880.09 & 0.75 & 0.96 & 0.72 \\
41997 & 108901 & 2001 & 523.40 & -0.20 & 49258.00 & 516099.27 & 1.06 & 0.99 & 1.05 \\
42638 & 108990 & 2001 & 25.80 & 0.02 & 2585.00 & 24540.85 & 1.00 & 0.95 & 0.95 \\
42646 & 108991 & 2001 & 559.30 & -0.38 & 53750.00 & 537501.30 & 1.04 & 0.96 & 1.00 \\
42654 & 108992 & 2001 & 9.40 & -0.24 & 1844.00 & 18441.75 & 0.51 & 1.96 & 1.00 \\
22187 & 102994 & 2001 & 73.10 & -0.14 & 6459.00 & 72914.77 & 1.13 & 1.00 & 1.13 \\
41890 & 108868 & 2001 & 638.90 & 0.18 & 93833.00 & 932329.90 & 0.68 & 1.46 & 0.99 \\
13354 & 101729 & 2001 & 187.20 & -0.23 & 17945.00 & 179433.16 & 1.04 & 0.96 & 1.00 \\
22277 & 102999 & 2001 & 182.40 & -0.09 & 18432.00 & 181614.73 & 0.99 & 1.00 & 0.99 \\
41839 & 108860 & 2001 & 77.10 & -0.39 & 5764.00 & 74035.08 & 1.34 & 0.96 & 1.28 \\
63252 & 500491 & 2001 & 636.40 & -0.07 & 61476.00 & 656374.66 & 1.04 & 1.03 & 1.07 \\
22298 & 103005 & 2001 & 274.60 & 0.04 & 23838.00 & 236594.75 & 1.15 & 0.86 & 0.99 \\
61907 & 500306 & 2001 & 3.20 & -0.06 & 246.00 & 2040.22 & 1.30 & 0.64 & 0.83 \\
41832 & 108859 & 2001 & 33.40 & -0.46 & 2516.00 & 33360.83 & 1.33 & 1.00 & 1.33 \\
63291 & 500493 & 2001 & 1367.10 & -0.13 & 89833.00 & 1129482.48 & 1.52 & 0.83 & 1.26 \\
21006 & 102821 & 2001 & 142.00 & -0.21 & 14300.00 & 138092.91 & 0.99 & 0.97 & 0.97 \\
22334 & 103007 & 2001 & 1241.30 & -0.11 & 117746.00 & 1146457.41 & 1.05 & 0.92 & 0.97 \\
41784 & 108856 & 2001 & 122.80 & 0.14 & 12274.00 & 109730.73 & 1.00 & 0.89 & 0.89 \\
13383 & 101736 & 2001 & 55.20 & -0.20 & 5517.00 & 51725.49 & 1.00 & 0.94 & 0.94 \\
41765 & 108855 & 2001 & 193.90 & 0.29 & 25820.00 & 208136.49 & 0.75 & 1.07 & 0.81 \\
20995 & 102818 & 2001 & 33.60 & -0.21 & 3358.00 & 32669.87 & 1.00 & 0.97 & 0.97 \\
41757 & 108853 & 2001 & 336.50 & -0.18 & 32270.00 & 302687.07 & 1.04 & 0.90 & 0.94 \\
48645 & 240116 & 2001 & 146.00 & -0.14 & 12748.00 & 121830.61 & 1.15 & 0.83 & 0.96 \\
3397 & 100431 & 2001 & 295.50 & -0.09 & 24670.00 & 280291.04 & 1.20 & 0.95 & 1.14 \\
63314 & 500494 & 2001 & 113.20 & -0.09 & 11086.00 & 104808.32 & 1.02 & 0.93 & 0.95 \\
42748 & 109015 & 2001 & 177.60 & -0.13 & 17667.00 & 175468.81 & 1.01 & 0.99 & 0.99 \\
2753 & 100357 & 2001 & 202.80 & -0.21 & 25694.00 & 221708.21 & 0.79 & 1.09 & 0.86 \\
20979 & 102814 & 2001 & 158.50 & 0.05 & 15755.00 & 147590.33 & 1.01 & 0.93 & 0.94 \\
13015 & 101621 & 2001 & 3433.80 & -0.14 & 342161.00 & 3141193.44 & 1.00 & 0.91 & 0.92 \\
54597 & 377010 & 2001 & 23.60 & 0.03 & 2344.00 & 20319.00 & 1.01 & 0.86 & 0.87 \\
7611 & 101048 & 2001 & 12324.80 & -0.18 & 1227489.00 & 11947668.43 & 1.00 & 0.97 & 0.97 \\
21025 & 102824 & 2001 & 64.50 & -0.01 & 6408.00 & 62470.72 & 1.01 & 0.97 & 0.97 \\
42743 & 109011 & 2001 & 47.70 & 0.21 & 4724.00 & 47189.83 & 1.01 & 0.99 & 1.00 \\
42738 & 109010 & 2001 & 16.20 & -0.11 & 1614.00 & 15731.31 & 1.00 & 0.97 & 0.97 \\
21090 & 102828 & 2001 & 190.20 & -0.25 & 19029.00 & 182484.98 & 1.00 & 0.96 & 0.96 \\
63229 & 500490 & 2001 & 292.40 & -0.11 & 24615.00 & 247993.68 & 1.19 & 0.85 & 1.01 \\
2830 & 100362 & 2001 & 44.00 & 0.29 & 3885.00 & 41078.17 & 1.13 & 0.93 & 1.06 \\
42701 & 108996 & 2001 & 67.90 & -0.16 & 6768.00 & 66075.50 & 1.00 & 0.97 & 0.98 \\
48879 & 240149 & 2001 & 64.20 & 0.28 & 6016.00 & 60017.97 & 1.07 & 0.93 & 1.00 \\
2819 & 100360 & 2001 & 740.00 & -0.34 & 73795.00 & 717276.55 & 1.00 & 0.97 & 0.97 \\
42705 & 108997 & 2001 & 112.30 & -0.40 & 11241.00 & 101573.67 & 1.00 & 0.90 & 0.90 \\
22218 & 102996 & 2001 & 805.20 & -0.16 & 80093.00 & 776862.10 & 1.01 & 0.96 & 0.97 \\
41884 & 108867 & 2001 & 930.80 & -0.32 & 77551.00 & 913886.70 & 1.20 & 0.98 & 1.18 \\
42708 & 109006 & 2001 & 61.00 & -0.35 & 6101.00 & 60977.34 & 1.00 & 1.00 & 1.00 \\
13050 & 101623 & 2001 & 1890.60 & -0.05 & 188914.00 & 1839729.97 & 1.00 & 0.97 & 0.97 \\
2810 & 100359 & 2001 & 352.50 & -0.26 & 35248.00 & 334084.65 & 1.00 & 0.95 & 0.95 \\
7464 & 101040 & 2001 & 2991.60 & -0.14 & 284774.00 & 2932132.87 & 1.05 & 0.98 & 1.03 \\
61930 & 500309 & 2001 & 89.40 & 0.05 & 8933.00 & 88574.93 & 1.00 & 0.99 & 0.99 \\
2797 & 100358 & 2001 & 836.50 & -0.32 & 83648.00 & 776667.93 & 1.00 & 0.93 & 0.93 \\
54825 & 400017 & 2001 & 156.20 & -0.09 & 13491.00 & 118824.89 & 1.16 & 0.76 & 0.88 \\
22246 & 102997 & 2001 & 6328.10 & 0.03 & 634365.00 & 5816788.91 & 1.00 & 0.92 & 0.92 \\
41859 & 108866 & 2001 & 263.00 & 0.02 & 25257.00 & 251610.06 & 1.04 & 0.96 & 1.00 \\
61925 & 500308 & 2001 & 197.90 & -0.12 & 19768.00 & 195581.95 & 1.00 & 0.99 & 0.99 \\
41853 & 108861 & 2001 & 80.60 & -0.11 & 7223.00 & 81163.93 & 1.12 & 1.01 & 1.12 \\
13036 & 101622 & 2001 & 1625.60 & -0.33 & 169409.00 & 1442769.56 & 0.96 & 0.89 & 0.85 \\
42715 & 109009 & 2001 & 92.30 & -0.09 & 9040.00 & 86204.55 & 1.02 & 0.93 & 0.95 \\
21067 & 102827 & 2001 & 121.00 & -0.12 & 12101.00 & 118116.23 & 1.00 & 0.98 & 0.98 \\
42763 & 109016 & 2001 & 57.50 & 0.07 & 4538.00 & 47089.73 & 1.27 & 0.82 & 1.04 \\
42063 & 108915 & 2001 & 28.20 & -0.33 & 2821.00 & 26217.23 & 1.00 & 0.93 & 0.93 \\
42071 & 108918 & 2001 & 26.80 & -0.06 & 2683.00 & 26686.75 & 1.00 & 1.00 & 0.99 \\
42383 & 108959 & 2001 & 12.60 & -0.31 & 1018.00 & 11553.09 & 1.24 & 0.92 & 1.13 \\
48727 & 240134 & 2001 & 96.50 & 0.13 & 9681.00 & 95333.54 & 1.00 & 0.99 & 0.98 \\
21770 & 102951 & 2001 & 7279.50 & -0.12 & 687703.00 & 7229671.35 & 1.06 & 0.99 & 1.05 \\
42194 & 108934 & 2001 & 69.10 & -0.12 & 4909.00 & 44141.02 & 1.41 & 0.64 & 0.90 \\
21533 & 102893 & 2001 & 38.90 & 0.19 & 3813.00 & 38130.79 & 1.02 & 0.98 & 1.00 \\
42188 & 108933 & 2001 & 256.30 & -0.22 & 16768.00 & 207491.16 & 1.53 & 0.81 & 1.24 \\
21509 & 102876 & 2001 & 21.60 & -0.25 & 1983.00 & 22006.15 & 1.09 & 1.02 & 1.11 \\
3055 & 100401 & 2001 & 366.60 & -0.05 & 30670.00 & 299835.62 & 1.20 & 0.82 & 0.98 \\
13183 & 101703 & 2001 & 56287.40 & -0.05 & 4844676.00 & 40842919.03 & 1.16 & 0.73 & 0.84 \\
3184 & 100413 & 2001 & 60.20 & -0.19 & 5310.00 & 47975.02 & 1.13 & 0.80 & 0.90 \\
3046 & 100400 & 2001 & 814.20 & -0.31 & 71075.00 & 773399.54 & 1.15 & 0.95 & 1.09 \\
21498 & 102875 & 2001 & 14.00 & -0.10 & 1198.00 & 13226.14 & 1.17 & 0.94 & 1.10 \\
42393 & 108963 & 2001 & 45.10 & -0.17 & 4040.00 & 35331.06 & 1.12 & 0.78 & 0.87 \\
42417 & 108964 & 2001 & 753.40 & -0.12 & 75210.00 & 725202.08 & 1.00 & 0.96 & 0.96 \\
21814 & 102952 & 2001 & 391.50 & -0.29 & 44656.00 & 370678.76 & 0.88 & 0.95 & 0.83 \\
63005 & 500466 & 2001 & 395.30 & -0.22 & 35510.00 & 365061.70 & 1.11 & 0.92 & 1.03 \\
13170 & 101698 & 2001 & 191.80 & -0.06 & 19153.00 & 179750.54 & 1.00 & 0.94 & 0.94 \\
42430 & 108965 & 2001 & 51.20 & -0.43 & 3531.00 & 47097.31 & 1.45 & 0.92 & 1.33 \\
21454 & 102872 & 2001 & 799.00 & 0.10 & 79746.00 & 696299.66 & 1.00 & 0.87 & 0.87 \\
21831 & 102954 & 2001 & 96.10 & 0.16 & 9645.00 & 77321.64 & 1.00 & 0.80 & 0.80 \\
42163 & 108932 & 2001 & 778.80 & 0.05 & 67326.00 & 675729.78 & 1.16 & 0.87 & 1.00 \\
42434 & 108966 & 2001 & 508.30 & -0.19 & 50990.00 & 509903.81 & 1.00 & 1.00 & 1.00 \\
21752 & 102949 & 2001 & 2168.00 & -0.13 & 267354.00 & 2412367.34 & 0.81 & 1.11 & 0.90 \\
42293 & 108950 & 2001 & 54.70 & 0.26 & 3879.00 & 36068.35 & 1.41 & 0.66 & 0.93 \\
42318 & 108951 & 2001 & 188.20 & -0.02 & 17898.00 & 178981.47 & 1.05 & 0.95 & 1.00 \\
21678 & 102939 & 2001 & 6216.30 & -0.12 & 620968.00 & 6134146.47 & 1.00 & 0.99 & 0.99 \\
21638 & 102937 & 2001 & 39.40 & 0.12 & 4029.00 & 35383.83 & 0.98 & 0.90 & 0.88 \\
7559 & 101045 & 2001 & 12211.70 & -0.07 & 1386924.00 & 11731848.80 & 0.88 & 0.96 & 0.85 \\
42260 & 108946 & 2001 & 41.60 & -0.14 & 4313.00 & 36472.81 & 0.96 & 0.88 & 0.85 \\
21623 & 102901 & 2001 & 96.30 & -0.12 & 9647.00 & 95589.63 & 1.00 & 0.99 & 0.99 \\
21611 & 102897 & 2001 & 180.10 & -0.08 & 18011.00 & 171085.87 & 1.00 & 0.95 & 0.95 \\
21708 & 102940 & 2001 & 1740.70 & -0.01 & 173868.00 & 1737139.89 & 1.00 & 1.00 & 1.00 \\
40663 & 108144 & 2001 & 363.70 & -0.17 & 35634.00 & 356342.74 & 1.02 & 0.98 & 1.00 \\
42210 & 108943 & 2001 & 728.70 & 0.13 & 56067.00 & 531441.79 & 1.30 & 0.73 & 0.95 \\
21597 & 102895 & 2001 & 1121.10 & -0.10 & 112275.00 & 1078101.75 & 1.00 & 0.96 & 0.96 \\
3142 & 100411 & 2001 & 4265.20 & -0.21 & 426592.00 & 4077315.77 & 1.00 & 0.96 & 0.96 \\
3085 & 100408 & 2001 & 204.50 & -0.16 & 20835.00 & 176555.81 & 0.98 & 0.86 & 0.85 \\
42334 & 108952 & 2001 & 439.10 & -0.02 & 41582.00 & 414715.95 & 1.06 & 0.94 & 1.00 \\
42359 & 108953 & 2001 & 265.90 & -0.31 & 26326.00 & 263258.37 & 1.01 & 0.99 & 1.00 \\
42365 & 108954 & 2001 & 42.50 & -0.13 & 4434.00 & 40880.80 & 0.96 & 0.96 & 0.92 \\
21563 & 102894 & 2001 & 1000.50 & -0.35 & 100134.00 & 985059.71 & 1.00 & 0.98 & 0.98 \\
48677 & 240118 & 2001 & 43.60 & 0.08 & 4259.00 & 41664.28 & 1.02 & 0.96 & 0.98 \\
42157 & 108931 & 2001 & 50.20 & -0.16 & 5018.00 & 45801.53 & 1.00 & 0.91 & 0.91 \\
3029 & 100399 & 2001 & 451.50 & -0.15 & 42481.00 & 343582.07 & 1.06 & 0.76 & 0.81 \\
42528 & 108975 & 2001 & 38.50 & -0.51 & 3866.00 & 37418.21 & 1.00 & 0.97 & 0.97 \\
42531 & 108976 & 2001 & 9.70 & -0.18 & 963.00 & 9194.30 & 1.01 & 0.95 & 0.95 \\
21958 & 102981 & 2001 & 198.60 & -0.15 & 20628.00 & 195642.81 & 0.96 & 0.99 & 0.95 \\
42104 & 108920 & 2001 & 43.00 & -0.62 & 4280.00 & 39571.26 & 1.00 & 0.92 & 0.92 \\
21331 & 102852 & 2001 & 2497.70 & -0.27 & 261942.00 & 2242344.33 & 0.95 & 0.90 & 0.86 \\
42096 & 108919 & 2001 & 230.80 & -0.47 & 22175.00 & 219605.49 & 1.04 & 0.95 & 0.99 \\
13126 & 101668 & 2001 & 107.30 & -0.04 & 10735.00 & 97446.10 & 1.00 & 0.91 & 0.91 \\
54769 & 400008 & 2001 & 60.60 & -0.03 & 6041.00 & 52545.16 & 1.00 & 0.87 & 0.87 \\
63179 & 500486 & 2001 & 304.30 & -0.11 & 31832.00 & 309495.81 & 0.96 & 1.02 & 0.97 \\
21312 & 102847 & 2001 & 24.90 & -0.19 & 2502.00 & 24363.87 & 1.00 & 0.98 & 0.97 \\
21991 & 102983 & 2001 & 16.40 & -0.26 & 1643.00 & 15214.85 & 1.00 & 0.93 & 0.93 \\
2916 & 100379 & 2001 & 474.10 & -0.09 & 46956.00 & 456820.16 & 1.01 & 0.96 & 0.97 \\
48804 & 240143 & 2001 & 195.90 & -0.23 & 19708.00 & 178033.21 & 0.99 & 0.91 & 0.90 \\
22003 & 102984 & 2001 & 21.80 & -0.09 & 2179.00 & 19601.95 & 1.00 & 0.90 & 0.90 \\
2891 & 100369 & 2001 & 363.80 & -0.13 & 35843.00 & 354490.41 & 1.01 & 0.97 & 0.99 \\
42543 & 108977 & 2001 & 12.50 & -0.18 & 1252.00 & 11449.88 & 1.00 & 0.92 & 0.91 \\
22017 & 102987 & 2001 & 1003.10 & -0.12 & 99336.00 & 968523.26 & 1.01 & 0.97 & 0.97 \\
42562 & 108979 & 2001 & 112.30 & -0.08 & 11233.00 & 108125.31 & 1.00 & 0.96 & 0.96 \\
42512 & 108973 & 2001 & 66.90 & -0.19 & 6229.00 & 65882.95 & 1.07 & 0.98 & 1.06 \\
2955 & 100389 & 2001 & 426.10 & -0.26 & 45601.00 & 449507.06 & 0.93 & 1.05 & 0.99 \\
63125 & 500483 & 2001 & 819.30 & -0.14 & 109195.00 & 723779.09 & 0.75 & 0.88 & 0.66 \\
3242 & 100417 & 2001 & 14.50 & -0.18 & 1189.00 & 12947.14 & 1.22 & 0.89 & 1.09 \\
42452 & 108968 & 2001 & 15.30 & -0.13 & 1302.00 & 13832.40 & 1.18 & 0.90 & 1.06 \\
21423 & 102871 & 2001 & 208.80 & 0.23 & 20845.00 & 182899.40 & 1.00 & 0.88 & 0.88 \\
21855 & 102957 & 2001 & 866.00 & -0.23 & 96198.00 & 808098.53 & 0.90 & 0.93 & 0.84 \\
42150 & 108930 & 2001 & 135.80 & -0.10 & 13526.00 & 131050.08 & 1.00 & 0.97 & 0.97 \\
42139 & 108929 & 2001 & 192.10 & 0.01 & 19919.00 & 183868.14 & 0.96 & 0.96 & 0.92 \\
42135 & 108926 & 2001 & 132.00 & -0.01 & 13368.00 & 106963.85 & 0.99 & 0.81 & 0.80 \\
13274 & 101716 & 2001 & 36.30 & -0.06 & 3241.00 & 34514.10 & 1.12 & 0.95 & 1.06 \\
3214 & 100415 & 2001 & 120.40 & 0.02 & 11954.00 & 114989.20 & 1.01 & 0.96 & 0.96 \\
13157 & 101681 & 2001 & 1239.10 & -0.21 & 134603.00 & 1184797.82 & 0.92 & 0.96 & 0.88 \\
21395 & 102861 & 2001 & 117.50 & -0.33 & 11688.00 & 116504.93 & 1.01 & 0.99 & 1.00 \\
21278 & 102844 & 2001 & 732.40 & -0.13 & 65997.00 & 726801.49 & 1.11 & 0.99 & 1.10 \\
13283 & 101717 & 2001 & 87.40 & -0.09 & 8085.00 & 85246.33 & 1.08 & 0.98 & 1.05 \\
2995 & 100395 & 2001 & 883.60 & -0.04 & 88717.00 & 840266.66 & 1.00 & 0.95 & 0.95 \\
42475 & 108970 & 2001 & 102.30 & -0.29 & 7666.00 & 91031.92 & 1.33 & 0.89 & 1.19 \\
42487 & 108971 & 2001 & 109.10 & -0.11 & 9251.00 & 100901.93 & 1.18 & 0.92 & 1.09 \\
48758 & 240139 & 2001 & 27.10 & 0.22 & 2710.00 & 21961.21 & 1.00 & 0.81 & 0.81 \\
21362 & 102854 & 2001 & 297.40 & -0.20 & 40750.00 & 367639.35 & 0.73 & 1.24 & 0.90 \\
21910 & 102979 & 2001 & 55.50 & -0.06 & 5643.00 & 53619.64 & 0.98 & 0.97 & 0.95 \\
42128 & 108925 & 2001 & 841.20 & -0.30 & 84498.00 & 830325.16 & 1.00 & 0.99 & 0.98 \\
48671 & 240117 & 2001 & 218.60 & 0.08 & 35328.00 & 293010.61 & 0.62 & 1.34 & 0.83 \\
42116 & 108924 & 2001 & 119.10 & 0.08 & 11804.00 & 115676.77 & 1.01 & 0.97 & 0.98 \\
42111 & 108923 & 2001 & 32.60 & -0.15 & 3264.00 & 32480.32 & 1.00 & 1.00 & 1.00 \\
22378 & 103008 & 2001 & 186.70 & -0.25 & 18634.00 & 186309.93 & 1.00 & 1.00 & 1.00 \\
41751 & 108852 & 2001 & 59.20 & -0.23 & 6511.00 & 58454.31 & 0.91 & 0.99 & 0.90 \\
42790 & 109018 & 2001 & 10.70 & -0.34 & 773.00 & 7621.99 & 1.38 & 0.71 & 0.99 \\
41372 & 108736 & 2001 & 42.20 & -0.17 & 3151.00 & 39323.27 & 1.34 & 0.93 & 1.25 \\
22862 & 103073 & 2001 & 704.80 & -0.03 & 67556.00 & 707270.79 & 1.04 & 1.00 & 1.05 \\
12883 & 101603 & 2001 & 1727.70 & -0.13 & 172787.00 & 1675260.90 & 1.00 & 0.97 & 0.97 \\
2507 & 100336 & 2001 & 30.50 & -0.03 & 2785.00 & 25435.27 & 1.10 & 0.83 & 0.91 \\
3623 & 100463 & 2001 & 934.30 & -0.08 & 87150.00 & 871505.78 & 1.07 & 0.93 & 1.00 \\
61568 & 500096 & 2001 & 55.30 & 0.07 & 5651.00 & 54567.52 & 0.98 & 0.99 & 0.97 \\
22893 & 103084 & 2001 & 373.70 & 0.14 & 34985.00 & 349849.72 & 1.07 & 0.94 & 1.00 \\
20501 & 102760 & 2001 & 1945.00 & -0.31 & 246898.00 & 2045072.20 & 0.79 & 1.05 & 0.83 \\
48537 & 240105 & 2001 & 248.50 & -0.54 & 25271.00 & 232685.87 & 0.98 & 0.94 & 0.92 \\
43079 & 109061 & 2001 & 201.70 & -0.21 & 13779.00 & 186863.01 & 1.46 & 0.93 & 1.36 \\
41324 & 108728 & 2001 & 14.40 & -0.05 & 1429.00 & 14125.96 & 1.01 & 0.98 & 0.99 \\
43084 & 109062 & 2001 & 19.50 & -0.10 & 1955.00 & 18669.00 & 1.00 & 0.96 & 0.95 \\
63844 & 500560 & 2001 & 10.80 & -0.13 & 1080.00 & 10761.46 & 1.00 & 1.00 & 1.00 \\
20472 & 102757 & 2001 & 16691.20 & -0.30 & 2055039.00 & 17460677.38 & 0.81 & 1.05 & 0.85 \\
63855 & 500561 & 2001 & 7.90 & -0.07 & 782.00 & 7804.65 & 1.01 & 0.99 & 1.00 \\
61557 & 500094 & 2001 & 32.10 & 0.12 & 3198.00 & 30239.18 & 1.00 & 0.94 & 0.95 \\
63865 & 500562 & 2001 & 9.60 & -0.13 & 955.00 & 9397.73 & 1.01 & 0.98 & 0.98 \\
43096 & 109063 & 2001 & 15.40 & -0.39 & 1884.00 & 12528.23 & 0.82 & 0.81 & 0.66 \\
22949 & 103090 & 2001 & 1188.40 & -0.09 & 128416.00 & 1196397.03 & 0.93 & 1.01 & 0.93 \\
48935 & 240154 & 2001 & 52.80 & 0.13 & 5279.00 & 52455.62 & 1.00 & 0.99 & 0.99 \\
63876 & 500563 & 2001 & 7.00 & 0.05 & 702.00 & 6867.91 & 1.00 & 0.98 & 0.98 \\
20523 & 102761 & 2001 & 28725.70 & -0.07 & 3187159.00 & 29161479.23 & 0.90 & 1.02 & 0.91 \\
20450 & 102744 & 2001 & 1498.00 & -0.16 & 101423.00 & 1022167.58 & 1.48 & 0.68 & 1.01 \\
43053 & 109058 & 2001 & 32.60 & -0.12 & 3263.00 & 31315.59 & 1.00 & 0.96 & 0.96 \\
41383 & 108742 & 2001 & 7.20 & -0.22 & 1052.00 & 8475.20 & 0.68 & 1.18 & 0.81 \\
63731 & 500550 & 2001 & 30439.30 & -0.06 & 2486459.00 & 27348400.75 & 1.22 & 0.90 & 1.10 \\
3566 & 100456 & 2001 & 1.70 & -0.16 & 336.00 & 3021.85 & 0.51 & 1.78 & 0.90 \\
43022 & 109051 & 2001 & 70.00 & -0.27 & 6761.00 & 63585.75 & 1.04 & 0.91 & 0.94 \\
2541 & 100343 & 2001 & 459.50 & 0.00 & 46041.00 & 441364.20 & 1.00 & 0.96 & 0.96 \\
20600 & 102774 & 2001 & 6699.70 & -0.34 & 536545.00 & 6794289.09 & 1.25 & 1.01 & 1.27 \\
22795 & 103065 & 2001 & 252.90 & -0.33 & 24627.00 & 240616.39 & 1.03 & 0.95 & 0.98 \\
41424 & 108752 & 2001 & 22.30 & -0.15 & 2035.00 & 20140.50 & 1.10 & 0.90 & 0.99 \\
54888 & 400020 & 2001 & 45.70 & -0.17 & 4559.00 & 42004.06 & 1.00 & 0.92 & 0.92 \\
3579 & 100457 & 2001 & 249.90 & -0.28 & 25082.00 & 243689.37 & 1.00 & 0.98 & 0.97 \\
13484 & 101741 & 2001 & 7148.00 & -0.14 & 654931.00 & 5406526.16 & 1.09 & 0.76 & 0.83 \\
3591 & 100460 & 2001 & 299.90 & 0.03 & 30175.00 & 260064.41 & 0.99 & 0.87 & 0.86 \\
43026 & 109052 & 2001 & 22.60 & -0.12 & 2496.00 & 19828.78 & 0.91 & 0.88 & 0.79 \\
43030 & 109053 & 2001 & 444.50 & -0.20 & 45714.00 & 478122.34 & 0.97 & 1.08 & 1.05 \\
48929 & 240153 & 2001 & 124.50 & -0.12 & 12406.00 & 123643.02 & 1.00 & 0.99 & 1.00 \\
43032 & 109056 & 2001 & 642.30 & 0.10 & 54235.00 & 521741.76 & 1.18 & 0.81 & 0.96 \\
20560 & 102767 & 2001 & 9146.10 & -0.25 & 826259.00 & 9215808.04 & 1.11 & 1.01 & 1.12 \\
22830 & 103067 & 2001 & 95.50 & -0.28 & 9416.00 & 93176.69 & 1.01 & 0.98 & 0.99 \\
61673 & 500114 & 2001 & 66.20 & -0.13 & 7272.00 & 63404.21 & 0.91 & 0.96 & 0.87 \\
61614 & 500107 & 2001 & 282.40 & -0.31 & 22449.00 & 278500.78 & 1.26 & 0.99 & 1.24 \\
43100 & 109064 & 2001 & 101.00 & -0.04 & 7275.00 & 91893.35 & 1.39 & 0.91 & 1.26 \\
22964 & 103099 & 2001 & 569.00 & -0.13 & 56568.00 & 563714.84 & 1.01 & 0.99 & 1.00 \\
20337 & 102716 & 2001 & 1256.60 & -0.03 & 119993.00 & 1196751.51 & 1.05 & 0.95 & 1.00 \\
41238 & 108690 & 2001 & 38.60 & 0.13 & 3469.00 & 32968.89 & 1.11 & 0.85 & 0.95 \\
48494 & 240090 & 2001 & 31.90 & -0.02 & 3182.00 & 31820.05 & 1.00 & 1.00 & 1.00 \\
2419 & 100323 & 2001 & 4671.50 & -0.23 & 422398.00 & 4658781.48 & 1.11 & 1.00 & 1.10 \\
12826 & 101601 & 2001 & 247.70 & -0.10 & 24987.00 & 237356.08 & 0.99 & 0.96 & 0.95 \\
3682 & 100468 & 2001 & 771.30 & 0.16 & 55788.00 & 633990.50 & 1.38 & 0.82 & 1.14 \\
23075 & 103110 & 2001 & 1051.50 & -0.01 & 106881.00 & 953190.10 & 0.98 & 0.91 & 0.89 \\
43226 & 109085 & 2001 & 34.20 & -0.21 & 3385.00 & 33849.66 & 1.01 & 0.99 & 1.00 \\
20303 & 102715 & 2001 & 2934.50 & -0.12 & 283906.00 & 2809196.00 & 1.03 & 0.96 & 0.99 \\
23103 & 103122 & 2001 & 470.20 & -0.17 & 47019.00 & 454067.46 & 1.00 & 0.97 & 0.97 \\
41186 & 108670 & 2001 & 373.40 & -0.22 & 37751.00 & 364640.48 & 0.99 & 0.98 & 0.97 \\
54901 & 400021 & 2001 & 85.00 & -0.16 & 10954.00 & 102937.63 & 0.78 & 1.21 & 0.94 \\
61399 & 500048 & 2001 & 50.10 & -0.18 & 5575.00 & 48470.23 & 0.90 & 0.97 & 0.87 \\
61373 & 500047 & 2001 & 2.10 & -0.16 & 244.00 & 1838.75 & 0.86 & 0.88 & 0.75 \\
7393 & 101038 & 2001 & 3978.30 & -0.18 & 367991.00 & 3788482.78 & 1.08 & 0.95 & 1.03 \\
2399 & 100322 & 2001 & 432.80 & -0.03 & 43172.00 & 392033.87 & 1.00 & 0.91 & 0.91 \\
13554 & 101743 & 2001 & 15082.30 & -0.29 & 1462302.00 & 12016443.69 & 1.03 & 0.80 & 0.82 \\
12805 & 101600 & 2001 & 2950.00 & -0.17 & 298895.00 & 2927627.30 & 0.99 & 0.99 & 0.98 \\
41266 & 108710 & 2001 & 867.90 & -0.43 & 87055.00 & 838370.52 & 1.00 & 0.97 & 0.96 \\
43186 & 109077 & 2001 & 17.80 & -0.43 & 1810.00 & 17370.63 & 0.98 & 0.98 & 0.96 \\
61423 & 500064 & 2001 & 14.70 & 0.15 & 1149.00 & 12667.76 & 1.28 & 0.86 & 1.10 \\
43123 & 109065 & 2001 & 30.00 & -0.15 & 2973.00 & 29394.85 & 1.01 & 0.98 & 0.99 \\
41313 & 108726 & 2001 & 11.20 & 0.17 & 1126.00 & 10650.46 & 0.99 & 0.95 & 0.95 \\
43146 & 109066 & 2001 & 31.70 & -0.21 & 3163.00 & 29456.05 & 1.00 & 0.93 & 0.93 \\
43148 & 109067 & 2001 & 95.90 & -0.31 & 11293.00 & 94033.48 & 0.85 & 0.98 & 0.83 \\
7738 & 101056 & 2001 & 34255.10 & -0.09 & 3716893.00 & 30869847.20 & 0.92 & 0.90 & 0.83 \\
20420 & 102737 & 2001 & 2443.30 & -0.13 & 231216.00 & 2300227.08 & 1.06 & 0.94 & 0.99 \\
43151 & 109069 & 2001 & 107.70 & 0.03 & 10852.00 & 106131.03 & 0.99 & 0.99 & 0.98 \\
22983 & 103100 & 2001 & 478.00 & -0.09 & 47483.00 & 468896.59 & 1.01 & 0.98 & 0.99 \\
43158 & 109070 & 2001 & 1.90 & 0.01 & 187.00 & 1853.66 & 1.02 & 0.98 & 0.99 \\
43160 & 109071 & 2001 & 14.40 & -0.31 & 1428.00 & 14142.34 & 1.01 & 0.98 & 0.99 \\
23001 & 103101 & 2001 & 219.40 & 0.06 & 22124.00 & 209975.66 & 0.99 & 0.96 & 0.95 \\
48584 & 240111 & 2001 & 2418.00 & -0.18 & 252356.00 & 2092095.20 & 0.96 & 0.87 & 0.83 \\
43166 & 109072 & 2001 & 124.80 & -0.12 & 12375.00 & 117997.36 & 1.01 & 0.95 & 0.95 \\
43171 & 109074 & 2001 & 20.90 & -0.35 & 2096.00 & 19822.46 & 1.00 & 0.95 & 0.95 \\
20388 & 102733 & 2001 & 3706.90 & -0.19 & 362118.00 & 3452491.12 & 1.02 & 0.93 & 0.95 \\
41278 & 108719 & 2001 & 97.80 & -0.37 & 9368.00 & 95717.83 & 1.04 & 0.98 & 1.02 \\
41274 & 108716 & 2001 & 18.70 & -0.26 & 1859.00 & 18518.70 & 1.01 & 0.99 & 1.00 \\
48525 & 240103 & 2001 & 112.50 & 0.16 & 9423.00 & 94570.65 & 1.19 & 0.84 & 1.00 \\
61491 & 500083 & 2001 & 9.30 & 0.04 & 907.00 & 8531.00 & 1.03 & 0.92 & 0.94 \\
2442 & 100330 & 2001 & 3094.00 & -0.10 & 268066.00 & 3142599.55 & 1.15 & 1.02 & 1.17 \\
12839 & 101602 & 2001 & 2331.40 & 0.02 & 234555.00 & 2146289.94 & 0.99 & 0.92 & 0.92 \\
23032 & 103103 & 2001 & 409.10 & -0.14 & 40890.00 & 405591.78 & 1.00 & 0.99 & 0.99 \\
61481 & 500082 & 2001 & 94.40 & -0.05 & 8650.00 & 90501.69 & 1.09 & 0.96 & 1.05 \\
41303 & 108723 & 2001 & 83.70 & -0.13 & 8095.00 & 79833.27 & 1.03 & 0.95 & 0.99 \\
20620 & 102775 & 2001 & 2050.70 & -0.02 & 191220.00 & 1775884.18 & 1.07 & 0.87 & 0.93 \\
7704 & 101055 & 2001 & 22741.60 & -0.25 & 2239909.00 & 22349955.13 & 1.02 & 0.98 & 1.00 \\
22487 & 103015 & 2001 & 459.70 & -0.10 & 44448.00 & 470314.31 & 1.03 & 1.02 & 1.06 \\
20858 & 102797 & 2001 & 50.00 & 0.28 & 5054.00 & 46133.38 & 0.99 & 0.92 & 0.91 \\
63475 & 500511 & 2001 & 882.30 & -0.15 & 115486.00 & 1049708.67 & 0.76 & 1.19 & 0.91 \\
48610 & 240114 & 2001 & 365.90 & -0.19 & 36491.00 & 364907.35 & 1.00 & 1.00 & 1.00 \\
22502 & 103016 & 2001 & 278.60 & -0.01 & 25499.00 & 267536.94 & 1.09 & 0.96 & 1.05 \\
41669 & 108827 & 2001 & 723.30 & -0.21 & 72509.00 & 664872.25 & 1.00 & 0.92 & 0.92 \\
41644 & 108826 & 2001 & 211.00 & -0.03 & 21238.00 & 200284.58 & 0.99 & 0.95 & 0.94 \\
22529 & 103017 & 2001 & 2226.10 & -0.12 & 293995.00 & 2650894.38 & 0.76 & 1.19 & 0.90 \\
48603 & 240113 & 2001 & 59.00 & -0.13 & 5667.00 & 56666.12 & 1.04 & 0.96 & 1.00 \\
3463 & 100439 & 2001 & 22.70 & -0.28 & 2054.00 & 16920.67 & 1.11 & 0.75 & 0.82 \\
42864 & 109026 & 2001 & 8.20 & -0.72 & 822.00 & 8223.38 & 1.00 & 1.00 & 1.00 \\
42866 & 109028 & 2001 & 219.60 & -0.02 & 22502.00 & 199363.83 & 0.98 & 0.91 & 0.89 \\
2690 & 100352 & 2001 & 1798.40 & -0.11 & 201450.00 & 1860438.82 & 0.89 & 1.03 & 0.92 \\
20808 & 102795 & 2001 & 350.40 & 0.12 & 31384.00 & 303114.46 & 1.12 & 0.87 & 0.97 \\
63499 & 500512 & 2001 & 692.00 & -0.02 & 73981.00 & 737928.13 & 0.94 & 1.07 & 1.00 \\
2670 & 100351 & 2001 & 98.60 & -0.17 & 9687.00 & 96811.29 & 1.02 & 0.98 & 1.00 \\
42826 & 109023 & 2001 & 8.90 & -0.06 & 779.00 & 7790.37 & 1.14 & 0.88 & 1.00 \\
42824 & 109022 & 2001 & 8.20 & -0.15 & 817.00 & 8168.37 & 1.00 & 1.00 & 1.00 \\
42821 & 109021 & 2001 & 21.80 & 0.02 & 1773.00 & 19912.47 & 1.23 & 0.91 & 1.12 \\
41677 & 108839 & 2001 & 613.90 & -0.56 & 61031.00 & 609432.51 & 1.01 & 0.99 & 1.00 \\
20949 & 102813 & 2001 & 747.90 & 0.19 & 74783.00 & 669464.72 & 1.00 & 0.90 & 0.90 \\
63350 & 500500 & 2001 & 167.00 & -0.11 & 14454.00 & 163789.08 & 1.16 & 0.98 & 1.13 \\
63368 & 500502 & 2001 & 25.10 & -0.18 & 2491.00 & 21795.57 & 1.01 & 0.87 & 0.87 \\
63401 & 500506 & 2001 & 12.00 & -0.01 & 1249.00 & 10012.88 & 0.96 & 0.83 & 0.80 \\
22414 & 103011 & 2001 & 66.70 & -0.08 & 7104.00 & 62920.59 & 0.94 & 0.94 & 0.89 \\
2722 & 100355 & 2001 & 5945.40 & -0.08 & 650121.00 & 5965393.68 & 0.91 & 1.00 & 0.92 \\
20928 & 102802 & 2001 & 275.80 & 0.01 & 27574.00 & 260951.60 & 1.00 & 0.95 & 0.95 \\
13001 & 101618 & 2001 & 200.10 & -0.14 & 20137.00 & 196294.39 & 0.99 & 0.98 & 0.97 \\
22570 & 103021 & 2001 & 65.00 & 0.23 & 6429.00 & 56952.19 & 1.01 & 0.88 & 0.89 \\
54843 & 400018 & 2001 & 168.40 & -0.11 & 16752.00 & 158097.25 & 1.01 & 0.94 & 0.94 \\
41726 & 108849 & 2001 & 621.70 & 0.05 & 55095.00 & 532921.89 & 1.13 & 0.86 & 0.97 \\
41722 & 108841 & 2001 & 110.30 & -0.27 & 13480.00 & 108463.46 & 0.82 & 0.98 & 0.80 \\
41702 & 108840 & 2001 & 53.10 & 0.07 & 4776.00 & 49517.05 & 1.11 & 0.93 & 1.04 \\
42798 & 109019 & 2001 & 63.70 & 0.10 & 5676.00 & 59699.30 & 1.12 & 0.94 & 1.05 \\
63431 & 500508 & 2001 & 3024.00 & -0.04 & 273012.00 & 3014793.95 & 1.11 & 1.00 & 1.10 \\
20886 & 102798 & 2001 & 499.20 & -0.25 & 75126.00 & 668047.57 & 0.66 & 1.34 & 0.89 \\
7641 & 101050 & 2001 & 424.80 & -0.12 & 43670.00 & 400211.53 & 0.97 & 0.94 & 0.92 \\
13415 & 101738 & 2001 & 3209.20 & -0.40 & 306135.00 & 3142383.52 & 1.05 & 0.98 & 1.03 \\
22469 & 103014 & 2001 & 316.50 & -0.10 & 28375.00 & 278530.44 & 1.12 & 0.88 & 0.98 \\
20914 & 102799 & 2001 & 81.00 & -0.20 & 14213.00 & 126775.70 & 0.57 & 1.57 & 0.89 \\
23143 & 103134 & 2001 & 490.40 & -0.13 & 49049.00 & 467489.99 & 1.00 & 0.95 & 0.95 \\
42882 & 109030 & 2001 & 49.10 & -0.22 & 4392.00 & 46687.09 & 1.12 & 0.95 & 1.06 \\
41607 & 108780 & 2001 & 97.70 & -0.12 & 8889.00 & 96769.07 & 1.10 & 0.99 & 1.09 \\
2632 & 100348 & 2001 & 336.00 & 0.02 & 32722.00 & 325983.06 & 1.03 & 0.97 & 1.00 \\
42965 & 109044 & 2001 & 65.90 & -0.09 & 6391.00 & 63618.54 & 1.03 & 0.97 & 1.00 \\
41532 & 108764 & 2001 & 329.80 & -0.13 & 33665.00 & 325298.28 & 0.98 & 0.99 & 0.97 \\
41485 & 108761 & 2001 & 424.80 & 0.48 & 35974.00 & 347713.44 & 1.18 & 0.82 & 0.97 \\
61696 & 500115 & 2001 & 17.10 & -0.05 & 2162.00 & 16943.77 & 0.79 & 0.99 & 0.78 \\
13459 & 101740 & 2001 & 25622.00 & -0.26 & 2350788.00 & 25512355.09 & 1.09 & 1.00 & 1.09 \\
20656 & 102777 & 2001 & 3283.50 & -0.46 & 386677.00 & 3173664.08 & 0.85 & 0.97 & 0.82 \\
2613 & 100347 & 2001 & 963.20 & -0.12 & 93254.00 & 930885.81 & 1.03 & 0.97 & 1.00 \\
48594 & 240112 & 2001 & 24.60 & -0.48 & 2497.00 & 24148.46 & 0.99 & 0.98 & 0.97 \\
12916 & 101606 & 2001 & 5336.90 & -0.31 & 543169.00 & 5125439.90 & 0.98 & 0.96 & 0.94 \\
63649 & 500539 & 2001 & 8.40 & -0.08 & 814.00 & 8138.15 & 1.03 & 0.97 & 1.00 \\
3557 & 100455 & 2001 & 4.20 & -0.09 & 466.00 & 3559.92 & 0.90 & 0.85 & 0.76 \\
42979 & 109046 & 2001 & 106.70 & -0.28 & 8383.00 & 106995.70 & 1.27 & 1.00 & 1.28 \\
41460 & 108760 & 2001 & 116.40 & -0.11 & 12102.00 & 120787.62 & 0.96 & 1.04 & 1.00 \\
43003 & 109048 & 2001 & 245.50 & -0.20 & 24128.00 & 257461.86 & 1.02 & 1.05 & 1.07 \\
41436 & 108759 & 2001 & 36.10 & 0.13 & 3542.00 & 35422.16 & 1.02 & 0.98 & 1.00 \\
63526 & 500514 & 2001 & 112.80 & -0.08 & 13661.00 & 128408.40 & 0.83 & 1.14 & 0.94 \\
61710 & 500116 & 2001 & 444.60 & 0.13 & 29122.00 & 357953.72 & 1.53 & 0.81 & 1.23 \\
41550 & 108765 & 2001 & 1221.30 & -0.12 & 103079.00 & 1092249.76 & 1.18 & 0.89 & 1.06 \\
41597 & 108777 & 2001 & 28.30 & -0.02 & 2826.00 & 28244.67 & 1.00 & 1.00 & 1.00 \\
42908 & 109033 & 2001 & 21.80 & -0.97 & 1495.00 & 14946.56 & 1.46 & 0.69 & 1.00 \\
20769 & 102789 & 2001 & 1052.50 & -0.08 & 106404.00 & 894662.01 & 0.99 & 0.85 & 0.84 \\
12951 & 101616 & 2001 & 22318.60 & -0.16 & 2253072.00 & 21242226.14 & 0.99 & 0.95 & 0.94 \\
3497 & 100441 & 2001 & 478.40 & -0.14 & 47710.00 & 471228.10 & 1.00 & 0.99 & 0.99 \\
61753 & 500120 & 2001 & 15.80 & -0.13 & 1438.00 & 15494.83 & 1.10 & 0.98 & 1.08 \\
54863 & 400019 & 2001 & 786.20 & 0.02 & 58864.00 & 527437.62 & 1.34 & 0.67 & 0.90 \\
3526 & 100453 & 2001 & 166.10 & -0.15 & 16792.00 & 150661.10 & 0.99 & 0.91 & 0.90 \\
22601 & 103024 & 2001 & 567.40 & 0.18 & 56777.00 & 508999.86 & 1.00 & 0.90 & 0.90 \\
42932 & 109037 & 2001 & 127.70 & -0.32 & 13005.00 & 113299.44 & 0.98 & 0.89 & 0.87 \\
41621 & 108782 & 2001 & 613.80 & -0.07 & 62351.00 & 602528.37 & 0.98 & 0.98 & 0.97 \\
42942 & 109038 & 2001 & 26.80 & -0.07 & 2243.00 & 25514.83 & 1.19 & 0.95 & 1.14 \\
41572 & 108776 & 2001 & 253.80 & 0.13 & 20616.00 & 219457.41 & 1.23 & 0.86 & 1.06 \\
20730 & 102788 & 2001 & 224.80 & -0.02 & 21060.00 & 199555.72 & 1.07 & 0.89 & 0.95 \\
7431 & 101039 & 2001 & 4988.10 & -0.10 & 455627.00 & 4507801.52 & 1.09 & 0.90 & 0.99 \\
63523 & 500513 & 2001 & 79.00 & -0.08 & 8002.00 & 77666.34 & 0.99 & 0.98 & 0.97 \\
2651 & 100350 & 2001 & 261.40 & -0.25 & 25992.00 & 259921.92 & 1.01 & 0.99 & 1.00 \\
42950 & 109039 & 2001 & 8.00 & -0.10 & 746.00 & 6767.53 & 1.07 & 0.85 & 0.91 \\
42953 & 109040 & 2001 & 7.80 & -0.08 & 844.00 & 7402.32 & 0.92 & 0.95 & 0.88 \\
20711 & 102784 & 2001 & 25469.80 & -0.23 & 2342503.00 & 25524888.64 & 1.09 & 1.00 & 1.09 \\
22638 & 103027 & 2001 & 6624.70 & -0.05 & 589545.00 & 5622878.22 & 1.12 & 0.85 & 0.95 \\
41565 & 108773 & 2001 & 45.70 & 0.28 & 4599.00 & 44206.13 & 0.99 & 0.97 & 0.96 \\
41557 & 108766 & 2001 & 338.40 & -0.23 & 27312.00 & 351712.61 & 1.24 & 1.04 & 1.29 \\
7666 & 101054 & 2001 & 12452.30 & -0.11 & 1447569.00 & 12571838.97 & 0.86 & 1.01 & 0.87 \\
42235 & 108944 & 2001 & 14.50 & -0.16 & 1762.00 & 12979.14 & 0.82 & 0.90 & 0.74 \\
49564 & 240313 & 2001 & 11.70 & -0.13 & 1174.00 & 11468.96 & 1.00 & 0.98 & 0.98 \\
64304 & 500596 & 2001 & 502.20 & -0.07 & 38774.00 & 421071.26 & 1.30 & 0.84 & 1.09 \\
29755 & 105644 & 2001 & 89.70 & -0.03 & 8883.00 & 84541.72 & 1.01 & 0.94 & 0.95 \\
10672 & 101307 & 2001 & 513.10 & 0.07 & 51325.00 & 470809.83 & 1.00 & 0.92 & 0.92 \\
35512 & 106369 & 2001 & 197.60 & -0.08 & 19239.00 & 187840.28 & 1.03 & 0.95 & 0.98 \\
29733 & 105643 & 2001 & 1189.50 & -0.01 & 116786.00 & 1096633.00 & 1.02 & 0.92 & 0.94 \\
35520 & 106370 & 2001 & 56.10 & -0.13 & 6380.00 & 63783.54 & 0.88 & 1.14 & 1.00 \\
35547 & 106372 & 2001 & 9.10 & -0.13 & 764.00 & 8020.43 & 1.19 & 0.88 & 1.05 \\
35555 & 106375 & 2001 & 26.60 & -0.27 & 2335.00 & 23198.90 & 1.14 & 0.87 & 0.99 \\
29683 & 105635 & 2001 & 196.90 & -0.28 & 19639.00 & 196388.33 & 1.00 & 1.00 & 1.00 \\
35580 & 106376 & 2001 & 7.20 & -0.46 & 616.00 & 7219.11 & 1.17 & 1.00 & 1.17 \\
29659 & 105631 & 2001 & 6.40 & -0.35 & 1060.00 & 8490.20 & 0.60 & 1.33 & 0.80 \\
10704 & 101312 & 2001 & 9884.20 & -0.21 & 988416.00 & 8582896.74 & 1.00 & 0.87 & 0.87 \\
29647 & 105630 & 2001 & 5.70 & -0.39 & 501.00 & 4276.65 & 1.14 & 0.75 & 0.85 \\
35603 & 106379 & 2001 & 238.50 & -0.14 & 23884.00 & 216121.66 & 1.00 & 0.91 & 0.90 \\
35613 & 106380 & 2001 & 103.20 & 0.06 & 10235.00 & 101857.39 & 1.01 & 0.99 & 1.00 \\
29765 & 105645 & 2001 & 5208.90 & -0.10 & 522359.00 & 4177818.11 & 1.00 & 0.80 & 0.80 \\
35495 & 106366 & 2001 & 23.80 & -0.10 & 1758.00 & 19557.39 & 1.35 & 0.82 & 1.11 \\
35489 & 106365 & 2001 & 20.80 & 0.16 & 1644.00 & 18740.85 & 1.27 & 0.90 & 1.14 \\
231 & 100019 & 2001 & 4960.80 & 0.05 & 494780.00 & 4285032.92 & 1.00 & 0.86 & 0.87 \\
35350 & 106353 & 2001 & 151.70 & 0.19 & 14299.00 & 142986.96 & 1.06 & 0.94 & 1.00 \\
35384 & 106356 & 2001 & 19.30 & 0.19 & 1606.00 & 17963.59 & 1.20 & 0.93 & 1.12 \\
29895 & 105656 & 2001 & 154.00 & 0.12 & 14874.00 & 146150.75 & 1.04 & 0.95 & 0.98 \\
35410 & 106358 & 2001 & 4.10 & -0.00 & 405.00 & 4054.09 & 1.01 & 0.99 & 1.00 \\
51092 & 240482 & 2001 & 2.20 & -0.03 & 221.00 & 1983.42 & 1.00 & 0.90 & 0.90 \\
35417 & 106359 & 2001 & 309.30 & -0.22 & 30925.00 & 307900.91 & 1.00 & 1.00 & 1.00 \\
29866 & 105655 & 2001 & 725.10 & 0.09 & 67867.00 & 622915.86 & 1.07 & 0.86 & 0.92 \\
29618 & 105627 & 2001 & 536.40 & 0.26 & 53807.00 & 535660.67 & 1.00 & 1.00 & 1.00 \\
74910 & 601197 & 2001 & 44.60 & -0.18 & 4177.00 & 41764.66 & 1.07 & 0.94 & 1.00 \\
35425 & 106360 & 2001 & 114.70 & 0.11 & 9518.00 & 99198.57 & 1.21 & 0.86 & 1.04 \\
29849 & 105654 & 2001 & 98.40 & -0.15 & 11158.00 & 101896.59 & 0.88 & 1.04 & 0.91 \\
10642 & 101302 & 2001 & 399.30 & 0.13 & 39932.00 & 385312.48 & 1.00 & 0.96 & 0.96 \\
35452 & 106361 & 2001 & 167.90 & -0.21 & 12975.00 & 158337.31 & 1.29 & 0.94 & 1.22 \\
29823 & 105652 & 2001 & 544.40 & -0.11 & 47824.00 & 526837.56 & 1.14 & 0.97 & 1.10 \\
35474 & 106363 & 2001 & 339.00 & -0.25 & 34183.00 & 323246.71 & 0.99 & 0.95 & 0.95 \\
35482 & 106364 & 2001 & 16.70 & -0.13 & 1666.00 & 14763.48 & 1.00 & 0.88 & 0.89 \\
53100 & 338393 & 2001 & 26.80 & -0.29 & 3297.00 & 24853.41 & 0.81 & 0.93 & 0.75 \\
29606 & 105623 & 2001 & 26.10 & -0.15 & 4667.00 & 37933.96 & 0.56 & 1.45 & 0.81 \\
35787 & 106402 & 2001 & 29.30 & 0.13 & 2443.00 & 22825.81 & 1.20 & 0.78 & 0.93 \\
35805 & 106413 & 2001 & 1997.30 & -0.03 & 193554.00 & 1935244.62 & 1.03 & 0.97 & 1.00 \\
29414 & 105594 & 2001 & 3.20 & 0.14 & 285.00 & 3140.87 & 1.12 & 0.98 & 1.10 \\
8680 & 101094 & 2001 & 964.20 & -0.07 & 86209.00 & 932936.63 & 1.12 & 0.97 & 1.08 \\
29407 & 105593 & 2001 & 57.80 & -0.07 & 5770.00 & 53081.79 & 1.00 & 0.92 & 0.92 \\
29400 & 105592 & 2001 & 363.60 & -0.24 & 36445.00 & 362169.13 & 1.00 & 1.00 & 0.99 \\
10801 & 101331 & 2001 & 86.90 & -0.13 & 7754.00 & 67527.03 & 1.12 & 0.78 & 0.87 \\
35831 & 106415 & 2001 & 441.00 & -0.22 & 44583.00 & 368507.56 & 0.99 & 0.84 & 0.83 \\
35843 & 106417 & 2001 & 198.60 & -0.12 & 19794.00 & 192357.57 & 1.00 & 0.97 & 0.97 \\
35847 & 106418 & 2001 & 741.10 & -0.26 & 74062.00 & 718829.39 & 1.00 & 0.97 & 0.97 \\
29362 & 105589 & 2001 & 264.70 & -0.19 & 27275.00 & 259122.01 & 0.97 & 0.98 & 0.95 \\
29356 & 105588 & 2001 & 9.00 & -0.21 & 844.00 & 8268.40 & 1.07 & 0.92 & 0.98 \\
35861 & 106419 & 2001 & 1709.50 & -0.12 & 170936.00 & 1687091.11 & 1.00 & 0.99 & 0.99 \\
35866 & 106420 & 2001 & 44.60 & -0.14 & 5226.00 & 41914.85 & 0.85 & 0.94 & 0.80 \\
35874 & 106421 & 2001 & 16.00 & 0.07 & 1595.00 & 14328.53 & 1.00 & 0.90 & 0.90 \\
29330 & 105587 & 2001 & 46.80 & 0.30 & 4125.00 & 39709.05 & 1.13 & 0.85 & 0.96 \\
35761 & 106401 & 2001 & 708.00 & 0.05 & 58494.00 & 595509.53 & 1.21 & 0.84 & 1.02 \\
35755 & 106400 & 2001 & 252.30 & -0.26 & 25414.00 & 239537.95 & 0.99 & 0.95 & 0.94 \\
29462 & 105597 & 2001 & 106.10 & -0.18 & 10627.00 & 102495.88 & 1.00 & 0.97 & 0.96 \\
29577 & 105616 & 2001 & 24.60 & -0.15 & 2553.00 & 22380.09 & 0.96 & 0.91 & 0.88 \\
275 & 100030 & 2001 & 372.70 & -0.42 & 36704.00 & 378093.08 & 1.02 & 1.01 & 1.03 \\
29548 & 105611 & 2001 & 264.80 & 0.21 & 26485.00 & 220228.82 & 1.00 & 0.83 & 0.83 \\
29536 & 105610 & 2001 & 341.40 & -0.23 & 34431.00 & 344308.87 & 0.99 & 1.01 & 1.00 \\
29529 & 105607 & 2001 & 6.10 & -0.52 & 582.00 & 5821.69 & 1.05 & 0.95 & 1.00 \\
35640 & 106381 & 2001 & 4.80 & 0.12 & 479.00 & 4340.78 & 1.00 & 0.90 & 0.91 \\
29520 & 105604 & 2001 & 19.00 & -0.25 & 1761.00 & 18614.00 & 1.08 & 0.98 & 1.06 \\
29514 & 105603 & 2001 & 2.10 & -0.13 & 205.00 & 1988.06 & 1.02 & 0.95 & 0.97 \\
35704 & 106391 & 2001 & 72.20 & -0.10 & 7818.00 & 70990.70 & 0.92 & 0.98 & 0.91 \\
10769 & 101330 & 2001 & 2851.00 & 0.02 & 284790.00 & 2727625.27 & 1.00 & 0.96 & 0.96 \\
35731 & 106392 & 2001 & 115.70 & -0.22 & 10495.00 & 116016.58 & 1.10 & 1.00 & 1.11 \\
29488 & 105598 & 2001 & 387.60 & -0.09 & 38140.00 & 381384.69 & 1.02 & 0.98 & 1.00 \\
50830 & 240458 & 2001 & 288.10 & 0.07 & 26110.00 & 243485.47 & 1.10 & 0.85 & 0.93 \\
35736 & 106394 & 2001 & 117.00 & -0.19 & 10946.00 & 110174.28 & 1.07 & 0.94 & 1.01 \\
35747 & 106398 & 2001 & 5.50 & -0.15 & 539.00 & 5266.68 & 1.02 & 0.96 & 0.98 \\
53234 & 342127 & 2001 & 143.90 & -0.01 & 14381.00 & 138883.30 & 1.00 & 0.97 & 0.97 \\
29929 & 105658 & 2001 & 139.60 & -0.26 & 16270.00 & 136240.12 & 0.86 & 0.98 & 0.84 \\
29944 & 105659 & 2001 & 213.70 & -0.11 & 22852.00 & 231393.72 & 0.94 & 1.08 & 1.01 \\
10384 & 101284 & 2001 & 1894.00 & -0.24 & 189399.00 & 1753814.64 & 1.00 & 0.93 & 0.93 \\
35034 & 106309 & 2001 & 312.50 & 0.16 & 24251.00 & 222914.10 & 1.29 & 0.71 & 0.92 \\
30396 & 105753 & 2001 & 53.80 & -0.00 & 5311.00 & 49999.76 & 1.01 & 0.93 & 0.94 \\
30374 & 105746 & 2001 & 192.10 & -0.12 & 19142.00 & 185679.88 & 1.00 & 0.97 & 0.97 \\
160 & 100016 & 2001 & 191.80 & -0.16 & 19173.00 & 187634.62 & 1.00 & 0.98 & 0.98 \\
30357 & 105741 & 2001 & 193.00 & -0.13 & 19300.00 & 180903.10 & 1.00 & 0.94 & 0.94 \\
35061 & 106310 & 2001 & 10.40 & -0.13 & 1137.00 & 10950.46 & 0.91 & 1.05 & 0.96 \\
35067 & 106317 & 2001 & 219.40 & -0.26 & 21702.00 & 210550.44 & 1.01 & 0.96 & 0.97 \\
35076 & 106318 & 2001 & 146.60 & -0.11 & 12455.00 & 140602.23 & 1.18 & 0.96 & 1.13 \\
30324 & 105737 & 2001 & 5.20 & -0.15 & 517.00 & 5086.07 & 1.01 & 0.98 & 0.98 \\
35117 & 106321 & 2001 & 5.10 & -0.11 & 526.00 & 4897.74 & 0.97 & 0.96 & 0.93 \\
30295 & 105731 & 2001 & 2167.20 & -0.08 & 208850.00 & 1918587.79 & 1.04 & 0.89 & 0.92 \\
35021 & 106306 & 2001 & 22.30 & -0.14 & 2199.00 & 21983.55 & 1.01 & 0.99 & 1.00 \\
30593 & 105775 & 2001 & 2098.60 & -0.18 & 210191.00 & 1899047.57 & 1.00 & 0.90 & 0.90 \\
10323 & 101279 & 2001 & 52.00 & 0.23 & 3730.00 & 45863.87 & 1.39 & 0.88 & 1.23 \\
34886 & 106284 & 2001 & 245.20 & -0.27 & 22734.00 & 230951.34 & 1.08 & 0.94 & 1.02 \\
34901 & 106292 & 2001 & 16.70 & 0.16 & 1753.00 & 17323.47 & 0.95 & 1.04 & 0.99 \\
30564 & 105770 & 2001 & 26.80 & -0.11 & 2677.00 & 22058.28 & 1.00 & 0.82 & 0.82 \\
30554 & 105769 & 2001 & 62.80 & -0.13 & 4547.00 & 59840.29 & 1.38 & 0.95 & 1.32 \\
96664 & 611002 & 2001 & 4247.60 & -0.15 & 484914.00 & 4118606.11 & 0.88 & 0.97 & 0.85 \\
34927 & 106293 & 2001 & 37.30 & -0.27 & 5296.00 & 52118.38 & 0.70 & 1.40 & 0.98 \\
30269 & 105723 & 2001 & 322.30 & 0.06 & 23983.00 & 258566.02 & 1.34 & 0.80 & 1.08 \\
34934 & 106294 & 2001 & 81.90 & 0.06 & 8312.00 & 80434.60 & 0.99 & 0.98 & 0.97 \\
34961 & 106297 & 2001 & 20.40 & 0.04 & 2711.00 & 27226.08 & 0.75 & 1.33 & 1.00 \\
34972 & 106298 & 2001 & 23.60 & -0.03 & 2568.00 & 23053.86 & 0.92 & 0.98 & 0.90 \\
30498 & 105762 & 2001 & 357.30 & -0.15 & 36629.00 & 342584.00 & 0.98 & 0.96 & 0.94 \\
10354 & 101283 & 2001 & 1777.90 & 0.18 & 186428.00 & 1475346.57 & 0.95 & 0.83 & 0.79 \\
34990 & 106303 & 2001 & 6.60 & -0.11 & 663.00 & 6463.26 & 1.00 & 0.98 & 0.97 \\
30470 & 105761 & 2001 & 967.00 & -0.04 & 103987.00 & 987844.82 & 0.93 & 1.02 & 0.95 \\
34994 & 106305 & 2001 & 105.70 & 0.11 & 12234.00 & 114921.27 & 0.86 & 1.09 & 0.94 \\
8760 & 101097 & 2001 & 558.60 & -0.28 & 61299.00 & 544892.75 & 0.91 & 0.98 & 0.89 \\
30267 & 105722 & 2001 & 19.40 & -0.12 & 1926.00 & 17155.51 & 1.01 & 0.88 & 0.89 \\
35144 & 106329 & 2001 & 27.20 & -0.22 & 2643.00 & 26427.21 & 1.03 & 0.97 & 1.00 \\
30092 & 105684 & 2001 & 34.70 & -0.12 & 3192.00 & 32985.32 & 1.09 & 0.95 & 1.03 \\
30085 & 105682 & 2001 & 294.70 & -0.15 & 29320.00 & 286770.95 & 1.01 & 0.97 & 0.98 \\
51288 & 240498 & 2001 & 59.00 & 0.00 & 5772.00 & 56168.64 & 1.02 & 0.95 & 0.97 \\
35275 & 106341 & 2001 & 34.50 & -0.22 & 3940.00 & 38901.81 & 0.88 & 1.13 & 0.99 \\
35279 & 106344 & 2001 & 813.10 & 0.11 & 66169.00 & 659156.66 & 1.23 & 0.81 & 1.00 \\
30047 & 105679 & 2001 & 428.20 & -0.17 & 42944.00 & 426847.06 & 1.00 & 1.00 & 0.99 \\
35306 & 106345 & 2001 & 365.60 & 0.20 & 36402.00 & 364082.87 & 1.00 & 1.00 & 1.00 \\
35321 & 106347 & 2001 & 109.60 & -0.24 & 9696.00 & 112595.63 & 1.13 & 1.03 & 1.16 \\
30019 & 105678 & 2001 & 17.20 & 0.11 & 1787.00 & 15506.28 & 0.96 & 0.90 & 0.87 \\
30010 & 105677 & 2001 & 53.40 & -0.08 & 5327.00 & 50665.87 & 1.00 & 0.95 & 0.95 \\
10568 & 101299 & 2001 & 2361.80 & -0.10 & 236185.00 & 2237841.14 & 1.00 & 0.95 & 0.95 \\
30002 & 105676 & 2001 & 329.80 & 0.26 & 25775.00 & 296531.06 & 1.28 & 0.90 & 1.15 \\
29990 & 105665 & 2001 & 110.80 & -0.18 & 11247.00 & 109231.29 & 0.99 & 0.99 & 0.97 \\
35334 & 106348 & 2001 & 134.80 & -0.22 & 10928.00 & 133915.66 & 1.23 & 0.99 & 1.23 \\
29972 & 105664 & 2001 & 134.90 & -0.07 & 11527.00 & 132462.92 & 1.17 & 0.98 & 1.15 \\
29956 & 105662 & 2001 & 104.80 & -0.11 & 10443.00 & 104433.85 & 1.00 & 1.00 & 1.00 \\
196 & 100018 & 2001 & 190.60 & -0.14 & 19167.00 & 179283.02 & 0.99 & 0.94 & 0.94 \\
30101 & 105686 & 2001 & 59.60 & -0.39 & 5968.00 & 56757.38 & 1.00 & 0.95 & 0.95 \\
30108 & 105694 & 2001 & 31.90 & 0.18 & 3195.00 & 31096.55 & 1.00 & 0.97 & 0.97 \\
10463 & 101286 & 2001 & 1244.40 & -0.23 & 124446.00 & 1037434.54 & 1.00 & 0.83 & 0.83 \\
35170 & 106330 & 2001 & 137.70 & 0.07 & 9983.00 & 111185.94 & 1.38 & 0.81 & 1.11 \\
30236 & 105720 & 2001 & 1461.60 & -0.48 & 146118.00 & 1343321.57 & 1.00 & 0.92 & 0.92 \\
30227 & 105718 & 2001 & 50.40 & -0.66 & 9487.00 & 43418.03 & 0.53 & 0.86 & 0.46 \\
30206 & 105708 & 2001 & 172.00 & -0.52 & 13580.00 & 131871.86 & 1.27 & 0.77 & 0.97 \\
178 & 100017 & 2001 & 143.80 & -0.10 & 14153.00 & 138959.43 & 1.02 & 0.97 & 0.98 \\
8821 & 101100 & 2001 & 515.40 & -0.21 & 46768.00 & 452977.57 & 1.10 & 0.88 & 0.97 \\
35186 & 106333 & 2001 & 91.30 & 0.10 & 7442.00 & 89076.95 & 1.23 & 0.98 & 1.20 \\
30179 & 105705 & 2001 & 124.90 & -0.22 & 15155.00 & 107072.26 & 0.82 & 0.86 & 0.71 \\
35221 & 106335 & 2001 & 458.90 & -0.32 & 45898.00 & 450349.81 & 1.00 & 0.98 & 0.98 \\
30147 & 105703 & 2001 & 272.30 & 0.41 & 27228.00 & 230777.50 & 1.00 & 0.85 & 0.85 \\
52993 & 336593 & 2001 & 5.00 & 0.01 & 570.00 & 4954.64 & 0.88 & 0.99 & 0.87 \\
30139 & 105702 & 2001 & 76.70 & -0.13 & 6612.00 & 78320.18 & 1.16 & 1.02 & 1.18 \\
30126 & 105701 & 2001 & 946.50 & -0.20 & 95032.00 & 934555.03 & 1.00 & 0.99 & 0.98 \\
30116 & 105700 & 2001 & 257.40 & 0.05 & 25736.00 & 239554.62 & 1.00 & 0.93 & 0.93 \\
35892 & 106422 & 2001 & 182.50 & -0.00 & 18536.00 & 171922.74 & 0.98 & 0.94 & 0.93 \\
35898 & 106424 & 2001 & 464.70 & 0.08 & 46204.00 & 426419.12 & 1.01 & 0.92 & 0.92 \\
29304 & 105585 & 2001 & 11.90 & -0.08 & 1119.00 & 12237.27 & 1.06 & 1.03 & 1.09 \\
74776 & 601168 & 2001 & 64.60 & -0.21 & 6466.00 & 59335.89 & 1.00 & 0.92 & 0.92 \\
8568 & 101090 & 2001 & 1159.10 & -0.27 & 121207.00 & 1168039.84 & 0.96 & 1.01 & 0.96 \\
28540 & 105437 & 2001 & 2052.10 & -0.07 & 200276.00 & 1930450.13 & 1.02 & 0.94 & 0.96 \\
11123 & 101368 & 2001 & 1270.40 & 0.05 & 128359.00 & 1113148.76 & 0.99 & 0.88 & 0.87 \\
36536 & 106560 & 2001 & 26.40 & -0.12 & 2782.00 & 25204.63 & 0.95 & 0.95 & 0.91 \\
28518 & 105432 & 2001 & 11.60 & -0.03 & 1157.00 & 11287.75 & 1.00 & 0.97 & 0.98 \\
418 & 100055 & 2001 & 12283.50 & -0.08 & 1041358.00 & 11473947.33 & 1.18 & 0.93 & 1.10 \\
36555 & 106561 & 2001 & 3.60 & 0.11 & 349.00 & 3434.52 & 1.03 & 0.95 & 0.98 \\
28489 & 105427 & 2001 & 205.70 & -0.28 & 20179.00 & 192380.60 & 1.02 & 0.94 & 0.95 \\
36586 & 106567 & 2001 & 2.20 & -0.11 & 316.00 & 2787.84 & 0.70 & 1.27 & 0.88 \\
28460 & 105426 & 2001 & 933.80 & 0.12 & 89561.00 & 873797.70 & 1.04 & 0.94 & 0.98 \\
36593 & 106568 & 2001 & 15.40 & 0.10 & 904.00 & 9091.22 & 1.70 & 0.59 & 1.01 \\
53431 & 349198 & 2001 & 443.80 & -0.25 & 36864.00 & 368593.00 & 1.20 & 0.83 & 1.00 \\
36619 & 106569 & 2001 & 40.30 & -0.39 & 4027.00 & 37487.09 & 1.00 & 0.93 & 0.93 \\
391 & 100048 & 2001 & 378.10 & -0.19 & 37869.00 & 358506.69 & 1.00 & 0.95 & 0.95 \\
28586 & 105448 & 2001 & 78.60 & 0.10 & 7345.00 & 70763.06 & 1.07 & 0.90 & 0.96 \\
28705 & 105469 & 2001 & 54.30 & -0.20 & 5436.00 & 50004.48 & 1.00 & 0.92 & 0.92 \\
74796 & 601172 & 2001 & 75.00 & 0.08 & 4002.00 & 38701.08 & 1.87 & 0.52 & 0.97 \\
28699 & 105465 & 2001 & 45.30 & 0.13 & 4001.00 & 47250.35 & 1.13 & 1.04 & 1.18 \\
28689 & 105464 & 2001 & 24.80 & 0.11 & 2082.00 & 17935.66 & 1.19 & 0.72 & 0.86 \\
36449 & 106529 & 2001 & 186.30 & -0.12 & 15828.00 & 158698.65 & 1.18 & 0.85 & 1.00 \\
36624 & 106570 & 2001 & 27.30 & -0.40 & 2533.00 & 24862.56 & 1.08 & 0.91 & 0.98 \\
28651 & 105458 & 2001 & 510.50 & 0.01 & 42310.00 & 497886.65 & 1.21 & 0.98 & 1.18 \\
28636 & 105457 & 2001 & 1949.50 & -0.22 & 194844.00 & 1929923.12 & 1.00 & 0.99 & 0.99 \\
11089 & 101367 & 2001 & 440.20 & 0.05 & 44085.00 & 408843.74 & 1.00 & 0.93 & 0.93 \\
28615 & 105450 & 2001 & 35.80 & -0.15 & 4106.00 & 30250.11 & 0.87 & 0.84 & 0.74 \\
36510 & 106545 & 2001 & 23.10 & -0.52 & 2278.00 & 20034.27 & 1.01 & 0.87 & 0.88 \\
36470 & 106535 & 2001 & 161.20 & 0.03 & 13824.00 & 142843.07 & 1.17 & 0.89 & 1.03 \\
28718 & 105471 & 2001 & 4.30 & -0.14 & 410.00 & 4097.26 & 1.05 & 0.95 & 1.00 \\
11159 & 101369 & 2001 & 2957.80 & -0.24 & 295595.00 & 2955827.87 & 1.00 & 1.00 & 1.00 \\
36628 & 106571 & 2001 & 37.30 & -0.14 & 3726.00 & 29967.56 & 1.00 & 0.80 & 0.80 \\
11220 & 101376 & 2001 & 1287.00 & -0.04 & 147024.00 & 1369739.70 & 0.88 & 1.06 & 0.93 \\
36817 & 106597 & 2001 & 144.80 & -0.06 & 14456.00 & 144143.23 & 1.00 & 1.00 & 1.00 \\
36824 & 106602 & 2001 & 81.10 & -0.56 & 10667.00 & 106627.55 & 0.76 & 1.31 & 1.00 \\
28249 & 105399 & 2001 & 56.80 & 0.34 & 3930.00 & 58425.04 & 1.45 & 1.03 & 1.49 \\
74763 & 601163 & 2001 & 21.60 & -0.28 & 1591.00 & 19498.50 & 1.36 & 0.90 & 1.23 \\
36832 & 106604 & 2001 & 8.40 & 0.18 & 880.00 & 8707.03 & 0.95 & 1.04 & 0.99 \\
28236 & 105397 & 2001 & 67.10 & 0.00 & 5600.00 & 54426.64 & 1.20 & 0.81 & 0.97 \\
50089 & 240392 & 2001 & 259.70 & -0.08 & 28703.00 & 280748.35 & 0.90 & 1.08 & 0.98 \\
28228 & 105394 & 2001 & 15.20 & -0.30 & 1527.00 & 15795.28 & 1.00 & 1.04 & 1.03 \\
74757 & 601160 & 2001 & 60.60 & -0.17 & 6830.00 & 57340.29 & 0.89 & 0.95 & 0.84 \\
28209 & 105393 & 2001 & 1015.90 & 0.25 & 101684.00 & 874996.41 & 1.00 & 0.86 & 0.86 \\
50064 & 240391 & 2001 & 147.70 & -0.09 & 15898.00 & 150507.80 & 0.93 & 1.02 & 0.95 \\
28200 & 105391 & 2001 & 16.20 & -0.16 & 1618.00 & 16184.54 & 1.00 & 1.00 & 1.00 \\
36852 & 106605 & 2001 & 27.70 & -0.16 & 3356.00 & 29413.50 & 0.83 & 1.06 & 0.88 \\
11250 & 101379 & 2001 & 819.10 & -0.12 & 75404.00 & 797851.57 & 1.09 & 0.97 & 1.06 \\
36873 & 106619 & 2001 & 296.30 & -0.46 & 41192.00 & 285849.47 & 0.72 & 0.96 & 0.69 \\
36793 & 106595 & 2001 & 8.50 & 0.05 & 600.00 & 5557.11 & 1.42 & 0.65 & 0.93 \\
53445 & 350408 & 2001 & 15.50 & -0.05 & 1197.00 & 11727.23 & 1.29 & 0.76 & 0.98 \\
36789 & 106594 & 2001 & 3.30 & -0.07 & 200.00 & 2006.87 & 1.65 & 0.61 & 1.00 \\
36773 & 106590 & 2001 & 40.80 & -0.06 & 3462.00 & 36914.08 & 1.18 & 0.90 & 1.07 \\
50160 & 240398 & 2001 & 594.00 & 0.06 & 59093.00 & 546765.03 & 1.01 & 0.92 & 0.93 \\
36654 & 106573 & 2001 & 16.80 & 0.10 & 1359.00 & 15417.70 & 1.24 & 0.92 & 1.13 \\
28402 & 105421 & 2001 & 49.10 & 0.06 & 4779.00 & 46567.13 & 1.03 & 0.95 & 0.97 \\
36680 & 106574 & 2001 & 19.40 & -0.06 & 1935.00 & 17839.61 & 1.00 & 0.92 & 0.92 \\
50154 & 240397 & 2001 & 39.70 & -0.23 & 3752.00 & 36285.50 & 1.06 & 0.91 & 0.97 \\
50150 & 240396 & 2001 & 15.90 & -0.08 & 1578.00 & 15050.10 & 1.01 & 0.95 & 0.95 \\
36689 & 106577 & 2001 & 1651.50 & -0.19 & 153021.00 & 1554872.77 & 1.08 & 0.94 & 1.02 \\
28376 & 105420 & 2001 & 72.30 & -0.28 & 7376.00 & 74788.65 & 0.98 & 1.03 & 1.01 \\
28431 & 105424 & 2001 & 4066.70 & -0.26 & 394824.00 & 3935179.98 & 1.03 & 0.97 & 1.00 \\
36715 & 106580 & 2001 & 93.70 & 0.06 & 9265.00 & 92481.62 & 1.01 & 0.99 & 1.00 \\
28350 & 105419 & 2001 & 90.40 & -0.04 & 8142.00 & 76280.90 & 1.11 & 0.84 & 0.94 \\
36745 & 106584 & 2001 & 51.50 & 0.19 & 4352.00 & 43757.31 & 1.18 & 0.85 & 1.01 \\
28338 & 105416 & 2001 & 2952.60 & 0.15 & 278338.00 & 2760227.16 & 1.06 & 0.93 & 0.99 \\
53439 & 349609 & 2001 & 25.80 & -0.11 & 2255.00 & 24582.38 & 1.14 & 0.95 & 1.09 \\
28324 & 105412 & 2001 & 168.30 & -0.19 & 18521.00 & 169446.84 & 0.91 & 1.01 & 0.91 \\
28307 & 105401 & 2001 & 16.80 & -0.19 & 1687.00 & 16773.53 & 1.00 & 1.00 & 0.99 \\
8531 & 101089 & 2001 & 115.00 & -0.05 & 13334.00 & 124159.73 & 0.86 & 1.08 & 0.93 \\
74767 & 601164 & 2001 & 5.10 & -0.15 & 526.00 & 4425.48 & 0.97 & 0.87 & 0.84 \\
36731 & 106583 & 2001 & 25.50 & -0.29 & 2823.00 & 28230.80 & 0.90 & 1.11 & 1.00 \\
50399 & 240415 & 2001 & 13.20 & 0.06 & 1323.00 & 12646.19 & 1.00 & 0.96 & 0.96 \\
28726 & 105472 & 2001 & 268.90 & -0.31 & 25961.00 & 259574.56 & 1.04 & 0.97 & 1.00 \\
36440 & 106528 & 2001 & 629.00 & -0.25 & 76655.00 & 643909.90 & 0.82 & 1.02 & 0.84 \\
36050 & 106451 & 2001 & 302.20 & -0.24 & 32634.00 & 325991.32 & 0.93 & 1.08 & 1.00 \\
29129 & 105531 & 2001 & 156.60 & -0.06 & 15731.00 & 155465.08 & 1.00 & 0.99 & 0.99 \\
10898 & 101345 & 2001 & 889.20 & 0.00 & 65133.00 & 635022.59 & 1.37 & 0.71 & 0.97 \\
36076 & 106458 & 2001 & 16.20 & -0.27 & 2138.00 & 19588.89 & 0.76 & 1.21 & 0.92 \\
36080 & 106459 & 2001 & 4.80 & 0.13 & 372.00 & 4744.02 & 1.29 & 0.99 & 1.28 \\
36085 & 106461 & 2001 & 173.70 & -0.20 & 14756.00 & 141803.45 & 1.18 & 0.82 & 0.96 \\
29106 & 105527 & 2001 & 5.90 & -0.02 & 586.00 & 5332.79 & 1.01 & 0.90 & 0.91 \\
50600 & 240430 & 2001 & 43.40 & 0.05 & 4636.00 & 42747.29 & 0.94 & 0.98 & 0.92 \\
36104 & 106464 & 2001 & 344.70 & -0.39 & 36950.00 & 340961.98 & 0.93 & 0.99 & 0.92 \\
29078 & 105525 & 2001 & 507.90 & 0.15 & 51728.00 & 455412.50 & 0.98 & 0.90 & 0.88 \\
74817 & 601183 & 2001 & 15.50 & -0.16 & 1586.00 & 13138.01 & 0.98 & 0.85 & 0.83 \\
29068 & 105523 & 2001 & 121.50 & -0.15 & 15546.00 & 137353.91 & 0.78 & 1.13 & 0.88 \\
36130 & 106467 & 2001 & 161.80 & -0.33 & 18688.00 & 175560.29 & 0.87 & 1.09 & 0.94 \\
36139 & 106470 & 2001 & 509.50 & -0.19 & 47131.00 & 509267.11 & 1.08 & 1.00 & 1.08 \\
29149 & 105533 & 2001 & 373.00 & -0.28 & 32024.00 & 373157.82 & 1.16 & 1.00 & 1.17 \\
36045 & 106450 & 2001 & 24.40 & -0.06 & 2468.00 & 23630.08 & 0.99 & 0.97 & 0.96 \\
35916 & 106428 & 2001 & 345.30 & -0.09 & 35334.00 & 311790.43 & 0.98 & 0.90 & 0.88 \\
35920 & 106429 & 2001 & 37.50 & -0.05 & 3753.00 & 37350.24 & 1.00 & 1.00 & 1.00 \\
50691 & 240440 & 2001 & 22.20 & 0.01 & 1849.00 & 17290.52 & 1.20 & 0.78 & 0.94 \\
35924 & 106434 & 2001 & 856.60 & -0.05 & 85652.00 & 802423.11 & 1.00 & 0.94 & 0.94 \\
29278 & 105581 & 2001 & 302.60 & 0.05 & 30628.00 & 298552.91 & 0.99 & 0.99 & 0.97 \\
50677 & 240439 & 2001 & 61.70 & -0.39 & 6160.00 & 58113.20 & 1.00 & 0.94 & 0.94 \\
29265 & 105574 & 2001 & 140.20 & 0.22 & 10134.00 & 113847.95 & 1.38 & 0.81 & 1.12 \\
29238 & 105561 & 2001 & 35.70 & 0.06 & 3383.00 & 33427.95 & 1.06 & 0.94 & 0.99 \\
10863 & 101340 & 2001 & 20340.30 & -0.10 & 1629109.00 & 15149072.09 & 1.25 & 0.74 & 0.93 \\
291 & 100033 & 2001 & 370.00 & -0.34 & 46373.00 & 389829.24 & 0.80 & 1.05 & 0.84 \\
29225 & 105545 & 2001 & 1038.00 & -0.16 & 94914.00 & 983569.09 & 1.09 & 0.95 & 1.04 \\
35986 & 106444 & 2001 & 118.10 & -0.15 & 11847.00 & 116751.62 & 1.00 & 0.99 & 0.99 \\
29199 & 105536 & 2001 & 488.60 & -0.41 & 48913.00 & 459389.37 & 1.00 & 0.94 & 0.94 \\
36009 & 106447 & 2001 & 72.30 & -0.72 & 7381.00 & 71419.40 & 0.98 & 0.99 & 0.97 \\
29183 & 105535 & 2001 & 286.50 & -0.29 & 28524.00 & 264372.37 & 1.00 & 0.92 & 0.93 \\
35958 & 106442 & 2001 & 3385.70 & -0.13 & 380477.00 & 3287837.98 & 0.89 & 0.97 & 0.86 \\
29031 & 105520 & 2001 & 20.90 & -0.04 & 1909.00 & 19762.44 & 1.09 & 0.95 & 1.04 \\
10927 & 101354 & 2001 & 1436.60 & -0.19 & 143680.00 & 1370894.25 & 1.00 & 0.95 & 0.95 \\
36150 & 106471 & 2001 & 177.30 & -0.05 & 19776.00 & 177004.69 & 0.90 & 1.00 & 0.90 \\
28873 & 105498 & 2001 & 104.60 & -0.20 & 10457.00 & 101315.60 & 1.00 & 0.97 & 0.97 \\
10985 & 101358 & 2001 & 428.50 & 0.05 & 42887.00 & 426952.21 & 1.00 & 1.00 & 1.00 \\
28855 & 105487 & 2001 & 283.30 & -0.35 & 14833.00 & 133434.73 & 1.91 & 0.47 & 0.90 \\
36325 & 106483 & 2001 & 6.30 & 0.10 & 538.00 & 5377.75 & 1.17 & 0.85 & 1.00 \\
340 & 100040 & 2001 & 3487.10 & -0.39 & 348632.00 & 3486094.35 & 1.00 & 1.00 & 1.00 \\
36335 & 106485 & 2001 & 104.40 & -0.23 & 10380.00 & 103802.71 & 1.01 & 0.99 & 1.00 \\
36351 & 106487 & 2001 & 21.70 & 0.12 & 2107.00 & 21068.20 & 1.03 & 0.97 & 1.00 \\
28883 & 105502 & 2001 & 1754.00 & -0.13 & 173762.00 & 1463035.90 & 1.01 & 0.83 & 0.84 \\
28808 & 105478 & 2001 & 49.80 & -0.15 & 4961.00 & 48555.04 & 1.00 & 0.98 & 0.98 \\
36365 & 106519 & 2001 & 27.60 & -0.13 & 2754.00 & 27541.36 & 1.00 & 1.00 & 1.00 \\
11025 & 101360 & 2001 & 1911.00 & 0.00 & 191161.00 & 1647849.31 & 1.00 & 0.86 & 0.86 \\
28777 & 105476 & 2001 & 108.70 & 0.09 & 10286.00 & 100590.54 & 1.06 & 0.93 & 0.98 \\
36394 & 106523 & 2001 & 35.60 & 0.23 & 1815.00 & 16368.49 & 1.96 & 0.46 & 0.90 \\
36414 & 106524 & 2001 & 13.30 & 0.24 & 1424.00 & 10714.49 & 0.93 & 0.81 & 0.75 \\
28748 & 105475 & 2001 & 541.50 & -0.06 & 48458.00 & 547680.20 & 1.12 & 1.01 & 1.13 \\
74807 & 601178 & 2001 & 14.20 & 0.01 & 1357.00 & 12116.92 & 1.05 & 0.85 & 0.89 \\
36435 & 106527 & 2001 & 18.50 & -0.01 & 1435.00 & 16873.59 & 1.29 & 0.91 & 1.18 \\
28801 & 105477 & 2001 & 5.80 & -0.15 & 588.00 & 5848.08 & 0.99 & 1.01 & 0.99 \\
34866 & 106283 & 2001 & 208.10 & 0.00 & 20841.00 & 202328.36 & 1.00 & 0.97 & 0.97 \\
36299 & 106482 & 2001 & 87.70 & 0.10 & 8399.00 & 83992.28 & 1.04 & 0.96 & 1.00 \\
36292 & 106481 & 2001 & 100.10 & 0.17 & 9075.00 & 90749.23 & 1.10 & 0.91 & 1.00 \\
36158 & 106474 & 2001 & 139.40 & -0.13 & 13932.00 & 135030.69 & 1.00 & 0.97 & 0.97 \\
29004 & 105512 & 2001 & 8.60 & -0.00 & 865.00 & 7766.75 & 0.99 & 0.90 & 0.90 \\
28997 & 105511 & 2001 & 25.60 & 0.09 & 2546.00 & 24906.61 & 1.01 & 0.97 & 0.98 \\
28986 & 105510 & 2001 & 26.20 & -0.18 & 2602.00 & 24841.48 & 1.01 & 0.95 & 0.95 \\
36168 & 106476 & 2001 & 5.30 & -0.03 & 519.00 & 5070.67 & 1.02 & 0.96 & 0.98 \\
36194 & 106477 & 2001 & 1102.70 & -0.11 & 102911.00 & 1015424.80 & 1.07 & 0.92 & 0.99 \\
10974 & 101357 & 2001 & 291.50 & -0.08 & 29919.00 & 300068.29 & 0.97 & 1.03 & 1.00 \\
28960 & 105508 & 2001 & 45.50 & 0.12 & 4520.00 & 42512.63 & 1.01 & 0.93 & 0.94 \\
36240 & 106479 & 2001 & 6.00 & 0.05 & 784.00 & 8040.93 & 0.77 & 1.34 & 1.03 \\
28941 & 105507 & 2001 & 823.30 & -0.12 & 73499.00 & 785439.49 & 1.12 & 0.95 & 1.07 \\
323 & 100036 & 2001 & 74.90 & -0.20 & 7499.00 & 73813.71 & 1.00 & 0.99 & 0.98 \\
10959 & 101356 & 2001 & 487.70 & -0.00 & 48915.00 & 475379.88 & 1.00 & 0.97 & 0.97 \\
36266 & 106480 & 2001 & 451.40 & -0.09 & 46355.00 & 435825.37 & 0.97 & 0.97 & 0.94 \\
8631 & 101092 & 2001 & 362.40 & -0.03 & 44710.00 & 437789.94 & 0.81 & 1.21 & 0.98 \\
36217 & 106478 & 2001 & 5.60 & 0.03 & 732.00 & 7386.94 & 0.77 & 1.32 & 1.01 \\
28171 & 105390 & 2001 & 525.60 & 0.05 & 48560.00 & 511218.33 & 1.08 & 0.97 & 1.05 \\
34839 & 106282 & 2001 & 508.20 & -0.03 & 50966.00 & 470897.05 & 1.00 & 0.93 & 0.92 \\
30617 & 105779 & 2001 & 2206.10 & -0.19 & 243283.00 & 2233600.19 & 0.91 & 1.01 & 0.92 \\
33552 & 106148 & 2001 & 563.60 & -0.01 & 56293.00 & 539151.85 & 1.00 & 0.96 & 0.96 \\
32152 & 105990 & 2001 & 229.30 & -0.02 & 32509.00 & 298957.66 & 0.71 & 1.30 & 0.92 \\
32146 & 105987 & 2001 & 156.40 & -0.12 & 17882.00 & 164863.57 & 0.87 & 1.05 & 0.92 \\
32132 & 105984 & 2001 & 135.40 & -0.17 & 13524.00 & 134579.02 & 1.00 & 0.99 & 1.00 \\
9663 & 101161 & 2001 & 1158.00 & -0.08 & 111873.00 & 1167775.44 & 1.04 & 1.01 & 1.04 \\
33563 & 106149 & 2001 & 388.40 & -0.35 & 38901.00 & 350527.89 & 1.00 & 0.90 & 0.90 \\
33575 & 106150 & 2001 & 201.70 & -0.25 & 16065.00 & 142702.20 & 1.26 & 0.71 & 0.89 \\
32105 & 105983 & 2001 & 217.50 & -0.18 & 21686.00 & 210166.75 & 1.00 & 0.97 & 0.97 \\
9184 & 101116 & 2001 & 1427.10 & -0.15 & 177826.00 & 1380810.85 & 0.80 & 0.97 & 0.78 \\
9682 & 101165 & 2001 & 2110.90 & -0.13 & 194121.00 & 2057623.97 & 1.09 & 0.97 & 1.06 \\
33586 & 106151 & 2001 & 1154.70 & 0.05 & 121320.00 & 1117704.83 & 0.95 & 0.97 & 0.92 \\
33600 & 106152 & 2001 & 524.10 & -0.16 & 52446.00 & 511660.58 & 1.00 & 0.98 & 0.98 \\
32073 & 105980 & 2001 & 351.40 & 0.09 & 35171.00 & 333793.04 & 1.00 & 0.95 & 0.95 \\
32 & 100003 & 2001 & 848.80 & -0.17 & 85607.00 & 797034.72 & 0.99 & 0.94 & 0.93 \\
52175 & 302578 & 2001 & 2.50 & -0.26 & 285.00 & 2541.28 & 0.88 & 1.02 & 0.89 \\
32061 & 105978 & 2001 & 147.30 & -0.35 & 14321.00 & 143198.85 & 1.03 & 0.97 & 1.00 \\
33538 & 106147 & 2001 & 102.90 & -0.10 & 10298.00 & 98635.18 & 1.00 & 0.96 & 0.96 \\
33533 & 106144 & 2001 & 217.50 & -0.17 & 22079.00 & 217183.72 & 0.99 & 1.00 & 0.98 \\
52216 & 302677 & 2001 & 6.30 & -0.09 & 657.00 & 5899.29 & 0.96 & 0.94 & 0.90 \\
96780 & 611013 & 2001 & 65.90 & -0.13 & 6581.00 & 65693.98 & 1.00 & 1.00 & 1.00 \\
32295 & 106009 & 2001 & 2164.10 & -0.14 & 212640.00 & 2126370.08 & 1.02 & 0.98 & 1.00 \\
52268 & 302731 & 2001 & 789.00 & 0.18 & 80964.00 & 722104.02 & 0.97 & 0.92 & 0.89 \\
33401 & 106129 & 2001 & 271.90 & -0.06 & 27213.00 & 267148.45 & 1.00 & 0.98 & 0.98 \\
32267 & 106008 & 2001 & 295.00 & -0.16 & 28449.00 & 272021.40 & 1.04 & 0.92 & 0.96 \\
52242 & 302698 & 2001 & 218.60 & -0.32 & 21883.00 & 217526.39 & 1.00 & 1.00 & 0.99 \\
33413 & 106133 & 2001 & 303.00 & -0.22 & 32215.00 & 271897.40 & 0.94 & 0.90 & 0.84 \\
33420 & 106135 & 2001 & 47.30 & 0.04 & 4603.00 & 44511.88 & 1.03 & 0.94 & 0.97 \\
32045 & 105977 & 2001 & 1218.00 & -0.09 & 107813.00 & 1144069.31 & 1.13 & 0.94 & 1.06 \\
9613 & 101158 & 2001 & 438.90 & -0.19 & 40750.00 & 434683.62 & 1.08 & 0.99 & 1.07 \\
33447 & 106136 & 2001 & 114.80 & 0.09 & 8439.00 & 95920.10 & 1.36 & 0.84 & 1.14 \\
33474 & 106137 & 2001 & 4.30 & -0.37 & 536.00 & 4257.35 & 0.80 & 0.99 & 0.79 \\
52480 & 302964 & 2001 & 33.10 & -0.03 & 3219.00 & 31403.07 & 1.03 & 0.95 & 0.98 \\
33480 & 106140 & 2001 & 1752.60 & -0.05 & 133358.00 & 1504186.50 & 1.31 & 0.86 & 1.13 \\
32211 & 106000 & 2001 & 618.00 & -0.14 & 58222.00 & 582206.11 & 1.06 & 0.94 & 1.00 \\
9632 & 101160 & 2001 & 348.90 & -0.01 & 31983.00 & 343509.13 & 1.09 & 0.98 & 1.07 \\
33506 & 106143 & 2001 & 173.00 & -0.13 & 19947.00 & 175877.68 & 0.87 & 1.02 & 0.88 \\
32183 & 105999 & 2001 & 14.00 & 0.13 & 1286.00 & 12637.80 & 1.09 & 0.90 & 0.98 \\
32239 & 106007 & 2001 & 3858.90 & -0.07 & 370001.00 & 3526752.51 & 1.04 & 0.91 & 0.95 \\
32308 & 106010 & 2001 & 962.40 & -0.05 & 90215.00 & 898965.86 & 1.07 & 0.93 & 1.00 \\
32030 & 105974 & 2001 & 44.10 & -0.27 & 4409.00 & 43914.87 & 1.00 & 1.00 & 1.00 \\
33745 & 106165 & 2001 & 140.20 & -0.31 & 14136.00 & 139829.06 & 0.99 & 1.00 & 0.99 \\
31865 & 105949 & 2001 & 289.80 & 0.18 & 28951.00 & 266196.20 & 1.00 & 0.92 & 0.92 \\
31854 & 105948 & 2001 & 29.20 & -0.12 & 2921.00 & 27730.14 & 1.00 & 0.95 & 0.95 \\
31843 & 105947 & 2001 & 40.30 & -0.15 & 4030.00 & 39572.72 & 1.00 & 0.98 & 0.98 \\
31832 & 105946 & 2001 & 122.00 & 0.00 & 12329.00 & 120556.40 & 0.99 & 0.99 & 0.98 \\
9800 & 101193 & 2001 & 510.80 & -0.16 & 47220.00 & 418513.68 & 1.08 & 0.82 & 0.89 \\
31826 & 105945 & 2001 & 12.00 & -0.18 & 1160.00 & 11602.47 & 1.03 & 0.97 & 1.00 \\
52139 & 302067 & 2001 & 14.70 & -0.14 & 1252.00 & 14575.26 & 1.17 & 0.99 & 1.16 \\
31815 & 105943 & 2001 & 60.60 & -0.01 & 5610.00 & 57107.88 & 1.08 & 0.94 & 1.02 \\
31807 & 105941 & 2001 & 14.80 & -0.30 & 1401.00 & 13999.81 & 1.06 & 0.95 & 1.00 \\
52555 & 303124 & 2001 & 55.60 & 0.30 & 3644.00 & 32117.63 & 1.53 & 0.58 & 0.88 \\
31792 & 105936 & 2001 & 102.00 & -0.19 & 8937.00 & 99745.90 & 1.14 & 0.98 & 1.12 \\
96704 & 611008 & 2001 & 40.70 & -0.10 & 4471.00 & 44708.69 & 0.91 & 1.10 & 1.00 \\
33753 & 106167 & 2001 & 179.30 & 0.13 & 18067.00 & 177484.59 & 0.99 & 0.99 & 0.98 \\
33729 & 106164 & 2001 & 67.00 & -0.23 & 6853.00 & 67455.47 & 0.98 & 1.01 & 0.98 \\
31890 & 105951 & 2001 & 335.40 & -0.25 & 33258.00 & 322467.53 & 1.01 & 0.96 & 0.97 \\
31897 & 105954 & 2001 & 4.10 & 0.16 & 410.00 & 3747.45 & 1.00 & 0.91 & 0.91 \\
33613 & 106156 & 2001 & 11.20 & -0.28 & 960.00 & 10811.30 & 1.17 & 0.97 & 1.13 \\
52539 & 303121 & 2001 & 229.20 & -0.13 & 21233.00 & 223493.64 & 1.08 & 0.98 & 1.05 \\
33625 & 106157 & 2001 & 749.00 & -0.28 & 58941.00 & 714417.44 & 1.27 & 0.95 & 1.21 \\
32003 & 105973 & 2001 & 11.60 & 0.19 & 1014.00 & 10688.15 & 1.14 & 0.92 & 1.05 \\
9727 & 101179 & 2001 & 926.10 & -0.13 & 96542.00 & 859820.82 & 0.96 & 0.93 & 0.89 \\
33645 & 106158 & 2001 & 221.40 & -0.10 & 22098.00 & 209865.75 & 1.00 & 0.95 & 0.95 \\
9150 & 101115 & 2001 & 16417.20 & -0.12 & 1604339.00 & 15916161.09 & 1.02 & 0.97 & 0.99 \\
33669 & 106160 & 2001 & 20.70 & 0.07 & 2042.00 & 20510.47 & 1.01 & 0.99 & 1.00 \\
32036 & 105976 & 2001 & 11.70 & -0.23 & 1172.00 & 11718.49 & 1.00 & 1.00 & 1.00 \\
31973 & 105965 & 2001 & 77.80 & 0.10 & 7065.00 & 66893.63 & 1.10 & 0.86 & 0.95 \\
52545 & 303123 & 2001 & 103.70 & -0.03 & 10369.00 & 97743.44 & 1.00 & 0.94 & 0.94 \\
9752 & 101186 & 2001 & 573.00 & -0.20 & 53363.00 & 486561.28 & 1.07 & 0.85 & 0.91 \\
33702 & 106163 & 2001 & 168.80 & 0.12 & 15551.00 & 166457.23 & 1.09 & 0.99 & 1.07 \\
31938 & 105963 & 2001 & 646.50 & -0.18 & 60768.00 & 614667.46 & 1.06 & 0.95 & 1.01 \\
31927 & 105961 & 2001 & 176.60 & -0.24 & 13486.00 & 156210.65 & 1.31 & 0.88 & 1.16 \\
52162 & 302545 & 2001 & 98.10 & -0.06 & 9790.00 & 90952.35 & 1.00 & 0.93 & 0.93 \\
31917 & 105960 & 2001 & 247.30 & -0.15 & 22083.00 & 244808.50 & 1.12 & 0.99 & 1.11 \\
9770 & 101192 & 2001 & 161.30 & -0.13 & 12902.00 & 129904.05 & 1.25 & 0.81 & 1.01 \\
31965 & 105964 & 2001 & 155.90 & -0.20 & 14185.00 & 141417.07 & 1.10 & 0.91 & 1.00 \\
33765 & 106169 & 2001 & 55.90 & 0.15 & 4433.00 & 47616.48 & 1.26 & 0.85 & 1.07 \\
33382 & 106127 & 2001 & 108.50 & -0.13 & 11107.00 & 103599.93 & 0.98 & 0.95 & 0.93 \\
9584 & 101151 & 2001 & 144.60 & -0.34 & 14424.00 & 131316.30 & 1.00 & 0.91 & 0.91 \\
32690 & 106049 & 2001 & 456.90 & -0.12 & 44033.00 & 460164.19 & 1.04 & 1.01 & 1.05 \\
52350 & 302811 & 2001 & 39.80 & -0.11 & 3982.00 & 39818.66 & 1.00 & 1.00 & 1.00 \\
32969 & 106084 & 2001 & 1639.10 & -0.27 & 189619.00 & 1621732.17 & 0.86 & 0.99 & 0.86 \\
32993 & 106085 & 2001 & 420.90 & -0.16 & 52314.00 & 413836.73 & 0.80 & 0.98 & 0.79 \\
32663 & 106047 & 2001 & 6.70 & -0.37 & 930.00 & 6917.89 & 0.72 & 1.03 & 0.74 \\
9449 & 101135 & 2001 & 1703.70 & -0.15 & 169427.00 & 1618885.04 & 1.01 & 0.95 & 0.96 \\
32644 & 106045 & 2001 & 29.90 & -0.13 & 2780.00 & 30295.48 & 1.08 & 1.01 & 1.09 \\
52395 & 302879 & 2001 & 32.10 & 0.03 & 3221.00 & 25984.10 & 1.00 & 0.81 & 0.81 \\
33020 & 106086 & 2001 & 42.90 & -0.18 & 3691.00 & 41213.97 & 1.16 & 0.96 & 1.12 \\
9462 & 101137 & 2001 & 27.40 & -0.40 & 2755.00 & 27151.87 & 0.99 & 0.99 & 0.99 \\
33047 & 106088 & 2001 & 95.20 & 0.06 & 11029.00 & 92795.49 & 0.86 & 0.97 & 0.84 \\
52341 & 302780 & 2001 & 115.70 & 0.09 & 12108.00 & 96861.80 & 0.96 & 0.84 & 0.80 \\
33083 & 106090 & 2001 & 87.70 & 0.24 & 8818.00 & 83428.79 & 0.99 & 0.95 & 0.95 \\
32702 & 106050 & 2001 & 421.70 & -0.10 & 42448.00 & 413393.09 & 0.99 & 0.98 & 0.97 \\
32942 & 106083 & 2001 & 1238.30 & -0.15 & 124393.00 & 1178835.71 & 1.00 & 0.95 & 0.95 \\
32715 & 106051 & 2001 & 62.80 & -0.13 & 6427.00 & 58610.58 & 0.98 & 0.93 & 0.91 \\
32851 & 106068 & 2001 & 26.60 & -0.09 & 2884.00 & 23761.63 & 0.92 & 0.89 & 0.82 \\
32801 & 106064 & 2001 & 169.80 & -0.23 & 18414.00 & 179802.52 & 0.92 & 1.06 & 0.98 \\
9351 & 101132 & 2001 & 351.40 & -0.18 & 34990.00 & 333265.78 & 1.00 & 0.95 & 0.95 \\
32873 & 106075 & 2001 & 398.40 & -0.16 & 43380.00 & 395520.74 & 0.92 & 0.99 & 0.91 \\
32755 & 106061 & 2001 & 260.40 & -0.20 & 25316.00 & 253161.99 & 1.03 & 0.97 & 1.00 \\
52362 & 302813 & 2001 & 20.50 & -0.15 & 2063.00 & 20316.15 & 0.99 & 0.99 & 0.98 \\
52386 & 302826 & 2001 & 205.60 & -0.22 & 20578.00 & 200219.00 & 1.00 & 0.97 & 0.97 \\
32915 & 106082 & 2001 & 697.50 & -0.08 & 75869.00 & 652323.26 & 0.92 & 0.94 & 0.86 \\
32740 & 106057 & 2001 & 463.10 & -0.30 & 35895.00 & 462463.72 & 1.29 & 1.00 & 1.29 \\
9324 & 101131 & 2001 & 6267.60 & -0.33 & 625949.00 & 5926555.37 & 1.00 & 0.95 & 0.95 \\
32721 & 106052 & 2001 & 158.00 & -0.12 & 15804.00 & 150121.72 & 1.00 & 0.95 & 0.95 \\
9388 & 101133 & 2001 & 1266.00 & -0.10 & 126502.00 & 1176899.79 & 1.00 & 0.93 & 0.93 \\
33356 & 106124 & 2001 & 589.60 & 0.17 & 60513.00 & 565230.48 & 0.97 & 0.96 & 0.93 \\
11 & 100001 & 2001 & 3766.20 & -0.14 & 376593.00 & 3668957.78 & 1.00 & 0.97 & 0.97 \\
33097 & 106091 & 2001 & 130.70 & 0.16 & 13071.00 & 124102.31 & 1.00 & 0.95 & 0.95 \\
9219 & 101119 & 2001 & 63.60 & 0.19 & 6406.00 & 56659.76 & 0.99 & 0.89 & 0.88 \\
33248 & 106108 & 2001 & 53.00 & 0.03 & 4704.00 & 43376.95 & 1.13 & 0.82 & 0.92 \\
9546 & 101149 & 2001 & 3465.00 & -0.45 & 346291.00 & 3282361.20 & 1.00 & 0.95 & 0.95 \\
33269 & 106109 & 2001 & 39.80 & 0.16 & 2736.00 & 22957.17 & 1.45 & 0.58 & 0.84 \\
32405 & 106023 & 2001 & 179.60 & -0.27 & 15890.00 & 185465.63 & 1.13 & 1.03 & 1.17 \\
52322 & 302763 & 2001 & 53.70 & -0.02 & 5160.00 & 49659.83 & 1.04 & 0.92 & 0.96 \\
33284 & 106113 & 2001 & 490.10 & -0.12 & 47267.00 & 471831.85 & 1.04 & 0.96 & 1.00 \\
52296 & 302760 & 2001 & 558.00 & 0.34 & 48615.00 & 459954.02 & 1.15 & 0.82 & 0.95 \\
33311 & 106114 & 2001 & 97.30 & 0.05 & 9762.00 & 94064.99 & 1.00 & 0.97 & 0.96 \\
32361 & 106014 & 2001 & 98.10 & -0.16 & 8029.00 & 94559.18 & 1.22 & 0.96 & 1.18 \\
52289 & 302732 & 2001 & 371.80 & -0.41 & 37986.00 & 352940.45 & 0.98 & 0.95 & 0.93 \\
33338 & 106116 & 2001 & 8.10 & 0.24 & 512.00 & 8047.72 & 1.58 & 0.99 & 1.57 \\
32333 & 106011 & 2001 & 878.80 & -0.39 & 87493.00 & 874833.98 & 1.00 & 1.00 & 1.00 \\
33349 & 106123 & 2001 & 491.30 & -0.20 & 51291.00 & 470683.10 & 0.96 & 0.96 & 0.92 \\
32446 & 106028 & 2001 & 353.30 & 0.03 & 35342.00 & 316490.13 & 1.00 & 0.90 & 0.90 \\
33238 & 106107 & 2001 & 39.70 & -0.19 & 3744.00 & 35405.44 & 1.06 & 0.89 & 0.95 \\
52447 & 302942 & 2001 & 466.30 & -0.01 & 45954.00 & 439117.34 & 1.01 & 0.94 & 0.96 \\
33233 & 106106 & 2001 & 3.30 & -0.09 & 315.00 & 2990.69 & 1.05 & 0.91 & 0.95 \\
9281 & 101127 & 2001 & 121.90 & -0.21 & 15430.00 & 121077.97 & 0.79 & 0.99 & 0.78 \\
33124 & 106092 & 2001 & 843.20 & -0.03 & 74557.00 & 801291.85 & 1.13 & 0.95 & 1.07 \\
32562 & 106041 & 2001 & 345.00 & 0.12 & 35833.00 & 315018.80 & 0.96 & 0.91 & 0.88 \\
9485 & 101139 & 2001 & 17.20 & -0.32 & 1953.00 & 18303.07 & 0.88 & 1.06 & 0.94 \\
32546 & 106039 & 2001 & 1211.40 & 0.01 & 112405.00 & 1124048.32 & 1.08 & 0.93 & 1.00 \\
9497 & 101140 & 2001 & 924.50 & -0.49 & 93013.00 & 929871.35 & 0.99 & 1.01 & 1.00 \\
52409 & 302881 & 2001 & 177.20 & -0.14 & 17689.00 & 175942.11 & 1.00 & 0.99 & 0.99 \\
32518 & 106038 & 2001 & 502.50 & 0.14 & 37767.00 & 341568.51 & 1.33 & 0.68 & 0.90 \\
32509 & 106037 & 2001 & 38.10 & 0.16 & 3814.00 & 38079.04 & 1.00 & 1.00 & 1.00 \\
9510 & 101141 & 2001 & 3490.70 & -0.12 & 348481.00 & 2998556.55 & 1.00 & 0.86 & 0.86 \\
33217 & 106103 & 2001 & 12.50 & -0.22 & 1882.00 & 16446.20 & 0.66 & 1.32 & 0.87 \\
32474 & 106033 & 2001 & 773.10 & 0.08 & 63535.00 & 751505.49 & 1.22 & 0.97 & 1.18 \\
52441 & 302941 & 2001 & 16.10 & -0.17 & 1438.00 & 14409.69 & 1.12 & 0.90 & 1.00 \\
52415 & 302907 & 2001 & 659.00 & 0.34 & 60493.00 & 604929.06 & 1.09 & 0.92 & 1.00 \\
31764 & 105935 & 2001 & 353.40 & -0.09 & 31516.00 & 342644.95 & 1.12 & 0.97 & 1.09 \\
9831 & 101194 & 2001 & 176.60 & -0.20 & 17432.00 & 154611.32 & 1.01 & 0.88 & 0.89 \\
31753 & 105933 & 2001 & 2282.00 & -0.25 & 180519.00 & 2096026.50 & 1.26 & 0.92 & 1.16 \\
31041 & 105852 & 2001 & 54.60 & -0.05 & 4599.00 & 51831.59 & 1.19 & 0.95 & 1.13 \\
96676 & 611003 & 2001 & 203.90 & 0.01 & 13500.00 & 140818.64 & 1.51 & 0.69 & 1.04 \\
10130 & 101262 & 2001 & 17.20 & -0.08 & 1718.00 & 16838.22 & 1.00 & 0.98 & 0.98 \\
31023 & 105848 & 2001 & 73.60 & 0.12 & 7397.00 & 67975.44 & 0.99 & 0.92 & 0.92 \\
31015 & 105847 & 2001 & 35.80 & -0.06 & 3593.00 & 31853.60 & 1.00 & 0.89 & 0.89 \\
34525 & 106249 & 2001 & 87.10 & -0.13 & 7572.00 & 80618.50 & 1.15 & 0.93 & 1.06 \\
30999 & 105846 & 2001 & 2783.10 & -0.14 & 279231.00 & 2782009.69 & 1.00 & 1.00 & 1.00 \\
10147 & 101263 & 2001 & 291.60 & -0.18 & 31454.00 & 287537.94 & 0.93 & 0.99 & 0.91 \\
30988 & 105845 & 2001 & 3.20 & -0.16 & 418.00 & 4275.40 & 0.77 & 1.34 & 1.02 \\
30981 & 105843 & 2001 & 5.10 & -0.10 & 576.00 & 5703.80 & 0.89 & 1.12 & 0.99 \\
34534 & 106250 & 2001 & 63.90 & -0.16 & 5319.00 & 52191.62 & 1.20 & 0.82 & 0.98 \\
30961 & 105842 & 2001 & 980.80 & 0.10 & 97940.00 & 881764.90 & 1.00 & 0.90 & 0.90 \\
10168 & 101264 & 2001 & 1084.20 & -0.25 & 123576.00 & 982298.71 & 0.88 & 0.91 & 0.79 \\
30953 & 105841 & 2001 & 5.60 & -0.35 & 625.00 & 6353.57 & 0.90 & 1.13 & 1.02 \\
30947 & 105840 & 2001 & 34.90 & -0.08 & 3481.00 & 34146.17 & 1.00 & 0.98 & 0.98 \\
30941 & 105839 & 2001 & 48.30 & -0.19 & 4828.00 & 48021.91 & 1.00 & 0.99 & 0.99 \\
34498 & 106248 & 2001 & 332.10 & 0.15 & 24299.00 & 227134.15 & 1.37 & 0.68 & 0.93 \\
31069 & 105854 & 2001 & 844.30 & -0.12 & 67504.00 & 765809.84 & 1.25 & 0.91 & 1.13 \\
34472 & 106244 & 2001 & 3.10 & -0.22 & 310.00 & 2882.62 & 1.00 & 0.93 & 0.93 \\
52641 & 305586 & 2001 & 98.00 & -0.16 & 9012.00 & 83641.33 & 1.09 & 0.85 & 0.93 \\
34331 & 106223 & 2001 & 78.90 & -0.01 & 7884.00 & 78044.52 & 1.00 & 0.99 & 0.99 \\
31190 & 105866 & 2001 & 7643.90 & -0.12 & 763830.00 & 7615054.40 & 1.00 & 1.00 & 1.00 \\
52651 & 305590 & 2001 & 112.20 & 0.05 & 10472.00 & 104200.07 & 1.07 & 0.93 & 1.00 \\
34351 & 106224 & 2001 & 41.60 & -0.02 & 4139.00 & 36577.83 & 1.01 & 0.88 & 0.88 \\
52674 & 305766 & 2001 & 82.00 & -0.10 & 8203.00 & 82007.54 & 1.00 & 1.00 & 1.00 \\
34369 & 106230 & 2001 & 296.10 & -0.10 & 29593.00 & 287624.24 & 1.00 & 0.97 & 0.97 \\
31162 & 105865 & 2001 & 135.20 & -0.02 & 13684.00 & 130347.35 & 0.99 & 0.96 & 0.95 \\
8969 & 101107 & 2001 & 1004.90 & -0.12 & 111308.00 & 1013673.78 & 0.90 & 1.01 & 0.91 \\
31155 & 105864 & 2001 & 22.50 & -0.20 & 2224.00 & 22239.01 & 1.01 & 0.99 & 1.00 \\
31139 & 105861 & 2001 & 243.30 & 0.15 & 21999.00 & 217425.95 & 1.11 & 0.89 & 0.99 \\
51909 & 300102 & 2001 & 44.00 & -0.22 & 5328.00 & 47756.66 & 0.83 & 1.09 & 0.90 \\
34396 & 106231 & 2001 & 190.80 & 0.16 & 19332.00 & 187215.30 & 0.99 & 0.98 & 0.97 \\
9000 & 101108 & 2001 & 958.00 & 0.06 & 78320.00 & 825023.13 & 1.22 & 0.86 & 1.05 \\
31111 & 105860 & 2001 & 3704.40 & -0.33 & 442352.00 & 3481773.34 & 0.84 & 0.94 & 0.79 \\
34429 & 106239 & 2001 & 76.80 & -0.50 & 6007.00 & 74096.69 & 1.28 & 0.96 & 1.23 \\
34445 & 106240 & 2001 & 175.60 & -0.08 & 17572.00 & 165620.88 & 1.00 & 0.94 & 0.94 \\
31084 & 105857 & 2001 & 191.50 & 0.15 & 19151.00 & 172983.66 & 1.00 & 0.90 & 0.90 \\
10078 & 101258 & 2001 & 4193.80 & -0.07 & 328347.00 & 2864520.40 & 1.28 & 0.68 & 0.87 \\
31218 & 105867 & 2001 & 4.80 & -0.25 & 489.00 & 4697.32 & 0.98 & 0.98 & 0.96 \\
120 & 100009 & 2001 & 253.20 & -0.16 & 25532.00 & 247757.66 & 0.99 & 0.98 & 0.97 \\
30753 & 105793 & 2001 & 3437.40 & -0.32 & 309111.00 & 3285276.20 & 1.11 & 0.96 & 1.06 \\
34677 & 106270 & 2001 & 16.50 & -0.15 & 1233.00 & 15832.43 & 1.34 & 0.96 & 1.28 \\
30732 & 105791 & 2001 & 3.80 & -0.07 & 362.00 & 3663.09 & 1.05 & 0.96 & 1.01 \\
144 & 100010 & 2001 & 559.60 & -0.03 & 55903.00 & 542818.05 & 1.00 & 0.97 & 0.97 \\
34710 & 106272 & 2001 & 1881.80 & -0.12 & 166710.00 & 1820047.37 & 1.13 & 0.97 & 1.09 \\
30711 & 105788 & 2001 & 13.00 & -0.05 & 1329.00 & 12206.05 & 0.98 & 0.94 & 0.92 \\
34739 & 106275 & 2001 & 35.60 & 0.10 & 2919.00 & 32376.11 & 1.22 & 0.91 & 1.11 \\
52807 & 330079 & 2001 & 31.30 & 0.01 & 3130.00 & 29712.95 & 1.00 & 0.95 & 0.95 \\
30684 & 105783 & 2001 & 1376.00 & 0.12 & 120931.00 & 1167307.24 & 1.14 & 0.85 & 0.97 \\
10294 & 101278 & 2001 & 89.50 & 0.22 & 7355.00 & 73554.23 & 1.22 & 0.82 & 1.00 \\
34793 & 106277 & 2001 & 41.30 & -0.03 & 3211.00 & 33612.46 & 1.29 & 0.81 & 1.05 \\
30653 & 105781 & 2001 & 338.30 & 0.26 & 24143.00 & 279756.08 & 1.40 & 0.83 & 1.16 \\
30644 & 105780 & 2001 & 315.90 & -0.22 & 32678.00 & 294872.59 & 0.97 & 0.93 & 0.90 \\
51615 & 240535 & 2001 & 25.50 & 0.02 & 2562.00 & 23489.85 & 1.00 & 0.92 & 0.92 \\
34816 & 106278 & 2001 & 111.10 & 0.04 & 9632.00 & 113714.17 & 1.15 & 1.02 & 1.18 \\
34828 & 106281 & 2001 & 19.60 & -0.12 & 1751.00 & 15826.41 & 1.12 & 0.81 & 0.90 \\
30763 & 105794 & 2001 & 32.90 & 0.20 & 3327.00 & 28456.74 & 0.99 & 0.86 & 0.86 \\
34661 & 106268 & 2001 & 179.30 & 0.02 & 15185.00 & 125392.37 & 1.18 & 0.70 & 0.83 \\
10259 & 101276 & 2001 & 551.60 & 0.38 & 38469.00 & 488603.60 & 1.43 & 0.89 & 1.27 \\
34655 & 106267 & 2001 & 9.30 & -0.05 & 926.00 & 8876.70 & 1.00 & 0.95 & 0.96 \\
10198 & 101268 & 2001 & 910.50 & -0.25 & 109235.00 & 973039.96 & 0.83 & 1.07 & 0.89 \\
30901 & 105807 & 2001 & 21.40 & -0.08 & 2374.00 & 24844.81 & 0.90 & 1.16 & 1.05 \\
34543 & 106251 & 2001 & 64.00 & -0.09 & 5009.00 & 54806.12 & 1.28 & 0.86 & 1.09 \\
30884 & 105806 & 2001 & 99.20 & -0.22 & 7665.00 & 84997.26 & 1.29 & 0.86 & 1.11 \\
10212 & 101274 & 2001 & 319.80 & -0.07 & 30174.00 & 289954.81 & 1.06 & 0.91 & 0.96 \\
34557 & 106255 & 2001 & 661.20 & -0.22 & 69359.00 & 714669.00 & 0.95 & 1.08 & 1.03 \\
34583 & 106256 & 2001 & 84.00 & -0.12 & 8315.00 & 82400.65 & 1.01 & 0.98 & 0.99 \\
34590 & 106257 & 2001 & 248.70 & -0.14 & 21811.00 & 237475.26 & 1.14 & 0.95 & 1.09 \\
10231 & 101275 & 2001 & 1130.60 & -0.14 & 108294.00 & 1060793.55 & 1.04 & 0.94 & 0.98 \\
30828 & 105804 & 2001 & 570.80 & -0.18 & 57269.00 & 529021.03 & 1.00 & 0.93 & 0.92 \\
34621 & 106261 & 2001 & 363.10 & -0.12 & 32310.00 & 369360.20 & 1.12 & 1.02 & 1.14 \\
34632 & 106262 & 2001 & 423.20 & -0.25 & 39322.00 & 405143.63 & 1.08 & 0.96 & 1.03 \\
30785 & 105798 & 2001 & 1496.50 & -0.40 & 151904.00 & 1518944.01 & 0.99 & 1.01 & 1.00 \\
30856 & 105805 & 2001 & 48.80 & 0.24 & 2871.00 & 28291.21 & 1.70 & 0.58 & 0.99 \\
10048 & 101256 & 2001 & 7.20 & -0.12 & 699.00 & 6926.86 & 1.03 & 0.96 & 0.99 \\
31223 & 105868 & 2001 & 141.90 & -0.16 & 14194.00 & 138875.47 & 1.00 & 0.98 & 0.98 \\
31229 & 105869 & 2001 & 84.50 & -0.07 & 11820.00 & 118254.92 & 0.71 & 1.40 & 1.00 \\
33956 & 106193 & 2001 & 37.90 & -0.14 & 3706.00 & 36898.41 & 1.02 & 0.97 & 1.00 \\
52569 & 303130 & 2001 & 48.60 & -0.00 & 4890.00 & 44909.54 & 0.99 & 0.92 & 0.92 \\
9906 & 101211 & 2001 & 345.80 & 0.03 & 34623.00 & 328306.92 & 1.00 & 0.95 & 0.95 \\
31562 & 105909 & 2001 & 64.90 & 0.02 & 5967.00 & 60348.28 & 1.09 & 0.93 & 1.01 \\
33999 & 106197 & 2001 & 150.70 & 0.17 & 13585.00 & 108697.05 & 1.11 & 0.72 & 0.80 \\
34026 & 106198 & 2001 & 488.00 & 0.21 & 48899.00 & 420044.53 & 1.00 & 0.86 & 0.86 \\
52592 & 303140 & 2001 & 359.90 & -0.13 & 27536.00 & 319986.03 & 1.31 & 0.89 & 1.16 \\
9924 & 101212 & 2001 & 1006.00 & -0.30 & 100847.00 & 957311.70 & 1.00 & 0.95 & 0.95 \\
34053 & 106199 & 2001 & 181.60 & 0.10 & 18513.00 & 178968.57 & 0.98 & 0.99 & 0.97 \\
31497 & 105900 & 2001 & 9.90 & -0.13 & 993.00 & 9261.63 & 1.00 & 0.94 & 0.93 \\
31612 & 105917 & 2001 & 61.60 & -0.02 & 6142.00 & 55817.65 & 1.00 & 0.91 & 0.91 \\
9889 & 101200 & 2001 & 38.20 & -0.24 & 3721.00 & 37213.57 & 1.03 & 0.97 & 1.00 \\
31622 & 105918 & 2001 & 1180.40 & -0.44 & 117814.00 & 1068521.48 & 1.00 & 0.91 & 0.91 \\
33929 & 106192 & 2001 & 2183.00 & -0.09 & 253192.00 & 2107555.94 & 0.86 & 0.97 & 0.83 \\
52103 & 301571 & 2001 & 399.90 & 0.05 & 31495.00 & 385723.56 & 1.27 & 0.96 & 1.22 \\
33792 & 106170 & 2001 & 252.10 & 0.05 & 22513.00 & 233046.68 & 1.12 & 0.92 & 1.04 \\
33805 & 106172 & 2001 & 79.10 & -0.20 & 7520.00 & 73763.67 & 1.05 & 0.93 & 0.98 \\
31726 & 105932 & 2001 & 184.30 & -0.12 & 16298.00 & 165737.99 & 1.13 & 0.90 & 1.02 \\
52090 & 301560 & 2001 & 522.50 & 0.11 & 43975.00 & 491815.77 & 1.19 & 0.94 & 1.12 \\
33832 & 106173 & 2001 & 428.10 & 0.06 & 32063.00 & 372459.48 & 1.34 & 0.87 & 1.16 \\
31699 & 105931 & 2001 & 2622.20 & -0.08 & 199695.00 & 1916735.78 & 1.31 & 0.73 & 0.96 \\
52019 & 300777 & 2001 & 17.40 & -0.16 & 1362.00 & 14565.40 & 1.28 & 0.84 & 1.07 \\
9117 & 101112 & 2001 & 1076.20 & -0.14 & 97429.00 & 980552.39 & 1.10 & 0.91 & 1.01 \\
31680 & 105930 & 2001 & 608.80 & -0.34 & 60491.00 & 604896.94 & 1.01 & 0.99 & 1.00 \\
9861 & 101198 & 2001 & 215.60 & -0.23 & 19249.00 & 193964.47 & 1.12 & 0.90 & 1.01 \\
96701 & 611007 & 2001 & 155.40 & 0.42 & 17963.00 & 144710.44 & 0.87 & 0.93 & 0.81 \\
33883 & 106180 & 2001 & 54.10 & -0.14 & 5224.00 & 54191.55 & 1.04 & 1.00 & 1.04 \\
33909 & 106182 & 2001 & 794.90 & 0.19 & 47329.00 & 390325.18 & 1.68 & 0.49 & 0.82 \\
31646 & 105920 & 2001 & 4694.00 & -0.19 & 467905.00 & 4679051.54 & 1.00 & 1.00 & 1.00 \\
52060 & 301438 & 2001 & 129.60 & -0.06 & 12817.00 & 126226.83 & 1.01 & 0.97 & 0.98 \\
33922 & 106189 & 2001 & 599.30 & -0.19 & 64354.00 & 491235.99 & 0.93 & 0.82 & 0.76 \\
33874 & 106179 & 2001 & 85.10 & -0.16 & 8509.00 & 79779.61 & 1.00 & 0.94 & 0.94 \\
51996 & 300695 & 2001 & 131.30 & 0.08 & 13123.00 & 118691.77 & 1.00 & 0.90 & 0.90 \\
34095 & 106207 & 2001 & 23.70 & -0.08 & 2373.00 & 22598.62 & 1.00 & 0.95 & 0.95 \\
34234 & 106214 & 2001 & 134.70 & 0.10 & 9600.00 & 85391.92 & 1.40 & 0.63 & 0.89 \\
10006 & 101252 & 2001 & 102.90 & 0.04 & 9454.00 & 94543.84 & 1.09 & 0.92 & 1.00 \\
34261 & 106216 & 2001 & 595.40 & -0.21 & 51245.00 & 559777.40 & 1.16 & 0.94 & 1.09 \\
31318 & 105878 & 2001 & 1028.90 & 0.18 & 103492.00 & 897977.32 & 0.99 & 0.87 & 0.87 \\
51953 & 300673 & 2001 & 123.40 & 0.15 & 12426.00 & 117098.86 & 0.99 & 0.95 & 0.94 \\
96691 & 611006 & 2001 & 166.80 & -0.22 & 13110.00 & 161007.70 & 1.27 & 0.97 & 1.23 \\
31303 & 105876 & 2001 & 81.00 & -0.18 & 8107.00 & 74239.37 & 1.00 & 0.92 & 0.92 \\
10002 & 101251 & 2001 & 37.40 & -0.11 & 3307.00 & 36432.89 & 1.13 & 0.97 & 1.10 \\
31294 & 105875 & 2001 & 50.20 & -0.29 & 5040.00 & 48438.31 & 1.00 & 0.96 & 0.96 \\
31280 & 105874 & 2001 & 74.70 & 0.08 & 5506.00 & 53681.17 & 1.36 & 0.72 & 0.97 \\
31265 & 105872 & 2001 & 14.20 & -0.15 & 1415.00 & 13192.66 & 1.00 & 0.93 & 0.93 \\
31256 & 105871 & 2001 & 10.40 & 0.12 & 1004.00 & 9940.74 & 1.04 & 0.96 & 0.99 \\
34288 & 106220 & 2001 & 489.00 & -0.35 & 48184.00 & 475423.36 & 1.01 & 0.97 & 0.99 \\
34294 & 106221 & 2001 & 117.90 & -0.06 & 10896.00 & 116179.22 & 1.08 & 0.99 & 1.07 \\
9037 & 101109 & 2001 & 132.20 & -0.02 & 23346.00 & 204626.54 & 0.57 & 1.55 & 0.88 \\
34321 & 106222 & 2001 & 145.00 & -0.19 & 14518.00 & 142173.51 & 1.00 & 0.98 & 0.98 \\
51945 & 300657 & 2001 & 26.80 & -0.23 & 2395.00 & 27765.28 & 1.12 & 1.04 & 1.16 \\
52853 & 330794 & 2001 & 28.00 & -0.10 & 2737.00 & 27207.97 & 1.02 & 0.97 & 0.99 \\
52628 & 305184 & 2001 & 23.20 & -0.32 & 2455.00 & 20121.87 & 0.95 & 0.87 & 0.82 \\
31467 & 105895 & 2001 & 851.50 & 0.02 & 54458.00 & 550938.40 & 1.56 & 0.65 & 1.01 \\
9085 & 101111 & 2001 & 492.50 & 0.00 & 45572.00 & 432789.86 & 1.08 & 0.88 & 0.95 \\
34111 & 106208 & 2001 & 27.40 & -0.02 & 2752.00 & 26971.61 & 1.00 & 0.98 & 0.98 \\
31450 & 105890 & 2001 & 95.80 & -0.03 & 8060.00 & 97798.80 & 1.19 & 1.02 & 1.21 \\
67 & 100004 & 2001 & 1866.00 & -0.27 & 168034.00 & 1887763.08 & 1.11 & 1.01 & 1.12 \\
31440 & 105886 & 2001 & 50.00 & -0.14 & 5401.00 & 49523.88 & 0.93 & 0.99 & 0.92 \\
9959 & 101215 & 2001 & 33.00 & -0.13 & 4846.00 & 43782.34 & 0.68 & 1.33 & 0.90 \\
31431 & 105883 & 2001 & 48.20 & -0.12 & 4105.00 & 41017.92 & 1.17 & 0.85 & 1.00 \\
31346 & 105879 & 2001 & 1361.60 & -0.10 & 136155.00 & 1310627.97 & 1.00 & 0.96 & 0.96 \\
31418 & 105882 & 2001 & 155.90 & -0.18 & 13942.00 & 153264.99 & 1.12 & 0.98 & 1.10 \\
34138 & 106209 & 2001 & 134.70 & 0.24 & 13464.00 & 116223.09 & 1.00 & 0.86 & 0.86 \\
31391 & 105881 & 2001 & 517.80 & 0.12 & 39592.00 & 388030.66 & 1.31 & 0.75 & 0.98 \\
34192 & 106211 & 2001 & 76.70 & -0.10 & 7658.00 & 70384.65 & 1.00 & 0.92 & 0.92 \\
34204 & 106212 & 2001 & 96.20 & -0.07 & 9548.00 & 93922.27 & 1.01 & 0.98 & 0.98 \\
31364 & 105880 & 2001 & 1346.30 & 0.17 & 134332.00 & 1317584.87 & 1.00 & 0.98 & 0.98 \\
52618 & 303175 & 2001 & 880.00 & -0.02 & 82732.00 & 827319.04 & 1.06 & 0.94 & 1.00 \\
11261 & 101380 & 2001 & 272.30 & 0.06 & 23575.00 & 269982.01 & 1.16 & 0.99 & 1.15 \\
463 & 100068 & 2001 & 92.60 & -0.01 & 9171.00 & 86467.68 & 1.01 & 0.93 & 0.94 \\
36879 & 106620 & 2001 & 246.30 & -0.11 & 24243.00 & 224890.98 & 1.02 & 0.91 & 0.93 \\
39468 & 107697 & 2001 & 1.10 & -0.04 & 97.00 & 968.57 & 1.13 & 0.88 & 1.00 \\
1450 & 100200 & 2001 & 265.60 & 0.08 & 26504.00 & 263463.85 & 1.00 & 0.99 & 0.99 \\
12160 & 101513 & 2001 & 159.90 & 0.04 & 16085.00 & 152128.46 & 0.99 & 0.95 & 0.95 \\
39471 & 107699 & 2001 & 20.10 & -0.22 & 2082.00 & 17921.55 & 0.97 & 0.89 & 0.86 \\
39474 & 107701 & 2001 & 22.00 & -0.47 & 2305.00 & 22972.21 & 0.95 & 1.04 & 1.00 \\
39477 & 107702 & 2001 & 539.30 & -0.11 & 53778.00 & 528557.57 & 1.00 & 0.98 & 0.98 \\
39502 & 107710 & 2001 & 23.10 & -0.55 & 2351.00 & 23695.51 & 0.98 & 1.03 & 1.01 \\
49339 & 240284 & 2001 & 61.20 & -0.13 & 6152.00 & 62344.60 & 0.99 & 1.02 & 1.01 \\
25200 & 103460 & 2001 & 1037.50 & -0.33 & 115891.00 & 1090391.31 & 0.90 & 1.05 & 0.94 \\
1481 & 100207 & 2001 & 2331.10 & -0.17 & 233400.00 & 2100431.01 & 1.00 & 0.90 & 0.90 \\
1519 & 100209 & 2001 & 10746.70 & -0.33 & 1000583.00 & 10005089.43 & 1.07 & 0.93 & 1.00 \\
39505 & 107711 & 2001 & 25.70 & -0.14 & 2564.00 & 25041.52 & 1.00 & 0.97 & 0.98 \\
39510 & 107712 & 2001 & 16.40 & -0.10 & 1639.00 & 15548.93 & 1.00 & 0.95 & 0.95 \\
49318 & 240269 & 2001 & 187.60 & 0.10 & 16767.00 & 167665.18 & 1.12 & 0.89 & 1.00 \\
39456 & 107694 & 2001 & 5.60 & -0.01 & 556.00 & 5260.67 & 1.01 & 0.94 & 0.95 \\
53917 & 360123 & 2001 & 18.40 & -0.29 & 1606.00 & 17978.25 & 1.15 & 0.98 & 1.12 \\
65063 & 500659 & 2001 & 10.50 & -0.14 & 951.00 & 9008.94 & 1.10 & 0.86 & 0.95 \\
39335 & 107670 & 2001 & 620.40 & 0.01 & 62343.00 & 604276.73 & 1.00 & 0.97 & 0.97 \\
25381 & 103481 & 2001 & 119.10 & 0.02 & 11919.00 & 112195.66 & 1.00 & 0.94 & 0.94 \\
53894 & 360021 & 2001 & 14.10 & 0.02 & 1418.00 & 13220.18 & 0.99 & 0.94 & 0.93 \\
39347 & 107671 & 2001 & 17.60 & -0.11 & 1807.00 & 16077.97 & 0.97 & 0.91 & 0.89 \\
25354 & 103478 & 2001 & 2221.30 & -0.05 & 221977.00 & 2193655.68 & 1.00 & 0.99 & 0.99 \\
12126 & 101511 & 2001 & 173.00 & -0.27 & 14467.00 & 134906.18 & 1.20 & 0.78 & 0.93 \\
1369 & 100192 & 2001 & 84.70 & -0.06 & 8400.00 & 84004.79 & 1.01 & 0.99 & 1.00 \\
25148 & 103439 & 2001 & 76.50 & -0.02 & 7253.00 & 72532.90 & 1.05 & 0.95 & 1.00 \\
39368 & 107673 & 2001 & 69.10 & -0.01 & 5216.00 & 58916.76 & 1.32 & 0.85 & 1.13 \\
25313 & 103466 & 2001 & 1065.80 & -0.06 & 111808.00 & 984949.70 & 0.95 & 0.92 & 0.88 \\
49371 & 240286 & 2001 & 16.40 & -0.11 & 1601.00 & 15604.33 & 1.02 & 0.95 & 0.97 \\
39391 & 107680 & 2001 & 293.50 & -0.20 & 28022.00 & 288719.68 & 1.05 & 0.98 & 1.03 \\
39393 & 107691 & 2001 & 79.20 & -0.50 & 9679.00 & 77821.67 & 0.82 & 0.98 & 0.80 \\
39418 & 107692 & 2001 & 19.30 & -0.00 & 1918.00 & 17785.31 & 1.01 & 0.92 & 0.93 \\
25282 & 103464 & 2001 & 1072.20 & -0.24 & 119183.00 & 1109474.66 & 0.90 & 1.03 & 0.93 \\
1420 & 100196 & 2001 & 2642.80 & -0.07 & 252477.00 & 2475916.51 & 1.05 & 0.94 & 0.98 \\
39375 & 107677 & 2001 & 170.10 & 0.02 & 13847.00 & 148806.59 & 1.23 & 0.87 & 1.07 \\
39331 & 107666 & 2001 & 56.30 & -0.39 & 5630.00 & 55634.56 & 1.00 & 0.99 & 0.99 \\
1537 & 100213 & 2001 & 228.70 & -0.15 & 23013.00 & 210876.04 & 0.99 & 0.92 & 0.92 \\
12191 & 101518 & 2001 & 331.60 & -0.03 & 33154.00 & 314575.70 & 1.00 & 0.95 & 0.95 \\
39627 & 107830 & 2001 & 416.40 & -0.12 & 40569.00 & 421883.96 & 1.03 & 1.01 & 1.04 \\
24994 & 103406 & 2001 & 2126.10 & -0.16 & 202013.00 & 1747935.45 & 1.05 & 0.82 & 0.87 \\
64888 & 500639 & 2001 & 13.50 & -0.16 & 1614.00 & 13541.79 & 0.84 & 1.00 & 0.84 \\
12231 & 101528 & 2001 & 75.00 & -0.09 & 7705.00 & 68593.77 & 0.97 & 0.91 & 0.89 \\
53959 & 362424 & 2001 & 24.00 & 0.04 & 2466.00 & 22145.68 & 0.97 & 0.92 & 0.90 \\
64865 & 500638 & 2001 & 45.00 & -0.05 & 4789.00 & 44945.55 & 0.94 & 1.00 & 0.94 \\
39637 & 107832 & 2001 & 155.10 & -0.04 & 12877.00 & 145985.64 & 1.20 & 0.94 & 1.13 \\
24942 & 103395 & 2001 & 111.80 & -0.09 & 11173.00 & 99206.20 & 1.00 & 0.89 & 0.89 \\
12246 & 101530 & 2001 & 1093.20 & 0.29 & 83714.00 & 918309.10 & 1.31 & 0.84 & 1.10 \\
1685 & 100223 & 2001 & 3315.80 & -0.05 & 333768.00 & 3052847.03 & 0.99 & 0.92 & 0.91 \\
39664 & 107833 & 2001 & 434.70 & -0.18 & 38910.00 & 441609.11 & 1.12 & 1.02 & 1.13 \\
39675 & 107834 & 2001 & 231.90 & -0.13 & 20767.00 & 223858.94 & 1.12 & 0.97 & 1.08 \\
49283 & 240264 & 2001 & 646.00 & -0.24 & 64572.00 & 555020.64 & 1.00 & 0.86 & 0.86 \\
12222 & 101523 & 2001 & 162.70 & -0.08 & 26626.00 & 244263.23 & 0.61 & 1.50 & 0.92 \\
64934 & 500646 & 2001 & 1.90 & -0.18 & 182.00 & 1796.66 & 1.04 & 0.95 & 0.99 \\
39524 & 107719 & 2001 & 232.10 & -0.16 & 17247.00 & 176515.86 & 1.35 & 0.76 & 1.02 \\
39549 & 107720 & 2001 & 74.30 & 0.01 & 7435.00 & 72069.47 & 1.00 & 0.97 & 0.97 \\
25113 & 103432 & 2001 & 1847.40 & -0.25 & 170745.00 & 1773653.35 & 1.08 & 0.96 & 1.04 \\
64956 & 500651 & 2001 & 9.20 & -0.23 & 1183.00 & 9922.76 & 0.78 & 1.08 & 0.84 \\
64946 & 500648 & 2001 & 1.90 & 0.12 & 139.00 & 1309.14 & 1.37 & 0.69 & 0.94 \\
39563 & 107722 & 2001 & 109.00 & -0.15 & 8960.00 & 102551.77 & 1.22 & 0.94 & 1.14 \\
25138 & 103436 & 2001 & 9.20 & -0.22 & 920.00 & 8191.92 & 1.00 & 0.89 & 0.89 \\
25075 & 103429 & 2001 & 2660.80 & -0.14 & 237536.00 & 2141969.05 & 1.12 & 0.81 & 0.90 \\
8099 & 101074 & 2001 & 197.30 & -0.23 & 27411.00 & 236270.26 & 0.72 & 1.20 & 0.86 \\
53932 & 361852 & 2001 & 6.20 & -0.08 & 628.00 & 6090.13 & 0.99 & 0.98 & 0.97 \\
1568 & 100214 & 2001 & 228.60 & -0.22 & 22090.00 & 188955.37 & 1.03 & 0.83 & 0.86 \\
39579 & 107726 & 2001 & 906.00 & -0.07 & 89876.00 & 890022.30 & 1.01 & 0.98 & 0.99 \\
25037 & 103426 & 2001 & 1267.50 & -0.19 & 109121.00 & 978661.68 & 1.16 & 0.77 & 0.90 \\
53936 & 361995 & 2001 & 12.90 & 0.02 & 1615.00 & 16092.62 & 0.80 & 1.25 & 1.00 \\
1597 & 100217 & 2001 & 34.70 & -0.33 & 3872.00 & 32171.50 & 0.90 & 0.93 & 0.83 \\
64940 & 500647 & 2001 & 29.70 & -0.15 & 2776.00 & 23315.32 & 1.07 & 0.79 & 0.84 \\
49292 & 240266 & 2001 & 386.80 & -0.39 & 39293.00 & 392505.93 & 0.98 & 1.01 & 1.00 \\
1350 & 100190 & 2001 & 2548.70 & -0.21 & 252832.00 & 2518614.39 & 1.01 & 0.99 & 1.00 \\
49379 & 240287 & 2001 & 11.40 & -0.10 & 1115.00 & 11432.09 & 1.02 & 1.00 & 1.03 \\
1104 & 100153 & 2001 & 173.10 & -0.04 & 17326.00 & 169390.95 & 1.00 & 0.98 & 0.98 \\
39058 & 107598 & 2001 & 34.00 & -0.07 & 2950.00 & 31418.06 & 1.15 & 0.92 & 1.07 \\
39083 & 107604 & 2001 & 371.90 & -0.36 & 32451.00 & 358308.86 & 1.15 & 0.96 & 1.10 \\
25760 & 103521 & 2001 & 4469.60 & -0.12 & 512885.00 & 4880700.13 & 0.87 & 1.09 & 0.95 \\
49407 & 240293 & 2001 & 1248.70 & -0.15 & 127152.00 & 1189111.93 & 0.98 & 0.95 & 0.94 \\
39095 & 107605 & 2001 & 26.80 & 0.04 & 2618.00 & 25279.17 & 1.02 & 0.94 & 0.97 \\
39120 & 107607 & 2001 & 52.60 & 0.03 & 5273.00 & 49396.34 & 1.00 & 0.94 & 0.94 \\
25728 & 103520 & 2001 & 10271.80 & -0.13 & 1077047.00 & 10494292.63 & 0.95 & 1.02 & 0.97 \\
65485 & 500700 & 2001 & 45.90 & -0.07 & 2473.00 & 19828.99 & 1.86 & 0.43 & 0.80 \\
49397 & 240291 & 2001 & 101.70 & -0.41 & 7939.00 & 94207.02 & 1.28 & 0.93 & 1.19 \\
39130 & 107608 & 2001 & 66.40 & 0.29 & 6653.00 & 63274.92 & 1.00 & 0.95 & 0.95 \\
39146 & 107611 & 2001 & 879.50 & 0.05 & 79389.00 & 772714.33 & 1.11 & 0.88 & 0.97 \\
25696 & 103514 & 2001 & 2491.40 & -0.02 & 236431.00 & 2254854.61 & 1.05 & 0.91 & 0.95 \\
1139 & 100155 & 2001 & 1693.30 & -0.13 & 169602.00 & 1635601.39 & 1.00 & 0.97 & 0.96 \\
1154 & 100157 & 2001 & 1046.60 & -0.12 & 104783.00 & 1029601.65 & 1.00 & 0.98 & 0.98 \\
11991 & 101476 & 2001 & 3378.70 & -0.11 & 378261.00 & 2864983.61 & 0.89 & 0.85 & 0.76 \\
25794 & 103523 & 2001 & 4596.30 & -0.09 & 522775.00 & 4417397.67 & 0.88 & 0.96 & 0.84 \\
8210 & 101079 & 2001 & 226.20 & -0.12 & 24062.00 & 208741.56 & 0.94 & 0.92 & 0.87 \\
39029 & 107573 & 2001 & 327.40 & -0.15 & 32763.00 & 325011.63 & 1.00 & 0.99 & 0.99 \\
1074 & 100150 & 2001 & 21.10 & -0.12 & 2073.00 & 18220.16 & 1.02 & 0.86 & 0.88 \\
38923 & 107354 & 2001 & 37.50 & -0.02 & 3516.00 & 35852.43 & 1.07 & 0.96 & 1.02 \\
38948 & 107358 & 2001 & 8.90 & 0.12 & 832.00 & 6859.57 & 1.07 & 0.77 & 0.82 \\
25895 & 103526 & 2001 & 3493.00 & -0.14 & 429381.00 & 3664785.08 & 0.81 & 1.05 & 0.85 \\
38973 & 107361 & 2001 & 40.70 & 0.14 & 4751.00 & 45189.94 & 0.86 & 1.11 & 0.95 \\
49423 & 240296 & 2001 & 667.90 & 0.07 & 65345.00 & 655338.41 & 1.02 & 0.98 & 1.00 \\
38976 & 107362 & 2001 & 81.80 & -0.11 & 8167.00 & 78808.38 & 1.00 & 0.96 & 0.96 \\
65462 & 500699 & 2001 & 73.10 & -0.16 & 8872.00 & 73191.08 & 0.82 & 1.00 & 0.82 \\
65556 & 500704 & 2001 & 1.30 & 0.01 & 115.00 & 1207.27 & 1.13 & 0.93 & 1.05 \\
65531 & 500702 & 2001 & 18.20 & 0.05 & 2656.00 & 25057.38 & 0.69 & 1.38 & 0.94 \\
38986 & 107470 & 2001 & 3.00 & -0.54 & 304.00 & 3003.59 & 0.99 & 1.00 & 0.99 \\
49415 & 240295 & 2001 & 1354.70 & -0.11 & 131803.00 & 1271233.92 & 1.03 & 0.94 & 0.96 \\
38992 & 107563 & 2001 & 936.80 & -0.12 & 83706.00 & 879729.74 & 1.12 & 0.94 & 1.05 \\
25828 & 103524 & 2001 & 87013.90 & -0.12 & 9150702.00 & 85048603.20 & 0.95 & 0.98 & 0.93 \\
65508 & 500701 & 2001 & 20.00 & -0.02 & 3327.00 & 29400.45 & 0.60 & 1.47 & 0.88 \\
39019 & 107565 & 2001 & 1.80 & -0.07 & 166.00 & 1772.79 & 1.08 & 0.98 & 1.07 \\
39023 & 107566 & 2001 & 2.70 & 0.11 & 237.00 & 2568.27 & 1.14 & 0.95 & 1.08 \\
25861 & 103525 & 2001 & 35513.30 & -0.12 & 3606048.00 & 33387463.62 & 0.98 & 0.94 & 0.93 \\
53877 & 360020 & 2001 & 52.00 & -0.10 & 5209.00 & 51460.92 & 1.00 & 0.99 & 0.99 \\
53789 & 357122 & 2001 & 435.80 & 0.03 & 23375.00 & 208236.19 & 1.86 & 0.48 & 0.89 \\
39300 & 107650 & 2001 & 122.00 & -0.34 & 12217.00 & 117384.80 & 1.00 & 0.96 & 0.96 \\
25498 & 103494 & 2001 & 303.00 & -0.13 & 30336.00 & 296256.02 & 1.00 & 0.98 & 0.98 \\
1255 & 100167 & 2001 & 385.70 & -0.17 & 34976.00 & 388639.31 & 1.10 & 1.01 & 1.11 \\
65295 & 500682 & 2001 & 7.60 & 0.00 & 694.00 & 6689.60 & 1.10 & 0.88 & 0.96 \\
53870 & 359773 & 2001 & 3.80 & -0.06 & 404.00 & 3834.31 & 0.94 & 1.01 & 0.95 \\
1273 & 100171 & 2001 & 733.90 & 0.02 & 59696.00 & 627633.92 & 1.23 & 0.86 & 1.05 \\
39306 & 107652 & 2001 & 144.60 & -0.18 & 14487.00 & 141181.62 & 1.00 & 0.98 & 0.97 \\
39312 & 107653 & 2001 & 36.60 & -0.11 & 3604.00 & 35542.57 & 1.02 & 0.97 & 0.99 \\
12083 & 101497 & 2001 & 1784.60 & -0.06 & 178092.00 & 1579020.16 & 1.00 & 0.88 & 0.89 \\
49387 & 240288 & 2001 & 4.50 & -0.08 & 430.00 & 4407.01 & 1.05 & 0.98 & 1.02 \\
25435 & 103487 & 2001 & 112.20 & -0.14 & 11246.00 & 107068.35 & 1.00 & 0.95 & 0.95 \\
65134 & 500664 & 2001 & 1869.40 & -0.04 & 195188.00 & 1737865.56 & 0.96 & 0.93 & 0.89 \\
25411 & 103483 & 2001 & 686.90 & -0.16 & 68722.00 & 675913.84 & 1.00 & 0.98 & 0.98 \\
12103 & 101503 & 2001 & 208.20 & -0.34 & 21024.00 & 203570.76 & 0.99 & 0.98 & 0.97 \\
39284 & 107648 & 2001 & 538.70 & -0.65 & 53729.00 & 537212.15 & 1.00 & 1.00 & 1.00 \\
12066 & 101494 & 2001 & 323.90 & -0.31 & 32849.00 & 309578.37 & 0.99 & 0.96 & 0.94 \\
39280 & 107641 & 2001 & 480.40 & -0.12 & 45219.00 & 443542.68 & 1.06 & 0.92 & 0.98 \\
39171 & 107616 & 2001 & 48.40 & 0.05 & 4913.00 & 46573.35 & 0.99 & 0.96 & 0.95 \\
39196 & 107618 & 2001 & 2809.70 & -0.15 & 314467.00 & 2667035.13 & 0.89 & 0.95 & 0.85 \\
25630 & 103498 & 2001 & 261.80 & -0.18 & 26170.00 & 255286.80 & 1.00 & 0.98 & 0.98 \\
65398 & 500694 & 2001 & 17.70 & 0.02 & 1801.00 & 17450.74 & 0.98 & 0.99 & 0.97 \\
53809 & 357133 & 2001 & 44.10 & -0.15 & 4415.00 & 43776.24 & 1.00 & 0.99 & 0.99 \\
1186 & 100159 & 2001 & 85.00 & -0.13 & 8564.00 & 85636.39 & 0.99 & 1.01 & 1.00 \\
53832 & 357756 & 2001 & 36.70 & -0.13 & 3559.00 & 35286.96 & 1.03 & 0.96 & 0.99 \\
1224 & 100166 & 2001 & 7924.40 & -0.05 & 636764.00 & 5530455.59 & 1.24 & 0.70 & 0.87 \\
39229 & 107623 & 2001 & 6.10 & -0.19 & 612.00 & 6020.29 & 1.00 & 0.99 & 0.98 \\
25586 & 103497 & 2001 & 57.60 & 0.04 & 5784.00 & 53148.22 & 1.00 & 0.92 & 0.92 \\
39255 & 107627 & 2001 & 627.90 & 0.03 & 64249.00 & 524448.56 & 0.98 & 0.84 & 0.82 \\
39276 & 107636 & 2001 & 20.70 & -0.52 & 1761.00 & 18903.76 & 1.18 & 0.91 & 1.07 \\
25555 & 103496 & 2001 & 495.70 & -0.11 & 51905.00 & 519047.08 & 0.96 & 1.05 & 1.00 \\
53853 & 359285 & 2001 & 160.20 & 0.10 & 16033.00 & 134368.70 & 1.00 & 0.84 & 0.84 \\
65369 & 500692 & 2001 & 58.60 & -0.14 & 5030.00 & 41472.34 & 1.17 & 0.71 & 0.82 \\
65297 & 500684 & 2001 & 42.20 & -0.01 & 4317.00 & 42332.16 & 0.98 & 1.00 & 0.98 \\
39238 & 107626 & 2001 & 435.60 & -0.05 & 44046.00 & 372122.28 & 0.99 & 0.85 & 0.84 \\
24883 & 103383 & 2001 & 1313.00 & -0.28 & 126043.00 & 1120174.83 & 1.04 & 0.85 & 0.89 \\
49268 & 240261 & 2001 & 136.40 & -0.02 & 13629.00 & 124086.44 & 1.00 & 0.91 & 0.91 \\
1704 & 100226 & 2001 & 9142.80 & 0.06 & 779877.00 & 6354320.77 & 1.17 & 0.70 & 0.81 \\
2008 & 100280 & 2001 & 40.80 & -0.02 & 4126.00 & 40077.90 & 0.99 & 0.98 & 0.97 \\
40223 & 108074 & 2001 & 12.50 & -0.19 & 1247.00 & 12450.69 & 1.00 & 1.00 & 1.00 \\
40234 & 108082 & 2001 & 18.70 & -0.18 & 1650.00 & 15807.67 & 1.13 & 0.85 & 0.96 \\
24165 & 103294 & 2001 & 2626.80 & 0.19 & 141107.00 & 1321337.12 & 1.86 & 0.50 & 0.94 \\
12507 & 101544 & 2001 & 252.90 & -0.17 & 25043.00 & 250407.65 & 1.01 & 0.99 & 1.00 \\
64481 & 500605 & 2001 & 73.70 & -0.05 & 6619.00 & 62993.49 & 1.11 & 0.85 & 0.95 \\
40261 & 108083 & 2001 & 56.70 & 0.03 & 5305.00 & 56509.62 & 1.07 & 1.00 & 1.07 \\
12519 & 101545 & 2001 & 170.10 & 0.12 & 16606.00 & 165925.00 & 1.02 & 0.98 & 1.00 \\
40291 & 108087 & 2001 & 70.20 & -0.55 & 5678.00 & 63851.11 & 1.24 & 0.91 & 1.12 \\
24115 & 103267 & 2001 & 1391.50 & -0.19 & 136380.00 & 1363527.08 & 1.02 & 0.98 & 1.00 \\
49064 & 240212 & 2001 & 5733.30 & -0.20 & 587931.00 & 5603994.59 & 0.98 & 0.98 & 0.95 \\
40317 & 108109 & 2001 & 35.50 & 0.11 & 3137.00 & 27858.96 & 1.13 & 0.78 & 0.89 \\
24099 & 103266 & 2001 & 407.40 & -0.34 & 39746.00 & 397400.91 & 1.03 & 0.98 & 1.00 \\
64442 & 500603 & 2001 & 104.10 & -0.06 & 7880.00 & 82455.84 & 1.32 & 0.79 & 1.05 \\
24087 & 103264 & 2001 & 919.50 & -0.16 & 92118.00 & 898744.32 & 1.00 & 0.98 & 0.98 \\
54087 & 364393 & 2001 & 55.10 & -0.06 & 4236.00 & 44883.90 & 1.30 & 0.81 & 1.06 \\
24200 & 103296 & 2001 & 1486.80 & -0.29 & 185807.00 & 1697208.50 & 0.80 & 1.14 & 0.91 \\
40209 & 108073 & 2001 & 160.00 & 0.07 & 15010.00 & 150096.44 & 1.07 & 0.94 & 1.00 \\
7979 & 101068 & 2001 & 79515.20 & -0.12 & 7171298.00 & 74894329.09 & 1.11 & 0.94 & 1.04 \\
64527 & 500607 & 2001 & 400.10 & -0.26 & 35977.00 & 420053.76 & 1.11 & 1.05 & 1.17 \\
40097 & 108029 & 2001 & 502.80 & -0.11 & 45848.00 & 486414.36 & 1.10 & 0.97 & 1.06 \\
24318 & 103308 & 2001 & 6764.60 & -0.16 & 613721.00 & 6542004.46 & 1.10 & 0.97 & 1.07 \\
49107 & 240222 & 2001 & 712.10 & -0.12 & 71782.00 & 700560.66 & 0.99 & 0.98 & 0.98 \\
40124 & 108030 & 2001 & 20.60 & -0.02 & 1993.00 & 19925.06 & 1.03 & 0.97 & 1.00 \\
40129 & 108037 & 2001 & 153.40 & -0.38 & 9347.00 & 146137.47 & 1.64 & 0.95 & 1.56 \\
24284 & 103304 & 2001 & 39.30 & 0.16 & 3276.00 & 29654.33 & 1.20 & 0.75 & 0.91 \\
40343 & 108112 & 2001 & 21.40 & 0.14 & 1936.00 & 19359.97 & 1.11 & 0.90 & 1.00 \\
1971 & 100263 & 2001 & 5.60 & -0.15 & 576.00 & 5107.05 & 0.97 & 0.91 & 0.89 \\
24266 & 103301 & 2001 & 1441.50 & -0.29 & 176510.00 & 1383642.84 & 0.82 & 0.96 & 0.78 \\
54056 & 364292 & 2001 & 428.20 & 0.00 & 42266.00 & 421407.47 & 1.01 & 0.98 & 1.00 \\
1984 & 100278 & 2001 & 95.10 & -0.06 & 9494.00 & 93225.16 & 1.00 & 0.98 & 0.98 \\
12479 & 101542 & 2001 & 57.80 & -0.04 & 5435.00 & 53312.35 & 1.06 & 0.92 & 0.98 \\
40163 & 108070 & 2001 & 88.30 & 0.08 & 8618.00 & 84853.04 & 1.02 & 0.96 & 0.98 \\
24235 & 103299 & 2001 & 411.60 & -0.17 & 46482.00 & 390138.82 & 0.89 & 0.95 & 0.84 \\
64504 & 500606 & 2001 & 223.00 & 0.02 & 18986.00 & 185776.50 & 1.17 & 0.83 & 0.98 \\
40153 & 108051 & 2001 & 487.40 & -0.05 & 48642.00 & 480926.80 & 1.00 & 0.99 & 0.99 \\
64547 & 500609 & 2001 & 95.70 & -0.05 & 7436.00 & 79741.28 & 1.29 & 0.83 & 1.07 \\
54110 & 364518 & 2001 & 64.40 & -0.01 & 4411.00 & 60046.22 & 1.46 & 0.93 & 1.36 \\
24056 & 103259 & 2001 & 2342.30 & 0.13 & 198940.00 & 2119302.56 & 1.18 & 0.90 & 1.07 \\
23896 & 103228 & 2001 & 236.00 & -0.33 & 17893.00 & 196194.25 & 1.32 & 0.83 & 1.10 \\
2062 & 100287 & 2001 & 61.10 & -0.16 & 6137.00 & 57444.59 & 1.00 & 0.94 & 0.94 \\
7909 & 101065 & 2001 & 3081.50 & -0.25 & 301340.00 & 2840452.49 & 1.02 & 0.92 & 0.94 \\
40550 & 108138 & 2001 & 13.30 & -0.47 & 1299.00 & 12983.82 & 1.02 & 0.98 & 1.00 \\
23860 & 103224 & 2001 & 161.00 & -0.32 & 14483.00 & 127052.68 & 1.11 & 0.79 & 0.88 \\
40579 & 108140 & 2001 & 141.60 & -0.11 & 14102.00 & 141017.14 & 1.00 & 1.00 & 1.00 \\
40588 & 108141 & 2001 & 58.60 & 0.07 & 5173.00 & 56654.16 & 1.13 & 0.97 & 1.10 \\
23827 & 103214 & 2001 & 2108.90 & -0.34 & 212099.00 & 2065568.48 & 0.99 & 0.98 & 0.97 \\
40613 & 108142 & 2001 & 27.70 & -0.01 & 2770.00 & 26886.91 & 1.00 & 0.97 & 0.97 \\
40638 & 108143 & 2001 & 26.30 & -0.02 & 2575.00 & 25752.46 & 1.02 & 0.98 & 1.00 \\
23797 & 103213 & 2001 & 976.30 & -0.32 & 97757.00 & 933428.83 & 1.00 & 0.96 & 0.95 \\
12611 & 101560 & 2001 & 26.40 & 0.11 & 2453.00 & 25243.94 & 1.08 & 0.96 & 1.03 \\
64373 & 500600 & 2001 & 398.00 & -0.02 & 27058.00 & 279579.12 & 1.47 & 0.70 & 1.03 \\
23932 & 103242 & 2001 & 327.80 & 0.19 & 32708.00 & 322363.49 & 1.00 & 0.98 & 0.99 \\
2036 & 100286 & 2001 & 45.30 & -0.06 & 4525.00 & 42471.01 & 1.00 & 0.94 & 0.94 \\
40395 & 108117 & 2001 & 972.90 & -0.20 & 97294.00 & 880747.19 & 1.00 & 0.91 & 0.91 \\
24034 & 103255 & 2001 & 145.70 & -0.19 & 14380.00 & 143346.38 & 1.01 & 0.98 & 1.00 \\
12548 & 101553 & 2001 & 233.90 & -0.36 & 23364.00 & 218983.24 & 1.00 & 0.94 & 0.94 \\
40421 & 108118 & 2001 & 85.20 & -0.02 & 7200.00 & 75708.46 & 1.18 & 0.89 & 1.05 \\
54133 & 364519 & 2001 & 73.10 & 0.06 & 5861.00 & 69565.04 & 1.25 & 0.95 & 1.19 \\
40429 & 108119 & 2001 & 197.80 & -0.11 & 19923.00 & 178864.75 & 0.99 & 0.90 & 0.90 \\
24004 & 103253 & 2001 & 175.70 & 0.17 & 17492.00 & 173132.96 & 1.00 & 0.99 & 0.99 \\
40369 & 108115 & 2001 & 1443.50 & -0.12 & 143508.00 & 1430747.68 & 1.01 & 0.99 & 1.00 \\
64419 & 500602 & 2001 & 64.10 & -0.04 & 3408.00 & 35542.92 & 1.88 & 0.55 & 1.04 \\
23983 & 103252 & 2001 & 275.50 & 0.01 & 26799.00 & 267600.52 & 1.03 & 0.97 & 1.00 \\
40481 & 108122 & 2001 & 68.70 & 0.09 & 5519.00 & 52650.67 & 1.24 & 0.77 & 0.95 \\
23962 & 103251 & 2001 & 358.60 & -0.30 & 34564.00 & 345637.06 & 1.04 & 0.96 & 1.00 \\
54175 & 364809 & 2001 & 28.30 & -0.01 & 2951.00 & 27053.52 & 0.96 & 0.96 & 0.92 \\
12572 & 101554 & 2001 & 166.90 & -0.16 & 16614.00 & 162961.35 & 1.00 & 0.98 & 0.98 \\
40507 & 108134 & 2001 & 77.30 & -0.10 & 8143.00 & 68912.46 & 0.95 & 0.89 & 0.85 \\
40532 & 108136 & 2001 & 6.50 & -0.10 & 819.00 & 8187.71 & 0.79 & 1.26 & 1.00 \\
40455 & 108121 & 2001 & 1127.80 & 0.14 & 84641.00 & 826625.80 & 1.33 & 0.73 & 0.98 \\
12445 & 101541 & 2001 & 384.20 & -0.09 & 37721.00 & 377217.45 & 1.02 & 0.98 & 1.00 \\
24344 & 103315 & 2001 & 55.30 & -0.06 & 5538.00 & 54699.88 & 1.00 & 0.99 & 0.99 \\
40073 & 108021 & 2001 & 960.80 & -0.06 & 91499.00 & 875838.15 & 1.05 & 0.91 & 0.96 \\
39805 & 107875 & 2001 & 304.70 & -0.09 & 31502.00 & 254293.55 & 0.97 & 0.83 & 0.81 \\
39809 & 107876 & 2001 & 753.40 & -0.25 & 123888.00 & 1285114.00 & 0.61 & 1.71 & 1.04 \\
39811 & 107877 & 2001 & 4.50 & -0.25 & 454.00 & 4378.49 & 0.99 & 0.97 & 0.96 \\
24726 & 103376 & 2001 & 6394.40 & -0.19 & 584452.00 & 5750796.35 & 1.09 & 0.90 & 0.98 \\
53988 & 363013 & 2001 & 32.50 & -0.02 & 2618.00 & 26185.76 & 1.24 & 0.81 & 1.00 \\
1787 & 100237 & 2001 & 91.70 & -0.14 & 11636.00 & 102992.55 & 0.79 & 1.12 & 0.89 \\
64764 & 500628 & 2001 & 58.90 & -0.30 & 6654.00 & 52075.41 & 0.89 & 0.88 & 0.78 \\
12327 & 101536 & 2001 & 2103.30 & -0.11 & 210235.00 & 2053227.41 & 1.00 & 0.98 & 0.98 \\
24686 & 103375 & 2001 & 948.30 & -0.35 & 88046.00 & 864148.63 & 1.08 & 0.91 & 0.98 \\
64740 & 500625 & 2001 & 393.30 & -0.70 & 77329.00 & 398263.01 & 0.51 & 1.01 & 0.52 \\
49240 & 240254 & 2001 & 651.60 & 0.18 & 54020.00 & 645322.52 & 1.21 & 0.99 & 1.19 \\
1825 & 100244 & 2001 & 200.10 & -0.22 & 15459.00 & 199044.00 & 1.29 & 0.99 & 1.29 \\
39866 & 107892 & 2001 & 674.50 & -0.07 & 77743.00 & 563621.49 & 0.87 & 0.84 & 0.72 \\
24646 & 103373 & 2001 & 66.50 & -0.15 & 7358.00 & 68872.19 & 0.90 & 1.04 & 0.94 \\
39799 & 107874 & 2001 & 182.00 & 0.06 & 18622.00 & 160237.62 & 0.98 & 0.88 & 0.86 \\
1768 & 100228 & 2001 & 140.40 & -0.13 & 13804.00 & 140297.77 & 1.02 & 1.00 & 1.02 \\
24763 & 103377 & 2001 & 1379.60 & -0.26 & 128345.00 & 1237673.41 & 1.07 & 0.90 & 0.96 \\
64842 & 500636 & 2001 & 50.30 & -0.02 & 3880.00 & 37684.65 & 1.30 & 0.75 & 0.97 \\
39709 & 107837 & 2001 & 18.70 & 0.08 & 1853.00 & 16166.47 & 1.01 & 0.86 & 0.87 \\
39730 & 107858 & 2001 & 15.80 & 0.04 & 1752.00 & 13869.56 & 0.90 & 0.88 & 0.79 \\
24842 & 103381 & 2001 & 28042.40 & -0.15 & 2558083.00 & 26183780.83 & 1.10 & 0.93 & 1.02 \\
64819 & 500635 & 2001 & 21.40 & -0.12 & 2430.00 & 21379.83 & 0.88 & 1.00 & 0.88 \\
64796 & 500634 & 2001 & 25.30 & 0.03 & 2573.00 & 25426.21 & 0.98 & 1.00 & 0.99 \\
8069 & 101073 & 2001 & 8069.70 & -0.24 & 876559.00 & 7792545.47 & 0.92 & 0.97 & 0.89 \\
49213 & 240250 & 2001 & 358.50 & 0.13 & 36126.00 & 331377.54 & 0.99 & 0.92 & 0.92 \\
1748 & 100227 & 2001 & 145.40 & -0.26 & 13535.00 & 130663.74 & 1.07 & 0.90 & 0.97 \\
24802 & 103380 & 2001 & 5190.00 & -0.18 & 491193.00 & 4924178.67 & 1.06 & 0.95 & 1.00 \\
53982 & 362981 & 2001 & 315.80 & 0.02 & 29559.00 & 313812.85 & 1.07 & 0.99 & 1.06 \\
64773 & 500633 & 2001 & 22.00 & -0.07 & 1971.00 & 17688.20 & 1.12 & 0.80 & 0.90 \\
12309 & 101534 & 2001 & 1427.50 & 0.04 & 142338.00 & 1387178.48 & 1.00 & 0.97 & 0.97 \\
39768 & 107868 & 2001 & 45.60 & -0.11 & 4558.00 & 45581.12 & 1.00 & 1.00 & 1.00 \\
39773 & 107870 & 2001 & 413.10 & -0.30 & 37200.00 & 361080.75 & 1.11 & 0.87 & 0.97 \\
39789 & 107872 & 2001 & 60.40 & -0.07 & 5298.00 & 54000.15 & 1.14 & 0.89 & 1.02 \\
39761 & 107863 & 2001 & 589.00 & 0.04 & 62406.00 & 532342.16 & 0.94 & 0.90 & 0.85 \\
64716 & 500621 & 2001 & 95.80 & 0.00 & 6161.00 & 59854.81 & 1.55 & 0.62 & 0.97 \\
24474 & 103328 & 2001 & 780.20 & -0.13 & 75518.00 & 727156.26 & 1.03 & 0.93 & 0.96 \\
39991 & 107968 & 2001 & 80.10 & -0.08 & 7998.00 & 76139.88 & 1.00 & 0.95 & 0.95 \\
24452 & 103327 & 2001 & 1499.50 & -0.05 & 164924.00 & 1577649.34 & 0.91 & 1.05 & 0.96 \\
40001 & 107992 & 2001 & 36.50 & -0.15 & 3241.00 & 35863.74 & 1.13 & 0.98 & 1.11 \\
12411 & 101539 & 2001 & 1396.40 & -0.13 & 139134.00 & 1375034.42 & 1.00 & 0.98 & 0.99 \\
40009 & 107994 & 2001 & 1272.50 & -0.44 & 171338.00 & 1141950.14 & 0.74 & 0.90 & 0.67 \\
24422 & 103326 & 2001 & 1596.70 & -0.21 & 165672.00 & 1670694.61 & 0.96 & 1.05 & 1.01 \\
39981 & 107967 & 2001 & 166.90 & -0.18 & 16705.00 & 154558.16 & 1.00 & 0.93 & 0.93 \\
49148 & 240234 & 2001 & 89.70 & 0.23 & 8405.00 & 79651.91 & 1.07 & 0.89 & 0.95 \\
24406 & 103319 & 2001 & 279.00 & -0.11 & 26377.00 & 269943.88 & 1.06 & 0.97 & 1.02 \\
40037 & 108013 & 2001 & 235.40 & -0.15 & 21956.00 & 233794.33 & 1.07 & 0.99 & 1.06 \\
24372 & 103318 & 2001 & 1641.10 & -0.01 & 162336.00 & 1623360.19 & 1.01 & 0.99 & 1.00 \\
8011 & 101069 & 2001 & 8602.10 & -0.12 & 785807.00 & 8334453.60 & 1.09 & 0.97 & 1.06 \\
40045 & 108018 & 2001 & 308.60 & -0.03 & 34883.00 & 284030.48 & 0.88 & 0.92 & 0.81 \\
1936 & 100259 & 2001 & 140.80 & -0.07 & 14179.00 & 139335.60 & 0.99 & 0.99 & 0.98 \\
11957 & 101473 & 2001 & 2073.30 & -0.22 & 207953.00 & 1984543.56 & 1.00 & 0.96 & 0.95 \\
39973 & 107964 & 2001 & 544.40 & -0.19 & 54521.00 & 545204.13 & 1.00 & 1.00 & 1.00 \\
24617 & 103372 & 2001 & 936.50 & -0.00 & 93508.00 & 824986.45 & 1.00 & 0.88 & 0.88 \\
39895 & 107928 & 2001 & 2130.70 & -0.13 & 214579.00 & 2050020.70 & 0.99 & 0.96 & 0.96 \\
12360 & 101537 & 2001 & 602.20 & -0.07 & 60261.00 & 570382.81 & 1.00 & 0.95 & 0.95 \\
1859 & 100245 & 2001 & 498.50 & -0.15 & 44648.00 & 493466.41 & 1.12 & 0.99 & 1.11 \\
39921 & 107938 & 2001 & 147.50 & -0.13 & 11894.00 & 122440.45 & 1.24 & 0.83 & 1.03 \\
24574 & 103369 & 2001 & 728.20 & -0.27 & 62275.00 & 726742.69 & 1.17 & 1.00 & 1.17 \\
64666 & 500618 & 2001 & 58.00 & -0.14 & 4004.00 & 39133.60 & 1.45 & 0.67 & 0.98 \\
39957 & 107960 & 2001 & 55.70 & -0.12 & 4084.00 & 36396.29 & 1.36 & 0.65 & 0.89 \\
1888 & 100247 & 2001 & 868.90 & -0.24 & 73973.00 & 867722.47 & 1.17 & 1.00 & 1.17 \\
24536 & 103339 & 2001 & 1355.90 & -0.14 & 135569.00 & 1344660.93 & 1.00 & 0.99 & 0.99 \\
64655 & 500617 & 2001 & 4845.50 & -0.01 & 396736.00 & 4737286.13 & 1.22 & 0.98 & 1.19 \\
39932 & 107958 & 2001 & 13.40 & 0.11 & 1226.00 & 12231.35 & 1.09 & 0.91 & 1.00 \\
24504 & 103329 & 2001 & 646.50 & 0.17 & 65012.00 & 620108.85 & 0.99 & 0.96 & 0.95 \\
12391 & 101538 & 2001 & 143.20 & 0.01 & 14340.00 & 139049.73 & 1.00 & 0.97 & 0.97 \\
8030 & 101071 & 2001 & 6546.70 & -0.13 & 555088.00 & 4714843.08 & 1.18 & 0.72 & 0.85 \\
25924 & 103529 & 2001 & 6924.60 & -0.11 & 850100.00 & 7677413.11 & 0.81 & 1.11 & 0.90 \\
38906 & 107352 & 2001 & 197.90 & -0.01 & 16933.00 & 161288.02 & 1.17 & 0.81 & 0.95 \\
38882 & 107350 & 2001 & 2436.80 & -0.03 & 195730.00 & 1677050.84 & 1.24 & 0.69 & 0.86 \\
597 & 100079 & 2001 & 1947.90 & -0.34 & 164562.00 & 1878782.93 & 1.18 & 0.96 & 1.14 \\
37514 & 106944 & 2001 & 18.10 & -0.11 & 1825.00 & 15707.38 & 0.99 & 0.87 & 0.86 \\
8398 & 101085 & 2001 & 319.50 & -0.22 & 39607.00 & 337150.82 & 0.81 & 1.06 & 0.85 \\
37523 & 106948 & 2001 & 156.40 & 0.04 & 15854.00 & 142503.55 & 0.99 & 0.91 & 0.90 \\
27431 & 105278 & 2001 & 1137.80 & -0.25 & 113578.00 & 1064098.37 & 1.00 & 0.94 & 0.94 \\
37535 & 106961 & 2001 & 107.80 & -0.07 & 10784.00 & 100821.13 & 1.00 & 0.94 & 0.93 \\
37542 & 106962 & 2001 & 6.40 & -0.11 & 643.00 & 6377.58 & 1.00 & 1.00 & 0.99 \\
27402 & 105276 & 2001 & 1330.40 & 0.02 & 129302.00 & 1140342.27 & 1.03 & 0.86 & 0.88 \\
74647 & 601147 & 2001 & 109.20 & 0.22 & 12331.00 & 98935.18 & 0.89 & 0.91 & 0.80 \\
37550 & 106968 & 2001 & 69.10 & -0.29 & 6776.00 & 67714.30 & 1.02 & 0.98 & 1.00 \\
49773 & 240359 & 2001 & 2.50 & 0.01 & 255.00 & 2315.26 & 0.98 & 0.93 & 0.91 \\
37575 & 106969 & 2001 & 57.30 & -0.13 & 5606.00 & 56058.93 & 1.02 & 0.98 & 1.00 \\
37600 & 106972 & 2001 & 45.10 & -0.11 & 4515.00 & 44411.03 & 1.00 & 0.98 & 0.98 \\
27372 & 105275 & 2001 & 76.80 & -0.10 & 8723.00 & 80250.73 & 0.88 & 1.04 & 0.92 \\
53580 & 354336 & 2001 & 23.00 & -0.20 & 2309.00 & 22676.82 & 1.00 & 0.99 & 0.98 \\
27361 & 105271 & 2001 & 47.00 & -0.13 & 4696.00 & 44114.13 & 1.00 & 0.94 & 0.94 \\
27467 & 105280 & 2001 & 44.00 & 0.00 & 4360.00 & 43543.44 & 1.01 & 0.99 & 1.00 \\
49782 & 240360 & 2001 & 507.90 & -0.11 & 45035.00 & 438202.74 & 1.13 & 0.86 & 0.97 \\
27476 & 105281 & 2001 & 333.20 & 0.05 & 32848.00 & 297204.61 & 1.01 & 0.89 & 0.90 \\
37488 & 106934 & 2001 & 1232.50 & -0.35 & 125016.00 & 1090693.80 & 0.99 & 0.88 & 0.87 \\
37414 & 106869 & 2001 & 129.50 & 0.03 & 12822.00 & 115991.80 & 1.01 & 0.90 & 0.90 \\
53544 & 351713 & 2001 & 254.80 & -0.15 & 24236.00 & 214990.58 & 1.05 & 0.84 & 0.89 \\
37428 & 106914 & 2001 & 276.70 & -0.22 & 23109.00 & 225654.54 & 1.20 & 0.82 & 0.98 \\
11485 & 101422 & 2001 & 59.40 & -0.28 & 5907.00 & 58664.93 & 1.01 & 0.99 & 0.99 \\
37454 & 106919 & 2001 & 3.30 & 0.01 & 324.00 & 2928.79 & 1.02 & 0.89 & 0.90 \\
53584 & 354930 & 2001 & 39.30 & 0.16 & 5303.00 & 47076.16 & 0.74 & 1.20 & 0.89 \\
27543 & 105287 & 2001 & 162.60 & -0.56 & 16207.00 & 162055.77 & 1.00 & 1.00 & 1.00 \\
572 & 100076 & 2001 & 889.50 & -0.26 & 78584.00 & 892814.95 & 1.13 & 1.00 & 1.14 \\
27534 & 105286 & 2001 & 432.80 & -0.21 & 35251.00 & 423744.18 & 1.23 & 0.98 & 1.20 \\
53563 & 351891 & 2001 & 36.20 & -0.17 & 3773.00 & 37731.58 & 0.96 & 1.04 & 1.00 \\
11507 & 101425 & 2001 & 24.40 & -0.19 & 2145.00 & 23126.60 & 1.14 & 0.95 & 1.08 \\
37479 & 106931 & 2001 & 690.70 & -0.35 & 69192.00 & 682850.13 & 1.00 & 0.99 & 0.99 \\
53561 & 351750 & 2001 & 16.40 & -0.14 & 1405.00 & 15755.82 & 1.17 & 0.96 & 1.12 \\
27590 & 105295 & 2001 & 418.70 & -0.16 & 41854.00 & 405382.16 & 1.00 & 0.97 & 0.97 \\
11561 & 101430 & 2001 & 122.90 & -0.07 & 12267.00 & 122167.05 & 1.00 & 0.99 & 1.00 \\
658 & 100087 & 2001 & 5213.10 & -0.34 & 491952.00 & 5312172.15 & 1.06 & 1.02 & 1.08 \\
27202 & 105252 & 2001 & 15.70 & -0.13 & 3080.00 & 30876.06 & 0.51 & 1.97 & 1.00 \\
74629 & 601142 & 2001 & 253.60 & -0.37 & 24875.00 & 255646.13 & 1.02 & 1.01 & 1.03 \\
49724 & 240344 & 2001 & 465.20 & 0.01 & 49688.00 & 476222.96 & 0.94 & 1.02 & 0.96 \\
37745 & 107137 & 2001 & 60.10 & -0.12 & 5948.00 & 59473.55 & 1.01 & 0.99 & 1.00 \\
37750 & 107141 & 2001 & 853.70 & 0.31 & 85424.00 & 685011.78 & 1.00 & 0.80 & 0.80 \\
53637 & 355027 & 2001 & 395.70 & -0.30 & 62750.00 & 566978.25 & 0.63 & 1.43 & 0.90 \\
37775 & 107143 & 2001 & 4.70 & -0.02 & 475.00 & 4068.56 & 0.99 & 0.87 & 0.86 \\
37778 & 107144 & 2001 & 132.10 & -0.15 & 12385.00 & 128466.22 & 1.07 & 0.97 & 1.04 \\
37801 & 107145 & 2001 & 18.00 & 0.02 & 1794.00 & 17481.88 & 1.00 & 0.97 & 0.97 \\
27133 & 105246 & 2001 & 4626.00 & 0.26 & 417296.00 & 4145912.43 & 1.11 & 0.90 & 0.99 \\
11649 & 101455 & 2001 & 39829.60 & -0.13 & 3013572.00 & 27524942.46 & 1.32 & 0.69 & 0.91 \\
74615 & 601140 & 2001 & 15.30 & 0.01 & 1540.00 & 13995.11 & 0.99 & 0.91 & 0.91 \\
49717 & 240337 & 2001 & 4.30 & -0.42 & 441.00 & 3858.30 & 0.98 & 0.90 & 0.87 \\
27211 & 105253 & 2001 & 26.10 & -0.08 & 2732.00 & 25618.34 & 0.96 & 0.98 & 0.94 \\
27220 & 105256 & 2001 & 51.20 & 0.14 & 5157.00 & 48092.03 & 0.99 & 0.94 & 0.93 \\
37719 & 107135 & 2001 & 50.70 & 0.01 & 5105.00 & 50077.65 & 0.99 & 0.99 & 0.98 \\
37712 & 107004 & 2001 & 165.10 & -0.11 & 16480.00 & 164404.75 & 1.00 & 1.00 & 1.00 \\
27339 & 105269 & 2001 & 695.40 & 0.17 & 72044.00 & 651328.41 & 0.97 & 0.94 & 0.90 \\
49762 & 240358 & 2001 & 20.50 & -0.21 & 2050.00 & 20045.54 & 1.00 & 0.98 & 0.98 \\
37641 & 106983 & 2001 & 153.60 & -0.28 & 18440.00 & 169734.97 & 0.83 & 1.11 & 0.92 \\
49759 & 240356 & 2001 & 5.00 & -0.08 & 496.00 & 4956.73 & 1.01 & 0.99 & 1.00 \\
27314 & 105268 & 2001 & 735.20 & -0.04 & 73490.00 & 714462.24 & 1.00 & 0.97 & 0.97 \\
49756 & 240355 & 2001 & 4.00 & -0.00 & 395.00 & 3825.45 & 1.01 & 0.96 & 0.97 \\
49753 & 240354 & 2001 & 26.80 & -0.10 & 2599.00 & 25486.10 & 1.03 & 0.95 & 0.98 \\
37647 & 106984 & 2001 & 13.70 & 0.07 & 1138.00 & 11395.91 & 1.20 & 0.83 & 1.00 \\
37620 & 106978 & 2001 & 232.50 & -0.20 & 19916.00 & 238487.28 & 1.17 & 1.03 & 1.20 \\
37654 & 106985 & 2001 & 53.30 & -0.13 & 5389.00 & 50933.47 & 0.99 & 0.96 & 0.95 \\
37660 & 106992 & 2001 & 248.30 & -0.06 & 23570.00 & 210184.32 & 1.05 & 0.85 & 0.89 \\
27276 & 105260 & 2001 & 220.60 & 0.08 & 22190.00 & 190093.86 & 0.99 & 0.86 & 0.86 \\
632 & 100085 & 2001 & 18697.10 & -0.13 & 1799324.00 & 18505169.65 & 1.04 & 0.99 & 1.03 \\
8360 & 101084 & 2001 & 2714.80 & -0.32 & 260230.00 & 2636013.94 & 1.04 & 0.97 & 1.01 \\
37671 & 106993 & 2001 & 78.10 & 0.25 & 5595.00 & 52499.89 & 1.40 & 0.67 & 0.94 \\
37682 & 106995 & 2001 & 1371.80 & -0.04 & 127509.00 & 1240599.57 & 1.08 & 0.90 & 0.97 \\
11601 & 101431 & 2001 & 340.90 & -0.06 & 34114.00 & 333170.55 & 1.00 & 0.98 & 0.98 \\
53607 & 354931 & 2001 & 1.30 & 0.13 & 118.00 & 1125.49 & 1.10 & 0.87 & 0.95 \\
27600 & 105299 & 2001 & 3.00 & -0.03 & 215.00 & 2584.17 & 1.40 & 0.86 & 1.20 \\
27607 & 105303 & 2001 & 167.00 & 0.01 & 18219.00 & 147765.30 & 0.92 & 0.88 & 0.81 \\
37388 & 106747 & 2001 & 26.30 & 0.03 & 2671.00 & 26714.28 & 0.98 & 1.02 & 1.00 \\
489 & 100071 & 2001 & 5467.60 & -0.23 & 500846.00 & 5607461.06 & 1.09 & 1.03 & 1.12 \\
37069 & 106664 & 2001 & 20.80 & -0.21 & 1891.00 & 20406.08 & 1.10 & 0.98 & 1.08 \\
37074 & 106666 & 2001 & 23.20 & -0.15 & 2170.00 & 23120.85 & 1.07 & 1.00 & 1.07 \\
37078 & 106667 & 2001 & 140.30 & 0.18 & 10231.00 & 92175.81 & 1.37 & 0.66 & 0.90 \\
27984 & 105364 & 2001 & 159.60 & -0.13 & 15968.00 & 154449.15 & 1.00 & 0.97 & 0.97 \\
37087 & 106675 & 2001 & 41.00 & -0.08 & 3808.00 & 38509.61 & 1.08 & 0.94 & 1.01 \\
53489 & 351048 & 2001 & 66.90 & -0.08 & 8437.00 & 64141.74 & 0.79 & 0.96 & 0.76 \\
74718 & 601156 & 2001 & 103.00 & -0.00 & 10284.00 & 102829.56 & 1.00 & 1.00 & 1.00 \\
27933 & 105353 & 2001 & 60.10 & 0.06 & 5475.00 & 60311.65 & 1.10 & 1.00 & 1.10 \\
8463 & 101087 & 2001 & 462.30 & -0.09 & 43949.00 & 415172.15 & 1.05 & 0.90 & 0.94 \\
27920 & 105348 & 2001 & 11.00 & -0.26 & 1074.00 & 10742.94 & 1.02 & 0.98 & 1.00 \\
53513 & 351459 & 2001 & 749.90 & -0.11 & 81696.00 & 656634.54 & 0.92 & 0.88 & 0.80 \\
27910 & 105346 & 2001 & 1342.80 & 0.03 & 129242.00 & 1264901.94 & 1.04 & 0.94 & 0.98 \\
27901 & 105343 & 2001 & 59.50 & -0.10 & 5560.00 & 61394.14 & 1.07 & 1.03 & 1.10 \\
74711 & 601155 & 2001 & 27.80 & 0.06 & 2763.00 & 27346.87 & 1.01 & 0.98 & 0.99 \\
11317 & 101393 & 2001 & 683.10 & -0.09 & 61370.00 & 659760.12 & 1.11 & 0.97 & 1.08 \\
28016 & 105369 & 2001 & 241.10 & -0.20 & 22704.00 & 247583.75 & 1.06 & 1.03 & 1.09 \\
37061 & 106655 & 2001 & 131.50 & -0.48 & 9683.00 & 127543.17 & 1.36 & 0.97 & 1.32 \\
37035 & 106654 & 2001 & 412.80 & -0.12 & 42978.00 & 397199.46 & 0.96 & 0.96 & 0.92 \\
53461 & 350572 & 2001 & 60.10 & -0.18 & 5029.00 & 57656.56 & 1.20 & 0.96 & 1.15 \\
36905 & 106627 & 2001 & 488.30 & -0.20 & 41639.00 & 447591.90 & 1.17 & 0.92 & 1.07 \\
28139 & 105384 & 2001 & 54.40 & -0.11 & 5188.00 & 53332.44 & 1.05 & 0.98 & 1.03 \\
11272 & 101381 & 2001 & 48.10 & -0.32 & 4797.00 & 47972.58 & 1.00 & 1.00 & 1.00 \\
28125 & 105383 & 2001 & 165.50 & -0.11 & 16541.00 & 147612.75 & 1.00 & 0.89 & 0.89 \\
36923 & 106640 & 2001 & 92.20 & -0.16 & 9526.00 & 86681.13 & 0.97 & 0.94 & 0.91 \\
8495 & 101088 & 2001 & 5106.10 & -0.15 & 511340.00 & 4789671.75 & 1.00 & 0.94 & 0.94 \\
36935 & 106642 & 2001 & 523.40 & 0.15 & 47077.00 & 470660.60 & 1.11 & 0.90 & 1.00 \\
53537 & 351589 & 2001 & 9.90 & -0.13 & 1573.00 & 15701.16 & 0.63 & 1.59 & 1.00 \\
28096 & 105382 & 2001 & 202.00 & -0.25 & 20197.00 & 182524.79 & 1.00 & 0.90 & 0.90 \\
36953 & 106643 & 2001 & 241.40 & -0.15 & 24140.00 & 219641.79 & 1.00 & 0.91 & 0.91 \\
28077 & 105379 & 2001 & 391.40 & -0.09 & 37827.00 & 373666.90 & 1.03 & 0.95 & 0.99 \\
36979 & 106644 & 2001 & 2.20 & -0.52 & 268.00 & 2339.23 & 0.82 & 1.06 & 0.87 \\
37016 & 106650 & 2001 & 16.00 & 0.09 & 1586.00 & 15516.56 & 1.01 & 0.97 & 0.98 \\
28045 & 105370 & 2001 & 101.40 & -0.26 & 9393.00 & 105286.14 & 1.08 & 1.04 & 1.12 \\
37031 & 106652 & 2001 & 15.60 & 0.00 & 1557.00 & 14371.53 & 1.00 & 0.92 & 0.92 \\
11285 & 101390 & 2001 & 3969.50 & -0.09 & 396986.00 & 3878997.29 & 1.00 & 0.98 & 0.98 \\
11373 & 101399 & 2001 & 132.70 & -0.13 & 12374.00 & 129012.79 & 1.07 & 0.97 & 1.04 \\
37275 & 106725 & 2001 & 13.10 & -0.29 & 1297.00 & 12972.02 & 1.01 & 0.99 & 1.00 \\
8430 & 101086 & 2001 & 788.20 & -0.36 & 90767.00 & 856095.90 & 0.87 & 1.09 & 0.94 \\
37287 & 106726 & 2001 & 2033.60 & -0.13 & 203167.00 & 1991711.12 & 1.00 & 0.98 & 0.98 \\
27711 & 105317 & 2001 & 1472.10 & -0.03 & 147207.00 & 1362765.22 & 1.00 & 0.93 & 0.93 \\
11435 & 101402 & 2001 & 12.90 & 0.15 & 1020.00 & 9392.03 & 1.26 & 0.73 & 0.92 \\
27699 & 105311 & 2001 & 213.80 & -0.18 & 21447.00 & 207945.98 & 1.00 & 0.97 & 0.97 \\
27691 & 105310 & 2001 & 181.10 & -0.41 & 16480.00 & 182616.89 & 1.10 & 1.01 & 1.11 \\
27665 & 105309 & 2001 & 672.90 & 0.01 & 56416.00 & 571100.35 & 1.19 & 0.85 & 1.01 \\
11458 & 101414 & 2001 & 14.70 & -0.09 & 1424.00 & 14459.83 & 1.03 & 0.98 & 1.02 \\
37339 & 106730 & 2001 & 23.00 & 0.13 & 2306.00 & 22341.36 & 1.00 & 0.97 & 0.97 \\
37369 & 106740 & 2001 & 227.70 & -0.18 & 20402.00 & 237710.37 & 1.12 & 1.04 & 1.17 \\
27636 & 105306 & 2001 & 220.30 & -0.45 & 17789.00 & 221785.58 & 1.24 & 1.01 & 1.25 \\
546 & 100075 & 2001 & 3115.20 & -0.26 & 280587.00 & 3186190.05 & 1.11 & 1.02 & 1.14 \\
37377 & 106742 & 2001 & 115.80 & -0.05 & 10162.00 & 114029.83 & 1.14 & 0.98 & 1.12 \\
37314 & 106729 & 2001 & 897.60 & -0.14 & 90480.00 & 886035.21 & 0.99 & 0.99 & 0.98 \\
686 & 100090 & 2001 & 908.90 & -0.13 & 90886.00 & 860090.69 & 1.00 & 0.95 & 0.95 \\
37157 & 106701 & 2001 & 65.70 & -0.11 & 6549.00 & 65302.64 & 1.00 & 0.99 & 1.00 \\
27862 & 105335 & 2001 & 246.70 & -0.11 & 23406.00 & 236717.03 & 1.05 & 0.96 & 1.01 \\
74686 & 601151 & 2001 & 35.10 & 0.15 & 2839.00 & 30933.02 & 1.24 & 0.88 & 1.09 \\
27854 & 105333 & 2001 & 10.90 & -0.02 & 1160.00 & 10949.70 & 0.94 & 1.00 & 0.94 \\
49949 & 240377 & 2001 & 15.10 & 0.05 & 1278.00 & 14233.50 & 1.18 & 0.94 & 1.11 \\
37164 & 106706 & 2001 & 10.90 & 0.03 & 1078.00 & 10784.19 & 1.01 & 0.99 & 1.00 \\
37178 & 106707 & 2001 & 28.30 & -0.02 & 2903.00 & 26250.27 & 0.97 & 0.93 & 0.90 \\
27754 & 105321 & 2001 & 363.90 & -0.10 & 29950.00 & 339331.67 & 1.22 & 0.93 & 1.13 \\
27825 & 105332 & 2001 & 106.10 & -0.27 & 9683.00 & 107827.16 & 1.10 & 1.02 & 1.11 \\
37204 & 106708 & 2001 & 392.80 & -0.14 & 39094.00 & 387137.54 & 1.00 & 0.99 & 0.99 \\
37217 & 106710 & 2001 & 76.20 & -0.05 & 7626.00 & 77590.13 & 1.00 & 1.02 & 1.02 \\
27796 & 105331 & 2001 & 3.80 & 0.07 & 383.00 & 3635.51 & 0.99 & 0.96 & 0.95 \\
11401 & 101400 & 2001 & 274.90 & 0.11 & 24728.00 & 262383.51 & 1.11 & 0.95 & 1.06 \\
49923 & 240375 & 2001 & 12.50 & 0.15 & 1039.00 & 11444.69 & 1.20 & 0.92 & 1.10 \\
525 & 100072 & 2001 & 11998.50 & -0.24 & 1028595.00 & 12074705.34 & 1.17 & 1.01 & 1.17 \\
27772 & 105322 & 2001 & 32.70 & -0.06 & 3139.00 & 31387.39 & 1.04 & 0.96 & 1.00 \\
49926 & 240376 & 2001 & 5.50 & -0.07 & 351.00 & 3187.92 & 1.57 & 0.58 & 0.91 \\
37826 & 107147 & 2001 & 60.80 & -0.44 & 5714.00 & 57720.85 & 1.06 & 0.95 & 1.01 \\
38530 & 107274 & 2001 & 5.70 & 0.01 & 562.00 & 5938.28 & 1.01 & 1.04 & 1.06 \\
38541 & 107281 & 2001 & 14.60 & -0.21 & 1455.00 & 14245.12 & 1.00 & 0.98 & 0.98 \\
49494 & 240305 & 2001 & 70.40 & -0.23 & 6958.00 & 69571.91 & 1.01 & 0.99 & 1.00 \\
38570 & 107285 & 2001 & 5.20 & 0.04 & 562.00 & 5345.01 & 0.93 & 1.03 & 0.95 \\
38574 & 107290 & 2001 & 546.40 & -0.12 & 51487.00 & 489044.72 & 1.06 & 0.90 & 0.95 \\
26293 & 103564 & 2001 & 674.40 & -0.05 & 58293.00 & 669496.32 & 1.16 & 0.99 & 1.15 \\
11876 & 101464 & 2001 & 411.30 & 0.20 & 36964.00 & 348895.66 & 1.11 & 0.85 & 0.94 \\
936 & 100112 & 2001 & 9317.50 & -0.02 & 781460.00 & 8350213.54 & 1.19 & 0.90 & 1.07 \\
38588 & 107294 & 2001 & 165.20 & -0.10 & 15791.00 & 128311.01 & 1.05 & 0.78 & 0.81 \\
38621 & 107300 & 2001 & 75.40 & -0.07 & 6268.00 & 69668.32 & 1.20 & 0.92 & 1.11 \\
26246 & 103547 & 2001 & 10028.30 & -0.32 & 987411.00 & 10129844.57 & 1.02 & 1.01 & 1.03 \\
49486 & 240304 & 2001 & 223.50 & 0.01 & 21182.00 & 177613.89 & 1.06 & 0.79 & 0.84 \\
894 & 100101 & 2001 & 937.70 & -0.18 & 93771.00 & 885075.18 & 1.00 & 0.94 & 0.94 \\
26376 & 103572 & 2001 & 33.60 & -0.12 & 3439.00 & 32180.84 & 0.98 & 0.96 & 0.94 \\
11845 & 101463 & 2001 & 942.40 & -0.12 & 96254.00 & 975824.87 & 0.98 & 1.04 & 1.01 \\
38507 & 107266 & 2001 & 239.00 & -0.05 & 25957.00 & 255414.17 & 0.92 & 1.07 & 0.98 \\
38368 & 107246 & 2001 & 236.00 & -0.17 & 20117.00 & 204617.14 & 1.17 & 0.87 & 1.02 \\
11813 & 101462 & 2001 & 488.10 & -0.10 & 49859.00 & 504239.04 & 0.98 & 1.03 & 1.01 \\
73364 & 600006 & 2001 & 116.50 & 0.08 & 9289.00 & 108083.57 & 1.25 & 0.93 & 1.16 \\
38383 & 107250 & 2001 & 10.50 & 0.01 & 1053.00 & 10204.66 & 1.00 & 0.97 & 0.97 \\
38390 & 107253 & 2001 & 113.80 & -0.05 & 11378.00 & 102321.62 & 1.00 & 0.90 & 0.90 \\
26485 & 103582 & 2001 & 19.50 & -0.23 & 1942.00 & 18744.47 & 1.00 & 0.96 & 0.97 \\
38631 & 107302 & 2001 & 434.70 & -0.09 & 40953.00 & 449525.44 & 1.06 & 1.03 & 1.10 \\
49555 & 240312 & 2001 & 19.40 & -0.05 & 1551.00 & 15675.03 & 1.25 & 0.81 & 1.01 \\
864 & 100099 & 2001 & 87.40 & -0.28 & 8743.00 & 82540.44 & 1.00 & 0.94 & 0.94 \\
38399 & 107257 & 2001 & 344.00 & -0.10 & 34392.00 & 329407.47 & 1.00 & 0.96 & 0.96 \\
38424 & 107258 & 2001 & 121.10 & -0.20 & 11247.00 & 120411.52 & 1.08 & 0.99 & 1.07 \\
38433 & 107259 & 2001 & 286.20 & 0.31 & 21532.00 & 209469.94 & 1.33 & 0.73 & 0.97 \\
26434 & 103580 & 2001 & 397.70 & -0.16 & 36453.00 & 392250.52 & 1.09 & 0.99 & 1.08 \\
38458 & 107260 & 2001 & 6.10 & 0.31 & 607.00 & 5742.70 & 1.00 & 0.94 & 0.95 \\
38483 & 107263 & 2001 & 1038.40 & -0.06 & 102924.00 & 1029981.83 & 1.01 & 0.99 & 1.00 \\
26402 & 103579 & 2001 & 540.20 & -0.02 & 43023.00 & 475486.96 & 1.26 & 0.88 & 1.11 \\
53678 & 355987 & 2001 & 151.20 & -0.32 & 13041.00 & 132308.37 & 1.16 & 0.88 & 1.01 \\
38656 & 107303 & 2001 & 62.30 & -0.19 & 6578.00 & 60037.53 & 0.95 & 0.96 & 0.91 \\
38824 & 107331 & 2001 & 4.80 & 0.07 & 496.00 & 4625.90 & 0.97 & 0.96 & 0.93 \\
26030 & 103535 & 2001 & 2577.80 & -0.15 & 284397.00 & 2593571.60 & 0.91 & 1.01 & 0.91 \\
49433 & 240297 & 2001 & 705.60 & -0.03 & 67739.00 & 685934.63 & 1.04 & 0.97 & 1.01 \\
53713 & 356500 & 2001 & 376.80 & -0.23 & 36249.00 & 362488.73 & 1.04 & 0.96 & 1.00 \\
1024 & 100127 & 2001 & 6079.10 & -0.26 & 463062.00 & 5808708.41 & 1.31 & 0.96 & 1.25 \\
38858 & 107337 & 2001 & 26.00 & -0.11 & 2604.00 & 25454.73 & 1.00 & 0.98 & 0.98 \\
38866 & 107338 & 2001 & 36.00 & -0.15 & 3615.00 & 34371.79 & 1.00 & 0.95 & 0.95 \\
25962 & 103531 & 2001 & 1336.40 & -0.03 & 138930.00 & 1209029.90 & 0.96 & 0.90 & 0.87 \\
53770 & 357053 & 2001 & 63.60 & -0.02 & 6382.00 & 61886.96 & 1.00 & 0.97 & 0.97 \\
38876 & 107339 & 2001 & 79.70 & -0.38 & 7882.00 & 72302.02 & 1.01 & 0.91 & 0.92 \\
49442 & 240300 & 2001 & 96.90 & -0.32 & 7417.00 & 93306.98 & 1.31 & 0.96 & 1.26 \\
26060 & 103536 & 2001 & 2269.50 & -0.12 & 242217.00 & 2259681.29 & 0.94 & 1.00 & 0.93 \\
38820 & 107329 & 2001 & 46.30 & -0.21 & 4686.00 & 45462.79 & 0.99 & 0.98 & 0.97 \\
38808 & 107328 & 2001 & 29.90 & -0.05 & 2929.00 & 29024.86 & 1.02 & 0.97 & 0.99 \\
38664 & 107306 & 2001 & 342.10 & -0.01 & 34906.00 & 333965.90 & 0.98 & 0.98 & 0.96 \\
38687 & 107308 & 2001 & 1359.10 & -0.05 & 130725.00 & 1287444.17 & 1.04 & 0.95 & 0.98 \\
26175 & 103545 & 2001 & 22428.60 & -0.03 & 1806605.00 & 16525064.60 & 1.24 & 0.74 & 0.91 \\
49471 & 240302 & 2001 & 7.50 & -0.10 & 603.00 & 5334.57 & 1.24 & 0.71 & 0.88 \\
980 & 100113 & 2001 & 949.40 & -0.18 & 74769.00 & 873989.74 & 1.27 & 0.92 & 1.17 \\
38712 & 107309 & 2001 & 143.50 & -0.12 & 14582.00 & 145495.90 & 0.98 & 1.01 & 1.00 \\
38720 & 107310 & 2001 & 58.80 & -0.13 & 5207.00 & 54229.18 & 1.13 & 0.92 & 1.04 \\
38724 & 107316 & 2001 & 618.10 & -0.01 & 62002.00 & 611926.40 & 1.00 & 0.99 & 0.99 \\
26215 & 103546 & 2001 & 17729.80 & 0.06 & 2662147.00 & 22416168.73 & 0.67 & 1.26 & 0.84 \\
26141 & 103544 & 2001 & 25088.40 & -0.11 & 2420502.00 & 22149374.13 & 1.04 & 0.88 & 0.92 \\
38748 & 107319 & 2001 & 81.90 & -0.44 & 8199.00 & 81987.46 & 1.00 & 1.00 & 1.00 \\
38752 & 107320 & 2001 & 133.90 & -0.44 & 13368.00 & 131533.57 & 1.00 & 0.98 & 0.98 \\
38756 & 107321 & 2001 & 15.70 & -0.06 & 1406.00 & 14327.06 & 1.12 & 0.91 & 1.02 \\
26107 & 103539 & 2001 & 705.80 & -0.19 & 78063.00 & 660336.90 & 0.90 & 0.94 & 0.85 \\
49447 & 240301 & 2001 & 6.30 & -0.21 & 876.00 & 7821.98 & 0.72 & 1.24 & 0.89 \\
38760 & 107322 & 2001 & 12.30 & 0.06 & 1007.00 & 11320.95 & 1.22 & 0.92 & 1.12 \\
38783 & 107323 & 2001 & 60.90 & 0.12 & 5273.00 & 55808.88 & 1.15 & 0.92 & 1.06 \\
11906 & 101465 & 2001 & 101.70 & -0.13 & 10162.00 & 96373.24 & 1.00 & 0.95 & 0.95 \\
38343 & 107244 & 2001 & 495.20 & -0.27 & 49205.00 & 474681.12 & 1.01 & 0.96 & 0.96 \\
73392 & 600012 & 2001 & 157.60 & -0.29 & 19881.00 & 195043.47 & 0.79 & 1.24 & 0.98 \\
37958 & 107174 & 2001 & 42.60 & 0.05 & 3410.00 & 40294.55 & 1.25 & 0.95 & 1.18 \\
11715 & 101457 & 2001 & 397.90 & -0.14 & 45466.00 & 420989.71 & 0.88 & 1.06 & 0.93 \\
37963 & 107175 & 2001 & 1476.60 & -0.15 & 131408.00 & 1378538.01 & 1.12 & 0.93 & 1.05 \\
37975 & 107178 & 2001 & 9.80 & 0.06 & 785.00 & 9169.39 & 1.25 & 0.94 & 1.17 \\
26928 & 103628 & 2001 & 1048.00 & -0.05 & 104801.00 & 1008797.44 & 1.00 & 0.96 & 0.96 \\
26915 & 103621 & 2001 & 170.70 & -0.17 & 17485.00 & 174846.55 & 0.98 & 1.02 & 1.00 \\
26893 & 103620 & 2001 & 223.10 & -0.07 & 22500.00 & 213033.57 & 0.99 & 0.95 & 0.95 \\
774 & 100096 & 2001 & 106.70 & -0.12 & 10674.00 & 101551.32 & 1.00 & 0.95 & 0.95 \\
37985 & 107179 & 2001 & 829.10 & -0.13 & 83386.00 & 804418.71 & 0.99 & 0.97 & 0.96 \\
49649 & 240328 & 2001 & 4.20 & -0.24 & 459.00 & 3756.99 & 0.92 & 0.89 & 0.82 \\
37997 & 107181 & 2001 & 72.70 & 0.02 & 7275.00 & 69230.18 & 1.00 & 0.95 & 0.95 \\
26861 & 103614 & 2001 & 291.00 & -0.10 & 27558.00 & 275580.96 & 1.06 & 0.95 & 1.00 \\
38007 & 107185 & 2001 & 22.80 & -0.23 & 2092.00 & 23938.70 & 1.09 & 1.05 & 1.14 \\
26845 & 103609 & 2001 & 35.60 & 0.10 & 3889.00 & 35012.67 & 0.92 & 0.98 & 0.90 \\
26963 & 103638 & 2001 & 22.40 & -0.10 & 2226.00 & 21969.49 & 1.01 & 0.98 & 0.99 \\
744 & 100093 & 2001 & 208.10 & -0.32 & 20819.00 & 203097.56 & 1.00 & 0.98 & 0.98 \\
53643 & 355536 & 2001 & 18.60 & 0.02 & 1683.00 & 18228.39 & 1.11 & 0.98 & 1.08 \\
37864 & 107152 & 2001 & 73.80 & -0.11 & 6932.00 & 62167.68 & 1.06 & 0.84 & 0.90 \\
27070 & 103647 & 2001 & 18.30 & -0.13 & 2405.00 & 22153.14 & 0.76 & 1.21 & 0.92 \\
710 & 100092 & 2001 & 720.20 & -0.07 & 72022.00 & 616268.92 & 1.00 & 0.86 & 0.86 \\
37881 & 107153 & 2001 & 20.20 & -0.32 & 1854.00 & 19598.80 & 1.09 & 0.97 & 1.06 \\
74590 & 601139 & 2001 & 4035.40 & -0.02 & 405350.00 & 3522865.91 & 1.00 & 0.87 & 0.87 \\
37902 & 107160 & 2001 & 2921.60 & -0.00 & 245093.00 & 2351155.75 & 1.19 & 0.80 & 0.96 \\
8319 & 101082 & 2001 & 2511.20 & -0.15 & 255769.00 & 2404939.97 & 0.98 & 0.96 & 0.94 \\
49633 & 240327 & 2001 & 125.40 & -0.18 & 10515.00 & 124022.82 & 1.19 & 0.99 & 1.18 \\
37927 & 107162 & 2001 & 8.00 & -0.12 & 838.00 & 7799.27 & 0.95 & 0.97 & 0.93 \\
11682 & 101456 & 2001 & 51.30 & 0.00 & 5576.00 & 47737.13 & 0.92 & 0.93 & 0.86 \\
37931 & 107167 & 2001 & 10.50 & -0.26 & 1084.00 & 10329.48 & 0.97 & 0.98 & 0.95 \\
37936 & 107171 & 2001 & 41.40 & -0.13 & 4635.00 & 42923.54 & 0.89 & 1.04 & 0.93 \\
49685 & 240332 & 2001 & 84.10 & -0.15 & 9935.00 & 71626.12 & 0.85 & 0.85 & 0.72 \\
37940 & 107173 & 2001 & 182.00 & 0.09 & 18143.00 & 173083.04 & 1.00 & 0.95 & 0.95 \\
26988 & 103643 & 2001 & 93.80 & -0.17 & 11461.00 & 106643.35 & 0.82 & 1.14 & 0.93 \\
74562 & 601136 & 2001 & 40.50 & -0.11 & 4699.00 & 47110.40 & 0.86 & 1.16 & 1.00 \\
49681 & 240331 & 2001 & 7.30 & -0.15 & 729.00 & 7243.27 & 1.00 & 0.99 & 0.99 \\
27014 & 103644 & 2001 & 45.90 & -0.10 & 5387.00 & 50043.34 & 0.85 & 1.09 & 0.93 \\
38013 & 107187 & 2001 & 54.80 & -0.31 & 5496.00 & 49148.80 & 1.00 & 0.90 & 0.89 \\
38017 & 107192 & 2001 & 562.60 & -0.03 & 57663.00 & 571150.43 & 0.98 & 1.02 & 0.99 \\
26822 & 103608 & 2001 & 45.50 & -0.03 & 4548.00 & 43196.83 & 1.00 & 0.95 & 0.95 \\
8278 & 101081 & 2001 & 573.50 & -0.09 & 56901.00 & 515219.32 & 1.01 & 0.90 & 0.91 \\
26651 & 103595 & 2001 & 156.40 & -0.07 & 13495.00 & 151294.63 & 1.16 & 0.97 & 1.12 \\
11782 & 101461 & 2001 & 2920.10 & -0.15 & 291067.00 & 2902023.19 & 1.00 & 0.99 & 1.00 \\
38242 & 107226 & 2001 & 92.40 & 0.12 & 9364.00 & 89181.87 & 0.99 & 0.97 & 0.95 \\
38267 & 107227 & 2001 & 86.30 & -0.21 & 8614.00 & 85229.26 & 1.00 & 0.99 & 0.99 \\
26619 & 103593 & 2001 & 50086.00 & 0.01 & 4391456.00 & 44076580.24 & 1.14 & 0.88 & 1.00 \\
38211 & 107222 & 2001 & 99.30 & 0.07 & 9946.00 & 92743.67 & 1.00 & 0.93 & 0.93 \\
49583 & 240319 & 2001 & 116.10 & -0.35 & 11483.00 & 102479.31 & 1.01 & 0.88 & 0.89 \\
38281 & 107235 & 2001 & 100.70 & -0.11 & 10649.00 & 105578.60 & 0.95 & 1.05 & 0.99 \\
49575 & 240318 & 2001 & 145.80 & -0.50 & 10276.00 & 102684.46 & 1.42 & 0.70 & 1.00 \\
834 & 100098 & 2001 & 254.90 & -0.06 & 25489.00 & 235623.72 & 1.00 & 0.92 & 0.92 \\
38293 & 107242 & 2001 & 341.00 & 0.12 & 33942.00 & 322988.10 & 1.00 & 0.95 & 0.95 \\
38318 & 107243 & 2001 & 1263.20 & -0.11 & 134900.00 & 1123917.16 & 0.94 & 0.89 & 0.83 \\
26554 & 103591 & 2001 & 1243.80 & 0.29 & 98912.00 & 1070513.29 & 1.26 & 0.86 & 1.08 \\
49566 & 240314 & 2001 & 16.80 & -0.13 & 1666.00 & 16637.26 & 1.01 & 0.99 & 1.00 \\
38273 & 107234 & 2001 & 11.50 & -0.10 & 1315.00 & 11075.46 & 0.87 & 0.96 & 0.84 \\
38186 & 107215 & 2001 & 233.10 & 0.06 & 24159.00 & 220453.47 & 0.96 & 0.95 & 0.91 \\
26695 & 103600 & 2001 & 358.80 & -0.05 & 31738.00 & 340298.31 & 1.13 & 0.95 & 1.07 \\
38040 & 107196 & 2001 & 47.10 & 0.03 & 4711.00 & 44083.72 & 1.00 & 0.94 & 0.94 \\
26795 & 103607 & 2001 & 119.20 & -0.10 & 13402.00 & 112819.21 & 0.89 & 0.95 & 0.84 \\
11749 & 101460 & 2001 & 2464.70 & -0.02 & 235762.00 & 2327035.14 & 1.05 & 0.94 & 0.99 \\
73608 & 600482 & 2001 & 238.40 & -0.01 & 33138.00 & 202374.79 & 0.72 & 0.85 & 0.61 \\
73603 & 600473 & 2001 & 36.50 & -0.23 & 5519.00 & 47278.54 & 0.66 & 1.30 & 0.86 \\
38065 & 107198 & 2001 & 116.30 & 0.04 & 7539.00 & 94443.34 & 1.54 & 0.81 & 1.25 \\
38081 & 107199 & 2001 & 25.90 & -0.14 & 2292.00 & 25807.59 & 1.13 & 1.00 & 1.13 \\
26763 & 103606 & 2001 & 17.90 & -0.13 & 1788.00 & 16673.41 & 1.00 & 0.93 & 0.93 \\
804 & 100097 & 2001 & 204.70 & -0.17 & 20471.00 & 191493.01 & 1.00 & 0.94 & 0.94 \\
38090 & 107201 & 2001 & 306.10 & -0.07 & 31296.00 & 266098.86 & 0.98 & 0.87 & 0.85 \\
38115 & 107202 & 2001 & 18.40 & 0.24 & 1879.00 & 16740.58 & 0.98 & 0.91 & 0.89 \\
38131 & 107204 & 2001 & 407.90 & -0.94 & 46625.00 & 353238.64 & 0.87 & 0.87 & 0.76 \\
38154 & 107205 & 2001 & 726.60 & -0.29 & 91379.00 & 798405.42 & 0.80 & 1.10 & 0.87 \\
38157 & 107209 & 2001 & 167.80 & -0.13 & 16781.00 & 158166.79 & 1.00 & 0.94 & 0.94 \\
49625 & 240326 & 2001 & 46.50 & -0.41 & 5982.00 & 39316.86 & 0.78 & 0.85 & 0.66 \\
43626 & 109165 & 2001 & 23.30 & -0.34 & 1762.00 & 23019.88 & 1.32 & 0.99 & 1.31 \\
15046 & 101953 & 2001 & 352.80 & -0.15 & 43048.00 & 370879.91 & 0.82 & 1.05 & 0.86 \\
43628 & 109170 & 2001 & 57.40 & -0.01 & 4339.00 & 58599.11 & 1.32 & 1.02 & 1.35 \\
16309 & 102124 & 2001 & 1772.10 & -0.12 & 176697.00 & 1729509.51 & 1.00 & 0.98 & 0.98 \\
46480 & 200247 & 2001 & 2.90 & -0.17 & 293.00 & 2582.96 & 0.99 & 0.89 & 0.88 \\
44232 & 109277 & 2001 & 400.70 & 0.01 & 40140.00 & 388077.35 & 1.00 & 0.97 & 0.97 \\
44234 & 109278 & 2001 & 19.30 & 0.05 & 1839.00 & 16879.11 & 1.05 & 0.87 & 0.92 \\
46461 & 200246 & 2001 & 125.00 & -0.10 & 12490.00 & 122492.63 & 1.00 & 0.98 & 0.98 \\
44257 & 109279 & 2001 & 11.00 & 0.04 & 881.00 & 8739.05 & 1.25 & 0.79 & 0.99 \\
19024 & 102544 & 2001 & 707.20 & -0.19 & 75603.00 & 652802.48 & 0.94 & 0.92 & 0.86 \\
5825 & 100804 & 2001 & 3427.00 & -0.11 & 342370.00 & 3187351.17 & 1.00 & 0.93 & 0.93 \\
16363 & 102130 & 2001 & 430.20 & -0.29 & 42805.00 & 420731.34 & 1.01 & 0.98 & 0.98 \\
44268 & 109281 & 2001 & 54.10 & 0.06 & 4040.00 & 45733.09 & 1.34 & 0.85 & 1.13 \\
46428 & 200244 & 2001 & 9.70 & -0.54 & 755.00 & 9249.11 & 1.28 & 0.95 & 1.23 \\
46415 & 200239 & 2001 & 2.50 & 0.02 & 254.00 & 2276.05 & 0.98 & 0.91 & 0.90 \\
5785 & 100792 & 2001 & 401.80 & 0.14 & 38659.00 & 372994.13 & 1.04 & 0.93 & 0.96 \\
47676 & 217585 & 2001 & 281.70 & -0.12 & 28261.00 & 279381.99 & 1.00 & 0.99 & 0.99 \\
16443 & 102145 & 2001 & 164.60 & -0.08 & 15163.00 & 158190.87 & 1.09 & 0.96 & 1.04 \\
18960 & 102531 & 2001 & 24.00 & 0.01 & 2416.00 & 24254.64 & 0.99 & 1.01 & 1.00 \\
44357 & 109289 & 2001 & 90.50 & -0.12 & 9043.00 & 85443.75 & 1.00 & 0.94 & 0.94 \\
46394 & 200233 & 2001 & 4.50 & -0.15 & 454.00 & 3795.83 & 0.99 & 0.84 & 0.84 \\
46397 & 200236 & 2001 & 3.10 & -0.14 & 297.00 & 2876.50 & 1.04 & 0.93 & 0.97 \\
44334 & 109286 & 2001 & 98.70 & -0.14 & 8114.00 & 81574.21 & 1.22 & 0.83 & 1.01 \\
55097 & 400061 & 2001 & 181.90 & 0.28 & 16627.00 & 166252.64 & 1.09 & 0.91 & 1.00 \\
44326 & 109284 & 2001 & 7.90 & -0.08 & 876.00 & 7681.79 & 0.90 & 0.97 & 0.88 \\
47687 & 220681 & 2001 & 1031.70 & -0.10 & 136439.00 & 1135246.14 & 0.76 & 1.10 & 0.83 \\
4285 & 100600 & 2001 & 46.80 & 0.13 & 3765.00 & 39875.59 & 1.24 & 0.85 & 1.06 \\
7241 & 101015 & 2001 & 1136.50 & 0.12 & 85493.00 & 718399.50 & 1.33 & 0.63 & 0.84 \\
16406 & 102133 & 2001 & 47.30 & -0.17 & 4696.00 & 45742.37 & 1.01 & 0.97 & 0.97 \\
18992 & 102540 & 2001 & 15.90 & 0.34 & 1607.00 & 15104.75 & 0.99 & 0.95 & 0.94 \\
16398 & 102132 & 2001 & 43.90 & -0.26 & 4382.00 & 43252.79 & 1.00 & 0.99 & 0.99 \\
16415 & 102134 & 2001 & 79.30 & -0.11 & 7938.00 & 76952.27 & 1.00 & 0.97 & 0.97 \\
19048 & 102545 & 2001 & 426.10 & 0.15 & 32259.00 & 276893.92 & 1.32 & 0.65 & 0.86 \\
57242 & 400323 & 2001 & 363.70 & 0.08 & 44906.00 & 449196.07 & 0.81 & 1.24 & 1.00 \\
46507 & 200249 & 2001 & 51.40 & 0.08 & 4766.00 & 47659.31 & 1.08 & 0.93 & 1.00 \\
16202 & 102090 & 2001 & 6587.40 & -0.20 & 483081.00 & 4793648.36 & 1.36 & 0.73 & 0.99 \\
19108 & 102549 & 2001 & 244.30 & 0.10 & 24840.00 & 216089.69 & 0.98 & 0.88 & 0.87 \\
46516 & 200250 & 2001 & 92.70 & -0.12 & 7656.00 & 83813.33 & 1.21 & 0.90 & 1.09 \\
46522 & 200251 & 2001 & 10.40 & -0.16 & 1007.00 & 9906.55 & 1.03 & 0.95 & 0.98 \\
6770 & 100953 & 2001 & 32.30 & 0.25 & 2501.00 & 25705.83 & 1.29 & 0.80 & 1.03 \\
46545 & 200252 & 2001 & 61.80 & -0.08 & 5961.00 & 59610.89 & 1.04 & 0.96 & 1.00 \\
16171 & 102089 & 2001 & 215.00 & -0.05 & 20166.00 & 173134.77 & 1.07 & 0.81 & 0.86 \\
14742 & 101912 & 2001 & 4509.00 & 0.10 & 384204.00 & 3747392.53 & 1.17 & 0.83 & 0.98 \\
44178 & 109270 & 2001 & 84.10 & -0.11 & 7218.00 & 76307.72 & 1.17 & 0.91 & 1.06 \\
16152 & 102087 & 2001 & 876.10 & -0.03 & 80572.00 & 794527.60 & 1.09 & 0.91 & 0.99 \\
5943 & 100812 & 2001 & 306.90 & -0.12 & 30679.00 & 304621.93 & 1.00 & 0.99 & 0.99 \\
44183 & 109271 & 2001 & 134.60 & -0.24 & 11937.00 & 118039.25 & 1.13 & 0.88 & 0.99 \\
18947 & 102529 & 2001 & 28.20 & 0.18 & 2790.00 & 26165.39 & 1.01 & 0.93 & 0.94 \\
47644 & 216438 & 2001 & 637.50 & -0.13 & 58603.00 & 597083.96 & 1.09 & 0.94 & 1.02 \\
5904 & 100811 & 2001 & 1169.20 & -0.06 & 117694.00 & 1170627.03 & 0.99 & 1.00 & 0.99 \\
16274 & 102113 & 2001 & 165.70 & -0.31 & 16393.00 & 163927.56 & 1.01 & 0.99 & 1.00 \\
47661 & 216749 & 2001 & 47.00 & -0.11 & 4077.00 & 40193.64 & 1.15 & 0.86 & 0.99 \\
5872 & 100809 & 2001 & 1864.30 & -0.10 & 187781.00 & 1795027.72 & 0.99 & 0.96 & 0.96 \\
44189 & 109273 & 2001 & 9.10 & 0.30 & 837.00 & 8184.81 & 1.09 & 0.90 & 0.98 \\
13841 & 101769 & 2001 & 1621.20 & -0.23 & 154742.00 & 1642282.91 & 1.05 & 1.01 & 1.06 \\
59213 & 410445 & 2001 & 5.20 & 0.01 & 804.00 & 7784.26 & 0.65 & 1.50 & 0.97 \\
47653 & 216504 & 2001 & 39.70 & -0.43 & 5133.00 & 40245.89 & 0.77 & 1.01 & 0.78 \\
19080 & 102548 & 2001 & 964.50 & -0.20 & 104540.00 & 928678.22 & 0.92 & 0.96 & 0.89 \\
44209 & 109275 & 2001 & 57.40 & -0.11 & 6047.00 & 57468.72 & 0.95 & 1.00 & 0.95 \\
16243 & 102104 & 2001 & 351.30 & 0.18 & 34149.00 & 341415.06 & 1.03 & 0.97 & 1.00 \\
16258 & 102105 & 2001 & 145.30 & 0.01 & 14270.00 & 142720.95 & 1.02 & 0.98 & 1.00 \\
46553 & 200253 & 2001 & 3.80 & -0.01 & 369.00 & 3360.89 & 1.03 & 0.88 & 0.91 \\
46390 & 200230 & 2001 & 1.40 & -0.51 & 132.00 & 1196.68 & 1.06 & 0.85 & 0.91 \\
44360 & 109290 & 2001 & 322.90 & -0.11 & 32218.00 & 304477.41 & 1.00 & 0.94 & 0.95 \\
46302 & 200210 & 2001 & 6.50 & -0.18 & 650.00 & 6231.88 & 1.00 & 0.96 & 0.96 \\
14640 & 101904 & 2001 & 14.30 & -0.38 & 1432.00 & 13292.42 & 1.00 & 0.93 & 0.93 \\
16592 & 102157 & 2001 & 5.90 & 0.04 & 594.00 & 4945.03 & 0.99 & 0.84 & 0.83 \\
18850 & 102524 & 2001 & 2755.20 & -0.12 & 275162.00 & 2666984.00 & 1.00 & 0.97 & 0.97 \\
5711 & 100789 & 2001 & 5.00 & -0.43 & 672.00 & 5658.89 & 0.74 & 1.13 & 0.84 \\
59109 & 410433 & 2001 & 984.20 & 0.20 & 69044.00 & 822523.44 & 1.43 & 0.84 & 1.19 \\
14630 & 101903 & 2001 & 161.40 & -0.24 & 15936.00 & 157653.28 & 1.01 & 0.98 & 0.99 \\
16608 & 102160 & 2001 & 1.90 & -0.35 & 246.00 & 1928.04 & 0.77 & 1.01 & 0.78 \\
6836 & 100962 & 2001 & 4753.40 & 0.01 & 406579.00 & 4149661.20 & 1.17 & 0.87 & 1.02 \\
16616 & 102163 & 2001 & 218.10 & -0.20 & 25918.00 & 235089.06 & 0.84 & 1.08 & 0.91 \\
44469 & 109326 & 2001 & 4.40 & -0.06 & 390.00 & 4022.94 & 1.13 & 0.91 & 1.03 \\
44473 & 109327 & 2001 & 254.80 & -0.06 & 18803.00 & 179912.64 & 1.36 & 0.71 & 0.96 \\
16628 & 102166 & 2001 & 65.00 & -0.13 & 8690.00 & 74382.86 & 0.75 & 1.14 & 0.86 \\
44496 & 109330 & 2001 & 17.00 & -0.01 & 1712.00 & 15956.02 & 0.99 & 0.94 & 0.93 \\
44500 & 109332 & 2001 & 21.80 & 0.02 & 2227.00 & 21454.93 & 0.98 & 0.98 & 0.96 \\
14606 & 101902 & 2001 & 271.80 & -0.03 & 26700.00 & 229106.41 & 1.02 & 0.84 & 0.86 \\
55176 & 400066 & 2001 & 145.60 & -0.05 & 14464.00 & 143167.58 & 1.01 & 0.98 & 0.99 \\
4383 & 100614 & 2001 & 1286.50 & 0.02 & 128285.00 & 1151516.67 & 1.00 & 0.90 & 0.90 \\
16692 & 102178 & 2001 & 827.10 & -0.15 & 127840.00 & 1049080.92 & 0.65 & 1.27 & 0.82 \\
55577 & 400127 & 2001 & 10.80 & 0.05 & 794.00 & 9036.60 & 1.36 & 0.84 & 1.14 \\
46268 & 200205 & 2001 & 74.60 & -0.11 & 9639.00 & 63228.21 & 0.77 & 0.85 & 0.66 \\
4357 & 100611 & 2001 & 950.90 & -0.09 & 97218.00 & 860492.75 & 0.98 & 0.90 & 0.89 \\
48213 & 240051 & 2001 & 429.80 & 0.08 & 38024.00 & 371436.27 & 1.13 & 0.86 & 0.98 \\
46277 & 200207 & 2001 & 17.90 & -0.01 & 1659.00 & 16873.07 & 1.08 & 0.94 & 1.02 \\
16659 & 102175 & 2001 & 551.80 & -0.03 & 71657.00 & 652460.78 & 0.77 & 1.18 & 0.91 \\
5679 & 100785 & 2001 & 1584.70 & -0.12 & 172573.00 & 1625294.65 & 0.92 & 1.03 & 0.94 \\
16649 & 102173 & 2001 & 44.40 & -0.12 & 4271.00 & 42706.14 & 1.04 & 0.96 & 1.00 \\
18819 & 102523 & 2001 & 582.00 & 0.18 & 58470.00 & 524577.25 & 1.00 & 0.90 & 0.90 \\
55169 & 400065 & 2001 & 115.90 & -0.18 & 11594.00 & 115874.59 & 1.00 & 1.00 & 1.00 \\
44442 & 109324 & 2001 & 281.70 & -0.11 & 35066.00 & 272036.79 & 0.80 & 0.97 & 0.78 \\
44440 & 109321 & 2001 & 406.00 & -0.19 & 37359.00 & 345773.75 & 1.09 & 0.85 & 0.93 \\
18912 & 102527 & 2001 & 157.30 & -0.10 & 16287.00 & 162573.25 & 0.97 & 1.03 & 1.00 \\
16496 & 102151 & 2001 & 8.50 & -0.43 & 849.00 & 8151.68 & 1.00 & 0.96 & 0.96 \\
5753 & 100791 & 2001 & 4191.50 & 0.01 & 406898.00 & 3588031.18 & 1.03 & 0.86 & 0.88 \\
46350 & 200225 & 2001 & 4.90 & -0.17 & 431.00 & 4683.54 & 1.14 & 0.96 & 1.09 \\
46355 & 200226 & 2001 & 8.40 & -0.12 & 846.00 & 8262.63 & 0.99 & 0.98 & 0.98 \\
44367 & 109295 & 2001 & 68.60 & -0.20 & 6974.00 & 66170.35 & 0.98 & 0.96 & 0.95 \\
4305 & 100603 & 2001 & 1388.00 & 0.06 & 124400.00 & 1255415.74 & 1.12 & 0.90 & 1.01 \\
44366 & 109292 & 2001 & 28.40 & -0.01 & 2828.00 & 28261.87 & 1.00 & 1.00 & 1.00 \\
46380 & 200228 & 2001 & 1.90 & 0.05 & 191.00 & 1685.88 & 0.99 & 0.89 & 0.88 \\
55121 & 400062 & 2001 & 33.30 & 0.21 & 2605.00 & 28772.10 & 1.28 & 0.86 & 1.10 \\
18928 & 102528 & 2001 & 86.40 & -0.07 & 8585.00 & 83821.47 & 1.01 & 0.97 & 0.98 \\
44363 & 109291 & 2001 & 25.00 & 0.05 & 2499.00 & 24835.36 & 1.00 & 0.99 & 0.99 \\
14674 & 101908 & 2001 & 51.30 & 0.12 & 5137.00 & 50538.60 & 1.00 & 0.99 & 0.98 \\
46357 & 200227 & 2001 & 16.00 & 0.02 & 1603.00 & 15701.20 & 1.00 & 0.98 & 0.98 \\
13868 & 101781 & 2001 & 481.30 & -0.31 & 37996.00 & 461669.84 & 1.27 & 0.96 & 1.22 \\
46342 & 200224 & 2001 & 36.00 & 0.09 & 3436.00 & 33535.90 & 1.05 & 0.93 & 0.98 \\
14660 & 101906 & 2001 & 14.60 & -0.16 & 1493.00 & 14485.93 & 0.98 & 0.99 & 0.97 \\
16562 & 102156 & 2001 & 18.70 & 0.31 & 1428.00 & 12807.88 & 1.31 & 0.68 & 0.90 \\
14650 & 101905 & 2001 & 10.40 & -0.08 & 1005.00 & 9670.46 & 1.03 & 0.93 & 0.96 \\
5729 & 100790 & 2001 & 215.30 & 0.01 & 21102.00 & 197727.70 & 1.02 & 0.92 & 0.94 \\
46322 & 200213 & 2001 & 11.00 & -0.45 & 878.00 & 9666.43 & 1.25 & 0.88 & 1.10 \\
6792 & 100954 & 2001 & 738.00 & -0.15 & 70264.00 & 760265.90 & 1.05 & 1.03 & 1.08 \\
18881 & 102525 & 2001 & 633.40 & -0.01 & 62902.00 & 583901.29 & 1.01 & 0.92 & 0.93 \\
44413 & 109301 & 2001 & 112.70 & -0.07 & 11516.00 & 111834.85 & 0.98 & 0.99 & 0.97 \\
44399 & 109300 & 2001 & 523.90 & -0.07 & 52322.00 & 505621.62 & 1.00 & 0.97 & 0.97 \\
46334 & 200223 & 2001 & 14.70 & -0.25 & 1336.00 & 13359.61 & 1.10 & 0.91 & 1.00 \\
13885 & 101785 & 2001 & 2360.40 & -0.35 & 316441.00 & 2212247.24 & 0.75 & 0.94 & 0.70 \\
59141 & 410439 & 2001 & 13.60 & -0.16 & 1344.00 & 11854.97 & 1.01 & 0.87 & 0.88 \\
16518 & 102152 & 2001 & 255.80 & -0.17 & 23311.00 & 254734.80 & 1.10 & 1.00 & 1.09 \\
16541 & 102154 & 2001 & 213.80 & 0.01 & 21042.00 & 196806.04 & 1.02 & 0.92 & 0.94 \\
46560 & 200254 & 2001 & 2.20 & 0.02 & 185.00 & 1658.42 & 1.19 & 0.75 & 0.90 \\
7279 & 101018 & 2001 & 21958.60 & -0.17 & 2302624.00 & 21144201.75 & 0.95 & 0.96 & 0.92 \\
19146 & 102551 & 2001 & 493.50 & 0.18 & 40980.00 & 393366.63 & 1.20 & 0.80 & 0.96 \\
19436 & 102601 & 2001 & 7212.50 & -0.11 & 721236.00 & 7039716.39 & 1.00 & 0.98 & 0.98 \\
15680 & 102013 & 2001 & 4987.40 & -0.25 & 498746.00 & 4589967.88 & 1.00 & 0.92 & 0.92 \\
4112 & 100552 & 2001 & 136.30 & -0.14 & 13548.00 & 134780.55 & 1.01 & 0.99 & 0.99 \\
15700 & 102015 & 2001 & 355.80 & -0.36 & 35577.00 & 336469.99 & 1.00 & 0.95 & 0.95 \\
6163 & 100827 & 2001 & 332.80 & 0.29 & 31340.00 & 306706.06 & 1.06 & 0.92 & 0.98 \\
47540 & 212658 & 2001 & 6756.10 & -0.01 & 666342.00 & 5953403.92 & 1.01 & 0.88 & 0.89 \\
6150 & 100825 & 2001 & 84.00 & -0.08 & 8391.00 & 79670.82 & 1.00 & 0.95 & 0.95 \\
47567 & 212809 & 2001 & 2.90 & -0.01 & 260.00 & 2453.46 & 1.12 & 0.85 & 0.94 \\
15725 & 102016 & 2001 & 14412.10 & -0.25 & 1247505.00 & 13158373.11 & 1.16 & 0.91 & 1.05 \\
43931 & 109232 & 2001 & 615.20 & -0.21 & 61579.00 & 591468.39 & 1.00 & 0.96 & 0.96 \\
43934 & 109233 & 2001 & 144.60 & -0.12 & 12977.00 & 138449.54 & 1.11 & 0.96 & 1.07 \\
19402 & 102600 & 2001 & 775.30 & -0.06 & 77540.00 & 749081.63 & 1.00 & 0.97 & 0.97 \\
46827 & 200304 & 2001 & 6.30 & 0.03 & 632.00 & 6187.96 & 1.00 & 0.98 & 0.98 \\
46820 & 200301 & 2001 & 1.50 & -0.25 & 184.00 & 1551.46 & 0.82 & 1.03 & 0.84 \\
48293 & 240060 & 2001 & 234.50 & -0.21 & 23441.00 & 201486.23 & 1.00 & 0.86 & 0.86 \\
46741 & 200294 & 2001 & 34.90 & -0.02 & 3349.00 & 33494.98 & 1.04 & 0.96 & 1.00 \\
43978 & 109250 & 2001 & 226.40 & -0.08 & 23885.00 & 210378.16 & 0.95 & 0.93 & 0.88 \\
46754 & 200295 & 2001 & 75.50 & -0.02 & 7161.00 & 70594.32 & 1.05 & 0.94 & 0.99 \\
4131 & 100559 & 2001 & 54.50 & -0.14 & 5448.00 & 51114.42 & 1.00 & 0.94 & 0.94 \\
15790 & 102018 & 2001 & 792.10 & -0.24 & 67022.00 & 756092.84 & 1.18 & 0.95 & 1.13 \\
43882 & 109228 & 2001 & 20.90 & -0.06 & 1837.00 & 20906.46 & 1.14 & 1.00 & 1.14 \\
19368 & 102599 & 2001 & 1406.70 & 0.05 & 140669.00 & 1351827.38 & 1.00 & 0.96 & 0.96 \\
43971 & 109249 & 2001 & 179.70 & -0.36 & 20579.00 & 149149.82 & 0.87 & 0.83 & 0.72 \\
43957 & 109238 & 2001 & 8.50 & 0.01 & 847.00 & 7512.93 & 1.00 & 0.88 & 0.89 \\
15759 & 102017 & 2001 & 10581.80 & -0.17 & 1009040.00 & 9950117.52 & 1.05 & 0.94 & 0.99 \\
14832 & 101916 & 2001 & 380.00 & 0.14 & 33077.00 & 363615.14 & 1.15 & 0.96 & 1.10 \\
43939 & 109237 & 2001 & 6.00 & 0.01 & 523.00 & 4709.28 & 1.15 & 0.78 & 0.90 \\
43937 & 109236 & 2001 & 4.50 & -0.22 & 445.00 & 4505.26 & 1.01 & 1.00 & 1.01 \\
46777 & 200297 & 2001 & 14.70 & -0.04 & 1330.00 & 13300.29 & 1.11 & 0.90 & 1.00 \\
6702 & 100913 & 2001 & 466.70 & 0.04 & 46495.00 & 387452.23 & 1.00 & 0.83 & 0.83 \\
43867 & 109226 & 2001 & 58.50 & -0.03 & 6140.00 & 64397.80 & 0.95 & 1.10 & 1.05 \\
46918 & 200316 & 2001 & 7.80 & -0.57 & 521.00 & 6364.47 & 1.50 & 0.82 & 1.22 \\
47509 & 212408 & 2001 & 2941.10 & -0.21 & 294075.00 & 2910018.80 & 1.00 & 0.99 & 0.99 \\
43786 & 109221 & 2001 & 139.50 & 0.04 & 11254.00 & 111731.11 & 1.24 & 0.80 & 0.99 \\
6259 & 100833 & 2001 & 927.40 & -0.24 & 93672.00 & 898920.84 & 0.99 & 0.97 & 0.96 \\
43779 & 109219 & 2001 & 45.50 & -0.24 & 4565.00 & 45605.41 & 1.00 & 1.00 & 1.00 \\
15499 & 101999 & 2001 & 1543.00 & 0.04 & 154298.00 & 1516689.84 & 1.00 & 0.98 & 0.98 \\
15480 & 101998 & 2001 & 1254.00 & -0.07 & 125399.00 & 1176030.86 & 1.00 & 0.94 & 0.94 \\
15530 & 102000 & 2001 & 528.30 & -0.30 & 45137.00 & 515512.95 & 1.17 & 0.98 & 1.14 \\
6671 & 100908 & 2001 & 202.80 & -0.21 & 20304.00 & 187267.55 & 1.00 & 0.92 & 0.92 \\
15460 & 101992 & 2001 & 399.20 & -0.17 & 39919.00 & 383517.91 & 1.00 & 0.96 & 0.96 \\
14913 & 101922 & 2001 & 356.20 & -0.05 & 29935.00 & 331495.40 & 1.19 & 0.93 & 1.11 \\
13732 & 101759 & 2001 & 42.70 & -0.26 & 3712.00 & 43574.24 & 1.15 & 1.02 & 1.17 \\
19556 & 102624 & 2001 & 919.20 & -0.22 & 91926.00 & 918313.17 & 1.00 & 1.00 & 1.00 \\
4064 & 100544 & 2001 & 383.20 & 0.05 & 45728.00 & 428704.16 & 0.84 & 1.12 & 0.94 \\
6303 & 100847 & 2001 & 3.60 & 0.14 & 335.00 & 3689.56 & 1.07 & 1.02 & 1.10 \\
43985 & 109254 & 2001 & 73.00 & -0.43 & 5006.00 & 61435.16 & 1.46 & 0.84 & 1.23 \\
6232 & 100831 & 2001 & 153.40 & -0.09 & 15332.00 & 145646.12 & 1.00 & 0.95 & 0.95 \\
15640 & 102010 & 2001 & 9939.50 & -0.26 & 993956.00 & 9509569.89 & 1.00 & 0.96 & 0.96 \\
6198 & 100829 & 2001 & 604.30 & -0.10 & 60772.00 & 583833.05 & 0.99 & 0.97 & 0.96 \\
7315 & 101020 & 2001 & 2311.10 & -0.08 & 211187.00 & 2108246.71 & 1.09 & 0.91 & 1.00 \\
19470 & 102606 & 2001 & 4290.20 & -0.04 & 429024.00 & 4096643.30 & 1.00 & 0.95 & 0.95 \\
19510 & 102608 & 2001 & 78.40 & 0.07 & 7849.00 & 74010.82 & 1.00 & 0.94 & 0.94 \\
15619 & 102009 & 2001 & 639.30 & -0.02 & 66841.00 & 619535.00 & 0.96 & 0.97 & 0.93 \\
15589 & 102007 & 2001 & 4992.00 & -0.22 & 462495.00 & 4655159.28 & 1.08 & 0.93 & 1.01 \\
19493 & 102607 & 2001 & 731.10 & -0.12 & 73101.00 & 697807.01 & 1.00 & 0.95 & 0.95 \\
14876 & 101919 & 2001 & 1848.50 & -0.07 & 145306.00 & 1823404.54 & 1.27 & 0.99 & 1.25 \\
15574 & 102005 & 2001 & 759.90 & -0.25 & 75984.00 & 685578.66 & 1.00 & 0.90 & 0.90 \\
6688 & 100910 & 2001 & 125.80 & -0.12 & 12586.00 & 111364.29 & 1.00 & 0.89 & 0.88 \\
43841 & 109224 & 2001 & 4.70 & -0.05 & 458.00 & 4541.69 & 1.03 & 0.97 & 0.99 \\
43987 & 109255 & 2001 & 445.60 & -0.13 & 44592.00 & 443801.57 & 1.00 & 1.00 & 1.00 \\
6116 & 100823 & 2001 & 39.10 & 0.17 & 3872.00 & 38483.69 & 1.01 & 0.98 & 0.99 \\
44064 & 109264 & 2001 & 47.80 & 0.04 & 3706.00 & 30472.41 & 1.29 & 0.64 & 0.82 \\
48243 & 240056 & 2001 & 128.40 & -0.30 & 12848.00 & 103998.01 & 1.00 & 0.81 & 0.81 \\
6054 & 100821 & 2001 & 88.40 & -0.01 & 8717.00 & 86708.99 & 1.01 & 0.98 & 0.99 \\
4228 & 100590 & 2001 & 253.80 & -0.33 & 21655.00 & 263176.64 & 1.17 & 1.04 & 1.22 \\
46608 & 200262 & 2001 & 8.30 & -0.15 & 841.00 & 6714.53 & 0.99 & 0.81 & 0.80 \\
47608 & 215687 & 2001 & 305.20 & 0.05 & 30550.00 & 281609.93 & 1.00 & 0.92 & 0.92 \\
4214 & 100575 & 2001 & 15.70 & -0.16 & 1573.00 & 15417.65 & 1.00 & 0.98 & 0.98 \\
55033 & 400049 & 2001 & 41.40 & 0.08 & 4140.00 & 34616.07 & 1.00 & 0.84 & 0.84 \\
19223 & 102570 & 2001 & 314.10 & -0.37 & 24723.00 & 283509.81 & 1.27 & 0.90 & 1.15 \\
15983 & 102062 & 2001 & 208.50 & 0.15 & 19077.00 & 190745.38 & 1.09 & 0.91 & 1.00 \\
13805 & 101764 & 2001 & 822.70 & -0.17 & 101653.00 & 912819.55 & 0.81 & 1.11 & 0.90 \\
44046 & 109263 & 2001 & 8.70 & -0.14 & 720.00 & 6227.30 & 1.21 & 0.72 & 0.86 \\
16027 & 102073 & 2001 & 8750.30 & -0.01 & 793458.00 & 8704858.25 & 1.10 & 0.99 & 1.10 \\
16109 & 102080 & 2001 & 2777.50 & 0.30 & 200633.00 & 2240085.18 & 1.38 & 0.81 & 1.12 \\
47631 & 215952 & 2001 & 817.00 & -0.12 & 72922.00 & 643830.30 & 1.12 & 0.79 & 0.88 \\
6760 & 100950 & 2001 & 434.90 & 0.17 & 43183.00 & 420865.20 & 1.01 & 0.97 & 0.97 \\
47623 & 215696 & 2001 & 116.00 & -0.17 & 11612.00 & 113298.96 & 1.00 & 0.98 & 0.98 \\
5976 & 100815 & 2001 & 311.20 & -0.03 & 31131.00 & 303196.77 & 1.00 & 0.97 & 0.97 \\
16078 & 102079 & 2001 & 458.90 & 0.10 & 37209.00 & 430940.60 & 1.23 & 0.94 & 1.16 \\
46603 & 200261 & 2001 & 12.10 & -0.17 & 1218.00 & 11548.73 & 0.99 & 0.95 & 0.95 \\
14774 & 101913 & 2001 & 24.80 & -0.25 & 2276.00 & 25863.87 & 1.09 & 1.04 & 1.14 \\
5984 & 100817 & 2001 & 262.90 & -0.14 & 26233.00 & 260701.14 & 1.00 & 0.99 & 0.99 \\
6013 & 100818 & 2001 & 80.50 & -0.21 & 8012.00 & 78868.48 & 1.00 & 0.98 & 0.98 \\
44087 & 109265 & 2001 & 45.70 & -0.05 & 2792.00 & 27103.08 & 1.64 & 0.59 & 0.97 \\
6023 & 100820 & 2001 & 470.90 & -0.16 & 47105.00 & 470407.14 & 1.00 & 1.00 & 1.00 \\
44133 & 109268 & 2001 & 44.00 & 0.09 & 4898.00 & 43725.81 & 0.90 & 0.99 & 0.89 \\
5649 & 100784 & 2001 & 9395.70 & 0.09 & 934562.00 & 8297177.14 & 1.01 & 0.88 & 0.89 \\
46639 & 200267 & 2001 & 5.70 & -0.15 & 585.00 & 5549.27 & 0.97 & 0.97 & 0.95 \\
44044 & 109261 & 2001 & 30.00 & -0.05 & 2985.00 & 29746.42 & 1.01 & 0.99 & 1.00 \\
47582 & 215413 & 2001 & 7.20 & 0.05 & 707.00 & 6965.69 & 1.02 & 0.97 & 0.99 \\
6085 & 100822 & 2001 & 13.80 & -0.03 & 1386.00 & 13859.47 & 1.00 & 1.00 & 1.00 \\
4177 & 100567 & 2001 & 1339.40 & -0.35 & 117525.00 & 1320538.65 & 1.14 & 0.99 & 1.12 \\
46706 & 200286 & 2001 & 42.80 & -0.13 & 4354.00 & 40409.35 & 0.98 & 0.94 & 0.93 \\
19303 & 102588 & 2001 & 404.50 & 0.11 & 39556.00 & 379673.62 & 1.02 & 0.94 & 0.96 \\
44012 & 109258 & 2001 & 291.90 & 0.00 & 18505.00 & 149600.71 & 1.58 & 0.51 & 0.81 \\
4145 & 100561 & 2001 & 67.70 & 0.07 & 7393.00 & 61341.57 & 0.92 & 0.91 & 0.83 \\
48267 & 240058 & 2001 & 148.30 & 0.00 & 12112.00 & 117443.78 & 1.22 & 0.79 & 0.97 \\
19334 & 102597 & 2001 & 374.70 & -0.49 & 37463.00 & 374500.04 & 1.00 & 1.00 & 1.00 \\
44010 & 109256 & 2001 & 16.10 & -0.32 & 1284.00 & 12612.12 & 1.25 & 0.78 & 0.98 \\
13775 & 101763 & 2001 & 65.80 & 0.14 & 6600.00 & 63464.32 & 1.00 & 0.96 & 0.96 \\
46643 & 200268 & 2001 & 20.80 & -0.13 & 2226.00 & 20729.93 & 0.93 & 1.00 & 0.93 \\
19283 & 102579 & 2001 & 495.70 & -0.33 & 49625.00 & 470008.14 & 1.00 & 0.95 & 0.95 \\
14806 & 101914 & 2001 & 42.70 & -0.06 & 3846.00 & 41179.07 & 1.11 & 0.96 & 1.07 \\
46647 & 200270 & 2001 & 2.90 & -0.14 & 300.00 & 2989.68 & 0.97 & 1.03 & 1.00 \\
46649 & 200271 & 2001 & 4.20 & -0.31 & 427.00 & 4086.13 & 0.98 & 0.97 & 0.96 \\
44036 & 109260 & 2001 & 28.60 & -0.11 & 2875.00 & 28090.29 & 0.99 & 0.98 & 0.98 \\
15945 & 102061 & 2001 & 1135.00 & 0.12 & 101920.00 & 1019198.49 & 1.11 & 0.90 & 1.00 \\
6749 & 100947 & 2001 & 927.90 & -0.19 & 46527.00 & 458937.87 & 1.99 & 0.49 & 0.99 \\
59253 & 410448 & 2001 & 7.60 & 0.04 & 692.00 & 7707.70 & 1.10 & 1.01 & 1.11 \\
46693 & 200279 & 2001 & 59.60 & -0.23 & 5191.00 & 60874.38 & 1.15 & 1.02 & 1.17 \\
19255 & 102575 & 2001 & 251.80 & -0.15 & 19032.00 & 235164.79 & 1.32 & 0.93 & 1.24 \\
44028 & 109259 & 2001 & 42.10 & 0.04 & 3024.00 & 30618.28 & 1.39 & 0.73 & 1.01 \\
48257 & 240057 & 2001 & 177.90 & -0.13 & 17791.00 & 159595.14 & 1.00 & 0.90 & 0.90 \\
15915 & 102059 & 2001 & 562.90 & -0.07 & 56104.00 & 535671.62 & 1.00 & 0.95 & 0.95 \\
19271 & 102578 & 2001 & 89.30 & -0.32 & 6612.00 & 80026.53 & 1.35 & 0.90 & 1.21 \\
46662 & 200274 & 2001 & 14.30 & -0.07 & 1431.00 & 13290.95 & 1.00 & 0.93 & 0.93 \\
48319 & 240061 & 2001 & 314.60 & -0.10 & 26837.00 & 254678.24 & 1.17 & 0.81 & 0.95 \\
4401 & 100622 & 2001 & 390.50 & 0.06 & 39129.00 & 360190.28 & 1.00 & 0.92 & 0.92 \\
5624 & 100775 & 2001 & 585.30 & -0.11 & 61633.00 & 594359.10 & 0.95 & 1.02 & 0.96 \\
5052 & 100710 & 2001 & 598.90 & -0.13 & 57866.00 & 518812.57 & 1.03 & 0.87 & 0.90 \\
47976 & 225484 & 2001 & 70.20 & -0.07 & 7013.00 & 69212.50 & 1.00 & 0.99 & 0.99 \\
18322 & 102425 & 2001 & 2224.20 & -0.25 & 254890.00 & 1847047.49 & 0.87 & 0.83 & 0.72 \\
48105 & 240010 & 2001 & 160.60 & 0.09 & 16030.00 & 160145.40 & 1.00 & 1.00 & 1.00 \\
5028 & 100701 & 2001 & 43.70 & 0.09 & 5419.00 & 54194.59 & 0.81 & 1.24 & 1.00 \\
47989 & 225687 & 2001 & 658.40 & -0.52 & 66915.00 & 633978.36 & 0.98 & 0.96 & 0.95 \\
5018 & 100700 & 2001 & 108.30 & -0.04 & 10936.00 & 98440.59 & 0.99 & 0.91 & 0.90 \\
17550 & 102318 & 2001 & 3140.40 & -0.11 & 298422.00 & 2865607.16 & 1.05 & 0.91 & 0.96 \\
45545 & 200074 & 2001 & 56.30 & -0.23 & 5818.00 & 54769.02 & 0.97 & 0.97 & 0.94 \\
45536 & 200073 & 2001 & 532.00 & -0.10 & 41148.00 & 486242.35 & 1.29 & 0.91 & 1.18 \\
14244 & 101835 & 2001 & 1410.50 & -0.12 & 140955.00 & 1383455.07 & 1.00 & 0.98 & 0.98 \\
58045 & 410060 & 2001 & 10.80 & -0.41 & 1076.00 & 10743.81 & 1.00 & 0.99 & 1.00 \\
17584 & 102319 & 2001 & 577.70 & -0.03 & 56007.00 & 560073.58 & 1.03 & 0.97 & 1.00 \\
45510 & 200072 & 2001 & 2.50 & 0.18 & 217.00 & 2089.66 & 1.15 & 0.84 & 0.96 \\
58073 & 410075 & 2001 & 157.80 & -0.03 & 15744.00 & 154411.90 & 1.00 & 0.98 & 0.98 \\
48002 & 225696 & 2001 & 48.50 & -0.13 & 4892.00 & 45063.29 & 0.99 & 0.93 & 0.92 \\
45444 & 200060 & 2001 & 604.00 & -0.20 & 60374.00 & 567495.15 & 1.00 & 0.94 & 0.94 \\
45470 & 200061 & 2001 & 69.00 & -0.01 & 6899.00 & 62388.36 & 1.00 & 0.90 & 0.90 \\
4704 & 100667 & 2001 & 14.80 & 0.01 & 1513.00 & 13221.25 & 0.98 & 0.89 & 0.87 \\
17621 & 102321 & 2001 & 164.20 & 0.08 & 15660.00 & 156596.56 & 1.05 & 0.95 & 1.00 \\
55415 & 400094 & 2001 & 239.50 & 0.05 & 21801.00 & 205974.61 & 1.10 & 0.86 & 0.94 \\
7047 & 100992 & 2001 & 1008.30 & -0.23 & 101229.00 & 978863.85 & 1.00 & 0.97 & 0.97 \\
4999 & 100698 & 2001 & 35.80 & 0.46 & 3242.00 & 31797.82 & 1.10 & 0.89 & 0.98 \\
45476 & 200065 & 2001 & 5.50 & -0.05 & 574.00 & 5427.46 & 0.96 & 0.99 & 0.95 \\
45507 & 200071 & 2001 & 389.80 & -0.29 & 33749.00 & 393555.37 & 1.15 & 1.01 & 1.17 \\
14229 & 101834 & 2001 & 243.30 & -0.05 & 24300.00 & 235168.03 & 1.00 & 0.97 & 0.97 \\
45576 & 200078 & 2001 & 2.80 & -0.24 & 267.00 & 2672.01 & 1.05 & 0.95 & 1.00 \\
45577 & 200079 & 2001 & 8.70 & -0.38 & 617.00 & 8321.10 & 1.41 & 0.96 & 1.35 \\
4597 & 100642 & 2001 & 1448.30 & -0.12 & 165703.00 & 1451659.36 & 0.87 & 1.00 & 0.88 \\
14304 & 101843 & 2001 & 179.50 & -0.06 & 21929.00 & 190595.37 & 0.82 & 1.06 & 0.87 \\
5135 & 100726 & 2001 & 7156.30 & -0.27 & 721346.00 & 6264936.21 & 0.99 & 0.88 & 0.87 \\
18400 & 102447 & 2001 & 3196.80 & 0.07 & 284280.00 & 2518375.98 & 1.12 & 0.79 & 0.89 \\
44748 & 109371 & 2001 & 131.20 & 0.02 & 11800.00 & 118014.72 & 1.11 & 0.90 & 1.00 \\
5115 & 100724 & 2001 & 49.30 & -0.11 & 5131.00 & 46776.64 & 0.96 & 0.95 & 0.91 \\
17385 & 102284 & 2001 & 273.80 & 0.11 & 27589.00 & 271669.12 & 0.99 & 0.99 & 0.98 \\
57968 & 410010 & 2001 & 273.60 & 0.01 & 27596.00 & 275919.66 & 0.99 & 1.01 & 1.00 \\
44723 & 109368 & 2001 & 140.30 & 0.01 & 16072.00 & 143718.27 & 0.87 & 1.02 & 0.89 \\
44722 & 109367 & 2001 & 170.00 & -0.40 & 18941.00 & 156187.46 & 0.90 & 0.92 & 0.82 \\
7018 & 100985 & 2001 & 1352.40 & -0.07 & 134135.00 & 1155331.26 & 1.01 & 0.85 & 0.86 \\
5158 & 100727 & 2001 & 305.20 & -0.27 & 30808.00 & 285632.03 & 0.99 & 0.94 & 0.93 \\
47921 & 222809 & 2001 & 39.90 & 0.07 & 4026.00 & 36414.60 & 0.99 & 0.91 & 0.90 \\
44733 & 109370 & 2001 & 232.70 & 0.06 & 17781.00 & 157881.82 & 1.31 & 0.68 & 0.89 \\
48011 & 226438 & 2001 & 657.10 & -0.23 & 55540.00 & 637569.26 & 1.18 & 0.97 & 1.15 \\
4626 & 100657 & 2001 & 300.40 & 0.01 & 51409.00 & 419950.99 & 0.58 & 1.40 & 0.82 \\
17438 & 102306 & 2001 & 21789.10 & -0.12 & 1988048.00 & 20069591.53 & 1.10 & 0.92 & 1.01 \\
17507 & 102317 & 2001 & 18.00 & 0.01 & 1759.00 & 16936.21 & 1.02 & 0.94 & 0.96 \\
17499 & 102314 & 2001 & 261.30 & -0.33 & 26056.00 & 266019.11 & 1.00 & 1.02 & 1.02 \\
14277 & 101842 & 2001 & 2510.20 & -0.13 & 250966.00 & 2494219.84 & 1.00 & 0.99 & 0.99 \\
4643 & 100659 & 2001 & 602.80 & 0.06 & 80657.00 & 748431.82 & 0.75 & 1.24 & 0.93 \\
17488 & 102313 & 2001 & 312.90 & -0.44 & 30934.00 & 309315.22 & 1.01 & 0.99 & 1.00 \\
17475 & 102312 & 2001 & 108.10 & -0.43 & 7958.00 & 93496.69 & 1.36 & 0.86 & 1.17 \\
47948 & 225413 & 2001 & 57.10 & 0.06 & 5958.00 & 50013.79 & 0.96 & 0.88 & 0.84 \\
18356 & 102446 & 2001 & 16.00 & -0.02 & 1614.00 & 14442.49 & 0.99 & 0.90 & 0.89 \\
58034 & 410055 & 2001 & 75.60 & -0.14 & 7548.00 & 75468.39 & 1.00 & 1.00 & 1.00 \\
44773 & 109375 & 2001 & 14.80 & 0.08 & 1482.00 & 13806.29 & 1.00 & 0.93 & 0.93 \\
45592 & 200082 & 2001 & 6.60 & -0.03 & 331.00 & 2925.57 & 1.99 & 0.44 & 0.88 \\
44767 & 109374 & 2001 & 206.20 & -0.15 & 20447.00 & 204467.08 & 1.01 & 0.99 & 1.00 \\
5084 & 100723 & 2001 & 23.80 & -0.03 & 2166.00 & 22656.71 & 1.10 & 0.95 & 1.05 \\
14023 & 101800 & 2001 & 646.00 & -0.18 & 64982.00 & 615027.87 & 0.99 & 0.95 & 0.95 \\
44826 & 109393 & 2001 & 8.40 & 0.01 & 487.00 & 4743.61 & 1.72 & 0.56 & 0.97 \\
4966 & 100697 & 2001 & 36.20 & 0.02 & 3564.00 & 34737.41 & 1.02 & 0.96 & 0.97 \\
58137 & 410100 & 2001 & 30.00 & 0.03 & 2833.00 & 28333.26 & 1.06 & 0.94 & 1.00 \\
44952 & 109404 & 2001 & 19.20 & 0.01 & 1916.00 & 18673.85 & 1.00 & 0.97 & 0.97 \\
55355 & 400090 & 2001 & 10.10 & -0.09 & 1020.00 & 8960.80 & 0.99 & 0.89 & 0.88 \\
17844 & 102365 & 2001 & 504.10 & -0.23 & 44241.00 & 500159.34 & 1.14 & 0.99 & 1.13 \\
44957 & 109405 & 2001 & 18.10 & -0.01 & 1778.00 & 16724.96 & 1.02 & 0.92 & 0.94 \\
7084 & 100996 & 2001 & 2038.30 & -0.13 & 188319.00 & 1977577.12 & 1.08 & 0.97 & 1.05 \\
4890 & 100691 & 2001 & 977.00 & -0.25 & 97746.00 & 971799.74 & 1.00 & 0.99 & 0.99 \\
48041 & 227155 & 2001 & 39.20 & -0.10 & 3954.00 & 37712.96 & 0.99 & 0.96 & 0.95 \\
18131 & 102404 & 2001 & 2929.10 & -0.13 & 293318.00 & 2896233.72 & 1.00 & 0.99 & 0.99 \\
14133 & 101805 & 2001 & 740.40 & -0.17 & 74033.00 & 718322.08 & 1.00 & 0.97 & 0.97 \\
17872 & 102367 & 2001 & 160.90 & -0.01 & 14990.00 & 148991.79 & 1.07 & 0.93 & 0.99 \\
48070 & 235413 & 2001 & 87.20 & 0.16 & 8725.00 & 78111.70 & 1.00 & 0.90 & 0.90 \\
17885 & 102371 & 2001 & 320.10 & -0.00 & 28233.00 & 310830.93 & 1.13 & 0.97 & 1.10 \\
18044 & 102387 & 2001 & 64.20 & 0.34 & 4734.00 & 57650.99 & 1.36 & 0.90 & 1.22 \\
18009 & 102386 & 2001 & 31.70 & 0.08 & 4467.00 & 46576.33 & 0.71 & 1.47 & 1.04 \\
7115 & 100997 & 2001 & 375.90 & -0.01 & 26516.00 & 303021.27 & 1.42 & 0.81 & 1.14 \\
17995 & 102383 & 2001 & 7.50 & -0.04 & 741.00 & 6854.62 & 1.01 & 0.91 & 0.93 \\
4842 & 100685 & 2001 & 13.90 & 0.24 & 1152.00 & 13148.07 & 1.21 & 0.95 & 1.14 \\
17969 & 102377 & 2001 & 80.10 & 0.00 & 9158.00 & 78027.32 & 0.87 & 0.97 & 0.85 \\
55332 & 400085 & 2001 & 39.10 & 0.01 & 3925.00 & 35432.60 & 1.00 & 0.91 & 0.90 \\
18082 & 102396 & 2001 & 3229.00 & 0.20 & 322909.00 & 2865725.04 & 1.00 & 0.89 & 0.89 \\
17905 & 102372 & 2001 & 3635.10 & 0.18 & 302822.00 & 3155735.71 & 1.20 & 0.87 & 1.04 \\
55347 & 400088 & 2001 & 132.80 & 0.07 & 13514.00 & 131794.88 & 0.98 & 0.99 & 0.98 \\
45277 & 200011 & 2001 & 127.00 & -0.15 & 13420.00 & 122215.03 & 0.95 & 0.96 & 0.91 \\
4809 & 100682 & 2001 & 78.40 & -0.09 & 7423.00 & 69370.51 & 1.06 & 0.88 & 0.93 \\
44929 & 109402 & 2001 & 265.30 & 0.00 & 25952.00 & 240904.16 & 1.02 & 0.91 & 0.93 \\
17813 & 102364 & 2001 & 799.60 & -0.12 & 64304.00 & 687570.34 & 1.24 & 0.86 & 1.07 \\
55394 & 400093 & 2001 & 88.00 & -0.39 & 8267.00 & 75282.68 & 1.06 & 0.86 & 0.91 \\
48018 & 226504 & 2001 & 25.40 & -0.22 & 4084.00 & 24972.95 & 0.62 & 0.98 & 0.61 \\
45381 & 200055 & 2001 & 76.30 & -0.11 & 7865.00 & 74242.65 & 0.97 & 0.97 & 0.94 \\
17732 & 102350 & 2001 & 406.80 & -0.27 & 40224.00 & 402206.89 & 1.01 & 0.99 & 1.00 \\
4741 & 100670 & 2001 & 85.20 & -0.09 & 9730.00 & 90706.02 & 0.88 & 1.06 & 0.93 \\
45407 & 200057 & 2001 & 1155.90 & -0.34 & 139463.00 & 1000310.13 & 0.83 & 0.87 & 0.72 \\
45376 & 200051 & 2001 & 2.40 & -0.17 & 185.00 & 2249.26 & 1.30 & 0.94 & 1.22 \\
4725 & 100669 & 2001 & 391.20 & -0.09 & 39144.00 & 388643.68 & 1.00 & 0.99 & 0.99 \\
55270 & 400076 & 2001 & 280.40 & 0.00 & 28214.00 & 269699.46 & 0.99 & 0.96 & 0.96 \\
18228 & 102417 & 2001 & 1245.90 & -0.12 & 146513.00 & 1137629.30 & 0.85 & 0.91 & 0.78 \\
14198 & 101820 & 2001 & 292.70 & -0.12 & 38926.00 & 365401.49 & 0.75 & 1.25 & 0.94 \\
45418 & 200058 & 2001 & 1401.00 & -0.21 & 143122.00 & 1372864.03 & 0.98 & 0.98 & 0.96 \\
44856 & 109395 & 2001 & 19.50 & 0.02 & 2249.00 & 16507.08 & 0.87 & 0.85 & 0.73 \\
17709 & 102349 & 2001 & 880.70 & -0.17 & 87126.00 & 870687.35 & 1.01 & 0.99 & 1.00 \\
58100 & 410093 & 2001 & 94.80 & -0.56 & 7606.00 & 77173.79 & 1.25 & 0.81 & 1.01 \\
44879 & 109397 & 2001 & 38.40 & -0.01 & 4098.00 & 35426.20 & 0.94 & 0.92 & 0.86 \\
4910 & 100692 & 2001 & 991.20 & -0.25 & 98367.00 & 946925.42 & 1.01 & 0.96 & 0.96 \\
58133 & 410095 & 2001 & 65.40 & -0.01 & 3849.00 & 37230.07 & 1.70 & 0.57 & 0.97 \\
4763 & 100671 & 2001 & 261.00 & 0.03 & 26611.00 & 226704.23 & 0.98 & 0.87 & 0.85 \\
45335 & 200039 & 2001 & 105.00 & -0.08 & 10520.00 & 103872.46 & 1.00 & 0.99 & 0.99 \\
4940 & 100695 & 2001 & 176.40 & -0.07 & 17646.00 & 158235.55 & 1.00 & 0.90 & 0.90 \\
17784 & 102357 & 2001 & 1576.10 & -0.06 & 157286.00 & 1531888.35 & 1.00 & 0.97 & 0.97 \\
45340 & 200047 & 2001 & 25.00 & -0.23 & 2482.00 & 21805.51 & 1.01 & 0.87 & 0.88 \\
44912 & 109401 & 2001 & 10.90 & 0.06 & 1127.00 & 10008.21 & 0.97 & 0.92 & 0.89 \\
45366 & 200050 & 2001 & 72.60 & 0.13 & 7443.00 & 69290.57 & 0.98 & 0.95 & 0.93 \\
44902 & 109399 & 2001 & 18.40 & 0.02 & 1826.00 & 16868.86 & 1.01 & 0.92 & 0.92 \\
17753 & 102356 & 2001 & 3.90 & -0.12 & 344.00 & 3454.72 & 1.13 & 0.89 & 1.00 \\
18430 & 102452 & 2001 & 97.80 & -0.24 & 9813.00 & 90939.55 & 1.00 & 0.93 & 0.93 \\
17341 & 102282 & 2001 & 115.40 & -0.03 & 11202.00 & 113909.89 & 1.03 & 0.99 & 1.02 \\
5176 & 100730 & 2001 & 604.40 & -0.26 & 60539.00 & 586103.25 & 1.00 & 0.97 & 0.97 \\
59034 & 410401 & 2001 & 118.20 & -0.09 & 11019.00 & 97890.19 & 1.07 & 0.83 & 0.89 \\
7202 & 101013 & 2001 & 9711.40 & -0.15 & 986587.00 & 9454471.54 & 0.98 & 0.97 & 0.96 \\
13937 & 101788 & 2001 & 808.40 & -0.12 & 85629.00 & 799621.67 & 0.94 & 0.99 & 0.93 \\
4439 & 100625 & 2001 & 1930.70 & -0.01 & 196249.00 & 1721820.59 & 0.98 & 0.89 & 0.88 \\
5469 & 100764 & 2001 & 332.20 & 0.11 & 33185.00 & 307600.30 & 1.00 & 0.93 & 0.93 \\
47768 & 221210 & 2001 & 159.00 & -0.02 & 15928.00 & 159262.02 & 1.00 & 1.00 & 1.00 \\
48181 & 240040 & 2001 & 343.10 & -0.11 & 40108.00 & 337824.49 & 0.86 & 0.98 & 0.84 \\
4454 & 100633 & 2001 & 681.40 & -0.18 & 64969.00 & 645897.73 & 1.05 & 0.95 & 0.99 \\
44594 & 109347 & 2001 & 198.10 & 0.06 & 16806.00 & 166952.61 & 1.18 & 0.84 & 0.99 \\
18656 & 102500 & 2001 & 200.60 & -0.19 & 19826.00 & 187916.81 & 1.01 & 0.94 & 0.95 \\
13956 & 101789 & 2001 & 509.60 & -0.13 & 58618.00 & 547392.72 & 0.87 & 1.07 & 0.93 \\
5447 & 100763 & 2001 & 1377.00 & -0.15 & 136218.00 & 1255292.07 & 1.01 & 0.91 & 0.92 \\
6922 & 100969 & 2001 & 49.50 & -0.13 & 5832.00 & 51764.53 & 0.85 & 1.05 & 0.89 \\
44641 & 109350 & 2001 & 10.60 & 0.02 & 1057.00 & 10450.75 & 1.00 & 0.99 & 0.99 \\
17005 & 102230 & 2001 & 27.40 & -0.13 & 2740.00 & 23609.55 & 1.00 & 0.86 & 0.86 \\
57883 & 401361 & 2001 & 189.60 & -0.15 & 25017.00 & 198260.30 & 0.76 & 1.05 & 0.79 \\
57881 & 401360 & 2001 & 13.50 & -0.50 & 1243.00 & 12434.72 & 1.09 & 0.92 & 1.00 \\
6912 & 100968 & 2001 & 198.80 & 0.04 & 18337.00 & 200617.02 & 1.08 & 1.01 & 1.09 \\
57879 & 401359 & 2001 & 3.00 & -0.24 & 298.00 & 2838.83 & 1.01 & 0.95 & 0.95 \\
18641 & 102493 & 2001 & 2027.60 & -0.12 & 202964.00 & 1950827.07 & 1.00 & 0.96 & 0.96 \\
16969 & 102224 & 2001 & 4676.10 & 0.03 & 488792.00 & 4446920.14 & 0.96 & 0.95 & 0.91 \\
57876 & 401355 & 2001 & 100.60 & -0.10 & 10136.00 & 104853.06 & 0.99 & 1.04 & 1.03 \\
47799 & 221485 & 2001 & 662.80 & -0.11 & 66350.00 & 657324.60 & 1.00 & 0.99 & 0.99 \\
57867 & 401296 & 2001 & 2.10 & 0.02 & 227.00 & 2094.03 & 0.93 & 1.00 & 0.92 \\
57866 & 401293 & 2001 & 1.70 & -0.27 & 172.00 & 1643.13 & 0.99 & 0.97 & 0.96 \\
4479 & 100634 & 2001 & 1325.80 & -0.21 & 127429.00 & 1271747.85 & 1.04 & 0.96 & 1.00 \\
57855 & 401189 & 2001 & 83.10 & -0.23 & 7526.00 & 80803.54 & 1.10 & 0.97 & 1.07 \\
5566 & 100772 & 2001 & 859.70 & 0.13 & 100169.00 & 951412.21 & 0.86 & 1.11 & 0.95 \\
44505 & 109333 & 2001 & 16.10 & 0.06 & 1669.00 & 15362.83 & 0.96 & 0.95 & 0.92 \\
6880 & 100967 & 2001 & 501.40 & 0.15 & 38288.00 & 407302.01 & 1.31 & 0.81 & 1.06 \\
59050 & 410418 & 2001 & 1089.70 & -0.10 & 146976.00 & 1476104.59 & 0.74 & 1.35 & 1.00 \\
14556 & 101876 & 2001 & 186.30 & -0.14 & 19381.00 & 167768.73 & 0.96 & 0.90 & 0.87 \\
13918 & 101787 & 2001 & 705.50 & -0.04 & 74252.00 & 717389.72 & 0.95 & 1.02 & 0.97 \\
16776 & 102191 & 2001 & 58.60 & -0.04 & 5867.00 & 52443.30 & 1.00 & 0.89 & 0.89 \\
18758 & 102507 & 2001 & 297.10 & -0.31 & 26190.00 & 309353.02 & 1.13 & 1.04 & 1.18 \\
5605 & 100773 & 2001 & 1662.70 & 0.23 & 165917.00 & 1650254.83 & 1.00 & 0.99 & 0.99 \\
16743 & 102183 & 2001 & 264.20 & -0.12 & 22350.00 & 197622.08 & 1.18 & 0.75 & 0.88 \\
46226 & 200199 & 2001 & 8297.20 & -0.18 & 669247.00 & 5904408.84 & 1.24 & 0.71 & 0.88 \\
16726 & 102182 & 2001 & 126.00 & -0.02 & 11237.00 & 96629.58 & 1.12 & 0.77 & 0.86 \\
6872 & 100966 & 2001 & 6.50 & -0.04 & 857.00 & 7204.60 & 0.76 & 1.11 & 0.84 \\
5403 & 100760 & 2001 & 649.50 & -0.17 & 64875.00 & 619304.47 & 1.00 & 0.95 & 0.95 \\
44516 & 109334 & 2001 & 24.80 & 0.03 & 2224.00 & 19251.01 & 1.12 & 0.78 & 0.87 \\
16878 & 102213 & 2001 & 1331.10 & -0.19 & 133109.00 & 1274366.26 & 1.00 & 0.96 & 0.96 \\
55526 & 400116 & 2001 & 30.20 & -0.21 & 2938.00 & 29374.21 & 1.03 & 0.97 & 1.00 \\
57850 & 401145 & 2001 & 28.50 & -0.52 & 2694.00 & 26938.31 & 1.06 & 0.95 & 1.00 \\
14514 & 101871 & 2001 & 287.40 & -0.02 & 31433.00 & 295378.00 & 0.91 & 1.03 & 0.94 \\
5507 & 100769 & 2001 & 3135.70 & 0.23 & 340257.00 & 3253913.14 & 0.92 & 1.04 & 0.96 \\
16790 & 102192 & 2001 & 538.90 & 0.09 & 39869.00 & 493676.45 & 1.35 & 0.92 & 1.24 \\
44579 & 109343 & 2001 & 45.10 & 0.07 & 3761.00 & 34211.33 & 1.20 & 0.76 & 0.91 \\
16848 & 102197 & 2001 & 386.20 & 0.02 & 27886.00 & 251718.60 & 1.38 & 0.65 & 0.90 \\
47746 & 221051 & 2001 & 5485.60 & -0.27 & 538563.00 & 5306859.95 & 1.02 & 0.97 & 0.99 \\
16819 & 102193 & 2001 & 379.80 & -0.15 & 37580.00 & 371712.93 & 1.01 & 0.98 & 0.99 \\
5538 & 100771 & 2001 & 290.30 & 0.13 & 29228.00 & 286302.06 & 0.99 & 0.99 & 0.98 \\
44570 & 109341 & 2001 & 173.30 & 0.07 & 16335.00 & 154591.83 & 1.06 & 0.89 & 0.95 \\
44649 & 109351 & 2001 & 16.70 & -0.02 & 1674.00 & 16540.68 & 1.00 & 0.99 & 0.99 \\
14470 & 101861 & 2001 & 981.70 & -0.05 & 96648.00 & 838181.12 & 1.02 & 0.85 & 0.87 \\
57884 & 401362 & 2001 & 8.40 & -0.47 & 847.00 & 7455.45 & 0.99 & 0.89 & 0.88 \\
57934 & 410003 & 2001 & 739.10 & -0.14 & 61401.00 & 756529.72 & 1.20 & 1.02 & 1.23 \\
47889 & 222658 & 2001 & 308.30 & -0.01 & 29326.00 & 274400.09 & 1.05 & 0.89 & 0.94 \\
5249 & 100741 & 2001 & 215.20 & -0.13 & 21525.00 & 200661.56 & 1.00 & 0.93 & 0.93 \\
7166 & 101000 & 2001 & 1610.70 & 0.01 & 142051.00 & 1516376.71 & 1.13 & 0.94 & 1.07 \\
18495 & 102465 & 2001 & 334.40 & -0.23 & 37708.00 & 358180.49 & 0.89 & 1.07 & 0.95 \\
17226 & 102271 & 2001 & 1105.50 & -0.17 & 97521.00 & 1113590.50 & 1.13 & 1.01 & 1.14 \\
44665 & 109358 & 2001 & 51.40 & -0.07 & 5266.00 & 42693.19 & 0.98 & 0.83 & 0.81 \\
47863 & 222408 & 2001 & 707.60 & -0.09 & 70725.00 & 689948.52 & 1.00 & 0.98 & 0.98 \\
44657 & 109357 & 2001 & 142.40 & 0.03 & 10054.00 & 95067.44 & 1.42 & 0.67 & 0.95 \\
45808 & 200142 & 2001 & 82.10 & -0.23 & 8293.00 & 66050.17 & 0.99 & 0.80 & 0.80 \\
18514 & 102469 & 2001 & 105.00 & -0.04 & 10534.00 & 94996.88 & 1.00 & 0.90 & 0.90 \\
58782 & 410217 & 2001 & 3.10 & 0.18 & 304.00 & 2872.95 & 1.02 & 0.93 & 0.95 \\
17192 & 102270 & 2001 & 962.10 & -0.17 & 96265.00 & 940970.28 & 1.00 & 0.98 & 0.98 \\
48142 & 240027 & 2001 & 161.60 & 0.08 & 16191.00 & 151913.68 & 1.00 & 0.94 & 0.94 \\
5271 & 100745 & 2001 & 1756.80 & -0.21 & 223564.00 & 1578554.48 & 0.79 & 0.90 & 0.71 \\
55444 & 400097 & 2001 & 18.70 & -0.03 & 1744.00 & 17421.35 & 1.07 & 0.93 & 1.00 \\
17253 & 102274 & 2001 & 3414.60 & 0.12 & 274554.00 & 3013151.84 & 1.24 & 0.88 & 1.10 \\
45778 & 200133 & 2001 & 16.90 & -0.21 & 1696.00 & 16679.04 & 1.00 & 0.99 & 0.98 \\
17311 & 102280 & 2001 & 2162.70 & -0.03 & 193360.00 & 2168385.70 & 1.12 & 1.00 & 1.12 \\
14330 & 101850 & 2001 & 571.80 & -0.03 & 51745.00 & 552872.07 & 1.11 & 0.97 & 1.07 \\
55247 & 400075 & 2001 & 907.60 & -0.02 & 90783.00 & 868330.39 & 1.00 & 0.96 & 0.96 \\
18456 & 102461 & 2001 & 2328.80 & -0.35 & 233167.00 & 1959060.58 & 1.00 & 0.84 & 0.84 \\
17293 & 102278 & 2001 & 194.00 & 0.01 & 14882.00 & 148562.64 & 1.30 & 0.77 & 1.00 \\
6992 & 100981 & 2001 & 67.50 & -0.26 & 5856.00 & 67703.30 & 1.15 & 1.00 & 1.16 \\
14355 & 101851 & 2001 & 2337.00 & 0.03 & 180226.00 & 2125588.14 & 1.30 & 0.91 & 1.18 \\
44699 & 109366 & 2001 & 11.70 & -0.25 & 1133.00 & 11326.61 & 1.03 & 0.97 & 1.00 \\
5216 & 100736 & 2001 & 761.00 & -0.08 & 76230.00 & 713523.14 & 1.00 & 0.94 & 0.94 \\
4563 & 100639 & 2001 & 1608.90 & -0.04 & 160232.00 & 1591484.87 & 1.00 & 0.99 & 0.99 \\
18473 & 102462 & 2001 & 8.70 & 0.19 & 910.00 & 8513.61 & 0.96 & 0.98 & 0.94 \\
5198 & 100731 & 2001 & 10724.70 & -0.09 & 1071280.00 & 10483403.31 & 1.00 & 0.98 & 0.98 \\
5292 & 100746 & 2001 & 1137.60 & -0.03 & 114230.00 & 1100497.02 & 1.00 & 0.97 & 0.96 \\
17163 & 102261 & 2001 & 963.90 & -0.08 & 85403.00 & 874302.44 & 1.13 & 0.91 & 1.02 \\
5356 & 100757 & 2001 & 3.00 & -0.14 & 254.00 & 3023.33 & 1.18 & 1.01 & 1.19 \\
5370 & 100758 & 2001 & 63.80 & -0.04 & 6119.00 & 63696.28 & 1.04 & 1.00 & 1.04 \\
18591 & 102490 & 2001 & 75.20 & -0.13 & 6597.00 & 77007.25 & 1.14 & 1.02 & 1.17 \\
45908 & 200162 & 2001 & 10.50 & -0.10 & 1028.00 & 10098.62 & 1.02 & 0.96 & 0.98 \\
45910 & 200164 & 2001 & 13.20 & -0.02 & 1322.00 & 12635.92 & 1.00 & 0.96 & 0.96 \\
17041 & 102231 & 2001 & 827.40 & -0.20 & 84895.00 & 823202.68 & 0.97 & 0.99 & 0.97 \\
55204 & 400072 & 2001 & 556.10 & -0.04 & 58933.00 & 572482.33 & 0.94 & 1.03 & 0.97 \\
57889 & 401368 & 2001 & 20.00 & -0.02 & 2023.00 & 18432.43 & 0.99 & 0.92 & 0.91 \\
18607 & 102491 & 2001 & 427.80 & -0.07 & 41663.00 & 416653.49 & 1.03 & 0.97 & 1.00 \\
57886 & 401363 & 2001 & 7.80 & -0.57 & 457.00 & 7287.78 & 1.71 & 0.93 & 1.59 \\
4490 & 100635 & 2001 & 621.60 & 0.05 & 56170.00 & 561695.53 & 1.11 & 0.90 & 1.00 \\
57921 & 402013 & 2001 & 48.60 & -0.46 & 4067.00 & 46246.36 & 1.19 & 0.95 & 1.14 \\
6941 & 100973 & 2001 & 61.10 & 0.11 & 6117.00 & 57705.16 & 1.00 & 0.94 & 0.94 \\
45890 & 200156 & 2001 & 43.70 & -0.18 & 5087.00 & 44543.05 & 0.86 & 1.02 & 0.88 \\
14405 & 101854 & 2001 & 11577.30 & -0.09 & 1176187.00 & 10913216.67 & 0.98 & 0.94 & 0.93 \\
17149 & 102259 & 2001 & 797.80 & -0.20 & 59401.00 & 666352.84 & 1.34 & 0.84 & 1.12 \\
4529 & 100637 & 2001 & 786.10 & -0.21 & 77228.00 & 757267.60 & 1.02 & 0.96 & 0.98 \\
47838 & 222351 & 2001 & 247.20 & 0.01 & 23371.00 & 231452.56 & 1.06 & 0.94 & 0.99 \\
55227 & 400074 & 2001 & 1806.00 & -0.06 & 183625.00 & 1822701.22 & 0.98 & 1.01 & 0.99 \\
17084 & 102255 & 2001 & 126.60 & -0.24 & 11131.00 & 125577.83 & 1.14 & 0.99 & 1.13 \\
17139 & 102258 & 2001 & 519.20 & -0.16 & 46384.00 & 529278.58 & 1.12 & 1.02 & 1.14 \\
47813 & 222027 & 2001 & 2575.20 & -0.20 & 179633.00 & 2229455.14 & 1.43 & 0.87 & 1.24 \\
45871 & 200151 & 2001 & 24.90 & -0.17 & 2267.00 & 21777.07 & 1.10 & 0.87 & 0.96 \\
45882 & 200153 & 2001 & 53.10 & -0.06 & 5322.00 & 53001.93 & 1.00 & 1.00 & 1.00 \\
17107 & 102257 & 2001 & 2015.60 & -0.13 & 181415.00 & 1942235.04 & 1.11 & 0.96 & 1.07 \\
5347 & 100754 & 2001 & 710.60 & -0.07 & 70970.00 & 695043.59 & 1.00 & 0.98 & 0.98 \\
13979 & 101794 & 2001 & 1658.80 & -0.16 & 145798.00 & 1648458.52 & 1.14 & 0.99 & 1.13 \\
15426 & 101990 & 2001 & 192.40 & 0.22 & 15950.00 & 173335.22 & 1.21 & 0.90 & 1.09 \\
44417 & 109307 & 2001 & 31.20 & 0.01 & 3070.00 & 27249.03 & 1.02 & 0.87 & 0.89 \\
43673 & 109189 & 2001 & 556.60 & 0.00 & 52784.00 & 527842.20 & 1.05 & 0.95 & 1.00 \\
15097 & 101956 & 2001 & 2465.30 & -0.24 & 248248.00 & 2468596.82 & 0.99 & 1.00 & 0.99 \\
47453 & 211210 & 2001 & 51.80 & -0.23 & 5706.00 & 55991.02 & 0.91 & 1.08 & 0.98 \\
15176 & 101964 & 2001 & 676.50 & 0.06 & 65981.00 & 639479.33 & 1.03 & 0.95 & 0.97 \\
6380 & 100856 & 2001 & 126.10 & -0.11 & 12602.00 & 126016.29 & 1.00 & 1.00 & 1.00 \\
56067 & 400171 & 2001 & 2.40 & 0.04 & 248.00 & 2340.11 & 0.97 & 0.98 & 0.94 \\
43696 & 109190 & 2001 & 11.20 & 0.04 & 1083.00 & 10825.38 & 1.03 & 0.97 & 1.00 \\
43740 & 109212 & 2001 & 1.30 & -0.14 & 99.00 & 1170.56 & 1.31 & 0.90 & 1.18 \\
19595 & 102635 & 2001 & 716.20 & -0.16 & 71247.00 & 725840.78 & 1.01 & 1.01 & 1.02 \\
4050 & 100543 & 2001 & 353.40 & -0.24 & 40883.00 & 401986.03 & 0.86 & 1.14 & 0.98 \\
19605 & 102636 & 2001 & 627.70 & -0.04 & 49199.00 & 549692.27 & 1.28 & 0.88 & 1.12 \\
15063 & 101955 & 2001 & 14294.50 & -0.11 & 1558897.00 & 14012772.14 & 0.92 & 0.98 & 0.90 \\
15368 & 101988 & 2001 & 389.70 & -0.07 & 40150.00 & 401308.87 & 0.97 & 1.03 & 1.00 \\
43764 & 109218 & 2001 & 22.90 & -0.08 & 2284.00 & 21630.16 & 1.00 & 0.94 & 0.95 \\
60799 & 410727 & 2001 & 6.60 & -0.00 & 621.00 & 5892.94 & 1.06 & 0.89 & 0.95 \\
15115 & 101958 & 2001 & 951.70 & -0.13 & 94724.00 & 906077.81 & 1.00 & 0.95 & 0.96 \\
47412 & 210770 & 2001 & 1572.50 & 0.04 & 163474.00 & 1624922.63 & 0.96 & 1.03 & 0.99 \\
43703 & 109191 & 2001 & 1.10 & -0.18 & 112.00 & 1119.35 & 0.98 & 1.02 & 1.00 \\
47472 & 212027 & 2001 & 197.30 & -0.22 & 24305.00 & 194765.07 & 0.81 & 0.99 & 0.80 \\
19658 & 102641 & 2001 & 510.40 & -0.18 & 60720.00 & 544843.36 & 0.84 & 1.07 & 0.90 \\
15302 & 101982 & 2001 & 376.40 & -0.13 & 38113.00 & 333776.86 & 0.99 & 0.89 & 0.88 \\
43716 & 109208 & 2001 & 12.20 & -0.15 & 1315.00 & 11962.60 & 0.93 & 0.98 & 0.91 \\
15218 & 101968 & 2001 & 187.70 & -0.32 & 16610.00 & 184031.78 & 1.13 & 0.98 & 1.11 \\
43630 & 109175 & 2001 & 7.50 & -0.09 & 641.00 & 6991.28 & 1.17 & 0.93 & 1.09 \\
43712 & 109205 & 2001 & 55.00 & -0.26 & 5205.00 & 56566.13 & 1.06 & 1.03 & 1.09 \\
47188 & 200342 & 2001 & 5237.50 & 0.12 & 457480.00 & 4666569.66 & 1.14 & 0.89 & 1.02 \\
47381 & 210681 & 2001 & 32320.90 & -0.13 & 3422730.00 & 27172970.90 & 0.94 & 0.84 & 0.79 \\
43709 & 109199 & 2001 & 19.30 & -0.02 & 1377.00 & 13869.60 & 1.40 & 0.72 & 1.01 \\
43714 & 109207 & 2001 & 2.40 & -0.13 & 280.00 & 2454.40 & 0.86 & 1.02 & 0.88 \\
47253 & 200344 & 2001 & 5712.30 & -0.02 & 566836.00 & 4785141.47 & 1.01 & 0.84 & 0.84 \\
15253 & 101972 & 2001 & 2312.10 & -0.08 & 189697.00 & 2072060.52 & 1.22 & 0.90 & 1.09 \\
43639 & 109176 & 2001 & 16.30 & -0.13 & 1630.00 & 15619.55 & 1.00 & 0.96 & 0.96 \\
19671 & 102645 & 2001 & 361.10 & -0.07 & 43410.00 & 345959.45 & 0.83 & 0.96 & 0.80 \\
47460 & 211485 & 2001 & 93.70 & -0.09 & 9375.00 & 92664.06 & 1.00 & 0.99 & 0.99 \\
48348 & 240065 & 2001 & 814.10 & -0.06 & 81445.00 & 808312.65 & 1.00 & 0.99 & 0.99 \\
47477 & 212351 & 2001 & 224.50 & -0.22 & 22164.00 & 221639.50 & 1.01 & 0.99 & 1.00 \\
47437 & 211051 & 2001 & 377.30 & -0.28 & 37350.00 & 368590.37 & 1.01 & 0.98 & 0.99 \\
15398 & 101989 & 2001 & 322.60 & -0.11 & 29398.00 & 283497.27 & 1.10 & 0.88 & 0.96 \\
14986 & 101926 & 2001 & 390.90 & -0.05 & 42247.00 & 432475.30 & 0.93 & 1.11 & 1.02 \\
15338 & 101987 & 2001 & 1982.40 & -0.30 & 198238.00 & 1928231.32 & 1.00 & 0.97 & 0.97 \\
14999 & 101930 & 2001 & 1168.90 & -0.18 & 99522.00 & 1106767.83 & 1.17 & 0.95 & 1.11 \\
55735 & 400144 & 2001 & 138.00 & -0.10 & 11251.00 & 99186.00 & 1.23 & 0.72 & 0.88 \\
48327 & 240062 & 2001 & 940.10 & 0.04 & 80463.00 & 720550.15 & 1.17 & 0.77 & 0.90 \\
4016 & 100538 & 2001 & 606.50 & 0.16 & 57461.00 & 574492.72 & 1.06 & 0.95 & 1.00 \\
19735 & 102650 & 2001 & 10436.40 & 0.02 & 1037165.00 & 9610057.64 & 1.01 & 0.92 & 0.93 \\
43741 & 109217 & 2001 & 32.90 & -0.14 & 3286.00 & 32862.60 & 1.00 & 1.00 & 1.00 \\
48338 & 240063 & 2001 & 821.50 & -0.54 & 66186.00 & 831226.28 & 1.24 & 1.01 & 1.26 \\
47299 & 200503 & 2001 & 30.70 & -0.12 & 2770.00 & 29214.47 & 1.11 & 0.95 & 1.05 \\
15328 & 101984 & 2001 & 195.40 & -0.21 & 19005.00 & 195115.29 & 1.03 & 1.00 & 1.03 \\
6506 & 100878 & 2001 & 2552.80 & -0.06 & 255158.00 & 2469736.70 & 1.00 & 0.97 & 0.97 \\
19701 & 102649 & 2001 & 692.10 & -0.04 & 72591.00 & 632345.34 & 0.95 & 0.91 & 0.87 \\
13704 & 101758 & 2001 & 646.40 & -0.09 & 56668.00 & 576289.44 & 1.14 & 0.89 & 1.02 \\
15010 & 101933 & 2001 & 194.60 & -0.20 & 15251.00 & 188468.36 & 1.28 & 0.97 & 1.24 \\
15237 & 101970 & 2001 & 63.60 & 0.03 & 5637.00 & 59334.95 & 1.13 & 0.93 & 1.05 \\
6638 & 100906 & 2001 & 2009.00 & -0.07 & 164023.00 & 2038528.05 & 1.22 & 1.01 & 1.24 \\
19637 & 102639 & 2001 & 80.30 & 0.10 & 7964.00 & 78685.06 & 1.01 & 0.98 & 0.99 \\
60781 & 410724 & 2001 & 98.70 & -0.43 & 10172.00 & 89890.25 & 0.97 & 0.91 & 0.88 \\
6551 & 100890 & 2001 & 1340.00 & 0.00 & 134263.00 & 1285774.10 & 1.00 & 0.96 & 0.96 \\
6332 & 100849 & 2001 & 82.00 & -0.24 & 9487.00 & 80934.71 & 0.86 & 0.99 & 0.85 \\
6410 & 100864 & 2001 & 425.20 & -0.20 & 42014.00 & 384446.23 & 1.01 & 0.90 & 0.92 \\
6457 & 100875 & 2001 & 115.40 & 0.14 & 10480.00 & 118947.22 & 1.10 & 1.03 & 1.13 \\
47337 & 210203 & 2001 & 6792.20 & -0.18 & 679143.00 & 6724043.18 & 1.00 & 0.99 & 0.99 \\
6594 & 100900 & 2001 & 60.50 & 0.02 & 4987.00 & 42200.22 & 1.21 & 0.70 & 0.85 \\
60783 & 410725 & 2001 & 6.90 & -0.06 & 645.00 & 7224.51 & 1.07 & 1.05 & 1.12 \\
15147 & 101963 & 2001 & 896.60 & -0.02 & 87968.00 & 804580.69 & 1.02 & 0.90 & 0.91 \\
54361 & 367231 & 2002 & 12.70 & -0.11 & 1273.00 & 12531.47 & 1.00 & 0.99 & 0.98 \\
38319 & 107243 & 2002 & 987.20 & -0.20 & 134900.00 & 920091.01 & 0.73 & 0.93 & 0.68 \\
44416 & 109305 & 2002 & 362.60 & -0.23 & 36037.00 & 360365.94 & 1.01 & 0.99 & 1.00 \\
54364 & 367500 & 2002 & 71.10 & -0.18 & 9682.00 & 71031.94 & 0.73 & 1.00 & 0.73 \\
2811 & 100359 & 2002 & 183.10 & -0.34 & 18311.00 & 181848.69 & 1.00 & 0.99 & 0.99 \\
13037 & 101622 & 2002 & 831.40 & -0.32 & 86163.00 & 788383.27 & 0.96 & 0.95 & 0.91 \\
28874 & 105498 & 2002 & 68.70 & -0.20 & 6861.00 & 66556.64 & 1.00 & 0.97 & 0.97 \\
18882 & 102525 & 2002 & 735.30 & -0.15 & 73363.00 & 691627.10 & 1.00 & 0.94 & 0.94 \\
14651 & 101905 & 2002 & 8.50 & -0.01 & 774.00 & 7537.36 & 1.10 & 0.89 & 0.97 \\
49567 & 240314 & 2002 & 14.60 & -0.14 & 1457.00 & 14515.56 & 1.00 & 0.99 & 1.00 \\
28706 & 105469 & 2002 & 60.00 & -0.12 & 5402.00 & 57411.68 & 1.11 & 0.96 & 1.06 \\
13051 & 101623 & 2002 & 1154.30 & -0.26 & 115734.00 & 1141276.28 & 1.00 & 0.99 & 0.99 \\
50392 & 240414 & 2002 & 50.20 & -0.10 & 5223.00 & 45641.04 & 0.96 & 0.91 & 0.87 \\
40823 & 108155 & 2002 & 59.00 & -0.22 & 5893.00 & 58931.73 & 1.00 & 1.00 & 1.00 \\
5754 & 100791 & 2002 & 3665.10 & -0.19 & 419073.00 & 3522250.38 & 0.87 & 0.96 & 0.84 \\
19659 & 102641 & 2002 & 337.00 & -0.30 & 51046.00 & 351773.85 & 0.66 & 1.04 & 0.69 \\
38877 & 107339 & 2002 & 43.30 & -0.31 & 4204.00 & 42035.47 & 1.03 & 0.97 & 1.00 \\
33098 & 106091 & 2002 & 128.10 & -0.22 & 12848.00 & 124422.10 & 1.00 & 0.97 & 0.97 \\
25925 & 103529 & 2002 & 5154.50 & -0.22 & 744675.00 & 5070541.58 & 0.69 & 0.98 & 0.68 \\
33312 & 106114 & 2002 & 108.50 & -0.14 & 9162.00 & 94337.49 & 1.18 & 0.87 & 1.03 \\
10986 & 101358 & 2002 & 311.00 & -0.30 & 31258.00 & 302607.36 & 0.99 & 0.97 & 0.97 \\
41840 & 108860 & 2002 & 30.00 & -0.46 & 3039.00 & 29426.48 & 0.99 & 0.98 & 0.97 \\
55447 & 400099 & 2002 & 2.50 & -0.15 & 251.00 & 2499.49 & 1.00 & 1.00 & 1.00 \\
53983 & 362981 & 2002 & 236.70 & -0.22 & 23145.00 & 238156.68 & 1.02 & 1.01 & 1.03 \\
4705 & 100667 & 2002 & 12.20 & -0.17 & 1234.00 & 11165.50 & 0.99 & 0.92 & 0.90 \\
15254 & 101972 & 2002 & 2015.80 & -0.22 & 201590.00 & 1850483.46 & 1.00 & 0.92 & 0.92 \\
16542 & 102154 & 2002 & 182.70 & -0.13 & 22428.00 & 170162.33 & 0.81 & 0.93 & 0.76 \\
26555 & 103591 & 2002 & 1078.70 & -0.23 & 107937.00 & 1027109.77 & 1.00 & 0.95 & 0.95 \\
54111 & 364518 & 2002 & 50.50 & -0.36 & 6184.00 & 57579.56 & 0.82 & 1.14 & 0.93 \\
44414 & 109301 & 2002 & 62.30 & -0.27 & 6193.00 & 61928.32 & 1.01 & 0.99 & 1.00 \\
38883 & 107350 & 2002 & 2791.90 & -0.21 & 288221.00 & 2495670.92 & 0.97 & 0.89 & 0.87 \\
41860 & 108866 & 2002 & 205.60 & -0.21 & 21034.00 & 202474.93 & 0.98 & 0.98 & 0.96 \\
24057 & 103259 & 2002 & 2101.70 & -0.13 & 254115.00 & 1961350.03 & 0.83 & 0.93 & 0.77 \\
5730 & 100790 & 2002 & 181.40 & -0.16 & 21102.00 & 168628.47 & 0.86 & 0.93 & 0.80 \\
27134 & 105246 & 2002 & 4077.00 & -0.12 & 417296.00 & 3775514.89 & 0.98 & 0.93 & 0.90 \\
41854 & 108861 & 2002 & 62.30 & -0.27 & 5751.00 & 62241.10 & 1.08 & 1.00 & 1.08 \\
28884 & 105502 & 2002 & 1188.40 & -0.22 & 118666.00 & 1004104.98 & 1.00 & 0.84 & 0.85 \\
33339 & 106116 & 2002 & 4.50 & -0.41 & 345.00 & 4269.44 & 1.30 & 0.95 & 1.24 \\
49424 & 240296 & 2002 & 659.40 & 0.03 & 68689.00 & 621122.57 & 0.96 & 0.94 & 0.90 \\
41948 & 108874 & 2002 & 30.00 & -0.28 & 3002.00 & 28987.63 & 1.00 & 0.97 & 0.97 \\
22247 & 102997 & 2002 & 3283.10 & -0.37 & 331287.00 & 2914514.69 & 0.99 & 0.89 & 0.88 \\
46323 & 200213 & 2002 & 7.00 & -0.19 & 728.00 & 6953.25 & 0.96 & 0.99 & 0.96 \\
44749 & 109371 & 2002 & 277.30 & -0.19 & 28030.00 & 268109.22 & 0.99 & 0.97 & 0.96 \\
13619 & 101748 & 2002 & 1050.80 & -0.11 & 105965.00 & 1038049.85 & 0.99 & 0.99 & 0.98 \\
45477 & 200065 & 2002 & 7.30 & -0.13 & 706.00 & 6901.98 & 1.03 & 0.95 & 0.98 \\
28700 & 105465 & 2002 & 31.50 & -0.14 & 3234.00 & 31107.08 & 0.97 & 0.99 & 0.96 \\
41885 & 108867 & 2002 & 470.80 & -0.33 & 48350.00 & 468567.30 & 0.97 & 1.00 & 0.97 \\
74797 & 601172 & 2002 & 205.20 & 0.02 & 20316.00 & 192402.79 & 1.01 & 0.94 & 0.95 \\
44368 & 109295 & 2002 & 53.60 & -0.10 & 5990.00 & 50807.55 & 0.89 & 0.95 & 0.85 \\
2798 & 100358 & 2002 & 432.20 & -0.40 & 43216.00 & 431680.23 & 1.00 & 1.00 & 1.00 \\
24100 & 103266 & 2002 & 171.00 & -0.51 & 17126.00 & 168595.81 & 1.00 & 0.99 & 0.98 \\
46351 & 200225 & 2002 & 3.30 & -0.35 & 295.00 & 2966.51 & 1.12 & 0.90 & 1.01 \\
38907 & 107352 & 2002 & 154.90 & -0.20 & 15644.00 & 138575.79 & 0.99 & 0.89 & 0.89 \\
9585 & 101151 & 2002 & 85.30 & -0.23 & 9644.00 & 74213.08 & 0.88 & 0.87 & 0.77 \\
36300 & 106482 & 2002 & 134.70 & 0.05 & 13214.00 & 132140.11 & 1.02 & 0.98 & 1.00 \\
48143 & 240027 & 2002 & 308.80 & 0.04 & 32438.00 & 294449.63 & 0.95 & 0.95 & 0.91 \\
46343 & 200224 & 2002 & 23.00 & -0.09 & 3402.00 & 28523.91 & 0.68 & 1.24 & 0.84 \\
17622 & 102321 & 2002 & 107.50 & -0.24 & 10394.00 & 103970.33 & 1.03 & 0.97 & 1.00 \\
6793 & 100954 & 2002 & 470.60 & -0.17 & 47249.00 & 435852.68 & 1.00 & 0.93 & 0.92 \\
28809 & 105478 & 2002 & 38.40 & -0.25 & 3839.00 & 38385.34 & 1.00 & 1.00 & 1.00 \\
55445 & 400097 & 2002 & 10.90 & -0.05 & 1104.00 & 10892.30 & 0.99 & 1.00 & 0.99 \\
16497 & 102151 & 2002 & 5.50 & -0.27 & 579.00 & 5125.26 & 0.95 & 0.93 & 0.89 \\
36366 & 106519 & 2002 & 20.80 & -0.24 & 2082.00 & 20803.66 & 1.00 & 1.00 & 1.00 \\
64420 & 500602 & 2002 & 88.70 & -0.15 & 6808.00 & 69642.82 & 1.30 & 0.79 & 1.02 \\
38924 & 107354 & 2002 & 29.60 & -0.12 & 3146.00 & 28111.45 & 0.94 & 0.95 & 0.89 \\
59142 & 410439 & 2002 & 12.30 & -0.17 & 1432.00 & 9991.16 & 0.86 & 0.81 & 0.70 \\
14661 & 101906 & 2002 & 7.00 & -0.33 & 701.00 & 6887.46 & 1.00 & 0.98 & 0.98 \\
36326 & 106483 & 2002 & 6.00 & -0.11 & 566.00 & 5659.52 & 1.06 & 0.94 & 1.00 \\
55811 & 400153 & 2002 & 11.30 & -0.11 & 1231.00 & 11371.13 & 0.92 & 1.01 & 0.92 \\
54088 & 364393 & 2002 & 59.90 & -0.24 & 6077.00 & 58161.03 & 0.99 & 0.97 & 0.96 \\
36336 & 106485 & 2002 & 66.00 & -0.17 & 6353.00 & 63515.46 & 1.04 & 0.96 & 1.00 \\
25896 & 103526 & 2002 & 2839.40 & -0.23 & 378190.00 & 2656923.84 & 0.75 & 0.94 & 0.70 \\
55248 & 400075 & 2002 & 915.90 & -0.19 & 92324.00 & 923252.01 & 0.99 & 1.01 & 1.00 \\
24088 & 103264 & 2002 & 556.70 & -0.28 & 57015.00 & 567332.89 & 0.98 & 1.02 & 1.00 \\
41891 & 108868 & 2002 & 762.30 & -0.19 & 80761.00 & 738075.61 & 0.94 & 0.97 & 0.91 \\
13074 & 101626 & 2002 & 1176.80 & -0.22 & 119256.00 & 1031283.23 & 0.99 & 0.88 & 0.86 \\
4306 & 100603 & 2002 & 1258.00 & 0.03 & 159304.00 & 1359230.35 & 0.79 & 1.08 & 0.85 \\
33350 & 106123 & 2002 & 255.40 & -0.25 & 25513.00 & 249042.54 & 1.00 & 0.98 & 0.98 \\
36352 & 106487 & 2002 & 10.60 & -0.08 & 1060.00 & 10598.73 & 1.00 & 1.00 & 1.00 \\
74616 & 601140 & 2002 & 13.20 & -0.13 & 1256.00 & 12283.43 & 1.05 & 0.93 & 0.98 \\
7048 & 100992 & 2002 & 594.10 & -0.27 & 59261.00 & 580650.58 & 1.00 & 0.98 & 0.98 \\
7592 & 101047 & 2002 & 207.50 & -0.17 & 23583.00 & 211318.44 & 0.88 & 1.02 & 0.90 \\
41921 & 108870 & 2002 & 24.90 & -0.13 & 2265.00 & 24160.84 & 1.10 & 0.97 & 1.07 \\
22188 & 102994 & 2002 & 60.20 & -0.18 & 7320.00 & 59342.82 & 0.82 & 0.99 & 0.81 \\
36441 & 106528 & 2002 & 261.40 & -0.40 & 26172.00 & 233753.43 & 1.00 & 0.89 & 0.89 \\
16519 & 102152 & 2002 & 201.20 & -0.23 & 25997.00 & 194600.53 & 0.77 & 0.97 & 0.75 \\
1075 & 100150 & 2002 & 17.80 & -0.16 & 1545.00 & 15451.81 & 1.15 & 0.87 & 1.00 \\
11958 & 101473 & 2002 & 1301.10 & -0.22 & 165480.00 & 1436171.57 & 0.79 & 1.10 & 0.87 \\
48880 & 240149 & 2002 & 100.70 & 0.06 & 10014.00 & 99965.98 & 1.01 & 0.99 & 1.00 \\
2820 & 100360 & 2002 & 290.50 & -0.48 & 32679.00 & 305987.55 & 0.89 & 1.05 & 0.94 \\
44400 & 109300 & 2002 & 356.20 & -0.21 & 35654.00 & 347622.24 & 1.00 & 0.98 & 0.97 \\
32722 & 106052 & 2002 & 103.00 & -0.24 & 10091.00 & 100416.78 & 1.02 & 0.97 & 1.00 \\
74808 & 601178 & 2002 & 27.60 & -0.18 & 2610.00 & 26103.49 & 1.06 & 0.95 & 1.00 \\
22144 & 102993 & 2002 & 4353.10 & -0.17 & 555858.00 & 3633127.00 & 0.78 & 0.83 & 0.65 \\
40344 & 108112 & 2002 & 21.00 & 0.02 & 2084.00 & 20256.06 & 1.01 & 0.96 & 0.97 \\
20079 & 102665 & 2002 & 12.90 & -0.12 & 1301.00 & 11455.81 & 0.99 & 0.89 & 0.88 \\
22219 & 102996 & 2002 & 521.80 & -0.34 & 53118.00 & 506472.78 & 0.98 & 0.97 & 0.95 \\
28749 & 105475 & 2002 & 598.90 & -0.17 & 61102.00 & 583474.41 & 0.98 & 0.97 & 0.95 \\
2831 & 100362 & 2002 & 33.50 & -0.07 & 4397.00 & 36553.26 & 0.76 & 1.09 & 0.83 \\
36415 & 106524 & 2002 & 10.80 & 0.11 & 1346.00 & 13940.48 & 0.80 & 1.29 & 1.04 \\
47454 & 211210 & 2002 & 35.10 & -0.37 & 4930.00 & 36291.61 & 0.71 & 1.03 & 0.74 \\
11026 & 101360 & 2002 & 1502.90 & -0.14 & 152678.00 & 1400931.02 & 0.98 & 0.93 & 0.92 \\
3974 & 100535 & 2002 & 273.30 & -0.13 & 26760.00 & 251962.29 & 1.02 & 0.92 & 0.94 \\
48349 & 240065 & 2002 & 627.10 & -0.25 & 52899.00 & 551424.41 & 1.19 & 0.88 & 1.04 \\
5000 & 100698 & 2002 & 34.40 & -0.17 & 3189.00 & 31371.15 & 1.08 & 0.91 & 0.98 \\
43311 & 109093 & 2002 & 15.30 & -0.35 & 1522.00 & 14661.41 & 1.01 & 0.96 & 0.96 \\
18913 & 102527 & 2002 & 104.80 & -0.22 & 10240.00 & 108030.87 & 1.02 & 1.03 & 1.05 \\
28778 & 105476 & 2002 & 145.50 & 0.07 & 13822.00 & 137576.96 & 1.05 & 0.95 & 1.00 \\
28856 & 105487 & 2002 & 142.80 & -0.35 & 10855.00 & 103953.74 & 1.32 & 0.73 & 0.96 \\
13886 & 101785 & 2002 & 1234.10 & -0.39 & 183682.00 & 1291234.48 & 0.67 & 1.05 & 0.70 \\
63230 & 500490 & 2002 & 288.80 & -0.15 & 46240.00 & 254263.21 & 0.62 & 0.88 & 0.55 \\
23311 & 103160 & 2002 & 90.10 & -0.25 & 9128.00 & 86096.18 & 0.99 & 0.96 & 0.94 \\
46335 & 200223 & 2002 & 12.10 & -0.43 & 1206.00 & 11089.64 & 1.00 & 0.92 & 0.92 \\
27118 & 103658 & 2002 & 97.40 & -0.22 & 9739.00 & 95414.26 & 1.00 & 0.98 & 0.98 \\
40318 & 108109 & 2002 & 41.90 & -0.02 & 3518.00 & 31764.52 & 1.19 & 0.76 & 0.90 \\
36436 & 106527 & 2002 & 14.10 & -0.33 & 1390.00 & 13599.43 & 1.01 & 0.96 & 0.98 \\
43474 & 109125 & 2002 & 134.60 & -0.32 & 14381.00 & 147632.58 & 0.94 & 1.10 & 1.03 \\
33285 & 106113 & 2002 & 362.50 & -0.23 & 36944.00 & 354969.47 & 0.98 & 0.98 & 0.96 \\
40430 & 108119 & 2002 & 167.70 & -0.21 & 16747.00 & 163294.26 & 1.00 & 0.97 & 0.98 \\
41752 & 108852 & 2002 & 92.50 & -0.20 & 5189.00 & 47447.13 & 1.78 & 0.51 & 0.91 \\
6552 & 100890 & 2002 & 1036.40 & -0.23 & 102792.00 & 964538.45 & 1.01 & 0.93 & 0.94 \\
419 & 100055 & 2002 & 10944.30 & -0.19 & 985463.00 & 10512047.14 & 1.11 & 0.96 & 1.07 \\
24005 & 103253 & 2002 & 148.10 & -0.08 & 14379.00 & 143788.89 & 1.03 & 0.97 & 1.00 \\
36556 & 106561 & 2002 & 10.20 & 0.04 & 964.00 & 8990.53 & 1.06 & 0.88 & 0.93 \\
43640 & 109176 & 2002 & 9.70 & -0.30 & 1070.00 & 10575.32 & 0.91 & 1.09 & 0.99 \\
44497 & 109330 & 2002 & 13.40 & -0.27 & 1344.00 & 13211.97 & 1.00 & 0.99 & 0.98 \\
54134 & 364519 & 2002 & 57.30 & -0.16 & 8201.00 & 59361.48 & 0.70 & 1.04 & 0.72 \\
63351 & 500500 & 2002 & 186.50 & -0.25 & 17829.00 & 178728.74 & 1.05 & 0.96 & 1.00 \\
28490 & 105427 & 2002 & 152.70 & -0.26 & 15152.00 & 140306.53 & 1.01 & 0.92 & 0.93 \\
64236 & 500593 & 2002 & 607.00 & -0.12 & 48616.00 & 557952.86 & 1.25 & 0.92 & 1.15 \\
28519 & 105432 & 2002 & 12.80 & -0.15 & 1103.00 & 11846.84 & 1.16 & 0.93 & 1.07 \\
58046 & 410060 & 2002 & 5.60 & -0.27 & 588.00 & 5841.66 & 0.95 & 1.04 & 0.99 \\
36537 & 106560 & 2002 & 80.70 & -0.22 & 7458.00 & 79224.96 & 1.08 & 0.98 & 1.06 \\
22379 & 103008 & 2002 & 125.80 & -0.25 & 12217.00 & 122167.40 & 1.03 & 0.97 & 1.00 \\
28541 & 105437 & 2002 & 1297.90 & -0.20 & 130442.00 & 1187482.11 & 1.00 & 0.91 & 0.91 \\
38859 & 107337 & 2002 & 19.10 & -0.28 & 1895.00 & 18950.72 & 1.01 & 0.99 & 1.00 \\
43358 & 109100 & 2002 & 4.60 & -0.16 & 459.00 & 4377.43 & 1.00 & 0.95 & 0.95 \\
17585 & 102319 & 2002 & 416.60 & -0.20 & 40289.00 & 406766.55 & 1.03 & 0.98 & 1.01 \\
14245 & 101835 & 2002 & 1128.40 & -0.22 & 112396.00 & 1079041.59 & 1.00 & 0.96 & 0.96 \\
6837 & 100962 & 2002 & 3839.90 & -0.25 & 499264.00 & 3449527.11 & 0.77 & 0.90 & 0.69 \\
1025 & 100127 & 2002 & 2760.80 & -0.43 & 275503.00 & 2518264.05 & 1.00 & 0.91 & 0.91 \\
18401 & 102447 & 2002 & 3439.50 & -0.14 & 344425.00 & 3117776.64 & 1.00 & 0.91 & 0.91 \\
11124 & 101368 & 2002 & 1027.60 & -0.16 & 105614.00 & 992889.53 & 0.97 & 0.97 & 0.94 \\
45625 & 200086 & 2002 & 3.50 & -0.23 & 257.00 & 2132.83 & 1.36 & 0.61 & 0.83 \\
44470 & 109326 & 2002 & 3.50 & -0.84 & 432.00 & 3438.11 & 0.81 & 0.98 & 0.80 \\
14631 & 101903 & 2002 & 89.40 & -0.32 & 8953.00 & 88984.61 & 1.00 & 1.00 & 0.99 \\
45511 & 200072 & 2002 & 3.20 & -0.31 & 322.00 & 2928.80 & 0.99 & 0.92 & 0.91 \\
36620 & 106569 & 2002 & 84.30 & -0.46 & 8432.00 & 81567.44 & 1.00 & 0.97 & 0.97 \\
53432 & 349198 & 2002 & 380.50 & -0.28 & 33469.00 & 344237.88 & 1.14 & 0.90 & 1.03 \\
52297 & 302760 & 2002 & 490.90 & -0.16 & 49115.00 & 416037.15 & 1.00 & 0.85 & 0.85 \\
53608 & 354931 & 2002 & 1.60 & -0.07 & 127.00 & 1269.89 & 1.26 & 0.79 & 1.00 \\
28432 & 105424 & 2002 & 3398.00 & -0.29 & 340243.00 & 3188288.40 & 1.00 & 0.94 & 0.94 \\
38825 & 107331 & 2002 & 7.80 & -0.19 & 779.00 & 7450.47 & 1.00 & 0.96 & 0.96 \\
11160 & 101369 & 2002 & 1680.10 & -0.34 & 168121.00 & 1654341.86 & 1.00 & 0.98 & 0.98 \\
45537 & 200073 & 2002 & 315.40 & -0.25 & 27302.00 & 290084.05 & 1.16 & 0.92 & 1.06 \\
50161 & 240398 & 2002 & 615.40 & 0.03 & 59093.00 & 575786.53 & 1.04 & 0.94 & 0.97 \\
36629 & 106571 & 2002 & 60.80 & -0.14 & 6142.00 & 53660.08 & 0.99 & 0.88 & 0.87 \\
41727 & 108849 & 2002 & 630.70 & -0.14 & 63407.00 & 596110.77 & 0.99 & 0.95 & 0.94 \\
23984 & 103252 & 2002 & 229.80 & -0.14 & 22576.00 & 225757.19 & 1.02 & 0.98 & 1.00 \\
37746 & 107137 & 2002 & 35.40 & -0.22 & 3546.00 & 35461.56 & 1.00 & 1.00 & 1.00 \\
16650 & 102173 & 2002 & 29.60 & -0.23 & 2910.00 & 29104.65 & 1.02 & 0.98 & 1.00 \\
15238 & 101970 & 2002 & 47.50 & -0.24 & 4749.00 & 46865.76 & 1.00 & 0.99 & 0.99 \\
60782 & 410724 & 2002 & 45.00 & -0.41 & 7925.00 & 36717.19 & 0.57 & 0.82 & 0.46 \\
20051 & 102664 & 2002 & 2553.70 & -0.02 & 254113.00 & 2376440.44 & 1.00 & 0.93 & 0.94 \\
53714 & 356500 & 2002 & 270.90 & -0.34 & 26813.00 & 260778.91 & 1.01 & 0.96 & 0.97 \\
23284 & 103154 & 2002 & 300.50 & -0.30 & 30115.00 & 299552.25 & 1.00 & 1.00 & 0.99 \\
36587 & 106567 & 2002 & 1.80 & -0.40 & 186.00 & 1719.38 & 0.97 & 0.96 & 0.92 \\
26031 & 103535 & 2002 & 2001.10 & -0.23 & 258646.00 & 2007599.18 & 0.77 & 1.00 & 0.78 \\
37751 & 107141 & 2002 & 769.90 & -0.26 & 76729.00 & 734779.06 & 1.00 & 0.95 & 0.96 \\
13002 & 101618 & 2002 & 134.80 & -0.14 & 13221.00 & 137247.06 & 1.02 & 1.02 & 1.04 \\
63369 & 500502 & 2002 & 21.20 & -0.29 & 2599.00 & 20392.39 & 0.82 & 0.96 & 0.78 \\
49434 & 240297 & 2002 & 564.80 & -0.20 & 72501.00 & 521435.17 & 0.78 & 0.92 & 0.72 \\
22415 & 103011 & 2002 & 48.70 & -0.21 & 6139.00 & 47196.56 & 0.79 & 0.97 & 0.77 \\
28461 & 105426 & 2002 & 696.60 & -0.12 & 69705.00 & 687817.12 & 1.00 & 0.99 & 0.99 \\
38846 & 107336 & 2002 & 260.80 & -0.27 & 26146.00 & 255600.16 & 1.00 & 0.98 & 0.98 \\
14607 & 101902 & 2002 & 174.70 & -0.20 & 17444.00 & 147884.01 & 1.00 & 0.85 & 0.85 \\
36594 & 106568 & 2002 & 121.60 & 0.03 & 7443.00 & 73510.37 & 1.63 & 0.60 & 0.99 \\
19842 & 102653 & 2002 & 3729.00 & -0.23 & 369539.00 & 3309674.40 & 1.01 & 0.89 & 0.90 \\
18820 & 102523 & 2002 & 727.90 & -0.10 & 73224.00 & 661197.03 & 0.99 & 0.91 & 0.90 \\
38344 & 107244 & 2002 & 263.40 & -0.29 & 26457.00 & 253583.93 & 1.00 & 0.96 & 0.96 \\
44418 & 109307 & 2002 & 29.40 & -0.17 & 2958.00 & 25857.51 & 0.99 & 0.88 & 0.87 \\
32716 & 106051 & 2002 & 46.80 & -0.22 & 4673.00 & 44929.74 & 1.00 & 0.96 & 0.96 \\
13016 & 101621 & 2002 & 2277.40 & -0.22 & 229948.00 & 2018576.41 & 0.99 & 0.89 & 0.88 \\
63253 & 500491 & 2002 & 451.10 & -0.16 & 58141.00 & 429736.37 & 0.78 & 0.95 & 0.74 \\
46318 & 200211 & 2002 & 5.10 & -0.46 & 489.00 & 4119.75 & 1.04 & 0.81 & 0.84 \\
36471 & 106535 & 2002 & 211.80 & -0.16 & 16945.00 & 164063.96 & 1.25 & 0.77 & 0.97 \\
53771 & 357053 & 2002 & 45.70 & -0.22 & 4572.00 & 45709.53 & 1.00 & 1.00 & 1.00 \\
44447 & 109325 & 2002 & 178.30 & -0.23 & 11636.00 & 161415.92 & 1.53 & 0.91 & 1.39 \\
22299 & 103005 & 2002 & 402.40 & -0.15 & 34666.00 & 346619.03 & 1.16 & 0.86 & 1.00 \\
28637 & 105457 & 2002 & 1628.20 & -0.28 & 162745.00 & 1612420.04 & 1.00 & 0.99 & 0.99 \\
53408 & 348766 & 2002 & 422.90 & -0.28 & 35696.00 & 356973.32 & 1.18 & 0.84 & 1.00 \\
25963 & 103531 & 2002 & 1408.20 & -0.01 & 136248.00 & 1345339.24 & 1.03 & 0.96 & 0.99 \\
46303 & 200210 & 2002 & 6.70 & -0.17 & 660.00 & 6237.99 & 1.02 & 0.93 & 0.95 \\
23299 & 103158 & 2002 & 1399.80 & -0.27 & 140057.00 & 1363238.36 & 1.00 & 0.97 & 0.97 \\
11090 & 101367 & 2002 & 453.10 & -0.05 & 45388.00 & 411777.98 & 1.00 & 0.91 & 0.91 \\
65617 & 500708 & 2002 & 96.40 & -0.10 & 5591.00 & 78674.02 & 1.72 & 0.82 & 1.41 \\
41826 & 108858 & 2002 & 36.00 & -0.32 & 3504.00 & 35172.59 & 1.03 & 0.98 & 1.00 \\
49565 & 240313 & 2002 & 8.90 & -0.23 & 829.00 & 8641.22 & 1.07 & 0.97 & 1.04 \\
9415 & 101134 & 2002 & 129.80 & 0.59 & 11764.00 & 103118.30 & 1.10 & 0.79 & 0.88 \\
5293 & 100746 & 2002 & 1005.20 & -0.16 & 100511.00 & 996607.15 & 1.00 & 0.99 & 0.99 \\
50327 & 240411 & 2002 & 15.60 & -0.19 & 1867.00 & 14748.38 & 0.84 & 0.95 & 0.79 \\
36450 & 106529 & 2002 & 149.90 & -0.22 & 14986.00 & 148324.23 & 1.00 & 0.99 & 0.99 \\
14230 & 101834 & 2002 & 236.70 & -0.13 & 20469.00 & 200522.44 & 1.16 & 0.85 & 0.98 \\
2336 & 100319 & 2002 & 186.70 & -0.03 & 18648.00 & 172961.13 & 1.00 & 0.93 & 0.93 \\
43327 & 109095 & 2002 & 29.40 & -0.12 & 2949.00 & 28897.87 & 1.00 & 0.98 & 0.98 \\
44441 & 109321 & 2002 & 215.00 & -0.33 & 21491.00 & 202040.45 & 1.00 & 0.94 & 0.94 \\
96781 & 611013 & 2002 & 62.30 & -0.22 & 6213.00 & 61160.60 & 1.00 & 0.98 & 0.98 \\
22278 & 102999 & 2002 & 104.40 & -0.28 & 10695.00 & 105175.34 & 0.98 & 1.01 & 0.98 \\
37802 & 107145 & 2002 & 38.50 & 0.06 & 3935.00 & 38194.48 & 0.98 & 0.99 & 0.97 \\
2037 & 100286 & 2002 & 33.70 & -0.22 & 3356.00 & 32781.94 & 1.00 & 0.97 & 0.98 \\
7612 & 101048 & 2002 & 8212.60 & -0.31 & 1127687.00 & 8514237.62 & 0.73 & 1.04 & 0.76 \\
28652 & 105458 & 2002 & 313.40 & -0.30 & 31220.00 & 306197.84 & 1.00 & 0.98 & 0.98 \\
47302 & 200505 & 2002 & 127.30 & -0.23 & 12714.00 & 124332.74 & 1.00 & 0.98 & 0.98 \\
41833 & 108859 & 2002 & 12.20 & -0.51 & 1198.00 & 11976.39 & 1.02 & 0.98 & 1.00 \\
36465 & 106533 & 2002 & 15.30 & -0.27 & 2042.00 & 20422.43 & 0.75 & 1.33 & 1.00 \\
8569 & 101090 & 2002 & 571.40 & -0.42 & 97191.00 & 504987.78 & 0.59 & 0.88 & 0.52 \\
37779 & 107144 & 2002 & 58.00 & -0.44 & 6218.00 & 50494.90 & 0.93 & 0.87 & 0.81 \\
36499 & 106541 & 2002 & 413.30 & -0.21 & 45027.00 & 458351.62 & 0.92 & 1.11 & 1.02 \\
40396 & 108117 & 2002 & 475.90 & -0.42 & 47588.00 & 404576.95 & 1.00 & 0.85 & 0.85 \\
36531 & 106557 & 2002 & 54.20 & -0.14 & 5430.00 & 53389.01 & 1.00 & 0.99 & 0.98 \\
48376 & 240067 & 2002 & 269.50 & -0.20 & 27131.00 & 259562.93 & 0.99 & 0.96 & 0.96 \\
28571 & 105444 & 2002 & 63.70 & -0.20 & 7677.00 & 58647.97 & 0.83 & 0.92 & 0.76 \\
25985 & 103532 & 2002 & 527.20 & 0.33 & 45072.00 & 372003.01 & 1.17 & 0.71 & 0.83 \\
41766 & 108855 & 2002 & 245.00 & 0.04 & 36404.00 & 210435.68 & 0.67 & 0.86 & 0.58 \\
54386 & 367567 & 2002 & 198.70 & -0.05 & 28669.00 & 186355.91 & 0.69 & 0.94 & 0.65 \\
59110 & 410433 & 2002 & 926.50 & -0.19 & 92667.00 & 921220.87 & 1.00 & 0.99 & 0.99 \\
50237 & 240406 & 2002 & 5.40 & -0.14 & 1018.00 & 8462.28 & 0.53 & 1.57 & 0.83 \\
57922 & 402013 & 2002 & 28.40 & -0.51 & 3094.00 & 28294.81 & 0.92 & 1.00 & 0.91 \\
74777 & 601168 & 2002 & 27.80 & -0.34 & 2944.00 & 30179.41 & 0.94 & 1.09 & 1.03 \\
38867 & 107338 & 2002 & 43.20 & -0.28 & 3833.00 & 37242.83 & 1.13 & 0.86 & 0.97 \\
41758 & 108853 & 2002 & 225.30 & -0.17 & 22575.00 & 204176.75 & 1.00 & 0.91 & 0.90 \\
52290 & 302732 & 2002 & 190.20 & -0.31 & 18527.00 & 169582.11 & 1.03 & 0.89 & 0.92 \\
45508 & 200071 & 2002 & 231.00 & -0.31 & 22324.00 & 223208.45 & 1.03 & 0.97 & 1.00 \\
392 & 100048 & 2002 & 230.90 & -0.26 & 23084.00 & 227551.00 & 1.00 & 0.99 & 0.99 \\
22335 & 103007 & 2002 & 934.20 & -0.18 & 87132.00 & 841467.67 & 1.07 & 0.90 & 0.97 \\
24035 & 103255 & 2002 & 100.90 & -0.33 & 9678.00 & 96781.30 & 1.04 & 0.96 & 1.00 \\
40818 & 108154 & 2002 & 31.20 & -0.27 & 3076.00 & 31133.28 & 1.01 & 1.00 & 1.01 \\
41807 & 108857 & 2002 & 23.40 & 0.01 & 1829.00 & 18293.39 & 1.28 & 0.78 & 1.00 \\
1040 & 100128 & 2002 & 197.80 & -0.20 & 19814.00 & 158537.26 & 1.00 & 0.80 & 0.80 \\
36511 & 106545 & 2002 & 14.40 & -0.21 & 1443.00 & 12490.32 & 1.00 & 0.87 & 0.87 \\
44798 & 109389 & 2002 & 12.00 & -0.10 & 1196.00 & 11365.93 & 1.00 & 0.95 & 0.95 \\
41785 & 108856 & 2002 & 230.50 & -0.15 & 23020.00 & 216926.07 & 1.00 & 0.94 & 0.94 \\
55613 & 400131 & 2002 & 17.10 & -0.15 & 1696.00 & 16642.41 & 1.01 & 0.97 & 0.98 \\
14641 & 101904 & 2002 & 8.30 & -0.32 & 832.00 & 7465.54 & 1.00 & 0.90 & 0.90 \\
12549 & 101553 & 2002 & 175.10 & -0.22 & 17492.00 & 167436.25 & 1.00 & 0.96 & 0.96 \\
18851 & 102524 & 2002 & 2185.80 & -0.23 & 218209.00 & 2119236.38 & 1.00 & 0.97 & 0.97 \\
28587 & 105448 & 2002 & 238.90 & 0.08 & 24363.00 & 204246.15 & 0.98 & 0.85 & 0.84 \\
54598 & 377010 & 2002 & 29.30 & -0.16 & 2571.00 & 27623.26 & 1.14 & 0.94 & 1.07 \\
43335 & 109099 & 2002 & 25.40 & -0.20 & 2592.00 & 21111.79 & 0.98 & 0.83 & 0.81 \\
16593 & 102157 & 2002 & 4.20 & -0.22 & 403.00 & 3616.12 & 1.04 & 0.86 & 0.90 \\
19924 & 102655 & 2002 & 984.80 & -0.06 & 98373.00 & 922257.70 & 1.00 & 0.94 & 0.94 \\
45471 & 200061 & 2002 & 25.40 & -0.19 & 2540.00 & 24787.48 & 1.00 & 0.98 & 0.98 \\
43264 & 109088 & 2002 & 1267.00 & -0.24 & 127674.00 & 1168421.28 & 0.99 & 0.92 & 0.92 \\
55098 & 400061 & 2002 & 231.90 & -0.07 & 21973.00 & 224234.22 & 1.06 & 0.97 & 1.02 \\
29305 & 105585 & 2002 & 11.10 & -0.20 & 1153.00 & 10002.03 & 0.96 & 0.90 & 0.87 \\
49398 & 240291 & 2002 & 50.30 & -0.34 & 5036.00 & 50435.88 & 1.00 & 1.00 & 1.00 \\
13158 & 101681 & 2002 & 843.00 & -0.32 & 107804.00 & 793937.13 & 0.78 & 0.94 & 0.74 \\
5826 & 100804 & 2002 & 3143.40 & -0.17 & 313738.00 & 2941170.08 & 1.00 & 0.94 & 0.94 \\
17386 & 102284 & 2002 & 218.70 & -0.08 & 21869.00 & 210336.04 & 1.00 & 0.96 & 0.96 \\
10832 & 101334 & 2002 & 406.00 & -0.33 & 40666.00 & 348247.81 & 1.00 & 0.86 & 0.86 \\
50692 & 240440 & 2002 & 203.50 & -0.16 & 18991.00 & 189915.41 & 1.07 & 0.93 & 1.00 \\
35925 & 106434 & 2002 & 589.20 & 7.11 & 59268.00 & 570245.46 & 0.99 & 0.97 & 0.96 \\
2009 & 100280 & 2002 & 35.10 & -0.12 & 4103.00 & 40137.45 & 0.86 & 1.14 & 0.98 \\
25729 & 103520 & 2002 & 7519.10 & -0.24 & 1030307.00 & 7543342.53 & 0.73 & 1.00 & 0.73 \\
29279 & 105581 & 2002 & 175.00 & -0.40 & 17534.00 & 168684.52 & 1.00 & 0.96 & 0.96 \\
44857 & 109395 & 2002 & 51.80 & -0.21 & 4083.00 & 39169.71 & 1.27 & 0.76 & 0.96 \\
48728 & 240134 & 2002 & 129.10 & 0.02 & 13037.00 & 128118.89 & 0.99 & 0.99 & 0.98 \\
42136 & 108926 & 2002 & 87.90 & -0.23 & 8800.00 & 83286.32 & 1.00 & 0.95 & 0.95 \\
49076 & 240218 & 2002 & 213.00 & -0.41 & 21480.00 & 183914.98 & 0.99 & 0.86 & 0.86 \\
49692 & 240333 & 2002 & 23.70 & -0.08 & 2709.00 & 25801.75 & 0.87 & 1.09 & 0.95 \\
19025 & 102544 & 2002 & 621.70 & -0.24 & 70504.00 & 574094.01 & 0.88 & 0.92 & 0.81 \\
35862 & 106419 & 2002 & 1512.00 & -0.26 & 151130.00 & 1469038.81 & 1.00 & 0.97 & 0.97 \\
35867 & 106420 & 2002 & 42.50 & -0.21 & 4245.00 & 39665.83 & 1.00 & 0.93 & 0.93 \\
21856 & 102957 & 2002 & 504.40 & -0.30 & 83130.00 & 517139.42 & 0.61 & 1.03 & 0.62 \\
37865 & 107152 & 2002 & 57.50 & -0.21 & 6932.00 & 49803.77 & 0.83 & 0.87 & 0.72 \\
35875 & 106421 & 2002 & 20.70 & 0.05 & 2031.00 & 19772.86 & 1.02 & 0.96 & 0.97 \\
52387 & 302826 & 2002 & 200.30 & -0.28 & 18299.00 & 180229.86 & 1.09 & 0.90 & 0.98 \\
50678 & 240439 & 2002 & 27.10 & -0.35 & 2556.00 & 22309.67 & 1.06 & 0.82 & 0.87 \\
14199 & 101820 & 2002 & 191.70 & -0.26 & 25615.00 & 186364.72 & 0.75 & 0.97 & 0.73 \\
42151 & 108930 & 2002 & 124.40 & -0.22 & 12476.00 & 124430.02 & 1.00 & 1.00 & 1.00 \\
46429 & 200244 & 2002 & 4.60 & -0.55 & 834.00 & 4129.34 & 0.55 & 0.90 & 0.50 \\
50706 & 240441 & 2002 & 589.20 & -0.14 & 58885.00 & 555852.93 & 1.00 & 0.94 & 0.94 \\
35893 & 106422 & 2002 & 128.40 & -0.25 & 18373.00 & 130939.57 & 0.70 & 1.02 & 0.71 \\
39131 & 107608 & 2002 & 61.00 & -0.19 & 6139.00 & 60220.67 & 0.99 & 0.99 & 0.98 \\
35899 & 106424 & 2002 & 310.40 & -0.20 & 31027.00 & 299821.97 & 1.00 & 0.97 & 0.97 \\
42140 & 108929 & 2002 & 308.30 & -0.22 & 31287.00 & 301547.82 & 0.99 & 0.98 & 0.96 \\
29331 & 105587 & 2002 & 48.80 & -0.26 & 4890.00 & 48328.78 & 1.00 & 0.99 & 0.99 \\
32268 & 106008 & 2002 & 181.30 & -0.31 & 18287.00 & 169211.25 & 0.99 & 0.93 & 0.93 \\
35935 & 106441 & 2002 & 384.00 & -0.18 & 40240.00 & 386786.78 & 0.95 & 1.01 & 0.96 \\
44269 & 109281 & 2002 & 93.20 & -0.13 & 7170.00 & 73568.32 & 1.30 & 0.79 & 1.03 \\
292 & 100033 & 2002 & 272.20 & -0.46 & 30590.00 & 309432.74 & 0.89 & 1.14 & 1.01 \\
25761 & 103521 & 2002 & 3170.20 & -0.21 & 446518.00 & 3309804.36 & 0.71 & 1.04 & 0.74 \\
42117 & 108924 & 2002 & 143.80 & -0.02 & 14253.00 & 139021.93 & 1.01 & 0.97 & 0.98 \\
29200 & 105536 & 2002 & 240.50 & -0.34 & 24086.00 & 222554.44 & 1.00 & 0.93 & 0.92 \\
42112 & 108923 & 2002 & 17.90 & -0.22 & 1779.00 & 18292.90 & 1.01 & 1.02 & 1.03 \\
39084 & 107604 & 2002 & 235.20 & -0.27 & 23374.00 & 220800.37 & 1.01 & 0.94 & 0.94 \\
43710 & 109199 & 2002 & 19.20 & -0.83 & 2449.00 & 17144.03 & 0.78 & 0.89 & 0.70 \\
29184 & 105535 & 2002 & 174.50 & -0.29 & 17450.00 & 170006.25 & 1.00 & 0.97 & 0.97 \\
42129 & 108925 & 2002 & 481.80 & -0.23 & 50103.00 & 430314.65 & 0.96 & 0.89 & 0.86 \\
35987 & 106444 & 2002 & 131.20 & -0.21 & 13112.00 & 128418.83 & 1.00 & 0.98 & 0.98 \\
49408 & 240293 & 2002 & 1056.10 & -0.23 & 122919.00 & 908008.72 & 0.86 & 0.86 & 0.74 \\
40224 & 108074 & 2002 & 20.90 & -0.30 & 2100.00 & 20825.41 & 1.00 & 1.00 & 0.99 \\
2996 & 100395 & 2002 & 667.20 & -0.25 & 66765.00 & 637789.69 & 1.00 & 0.96 & 0.96 \\
35959 & 106442 & 2002 & 2700.50 & -0.26 & 380477.00 & 2679259.95 & 0.71 & 0.99 & 0.70 \\
54339 & 367206 & 2002 & 121.80 & -0.19 & 12187.00 & 116720.50 & 1.00 & 0.96 & 0.96 \\
39121 & 107607 & 2002 & 50.20 & -0.18 & 5031.00 & 48319.49 & 1.00 & 0.96 & 0.96 \\
29239 & 105561 & 2002 & 93.60 & 0.11 & 9357.00 & 87531.74 & 1.00 & 0.94 & 0.94 \\
835 & 100098 & 2002 & 219.80 & -0.29 & 21979.00 & 212657.85 & 1.00 & 0.97 & 0.97 \\
29266 & 105574 & 2002 & 172.50 & -0.06 & 16888.00 & 173359.03 & 1.02 & 1.00 & 1.03 \\
23659 & 103205 & 2002 & 44.90 & -0.10 & 4050.00 & 40494.54 & 1.11 & 0.90 & 1.00 \\
21911 & 102979 & 2002 & 42.10 & -0.25 & 4113.00 & 37933.25 & 1.02 & 0.90 & 0.92 \\
33402 & 106129 & 2002 & 183.80 & -0.25 & 18410.00 & 183809.67 & 1.00 & 1.00 & 1.00 \\
29226 & 105545 & 2002 & 605.00 & -0.29 & 61310.00 & 566635.54 & 0.99 & 0.94 & 0.92 \\
16364 & 102130 & 2002 & 264.40 & -0.29 & 25591.00 & 269828.53 & 1.03 & 1.02 & 1.05 \\
46452 & 200245 & 2002 & 17.30 & -0.22 & 1736.00 & 17118.76 & 1.00 & 0.99 & 0.99 \\
24201 & 103296 & 2002 & 843.90 & -0.29 & 109608.00 & 945582.03 & 0.77 & 1.12 & 0.86 \\
58111 & 410094 & 2002 & 341.80 & -0.20 & NaN & 199670.81 & 1.00 & 0.58 & 1.00 \\
35762 & 106401 & 2002 & 743.90 & -0.10 & 86038.00 & 652830.81 & 0.86 & 0.88 & 0.76 \\
64505 & 500606 & 2002 & 351.40 & -0.21 & 18986.00 & 357921.95 & 1.85 & 1.02 & 1.89 \\
29441 & 105595 & 2002 & 7.80 & -0.21 & 896.00 & 6669.67 & 0.87 & 0.86 & 0.74 \\
46485 & 200248 & 2002 & 48.40 & -0.21 & 4806.00 & 46977.09 & 1.01 & 0.97 & 0.98 \\
42189 & 108933 & 2002 & 144.00 & -0.32 & 13556.00 & 136683.24 & 1.06 & 0.95 & 1.01 \\
38282 & 107235 & 2002 & 105.30 & -0.33 & 10581.00 & 103309.23 & 1.00 & 0.98 & 0.98 \\
1155 & 100157 & 2002 & 858.40 & -0.23 & 85865.00 & 831152.77 & 1.00 & 0.97 & 0.97 \\
35788 & 106402 & 2002 & 35.60 & -0.15 & 3001.00 & 33384.37 & 1.19 & 0.94 & 1.11 \\
50759 & 240448 & 2002 & 21.70 & -0.01 & 1913.00 & 18021.32 & 1.13 & 0.83 & 0.94 \\
17710 & 102349 & 2002 & 594.90 & -0.23 & 59426.00 & 584663.84 & 1.00 & 0.98 & 0.98 \\
46481 & 200247 & 2002 & 1.40 & -0.41 & 150.00 & 1428.54 & 0.93 & 1.02 & 0.95 \\
37896 & 107159 & 2002 & 7.90 & -0.12 & 751.00 & 6048.61 & 1.05 & 0.77 & 0.81 \\
35806 & 106413 & 2002 & 1724.50 & -0.24 & 223934.00 & 2239233.03 & 0.77 & 1.30 & 1.00 \\
3056 & 100401 & 2002 & 1300.90 & -0.18 & 124704.00 & 1203896.93 & 1.04 & 0.93 & 0.97 \\
7980 & 101068 & 2002 & 56750.80 & -0.24 & 5362102.00 & 54548142.33 & 1.06 & 0.96 & 1.02 \\
35756 & 106400 & 2002 & 141.80 & -0.32 & 14592.00 & 146929.45 & 0.97 & 1.04 & 1.01 \\
29463 & 105597 & 2002 & 78.70 & -0.26 & 7928.00 & 75908.41 & 0.99 & 0.96 & 0.96 \\
45408 & 200057 & 2002 & 714.70 & -0.28 & 104341.00 & 596255.81 & 0.68 & 0.83 & 0.57 \\
29489 & 105598 & 2002 & 310.90 & -0.25 & 31014.00 & 309415.01 & 1.00 & 1.00 & 1.00 \\
35737 & 106394 & 2002 & 69.10 & -0.19 & 6913.00 & 67994.16 & 1.00 & 0.98 & 0.98 \\
2275 & 100305 & 2002 & 62.50 & -0.27 & 5544.00 & 58262.82 & 1.13 & 0.93 & 1.05 \\
5272 & 100745 & 2002 & 1038.20 & -0.20 & 103620.00 & 929969.19 & 1.00 & 0.90 & 0.90 \\
13705 & 101758 & 2002 & 420.30 & -0.20 & 41977.00 & 401230.75 & 1.00 & 0.95 & 0.96 \\
16285 & 102121 & 2002 & 48.90 & -0.33 & 3523.00 & 35232.22 & 1.39 & 0.72 & 1.00 \\
21771 & 102951 & 2002 & 4646.90 & -0.24 & 694177.00 & 4762326.52 & 0.67 & 1.02 & 0.69 \\
35748 & 106398 & 2002 & 4.40 & -0.18 & 413.00 & 4397.99 & 1.07 & 1.00 & 1.06 \\
50821 & 240453 & 2002 & 15.80 & -0.07 & 1573.00 & 14261.85 & 1.00 & 0.90 & 0.91 \\
65461 & 500698 & 2002 & 21.70 & -0.22 & 1505.00 & 19216.57 & 1.44 & 0.89 & 1.28 \\
59191 & 410444 & 2002 & 21.00 & -0.16 & 2089.00 & 20386.02 & 1.01 & 0.97 & 0.98 \\
4726 & 100669 & 2002 & 292.90 & -0.28 & 29288.00 & 290426.46 & 1.00 & 0.99 & 0.99 \\
57243 & 400323 & 2002 & 268.50 & -0.32 & 27066.00 & 216342.75 & 0.99 & 0.81 & 0.80 \\
44233 & 109277 & 2002 & 201.40 & -0.29 & 21482.00 & 207584.22 & 0.94 & 1.03 & 0.97 \\
52474 & 302944 & 2002 & 113.50 & -0.30 & 11483.00 & 109211.67 & 0.99 & 0.96 & 0.95 \\
21832 & 102954 & 2002 & 165.90 & 0.67 & 13514.00 & 125842.17 & 1.23 & 0.76 & 0.93 \\
40210 & 108073 & 2002 & 205.20 & -0.00 & 20475.00 & 197857.71 & 1.00 & 0.96 & 0.97 \\
25697 & 103514 & 2002 & 2235.80 & -0.16 & 263609.00 & 2066798.79 & 0.85 & 0.92 & 0.78 \\
46462 & 200246 & 2002 & 111.20 & -0.17 & 11110.00 & 107235.39 & 1.00 & 0.96 & 0.97 \\
42164 & 108932 & 2002 & 601.80 & -0.27 & 76838.00 & 602084.40 & 0.78 & 1.00 & 0.78 \\
19638 & 102639 & 2002 & 86.40 & 0.02 & 9808.00 & 97412.86 & 0.88 & 1.13 & 0.99 \\
44235 & 109278 & 2002 & 17.60 & -0.23 & 1770.00 & 17648.27 & 0.99 & 1.00 & 1.00 \\
37882 & 107156 & 2002 & 76.30 & -0.05 & 7644.00 & 65884.24 & 1.00 & 0.86 & 0.86 \\
26586 & 103592 & 2002 & 397.20 & -0.19 & 39719.00 & 327274.44 & 1.00 & 0.82 & 0.82 \\
29363 & 105589 & 2002 & 180.60 & -0.29 & 18103.00 & 175359.88 & 1.00 & 0.97 & 0.97 \\
42158 & 108931 & 2002 & 36.90 & -0.22 & 3696.00 & 34408.17 & 1.00 & 0.93 & 0.93 \\
39147 & 107611 & 2002 & 914.10 & -0.13 & 104960.00 & 877577.62 & 0.87 & 0.96 & 0.84 \\
29357 & 105588 & 2002 & 7.20 & -0.18 & 606.00 & 6059.98 & 1.19 & 0.84 & 1.00 \\
3030 & 100399 & 2002 & 360.60 & -0.18 & 36105.00 & 294107.16 & 1.00 & 0.82 & 0.81 \\
52243 & 302698 & 2002 & 82.40 & -0.37 & 8168.00 & 80284.92 & 1.01 & 0.97 & 0.98 \\
74591 & 601139 & 2002 & 3816.60 & -0.16 & 460708.00 & 3299956.98 & 0.83 & 0.86 & 0.72 \\
35848 & 106418 & 2002 & 846.00 & -0.26 & 84563.00 & 814206.12 & 1.00 & 0.96 & 0.96 \\
18229 & 102417 & 2002 & 835.50 & -0.24 & 123572.00 & 716341.58 & 0.68 & 0.86 & 0.58 \\
43560 & 109144 & 2002 & 50.60 & -0.31 & 3841.00 & 45554.08 & 1.32 & 0.90 & 1.19 \\
3047 & 100400 & 2002 & 507.90 & 0.09 & 65899.00 & 643227.37 & 0.77 & 1.27 & 0.98 \\
40184 & 108071 & 2002 & 84.10 & -0.36 & 10977.00 & 109772.59 & 0.77 & 1.31 & 1.00 \\
29408 & 105593 & 2002 & 41.20 & -0.28 & 4115.00 & 39174.87 & 1.00 & 0.95 & 0.95 \\
6771 & 100953 & 2002 & 38.30 & 0.03 & 3607.00 & 35873.40 & 1.06 & 0.94 & 0.99 \\
1140 & 100155 & 2002 & 1329.60 & -0.23 & 132717.00 & 1275578.04 & 1.00 & 0.96 & 0.96 \\
29401 & 105592 & 2002 & 221.00 & -0.34 & 22200.00 & 216587.24 & 1.00 & 0.98 & 0.98 \\
16310 & 102124 & 2002 & 1526.10 & -0.18 & 145225.00 & 1530854.44 & 1.05 & 1.00 & 1.05 \\
41010 & 108176 & 2002 & 15.50 & -0.35 & 2376.00 & 13764.73 & 0.65 & 0.89 & 0.58 \\
13171 & 101698 & 2002 & 100.80 & -0.20 & 14536.00 & 116941.66 & 0.69 & 1.16 & 0.80 \\
59168 & 410443 & 2002 & 5.30 & -0.41 & 515.00 & 5015.82 & 1.03 & 0.95 & 0.97 \\
10802 & 101331 & 2002 & 61.40 & -0.24 & 6139.00 & 59404.55 & 1.00 & 0.97 & 0.97 \\
47413 & 210770 & 2002 & 1616.60 & -0.15 & 174120.00 & 1421072.24 & 0.93 & 0.88 & 0.82 \\
35832 & 106415 & 2002 & 198.40 & -0.38 & 19982.00 & 173392.36 & 0.99 & 0.87 & 0.87 \\
21815 & 102952 & 2002 & 208.40 & -0.33 & 34518.00 & 186250.95 & 0.60 & 0.89 & 0.54 \\
44834 & 109394 & 2002 & 145.50 & -0.26 & 9269.00 & 91664.53 & 1.57 & 0.63 & 0.99 \\
46416 & 200239 & 2002 & 1.90 & -0.20 & 193.00 & 1846.11 & 0.98 & 0.97 & 0.96 \\
42027 & 108910 & 2002 & 25.20 & -0.26 & 2515.00 & 25105.85 & 1.00 & 1.00 & 1.00 \\
12642 & 101561 & 2002 & 73.60 & -0.28 & 7292.00 & 69685.19 & 1.01 & 0.95 & 0.96 \\
28987 & 105510 & 2002 & 17.30 & -0.25 & 1984.00 & 19296.84 & 0.87 & 1.12 & 0.97 \\
18948 & 102529 & 2002 & 21.40 & -0.17 & 2243.00 & 17752.24 & 0.95 & 0.83 & 0.79 \\
3818 & 100485 & 2002 & 506.90 & -0.23 & 49945.00 & 496926.96 & 1.01 & 0.98 & 0.99 \\
38993 & 107563 & 2002 & 907.00 & -0.22 & 73861.00 & 792175.88 & 1.23 & 0.87 & 1.07 \\
47277 & 200347 & 2002 & 2090.70 & -0.17 & NaN & 1912428.70 & 1.00 & 0.91 & 1.00 \\
40292 & 108087 & 2002 & 24.40 & -0.50 & 2244.00 & 21928.77 & 1.09 & 0.90 & 0.98 \\
48003 & 225696 & 2002 & 39.60 & -0.31 & 3961.00 & 37318.17 & 1.00 & 0.94 & 0.94 \\
13869 & 101781 & 2002 & 223.10 & -0.40 & 41022.00 & 223206.31 & 0.54 & 1.00 & 0.54 \\
36195 & 106477 & 2002 & 1033.00 & -0.21 & 107754.00 & 935300.71 & 0.96 & 0.91 & 0.87 \\
48829 & 240144 & 2002 & 13.40 & -0.15 & 1337.00 & 10915.59 & 1.00 & 0.81 & 0.82 \\
19880 & 102654 & 2002 & 761.00 & -0.29 & 74971.00 & 729842.41 & 1.02 & 0.96 & 0.97 \\
12520 & 101545 & 2002 & 138.20 & -0.20 & 13410.00 & 134116.59 & 1.03 & 0.97 & 1.00 \\
16444 & 102145 & 2002 & 136.90 & -0.18 & 12801.00 & 127315.98 & 1.07 & 0.93 & 0.99 \\
9325 & 101131 & 2002 & 3476.70 & -0.33 & 455595.00 & 3703473.97 & 0.76 & 1.07 & 0.81 \\
42040 & 108914 & 2002 & 23.80 & 0.08 & 2453.00 & 21297.08 & 0.97 & 0.89 & 0.87 \\
44358 & 109289 & 2002 & 48.80 & -0.37 & 4901.00 & 46465.84 & 1.00 & 0.95 & 0.95 \\
42037 & 108912 & 2002 & 22.90 & -0.44 & 2080.00 & 20653.22 & 1.10 & 0.90 & 0.99 \\
46391 & 200232 & 2002 & 2.10 & -0.07 & 200.00 & 1702.72 & 1.05 & 0.81 & 0.85 \\
36159 & 106474 & 2002 & 140.20 & -0.23 & 14532.00 & 126268.79 & 0.96 & 0.90 & 0.87 \\
18961 & 102531 & 2002 & 18.90 & -0.25 & 2258.00 & 19977.77 & 0.84 & 1.06 & 0.88 \\
28961 & 105508 & 2002 & 53.80 & -0.16 & 5382.00 & 53815.85 & 1.00 & 1.00 & 1.00 \\
33383 & 106127 & 2002 & 200.40 & -0.14 & 10937.00 & 87683.23 & 1.83 & 0.44 & 0.80 \\
58783 & 410217 & 2002 & 2.90 & -0.07 & 306.00 & 2710.52 & 0.95 & 0.93 & 0.89 \\
29005 & 105512 & 2002 & 7.80 & -0.18 & 756.00 & 7215.66 & 1.03 & 0.93 & 0.95 \\
59146 & 410442 & 2002 & 21.70 & -0.09 & 2217.00 & 20633.53 & 0.98 & 0.95 & 0.93 \\
22049 & 102988 & 2002 & 33.00 & -0.14 & 3335.00 & 31209.15 & 0.99 & 0.95 & 0.94 \\
25829 & 103524 & 2002 & 68145.50 & -0.22 & 8717331.00 & 67996825.37 & 0.78 & 1.00 & 0.78 \\
37827 & 107147 & 2002 & 24.60 & -0.48 & 1907.00 & 23700.77 & 1.29 & 0.96 & 1.24 \\
74812 & 601179 & 2002 & 31.40 & -0.15 & 3009.00 & 30090.83 & 1.04 & 0.96 & 1.00 \\
19702 & 102649 & 2002 & 531.30 & -0.18 & 52043.00 & 492471.35 & 1.02 & 0.93 & 0.95 \\
22077 & 102989 & 2002 & 1369.20 & -0.22 & 202969.00 & 1435049.41 & 0.67 & 1.05 & 0.71 \\
687 & 100090 & 2002 & 480.60 & -0.33 & 48070.00 & 479562.86 & 1.00 & 1.00 & 1.00 \\
38987 & 107470 & 2002 & 1.90 & -0.31 & 188.00 & 1722.79 & 1.01 & 0.91 & 0.92 \\
25862 & 103525 & 2002 & 29403.60 & -0.22 & 3559514.00 & 28860798.93 & 0.83 & 0.98 & 0.81 \\
46278 & 200207 & 2002 & 18.20 & -0.15 & 1844.00 & 16688.28 & 0.99 & 0.92 & 0.91 \\
64443 & 500603 & 2002 & 143.50 & -0.22 & 12114.00 & 118442.09 & 1.18 & 0.83 & 0.98 \\
18929 & 102528 & 2002 & 73.60 & -0.12 & 7636.00 & 74728.67 & 0.96 & 1.02 & 0.98 \\
47300 & 200503 & 2002 & 15.40 & -0.23 & 1560.00 & 15220.39 & 0.99 & 0.99 & 0.98 \\
22110 & 102990 & 2002 & 3424.80 & -0.18 & 341764.00 & 3165386.24 & 1.00 & 0.92 & 0.93 \\
36267 & 106480 & 2002 & 442.60 & -0.18 & 52346.00 & 359330.08 & 0.85 & 0.81 & 0.69 \\
46381 & 200228 & 2002 & 1.70 & -0.15 & 196.00 & 1713.78 & 0.87 & 1.01 & 0.87 \\
55122 & 400062 & 2002 & 38.50 & -0.11 & 3837.00 & 33833.58 & 1.00 & 0.88 & 0.88 \\
49065 & 240212 & 2002 & 3770.70 & -0.21 & 398726.00 & 3073488.23 & 0.95 & 0.82 & 0.77 \\
46358 & 200227 & 2002 & 10.30 & -0.28 & 1034.00 & 10136.23 & 1.00 & 0.98 & 0.98 \\
33357 & 106124 & 2002 & 655.60 & -0.17 & 69882.00 & 668132.68 & 0.94 & 1.02 & 0.96 \\
46356 & 200226 & 2002 & 6.50 & -0.23 & 610.00 & 6309.56 & 1.07 & 0.97 & 1.03 \\
41976 & 108886 & 2002 & 55.00 & -0.43 & 5495.00 & 53597.08 & 1.00 & 0.97 & 0.98 \\
36293 & 106481 & 2002 & 148.00 & -0.07 & 13604.00 & 136043.93 & 1.09 & 0.92 & 1.00 \\
49718 & 240337 & 2002 & 3.30 & -0.44 & 266.00 & 2533.26 & 1.24 & 0.77 & 0.95 \\
36151 & 106471 & 2002 & 128.50 & -0.11 & 17411.00 & 128832.35 & 0.74 & 1.00 & 0.74 \\
2316 & 100315 & 2002 & 190.20 & -0.26 & 18894.00 & 178020.70 & 1.01 & 0.94 & 0.94 \\
36218 & 106478 & 2002 & 36.90 & 0.07 & 3810.00 & 36549.39 & 0.97 & 0.99 & 0.96 \\
45445 & 200060 & 2002 & 622.40 & -0.17 & 66382.00 & 530224.34 & 0.94 & 0.85 & 0.80 \\
36241 & 106479 & 2002 & 25.70 & -0.17 & 2563.00 & 25376.02 & 1.00 & 0.99 & 0.99 \\
24116 & 103267 & 2002 & 1275.80 & -0.30 & 103298.00 & 1167752.82 & 1.24 & 0.92 & 1.13 \\
28942 & 105507 & 2002 & 663.20 & -0.22 & 66183.00 & 647438.14 & 1.00 & 0.98 & 0.98 \\
8632 & 101092 & 2002 & 352.60 & -0.01 & 40940.00 & 403456.26 & 0.86 & 1.14 & 0.99 \\
44361 & 109290 & 2002 & 195.50 & -0.26 & 19525.00 & 181772.96 & 1.00 & 0.93 & 0.93 \\
10960 & 101356 & 2002 & 289.30 & -0.34 & 28974.00 & 280743.48 & 1.00 & 0.97 & 0.97 \\
14675 & 101908 & 2002 & 66.70 & -0.21 & 6649.00 & 65949.98 & 1.00 & 0.99 & 0.99 \\
17193 & 102270 & 2002 & 610.80 & -0.29 & 61071.00 & 604037.97 & 1.00 & 0.99 & 0.99 \\
44364 & 109291 & 2002 & 30.10 & -0.15 & 3002.00 & 30019.96 & 1.00 & 1.00 & 1.00 \\
324 & 100036 & 2002 & 43.40 & -0.28 & 4562.00 & 44004.73 & 0.95 & 1.01 & 0.96 \\
49416 & 240295 & 2002 & 1165.70 & -0.23 & 130743.00 & 1075051.20 & 0.89 & 0.92 & 0.82 \\
28926 & 105506 & 2002 & 255.60 & -0.24 & 25699.00 & 245336.88 & 0.99 & 0.96 & 0.95 \\
41985 & 108889 & 2002 & 5.00 & -0.36 & 335.00 & 3154.40 & 1.49 & 0.63 & 0.94 \\
32309 & 106010 & 2002 & 1263.40 & -0.14 & 126478.00 & 1180378.79 & 1.00 & 0.93 & 0.93 \\
48855 & 240148 & 2002 & 247.30 & -0.25 & 19937.00 & 206698.20 & 1.24 & 0.84 & 1.04 \\
39020 & 107565 & 2002 & 1.40 & -0.20 & 121.00 & 1353.44 & 1.16 & 0.97 & 1.12 \\
7803 & 101061 & 2002 & 2615.70 & -0.20 & 284278.00 & 2292624.32 & 0.92 & 0.88 & 0.81 \\
47461 & 211485 & 2002 & 63.00 & -0.23 & 6302.00 & 62310.56 & 1.00 & 0.99 & 0.99 \\
16399 & 102132 & 2002 & 27.30 & -0.27 & 2748.00 & 27171.73 & 0.99 & 1.00 & 0.99 \\
10899 & 101345 & 2002 & 1738.40 & -0.11 & 143423.00 & 1260428.26 & 1.21 & 0.73 & 0.88 \\
27099 & 103652 & 2002 & 84.50 & -0.17 & 8459.00 & 83422.56 & 1.00 & 0.99 & 0.99 \\
36081 & 106459 & 2002 & 6.40 & -0.14 & 635.00 & 6244.14 & 1.01 & 0.98 & 0.98 \\
58101 & 410093 & 2002 & 46.80 & -0.44 & 7330.00 & 40441.98 & 0.64 & 0.86 & 0.55 \\
42097 & 108919 & 2002 & 106.50 & -0.40 & 10698.00 & 98192.00 & 1.00 & 0.92 & 0.92 \\
18993 & 102540 & 2002 & 19.20 & 0.00 & 1854.00 & 16639.44 & 1.04 & 0.87 & 0.90 \\
13127 & 101668 & 2002 & 90.90 & -0.16 & 9074.00 & 80793.30 & 1.00 & 0.89 & 0.89 \\
43707 & 109193 & 2002 & 10.40 & -0.32 & 1039.00 & 10113.20 & 1.00 & 0.97 & 0.97 \\
36086 & 106461 & 2002 & 217.30 & -0.16 & 20093.00 & 181229.11 & 1.08 & 0.83 & 0.90 \\
12508 & 101544 & 2002 & 224.30 & -0.28 & 22313.00 & 223148.69 & 1.01 & 0.99 & 1.00 \\
29107 & 105527 & 2002 & 4.70 & -0.26 & 432.00 & 4388.89 & 1.09 & 0.93 & 1.02 \\
42072 & 108918 & 2002 & 36.10 & -0.16 & 3609.00 & 35931.90 & 1.00 & 1.00 & 1.00 \\
16407 & 102133 & 2002 & 31.20 & -0.32 & 3120.00 & 28692.39 & 1.00 & 0.92 & 0.92 \\
7242 & 101015 & 2002 & 1034.30 & -0.26 & 146693.00 & 950389.64 & 0.71 & 0.92 & 0.65 \\
43304 & 109092 & 2002 & 44.20 & -0.15 & 4472.00 & 42308.85 & 0.99 & 0.96 & 0.95 \\
42105 & 108920 & 2002 & 20.00 & -0.50 & 2186.00 & 18188.37 & 0.91 & 0.91 & 0.83 \\
64482 & 500605 & 2002 & 104.90 & -0.15 & 10480.00 & 92909.03 & 1.00 & 0.89 & 0.89 \\
29130 & 105531 & 2002 & 115.80 & -0.28 & 11580.00 & 109405.25 & 1.00 & 0.94 & 0.94 \\
1105 & 100153 & 2002 & 130.20 & -0.20 & 13008.00 & 128682.40 & 1.00 & 0.99 & 0.99 \\
43287 & 109090 & 2002 & 37.20 & -0.07 & 3711.00 & 36945.69 & 1.00 & 0.99 & 1.00 \\
39059 & 107598 & 2002 & 53.10 & -0.18 & 3956.00 & 50274.93 & 1.34 & 0.95 & 1.27 \\
2956 & 100389 & 2002 & 313.00 & -0.29 & 27336.00 & 271732.19 & 1.15 & 0.87 & 0.99 \\
36036 & 106449 & 2002 & 104.40 & -0.44 & 13787.00 & 106486.00 & 0.76 & 1.02 & 0.77 \\
38294 & 107242 & 2002 & 640.90 & -0.10 & 57853.00 & 614134.98 & 1.11 & 0.96 & 1.06 \\
48782 & 240140 & 2002 & 28.00 & -0.12 & 2204.00 & 21201.96 & 1.27 & 0.76 & 0.96 \\
36046 & 106450 & 2002 & 9.40 & -0.43 & 941.00 & 8264.48 & 1.00 & 0.88 & 0.88 \\
5786 & 100792 & 2002 & 425.30 & -0.11 & 38471.00 & 398678.95 & 1.11 & 0.94 & 1.04 \\
29150 & 105533 & 2002 & 187.90 & -0.36 & 18787.00 & 185237.02 & 1.00 & 0.99 & 0.99 \\
11992 & 101476 & 2002 & 2378.10 & -0.18 & 336947.00 & 2131504.31 & 0.71 & 0.90 & 0.63 \\
74841 & 601186 & 2002 & 8.40 & -0.31 & 788.00 & 7654.69 & 1.07 & 0.91 & 0.97 \\
36051 & 106451 & 2002 & 153.10 & -0.32 & 19155.00 & 191321.18 & 0.80 & 1.25 & 1.00 \\
56068 & 400171 & 2002 & 7.10 & -0.15 & 705.00 & 6975.72 & 1.01 & 0.98 & 0.99 \\
44304 & 109283 & 2002 & 680.90 & -0.16 & 48955.00 & 585905.69 & 1.39 & 0.86 & 1.20 \\
21959 & 102981 & 2002 & 122.90 & -0.28 & 11415.00 & 123930.61 & 1.08 & 1.01 & 1.09 \\
47677 & 217585 & 2002 & 201.90 & -0.22 & 19103.00 & 201541.75 & 1.06 & 1.00 & 1.06 \\
4967 & 100697 & 2002 & 27.00 & -0.10 & 2353.00 & 25121.51 & 1.15 & 0.93 & 1.07 \\
32741 & 106057 & 2002 & 258.80 & -0.31 & 25877.00 & 253332.60 & 1.00 & 0.98 & 0.98 \\
44827 & 109393 & 2002 & 8.90 & -0.25 & 1051.00 & 8534.09 & 0.85 & 0.96 & 0.81 \\
52269 & 302731 & 2002 & 1167.30 & -0.07 & 117442.00 & 1092628.44 & 0.99 & 0.94 & 0.93 \\
36131 & 106467 & 2002 & 129.10 & -0.30 & 14268.00 & 107059.86 & 0.90 & 0.83 & 0.75 \\
58074 & 410075 & 2002 & 173.50 & 0.05 & 17334.00 & 166320.26 & 1.00 & 0.96 & 0.96 \\
36140 & 106470 & 2002 & 288.30 & -0.27 & 28764.00 & 286216.11 & 1.00 & 0.99 & 1.00 \\
20112 & 102667 & 2002 & 14377.80 & -0.25 & 1876762.00 & 12695566.66 & 0.77 & 0.88 & 0.68 \\
44335 & 109286 & 2002 & 98.50 & -0.21 & 10019.00 & 92255.30 & 0.98 & 0.94 & 0.92 \\
22018 & 102987 & 2002 & 669.00 & -0.34 & 67325.00 & 612576.48 & 0.99 & 0.92 & 0.91 \\
8211 & 101079 & 2002 & 197.80 & -0.11 & 22172.00 & 161070.11 & 0.89 & 0.81 & 0.73 \\
39024 & 107566 & 2002 & 2.60 & -0.05 & 251.00 & 2475.40 & 1.04 & 0.95 & 0.99 \\
46398 & 200236 & 2002 & 3.50 & -0.23 & 249.00 & 2483.46 & 1.41 & 0.71 & 1.00 \\
18515 & 102469 & 2002 & 64.60 & -0.26 & 6456.00 & 64560.36 & 1.00 & 1.00 & 1.00 \\
29032 & 105520 & 2002 & 14.10 & -0.23 & 1401.00 & 13728.87 & 1.01 & 0.97 & 0.98 \\
48012 & 226438 & 2002 & 421.90 & -0.30 & 42030.00 & 420326.47 & 1.00 & 1.00 & 1.00 \\
46395 & 200233 & 2002 & 3.80 & -0.23 & 385.00 & 3484.39 & 0.99 & 0.92 & 0.91 \\
42064 & 108915 & 2002 & 11.00 & -0.42 & 1649.00 & 16487.01 & 0.67 & 1.50 & 1.00 \\
10928 & 101354 & 2002 & 900.10 & -0.27 & 90517.00 & 882869.04 & 0.99 & 0.98 & 0.98 \\
29040 & 105522 & 2002 & 288.00 & -0.25 & 27065.00 & 271099.14 & 1.06 & 0.94 & 1.00 \\
40262 & 108083 & 2002 & 118.50 & -0.18 & 5991.00 & 109994.37 & 1.98 & 0.93 & 1.84 \\
39030 & 107573 & 2002 & 203.70 & -0.21 & 20346.00 & 203461.58 & 1.00 & 1.00 & 1.00 \\
4286 & 100600 & 2002 & 45.00 & -0.12 & 4363.00 & 41282.08 & 1.03 & 0.92 & 0.95 \\
25795 & 103523 & 2002 & 3914.40 & -0.18 & 474286.00 & 3676003.93 & 0.83 & 0.94 & 0.78 \\
2917 & 100379 & 2002 & 406.70 & -0.14 & 36058.00 & 353587.12 & 1.13 & 0.87 & 0.98 \\
47688 & 220681 & 2002 & 671.90 & -0.16 & 80707.00 & 658307.73 & 0.83 & 0.98 & 0.82 \\
44327 & 109284 & 2002 & 7.40 & -0.21 & 727.00 & 7167.92 & 1.02 & 0.97 & 0.99 \\
36105 & 106464 & 2002 & 214.60 & -0.39 & 22096.00 & 197159.85 & 0.97 & 0.92 & 0.89 \\
29079 & 105525 & 2002 & 1100.10 & -0.02 & 115640.00 & 1017843.88 & 0.95 & 0.93 & 0.88 \\
16416 & 102134 & 2002 & 64.10 & -0.16 & 6795.00 & 63290.42 & 0.94 & 0.99 & 0.93 \\
7350 & 101023 & 2002 & 17733.70 & -0.17 & 1853998.00 & 16557401.25 & 0.96 & 0.93 & 0.89 \\
32296 & 106009 & 2002 & 1566.00 & -0.25 & 156251.00 & 1542516.50 & 1.00 & 0.99 & 0.99 \\
48805 & 240143 & 2002 & 112.50 & -0.36 & 16846.00 & 106813.52 & 0.67 & 0.95 & 0.63 \\
55228 & 400074 & 2002 & 1255.70 & -0.23 & 125612.00 & 1248318.49 & 1.00 & 0.99 & 0.99 \\
61210 & 410907 & 2002 & 6.70 & -0.04 & 409.00 & 4064.56 & 1.64 & 0.61 & 0.99 \\
6639 & 100906 & 2002 & 1160.30 & -0.37 & 116021.00 & 1154243.75 & 1.00 & 0.99 & 0.99 \\
59073 & 410423 & 2002 & 2.70 & -0.18 & NaN & 2567.04 & 1.00 & 0.95 & 1.00 \\
54454 & 367992 & 2002 & 21.60 & -0.06 & 2190.00 & 18307.19 & 0.99 & 0.85 & 0.84 \\
49856 & 240368 & 2002 & 126.90 & -0.05 & 8343.00 & 78265.36 & 1.52 & 0.62 & 0.94 \\
40614 & 108142 & 2002 & 15.60 & -0.14 & 1559.00 & 15439.43 & 1.00 & 0.99 & 0.99 \\
58035 & 410055 & 2002 & 56.10 & -0.25 & 5597.00 & 55968.24 & 1.00 & 1.00 & 1.00 \\
63856 & 500561 & 2002 & 13.10 & -0.21 & 1281.00 & 12391.27 & 1.02 & 0.95 & 0.97 \\
16970 & 102224 & 2002 & 4219.10 & -0.20 & 467615.00 & 4062830.93 & 0.90 & 0.96 & 0.87 \\
49495 & 240305 & 2002 & 37.70 & -0.39 & 3779.00 & 36837.39 & 1.00 & 0.98 & 0.97 \\
37315 & 106729 & 2002 & 663.20 & -0.23 & 66910.00 & 658566.16 & 0.99 & 0.99 & 0.98 \\
47814 & 222027 & 2002 & 1480.40 & -0.48 & 148013.00 & 1432185.61 & 1.00 & 0.97 & 0.97 \\
63866 & 500562 & 2002 & 14.70 & -0.23 & 1446.00 & 13510.21 & 1.02 & 0.92 & 0.93 \\
27666 & 105309 & 2002 & 674.90 & -0.19 & 61335.00 & 575617.87 & 1.10 & 0.85 & 0.94 \\
11459 & 101414 & 2002 & 16.10 & -0.14 & 1704.00 & 14480.10 & 0.94 & 0.90 & 0.85 \\
8361 & 101084 & 2002 & 1458.70 & -0.38 & 222386.00 & 1293526.26 & 0.66 & 0.89 & 0.58 \\
44774 & 109375 & 2002 & 32.00 & -0.15 & 3147.00 & 30159.59 & 1.02 & 0.94 & 0.96 \\
18642 & 102493 & 2002 & 1586.80 & -0.25 & 159034.00 & 1554919.09 & 1.00 & 0.98 & 0.98 \\
27692 & 105310 & 2002 & 94.90 & -0.29 & 8786.00 & 87869.09 & 1.08 & 0.93 & 1.00 \\
49866 & 240369 & 2002 & 1.50 & -0.05 & 155.00 & 1516.90 & 0.97 & 1.01 & 0.98 \\
40589 & 108141 & 2002 & 64.50 & -0.07 & 6517.00 & 53594.42 & 0.99 & 0.83 & 0.82 \\
54950 & 400030 & 2002 & 21.70 & -0.14 & 1909.00 & 15440.65 & 1.14 & 0.71 & 0.81 \\
37288 & 106726 & 2002 & 1532.50 & -0.23 & 141754.00 & 1511815.13 & 1.08 & 0.99 & 1.07 \\
38575 & 107290 & 2002 & 407.10 & -0.22 & 51487.00 & 386100.27 & 0.79 & 0.95 & 0.75 \\
41325 & 108728 & 2002 & 23.70 & -0.17 & 2398.00 & 23702.06 & 0.99 & 1.00 & 0.99 \\
13957 & 101789 & 2002 & 359.90 & -0.24 & 49080.00 & 485448.00 & 0.73 & 1.35 & 0.99 \\
57868 & 401296 & 2002 & 1.10 & -0.42 & 94.00 & 921.48 & 1.17 & 0.84 & 0.98 \\
43425 & 109117 & 2002 & 137.10 & -0.24 & 14183.00 & 123979.08 & 0.97 & 0.90 & 0.87 \\
32510 & 106037 & 2002 & 33.10 & -0.10 & 3314.00 & 31756.57 & 1.00 & 0.96 & 0.96 \\
54461 & 368366 & 2002 & 82.50 & -0.26 & NaN & 49126.45 & 1.00 & 0.60 & 1.00 \\
47800 & 221485 & 2002 & 462.00 & -0.23 & 46102.00 & 456615.46 & 1.00 & 0.99 & 0.99 \\
27700 & 105311 & 2002 & 110.80 & -0.36 & 10889.00 & 108889.66 & 1.02 & 0.98 & 1.00 \\
48407 & 240076 & 2002 & 77.10 & -0.24 & 9810.00 & 71205.51 & 0.79 & 0.92 & 0.73 \\
63845 & 500560 & 2002 & 13.30 & -0.25 & 1312.00 & 12169.38 & 1.01 & 0.91 & 0.93 \\
57877 & 401355 & 2002 & 15.20 & -0.34 & 1517.00 & 14599.43 & 1.00 & 0.96 & 0.96 \\
5085 & 100723 & 2002 & 17.90 & -0.23 & 1800.00 & 17949.16 & 0.99 & 1.00 & 1.00 \\
40639 & 108143 & 2002 & 14.20 & -0.03 & 1422.00 & 14188.00 & 1.00 & 1.00 & 1.00 \\
12612 & 101560 & 2002 & 17.00 & -0.26 & 1705.00 & 16324.91 & 1.00 & 0.96 & 0.96 \\
27608 & 105303 & 2002 & 148.30 & -0.18 & 14853.00 & 147169.68 & 1.00 & 0.99 & 0.99 \\
49503 & 240307 & 2002 & 61.60 & -0.09 & 9265.00 & 49977.94 & 0.66 & 0.81 & 0.54 \\
7739 & 101056 & 2002 & 26268.70 & -0.18 & 3447438.00 & 24463111.34 & 0.76 & 0.93 & 0.71 \\
43427 & 109118 & 2002 & 314.20 & -0.26 & 28010.00 & 265776.24 & 1.12 & 0.85 & 0.95 \\
38531 & 107274 & 2002 & 4.50 & -0.12 & 451.00 & 4235.29 & 1.00 & 0.94 & 0.94 \\
37389 & 106747 & 2002 & 78.90 & 0.17 & 5113.00 & 53686.24 & 1.54 & 0.68 & 1.05 \\
22984 & 103100 & 2002 & 292.30 & -0.23 & 29434.00 & 274320.63 & 0.99 & 0.94 & 0.93 \\
27591 & 105295 & 2002 & 346.10 & -0.16 & 31567.00 & 322994.02 & 1.10 & 0.93 & 1.02 \\
64282 & 500595 & 2002 & 691.40 & -0.22 & 42934.00 & 632654.29 & 1.61 & 0.92 & 1.47 \\
57882 & 401360 & 2002 & 4.50 & -0.41 & 450.00 & 4225.61 & 1.00 & 0.94 & 0.94 \\
52416 & 302907 & 2002 & 382.70 & -0.27 & 35636.00 & 398698.73 & 1.07 & 1.04 & 1.12 \\
44768 & 109374 & 2002 & 96.20 & -0.34 & 9749.00 & 89865.58 & 0.99 & 0.93 & 0.92 \\
41304 & 108723 & 2002 & 97.90 & -0.23 & 8095.00 & 93520.34 & 1.21 & 0.96 & 1.16 \\
45872 & 200151 & 2002 & 32.00 & 0.06 & 3117.00 & 31907.18 & 1.03 & 1.00 & 1.02 \\
60800 & 410727 & 2002 & 3.50 & -0.45 & 540.00 & 3206.28 & 0.65 & 0.92 & 0.59 \\
32664 & 106047 & 2002 & 4.40 & -0.33 & 448.00 & 4170.52 & 0.98 & 0.95 & 0.93 \\
45591 & 200081 & 2002 & 4.70 & -0.27 & 469.00 & 4652.69 & 1.00 & 0.99 & 0.99 \\
37361 & 106737 & 2002 & 160.80 & -0.24 & 14784.00 & 168519.10 & 1.09 & 1.05 & 1.14 \\
32994 & 106085 & 2002 & 203.80 & -0.30 & 39806.00 & 203637.78 & 0.51 & 1.00 & 0.51 \\
48936 & 240154 & 2002 & 36.90 & -0.14 & 3739.00 & 38789.00 & 0.99 & 1.05 & 1.04 \\
37370 & 106740 & 2002 & 121.60 & -0.30 & 12214.00 & 118384.97 & 1.00 & 0.97 & 0.97 \\
14305 & 101843 & 2002 & 121.10 & -0.17 & 17796.00 & 125059.98 & 0.68 & 1.03 & 0.70 \\
63877 & 500563 & 2002 & 13.50 & -0.14 & 1327.00 & 12501.59 & 1.02 & 0.93 & 0.94 \\
38542 & 107281 & 2002 & 14.00 & -0.25 & 1405.00 & 13578.91 & 1.00 & 0.97 & 0.97 \\
41314 & 108726 & 2002 & 13.80 & -0.15 & 1212.00 & 11808.99 & 1.14 & 0.86 & 0.97 \\
27637 & 105306 & 2002 & 76.10 & -0.42 & 7707.00 & 73988.08 & 0.99 & 0.97 & 0.96 \\
4480 & 100634 & 2002 & 830.10 & -0.34 & 73993.00 & 760690.73 & 1.12 & 0.92 & 1.03 \\
23798 & 103213 & 2002 & 588.90 & -0.30 & 58208.00 & 557309.53 & 1.01 & 0.95 & 0.96 \\
547 & 100075 & 2002 & 2087.40 & -0.23 & 204850.00 & 2078228.04 & 1.02 & 1.00 & 1.01 \\
37378 & 106742 & 2002 & 101.50 & -0.14 & 10142.00 & 100481.58 & 1.00 & 0.99 & 0.99 \\
40768 & 108149 & 2002 & 27.10 & -0.14 & 2716.00 & 27070.73 & 1.00 & 1.00 & 1.00 \\
37385 & 106746 & 2002 & 18.90 & -0.27 & 1885.00 & 18196.16 & 1.00 & 0.96 & 0.97 \\
55184 & 400069 & 2002 & 45.10 & -0.67 & 3334.00 & 41845.87 & 1.35 & 0.93 & 1.26 \\
5448 & 100763 & 2002 & 1218.00 & -0.20 & 127437.00 & 1158588.19 & 0.96 & 0.95 & 0.91 \\
64305 & 500596 & 2002 & 743.60 & -0.28 & 49793.00 & 489017.19 & 1.49 & 0.66 & 0.98 \\
37179 & 106707 & 2002 & 63.00 & -0.13 & 6472.00 & 60261.58 & 0.97 & 0.96 & 0.93 \\
65794 & 500736 & 2002 & 5.10 & -0.12 & 787.00 & 4214.97 & 0.65 & 0.83 & 0.54 \\
17140 & 102258 & 2002 & 250.50 & -0.31 & 25095.00 & 248942.21 & 1.00 & 0.99 & 0.99 \\
41384 & 108742 & 2002 & 7.20 & -0.19 & 717.00 & 6436.66 & 1.00 & 0.89 & 0.90 \\
40576 & 108139 & 2002 & 25.10 & -0.36 & 2510.00 & 24591.93 & 1.00 & 0.98 & 0.98 \\
33125 & 106092 & 2002 & 514.10 & -0.25 & 50931.00 & 488828.37 & 1.01 & 0.95 & 0.96 \\
937 & 100112 & 2002 & 6552.00 & -0.27 & 655443.00 & 6456842.61 & 1.00 & 0.99 & 0.99 \\
49927 & 240376 & 2002 & 8.80 & -0.20 & 882.00 & 8709.95 & 1.00 & 0.99 & 0.99 \\
57863 & 401265 & 2002 & 16.30 & -0.22 & 1627.00 & 16169.47 & 1.00 & 0.99 & 0.99 \\
41373 & 108736 & 2002 & 47.10 & -0.19 & 3883.00 & 38420.29 & 1.21 & 0.82 & 0.99 \\
37205 & 106708 & 2002 & 337.80 & -0.22 & 33386.00 & 330670.47 & 1.01 & 0.98 & 0.99 \\
40580 & 108140 & 2002 & 119.70 & -0.18 & 10290.00 & 111859.77 & 1.16 & 0.93 & 1.09 \\
37218 & 106710 & 2002 & 38.20 & -0.28 & 4433.00 & 37358.14 & 0.86 & 0.98 & 0.84 \\
47769 & 221210 & 2002 & 115.30 & -0.07 & 11466.00 & 110660.34 & 1.01 & 0.96 & 0.97 \\
27248 & 105259 & 2002 & 197.50 & -0.24 & 18538.00 & 198741.74 & 1.07 & 1.01 & 1.07 \\
37165 & 106706 & 2002 & 19.10 & -0.13 & 1900.00 & 18794.91 & 1.01 & 0.98 & 0.99 \\
47338 & 210203 & 2002 & 4912.40 & -0.31 & 481651.00 & 4570873.67 & 1.02 & 0.93 & 0.95 \\
43416 & 109112 & 2002 & 30.30 & -0.43 & 3032.00 & 29005.89 & 1.00 & 0.96 & 0.96 \\
64328 & 500597 & 2002 & 5074.60 & -0.22 & 427978.00 & 3460515.19 & 1.19 & 0.68 & 0.81 \\
49972 & 240379 & 2002 & 15.30 & -0.23 & 1433.00 & 14572.81 & 1.07 & 0.95 & 1.02 \\
27855 & 105333 & 2002 & 9.00 & -0.13 & 827.00 & 8814.88 & 1.09 & 0.98 & 1.07 \\
13938 & 101788 & 2002 & 571.50 & -0.22 & 82479.00 & 581433.45 & 0.69 & 1.02 & 0.70 \\
41391 & 108745 & 2002 & 42.70 & -0.00 & 4432.00 & 44199.16 & 0.96 & 1.04 & 1.00 \\
63772 & 500553 & 2002 & 99.20 & -0.51 & 14604.00 & 91894.07 & 0.68 & 0.93 & 0.63 \\
37158 & 106701 & 2002 & 43.00 & -0.23 & 4331.00 & 42901.41 & 0.99 & 1.00 & 0.99 \\
11877 & 101464 & 2002 & 605.10 & -0.08 & 48625.00 & 554579.15 & 1.24 & 0.92 & 1.14 \\
2368 & 100320 & 2002 & 50.60 & -0.10 & 5020.00 & 49293.16 & 1.01 & 0.97 & 0.98 \\
4440 & 100625 & 2002 & 1769.90 & -0.22 & 176930.00 & 1672807.36 & 1.00 & 0.95 & 0.95 \\
22831 & 103067 & 2002 & 54.50 & -0.31 & 5457.00 & 52353.49 & 1.00 & 0.96 & 0.96 \\
49950 & 240377 & 2002 & 15.00 & -0.24 & 1661.00 & 14223.11 & 0.90 & 0.95 & 0.86 \\
38589 & 107294 & 2002 & 138.20 & -0.21 & 12014.00 & 105242.75 & 1.15 & 0.76 & 0.88 \\
5470 & 100764 & 2002 & 310.40 & -0.07 & 32509.00 & 293877.45 & 0.95 & 0.95 & 0.90 \\
48930 & 240153 & 2002 & 65.50 & -0.21 & 6541.00 & 65349.89 & 1.00 & 1.00 & 1.00 \\
27277 & 105260 & 2002 & 206.90 & -0.12 & 20451.00 & 181491.29 & 1.01 & 0.88 & 0.89 \\
27797 & 105331 & 2002 & 6.00 & -0.16 & 500.00 & 5033.30 & 1.20 & 0.84 & 1.01 \\
43627 & 109165 & 2002 & 16.20 & -0.32 & 1629.00 & 16016.49 & 0.99 & 0.99 & 0.98 \\
27755 & 105321 & 2002 & 290.30 & -0.20 & 26513.00 & 289720.38 & 1.09 & 1.00 & 1.09 \\
18357 & 102446 & 2002 & 12.50 & -0.20 & 1252.00 & 12059.85 & 1.00 & 0.96 & 0.96 \\
54479 & 372363 & 2002 & 8.90 & -0.07 & 1011.00 & 8295.82 & 0.88 & 0.93 & 0.82 \\
54478 & 371810 & 2002 & 41.20 & -0.04 & 3776.00 & 40864.64 & 1.09 & 0.99 & 1.08 \\
11814 & 101462 & 2002 & 390.20 & -0.12 & 47235.00 & 364185.55 & 0.83 & 0.93 & 0.77 \\
37250 & 106724 & 2002 & 1059.60 & -0.12 & 159188.00 & 1047864.25 & 0.67 & 0.99 & 0.66 \\
18657 & 102500 & 2002 & 91.60 & -0.32 & 9171.00 & 88432.61 & 1.00 & 0.97 & 0.96 \\
19982 & 102660 & 2002 & 5020.40 & -0.24 & 501771.00 & 4887868.33 & 1.00 & 0.97 & 0.97 \\
37672 & 106993 & 2002 & 62.50 & -0.09 & 8028.00 & 64552.79 & 0.78 & 1.03 & 0.80 \\
37276 & 106725 & 2002 & 8.60 & -0.30 & 640.00 & 7032.31 & 1.34 & 0.82 & 1.10 \\
5326 & 100753 & 2002 & 931.30 & -0.19 & 90191.00 & 901910.96 & 1.03 & 0.97 & 1.00 \\
27729 & 105320 & 2002 & 144.30 & -0.07 & 13922.00 & 135209.61 & 1.04 & 0.94 & 0.97 \\
23219 & 103145 & 2002 & 54.10 & -0.18 & 5331.00 & 53229.81 & 1.01 & 0.98 & 1.00 \\
8431 & 101086 & 2002 & 320.90 & -0.48 & 59513.00 & 315901.79 & 0.54 & 0.98 & 0.53 \\
22894 & 103084 & 2002 & 382.70 & -0.07 & 36718.00 & 374622.20 & 1.04 & 0.98 & 1.02 \\
12884 & 101603 & 2002 & 1572.70 & -0.14 & 157385.00 & 1521116.31 & 1.00 & 0.97 & 0.97 \\
2508 & 100336 & 2002 & 95.10 & -0.17 & 7698.00 & 83238.86 & 1.24 & 0.88 & 1.08 \\
4455 & 100633 & 2002 & 417.70 & -0.25 & 42147.00 & 432189.43 & 0.99 & 1.03 & 1.03 \\
22863 & 103073 & 2002 & 508.90 & -0.14 & 70412.00 & 698426.63 & 0.72 & 1.37 & 0.99 \\
9511 & 101141 & 2002 & 2431.00 & -0.20 & 300220.00 & 2294692.19 & 0.81 & 0.94 & 0.76 \\
26523 & 103590 & 2002 & 822.20 & -0.23 & 82251.00 & 731606.73 & 1.00 & 0.89 & 0.89 \\
11402 & 101400 & 2002 & 241.40 & -0.11 & 29478.00 & 235611.87 & 0.82 & 0.98 & 0.80 \\
23828 & 103214 & 2002 & 1474.30 & -0.25 & 145725.00 & 1421886.31 & 1.01 & 0.96 & 0.98 \\
17476 & 102312 & 2002 & 71.70 & -0.31 & 7158.00 & 71155.50 & 1.00 & 0.99 & 0.99 \\
26294 & 103564 & 2002 & 394.80 & -0.28 & 39368.00 & 411578.49 & 1.00 & 1.04 & 1.05 \\
63805 & 500556 & 2002 & 54.00 & -0.16 & 4251.00 & 43789.52 & 1.27 & 0.81 & 1.03 \\
526 & 100072 & 2002 & 7267.50 & -0.25 & 718851.00 & 7410944.11 & 1.01 & 1.02 & 1.03 \\
37683 & 106995 & 2002 & 1155.70 & -0.20 & 148840.00 & 1117772.64 & 0.78 & 0.97 & 0.75 \\
27773 & 105322 & 2002 & 21.80 & -0.28 & 2093.00 & 20930.85 & 1.04 & 0.96 & 1.00 \\
49924 & 240375 & 2002 & 14.30 & -0.14 & 1434.00 & 13454.34 & 1.00 & 0.94 & 0.94 \\
38508 & 107266 & 2002 & 201.40 & -0.18 & 22941.00 & 203972.79 & 0.88 & 1.01 & 0.89 \\
26435 & 103580 & 2002 & 250.00 & -0.23 & 25142.00 & 230731.73 & 0.99 & 0.92 & 0.92 \\
27403 & 105276 & 2002 & 1187.50 & -0.17 & 118439.00 & 1123673.27 & 1.00 & 0.95 & 0.95 \\
49754 & 240354 & 2002 & 16.90 & -0.13 & 1688.00 & 16249.79 & 1.00 & 0.96 & 0.96 \\
5357 & 100757 & 2002 & 2.30 & -0.24 & 221.00 & 2074.44 & 1.04 & 0.90 & 0.94 \\
38425 & 107258 & 2002 & 94.30 & -0.25 & 8315.00 & 85628.53 & 1.13 & 0.91 & 1.03 \\
41135 & 108203 & 2002 & 77.80 & -0.34 & 4467.00 & 44669.05 & 1.74 & 0.57 & 1.00 \\
55736 & 400144 & 2002 & 107.60 & -0.27 & 11337.00 & 109584.39 & 0.95 & 1.02 & 0.97 \\
2400 & 100322 & 2002 & 358.30 & -0.14 & 35595.00 & 336951.61 & 1.01 & 0.94 & 0.95 \\
37551 & 106968 & 2002 & 35.50 & -0.40 & 3566.00 & 35308.59 & 1.00 & 0.99 & 0.99 \\
23189 & 103144 & 2002 & 18.70 & -0.23 & 2150.00 & 18767.60 & 0.87 & 1.00 & 0.87 \\
14450 & 101858 & 2002 & 1085.70 & -0.25 & 130938.00 & 916381.34 & 0.83 & 0.84 & 0.70 \\
54426 & 367713 & 2002 & 1.30 & -0.07 & 105.00 & 1053.92 & 1.24 & 0.81 & 1.00 \\
37576 & 106969 & 2002 & 32.30 & -0.16 & 3224.00 & 31918.22 & 1.00 & 0.99 & 0.99 \\
38400 & 107257 & 2002 & 269.30 & -0.18 & 26972.00 & 262706.89 & 1.00 & 0.98 & 0.97 \\
40714 & 108146 & 2002 & 28.70 & -0.49 & 2870.00 & 28378.65 & 1.00 & 0.99 & 0.99 \\
37543 & 106962 & 2002 & 4.30 & -0.27 & 428.00 & 4131.45 & 1.00 & 0.96 & 0.97 \\
54935 & 400029 & 2002 & 2.60 & -0.17 & NaN & 2202.61 & 1.00 & 0.85 & 1.00 \\
37515 & 106944 & 2002 & 30.90 & -0.26 & 3095.00 & 29795.30 & 1.00 & 0.96 & 0.96 \\
38434 & 107259 & 2002 & 324.60 & -0.13 & 29274.00 & 290822.48 & 1.11 & 0.90 & 0.99 \\
53585 & 354930 & 2002 & 51.50 & -0.07 & 4055.00 & 34243.68 & 1.27 & 0.66 & 0.84 \\
8399 & 101085 & 2002 & 183.10 & -0.30 & 28611.00 & 175293.98 & 0.64 & 0.96 & 0.61 \\
37524 & 106948 & 2002 & 238.60 & -0.14 & 23737.00 & 224081.76 & 1.01 & 0.94 & 0.94 \\
2101 & 100291 & 2002 & 698.30 & -0.18 & 68224.00 & 682243.48 & 1.02 & 0.98 & 1.00 \\
12 & 100001 & 2002 & 2781.30 & -0.23 & 277935.00 & 2754078.45 & 1.00 & 0.99 & 0.99 \\
54430 & 367841 & 2002 & 179.40 & -0.02 & 18393.00 & 150445.12 & 0.98 & 0.84 & 0.82 \\
64259 & 500594 & 2002 & 789.70 & -0.12 & 89866.00 & 812154.92 & 0.88 & 1.03 & 0.90 \\
23104 & 103122 & 2002 & 232.00 & -0.36 & 23201.00 & 225268.06 & 1.00 & 0.97 & 0.97 \\
27432 & 105278 & 2002 & 596.30 & -0.33 & 60489.00 & 585962.26 & 0.99 & 0.98 & 0.97 \\
5371 & 100758 & 2002 & 50.60 & -0.14 & 4642.00 & 48121.35 & 1.09 & 0.95 & 1.04 \\
41187 & 108670 & 2002 & 229.30 & -0.28 & 23068.00 & 224604.16 & 0.99 & 0.98 & 0.97 \\
33148 & 106097 & 2002 & 281.70 & -0.23 & 29017.00 & 281429.61 & 0.97 & 1.00 & 0.97 \\
37536 & 106961 & 2002 & 88.40 & -0.16 & 8228.00 & 80158.97 & 1.07 & 0.91 & 0.97 \\
15116 & 101958 & 2002 & 589.00 & -0.23 & 59171.00 & 589417.27 & 1.00 & 1.00 & 1.00 \\
18592 & 102490 & 2002 & 52.50 & -0.24 & 4936.00 & 53235.32 & 1.06 & 1.01 & 1.08 \\
40739 & 108147 & 2002 & 11.00 & -0.23 & 1089.00 & 10809.19 & 1.01 & 0.98 & 0.99 \\
63936 & 500568 & 2002 & 6.60 & -0.28 & 1210.00 & 6715.30 & 0.55 & 1.02 & 0.55 \\
37601 & 106972 & 2002 & 39.20 & -0.18 & 3921.00 & 38602.04 & 1.00 & 0.98 & 0.98 \\
49556 & 240312 & 2002 & 48.20 & -0.11 & 3884.00 & 39028.21 & 1.24 & 0.81 & 1.00 \\
23726 & 103209 & 2002 & 133.40 & -0.12 & 13669.00 & 117754.06 & 0.98 & 0.88 & 0.86 \\
63949 & 500571 & 2002 & 95.60 & -0.32 & 9623.00 & 94822.40 & 0.99 & 0.99 & 0.99 \\
26486 & 103582 & 2002 & 11.70 & -0.36 & 1178.00 & 11284.48 & 0.99 & 0.96 & 0.96 \\
27340 & 105269 & 2002 & 776.80 & -0.11 & 83519.00 & 700599.35 & 0.93 & 0.90 & 0.84 \\
23160 & 103136 & 2002 & 192.40 & -0.26 & 19271.00 & 187160.97 & 1.00 & 0.97 & 0.97 \\
45883 & 200153 & 2002 & 54.40 & -0.23 & 5453.00 & 53709.25 & 1.00 & 0.99 & 0.98 \\
32547 & 106039 & 2002 & 1331.20 & -0.16 & 132931.00 & 1255407.80 & 1.00 & 0.94 & 0.94 \\
64003 & 500576 & 2002 & 120.50 & -0.03 & 14404.00 & 104060.82 & 0.84 & 0.86 & 0.72 \\
49763 & 240358 & 2002 & 19.50 & -0.23 & 2008.00 & 18084.51 & 0.97 & 0.93 & 0.90 \\
49000 & 240198 & 2002 & 464.50 & -0.24 & 45088.00 & 449481.10 & 1.03 & 0.97 & 1.00 \\
41112 & 108202 & 2002 & 31.60 & -0.37 & 4040.00 & 29825.96 & 0.78 & 0.94 & 0.74 \\
49760 & 240356 & 2002 & 8.80 & -0.37 & 844.00 & 8444.24 & 1.04 & 0.96 & 1.00 \\
37642 & 106983 & 2002 & 137.80 & -0.27 & 15069.00 & 130628.50 & 0.91 & 0.95 & 0.87 \\
41100 & 108200 & 2002 & 111.20 & -0.26 & 11055.00 & 110225.07 & 1.01 & 0.99 & 1.00 \\
11562 & 101430 & 2002 & 108.20 & -0.05 & 10818.00 & 105787.72 & 1.00 & 0.98 & 0.98 \\
4017 & 100538 & 2002 & 405.30 & -0.25 & 39361.00 & 394061.33 & 1.03 & 0.97 & 1.00 \\
49757 & 240355 & 2002 & 2.90 & -0.30 & 282.00 & 2821.91 & 1.03 & 0.97 & 1.00 \\
12806 & 101600 & 2002 & 1634.00 & -0.23 & 169932.00 & 1576605.11 & 0.96 & 0.96 & 0.93 \\
18580 & 102489 & 2002 & 57.70 & -0.25 & 5728.00 & 57452.66 & 1.01 & 1.00 & 1.00 \\
27373 & 105275 & 2002 & 57.30 & -0.22 & 5774.00 & 56779.14 & 0.99 & 0.99 & 0.98 \\
5348 & 100754 & 2002 & 455.40 & -0.25 & 46572.00 & 441831.35 & 0.98 & 0.97 & 0.95 \\
23144 & 103134 & 2002 & 339.40 & -0.23 & 33913.00 & 329633.46 & 1.00 & 0.97 & 0.97 \\
52410 & 302881 & 2002 & 150.50 & -0.23 & 14984.00 & 146661.50 & 1.00 & 0.97 & 0.98 \\
865 & 100099 & 2002 & 63.70 & -0.27 & 6370.00 & 59238.20 & 1.00 & 0.93 & 0.93 \\
13980 & 101794 & 2002 & 1027.70 & -0.27 & 100859.00 & 1003562.74 & 1.02 & 0.98 & 1.00 \\
17085 & 102255 & 2002 & 7.90 & -0.23 & 795.00 & 7721.41 & 0.99 & 0.98 & 0.97 \\
40743 & 108148 & 2002 & 32.20 & -0.20 & 3217.00 & 32291.72 & 1.00 & 1.00 & 1.00 \\
49533 & 240311 & 2002 & 9.10 & -0.37 & 979.00 & 8331.69 & 0.93 & 0.92 & 0.85 \\
27362 & 105271 & 2002 & 32.60 & -0.23 & 3251.00 & 30587.81 & 1.00 & 0.94 & 0.94 \\
27315 & 105268 & 2002 & 513.50 & -0.16 & 51276.00 & 493905.15 & 1.00 & 0.96 & 0.96 \\
45891 & 200156 & 2002 & 37.00 & -0.24 & 3597.00 & 38371.33 & 1.03 & 1.04 & 1.07 \\
48387 & 240074 & 2002 & 27.10 & -0.12 & 2861.00 & 23064.84 & 0.95 & 0.85 & 0.81 \\
6942 & 100973 & 2002 & 53.70 & -0.22 & 5406.00 & 51515.71 & 0.99 & 0.96 & 0.95 \\
53581 & 354336 & 2002 & 13.90 & -0.27 & 1366.00 & 13321.12 & 1.02 & 0.96 & 0.98 \\
49774 & 240359 & 2002 & 3.70 & -0.20 & 368.00 & 3311.85 & 1.01 & 0.90 & 0.90 \\
41212 & 108673 & 2002 & 104.90 & -0.23 & 11536.00 & 110411.01 & 0.91 & 1.05 & 0.96 \\
598 & 100079 & 2002 & 1436.80 & -0.30 & 139061.00 & 1318293.58 & 1.03 & 0.92 & 0.95 \\
45911 & 200164 & 2002 & 12.90 & -0.23 & 1140.00 & 11240.54 & 1.13 & 0.87 & 0.99 \\
573 & 100076 & 2002 & 499.70 & -0.30 & 52491.00 & 487005.22 & 0.95 & 0.97 & 0.93 \\
27535 & 105286 & 2002 & 237.00 & -0.31 & 20386.00 & 203864.44 & 1.16 & 0.86 & 1.00 \\
5404 & 100760 & 2002 & 418.30 & -0.25 & 44580.00 & 393193.61 & 0.94 & 0.94 & 0.88 \\
12840 & 101602 & 2002 & 1934.50 & -0.19 & 196984.00 & 1833769.80 & 0.98 & 0.95 & 0.93 \\
53562 & 351750 & 2002 & 10.90 & -0.31 & 1088.00 & 10517.74 & 1.00 & 0.96 & 0.97 \\
27544 & 105287 & 2002 & 94.60 & -0.42 & 8030.00 & 90944.62 & 1.18 & 0.96 & 1.13 \\
53564 & 351891 & 2002 & 25.90 & -0.20 & 2597.00 & 24292.63 & 1.00 & 0.94 & 0.94 \\
44650 & 109351 & 2002 & 25.30 & -0.22 & 2526.00 & 25615.05 & 1.00 & 1.01 & 1.01 \\
37455 & 106919 & 2002 & 31.00 & -0.18 & 3095.00 & 30948.10 & 1.00 & 1.00 & 1.00 \\
40689 & 108145 & 2002 & 52.10 & -0.11 & 5228.00 & 51970.64 & 1.00 & 1.00 & 0.99 \\
41094 & 108197 & 2002 & 15.20 & -0.23 & 1510.00 & 14840.79 & 1.01 & 0.98 & 0.98 \\
23033 & 103103 & 2002 & 270.40 & -0.25 & 27098.00 & 270992.66 & 1.00 & 1.00 & 1.00 \\
32519 & 106038 & 2002 & 992.80 & -0.07 & 95965.00 & 869714.21 & 1.03 & 0.88 & 0.91 \\
11846 & 101463 & 2002 & 620.50 & -0.22 & 86611.00 & 633673.80 & 0.72 & 1.02 & 0.73 \\
44642 & 109350 & 2002 & 21.40 & -0.14 & 2145.00 & 21319.42 & 1.00 & 1.00 & 0.99 \\
37423 & 106896 & 2002 & 55.20 & -0.33 & 5879.00 & 61648.16 & 0.94 & 1.12 & 1.05 \\
895 & 100101 & 2002 & 460.30 & -0.36 & 46025.00 & 443819.83 & 1.00 & 0.96 & 0.96 \\
23002 & 103101 & 2002 & 179.90 & -0.10 & 17962.00 & 170922.97 & 1.00 & 0.95 & 0.95 \\
26377 & 103572 & 2002 & 25.30 & -0.20 & 2631.00 & 24109.21 & 0.96 & 0.95 & 0.92 \\
37661 & 106992 & 2002 & 180.40 & -0.24 & 24318.00 & 183349.90 & 0.74 & 1.02 & 0.75 \\
55461 & 400100 & 2002 & 63.30 & -0.18 & 6367.00 & 58140.26 & 0.99 & 0.92 & 0.91 \\
11486 & 101422 & 2002 & 45.60 & -0.27 & 4559.00 & 45148.58 & 1.00 & 0.99 & 0.99 \\
40664 & 108144 & 2002 & 135.10 & -0.46 & 13530.00 & 129084.97 & 1.00 & 0.96 & 0.95 \\
53545 & 351713 & 2002 & 195.80 & -0.25 & 19777.00 & 188423.42 & 0.99 & 0.96 & 0.95 \\
41279 & 108719 & 2002 & 162.30 & -0.28 & 17228.00 & 163808.23 & 0.94 & 1.01 & 0.95 \\
47949 & 225413 & 2002 & 91.20 & -0.22 & 7499.00 & 68118.42 & 1.22 & 0.75 & 0.91 \\
48339 & 240063 & 2002 & 371.40 & -0.44 & 36672.00 & 341175.65 & 1.01 & 0.92 & 0.93 \\
48957 & 240174 & 2002 & 139.80 & -0.20 & 13786.00 & 131206.20 & 1.01 & 0.94 & 0.95 \\
41275 & 108716 & 2002 & 19.00 & -0.33 & 1919.00 & 18835.21 & 0.99 & 0.99 & 0.98 \\
41091 & 108194 & 2002 & 21.90 & -0.14 & 2144.00 & 22409.83 & 1.02 & 1.02 & 1.05 \\
2443 & 100330 & 2002 & 1944.80 & -0.20 & 262056.00 & 2023956.86 & 0.74 & 1.04 & 0.77 \\
6923 & 100969 & 2002 & 44.20 & -0.20 & 5438.00 & 40464.09 & 0.81 & 0.92 & 0.74 \\
4491 & 100635 & 2002 & 783.70 & 0.03 & 76597.00 & 728511.03 & 1.02 & 0.93 & 0.95 \\
37648 & 106984 & 2002 & 31.40 & -0.09 & 2077.00 & 20553.89 & 1.51 & 0.65 & 0.99 \\
49799 & 240362 & 2002 & 1.40 & -0.15 & NaN & 840.00 & 1.00 & 0.60 & 1.00 \\
37489 & 106934 & 2002 & 573.30 & -0.40 & 57584.00 & 520704.75 & 1.00 & 0.91 & 0.90 \\
54438 & 367842 & 2002 & 46.20 & -0.15 & 4595.00 & 44025.86 & 1.01 & 0.95 & 0.96 \\
44756 & 109373 & 2002 & 64.20 & -0.15 & 5372.00 & 55389.86 & 1.20 & 0.86 & 1.03 \\
38459 & 107260 & 2002 & 393.80 & 0.15 & 39778.00 & 334059.72 & 0.99 & 0.85 & 0.84 \\
23076 & 103110 & 2002 & 758.90 & -0.15 & 76225.00 & 699787.00 & 1.00 & 0.92 & 0.92 \\
15000 & 101930 & 2002 & 811.30 & -0.25 & 81142.00 & 793858.66 & 1.00 & 0.98 & 0.98 \\
27477 & 105281 & 2002 & 314.80 & 0.06 & 33025.00 & 284567.24 & 0.95 & 0.90 & 0.86 \\
17042 & 102231 & 2002 & 455.20 & -0.36 & 46287.00 & 449043.11 & 0.98 & 0.99 & 0.97 \\
17439 & 102306 & 2002 & 16538.40 & -0.17 & 1653943.00 & 13948721.22 & 1.00 & 0.84 & 0.84 \\
49783 & 240360 & 2002 & 473.60 & -0.17 & 39188.00 & 408055.22 & 1.21 & 0.86 & 1.04 \\
49013 & 240199 & 2002 & 300.30 & -0.28 & 51297.00 & 418828.16 & 0.59 & 1.39 & 0.82 \\
27468 & 105280 & 2002 & 30.60 & -0.20 & 3075.00 & 28572.24 & 1.00 & 0.93 & 0.93 \\
19952 & 102659 & 2002 & 3359.40 & -0.27 & 337598.00 & 3313384.26 & 1.00 & 0.99 & 0.98 \\
45909 & 200162 & 2002 & 6.10 & -0.25 & 663.00 & 5222.64 & 0.92 & 0.86 & 0.79 \\
9498 & 101140 & 2002 & 446.70 & -0.41 & 61448.00 & 512426.29 & 0.73 & 1.15 & 0.83 \\
55205 & 400072 & 2002 & 331.40 & -0.20 & 33386.00 & 332268.82 & 0.99 & 1.00 & 1.00 \\
12827 & 101601 & 2002 & 136.50 & -0.27 & 13864.00 & 134604.09 & 0.98 & 0.99 & 0.97 \\
53573 & 354018 & 2002 & 57.30 & -0.22 & 4873.00 & 51114.97 & 1.18 & 0.89 & 1.05 \\
26403 & 103579 & 2002 & 337.50 & -0.26 & 33842.00 & 292629.29 & 1.00 & 0.87 & 0.86 \\
41267 & 108710 & 2002 & 362.70 & -0.50 & 35698.00 & 342052.43 & 1.02 & 0.94 & 0.96 \\
18608 & 102491 & 2002 & 439.80 & -0.21 & 43646.00 & 432518.40 & 1.01 & 0.98 & 0.99 \\
43441 & 109121 & 2002 & 5.90 & -0.56 & 580.00 & 5542.89 & 1.02 & 0.94 & 0.96 \\
37655 & 106985 & 2002 & 61.60 & -0.24 & 3991.00 & 37570.04 & 1.54 & 0.61 & 0.94 \\
27507 & 105283 & 2002 & 6.30 & -0.23 & 861.00 & 6388.07 & 0.73 & 1.01 & 0.74 \\
57887 & 401363 & 2002 & 2.60 & -0.53 & 226.00 & 2710.42 & 1.15 & 1.04 & 1.20 \\
2420 & 100323 & 2002 & 2736.10 & -0.27 & 278920.00 & 2722494.50 & 0.98 & 1.00 & 0.98 \\
5116 & 100724 & 2002 & 21.10 & -0.27 & 2111.00 & 21105.75 & 1.00 & 1.00 & 1.00 \\
57890 & 401368 & 2002 & 11.60 & -0.36 & 1155.00 & 11060.06 & 1.00 & 0.95 & 0.96 \\
11508 & 101425 & 2002 & 16.40 & -0.27 & 1643.00 & 14751.97 & 1.00 & 0.90 & 0.90 \\
38484 & 107263 & 2002 & 946.80 & -0.14 & 109999.00 & 892667.97 & 0.86 & 0.94 & 0.81 \\
41239 & 108690 & 2002 & 54.80 & 0.61 & 5702.00 & 55089.26 & 0.96 & 1.01 & 0.97 \\
15177 & 101964 & 2002 & 850.30 & -0.08 & 83576.00 & 825244.09 & 1.02 & 0.97 & 0.99 \\
23762 & 103212 & 2002 & 2611.50 & -0.27 & 259843.00 & 2271853.25 & 1.01 & 0.87 & 0.87 \\
37480 & 106931 & 2002 & 323.90 & -0.52 & 31643.00 & 316446.77 & 1.02 & 0.98 & 1.00 \\
41723 & 108841 & 2002 & 86.00 & -0.29 & 8090.00 & 76692.52 & 1.06 & 0.89 & 0.95 \\
27863 & 105335 & 2002 & 206.40 & -0.17 & 20872.00 & 180404.97 & 0.99 & 0.87 & 0.86 \\
40793 & 108153 & 2002 & 13.80 & -0.11 & 1358.00 & 11648.84 & 1.02 & 0.84 & 0.86 \\
14569 & 101885 & 2002 & 1393.60 & -0.20 & 139210.00 & 1228105.04 & 1.00 & 0.88 & 0.88 \\
38753 & 107320 & 2002 & 49.80 & -0.49 & 7216.00 & 62851.68 & 0.69 & 1.26 & 0.87 \\
28237 & 105397 & 2002 & 55.80 & -0.04 & 5529.00 & 54245.81 & 1.01 & 0.97 & 0.98 \\
22571 & 103021 & 2002 & 83.30 & -0.12 & 8227.00 & 82282.25 & 1.01 & 0.99 & 1.00 \\
41058 & 108185 & 2002 & 1.60 & -0.45 & 150.00 & 1488.95 & 1.07 & 0.93 & 0.99 \\
43377 & 109108 & 2002 & 30.70 & -0.16 & 3095.00 & 30192.81 & 0.99 & 0.98 & 0.98 \\
41622 & 108782 & 2002 & 452.70 & -0.18 & 46728.00 & 446799.64 & 0.97 & 0.99 & 0.96 \\
16744 & 102183 & 2002 & 290.90 & -0.22 & 26507.00 & 214261.73 & 1.10 & 0.74 & 0.81 \\
47977 & 225484 & 2002 & 42.60 & -0.26 & 4290.00 & 42241.04 & 0.99 & 0.99 & 0.98 \\
28229 & 105394 & 2002 & 8.70 & -0.33 & 929.00 & 8773.89 & 0.94 & 1.01 & 0.94 \\
38749 & 107319 & 2002 & 40.60 & -0.35 & 5383.00 & 45853.72 & 0.75 & 1.13 & 0.85 \\
74760 & 601162 & 2002 & 136.70 & -0.05 & 13625.00 & 133853.40 & 1.00 & 0.98 & 0.98 \\
15098 & 101956 & 2002 & 1452.60 & -0.32 & 145985.00 & 1430300.38 & 1.00 & 0.98 & 0.98 \\
41608 & 108780 & 2002 & 53.90 & -0.25 & 9260.00 & 56521.77 & 0.58 & 1.05 & 0.61 \\
40533 & 108136 & 2002 & 11.90 & -0.17 & 833.00 & 7020.75 & 1.43 & 0.59 & 0.84 \\
2671 & 100351 & 2002 & 63.80 & -0.23 & 6379.00 & 61777.24 & 1.00 & 0.97 & 0.97 \\
50090 & 240392 & 2002 & 250.90 & -0.19 & 22952.00 & 227834.63 & 1.09 & 0.91 & 0.99 \\
43532 & 109142 & 2002 & 122.90 & -0.27 & 12520.00 & 130191.79 & 0.98 & 1.06 & 1.04 \\
12573 & 101554 & 2002 & 96.00 & -0.20 & 9275.00 & 91187.78 & 1.04 & 0.95 & 0.98 \\
16727 & 102182 & 2002 & 123.80 & -0.24 & 13011.00 & 109927.99 & 0.95 & 0.89 & 0.84 \\
36818 & 106597 & 2002 & 131.30 & -0.15 & 11841.00 & 128228.48 & 1.11 & 0.98 & 1.08 \\
26108 & 103539 & 2002 & 554.60 & -0.23 & 78063.00 & 514520.01 & 0.71 & 0.93 & 0.66 \\
54551 & 375941 & 2002 & 5.70 & -0.05 & 534.00 & 5310.58 & 1.07 & 0.93 & 0.99 \\
50065 & 240391 & 2002 & 153.00 & -0.21 & 15416.00 & 147887.62 & 0.99 & 0.97 & 0.96 \\
11221 & 101376 & 2002 & 1068.40 & -0.05 & 98386.00 & 1056556.25 & 1.09 & 0.99 & 1.07 \\
52540 & 303121 & 2002 & 154.90 & -0.26 & 15505.00 & 150840.11 & 1.00 & 0.97 & 0.97 \\
54549 & 375746 & 2002 & 22.00 & -0.09 & 2180.00 & 21225.13 & 1.01 & 0.96 & 0.97 \\
36825 & 106602 & 2002 & 31.80 & -0.47 & 3649.00 & 36487.13 & 0.87 & 1.15 & 1.00 \\
40508 & 108134 & 2002 & 66.70 & -0.18 & 6696.00 & 60824.91 & 1.00 & 0.91 & 0.91 \\
38757 & 107321 & 2002 & 12.60 & -0.19 & 1606.00 & 11600.71 & 0.78 & 0.92 & 0.72 \\
14024 & 101800 & 2002 & 426.70 & -0.24 & 42666.00 & 419728.30 & 1.00 & 0.98 & 0.98 \\
52351 & 302811 & 2002 & 29.70 & -0.27 & 2734.00 & 29172.39 & 1.09 & 0.98 & 1.07 \\
28210 & 105393 & 2002 & 1788.00 & -0.16 & 178898.00 & 1692733.35 & 1.00 & 0.95 & 0.95 \\
41573 & 108776 & 2002 & 288.30 & -0.08 & 30418.00 & 271233.12 & 0.95 & 0.94 & 0.89 \\
59035 & 410401 & 2002 & 105.90 & 0.00 & 10000.00 & 102895.95 & 1.06 & 0.97 & 1.03 \\
9547 & 101149 & 2002 & 1668.50 & -0.47 & 221300.00 & 1504404.44 & 0.75 & 0.90 & 0.68 \\
36874 & 106619 & 2002 & 159.00 & -0.40 & 29824.00 & 147811.83 & 0.53 & 0.93 & 0.50 \\
54176 & 364809 & 2002 & 28.80 & -0.29 & 2844.00 & 27470.65 & 1.01 & 0.95 & 0.97 \\
28172 & 105390 & 2002 & 411.60 & -0.15 & 39836.00 & 398334.88 & 1.03 & 0.97 & 1.00 \\
7667 & 101054 & 2002 & 8695.80 & -0.22 & 1052041.00 & 8967419.58 & 0.83 & 1.03 & 0.85 \\
38725 & 107316 & 2002 & 506.30 & -0.23 & 50587.00 & 501536.33 & 1.00 & 0.99 & 0.99 \\
5567 & 100772 & 2002 & 615.60 & -0.19 & 89255.00 & 775444.00 & 0.69 & 1.26 & 0.87 \\
44506 & 109333 & 2002 & 11.00 & -0.38 & 1102.00 & 10365.12 & 1.00 & 0.94 & 0.94 \\
11262 & 101380 & 2002 & 219.70 & -0.15 & 21969.00 & 219694.83 & 1.00 & 1.00 & 1.00 \\
63524 & 500513 & 2002 & 47.70 & -0.20 & 6714.00 & 49142.33 & 0.71 & 1.03 & 0.73 \\
64374 & 500600 & 2002 & 637.50 & -0.24 & 47845.00 & 477964.35 & 1.33 & 0.75 & 1.00 \\
20017 & 102663 & 2002 & 3186.10 & -0.41 & 306025.00 & 2820082.00 & 1.04 & 0.89 & 0.92 \\
464 & 100068 & 2002 & 69.30 & -0.20 & 7449.00 & 73627.57 & 0.93 & 1.06 & 0.99 \\
5053 & 100710 & 2002 & 810.50 & -0.19 & 81014.00 & 777525.95 & 1.00 & 0.96 & 0.96 \\
2652 & 100350 & 2002 & 196.20 & -0.26 & 19645.00 & 180882.76 & 1.00 & 0.92 & 0.92 \\
65682 & 500716 & 2002 & 3.40 & -0.30 & 417.00 & 3493.15 & 0.82 & 1.03 & 0.84 \\
53446 & 350408 & 2002 & 17.70 & -0.19 & 1526.00 & 14725.77 & 1.16 & 0.83 & 0.96 \\
54226 & 364947 & 2002 & 5.90 & -0.07 & 638.00 & 5558.86 & 0.92 & 0.94 & 0.87 \\
11907 & 101465 & 2002 & 89.00 & -0.21 & 8915.00 & 85345.28 & 1.00 & 0.96 & 0.96 \\
14557 & 101876 & 2002 & 124.60 & -0.26 & 12947.00 & 127050.16 & 0.96 & 1.02 & 0.98 \\
41598 & 108777 & 2002 & 25.00 & -0.19 & 2550.00 & 25573.27 & 0.98 & 1.02 & 1.00 \\
18759 & 102507 & 2002 & 175.10 & -0.25 & 17517.00 & 171961.80 & 1.00 & 0.98 & 0.98 \\
12952 & 101616 & 2002 & 15483.40 & -0.22 & 1560316.00 & 15044255.00 & 0.99 & 0.97 & 0.96 \\
19672 & 102645 & 2002 & 281.10 & -0.16 & 36112.00 & 268067.04 & 0.78 & 0.95 & 0.74 \\
11251 & 101379 & 2002 & 639.30 & -0.22 & 59872.00 & 619073.48 & 1.07 & 0.97 & 1.03 \\
59051 & 410418 & 2002 & 871.90 & -0.23 & 162739.00 & 1339351.00 & 0.54 & 1.54 & 0.82 \\
32943 & 106083 & 2002 & 838.90 & -0.18 & 121016.00 & 850539.26 & 0.69 & 1.01 & 0.70 \\
6881 & 100967 & 2002 & 469.80 & -0.05 & 47306.00 & 405180.58 & 0.99 & 0.86 & 0.86 \\
22602 & 103024 & 2002 & 645.90 & -0.09 & 65568.00 & 605836.09 & 0.99 & 0.94 & 0.92 \\
41060 & 108186 & 2002 & 9.30 & -0.36 & 927.00 & 9181.55 & 1.00 & 0.99 & 0.99 \\
28201 & 105391 & 2002 & 9.10 & -0.29 & 847.00 & 8663.63 & 1.07 & 0.95 & 1.02 \\
23933 & 103242 & 2002 & 512.30 & -0.20 & 51468.00 & 514683.64 & 1.00 & 1.00 & 1.00 \\
18323 & 102425 & 2002 & 1465.20 & -0.26 & 205259.00 & 1254619.90 & 0.71 & 0.86 & 0.61 \\
43385 & 109110 & 2002 & 21.80 & -0.16 & 2183.00 & 21651.66 & 1.00 & 0.99 & 0.99 \\
4402 & 100622 & 2002 & 367.90 & -0.07 & 36770.00 & 362275.88 & 1.00 & 0.98 & 0.99 \\
2691 & 100352 & 2002 & 1196.10 & -0.20 & 144352.00 & 1280321.43 & 0.83 & 1.07 & 0.89 \\
28377 & 105420 & 2002 & 41.00 & -0.31 & 4749.00 & 40931.11 & 0.86 & 1.00 & 0.86 \\
41678 & 108839 & 2002 & 220.50 & -0.55 & 18363.00 & 222339.49 & 1.20 & 1.01 & 1.21 \\
55170 & 400065 & 2002 & 73.90 & -0.31 & 7384.00 & 73846.80 & 1.00 & 1.00 & 1.00 \\
36716 & 106580 & 2002 & 75.50 & -0.13 & 7551.00 & 75511.94 & 1.00 & 1.00 & 1.00 \\
64052 & 500585 & 2002 & 608.40 & -0.30 & 43623.00 & 394545.76 & 1.39 & 0.65 & 0.90 \\
38809 & 107328 & 2002 & 26.90 & -0.14 & 2306.00 & 24684.80 & 1.17 & 0.92 & 1.07 \\
5136 & 100726 & 2002 & 4728.50 & -0.23 & 467277.00 & 3766682.15 & 1.01 & 0.80 & 0.81 \\
9282 & 101127 & 2002 & 92.30 & -0.25 & 12099.00 & 93825.08 & 0.76 & 1.02 & 0.78 \\
22488 & 103015 & 2002 & 320.80 & -0.24 & 42709.00 & 317772.96 & 0.75 & 0.99 & 0.74 \\
28351 & 105419 & 2002 & 54.90 & -0.29 & 8182.00 & 55008.84 & 0.67 & 1.00 & 0.67 \\
52323 & 302763 & 2002 & 38.00 & -0.25 & 3782.00 & 36639.99 & 1.00 & 0.96 & 0.97 \\
61186 & 410904 & 2002 & 181.40 & -0.13 & 17579.00 & 175785.83 & 1.03 & 0.97 & 1.00 \\
63476 & 500511 & 2002 & 603.10 & -0.22 & 83929.00 & 625570.84 & 0.72 & 1.04 & 0.75 \\
17164 & 102261 & 2002 & 537.90 & -0.20 & 53768.00 & 524384.22 & 1.00 & 0.97 & 0.98 \\
22470 & 103014 & 2002 & 228.20 & -0.22 & 30718.00 & 246609.33 & 0.74 & 1.08 & 0.80 \\
36655 & 106573 & 2002 & 9.60 & -0.37 & 869.00 & 9503.52 & 1.10 & 0.99 & 1.09 \\
54573 & 375967 & 2002 & 9.70 & -0.14 & 967.00 & 8006.90 & 1.00 & 0.83 & 0.83 \\
4678 & 100660 & 2002 & 589.90 & -0.21 & 116571.00 & 1186676.09 & 0.51 & 2.01 & 1.02 \\
54156 & 364633 & 2002 & 1.80 & -0.11 & 179.00 & 1788.84 & 1.01 & 0.99 & 1.00 \\
41703 & 108840 & 2002 & 58.00 & -0.14 & 5647.00 & 49190.15 & 1.03 & 0.85 & 0.87 \\
50155 & 240397 & 2002 & 11.10 & -0.48 & 1126.00 & 11263.15 & 0.99 & 1.01 & 1.00 \\
48214 & 240051 & 2002 & 482.20 & -0.08 & 49370.00 & 409132.13 & 0.98 & 0.85 & 0.83 \\
36690 & 106577 & 2002 & 1534.90 & -0.16 & 158811.00 & 1274870.27 & 0.97 & 0.83 & 0.80 \\
38821 & 107329 & 2002 & 34.20 & -0.29 & 3369.00 & 34566.05 & 1.02 & 1.01 & 1.03 \\
63432 & 500508 & 2002 & 4727.60 & -0.15 & 273012.00 & 4059682.68 & 1.73 & 0.86 & 1.49 \\
49443 & 240300 & 2002 & 28.00 & -0.40 & 2704.00 & 27034.90 & 1.04 & 0.97 & 1.00 \\
50151 & 240396 & 2002 & 9.50 & -0.48 & 956.00 & 9459.75 & 0.99 & 1.00 & 0.99 \\
24236 & 103299 & 2002 & 258.30 & -0.30 & 32207.00 & 258625.63 & 0.80 & 1.00 & 0.80 \\
4358 & 100611 & 2002 & 457.50 & -0.43 & 45817.00 & 438423.76 & 1.00 & 0.96 & 0.96 \\
26061 & 103536 & 2002 & 1762.50 & -0.22 & 227013.00 & 1798019.74 & 0.78 & 1.02 & 0.79 \\
36681 & 106574 & 2002 & 16.20 & -0.23 & 1571.00 & 15374.93 & 1.03 & 0.95 & 0.98 \\
16777 & 102191 & 2002 & 49.80 & -0.15 & 4972.00 & 44402.05 & 1.00 & 0.89 & 0.89 \\
28339 & 105416 & 2002 & 2469.40 & -0.20 & 248502.00 & 2438619.27 & 0.99 & 0.99 & 0.98 \\
16693 & 102178 & 2002 & 588.70 & -0.28 & 89616.00 & 555914.79 & 0.66 & 0.94 & 0.62 \\
43674 & 109189 & 2002 & 539.90 & -0.17 & 55327.00 & 535274.70 & 0.98 & 0.99 & 0.97 \\
47990 & 225687 & 2002 & 263.30 & -0.55 & 26338.00 & 256262.23 & 1.00 & 0.97 & 0.97 \\
74768 & 601164 & 2002 & 4.10 & -0.25 & 405.00 & 3376.04 & 1.01 & 0.82 & 0.83 \\
5625 & 100775 & 2002 & 457.50 & -0.23 & 42565.00 & 464064.26 & 1.07 & 1.01 & 1.09 \\
43372 & 109104 & 2002 & 25.80 & -0.38 & 2590.00 & 24866.05 & 1.00 & 0.96 & 0.96 \\
36774 & 106590 & 2002 & 43.80 & -0.25 & 4511.00 & 38845.86 & 0.97 & 0.89 & 0.86 \\
65661 & 500712 & 2002 & 8.80 & -0.26 & 663.00 & 5450.86 & 1.33 & 0.62 & 0.82 \\
32406 & 106023 & 2002 & 145.40 & -0.23 & 14999.00 & 143443.20 & 0.97 & 0.99 & 0.96 \\
65666 & 500713 & 2002 & 153.40 & -0.09 & 15309.00 & 144276.33 & 1.00 & 0.94 & 0.94 \\
36794 & 106595 & 2002 & 10.70 & -0.11 & 952.00 & 7870.41 & 1.12 & 0.74 & 0.83 \\
41641 & 108805 & 2002 & 3.30 & -0.56 & 335.00 & 3349.12 & 0.99 & 1.01 & 1.00 \\
18777 & 102508 & 2002 & 105.60 & -0.33 & 10558.00 & 102600.92 & 1.00 & 0.97 & 0.97 \\
65671 & 500714 & 2002 & 45.40 & -0.08 & 4544.00 & 40910.12 & 1.00 & 0.90 & 0.90 \\
32703 & 106050 & 2002 & 344.70 & -0.15 & 34069.00 & 351183.78 & 1.01 & 1.02 & 1.03 \\
659 & 100087 & 2002 & 2566.30 & -0.32 & 258368.00 & 2651093.89 & 0.99 & 1.03 & 1.03 \\
4384 & 100614 & 2002 & 1212.40 & -0.13 & 121152.00 & 1126835.23 & 1.00 & 0.93 & 0.93 \\
28308 & 105401 & 2002 & 18.00 & -0.22 & 1728.00 & 17272.84 & 1.04 & 0.96 & 1.00 \\
38761 & 107322 & 2002 & 11.50 & -0.10 & 1322.00 & 11071.09 & 0.87 & 0.96 & 0.84 \\
7148 & 100998 & 2002 & 68.70 & -0.21 & 6795.00 & 65961.94 & 1.01 & 0.96 & 0.97 \\
53440 & 349609 & 2002 & 18.00 & -0.25 & 1631.00 & 16208.55 & 1.10 & 0.90 & 0.99 \\
6458 & 100875 & 2002 & 111.30 & -0.03 & 12169.00 & 109369.95 & 0.91 & 0.98 & 0.90 \\
22503 & 103016 & 2002 & 246.20 & -0.15 & 30670.00 & 258225.80 & 0.80 & 1.05 & 0.84 \\
40482 & 108122 & 2002 & 131.70 & 0.05 & 13039.00 & 121869.21 & 1.01 & 0.93 & 0.93 \\
38784 & 107323 & 2002 & 78.50 & -0.14 & 7135.00 & 66177.53 & 1.10 & 0.84 & 0.93 \\
17551 & 102318 & 2002 & 2139.40 & -0.22 & 193510.00 & 1820959.38 & 1.11 & 0.85 & 0.94 \\
28325 & 105412 & 2002 & 123.60 & -0.29 & 16508.00 & 119868.33 & 0.75 & 0.97 & 0.73 \\
44501 & 109332 & 2002 & 24.70 & -0.48 & 2465.00 & 23111.10 & 1.00 & 0.94 & 0.94 \\
43445 & 109122 & 2002 & 5.60 & -0.28 & 555.00 & 5470.87 & 1.01 & 0.98 & 0.99 \\
55177 & 400066 & 2002 & 118.90 & -0.13 & 11876.00 & 118743.06 & 1.00 & 1.00 & 1.00 \\
5650 & 100784 & 2002 & 14036.30 & -0.11 & 1391126.00 & 12897567.19 & 1.01 & 0.92 & 0.93 \\
23963 & 103251 & 2002 & 243.70 & -0.32 & 23675.00 & 236746.63 & 1.03 & 0.97 & 1.00 \\
8532 & 101089 & 2002 & 106.80 & -0.06 & 14197.00 & 103069.72 & 0.75 & 0.97 & 0.73 \\
43369 & 109102 & 2002 & 11.80 & -0.22 & 1146.00 & 11021.45 & 1.03 & 0.93 & 0.96 \\
41039 & 108183 & 2002 & 61.30 & -0.27 & 6382.00 & 58663.09 & 0.96 & 0.96 & 0.92 \\
41645 & 108826 & 2002 & 188.50 & -0.11 & 19229.00 & 195218.49 & 0.98 & 1.04 & 1.02 \\
41670 & 108827 & 2002 & 564.80 & -0.17 & 57196.00 & 520692.93 & 0.99 & 0.92 & 0.91 \\
45546 & 200074 & 2002 & 32.80 & -0.24 & 5220.00 & 32705.17 & 0.63 & 1.00 & 0.63 \\
27212 & 105253 & 2002 & 17.90 & -0.26 & 1707.00 & 16148.14 & 1.05 & 0.90 & 0.95 \\
6595 & 100900 & 2002 & 64.00 & -0.10 & 5589.00 & 47791.38 & 1.15 & 0.75 & 0.86 \\
53490 & 351048 & 2002 & 57.40 & 0.01 & 7518.00 & 67673.45 & 0.76 & 1.18 & 0.90 \\
47839 & 222351 & 2002 & 258.30 & -0.06 & 25919.00 & 255412.44 & 1.00 & 0.99 & 0.99 \\
5508 & 100769 & 2002 & 2909.30 & -0.09 & 335464.00 & 3432193.95 & 0.87 & 1.18 & 1.02 \\
41461 & 108760 & 2002 & 78.50 & -0.21 & 8653.00 & 83962.43 & 0.91 & 1.07 & 0.97 \\
33239 & 106107 & 2002 & 30.50 & -0.23 & 4080.00 & 24699.94 & 0.75 & 0.81 & 0.61 \\
44580 & 109343 & 2002 & 59.10 & -0.16 & 6105.00 & 50549.75 & 0.97 & 0.86 & 0.83 \\
52448 & 302942 & 2002 & 491.50 & -0.15 & 47362.00 & 479306.61 & 1.04 & 0.98 & 1.01 \\
38632 & 107302 & 2002 & 255.60 & -0.23 & 25567.00 & 253444.78 & 1.00 & 0.99 & 0.99 \\
27934 & 105353 & 2002 & 50.10 & -0.10 & 4966.00 & 49318.81 & 1.01 & 0.98 & 0.99 \\
43503 & 109129 & 2002 & 114.20 & -0.24 & 11415.00 & 113502.01 & 1.00 & 0.99 & 0.99 \\
41437 & 108759 & 2002 & 25.80 & -0.16 & 2536.00 & 25360.53 & 1.02 & 0.98 & 1.00 \\
14278 & 101842 & 2002 & 1489.60 & -0.23 & 145357.00 & 1443816.71 & 1.02 & 0.97 & 0.99 \\
8464 & 101087 & 2002 & 349.30 & -0.17 & 41480.00 & 331640.28 & 0.84 & 0.95 & 0.80 \\
38622 & 107300 & 2002 & 51.40 & -0.30 & 5150.00 & 50806.48 & 1.00 & 0.99 & 0.99 \\
74719 & 601156 & 2002 & 74.10 & -0.28 & 7460.00 & 72287.99 & 0.99 & 0.98 & 0.97 \\
27953 & 105358 & 2002 & 1215.40 & -0.12 & 105988.00 & 897412.45 & 1.15 & 0.74 & 0.85 \\
11602 & 101431 & 2002 & 268.20 & -0.10 & 26809.00 & 257643.66 & 1.00 & 0.96 & 0.96 \\
2063 & 100287 & 2002 & 48.70 & -0.23 & 4901.00 & 46339.00 & 0.99 & 0.95 & 0.95 \\
27985 & 105364 & 2002 & 108.40 & -0.29 & 10740.00 & 106706.94 & 1.01 & 0.98 & 0.99 \\
7910 & 101065 & 2002 & 1537.30 & -0.41 & 243566.00 & 1234593.42 & 0.63 & 0.80 & 0.51 \\
37713 & 107004 & 2002 & 100.60 & -0.23 & 10335.00 & 96915.90 & 0.97 & 0.96 & 0.94 \\
49980 & 240381 & 2002 & 20.80 & 0.03 & 2321.00 & 19342.54 & 0.90 & 0.93 & 0.83 \\
63650 & 500539 & 2002 & 6.70 & -0.17 & 558.00 & 6131.06 & 1.20 & 0.92 & 1.10 \\
65741 & 500729 & 2002 & 20.30 & -0.14 & 2929.00 & 20950.80 & 0.69 & 1.03 & 0.72 \\
16849 & 102197 & 2002 & 461.40 & -0.20 & 46154.00 & 409435.63 & 1.00 & 0.89 & 0.89 \\
49480 & 240303 & 2002 & 15.10 & -0.34 & 1502.00 & 14862.32 & 1.01 & 0.98 & 0.99 \\
44571 & 109341 & 2002 & 196.90 & -0.37 & 20285.00 & 188812.59 & 0.97 & 0.96 & 0.93 \\
23879 & 103226 & 2002 & 115.80 & -0.30 & 18343.00 & 112676.66 & 0.63 & 0.97 & 0.61 \\
54484 & 372487 & 2002 & 2.10 & -0.15 & 201.00 & 1910.15 & 1.04 & 0.91 & 0.95 \\
37088 & 106675 & 2002 & 47.40 & -0.14 & 4741.00 & 43547.67 & 1.00 & 0.92 & 0.92 \\
4644 & 100659 & 2002 & 478.80 & -0.23 & 62333.00 & 642120.47 & 0.77 & 1.34 & 1.03 \\
57851 & 401145 & 2002 & 10.20 & -0.47 & 825.00 & 7983.12 & 1.24 & 0.78 & 0.97 \\
40551 & 108138 & 2002 & 48.40 & -0.43 & 4837.00 & 47799.28 & 1.00 & 0.99 & 0.99 \\
23861 & 103224 & 2002 & 98.20 & -0.25 & 16004.00 & 95315.33 & 0.61 & 0.97 & 0.60 \\
22796 & 103065 & 2002 & 146.80 & -0.35 & 14740.00 & 140388.35 & 1.00 & 0.96 & 0.95 \\
6913 & 100968 & 2002 & 216.30 & -0.09 & 20968.00 & 201506.08 & 1.03 & 0.93 & 0.96 \\
48106 & 240010 & 2002 & 236.40 & 0.03 & 23632.00 & 224616.77 & 1.00 & 0.95 & 0.95 \\
37129 & 106692 & 2002 & 281.70 & -0.23 & 51679.00 & 479729.84 & 0.55 & 1.70 & 0.93 \\
6507 & 100878 & 2002 & 1986.90 & -0.20 & 198568.00 & 1922305.06 & 1.00 & 0.97 & 0.97 \\
17489 & 102313 & 2002 & 166.00 & -0.32 & 16306.00 & 163055.57 & 1.02 & 0.98 & 1.00 \\
41425 & 108752 & 2002 & 8.80 & -0.22 & 1191.00 & 11551.99 & 0.74 & 1.31 & 0.97 \\
12598 & 101557 & 2002 & 42.00 & -0.22 & 5165.00 & 40300.19 & 0.81 & 0.96 & 0.78 \\
48182 & 240040 & 2002 & 265.50 & -0.12 & 33469.00 & 273042.23 & 0.79 & 1.03 & 0.82 \\
49976 & 240380 & 2002 & 8.10 & -0.23 & 765.00 & 7626.63 & 1.06 & 0.94 & 1.00 \\
12785 & 101595 & 2002 & 1153.50 & -0.31 & 102249.00 & 1022486.25 & 1.13 & 0.89 & 1.00 \\
11374 & 101399 & 2002 & 88.50 & -0.23 & 11557.00 & 84734.62 & 0.77 & 0.96 & 0.73 \\
44548 & 109338 & 2002 & 96.70 & -0.16 & 9311.00 & 93117.55 & 1.04 & 0.96 & 1.00 \\
32475 & 106033 & 2002 & 610.30 & -0.11 & 60071.00 & 620777.78 & 1.02 & 1.02 & 1.03 \\
53538 & 351589 & 2002 & 6.60 & -0.26 & 1030.00 & 6472.68 & 0.64 & 0.98 & 0.63 \\
633 & 100085 & 2002 & 9389.10 & -0.31 & 957105.00 & 9615957.83 & 0.98 & 1.02 & 1.00 \\
27911 & 105346 & 2002 & 1130.10 & -0.16 & 112912.00 & 988857.59 & 1.00 & 0.88 & 0.88 \\
55527 & 400116 & 2002 & 32.60 & -0.29 & 2538.00 & 26755.19 & 1.28 & 0.82 & 1.05 \\
52442 & 302941 & 2002 & 10.30 & -0.27 & 1175.00 & 10358.41 & 0.88 & 1.01 & 0.88 \\
63728 & 500549 & 2002 & 1.20 & -0.32 & 168.00 & 1018.16 & 0.71 & 0.85 & 0.61 \\
53514 & 351459 & 2002 & 669.80 & -0.22 & 74935.00 & 535991.16 & 0.89 & 0.80 & 0.72 \\
49487 & 240304 & 2002 & 193.20 & -0.18 & 19381.00 & 165879.83 & 1.00 & 0.86 & 0.86 \\
43393 & 109111 & 2002 & 33.40 & -0.30 & 3521.00 & 34338.35 & 0.95 & 1.03 & 0.98 \\
18697 & 102503 & 2002 & 303.20 & -0.28 & 41848.00 & 289800.17 & 0.72 & 0.96 & 0.69 \\
74712 & 601155 & 2002 & 40.80 & 0.07 & 4155.00 & 40942.26 & 0.98 & 1.00 & 0.99 \\
27902 & 105343 & 2002 & 81.40 & -0.21 & 7884.00 & 80868.76 & 1.03 & 0.99 & 1.03 \\
16879 & 102213 & 2002 & 702.60 & -0.35 & 70257.00 & 688973.31 & 1.00 & 0.98 & 0.98 \\
2542 & 100343 & 2002 & 403.00 & -0.20 & 36799.00 & 380186.63 & 1.10 & 0.94 & 1.03 \\
60784 & 410725 & 2002 & 4.50 & -0.15 & 423.00 & 4482.89 & 1.06 & 1.00 & 1.06 \\
26247 & 103547 & 2002 & 6033.30 & -0.36 & 873797.00 & 6032477.73 & 0.69 & 1.00 & 0.69 \\
7705 & 101055 & 2002 & 14798.00 & -0.33 & 1922475.00 & 14193055.32 & 0.77 & 0.96 & 0.74 \\
46094 & 200184 & 2002 & 10.20 & -0.43 & 1676.00 & 8842.01 & 0.61 & 0.87 & 0.53 \\
63732 & 500550 & 2002 & 39272.50 & -0.22 & 2486459.00 & 32891425.72 & 1.58 & 0.84 & 1.32 \\
32447 & 106028 & 2002 & 598.80 & -0.22 & 58930.00 & 556315.28 & 1.02 & 0.93 & 0.94 \\
12917 & 101606 & 2002 & 3168.90 & -0.24 & 296057.00 & 2777982.52 & 1.07 & 0.88 & 0.94 \\
65687 & 500719 & 2002 & 71.80 & -0.14 & 6158.00 & 58227.11 & 1.17 & 0.81 & 0.95 \\
15219 & 101968 & 2002 & 106.90 & -0.30 & 10688.00 & 103882.24 & 1.00 & 0.97 & 0.97 \\
44540 & 109336 & 2002 & 28.40 & -0.20 & 2817.00 & 28053.13 & 1.01 & 0.99 & 1.00 \\
36936 & 106642 & 2002 & 566.00 & -0.04 & 53703.00 & 536649.49 & 1.05 & 0.95 & 1.00 \\
17150 & 102259 & 2002 & 497.30 & -0.37 & 49806.00 & 487306.72 & 1.00 & 0.98 & 0.98 \\
28097 & 105382 & 2002 & 130.40 & -0.29 & 14165.00 & 119906.42 & 0.92 & 0.92 & 0.85 \\
11286 & 101390 & 2002 & 3162.70 & -0.19 & 315992.00 & 3084620.16 & 1.00 & 0.98 & 0.98 \\
41551 & 108765 & 2002 & 1781.80 & -0.28 & 184062.00 & 1840682.89 & 0.97 & 1.03 & 1.00 \\
63527 & 500514 & 2002 & 85.10 & -0.21 & 10776.00 & 79424.03 & 0.79 & 0.93 & 0.74 \\
36954 & 106643 & 2002 & 96.20 & -0.39 & 9533.00 & 94773.19 & 1.01 & 0.99 & 0.99 \\
981 & 100113 & 2002 & 462.20 & -0.33 & 46371.00 & 463693.18 & 1.00 & 1.00 & 1.00 \\
28078 & 105379 & 2002 & 495.10 & -0.20 & 45904.00 & 428305.32 & 1.08 & 0.87 & 0.93 \\
50041 & 240386 & 2002 & 6.40 & 0.01 & 627.00 & 5629.63 & 1.02 & 0.88 & 0.90 \\
2633 & 100348 & 2002 & 147.10 & -0.21 & 14905.00 & 144845.60 & 0.99 & 0.98 & 0.97 \\
27221 & 105256 & 2002 & 100.60 & -0.17 & 10083.00 & 95024.64 & 1.00 & 0.94 & 0.94 \\
49472 & 240302 & 2002 & 5.70 & -0.25 & 720.00 & 4827.86 & 0.79 & 0.85 & 0.67 \\
32691 & 106049 & 2002 & 373.40 & -0.23 & 37292.00 & 358291.68 & 1.00 & 0.96 & 0.96 \\
50038 & 240385 & 2002 & 1.10 & -0.12 & 106.00 & 1009.82 & 1.04 & 0.92 & 0.95 \\
36924 & 106640 & 2002 & 70.40 & -0.29 & 9252.00 & 64044.48 & 0.76 & 0.91 & 0.69 \\
36906 & 106627 & 2002 & 529.80 & -0.16 & 54358.00 & 486553.88 & 0.97 & 0.92 & 0.90 \\
43525 & 109133 & 2002 & 17.90 & -0.49 & 1758.00 & 16650.91 & 1.02 & 0.93 & 0.95 \\
26142 & 103544 & 2002 & 24077.80 & -0.21 & 2520653.00 & 21180240.24 & 0.96 & 0.88 & 0.84 \\
44517 & 109334 & 2002 & 40.80 & -0.16 & 3974.00 & 36176.44 & 1.03 & 0.89 & 0.91 \\
22639 & 103027 & 2002 & 4839.00 & -0.33 & 631150.00 & 4723544.28 & 0.77 & 0.98 & 0.75 \\
41064 & 108188 & 2002 & 391.10 & -0.30 & 41818.00 & 343957.21 & 0.94 & 0.88 & 0.82 \\
28140 & 105384 & 2002 & 39.30 & -0.21 & 3936.00 & 37357.24 & 1.00 & 0.95 & 0.95 \\
49740 & 240352 & 2002 & 1.40 & -0.18 & 126.00 & 1127.89 & 1.11 & 0.81 & 0.90 \\
40540 & 108137 & 2002 & 618.60 & -0.18 & 61845.00 & 619966.21 & 1.00 & 1.00 & 1.00 \\
16791 & 102192 & 2002 & 600.50 & -0.20 & 60682.00 & 596210.89 & 0.99 & 0.99 & 0.98 \\
28126 & 105383 & 2002 & 128.70 & -0.20 & 12479.00 & 114400.26 & 1.03 & 0.89 & 0.92 \\
38369 & 107246 & 2002 & 91.90 & -0.33 & 9039.00 & 77408.62 & 1.02 & 0.84 & 0.86 \\
41558 & 108766 & 2002 & 190.90 & -0.20 & 16428.00 & 175684.62 & 1.16 & 0.92 & 1.07 \\
38721 & 107310 & 2002 & 44.60 & -0.24 & 5207.00 & 39801.28 & 0.86 & 0.89 & 0.76 \\
36880 & 106620 & 2002 & 167.80 & -0.25 & 24243.00 & 155778.36 & 0.69 & 0.93 & 0.64 \\
36980 & 106644 & 2002 & 1.40 & -0.28 & 218.00 & 1429.33 & 0.64 & 1.02 & 0.66 \\
41486 & 108761 & 2002 & 255.90 & -0.16 & 25229.00 & 258952.21 & 1.01 & 1.01 & 1.03 \\
38665 & 107306 & 2002 & 227.30 & -0.25 & 22815.00 & 225833.34 & 1.00 & 0.99 & 0.99 \\
28017 & 105369 & 2002 & 171.90 & -0.22 & 17171.00 & 167025.72 & 1.00 & 0.97 & 0.97 \\
55416 & 400094 & 2002 & 248.00 & -0.21 & 19064.00 & 199926.76 & 1.30 & 0.81 & 1.05 \\
9220 & 101119 & 2002 & 110.60 & -0.06 & 11061.00 & 101859.00 & 1.00 & 0.92 & 0.92 \\
55537 & 400117 & 2002 & 13.90 & -0.25 & 1264.00 & 12419.22 & 1.10 & 0.89 & 0.98 \\
37062 & 106655 & 2002 & 48.50 & -0.43 & 4796.00 & 46532.68 & 1.01 & 0.96 & 0.97 \\
64351 & 500598 & 2002 & 1341.00 & -0.15 & 123037.00 & 1065681.48 & 1.09 & 0.79 & 0.87 \\
32563 & 106041 & 2002 & 265.30 & -0.28 & 22619.00 & 238136.31 & 1.17 & 0.90 & 1.05 \\
11318 & 101393 & 2002 & 568.50 & -0.20 & 52911.00 & 536610.11 & 1.07 & 0.94 & 1.01 \\
47747 & 221051 & 2002 & 4239.60 & -0.16 & 559315.00 & 3884246.87 & 0.76 & 0.92 & 0.69 \\
2614 & 100347 & 2002 & 781.60 & -0.22 & 81377.00 & 748189.71 & 0.96 & 0.96 & 0.92 \\
37075 & 106666 & 2002 & 8.30 & -0.49 & 785.00 & 8313.39 & 1.06 & 1.00 & 1.06 \\
37079 & 106667 & 2002 & 345.70 & 0.04 & 32675.00 & 286022.42 & 1.06 & 0.83 & 0.88 \\
490 & 100071 & 2002 & 3389.80 & -0.16 & 338565.00 & 3271584.27 & 1.00 & 0.97 & 0.97 \\
23692 & 103208 & 2002 & 1456.20 & -0.20 & 146184.00 & 1254709.39 & 1.00 & 0.86 & 0.86 \\
50002 & 240382 & 2002 & 17.30 & -0.08 & 1266.00 & 12143.50 & 1.37 & 0.70 & 0.96 \\
37720 & 107135 & 2002 & 109.40 & -0.15 & 10943.00 & 106844.01 & 1.00 & 0.98 & 0.98 \\
23252 & 103152 & 2002 & 2106.30 & -0.16 & 192611.00 & 2063638.90 & 1.09 & 0.98 & 1.07 \\
22683 & 103028 & 2002 & 4040.70 & -0.17 & 493174.00 & 3496248.75 & 0.82 & 0.87 & 0.71 \\
5539 & 100771 & 2002 & 413.80 & 0.04 & 39383.00 & 379819.58 & 1.05 & 0.92 & 0.96 \\
37017 & 106650 & 2002 & 28.80 & -0.06 & 2854.00 & 27957.37 & 1.01 & 0.97 & 0.98 \\
53679 & 355987 & 2002 & 124.40 & -0.28 & 11045.00 & 110427.01 & 1.13 & 0.89 & 1.00 \\
38688 & 107308 & 2002 & 1370.50 & -0.15 & 138649.00 & 1341580.30 & 0.99 & 0.98 & 0.97 \\
54404 & 367600 & 2002 & 1.10 & -0.02 & 90.00 & 937.54 & 1.22 & 0.85 & 1.04 \\
28046 & 105370 & 2002 & 62.90 & -0.27 & 6225.00 & 62436.78 & 1.01 & 0.99 & 1.00 \\
41533 & 108764 & 2002 & 264.90 & -0.23 & 26328.00 & 252669.24 & 1.01 & 0.95 & 0.96 \\
17508 & 102317 & 2002 & 15.90 & -0.16 & 1389.00 & 13888.85 & 1.14 & 0.87 & 1.00 \\
16820 & 102193 & 2002 & 145.30 & -0.27 & 14576.00 & 144990.81 & 1.00 & 1.00 & 0.99 \\
4530 & 100637 & 2002 & 842.10 & -0.25 & 56629.00 & 506820.50 & 1.49 & 0.60 & 0.89 \\
37036 & 106654 & 2002 & 357.80 & -0.22 & 41023.00 & 338855.49 & 0.87 & 0.95 & 0.83 \\
41508 & 108762 & 2002 & 109.20 & -0.05 & 9880.00 & 106639.25 & 1.11 & 0.98 & 1.08 \\
48414 & 240080 & 2002 & 44.20 & -0.20 & 3867.00 & 39190.21 & 1.14 & 0.89 & 1.01 \\
16275 & 102113 & 2002 & 89.70 & -0.32 & 9004.00 & 91236.76 & 1.00 & 1.02 & 1.01 \\
39172 & 107616 & 2002 & 85.20 & -0.11 & 9084.00 & 85423.37 & 0.94 & 1.00 & 0.94 \\
21753 & 102949 & 2002 & 1596.90 & -0.17 & 213562.00 & 1614890.20 & 0.75 & 1.01 & 0.76 \\
61771 & 500125 & 2002 & 68.70 & -0.15 & 6849.00 & 65669.47 & 1.00 & 0.96 & 0.96 \\
8279 & 101081 & 2002 & 425.00 & -0.11 & 55411.00 & 451609.33 & 0.77 & 1.06 & 0.82 \\
15726 & 102016 & 2002 & 9471.30 & -0.29 & 947124.00 & 8979449.27 & 1.00 & 0.95 & 0.95 \\
34332 & 106223 & 2002 & 67.60 & -0.18 & 6400.00 & 63224.73 & 1.06 & 0.94 & 0.99 \\
64957 & 500651 & 2002 & 5.80 & -0.36 & 585.00 & 5618.06 & 0.99 & 0.97 & 0.96 \\
43932 & 109232 & 2002 & 241.90 & -0.51 & 24115.00 & 240589.31 & 1.00 & 0.99 & 1.00 \\
38008 & 107185 & 2002 & 10.00 & -0.30 & 1096.00 & 10711.14 & 0.91 & 1.07 & 0.98 \\
43935 & 109233 & 2002 & 125.10 & -0.84 & 15072.00 & 112928.04 & 0.83 & 0.90 & 0.75 \\
20809 & 102795 & 2002 & 413.40 & -0.11 & 50014.00 & 386178.06 & 0.83 & 0.93 & 0.77 \\
31191 & 105866 & 2002 & 5938.10 & -0.22 & 555808.00 & 5960167.68 & 1.07 & 1.00 & 1.07 \\
52652 & 305590 & 2002 & 164.60 & -0.18 & 15479.00 & 156372.79 & 1.06 & 0.95 & 1.01 \\
42867 & 109028 & 2002 & 227.70 & -0.15 & 22598.00 & 216307.50 & 1.01 & 0.95 & 0.96 \\
25114 & 103432 & 2002 & 1355.00 & -0.21 & 136834.00 & 1254112.77 & 0.99 & 0.93 & 0.92 \\
61770 & 500123 & 2002 & 4.60 & -0.16 & 465.00 & 4434.84 & 0.99 & 0.96 & 0.95 \\
52642 & 305586 & 2002 & 57.40 & -0.42 & 5203.00 & 56683.60 & 1.10 & 0.99 & 1.09 \\
39896 & 107928 & 2002 & 2119.50 & -0.23 & 211986.00 & 1980780.90 & 1.00 & 0.93 & 0.93 \\
47541 & 212658 & 2002 & 5763.80 & -0.21 & 579904.00 & 5496866.58 & 0.99 & 0.95 & 0.95 \\
34322 & 106222 & 2002 & 88.90 & -0.30 & 8935.00 & 87975.05 & 0.99 & 0.99 & 0.98 \\
39564 & 107722 & 2002 & 117.90 & -0.21 & 12442.00 & 94404.22 & 0.95 & 0.80 & 0.76 \\
42901 & 109031 & 2002 & 15.00 & -0.25 & 1484.00 & 14505.72 & 1.01 & 0.97 & 0.98 \\
6151 & 100825 & 2002 & 63.00 & -0.18 & 6295.00 & 60454.42 & 1.00 & 0.96 & 0.96 \\
32046 & 105977 & 2002 & 858.70 & -0.22 & 86330.00 & 815631.21 & 0.99 & 0.95 & 0.94 \\
34352 & 106224 & 2002 & 87.90 & -0.21 & 7983.00 & 77652.49 & 1.10 & 0.88 & 0.97 \\
31224 & 105868 & 2002 & 109.40 & -0.28 & 10929.00 & 107177.19 & 1.00 & 0.98 & 0.98 \\
47568 & 212809 & 2002 & 5.30 & -0.07 & 441.00 & 4385.72 & 1.20 & 0.83 & 0.99 \\
31219 & 105867 & 2002 & 2.90 & -0.33 & 287.00 & 2773.19 & 1.01 & 0.96 & 0.97 \\
64121 & 500588 & 2002 & 351.60 & -0.20 & 27219.00 & 283312.98 & 1.29 & 0.81 & 1.04 \\
10049 & 101256 & 2002 & 4.40 & -0.36 & 386.00 & 3861.93 & 1.14 & 0.88 & 1.00 \\
48595 & 240112 & 2002 & 12.00 & -0.44 & 1203.00 & 10861.32 & 1.00 & 0.91 & 0.90 \\
3726 & 100475 & 2002 & 506.70 & -0.15 & 50794.00 & 493327.56 & 1.00 & 0.97 & 0.97 \\
19403 & 102600 & 2002 & 638.30 & -0.17 & 63825.00 & 608230.82 & 1.00 & 0.95 & 0.95 \\
24595 & 103370 & 2002 & 146.90 & -0.33 & 14629.00 & 143266.71 & 1.00 & 0.98 & 0.98 \\
46850 & 200309 & 2002 & 15.10 & -0.12 & 1556.00 & 14267.58 & 0.97 & 0.94 & 0.92 \\
48604 & 240113 & 2002 & 42.90 & -0.27 & 4306.00 & 41899.56 & 1.00 & 0.98 & 0.97 \\
33 & 100003 & 2002 & 712.60 & -0.16 & 71270.00 & 692732.26 & 1.00 & 0.97 & 0.97 \\
34397 & 106231 & 2002 & 184.50 & -0.18 & 18921.00 & 189237.17 & 0.98 & 1.03 & 1.00 \\
12361 & 101537 & 2002 & 431.80 & -0.19 & 43208.00 & 403536.32 & 1.00 & 0.93 & 0.93 \\
48294 & 240060 & 2002 & 138.60 & -0.30 & 14190.00 & 117437.63 & 0.98 & 0.85 & 0.83 \\
64984 & 500653 & 2002 & 31.60 & -0.28 & 2582.00 & 24411.70 & 1.22 & 0.77 & 0.95 \\
38212 & 107222 & 2002 & 167.30 & -0.08 & 16655.00 & 158352.40 & 1.00 & 0.95 & 0.95 \\
47189 & 200342 & 2002 & 5092.20 & -0.09 & 663423.00 & 4926660.58 & 0.77 & 0.97 & 0.74 \\
9001 & 101108 & 2002 & 841.30 & -0.21 & 114412.00 & 842477.81 & 0.74 & 1.00 & 0.74 \\
43938 & 109236 & 2002 & 1.90 & -0.52 & 226.00 & 2310.09 & 0.84 & 1.22 & 1.02 \\
39922 & 107938 & 2002 & 134.70 & -0.27 & 12764.00 & 113428.52 & 1.06 & 0.84 & 0.89 \\
20859 & 102797 & 2002 & 44.30 & -0.15 & 4296.00 & 43993.57 & 1.03 & 0.99 & 1.02 \\
31112 & 105860 & 2002 & 1928.70 & -0.30 & 196770.00 & 1867710.79 & 0.98 & 0.97 & 0.95 \\
23604 & 103202 & 2002 & 45.10 & -0.23 & 4662.00 & 40211.73 & 0.97 & 0.89 & 0.86 \\
14833 & 101916 & 2002 & 363.50 & -0.13 & 36527.00 & 353740.60 & 1.00 & 0.97 & 0.97 \\
48611 & 240114 & 2002 & 218.60 & -0.24 & 22105.00 & 216659.90 & 0.99 & 0.99 & 0.98 \\
43909 & 109230 & 2002 & 26.80 & 0.00 & 2552.00 & 25520.83 & 1.05 & 0.95 & 1.00 \\
3464 & 100439 & 2002 & 18.90 & -0.11 & 1998.00 & 20248.47 & 0.95 & 1.07 & 1.01 \\
52675 & 305766 & 2002 & 63.70 & -0.17 & 6362.00 & 63619.99 & 1.00 & 1.00 & 1.00 \\
5199 & 100731 & 2002 & 7163.40 & -0.19 & 703753.00 & 7482560.30 & 1.02 & 1.04 & 1.06 \\
4051 & 100543 & 2002 & 259.80 & -0.21 & 29525.00 & 268463.97 & 0.88 & 1.03 & 0.91 \\
34370 & 106230 & 2002 & 178.90 & -0.31 & 18268.00 & 182671.13 & 0.98 & 1.02 & 1.00 \\
39550 & 107720 & 2002 & 92.20 & -0.02 & 9233.00 & 90308.63 & 1.00 & 0.98 & 0.98 \\
31163 & 105865 & 2002 & 88.30 & -0.27 & 9083.00 & 89875.24 & 0.97 & 1.02 & 0.99 \\
46822 & 200303 & 2002 & 6.50 & -0.12 & 641.00 & 6144.04 & 1.01 & 0.95 & 0.96 \\
45256 & 109444 & 2002 & 2.10 & -0.03 & 193.00 & 1934.25 & 1.09 & 0.92 & 1.00 \\
31156 & 105864 & 2002 & 15.40 & -0.24 & 1548.00 & 15217.65 & 0.99 & 0.99 & 0.98 \\
10079 & 101258 & 2002 & 3235.20 & -0.27 & 445248.00 & 2628698.83 & 0.73 & 0.81 & 0.59 \\
44993 & 109410 & 2002 & 68.20 & -0.15 & 5666.00 & 59479.88 & 1.20 & 0.87 & 1.05 \\
51920 & 300653 & 2002 & 48.30 & -0.16 & 3736.00 & 40472.53 & 1.29 & 0.84 & 1.08 \\
26862 & 103614 & 2002 & 232.60 & -0.18 & 23122.00 & 228721.66 & 1.01 & 0.98 & 0.99 \\
39525 & 107719 & 2002 & 202.60 & -0.28 & 21628.00 & 166318.69 & 0.94 & 0.82 & 0.77 \\
31140 & 105861 & 2002 & 220.00 & -0.10 & 24971.00 & 234233.58 & 0.88 & 1.06 & 0.94 \\
46828 & 200304 & 2002 & 9.30 & -0.22 & 929.00 & 8725.23 & 1.00 & 0.94 & 0.94 \\
32062 & 105978 & 2002 & 94.00 & -0.36 & 7128.00 & 83829.21 & 1.32 & 0.89 & 1.18 \\
38014 & 107187 & 2002 & 30.30 & -0.30 & 3037.00 & 29635.08 & 1.00 & 0.98 & 0.98 \\
43883 & 109228 & 2002 & 15.10 & -0.28 & 1418.00 & 15089.25 & 1.06 & 1.00 & 1.06 \\
34262 & 106216 & 2002 & 367.30 & -0.34 & 36078.00 & 353433.32 & 1.02 & 0.96 & 0.98 \\
20712 & 102784 & 2002 & 16690.60 & -0.27 & 1690795.00 & 16287366.11 & 0.99 & 0.98 & 0.96 \\
31319 & 105878 & 2002 & 1319.20 & -0.19 & 133225.00 & 1241356.26 & 0.99 & 0.94 & 0.93 \\
54243 & 364993 & 2002 & 25.80 & -0.16 & 3007.00 & 23915.51 & 0.86 & 0.93 & 0.80 \\
39580 & 107726 & 2002 & 676.30 & -0.20 & 67278.00 & 668726.74 & 1.01 & 0.99 & 0.99 \\
96692 & 611006 & 2002 & 89.80 & -0.31 & 8982.00 & 82081.72 & 1.00 & 0.91 & 0.91 \\
19437 & 102601 & 2002 & 5381.00 & -0.21 & 538107.00 & 5269503.45 & 1.00 & 0.98 & 0.98 \\
42951 & 109039 & 2002 & 8.20 & -0.23 & 746.00 & 7001.74 & 1.10 & 0.85 & 0.94 \\
52521 & 302997 & 2002 & 175.20 & -0.20 & 17574.00 & 146147.88 & 1.00 & 0.83 & 0.83 \\
13750 & 101762 & 2002 & 3516.40 & -0.22 & 350999.00 & 3165838.83 & 1.00 & 0.90 & 0.90 \\
31311 & 105877 & 2002 & 6.00 & -0.26 & 587.00 & 6080.66 & 1.02 & 1.01 & 1.04 \\
18457 & 102461 & 2002 & 1244.70 & -0.30 & 124648.00 & 1212444.97 & 1.00 & 0.97 & 0.97 \\
10007 & 101252 & 2002 & 100.60 & -0.14 & 9593.00 & 95932.46 & 1.05 & 0.95 & 1.00 \\
25038 & 103426 & 2002 & 840.80 & -0.28 & 84327.00 & 746853.16 & 1.00 & 0.89 & 0.89 \\
64717 & 500621 & 2002 & 165.60 & 0.03 & 10690.00 & 139718.50 & 1.55 & 0.84 & 1.31 \\
34235 & 106214 & 2002 & 172.10 & -0.17 & 17214.00 & 170921.64 & 1.00 & 0.99 & 0.99 \\
60804 & 410730 & 2002 & 295.10 & -0.06 & 41877.00 & 298831.64 & 0.70 & 1.01 & 0.71 \\
42957 & 109042 & 2002 & 78.90 & -0.07 & 4884.00 & 48364.33 & 1.62 & 0.61 & 0.99 \\
49214 & 240250 & 2002 & 307.40 & -0.19 & 38019.00 & 306823.49 & 0.81 & 1.00 & 0.81 \\
31365 & 105880 & 2002 & 722.20 & -0.33 & 72418.00 & 724173.88 & 1.00 & 1.00 & 1.00 \\
52619 & 303175 & 2002 & 577.80 & -0.23 & 53754.00 & 555405.38 & 1.07 & 0.96 & 1.03 \\
18083 & 102396 & 2002 & 2790.90 & -0.15 & 269515.00 & 2551634.21 & 1.04 & 0.91 & 0.95 \\
1598 & 100217 & 2002 & 22.70 & -0.37 & 2205.00 & 20999.60 & 1.03 & 0.93 & 0.95 \\
46881 & 200311 & 2002 & 62.70 & -0.15 & 6296.00 & 58179.07 & 1.00 & 0.93 & 0.92 \\
51954 & 300673 & 2002 & 98.40 & -0.13 & 9859.00 & 96966.53 & 1.00 & 0.99 & 0.98 \\
34223 & 106213 & 2002 & 73.30 & -0.27 & 6105.00 & 61806.47 & 1.20 & 0.84 & 1.01 \\
18431 & 102452 & 2002 & 75.50 & -0.22 & 7533.00 & 72773.13 & 1.00 & 0.96 & 0.97 \\
53937 & 361995 & 2002 & 6.80 & -0.25 & 827.00 & 8274.55 & 0.82 & 1.22 & 1.00 \\
31347 & 105879 & 2002 & 1003.80 & -0.20 & 100503.00 & 985665.64 & 1.00 & 0.98 & 0.98 \\
45212 & 109438 & 2002 & 60.00 & -0.03 & 6054.00 & 57091.79 & 0.99 & 0.95 & 0.94 \\
61750 & 500119 & 2002 & 6.30 & -0.26 & 448.00 & 3654.41 & 1.41 & 0.58 & 0.82 \\
6703 & 100913 & 2002 & 431.90 & -0.12 & 43167.00 & 416184.83 & 1.00 & 0.96 & 0.96 \\
56415 & 400206 & 2002 & 4.40 & -0.13 & 630.00 & 3922.58 & 0.70 & 0.89 & 0.62 \\
61711 & 500116 & 2002 & 387.20 & -0.23 & 47061.00 & 407427.82 & 0.82 & 1.05 & 0.87 \\
7432 & 101039 & 2002 & 4294.60 & -0.16 & 512288.00 & 3526655.32 & 0.84 & 0.82 & 0.69 \\
31304 & 105876 & 2002 & 39.20 & -0.42 & 3919.00 & 38840.74 & 1.00 & 0.99 & 0.99 \\
61374 & 500047 & 2002 & 1.70 & -0.21 & 175.00 & 1424.45 & 0.97 & 0.84 & 0.81 \\
55339 & 400087 & 2002 & 36.60 & -0.04 & 3678.00 & 35593.71 & 1.00 & 0.97 & 0.97 \\
54864 & 400019 & 2002 & 646.60 & -0.27 & 64532.00 & 640615.32 & 1.00 & 0.99 & 0.99 \\
24618 & 103372 & 2002 & 717.00 & -0.15 & 72841.00 & 674284.02 & 0.98 & 0.94 & 0.93 \\
31257 & 105871 & 2002 & 7.60 & -0.23 & 899.00 & 6994.91 & 0.85 & 0.92 & 0.78 \\
61754 & 500120 & 2002 & 7.80 & -0.23 & 1438.00 & 7576.04 & 0.54 & 0.97 & 0.53 \\
15701 & 102015 & 2002 & 213.40 & -0.27 & 21344.00 & 211249.10 & 1.00 & 0.99 & 0.99 \\
49293 & 240266 & 2002 & 149.50 & -0.37 & 18657.00 & 180775.37 & 0.80 & 1.21 & 0.97 \\
52629 & 305184 & 2002 & 14.30 & -0.28 & 1457.00 & 13906.28 & 0.98 & 0.97 & 0.95 \\
34289 & 106220 & 2002 & 213.20 & -0.39 & 20955.00 & 196470.06 & 1.02 & 0.92 & 0.94 \\
45003 & 109413 & 2002 & 16.10 & -0.12 & 1520.00 & 14426.28 & 1.06 & 0.90 & 0.95 \\
6164 & 100827 & 2002 & 155.40 & -0.41 & 15623.00 & 156180.38 & 0.99 & 1.01 & 1.00 \\
32037 & 105976 & 2002 & 8.40 & -0.25 & 835.00 & 8359.17 & 1.01 & 1.00 & 1.00 \\
34295 & 106221 & 2002 & 103.40 & -0.19 & 9303.00 & 95355.85 & 1.11 & 0.92 & 1.03 \\
43906 & 109229 & 2002 & 45.00 & -0.20 & 3929.00 & 35017.79 & 1.15 & 0.78 & 0.89 \\
20770 & 102789 & 2002 & 819.90 & -0.19 & 106399.00 & 777567.67 & 0.77 & 0.95 & 0.73 \\
17294 & 102278 & 2002 & 129.90 & -0.20 & 12989.00 & 126360.37 & 1.00 & 0.97 & 0.97 \\
3498 & 100441 & 2002 & 295.40 & -0.27 & 29804.00 & 293520.41 & 0.99 & 0.99 & 0.98 \\
49634 & 240327 & 2002 & 78.40 & -0.35 & 11420.00 & 76783.92 & 0.69 & 0.98 & 0.67 \\
42909 & 109033 & 2002 & 20.30 & -0.44 & 2031.00 & 20143.42 & 1.00 & 0.99 & 0.99 \\
15681 & 102013 & 2002 & 2694.00 & -0.31 & 269378.00 & 2657409.11 & 1.00 & 0.99 & 0.99 \\
20731 & 102788 & 2002 & 219.10 & -0.15 & 24808.00 & 198482.65 & 0.88 & 0.91 & 0.80 \\
53933 & 361852 & 2002 & 3.70 & -0.29 & 367.00 & 3582.45 & 1.01 & 0.97 & 0.98 \\
31295 & 105875 & 2002 & 32.70 & -0.32 & 3166.00 & 31095.11 & 1.03 & 0.95 & 0.98 \\
42943 & 109038 & 2002 & 12.60 & -0.22 & 1267.00 & 12667.07 & 0.99 & 1.01 & 1.00 \\
43765 & 109218 & 2002 & 16.00 & -0.12 & 1605.00 & 15254.32 & 1.00 & 0.95 & 0.95 \\
46872 & 200310 & 2002 & 16.10 & -0.11 & 1636.00 & 15067.65 & 0.98 & 0.94 & 0.92 \\
64941 & 500647 & 2002 & 28.80 & -0.28 & 2829.00 & 27176.77 & 1.02 & 0.94 & 0.96 \\
2169 & 100293 & 2002 & 132.30 & -0.34 & 13238.00 & 131398.90 & 1.00 & 0.99 & 0.99 \\
42933 & 109037 & 2002 & 66.70 & -0.28 & 6676.00 & 65643.32 & 1.00 & 0.98 & 0.98 \\
4113 & 100552 & 2002 & 61.60 & -0.30 & 6160.00 & 61419.22 & 1.00 & 1.00 & 1.00 \\
48986 & 240197 & 2002 & 197.30 & -0.31 & 19551.00 & 195504.91 & 1.01 & 0.99 & 1.00 \\
1569 & 100214 & 2002 & 151.30 & -0.19 & 15165.00 & 128976.88 & 1.00 & 0.85 & 0.85 \\
31281 & 105874 & 2002 & 129.50 & -0.16 & 13168.00 & 110417.27 & 0.98 & 0.85 & 0.84 \\
26846 & 103609 & 2002 & 26.00 & -0.16 & 2743.00 & 24929.19 & 0.95 & 0.96 & 0.91 \\
31270 & 105873 & 2002 & 6.10 & -0.26 & 818.00 & 6389.41 & 0.75 & 1.05 & 0.78 \\
61332 & 500037 & 2002 & 1943.10 & -0.29 & 194708.00 & 1946784.74 & 1.00 & 1.00 & 1.00 \\
34430 & 106239 & 2002 & 41.30 & -0.37 & 4123.00 & 39565.21 & 1.00 & 0.96 & 0.96 \\
17886 & 102371 & 2002 & 220.00 & -0.24 & 31246.00 & 213069.53 & 0.70 & 0.97 & 0.68 \\
43986 & 109254 & 2002 & 34.90 & -0.46 & 3177.00 & 32294.57 & 1.10 & 0.93 & 1.02 \\
32074 & 105980 & 2002 & 303.30 & -0.22 & 30367.00 & 285383.37 & 1.00 & 0.94 & 0.94 \\
46716 & 200292 & 2002 & 20.90 & -0.19 & 1914.00 & 18514.17 & 1.09 & 0.89 & 0.97 \\
34544 & 106251 & 2002 & 62.10 & -0.29 & 6456.00 & 60699.14 & 0.96 & 0.98 & 0.94 \\
3398 & 100431 & 2002 & 215.50 & -0.22 & 29619.00 & 212896.91 & 0.73 & 0.99 & 0.72 \\
43988 & 109255 & 2002 & 355.60 & -0.20 & 33602.00 & 352455.04 & 1.06 & 0.99 & 1.05 \\
45668 & 200088 & 2002 & 8.30 & -0.01 & 769.00 & 7257.81 & 1.08 & 0.87 & 0.94 \\
48646 & 240116 & 2002 & 96.90 & -0.26 & 9856.00 & 78648.65 & 0.98 & 0.81 & 0.80 \\
19767 & 102651 & 2002 & 2708.40 & -0.22 & 271337.00 & 2647829.06 & 1.00 & 0.98 & 0.98 \\
32802 & 106064 & 2002 & 77.90 & -0.33 & 7792.00 & 76657.83 & 1.00 & 0.98 & 0.98 \\
30885 & 105806 & 2002 & 85.00 & -0.24 & 9713.00 & 80347.34 & 0.88 & 0.95 & 0.83 \\
20996 & 102818 & 2002 & 20.00 & -0.22 & 2305.00 & 22417.15 & 0.87 & 1.12 & 0.97 \\
4564 & 100639 & 2002 & 696.60 & -0.38 & 69773.00 & 655399.77 & 1.00 & 0.94 & 0.94 \\
48474 & 240087 & 2002 & 97.90 & -0.17 & 9665.00 & 86596.59 & 1.01 & 0.88 & 0.90 \\
56526 & 400222 & 2002 & 14.30 & -0.04 & 1218.00 & 12509.04 & 1.17 & 0.87 & 1.03 \\
46719 & 200293 & 2002 & 14.40 & -0.12 & 1481.00 & 14811.52 & 0.97 & 1.03 & 1.00 \\
20980 & 102814 & 2002 & 142.80 & -0.12 & 13971.00 & 134259.29 & 1.02 & 0.94 & 0.96 \\
30902 & 105807 & 2002 & 15.20 & -0.23 & 1442.00 & 13409.11 & 1.05 & 0.88 & 0.93 \\
42786 & 109017 & 2002 & 46.10 & -0.10 & 4491.00 & 46522.91 & 1.03 & 1.01 & 1.04 \\
46755 & 200295 & 2002 & 161.70 & -0.14 & 16215.00 & 155872.18 & 1.00 & 0.96 & 0.96 \\
1482 & 100207 & 2002 & 1531.50 & -0.26 & 153228.00 & 1382508.09 & 1.00 & 0.90 & 0.90 \\
1889 & 100247 & 2002 & 505.10 & -0.29 & 50792.00 & 476723.93 & 0.99 & 0.94 & 0.94 \\
43979 & 109250 & 2002 & 146.90 & -0.28 & 23053.00 & 130475.23 & 0.64 & 0.89 & 0.57 \\
45278 & 200011 & 2002 & 99.10 & -0.22 & 12698.00 & 99282.31 & 0.78 & 1.00 & 0.78 \\
30918 & 105836 & 2002 & 169.90 & -0.38 & 28310.00 & 177321.91 & 0.60 & 1.04 & 0.63 \\
42749 & 109015 & 2002 & 133.30 & -0.23 & 17114.00 & 130686.92 & 0.78 & 0.98 & 0.76 \\
25201 & 103460 & 2002 & 694.00 & -0.31 & 97810.00 & 624936.62 & 0.71 & 0.90 & 0.64 \\
46742 & 200294 & 2002 & 45.30 & -0.23 & 4526.00 & 44608.88 & 1.00 & 0.98 & 0.99 \\
121 & 100009 & 2002 & 159.80 & -0.27 & 15626.00 & 151855.18 & 1.02 & 0.95 & 0.97 \\
10199 & 101268 & 2002 & 468.90 & -0.29 & 91263.00 & 458097.01 & 0.51 & 0.98 & 0.50 \\
37976 & 107178 & 2002 & 15.60 & -0.18 & 1299.00 & 13285.97 & 1.20 & 0.85 & 1.02 \\
24537 & 103339 & 2002 & 788.20 & -0.32 & 78992.00 & 786173.48 & 1.00 & 1.00 & 1.00 \\
54234 & 364950 & 2002 & 4.50 & -0.08 & 447.00 & 4249.36 & 1.01 & 0.94 & 0.95 \\
39478 & 107702 & 2002 & 549.90 & -0.20 & 35780.00 & 504118.47 & 1.54 & 0.92 & 1.41 \\
13384 & 101736 & 2002 & 41.20 & -0.21 & 4126.00 & 38799.96 & 1.00 & 0.94 & 0.94 \\
64656 & 500617 & 2002 & 5050.20 & -0.15 & 396736.00 & 3939052.93 & 1.27 & 0.78 & 0.99 \\
30829 & 105804 & 2002 & 461.30 & -0.24 & 55985.00 & 414501.51 & 0.82 & 0.90 & 0.74 \\
42744 & 109011 & 2002 & 38.60 & -0.10 & 3844.00 & 37887.02 & 1.00 & 0.98 & 0.99 \\
10232 & 101275 & 2002 & 833.40 & -0.33 & 112601.00 & 767961.86 & 0.74 & 0.92 & 0.68 \\
52370 & 302819 & 2002 & 3.30 & -0.30 & 329.00 & 3289.10 & 1.00 & 1.00 & 1.00 \\
34622 & 106261 & 2002 & 302.30 & -0.22 & 29397.00 & 292160.50 & 1.03 & 0.97 & 0.99 \\
42739 & 109010 & 2002 & 10.20 & -0.33 & 985.00 & 9798.54 & 1.04 & 0.96 & 0.99 \\
14134 & 101805 & 2002 & 511.40 & -0.26 & 68181.00 & 622783.31 & 0.75 & 1.22 & 0.91 \\
17873 & 102367 & 2002 & 145.30 & -0.21 & 13456.00 & 132050.21 & 1.08 & 0.91 & 0.98 \\
34633 & 106262 & 2002 & 304.70 & -0.23 & 28524.00 & 293223.17 & 1.07 & 0.96 & 1.03 \\
48071 & 235413 & 2002 & 70.90 & -0.22 & 6374.00 & 61522.89 & 1.11 & 0.87 & 0.97 \\
42716 & 109009 & 2002 & 101.60 & -0.11 & 10855.00 & 97517.15 & 0.94 & 0.96 & 0.90 \\
30801 & 105803 & 2002 & 3167.80 & -0.19 & 356841.00 & 2731413.61 & 0.89 & 0.86 & 0.77 \\
39933 & 107958 & 2002 & 19.60 & -0.01 & 1898.00 & 19315.86 & 1.03 & 0.99 & 1.02 \\
26652 & 103595 & 2002 & 135.80 & -0.18 & 12563.00 & 121423.28 & 1.08 & 0.89 & 0.97 \\
9463 & 101137 & 2002 & 17.50 & -0.30 & 2236.00 & 17929.75 & 0.78 & 1.02 & 0.80 \\
30786 & 105798 & 2002 & 1025.50 & -0.34 & 106966.00 & 1051694.04 & 0.96 & 1.03 & 0.98 \\
33587 & 106151 & 2002 & 1006.70 & -0.16 & 105449.00 & 986821.59 & 0.95 & 0.98 & 0.94 \\
8970 & 101107 & 2002 & 729.60 & -0.21 & 86144.00 & 716845.85 & 0.85 & 0.98 & 0.83 \\
21026 & 102824 & 2002 & 45.90 & -0.26 & 4578.00 & 40712.67 & 1.00 & 0.89 & 0.89 \\
9683 & 101165 & 2002 & 1667.00 & -0.23 & 166080.00 & 1633570.04 & 1.00 & 0.98 & 0.98 \\
39472 & 107699 & 2002 & 6.40 & -0.40 & 646.00 & 5528.97 & 0.99 & 0.86 & 0.86 \\
21007 & 102821 & 2002 & 101.00 & -0.30 & 10108.00 & 93368.99 & 1.00 & 0.92 & 0.92 \\
2223 & 100296 & 2002 & 7.00 & -0.19 & 619.00 & 6624.15 & 1.13 & 0.95 & 1.07 \\
19335 & 102597 & 2002 & 143.50 & -0.50 & 12936.00 & 146859.55 & 1.11 & 1.02 & 1.14 \\
33601 & 106152 & 2002 & 313.60 & -0.22 & 31437.00 & 307884.61 & 1.00 & 0.98 & 0.98 \\
48268 & 240058 & 2002 & 144.70 & -0.06 & 14643.00 & 127925.75 & 0.99 & 0.88 & 0.87 \\
4146 & 100561 & 2002 & 60.80 & -0.12 & 7342.00 & 51258.78 & 0.83 & 0.84 & 0.70 \\
12161 & 101513 & 2002 & 142.30 & -0.21 & 13861.00 & 138609.89 & 1.03 & 0.97 & 1.00 \\
54287 & 366837 & 2002 & 8.00 & -0.15 & 1098.00 & 6908.69 & 0.73 & 0.86 & 0.63 \\
34591 & 106257 & 2002 & 187.50 & -0.25 & 17277.00 & 186242.02 & 1.09 & 0.99 & 1.08 \\
1451 & 100200 & 2002 & 336.40 & -0.02 & 26373.00 & 293266.25 & 1.28 & 0.87 & 1.11 \\
49659 & 240330 & 2002 & 6.10 & -0.02 & 549.00 & 5857.63 & 1.11 & 0.96 & 1.07 \\
54826 & 400017 & 2002 & 112.50 & -0.32 & 11277.00 & 109541.58 & 1.00 & 0.97 & 0.97 \\
34584 & 106256 & 2002 & 68.00 & -0.18 & 6812.00 & 66548.37 & 1.00 & 0.98 & 0.98 \\
55271 & 400076 & 2002 & 373.70 & -0.16 & 37485.00 & 369784.18 & 1.00 & 0.99 & 0.99 \\
46804 & 200299 & 2002 & 8.20 & -0.20 & 816.00 & 7715.47 & 1.00 & 0.94 & 0.95 \\
34499 & 106248 & 2002 & 395.00 & -0.11 & 38431.00 & 330382.62 & 1.03 & 0.84 & 0.86 \\
42815 & 109020 & 2002 & 271.20 & -0.33 & 39522.00 & 268645.44 & 0.69 & 0.99 & 0.68 \\
44700 & 109366 & 2002 & 19.60 & -0.33 & 1953.00 & 19384.55 & 1.00 & 0.99 & 0.99 \\
1860 & 100245 & 2002 & 329.70 & -0.25 & 32935.00 & 327099.81 & 1.00 & 0.99 & 0.99 \\
31042 & 105852 & 2002 & 37.10 & -0.26 & 3700.00 & 35748.86 & 1.00 & 0.96 & 0.97 \\
33048 & 106088 & 2002 & 99.20 & -0.19 & 13185.00 & 105877.85 & 0.75 & 1.07 & 0.80 \\
17906 & 102372 & 2002 & 3785.00 & -0.14 & 302822.00 & 3444315.36 & 1.25 & 0.91 & 1.14 \\
52681 & 306482 & 2002 & 6.20 & -0.10 & 661.00 & 6734.73 & 0.94 & 1.09 & 1.02 \\
42799 & 109019 & 2002 & 43.00 & -0.24 & 6691.00 & 40048.86 & 0.64 & 0.93 & 0.60 \\
43972 & 109249 & 2002 & 118.70 & -0.41 & 17465.00 & 116328.41 & 0.68 & 0.98 & 0.67 \\
49319 & 240269 & 2002 & 208.20 & 0.02 & 22251.00 & 212021.12 & 0.94 & 1.02 & 0.95 \\
46778 & 200297 & 2002 & 36.20 & -0.04 & 3261.00 & 32606.12 & 1.11 & 0.90 & 1.00 \\
10131 & 101262 & 2002 & 12.20 & -0.26 & 1220.00 & 12130.45 & 1.00 & 0.99 & 0.99 \\
46800 & 200298 & 2002 & 28.90 & -0.13 & 3444.00 & 24656.39 & 0.84 & 0.85 & 0.72 \\
37986 & 107179 & 2002 & 787.60 & -0.23 & 82174.00 & 773593.58 & 0.96 & 0.98 & 0.94 \\
34473 & 106244 & 2002 & 1.50 & -0.48 & 147.00 & 1421.12 & 1.02 & 0.95 & 0.97 \\
43940 & 109237 & 2002 & 41.40 & -0.11 & 4090.00 & 39821.69 & 1.01 & 0.96 & 0.97 \\
12192 & 101518 & 2002 & 249.90 & -0.21 & 25036.00 & 247631.81 & 1.00 & 0.99 & 0.99 \\
15760 & 102017 & 2002 & 6612.30 & -0.22 & 661224.00 & 6340248.34 & 1.00 & 0.96 & 0.96 \\
37998 & 107181 & 2002 & 77.30 & -0.09 & 8202.00 & 67882.45 & 0.94 & 0.88 & 0.83 \\
34446 & 106240 & 2002 & 142.10 & -0.16 & 18057.00 & 147361.47 & 0.79 & 1.04 & 0.82 \\
42827 & 109023 & 2002 & 14.50 & -0.27 & 1284.00 & 13817.83 & 1.13 & 0.95 & 1.08 \\
1538 & 100213 & 2002 & 157.50 & -0.24 & 15703.00 & 148785.93 & 1.00 & 0.94 & 0.95 \\
42825 & 109022 & 2002 & 5.10 & -0.40 & 417.00 & 5106.10 & 1.22 & 1.00 & 1.22 \\
20887 & 102798 & 2002 & 191.00 & -0.27 & 19102.00 & 186240.02 & 1.00 & 0.98 & 0.97 \\
2202 & 100295 & 2002 & 13.60 & -0.17 & 1728.00 & 11845.28 & 0.79 & 0.87 & 0.69 \\
42822 & 109021 & 2002 & 16.20 & -0.84 & 2331.00 & 16107.05 & 0.69 & 0.99 & 0.69 \\
8100 & 101074 & 2002 & 153.70 & -0.10 & 14928.00 & 127505.61 & 1.03 & 0.83 & 0.85 \\
24575 & 103369 & 2002 & 541.40 & -0.31 & 77959.00 & 524369.29 & 0.69 & 0.97 & 0.67 \\
31070 & 105854 & 2002 & 576.80 & -0.28 & 58336.00 & 542384.82 & 0.99 & 0.94 & 0.93 \\
25149 & 103439 & 2002 & 48.60 & -0.29 & 4861.00 & 48612.69 & 1.00 & 1.00 & 1.00 \\
13416 & 101738 & 2002 & 1626.00 & -0.28 & 187572.00 & 1616205.39 & 0.87 & 0.99 & 0.86 \\
43958 & 109238 & 2002 & 8.40 & -0.19 & 936.00 & 8738.90 & 0.90 & 1.04 & 0.93 \\
96677 & 611003 & 2002 & 193.20 & -0.24 & 16131.00 & 166405.71 & 1.20 & 0.86 & 1.03 \\
31085 & 105857 & 2002 & 281.30 & -0.10 & 28386.00 & 271381.79 & 0.99 & 0.96 & 0.96 \\
34205 & 106212 & 2002 & 74.40 & -0.21 & 7416.00 & 67646.03 & 1.00 & 0.91 & 0.91 \\
20257 & 102696 & 2002 & 220.30 & -0.24 & 22371.00 & 220453.32 & 0.98 & 1.00 & 0.99 \\
55348 & 400088 & 2002 & 67.50 & -0.31 & 6766.00 & 57847.17 & 1.00 & 0.86 & 0.85 \\
30982 & 105843 & 2002 & 3.50 & -0.25 & 326.00 & 3211.80 & 1.07 & 0.92 & 0.99 \\
26666 & 103597 & 2002 & 3235.90 & -0.17 & 323591.00 & 2752875.20 & 1.00 & 0.85 & 0.85 \\
54265 & 365483 & 2002 & 253.30 & -0.13 & 25630.00 & 244222.49 & 0.99 & 0.96 & 0.95 \\
34535 & 106250 & 2002 & 62.80 & -0.22 & 5941.00 & 58296.90 & 1.06 & 0.93 & 0.98 \\
32617 & 106043 & 2002 & 1911.70 & -0.22 & 243266.00 & 2115301.50 & 0.79 & 1.11 & 0.87 \\
1520 & 100209 & 2002 & 6397.50 & -0.34 & 642887.00 & 5917437.89 & 1.00 & 0.92 & 0.92 \\
65023 & 500656 & 2002 & 486.30 & -0.45 & 32634.00 & 390323.66 & 1.49 & 0.80 & 1.20 \\
30962 & 105842 & 2002 & 432.90 & -0.44 & 43121.00 & 412041.97 & 1.00 & 0.95 & 0.96 \\
5217 & 100736 & 2002 & 510.40 & -0.23 & 53143.00 & 492617.72 & 0.96 & 0.97 & 0.93 \\
6117 & 100823 & 2002 & 25.60 & -0.08 & 2544.00 & 25438.83 & 1.01 & 0.99 & 1.00 \\
20950 & 102813 & 2002 & 819.20 & -0.06 & 73512.00 & 733882.62 & 1.11 & 0.90 & 1.00 \\
10169 & 101264 & 2002 & 643.40 & -0.33 & 107512.00 & 619793.06 & 0.60 & 0.96 & 0.58 \\
26894 & 103620 & 2002 & 142.50 & -0.16 & 14382.00 & 130199.95 & 0.99 & 0.91 & 0.91 \\
32810 & 106066 & 2002 & 975.10 & -0.28 & 87542.00 & 812366.68 & 1.11 & 0.83 & 0.93 \\
31024 & 105848 & 2002 & 56.80 & -0.26 & 5787.00 & 52978.51 & 0.98 & 0.93 & 0.92 \\
45677 & 200089 & 2002 & 30.50 & -0.07 & 3213.00 & 33434.99 & 0.95 & 1.10 & 1.04 \\
31016 & 105847 & 2002 & 32.30 & -0.18 & 3240.00 & 31183.08 & 1.00 & 0.97 & 0.96 \\
20915 & 102799 & 2002 & 9.00 & -0.29 & 900.00 & 8894.73 & 1.00 & 0.99 & 0.99 \\
54844 & 400018 & 2002 & 126.80 & -0.22 & 12736.00 & 125277.59 & 1.00 & 0.99 & 0.98 \\
19369 & 102599 & 2002 & 1211.30 & -0.12 & 121129.00 & 1159121.46 & 1.00 & 0.96 & 0.96 \\
52689 & 306690 & 2002 & 24.60 & -0.18 & 2474.00 & 24422.83 & 0.99 & 0.99 & 0.99 \\
775 & 100096 & 2002 & 65.30 & -0.34 & 6528.00 & 63240.54 & 1.00 & 0.97 & 0.97 \\
10148 & 101263 & 2002 & 193.80 & -0.26 & 29266.00 & 178887.04 & 0.66 & 0.92 & 0.61 \\
48328 & 240062 & 2002 & 733.00 & -0.23 & 79025.00 & 589942.15 & 0.93 & 0.80 & 0.75 \\
34526 & 106249 & 2002 & 83.00 & -0.23 & 8671.00 & 69022.33 & 0.96 & 0.83 & 0.80 \\
39506 & 107711 & 2002 & 18.80 & -0.26 & 1886.00 & 18719.35 & 1.00 & 1.00 & 0.99 \\
15369 & 101988 & 2002 & 329.10 & -0.15 & 31111.00 & 322417.46 & 1.06 & 0.98 & 1.04 \\
4132 & 100559 & 2002 & 19.10 & -0.30 & 1913.00 & 18076.94 & 1.00 & 0.95 & 0.94 \\
20929 & 102802 & 2002 & 195.80 & -0.21 & 19604.00 & 189272.47 & 1.00 & 0.97 & 0.97 \\
15791 & 102018 & 2002 & 397.90 & -0.33 & 39783.00 & 373924.80 & 1.00 & 0.94 & 0.94 \\
46896 & 200312 & 2002 & 59.00 & -0.10 & 5952.00 & 56698.35 & 0.99 & 0.96 & 0.95 \\
38018 & 107192 & 2002 & 912.50 & -0.25 & 89296.00 & 755160.39 & 1.02 & 0.83 & 0.85 \\
43868 & 109226 & 2002 & 60.90 & -0.08 & 5817.00 & 59399.97 & 1.05 & 0.98 & 1.02 \\
43124 & 109065 & 2002 & 26.90 & -0.25 & 2671.00 & 26505.90 & 1.01 & 0.99 & 0.99 \\
33766 & 106169 & 2002 & 133.20 & -0.10 & 13334.00 & 130243.55 & 1.00 & 0.98 & 0.98 \\
26764 & 103606 & 2002 & 26.80 & -0.23 & 2678.00 & 25737.20 & 1.00 & 0.96 & 0.96 \\
23555 & 103186 & 2002 & 616.70 & -0.25 & 85079.00 & 711377.66 & 0.72 & 1.15 & 0.84 \\
43101 & 109064 & 2002 & 63.50 & -0.29 & 6871.00 & 61423.30 & 0.92 & 0.97 & 0.89 \\
61492 & 500083 & 2002 & 16.30 & 0.03 & 920.00 & 13135.62 & 1.77 & 0.81 & 1.43 \\
39806 & 107875 & 2002 & 199.90 & -0.18 & 25863.00 & 183003.63 & 0.77 & 0.92 & 0.71 \\
13516 & 101742 & 2002 & 4879.00 & -0.29 & 511300.00 & 4576492.44 & 0.95 & 0.94 & 0.90 \\
31754 & 105933 & 2002 & 1292.60 & -0.30 & 129247.00 & 1237698.02 & 1.00 & 0.96 & 0.96 \\
20451 & 102744 & 2002 & 905.40 & -0.24 & 87623.00 & 938059.38 & 1.03 & 1.04 & 1.07 \\
39800 & 107874 & 2002 & 175.30 & -0.10 & 17095.00 & 170485.21 & 1.03 & 0.97 & 1.00 \\
33754 & 106167 & 2002 & 271.20 & -0.13 & 27123.00 & 244048.61 & 1.00 & 0.90 & 0.90 \\
26725 & 103601 & 2002 & 2326.90 & -0.17 & 232692.00 & 1912827.67 & 1.00 & 0.82 & 0.82 \\
43152 & 109069 & 2002 & 154.00 & -0.23 & 15399.00 & 148946.98 & 1.00 & 0.97 & 0.97 \\
18010 & 102386 & 2002 & 53.30 & -0.08 & 5356.00 & 52082.90 & 1.00 & 0.98 & 0.97 \\
6672 & 100908 & 2002 & 157.00 & -0.18 & 15653.00 & 147837.23 & 1.00 & 0.94 & 0.94 \\
33646 & 106158 & 2002 & 250.70 & -0.21 & 25422.00 & 242867.64 & 0.99 & 0.97 & 0.96 \\
39710 & 107837 & 2002 & 30.40 & -0.00 & 3083.00 & 29255.61 & 0.99 & 0.96 & 0.95 \\
20421 & 102737 & 2002 & 1879.10 & -0.23 & 178890.00 & 1788932.72 & 1.05 & 0.95 & 1.00 \\
43629 & 109170 & 2002 & 29.80 & -0.38 & 5232.00 & 31120.17 & 0.57 & 1.04 & 0.59 \\
12275 & 101531 & 2002 & 57.70 & -0.04 & 6100.00 & 59333.21 & 0.95 & 1.03 & 0.97 \\
64820 & 500635 & 2002 & 14.50 & -0.32 & 2132.00 & 14250.88 & 0.68 & 0.98 & 0.67 \\
52132 & 302060 & 2002 & 82.70 & -0.31 & 8278.00 & 79988.31 & 1.00 & 0.97 & 0.97 \\
52104 & 301571 & 2002 & 299.70 & -0.26 & 26022.00 & 268809.82 & 1.15 & 0.90 & 1.03 \\
7116 & 100997 & 2002 & 380.70 & -0.15 & 37958.00 & 376338.20 & 1.00 & 0.99 & 0.99 \\
33793 & 106170 & 2002 & 222.40 & -0.13 & 22527.00 & 211473.17 & 0.99 & 0.95 & 0.94 \\
43085 & 109062 & 2002 & 16.40 & -0.22 & 1641.00 & 16160.16 & 1.00 & 0.99 & 0.98 \\
20304 & 102715 & 2002 & 1840.20 & -0.23 & 179096.00 & 1749945.13 & 1.03 & 0.95 & 0.98 \\
43787 & 109221 & 2002 & 266.20 & -0.09 & 23118.00 & 203221.88 & 1.15 & 0.76 & 0.88 \\
9118 & 101112 & 2002 & 724.30 & -0.21 & 93671.00 & 702277.93 & 0.77 & 0.97 & 0.75 \\
43080 & 109061 & 2002 & 115.30 & -0.34 & 10431.00 & 108797.51 & 1.11 & 0.94 & 1.04 \\
47510 & 212408 & 2002 & 1842.20 & -0.29 & 184588.00 & 1810796.53 & 1.00 & 0.98 & 0.98 \\
49269 & 240261 & 2002 & 100.70 & -0.15 & 10042.00 & 88831.89 & 1.00 & 0.88 & 0.88 \\
15531 & 102000 & 2002 & 296.80 & -0.31 & 29675.00 & 288905.00 & 1.00 & 0.97 & 0.97 \\
9728 & 101179 & 2002 & 608.10 & -0.27 & 57287.00 & 548832.31 & 1.06 & 0.90 & 0.96 \\
49607 & 240321 & 2002 & 2.10 & -0.42 & 285.00 & 2696.28 & 0.74 & 1.28 & 0.95 \\
33875 & 106179 & 2002 & 54.50 & -0.25 & 5448.00 & 53157.25 & 1.00 & 0.98 & 0.98 \\
45047 & 109415 & 2002 & 26.90 & -0.10 & 4022.00 & 37876.12 & 0.67 & 1.41 & 0.94 \\
31681 & 105930 & 2002 & 287.60 & -0.41 & 28981.00 & 286483.56 & 0.99 & 1.00 & 0.99 \\
48538 & 240105 & 2002 & 123.80 & -0.42 & 12772.00 & 110063.01 & 0.97 & 0.89 & 0.86 \\
9862 & 101198 & 2002 & 155.10 & -0.28 & 15527.00 & 148764.05 & 1.00 & 0.96 & 0.96 \\
49284 & 240264 & 2002 & 351.60 & -0.21 & 36120.00 & 324749.38 & 0.97 & 0.92 & 0.90 \\
61558 & 500094 & 2002 & 56.30 & 0.05 & 5616.00 & 52767.57 & 1.00 & 0.94 & 0.94 \\
31700 & 105931 & 2002 & 3597.30 & -0.14 & 359725.00 & 3199237.23 & 1.00 & 0.89 & 0.89 \\
53989 & 363013 & 2002 & 20.10 & -0.29 & 2010.00 & 19626.81 & 1.00 & 0.98 & 0.98 \\
43097 & 109063 & 2002 & 9.40 & -0.41 & 1259.00 & 8144.14 & 0.75 & 0.87 & 0.65 \\
33806 & 106172 & 2002 & 47.00 & -0.28 & 6332.00 & 53299.43 & 0.74 & 1.13 & 0.84 \\
45058 & 109416 & 2002 & 17.20 & -0.11 & 1530.00 & 14400.97 & 1.12 & 0.84 & 0.94 \\
31727 & 105932 & 2002 & 129.20 & -0.22 & 12925.00 & 123320.47 & 1.00 & 0.95 & 0.95 \\
24884 & 103383 & 2002 & 953.00 & -0.21 & 125501.00 & 963932.49 & 0.76 & 1.01 & 0.77 \\
43780 & 109219 & 2002 & 25.10 & -0.36 & 2511.00 & 25109.23 & 1.00 & 1.00 & 1.00 \\
52091 & 301560 & 2002 & 650.20 & -0.06 & 56806.00 & 571134.54 & 1.14 & 0.88 & 1.01 \\
15047 & 101953 & 2002 & 222.60 & -0.24 & 22272.00 & 217261.78 & 1.00 & 0.98 & 0.98 \\
56337 & 400192 & 2002 & 6.50 & -0.04 & 742.00 & 5328.32 & 0.88 & 0.82 & 0.72 \\
33833 & 106173 & 2002 & 557.60 & -0.15 & 55624.00 & 556253.21 & 1.00 & 1.00 & 1.00 \\
38066 & 107198 & 2002 & 92.60 & -0.18 & 9237.00 & 92366.25 & 1.00 & 1.00 & 1.00 \\
54889 & 400020 & 2002 & 43.10 & -0.25 & 4296.00 & 39530.55 & 1.00 & 0.92 & 0.92 \\
39682 & 107835 & 2002 & 259.20 & -0.17 & 25839.00 & 256575.76 & 1.00 & 0.99 & 0.99 \\
17996 & 102383 & 2002 & 5.00 & -0.21 & 482.00 & 4817.43 & 1.04 & 0.96 & 1.00 \\
20473 & 102757 & 2002 & 9170.80 & -0.31 & 919148.00 & 8539008.94 & 1.00 & 0.93 & 0.93 \\
6260 & 100833 & 2002 & 562.50 & -0.30 & 56280.00 & 552385.31 & 1.00 & 0.98 & 0.98 \\
24727 & 103376 & 2002 & 4328.40 & -0.26 & 577856.00 & 3987091.73 & 0.75 & 0.92 & 0.69 \\
96705 & 611008 & 2002 & 38.30 & -0.27 & 5164.00 & 51098.41 & 0.74 & 1.33 & 0.99 \\
52140 & 302067 & 2002 & 11.10 & -0.20 & 978.00 & 10865.89 & 1.13 & 0.98 & 1.11 \\
52396 & 302879 & 2002 & 60.50 & -0.11 & 6086.00 & 60149.00 & 0.99 & 0.99 & 0.99 \\
31939 & 105963 & 2002 & 563.60 & -0.23 & 62318.00 & 458275.20 & 0.90 & 0.81 & 0.74 \\
1749 & 100227 & 2002 & 99.70 & -0.26 & 9613.00 & 88424.45 & 1.04 & 0.89 & 0.92 \\
52163 & 302545 & 2002 & 82.90 & -0.25 & 8304.00 & 77327.03 & 1.00 & 0.93 & 0.93 \\
39774 & 107870 & 2002 & 285.00 & -0.23 & 41486.00 & 278002.90 & 0.69 & 0.98 & 0.67 \\
43187 & 109077 & 2002 & 6.80 & -0.62 & 821.00 & 7120.17 & 0.83 & 1.05 & 0.87 \\
31928 & 105961 & 2002 & 128.80 & -0.32 & 12870.00 & 124736.06 & 1.00 & 0.97 & 0.97 \\
45086 & 109427 & 2002 & 19.00 & -0.10 & 1194.00 & 10847.28 & 1.59 & 0.57 & 0.91 \\
24803 & 103380 & 2002 & 3650.30 & -0.23 & 478014.00 & 3340244.00 & 0.76 & 0.92 & 0.70 \\
17342 & 102282 & 2002 & 86.60 & -0.15 & 12213.00 & 97816.69 & 0.71 & 1.13 & 0.80 \\
43180 & 109076 & 2002 & 197.30 & -0.11 & 20098.00 & 190344.96 & 0.98 & 0.96 & 0.95 \\
47113 & 200334 & 2002 & 7.90 & -0.05 & 782.00 & 7596.65 & 1.01 & 0.96 & 0.97 \\
9771 & 101192 & 2002 & 170.20 & -0.20 & 16897.00 & 161647.54 & 1.01 & 0.95 & 0.96 \\
31903 & 105957 & 2002 & 7.70 & -0.29 & 740.00 & 7726.88 & 1.04 & 1.00 & 1.04 \\
43172 & 109074 & 2002 & 13.20 & -0.40 & 1294.00 & 12945.35 & 1.02 & 0.98 & 1.00 \\
55310 & 400084 & 2002 & 16.80 & -0.32 & 1598.00 & 16759.63 & 1.05 & 1.00 & 1.05 \\
31918 & 105960 & 2002 & 171.00 & -0.30 & 17096.00 & 165719.48 & 1.00 & 0.97 & 0.97 \\
31898 & 105954 & 2002 & 3.20 & -0.17 & 326.00 & 3115.24 & 0.98 & 0.97 & 0.96 \\
33703 & 106163 & 2002 & 459.10 & -0.05 & 45908.00 & 437615.58 & 1.00 & 0.95 & 0.95 \\
15427 & 101990 & 2002 & 214.10 & -0.12 & 18544.00 & 189759.80 & 1.15 & 0.89 & 1.02 \\
31966 & 105964 & 2002 & 103.80 & -0.30 & 10178.00 & 103564.62 & 1.02 & 1.00 & 1.02 \\
47478 & 212351 & 2002 & 137.30 & -0.25 & 13653.00 & 134215.07 & 1.01 & 0.98 & 0.98 \\
64774 & 500633 & 2002 & 15.70 & -0.39 & 2192.00 & 18244.08 & 0.72 & 1.16 & 0.83 \\
33670 & 106160 & 2002 & 17.60 & -0.16 & 1756.00 & 17526.22 & 1.00 & 1.00 & 1.00 \\
54236 & 364951 & 2002 & 238.90 & -0.08 & 16376.00 & 142149.77 & 1.46 & 0.60 & 0.87 \\
48320 & 240061 & 2002 & 220.90 & -0.18 & 21534.00 & 179317.94 & 1.03 & 0.81 & 0.83 \\
52546 & 303123 & 2002 & 87.70 & -0.20 & 8757.00 & 83157.89 & 1.00 & 0.95 & 0.95 \\
38116 & 107202 & 2002 & 16.10 & -0.16 & 1618.00 & 15228.53 & 1.00 & 0.95 & 0.94 \\
58248 & 410133 & 2002 & 6.50 & -0.08 & 652.00 & 6458.29 & 1.00 & 0.99 & 0.99 \\
3683 & 100468 & 2002 & 1472.10 & -0.10 & 147207.00 & 1412198.71 & 1.00 & 0.96 & 0.96 \\
43204 & 109084 & 2002 & 178.00 & -0.23 & 9675.00 & 87169.56 & 1.84 & 0.49 & 0.90 \\
31974 & 105965 & 2002 & 74.10 & -0.21 & 7006.00 & 68834.64 & 1.06 & 0.93 & 0.98 \\
23574 & 103193 & 2002 & 37.60 & -0.21 & 3460.00 & 34596.88 & 1.09 & 0.92 & 1.00 \\
48495 & 240090 & 2002 & 22.60 & -0.18 & 2181.00 & 19618.19 & 1.04 & 0.87 & 0.90 \\
4065 & 100544 & 2002 & 339.60 & 0.04 & 36634.00 & 357949.86 & 0.93 & 1.05 & 0.98 \\
12310 & 101534 & 2002 & 1330.40 & -0.13 & 132635.00 & 1305810.77 & 1.00 & 0.98 & 0.98 \\
20338 & 102716 & 2002 & 808.60 & -0.22 & 75227.00 & 804496.32 & 1.07 & 0.99 & 1.07 \\
9753 & 101186 & 2002 & 403.10 & -0.26 & 40794.00 & 384612.83 & 0.99 & 0.95 & 0.94 \\
39790 & 107872 & 2002 & 46.20 & -0.12 & 4656.00 & 45245.28 & 0.99 & 0.98 & 0.97 \\
49258 & 240256 & 2002 & 67.20 & -0.21 & 6404.00 & 69111.09 & 1.05 & 1.03 & 1.08 \\
20389 & 102733 & 2002 & 2282.80 & -0.26 & 219144.00 & 2105131.14 & 1.04 & 0.92 & 0.96 \\
43227 & 109085 & 2002 & 19.00 & -0.24 & 1924.00 & 19145.34 & 0.99 & 1.01 & 1.00 \\
18045 & 102387 & 2002 & 146.40 & -0.01 & 14587.00 & 135828.94 & 1.00 & 0.93 & 0.93 \\
31844 & 105947 & 2002 & 32.70 & -0.19 & 4030.00 & 31975.62 & 0.81 & 0.98 & 0.79 \\
24843 & 103381 & 2002 & 18913.60 & -0.23 & 2604506.00 & 18359484.37 & 0.73 & 0.97 & 0.70 \\
31855 & 105948 & 2002 & 23.50 & -0.14 & 2921.00 & 23724.58 & 0.80 & 1.01 & 0.81 \\
43167 & 109072 & 2002 & 135.20 & -0.22 & 13280.00 & 132802.79 & 1.02 & 0.98 & 1.00 \\
9151 & 101115 & 2002 & 12966.90 & -0.22 & 1610290.00 & 12579738.58 & 0.81 & 0.97 & 0.78 \\
2138 & 100292 & 2002 & 1190.10 & -0.05 & 119012.00 & 1134835.92 & 1.00 & 0.95 & 0.95 \\
31833 & 105946 & 2002 & 95.40 & -0.12 & 12329.00 & 79031.28 & 0.77 & 0.83 & 0.64 \\
55661 & 400135 & 2002 & 17.50 & -0.02 & 3499.00 & 17082.54 & 0.50 & 0.98 & 0.49 \\
43161 & 109071 & 2002 & 10.00 & -0.36 & 1003.00 & 9827.65 & 1.00 & 0.98 & 0.98 \\
9801 & 101193 & 2002 & 401.90 & -0.23 & 40072.00 & 352650.64 & 1.00 & 0.88 & 0.88 \\
43159 & 109070 & 2002 & 1.60 & -0.22 & 163.00 & 1608.53 & 0.98 & 1.01 & 0.99 \\
39731 & 107858 & 2002 & 14.60 & -0.15 & 1499.00 & 14313.48 & 0.97 & 0.98 & 0.95 \\
6304 & 100847 & 2002 & 3.20 & -0.11 & 347.00 & 3028.59 & 0.92 & 0.95 & 0.87 \\
47091 & 200333 & 2002 & 393.70 & -0.07 & NaN & 272373.79 & 1.00 & 0.69 & 1.00 \\
8070 & 101073 & 2002 & 5974.40 & -0.23 & 826431.00 & 5146320.80 & 0.72 & 0.86 & 0.62 \\
14914 & 101922 & 2002 & 364.20 & -0.15 & 30512.00 & 325806.59 & 1.19 & 0.89 & 1.07 \\
31891 & 105951 & 2002 & 174.80 & -0.39 & 17493.00 & 171952.43 & 1.00 & 0.98 & 0.98 \\
61424 & 500064 & 2002 & 31.50 & 0.04 & 3083.00 & 26288.17 & 1.02 & 0.83 & 0.85 \\
45706 & 200092 & 2002 & 3.80 & 0.00 & 384.00 & 3692.88 & 0.99 & 0.97 & 0.96 \\
56198 & 400181 & 2002 & 7.60 & -0.04 & 1427.00 & 14265.81 & 0.53 & 1.88 & 1.00 \\
33730 & 106164 & 2002 & 33.40 & -0.37 & 3312.00 & 32333.91 & 1.01 & 0.97 & 0.98 \\
52152 & 302206 & 2002 & 495.50 & -0.12 & 49171.00 & 399739.11 & 1.01 & 0.81 & 0.81 \\
48526 & 240103 & 2002 & 151.90 & -0.15 & 15503.00 & 135955.60 & 0.98 & 0.90 & 0.88 \\
805 & 100097 & 2002 & 119.70 & -0.34 & 11966.00 & 112257.52 & 1.00 & 0.94 & 0.94 \\
61482 & 500082 & 2002 & 102.20 & -0.22 & 8999.00 & 86679.74 & 1.14 & 0.85 & 0.96 \\
24764 & 103377 & 2002 & 1159.00 & -0.16 & 149131.00 & 1083432.64 & 0.78 & 0.93 & 0.73 \\
33746 & 106165 & 2002 & 74.30 & -0.38 & 7513.00 & 73470.89 & 0.99 & 0.99 & 0.98 \\
31866 & 105949 & 2002 & 323.80 & -0.19 & 32402.00 & 296037.04 & 1.00 & 0.91 & 0.91 \\
6333 & 100849 & 2002 & 50.60 & -0.33 & 4829.00 & 48292.60 & 1.05 & 0.95 & 1.00 \\
19557 & 102624 & 2002 & 156.00 & -0.27 & 15382.00 & 153820.75 & 1.01 & 0.99 & 1.00 \\
4843 & 100685 & 2002 & 13.70 & -0.12 & 1152.00 & 11701.14 & 1.19 & 0.85 & 1.02 \\
31468 & 105895 & 2002 & 572.50 & -0.39 & 101780.00 & 562276.78 & 0.56 & 0.98 & 0.55 \\
33021 & 106086 & 2002 & 25.30 & -0.25 & 2790.00 & 23732.33 & 0.91 & 0.94 & 0.85 \\
20621 & 102775 & 2002 & 2169.00 & -0.13 & 184934.00 & 1775424.59 & 1.17 & 0.82 & 0.96 \\
9086 & 101111 & 2002 & 396.90 & -0.08 & 47519.00 & 409651.01 & 0.84 & 1.03 & 0.86 \\
6689 & 100910 & 2002 & 98.20 & -0.15 & 9809.00 & 91071.05 & 1.00 & 0.93 & 0.93 \\
19471 & 102606 & 2002 & 3213.70 & -0.19 & 321374.00 & 2982735.76 & 1.00 & 0.93 & 0.93 \\
61400 & 500048 & 2002 & 39.10 & -0.20 & 4906.00 & 38329.32 & 0.80 & 0.98 & 0.78 \\
14331 & 101850 & 2002 & 445.10 & -0.18 & 39990.00 & 425655.15 & 1.11 & 0.96 & 1.06 \\
24995 & 103406 & 2002 & 1681.50 & -0.23 & 168364.00 & 1486557.03 & 1.00 & 0.88 & 0.88 \\
43004 & 109048 & 2002 & 183.40 & -0.22 & 27141.00 & 191754.18 & 0.68 & 1.05 & 0.71 \\
31451 & 105890 & 2002 & 36.50 & -0.30 & 3646.00 & 34243.92 & 1.00 & 0.94 & 0.94 \\
48585 & 240111 & 2002 & 1278.20 & -0.30 & 137923.00 & 1153606.23 & 0.93 & 0.90 & 0.84 \\
12232 & 101528 & 2002 & 66.50 & -0.09 & 6780.00 & 57778.57 & 0.98 & 0.87 & 0.85 \\
43842 & 109224 & 2002 & 3.70 & -0.26 & 371.00 & 3656.61 & 1.00 & 0.99 & 0.99 \\
64144 & 500589 & 2002 & 1220.90 & -0.15 & 65816.00 & 623411.05 & 1.86 & 0.51 & 0.95 \\
32838 & 106067 & 2002 & 4559.30 & -0.23 & 414676.00 & 4244860.89 & 1.10 & 0.93 & 1.02 \\
17312 & 102280 & 2002 & 1380.00 & -0.25 & 138002.00 & 1365613.18 & 1.00 & 0.99 & 0.99 \\
31504 & 105903 & 2002 & 64.30 & -0.18 & 7076.00 & 52897.52 & 0.91 & 0.82 & 0.75 \\
49241 & 240254 & 2002 & 431.40 & -0.18 & 58110.00 & 512197.21 & 0.74 & 1.19 & 0.88 \\
20601 & 102774 & 2002 & 3612.60 & -0.31 & 365775.00 & 3464624.41 & 0.99 & 0.96 & 0.95 \\
46931 & 200319 & 2002 & 41.50 & -0.11 & 4136.00 & 37773.43 & 1.00 & 0.91 & 0.91 \\
68 & 100004 & 2002 & 1122.60 & -0.35 & 177744.00 & 1130204.43 & 0.63 & 1.01 & 0.64 \\
52020 & 300777 & 2002 & 12.60 & -0.25 & 1175.00 & 11725.29 & 1.07 & 0.93 & 1.00 \\
34070 & 106203 & 2002 & 53.10 & -0.14 & 8266.00 & 46920.89 & 0.64 & 0.88 & 0.57 \\
43023 & 109051 & 2002 & 18.90 & -0.50 & 1919.00 & 17002.14 & 0.98 & 0.90 & 0.89 \\
55333 & 400085 & 2002 & 28.50 & -0.26 & 2780.00 & 27287.16 & 1.03 & 0.96 & 0.98 \\
34096 & 106207 & 2002 & 9.40 & -0.44 & 964.00 & 9308.70 & 0.98 & 0.99 & 0.97 \\
1826 & 100244 & 2002 & 130.70 & -0.29 & 13053.00 & 127898.26 & 1.00 & 0.98 & 0.98 \\
42980 & 109046 & 2002 & 56.30 & -0.30 & 5846.00 & 57840.56 & 0.96 & 1.03 & 0.99 \\
7316 & 101020 & 2002 & 1795.70 & -0.21 & 219232.00 & 1589081.18 & 0.82 & 0.88 & 0.72 \\
26696 & 103600 & 2002 & 195.20 & -0.29 & 27902.00 & 179099.90 & 0.70 & 0.92 & 0.64 \\
51965 & 300679 & 2002 & 998.00 & -0.15 & 87661.00 & 761428.40 & 1.14 & 0.76 & 0.87 \\
15641 & 102010 & 2002 & 4796.70 & -0.36 & 479646.00 & 4690677.80 & 1.00 & 0.98 & 0.98 \\
45180 & 109436 & 2002 & 22.90 & -0.19 & 2290.00 & 22595.68 & 1.00 & 0.99 & 0.99 \\
40893 & 108161 & 2002 & 215.30 & -0.21 & 21521.00 & 209393.79 & 1.00 & 0.97 & 0.97 \\
64935 & 500646 & 2002 & 1.30 & -0.42 & 133.00 & 1237.37 & 0.98 & 0.95 & 0.93 \\
56176 & 400180 & 2002 & 18.20 & -0.06 & 1873.00 & 18735.35 & 0.97 & 1.03 & 1.00 \\
31392 & 105881 & 2002 & 768.20 & -0.05 & 57594.00 & 710314.56 & 1.33 & 0.92 & 1.23 \\
38187 & 107215 & 2002 & 225.40 & 0.03 & 23146.00 & 238277.90 & 0.97 & 1.06 & 1.03 \\
46920 & 200317 & 2002 & 16.60 & -0.02 & 1676.00 & 16089.56 & 0.99 & 0.97 & 0.96 \\
46919 & 200316 & 2002 & 4.10 & -0.35 & 334.00 & 4125.31 & 1.23 & 1.01 & 1.24 \\
7394 & 101038 & 2002 & 2895.40 & -0.23 & 396801.00 & 2697777.36 & 0.73 & 0.93 & 0.68 \\
34193 & 106211 & 2002 & 63.60 & -0.19 & 6354.00 & 58762.97 & 1.00 & 0.92 & 0.92 \\
24647 & 103373 & 2002 & 48.50 & -0.26 & 4789.00 & 48605.64 & 1.01 & 1.00 & 1.01 \\
42966 & 109044 & 2002 & 119.60 & -0.25 & 12067.00 & 102070.92 & 0.99 & 0.85 & 0.85 \\
32031 & 105974 & 2002 & 28.00 & -0.25 & 2898.00 & 28655.49 & 0.97 & 1.02 & 0.99 \\
39601 & 107786 & 2002 & 1085.90 & -0.27 & 105110.00 & 1120100.68 & 1.03 & 1.03 & 1.07 \\
4810 & 100682 & 2002 & 61.00 & -0.14 & 6580.00 & 53684.18 & 0.93 & 0.88 & 0.82 \\
31441 & 105886 & 2002 & 35.20 & -0.25 & 5001.00 & 36847.36 & 0.70 & 1.05 & 0.74 \\
60803 & 410729 & 2002 & 18.80 & -0.14 & 1877.00 & 17782.80 & 1.00 & 0.95 & 0.95 \\
53942 & 362337 & 2002 & 4.70 & -0.14 & 492.00 & 4195.60 & 0.96 & 0.89 & 0.85 \\
26823 & 103608 & 2002 & 29.50 & -0.21 & 2942.00 & 28841.79 & 1.00 & 0.98 & 0.98 \\
39867 & 107892 & 2002 & 421.10 & -0.28 & 41689.00 & 349517.02 & 1.01 & 0.83 & 0.84 \\
31432 & 105883 & 2002 & 43.50 & -0.22 & 3901.00 & 39867.95 & 1.12 & 0.92 & 1.02 \\
39628 & 107830 & 2002 & 328.80 & -0.17 & 30066.00 & 321035.13 & 1.09 & 0.98 & 1.07 \\
6199 & 100829 & 2002 & 509.30 & -0.12 & 50909.00 & 471201.44 & 1.00 & 0.93 & 0.93 \\
45173 & 109435 & 2002 & 2.30 & -0.02 & 222.00 & 2287.66 & 1.04 & 0.99 & 1.03 \\
13460 & 101740 & 2002 & 15426.10 & -0.28 & 1634027.00 & 15699310.88 & 0.94 & 1.02 & 0.96 \\
20657 & 102777 & 2002 & 1813.70 & -0.33 & 182014.00 & 1528624.16 & 1.00 & 0.84 & 0.84 \\
31419 & 105882 & 2002 & 111.30 & -0.25 & 11140.00 & 107812.66 & 1.00 & 0.97 & 0.97 \\
46953 & 200322 & 2002 & 25.40 & -0.25 & 2328.00 & 25822.92 & 1.09 & 1.02 & 1.11 \\
52593 & 303140 & 2002 & 355.20 & -0.25 & 33169.00 & 342266.93 & 1.07 & 0.96 & 1.03 \\
39665 & 107833 & 2002 & 248.20 & -0.27 & 24894.00 & 234717.10 & 1.00 & 0.95 & 0.94 \\
33930 & 106192 & 2002 & 1496.20 & -0.19 & 188971.00 & 1259718.42 & 0.79 & 0.84 & 0.67 \\
61569 & 500096 & 2002 & 99.40 & 0.07 & 6585.00 & 84359.22 & 1.51 & 0.85 & 1.28 \\
31623 & 105918 & 2002 & 474.90 & -0.44 & 45260.00 & 458325.62 & 1.05 & 0.97 & 1.01 \\
64190 & 500591 & 2002 & 704.00 & -0.15 & 49772.00 & 479061.51 & 1.41 & 0.68 & 0.96 \\
39813 & 107880 & 2002 & 11.70 & -0.26 & NaN & 10150.97 & 1.00 & 0.87 & 1.00 \\
31613 & 105917 & 2002 & 61.60 & -0.18 & 6185.00 & 59703.74 & 1.00 & 0.97 & 0.97 \\
1686 & 100223 & 2002 & 2369.90 & -0.21 & 236921.00 & 2301806.51 & 1.00 & 0.97 & 0.97 \\
52038 & 301299 & 2002 & 6141.70 & -0.33 & 612274.00 & 6090701.93 & 1.00 & 0.99 & 0.99 \\
9890 & 101200 & 2002 & 23.90 & -0.26 & 2402.00 & 23420.87 & 1.00 & 0.98 & 0.98 \\
6233 & 100831 & 2002 & 115.20 & -0.18 & 11497.00 & 111063.16 & 1.00 & 0.96 & 0.97 \\
32004 & 105973 & 2002 & 15.90 & -0.11 & 1672.00 & 15507.37 & 0.95 & 0.98 & 0.93 \\
12247 & 101530 & 2002 & 923.80 & -0.11 & 109964.00 & 814676.28 & 0.84 & 0.88 & 0.74 \\
11750 & 101460 & 2002 & 1448.70 & -0.26 & 144863.00 & 1381245.54 & 1.00 & 0.95 & 0.95 \\
56346 & 400197 & 2002 & 7.70 & -0.09 & 919.00 & 7556.11 & 0.84 & 0.98 & 0.82 \\
61615 & 500107 & 2002 & 118.70 & -0.43 & 11847.00 & 103363.97 & 1.00 & 0.87 & 0.87 \\
33957 & 106193 & 2002 & 19.50 & -0.41 & 3231.00 & 18169.69 & 0.60 & 0.93 & 0.56 \\
43054 & 109058 & 2002 & 27.70 & -0.20 & 2782.00 & 27246.88 & 1.00 & 0.98 & 0.98 \\
45025 & 109414 & 2002 & 41.70 & -0.15 & 4461.00 & 41512.20 & 0.93 & 1.00 & 0.93 \\
20524 & 102761 & 2002 & 19652.00 & -0.19 & 1885131.00 & 20264484.17 & 1.04 & 1.03 & 1.07 \\
7873 & 101064 & 2002 & 697.00 & 0.64 & 43945.00 & 451289.66 & 1.59 & 0.65 & 1.03 \\
47025 & 200329 & 2002 & 105.50 & -0.15 & 10556.00 & 100890.68 & 1.00 & 0.96 & 0.96 \\
33884 & 106180 & 2002 & 32.00 & -0.39 & 3447.00 & 32409.01 & 0.93 & 1.01 & 0.94 \\
24905 & 103394 & 2002 & 37.90 & -0.12 & 3781.00 & 35622.27 & 1.00 & 0.94 & 0.94 \\
52061 & 301438 & 2002 & 181.30 & -0.13 & 15962.00 & 129921.32 & 1.14 & 0.72 & 0.81 \\
33910 & 106182 & 2002 & 667.00 & -0.18 & 66652.00 & 636673.49 & 1.00 & 0.95 & 0.96 \\
5177 & 100730 & 2002 & 363.50 & -0.28 & 36277.00 & 372744.94 & 1.00 & 1.03 & 1.03 \\
3624 & 100463 & 2002 & 2302.70 & -0.16 & 255133.00 & 2180834.03 & 0.90 & 0.95 & 0.85 \\
31647 & 105920 & 2002 & 2815.20 & -0.31 & 276122.00 & 2614643.45 & 1.02 & 0.93 & 0.95 \\
39676 & 107834 & 2002 & 241.70 & -0.22 & 21016.00 & 218534.78 & 1.15 & 0.90 & 1.04 \\
19511 & 102608 & 2002 & 84.30 & 0.03 & 8435.00 & 81426.00 & 1.00 & 0.97 & 0.97 \\
19578 & 102633 & 2002 & 146.80 & -0.17 & 14520.00 & 138061.44 & 1.01 & 0.94 & 0.95 \\
47005 & 200327 & 2002 & 2.70 & -0.03 & 270.00 & 2160.72 & 1.00 & 0.80 & 0.80 \\
33923 & 106189 & 2002 & 375.30 & -0.27 & 37309.00 & 361309.79 & 1.01 & 0.96 & 0.97 \\
64765 & 500628 & 2002 & 40.10 & -0.45 & 3091.00 & 32513.08 & 1.30 & 0.81 & 1.05 \\
52570 & 303130 & 2002 & 46.40 & -0.15 & 4710.00 & 44030.44 & 0.99 & 0.95 & 0.93 \\
19494 & 102607 & 2002 & 457.50 & -0.30 & 45748.00 & 421294.58 & 1.00 & 0.92 & 0.92 \\
33626 & 106157 & 2002 & 509.00 & -0.33 & 50647.00 & 506207.55 & 1.00 & 0.99 & 1.00 \\
43027 & 109052 & 2002 & 18.60 & -0.22 & 2147.00 & 18111.34 & 0.87 & 0.97 & 0.84 \\
15399 & 101989 & 2002 & 198.60 & -0.32 & 19582.00 & 184865.30 & 1.01 & 0.93 & 0.94 \\
3592 & 100460 & 2002 & 246.10 & -0.24 & 24361.00 & 233194.23 & 1.01 & 0.95 & 0.96 \\
38041 & 107196 & 2002 & 25.80 & -0.29 & 2571.00 & 25007.34 & 1.00 & 0.97 & 0.97 \\
31531 & 105905 & 2002 & 12.40 & -0.17 & 1177.00 & 12641.38 & 1.05 & 1.02 & 1.07 \\
64866 & 500638 & 2002 & 82.70 & -0.22 & 4499.00 & 66426.69 & 1.84 & 0.80 & 1.48 \\
12328 & 101536 & 2002 & 1824.70 & -0.21 & 183268.00 & 1786558.24 & 1.00 & 0.98 & 0.97 \\
46975 & 200324 & 2002 & 186.60 & -0.14 & 14984.00 & 147808.45 & 1.25 & 0.79 & 0.99 \\
3580 & 100457 & 2002 & 154.20 & -0.29 & 15422.00 & 151891.04 & 1.00 & 0.99 & 0.98 \\
34054 & 106199 & 2002 & 134.70 & 0.04 & 14197.00 & 137780.89 & 0.95 & 1.02 & 0.97 \\
17970 & 102377 & 2002 & 61.90 & -0.24 & 5377.00 & 54945.91 & 1.15 & 0.89 & 1.02 \\
15590 & 102007 & 2002 & 2992.10 & -0.25 & 299186.00 & 2849740.52 & 1.00 & 0.95 & 0.95 \\
61635 & 500109 & 2002 & 2126.90 & -0.21 & 267876.00 & 2531111.89 & 0.79 & 1.19 & 0.94 \\
24687 & 103375 & 2002 & 754.80 & -0.21 & 88046.00 & 693690.76 & 0.86 & 0.92 & 0.79 \\
34027 & 106198 & 2002 & 516.30 & -0.02 & 51616.00 & 494963.12 & 1.00 & 0.96 & 0.96 \\
45702 & 200091 & 2002 & 8.60 & -0.09 & 854.00 & 8537.65 & 1.01 & 0.99 & 1.00 \\
15575 & 102005 & 2002 & 448.70 & -0.30 & 44876.00 & 436911.17 & 1.00 & 0.97 & 0.97 \\
24943 & 103395 & 2002 & 83.60 & -0.20 & 8352.00 & 71088.43 & 1.00 & 0.85 & 0.85 \\
9907 & 101211 & 2002 & 260.80 & -0.18 & 34548.00 & 264162.95 & 0.75 & 1.01 & 0.76 \\
43033 & 109056 & 2002 & 495.10 & -0.24 & 72299.00 & 398151.46 & 0.68 & 0.80 & 0.55 \\
26796 & 103607 & 2002 & 89.10 & -0.18 & 9254.00 & 83809.15 & 0.96 & 0.94 & 0.91 \\
39638 & 107832 & 2002 & 138.60 & -0.13 & 13882.00 & 136224.00 & 1.00 & 0.98 & 0.98 \\
39848 & 107882 & 2002 & 188.90 & -0.27 & 29829.00 & 311454.26 & 0.63 & 1.65 & 1.04 \\
20561 & 102767 & 2002 & 5548.70 & -0.24 & 560815.00 & 5420558.04 & 0.99 & 0.98 & 0.97 \\
31563 & 105909 & 2002 & 46.30 & -0.30 & 4682.00 & 46675.90 & 0.99 & 1.01 & 1.00 \\
53960 & 362424 & 2002 & 61.80 & -0.10 & 5718.00 & 53931.75 & 1.08 & 0.87 & 0.94 \\
34000 & 106197 & 2002 & 191.00 & -0.11 & 14999.00 & 179576.93 & 1.27 & 0.94 & 1.20 \\
14877 & 101919 & 2002 & 1102.30 & -0.30 & 108085.00 & 1122999.82 & 1.02 & 1.02 & 1.04 \\
39457 & 107694 & 2002 & 5.60 & -0.24 & 566.00 & 5234.69 & 0.99 & 0.93 & 0.92 \\
47890 & 222658 & 2002 & 263.30 & -0.22 & 26379.00 & 258888.38 & 1.00 & 0.98 & 0.98 \\
64528 & 500607 & 2002 & 170.00 & -0.45 & 17214.00 & 169628.04 & 0.99 & 1.00 & 0.99 \\
29991 & 105665 & 2002 & 63.00 & -0.28 & 6321.00 & 62292.41 & 1.00 & 0.99 & 0.99 \\
35335 & 106348 & 2002 & 85.80 & -0.18 & 8662.00 & 81985.53 & 0.99 & 0.96 & 0.95 \\
21510 & 102876 & 2002 & 10.50 & -0.34 & 1139.00 & 11648.78 & 0.92 & 1.11 & 1.02 \\
60864 & 410733 & 2002 & 232.90 & -0.06 & 34471.00 & 238025.04 & 0.68 & 1.02 & 0.69 \\
12067 & 101494 & 2002 & 182.60 & -0.31 & 18332.00 & 176642.80 & 1.00 & 0.97 & 0.96 \\
39285 & 107648 & 2002 & 252.10 & -0.49 & 25404.00 & 250962.25 & 0.99 & 1.00 & 0.99 \\
29957 & 105662 & 2002 & 82.70 & -0.17 & 8764.00 & 81723.69 & 0.94 & 0.99 & 0.93 \\
51153 & 240487 & 2002 & 2.80 & -0.16 & 210.00 & 2431.06 & 1.33 & 0.87 & 1.16 \\
8761 & 101097 & 2002 & 386.50 & -0.29 & 31159.00 & 284811.83 & 1.24 & 0.74 & 0.91 \\
29945 & 105659 & 2002 & 196.10 & -0.26 & 17253.00 & 187618.31 & 1.14 & 0.96 & 1.09 \\
47438 & 211051 & 2002 & 303.50 & -0.13 & 38462.00 & 275269.89 & 0.79 & 0.91 & 0.72 \\
74924 & 601201 & 2002 & 14.20 & -0.18 & 1583.00 & 11521.95 & 0.90 & 0.81 & 0.73 \\
7280 & 101018 & 2002 & 16682.40 & -0.25 & 2170339.00 & 15745051.27 & 0.77 & 0.94 & 0.73 \\
40991 & 108172 & 2002 & 107.00 & -0.24 & 10724.00 & 94711.40 & 1.00 & 0.89 & 0.88 \\
3185 & 100413 & 2002 & 45.90 & -0.25 & 4396.00 & 37531.06 & 1.04 & 0.82 & 0.85 \\
35322 & 106347 & 2002 & 76.40 & -0.28 & 7650.00 & 75513.48 & 1.00 & 0.99 & 0.99 \\
53866 & 359485 & 2002 & 171.30 & -0.28 & 18118.00 & 156065.93 & 0.95 & 0.91 & 0.86 \\
37941 & 107173 & 2002 & 127.20 & -0.01 & 12655.00 & 122678.11 & 1.01 & 0.96 & 0.97 \\
30020 & 105678 & 2002 & 25.10 & -0.07 & 2499.00 & 23265.35 & 1.00 & 0.93 & 0.93 \\
25499 & 103494 & 2002 & 206.30 & -0.25 & 20691.00 & 166209.79 & 1.00 & 0.81 & 0.80 \\
12446 & 101541 & 2002 & 306.40 & -0.25 & 28962.00 & 289615.55 & 1.06 & 0.95 & 1.00 \\
30011 & 105677 & 2002 & 46.20 & -0.16 & 3825.00 & 42024.96 & 1.21 & 0.91 & 1.10 \\
33481 & 106140 & 2002 & 1082.90 & -0.34 & 108271.00 & 1033197.24 & 1.00 & 0.95 & 0.95 \\
46561 & 200254 & 2002 & 41.60 & 0.09 & 5601.00 & 39703.91 & 0.74 & 0.95 & 0.71 \\
5250 & 100741 & 2002 & 157.40 & -0.13 & 16585.00 & 163100.15 & 0.95 & 1.04 & 0.98 \\
64548 & 500609 & 2002 & 140.30 & -0.13 & 23623.00 & 143232.55 & 0.59 & 1.02 & 0.61 \\
10569 & 101299 & 2002 & 1687.20 & -0.25 & 168723.00 & 1677351.66 & 1.00 & 0.99 & 0.99 \\
7167 & 101000 & 2002 & 1194.60 & -0.19 & 119214.00 & 1166912.01 & 1.00 & 0.98 & 0.98 \\
39301 & 107650 & 2002 & 51.30 & -0.48 & 5222.00 & 51401.61 & 0.98 & 1.00 & 0.98 \\
19147 & 102551 & 2002 & 1022.00 & 0.02 & 83058.00 & 675472.97 & 1.23 & 0.66 & 0.81 \\
20177 & 102673 & 2002 & 388.40 & -0.10 & 38607.00 & 369532.71 & 1.01 & 0.95 & 0.96 \\
21499 & 102875 & 2002 & 11.20 & -0.21 & 1392.00 & 10387.77 & 0.80 & 0.93 & 0.75 \\
30003 & 105676 & 2002 & 361.90 & -0.13 & 25775.00 & 334093.79 & 1.40 & 0.92 & 1.30 \\
40098 & 108029 & 2002 & 327.10 & -0.26 & 48842.00 & 342118.96 & 0.67 & 1.05 & 0.70 \\
9633 & 101160 & 2002 & 288.30 & -0.10 & 27834.00 & 278329.84 & 1.04 & 0.97 & 1.00 \\
74911 & 601197 & 2002 & 25.00 & -0.35 & 2427.00 & 25122.58 & 1.03 & 1.00 & 1.04 \\
35411 & 106358 & 2002 & 2.10 & -0.14 & 222.00 & 1859.59 & 0.95 & 0.89 & 0.84 \\
65347 & 500689 & 2002 & 18.30 & -0.12 & 1847.00 & 18241.89 & 0.99 & 1.00 & 0.99 \\
49108 & 240222 & 2002 & 504.30 & -0.24 & 47666.00 & 468858.03 & 1.06 & 0.93 & 0.98 \\
35418 & 106359 & 2002 & 181.10 & -0.30 & 18157.00 & 179657.33 & 1.00 & 0.99 & 0.99 \\
29896 & 105656 & 2002 & 101.70 & -0.29 & 10070.00 & 98273.34 & 1.01 & 0.97 & 0.98 \\
18496 & 102465 & 2002 & 220.20 & -0.34 & 32159.00 & 203176.35 & 0.68 & 0.92 & 0.63 \\
48678 & 240118 & 2002 & 43.00 & 0.07 & 4216.00 & 41585.79 & 1.02 & 0.97 & 0.99 \\
16153 & 102087 & 2002 & 632.10 & -0.23 & 65860.00 & 599047.21 & 0.96 & 0.95 & 0.91 \\
21564 & 102894 & 2002 & 483.70 & -0.39 & 48303.00 & 474685.61 & 1.00 & 0.98 & 0.98 \\
65370 & 500692 & 2002 & 53.00 & -0.38 & 5013.00 & 43517.68 & 1.06 & 0.82 & 0.87 \\
35426 & 106360 & 2002 & 170.80 & 0.06 & 9518.00 & 159484.51 & 1.79 & 0.93 & 1.68 \\
42366 & 108954 & 2002 & 40.00 & -0.23 & 3993.00 & 38975.60 & 1.00 & 0.97 & 0.98 \\
44156 & 109269 & 2002 & 9.00 & -0.03 & 910.00 & 9148.86 & 0.99 & 1.02 & 1.01 \\
24345 & 103315 & 2002 & 68.00 & -0.15 & 6804.00 & 67464.79 & 1.00 & 0.99 & 0.99 \\
65296 & 500682 & 2002 & 9.30 & -0.31 & 694.00 & 8325.59 & 1.34 & 0.90 & 1.20 \\
35385 & 106356 & 2002 & 16.60 & -0.14 & 1515.00 & 16653.28 & 1.10 & 1.00 & 1.10 \\
13675 & 101757 & 2002 & 345.50 & -0.27 & 34575.00 & 336479.68 & 1.00 & 0.97 & 0.97 \\
29930 & 105658 & 2002 & 84.20 & -0.31 & 12709.00 & 79402.77 & 0.66 & 0.94 & 0.62 \\
45367 & 200050 & 2002 & 87.40 & 0.04 & 8673.00 & 79612.76 & 1.01 & 0.91 & 0.92 \\
5944 & 100812 & 2002 & 211.20 & -0.23 & 21120.00 & 207849.04 & 1.00 & 0.98 & 0.98 \\
232 & 100019 & 2002 & 5725.10 & -0.14 & 571352.00 & 4769917.36 & 1.00 & 0.83 & 0.83 \\
57969 & 410010 & 2002 & 315.60 & -0.16 & 31944.00 & 309187.05 & 0.99 & 0.98 & 0.97 \\
21534 & 102893 & 2002 & 125.30 & 0.48 & 12543.00 & 124853.94 & 1.00 & 1.00 & 1.00 \\
24319 & 103308 & 2002 & 4589.10 & -0.22 & 460864.00 & 4044721.86 & 1.00 & 0.88 & 0.88 \\
37937 & 107171 & 2002 & 23.30 & -0.22 & 3186.00 & 28289.42 & 0.73 & 1.21 & 0.89 \\
48449 & 240085 & 2002 & 152.40 & -0.23 & 15324.00 & 153882.37 & 0.99 & 1.01 & 1.00 \\
47632 & 215952 & 2002 & 698.40 & -0.18 & 86173.00 & 582840.38 & 0.81 & 0.83 & 0.68 \\
4256 & 100598 & 2002 & 4984.50 & -0.06 & 495744.00 & 4688512.65 & 1.01 & 0.94 & 0.95 \\
35351 & 106353 & 2002 & 326.50 & -0.19 & 31072.00 & 310717.82 & 1.05 & 0.95 & 1.00 \\
49686 & 240332 & 2002 & 55.10 & -0.30 & 5777.00 & 50180.87 & 0.95 & 0.91 & 0.87 \\
39277 & 107636 & 2002 & 8.00 & -0.51 & 729.00 & 6234.29 & 1.10 & 0.78 & 0.86 \\
39281 & 107641 & 2002 & 243.10 & -0.23 & 24278.00 & 237535.59 & 1.00 & 0.98 & 0.98 \\
35307 & 106345 & 2002 & 398.00 & 0.65 & 39888.00 & 385183.87 & 1.00 & 0.97 & 0.97 \\
30048 & 105679 & 2002 & 227.80 & -0.34 & 25667.00 & 255969.12 & 0.89 & 1.12 & 1.00 \\
49682 & 240331 & 2002 & 3.60 & -0.28 & 359.00 & 3380.01 & 1.00 & 0.94 & 0.94 \\
30148 & 105703 & 2002 & 240.90 & 0.04 & 24087.00 & 231398.68 & 1.00 & 0.96 & 0.96 \\
43717 & 109208 & 2002 & 10.40 & -0.33 & 1012.00 & 10116.08 & 1.03 & 0.97 & 1.00 \\
44088 & 109265 & 2002 & 29.40 & -0.40 & 4061.00 & 24762.36 & 0.72 & 0.84 & 0.61 \\
6014 & 100818 & 2002 & 54.80 & -0.25 & 5470.00 & 53768.10 & 1.00 & 0.98 & 0.98 \\
45336 & 200039 & 2002 & 41.20 & -0.32 & 4066.00 & 40669.15 & 1.01 & 0.99 & 1.00 \\
21424 & 102871 & 2002 & 333.00 & -0.14 & 33287.00 & 310964.79 & 1.00 & 0.93 & 0.93 \\
30140 & 105702 & 2002 & 41.30 & -0.36 & 3865.00 & 33842.19 & 1.07 & 0.82 & 0.88 \\
65207 & 500670 & 2002 & 168.60 & 0.90 & 17900.00 & 170725.70 & 0.94 & 1.01 & 0.95 \\
23415 & 103175 & 2002 & 692.50 & -0.16 & 61071.00 & 618502.84 & 1.13 & 0.89 & 1.01 \\
44724 & 109368 & 2002 & 287.30 & -0.18 & 28748.00 & 254006.30 & 1.00 & 0.88 & 0.88 \\
42453 & 108968 & 2002 & 16.20 & -0.24 & 1616.00 & 15415.92 & 1.00 & 0.95 & 0.95 \\
30127 & 105701 & 2002 & 678.70 & -0.22 & 67138.00 & 655656.21 & 1.01 & 0.97 & 0.98 \\
17785 & 102357 & 2002 & 1103.50 & -0.20 & 110046.00 & 1100454.00 & 1.00 & 1.00 & 1.00 \\
20190 & 102676 & 2002 & 130.10 & -0.26 & 11977.00 & 126368.73 & 1.09 & 0.97 & 1.06 \\
59232 & 410446 & 2002 & 15.10 & -0.40 & 1614.00 & 14598.78 & 0.94 & 0.97 & 0.90 \\
24373 & 103318 & 2002 & 1428.20 & -0.13 & 149458.00 & 1418282.13 & 0.96 & 0.99 & 0.95 \\
3215 & 100415 & 2002 & 72.70 & -0.17 & 8636.00 & 82023.08 & 0.84 & 1.13 & 0.95 \\
13275 & 101716 & 2002 & 26.90 & -0.24 & 2606.00 & 25806.12 & 1.03 & 0.96 & 0.99 \\
10480 & 101287 & 2002 & 554.00 & -0.16 & 55867.00 & 483567.77 & 0.99 & 0.87 & 0.87 \\
21396 & 102861 & 2002 & 51.00 & -0.30 & 5122.00 & 50141.58 & 1.00 & 0.98 & 0.98 \\
179 & 100017 & 2002 & 74.80 & -0.23 & 6888.00 & 72630.75 & 1.09 & 0.97 & 1.05 \\
35187 & 106333 & 2002 & 83.60 & -0.13 & 10328.00 & 83360.85 & 0.81 & 1.00 & 0.81 \\
49388 & 240288 & 2002 & 4.30 & -0.26 & 485.00 & 3627.96 & 0.89 & 0.84 & 0.75 \\
40038 & 108013 & 2002 & 107.60 & -0.23 & 11114.00 & 115196.53 & 0.97 & 1.07 & 1.04 \\
40852 & 108159 & 2002 & 23.90 & -0.32 & 2397.00 & 23946.78 & 1.00 & 1.00 & 1.00 \\
34656 & 106267 & 2002 & 7.50 & -0.19 & 751.00 & 7256.02 & 1.00 & 0.97 & 0.97 \\
74563 & 601136 & 2002 & 28.90 & -0.18 & 2894.00 & 28500.43 & 1.00 & 0.99 & 0.98 \\
12084 & 101497 & 2002 & 1259.90 & -0.19 & 125391.00 & 1136452.35 & 1.00 & 0.90 & 0.91 \\
6024 & 100820 & 2002 & 326.50 & -0.22 & 32665.00 & 329408.02 & 1.00 & 1.01 & 1.01 \\
52217 & 302677 & 2002 & 5.60 & -0.18 & 627.00 & 4602.13 & 0.89 & 0.82 & 0.73 \\
35222 & 106335 & 2002 & 147.70 & -0.29 & 14312.00 & 144080.84 & 1.03 & 0.98 & 1.01 \\
8012 & 101069 & 2002 & 6788.00 & -0.22 & 720270.00 & 6595663.96 & 0.94 & 0.97 & 0.92 \\
60844 & 410732 & 2002 & 1049.50 & -0.08 & 138450.00 & 972061.12 & 0.76 & 0.93 & 0.70 \\
16079 & 102079 & 2002 & 366.80 & -0.22 & 32529.00 & 365890.07 & 1.13 & 1.00 & 1.12 \\
54298 & 367166 & 2002 & 99.50 & -0.18 & 9939.00 & 92977.82 & 1.00 & 0.93 & 0.94 \\
51289 & 240498 & 2002 & 55.00 & -0.21 & 5772.00 & 53888.58 & 0.95 & 0.98 & 0.93 \\
197 & 100018 & 2002 & 111.80 & -0.15 & 11190.00 & 105869.80 & 1.00 & 0.95 & 0.95 \\
42418 & 108964 & 2002 & 590.00 & -0.22 & 58748.00 & 571957.81 & 1.00 & 0.97 & 0.97 \\
30086 & 105682 & 2002 & 202.50 & -0.31 & 20280.00 & 198824.76 & 1.00 & 0.98 & 0.98 \\
44134 & 109268 & 2002 & 440.40 & 0.14 & 45883.00 & 403288.94 & 0.96 & 0.92 & 0.88 \\
47624 & 215696 & 2002 & 48.20 & -0.33 & 4817.00 & 44341.37 & 1.00 & 0.92 & 0.92 \\
18182 & 102414 & 2002 & 5883.50 & -0.26 & 612276.00 & 4857612.75 & 0.96 & 0.83 & 0.79 \\
26620 & 103593 & 2002 & 46057.10 & -0.15 & 4620873.00 & 44118672.94 & 1.00 & 0.96 & 0.95 \\
35280 & 106344 & 2002 & 756.80 & -0.21 & 69597.00 & 684195.50 & 1.09 & 0.90 & 0.98 \\
40074 & 108021 & 2002 & 678.00 & -0.28 & 67753.00 & 653975.67 & 1.00 & 0.96 & 0.97 \\
2262 & 100303 & 2002 & 263.60 & -0.18 & 24301.00 & 218972.73 & 1.08 & 0.83 & 0.90 \\
1256 & 100167 & 2002 & 215.30 & -0.29 & 36686.00 & 221453.07 & 0.59 & 1.03 & 0.60 \\
33507 & 106143 & 2002 & 95.80 & -0.31 & 9604.00 & 92327.77 & 1.00 & 0.96 & 0.96 \\
45341 & 200047 & 2002 & 22.10 & -0.21 & 2171.00 & 19343.86 & 1.02 & 0.88 & 0.89 \\
30093 & 105684 & 2002 & 19.00 & -0.22 & 1869.00 & 18691.38 & 1.02 & 0.98 & 1.00 \\
3786 & 100481 & 2002 & 89.40 & -0.14 & 9429.00 & 85368.88 & 0.95 & 0.95 & 0.91 \\
39313 & 107653 & 2002 & 24.80 & -0.27 & 2348.00 & 20916.16 & 1.06 & 0.84 & 0.89 \\
30117 & 105700 & 2002 & 185.30 & -0.19 & 18522.00 & 169104.35 & 1.00 & 0.91 & 0.91 \\
3920 & 100514 & 2002 & 44.10 & -0.24 & 4412.00 & 44058.55 & 1.00 & 1.00 & 1.00 \\
30109 & 105694 & 2002 & 41.50 & -0.04 & 4399.00 & 38081.25 & 0.94 & 0.92 & 0.87 \\
39307 & 107652 & 2002 & 72.90 & -0.36 & 7296.00 & 71307.12 & 1.00 & 0.98 & 0.98 \\
5985 & 100817 & 2002 & 171.10 & -0.23 & 17112.00 & 168886.42 & 1.00 & 0.99 & 0.99 \\
42431 & 108965 & 2002 & 22.00 & -0.43 & 2275.00 & 19605.75 & 0.97 & 0.89 & 0.86 \\
26989 & 103643 & 2002 & 66.60 & -0.25 & 6207.00 & 64593.28 & 1.07 & 0.97 & 1.04 \\
30102 & 105686 & 2002 & 34.60 & -0.29 & 3466.00 & 33570.92 & 1.00 & 0.97 & 0.97 \\
32184 & 105999 & 2002 & 30.00 & -0.06 & 2867.00 & 26734.16 & 1.05 & 0.89 & 0.93 \\
7529 & 101043 & 2002 & 4131.10 & -0.41 & 764787.00 & 4056022.36 & 0.54 & 0.98 & 0.53 \\
14775 & 101913 & 2002 & 23.40 & -0.23 & 2349.00 & 22908.16 & 1.00 & 0.98 & 0.98 \\
33084 & 106090 & 2002 & 86.60 & -0.10 & 8649.00 & 80385.94 & 1.00 & 0.93 & 0.93 \\
51311 & 240499 & 2002 & 2.20 & -0.01 & 175.00 & 1811.59 & 1.26 & 0.82 & 1.04 \\
1274 & 100171 & 2002 & 556.00 & -0.16 & 52504.00 & 547137.57 & 1.06 & 0.98 & 1.04 \\
21455 & 102872 & 2002 & 849.00 & -0.08 & 85408.00 & 778121.15 & 0.99 & 0.92 & 0.91 \\
40046 & 108018 & 2002 & 264.50 & -0.10 & 26440.00 & 253606.11 & 1.00 & 0.96 & 0.96 \\
53101 & 338393 & 2002 & 13.60 & -0.26 & 1367.00 & 12544.66 & 0.99 & 0.92 & 0.92 \\
29619 & 105627 & 2002 & 642.40 & -0.11 & 64979.00 & 627553.67 & 0.99 & 0.98 & 0.97 \\
37903 & 107160 & 2002 & 3330.70 & -0.15 & 294533.00 & 3115433.86 & 1.13 & 0.94 & 1.06 \\
1187 & 100159 & 2002 & 43.90 & -0.34 & 4396.00 & 43542.09 & 1.00 & 0.99 & 0.99 \\
29607 & 105623 & 2002 & 18.10 & -0.27 & 1946.00 & 18359.53 & 0.93 & 1.01 & 0.94 \\
7560 & 101045 & 2002 & 9400.60 & -0.15 & 1170995.00 & 8690500.94 & 0.80 & 0.92 & 0.74 \\
4941 & 100695 & 2002 & 139.50 & -0.18 & 14089.00 & 125522.80 & 0.99 & 0.90 & 0.89 \\
54079 & 364391 & 2002 & 10.20 & -0.15 & 1036.00 & 9973.04 & 0.98 & 0.98 & 0.96 \\
47864 & 222408 & 2002 & 609.00 & -0.19 & 74141.00 & 575893.33 & 0.82 & 0.95 & 0.78 \\
35641 & 106381 & 2002 & 3.70 & -0.10 & 369.00 & 3663.09 & 1.00 & 0.99 & 0.99 \\
10734 & 101320 & 2002 & 61.90 & -0.05 & 6189.00 & 53196.03 & 1.00 & 0.86 & 0.86 \\
53810 & 357133 & 2002 & 30.00 & -0.28 & 3105.00 & 31048.87 & 0.97 & 1.03 & 1.00 \\
29578 & 105616 & 2002 & 19.30 & -0.25 & 1680.00 & 16783.12 & 1.15 & 0.87 & 1.00 \\
15148 & 101963 & 2002 & 621.40 & -0.23 & 62468.00 & 575043.09 & 0.99 & 0.93 & 0.92 \\
44210 & 109275 & 2002 & 56.30 & -0.24 & 5463.00 & 55721.06 & 1.03 & 0.99 & 1.02 \\
35614 & 106380 & 2002 & 171.50 & -0.15 & 17254.00 & 170978.63 & 0.99 & 1.00 & 0.99 \\
44201 & 109274 & 2002 & 26.10 & -0.23 & 2045.00 & 23940.03 & 1.28 & 0.92 & 1.17 \\
3108 & 100409 & 2002 & 239.60 & -0.13 & 24068.00 & 227914.19 & 1.00 & 0.95 & 0.95 \\
24267 & 103301 & 2002 & 766.80 & -0.33 & 105603.00 & 787340.71 & 0.73 & 1.03 & 0.75 \\
52363 & 302813 & 2002 & 16.20 & -0.26 & 1434.00 & 15194.68 & 1.13 & 0.94 & 1.06 \\
5905 & 100811 & 2002 & 777.10 & -0.25 & 77936.00 & 772477.23 & 1.00 & 0.99 & 0.99 \\
20156 & 102671 & 2002 & 196.90 & -0.24 & 19705.00 & 196292.12 & 1.00 & 1.00 & 1.00 \\
16244 & 102104 & 2002 & 410.50 & -0.20 & 40505.00 & 395923.15 & 1.01 & 0.96 & 0.98 \\
47645 & 216438 & 2002 & 496.80 & -0.24 & 49670.00 & 488670.28 & 1.00 & 0.98 & 0.98 \\
10705 & 101312 & 2002 & 6429.90 & -0.30 & 643545.00 & 5759731.32 & 1.00 & 0.90 & 0.90 \\
21679 & 102939 & 2002 & 4779.80 & -0.21 & 477458.00 & 4704810.31 & 1.00 & 0.98 & 0.99 \\
54057 & 364292 & 2002 & 295.10 & -0.26 & 27204.00 & 287799.40 & 1.08 & 0.98 & 1.06 \\
55395 & 400093 & 2002 & 41.30 & -0.42 & 4121.00 & 37588.74 & 1.00 & 0.91 & 0.91 \\
35604 & 106379 & 2002 & 146.80 & -0.27 & 14767.00 & 143678.93 & 0.99 & 0.98 & 0.97 \\
14955 & 101925 & 2002 & 3485.60 & -0.10 & 267651.00 & 2858065.96 & 1.30 & 0.82 & 1.07 \\
4742 & 100670 & 2002 & 64.80 & -0.16 & 7068.00 & 64761.04 & 0.92 & 1.00 & 0.92 \\
21709 & 102940 & 2002 & 1340.30 & -0.22 & 133939.00 & 1320567.37 & 1.00 & 0.99 & 0.99 \\
39197 & 107618 & 2002 & 2060.40 & -0.24 & 188024.00 & 2007769.51 & 1.10 & 0.97 & 1.07 \\
16259 & 102105 & 2002 & 104.00 & -0.24 & 10431.00 & 103336.15 & 1.00 & 0.99 & 0.99 \\
29521 & 105604 & 2002 & 13.50 & -0.24 & 1341.00 & 12944.11 & 1.01 & 0.96 & 0.97 \\
53790 & 357122 & 2002 & 450.90 & -0.15 & 23375.00 & 425304.05 & 1.93 & 0.94 & 1.82 \\
3086 & 100408 & 2002 & 197.30 & -0.20 & 19648.00 & 178293.58 & 1.00 & 0.90 & 0.91 \\
5873 & 100809 & 2002 & 1317.40 & -0.22 & 131701.00 & 1271592.20 & 1.00 & 0.97 & 0.97 \\
48437 & 240083 & 2002 & 235.50 & -0.27 & 21262.00 & 214433.77 & 1.11 & 0.91 & 1.01 \\
13842 & 101769 & 2002 & 936.20 & -0.30 & 151849.00 & 954591.09 & 0.62 & 1.02 & 0.63 \\
9614 & 101158 & 2002 & 334.20 & -0.21 & 33388.00 & 304600.30 & 1.00 & 0.91 & 0.91 \\
48717 & 240130 & 2002 & 482.00 & -0.45 & 43401.00 & 459424.13 & 1.11 & 0.95 & 1.06 \\
29515 & 105603 & 2002 & 1.20 & -0.09 & 114.00 & 1114.75 & 1.05 & 0.93 & 0.98 \\
38274 & 107234 & 2002 & 9.60 & -0.16 & 1182.00 & 9897.93 & 0.81 & 1.03 & 0.84 \\
6411 & 100864 & 2002 & 350.80 & -0.13 & 33229.00 & 280520.93 & 1.06 & 0.80 & 0.84 \\
35705 & 106391 & 2002 & 50.40 & -0.27 & 4601.00 & 49626.33 & 1.10 & 0.98 & 1.08 \\
17108 & 102257 & 2002 & 982.30 & -0.23 & 98143.00 & 950774.83 & 1.00 & 0.97 & 0.97 \\
47662 & 216749 & 2002 & 55.20 & -0.23 & 5300.00 & 45707.72 & 1.04 & 0.83 & 0.86 \\
10770 & 101330 & 2002 & 2350.60 & -0.03 & 234941.00 & 2306709.34 & 1.00 & 0.98 & 0.98 \\
53235 & 342127 & 2002 & 75.20 & -0.24 & 7526.00 & 73144.12 & 1.00 & 0.97 & 0.97 \\
15303 & 101982 & 2002 & 242.50 & -0.27 & 24250.00 & 241267.89 & 1.00 & 0.99 & 0.99 \\
8320 & 101082 & 2002 & 1824.90 & -0.18 & 237783.00 & 1824023.75 & 0.77 & 1.00 & 0.77 \\
29530 & 105607 & 2002 & 3.50 & -0.32 & 354.00 & 3401.22 & 0.99 & 0.97 & 0.96 \\
276 & 100030 & 2002 & 181.40 & -0.33 & 18167.00 & 183131.23 & 1.00 & 1.01 & 1.01 \\
44734 & 109370 & 2002 & 630.70 & -0.15 & 65520.00 & 596133.32 & 0.96 & 0.95 & 0.91 \\
19081 & 102548 & 2002 & 543.90 & -0.20 & 90603.00 & 741986.09 & 0.60 & 1.36 & 0.82 \\
32756 & 106061 & 2002 & 280.90 & -0.25 & 28603.00 & 274338.21 & 0.98 & 0.98 & 0.96 \\
12480 & 101542 & 2002 & 48.00 & -0.24 & 4694.00 & 45003.41 & 1.02 & 0.94 & 0.96 \\
29549 & 105611 & 2002 & 947.70 & -0.13 & 88469.00 & 809936.57 & 1.07 & 0.85 & 0.92 \\
47654 & 216504 & 2002 & 14.50 & -0.47 & 1859.00 & 14492.98 & 0.78 & 1.00 & 0.78 \\
42211 & 108943 & 2002 & 969.50 & -0.16 & 75100.00 & 800288.23 & 1.29 & 0.83 & 1.07 \\
48712 & 240123 & 2002 & 1.40 & -0.09 & 143.00 & 1305.99 & 0.98 & 0.93 & 0.91 \\
29537 & 105610 & 2002 & 237.50 & -0.21 & 23601.00 & 222529.26 & 1.01 & 0.94 & 0.94 \\
711 & 100092 & 2002 & 406.20 & -0.31 & 40626.00 & 418086.62 & 1.00 & 1.03 & 1.03 \\
42205 & 108939 & 2002 & 44.40 & -0.34 & 4486.00 & 45611.10 & 0.99 & 1.03 & 1.02 \\
32240 & 106007 & 2002 & 3382.40 & -0.23 & 339101.00 & 2947113.59 & 1.00 & 0.87 & 0.87 \\
59214 & 410445 & 2002 & 24.50 & -0.09 & 2422.00 & 21240.16 & 1.01 & 0.87 & 0.88 \\
40164 & 108070 & 2002 & 53.80 & -0.15 & 4891.00 & 52327.47 & 1.10 & 0.97 & 1.07 \\
54320 & 367168 & 2002 & 1.50 & -0.12 & 148.00 & 1409.11 & 1.01 & 0.94 & 0.95 \\
25631 & 103498 & 2002 & 242.40 & -0.15 & 24045.00 & 218848.81 & 1.01 & 0.90 & 0.91 \\
33448 & 106136 & 2002 & 79.30 & -0.29 & 7257.00 & 75874.90 & 1.09 & 0.96 & 1.05 \\
44658 & 109357 & 2002 & 240.70 & -0.16 & 24119.00 & 223745.87 & 1.00 & 0.93 & 0.93 \\
46546 & 200252 & 2002 & 59.00 & -0.23 & 5821.00 & 56130.07 & 1.01 & 0.95 & 0.96 \\
35483 & 106364 & 2002 & 13.90 & -0.11 & 1400.00 & 13060.02 & 0.99 & 0.94 & 0.93 \\
44880 & 109397 & 2002 & 120.60 & -0.15 & 9842.00 & 85583.26 & 1.23 & 0.71 & 0.87 \\
48687 & 240121 & 2002 & 525.40 & -0.32 & 55444.00 & 496050.71 & 0.95 & 0.94 & 0.89 \\
40130 & 108037 & 2002 & 91.90 & -0.45 & 9180.00 & 90452.25 & 1.00 & 0.98 & 0.99 \\
29795 & 105647 & 2002 & 380.60 & -0.14 & 38074.00 & 366587.46 & 1.00 & 0.96 & 0.96 \\
61261 & 500025 & 2002 & 2.00 & -0.15 & 149.00 & 1329.63 & 1.34 & 0.66 & 0.89 \\
49584 & 240319 & 2002 & 56.30 & -0.55 & 7674.00 & 57300.57 & 0.73 & 1.02 & 0.75 \\
21598 & 102895 & 2002 & 683.90 & -0.27 & 68538.00 & 660808.43 & 1.00 & 0.97 & 0.96 \\
19801 & 102652 & 2002 & 1675.20 & -0.24 & 165646.00 & 1603060.31 & 1.01 & 0.96 & 0.97 \\
46523 & 200251 & 2002 & 8.40 & -0.38 & 732.00 & 6847.46 & 1.15 & 0.82 & 0.94 \\
35496 & 106366 & 2002 & 12.20 & -0.29 & 919.00 & 9828.15 & 1.33 & 0.81 & 1.07 \\
21612 & 102897 & 2002 & 133.20 & -0.18 & 13356.00 & 129475.24 & 1.00 & 0.97 & 0.97 \\
47254 & 200344 & 2002 & 6239.20 & -0.20 & 622153.00 & 5902050.55 & 1.00 & 0.95 & 0.95 \\
35501 & 106367 & 2002 & 90.60 & -0.22 & 9135.00 & 90342.94 & 0.99 & 1.00 & 0.99 \\
54046 & 364291 & 2002 & 88.50 & -0.17 & 7792.00 & 82356.50 & 1.14 & 0.93 & 1.06 \\
35490 & 106365 & 2002 & 15.20 & -0.15 & 1521.00 & 14908.77 & 1.00 & 0.98 & 0.98 \\
39256 & 107627 & 2002 & 564.90 & -0.15 & 50091.00 & 508233.61 & 1.13 & 0.90 & 1.01 \\
25556 & 103496 & 2002 & 334.50 & -0.27 & 33486.00 & 328658.89 & 1.00 & 0.98 & 0.98 \\
42360 & 108953 & 2002 & 136.80 & -0.36 & 13505.00 & 134054.09 & 1.01 & 0.98 & 0.99 \\
10643 & 101302 & 2002 & 793.10 & -0.02 & 79314.00 & 779097.89 & 1.00 & 0.98 & 0.98 \\
9389 & 101133 & 2002 & 839.00 & -0.16 & 109041.00 & 824248.40 & 0.77 & 0.98 & 0.76 \\
32212 & 106000 & 2002 & 493.50 & -0.27 & 51421.00 & 496917.37 & 0.96 & 1.01 & 0.97 \\
35453 & 106361 & 2002 & 129.20 & -0.26 & 12940.00 & 117860.19 & 1.00 & 0.91 & 0.91 \\
42335 & 108952 & 2002 & 236.50 & -0.32 & 23141.00 & 230664.30 & 1.02 & 0.98 & 1.00 \\
44179 & 109270 & 2002 & 57.50 & -0.25 & 7722.00 & 59312.35 & 0.74 & 1.03 & 0.77 \\
40843 & 108158 & 2002 & 23.30 & -0.34 & 2325.00 & 23141.38 & 1.00 & 0.99 & 1.00 \\
41003 & 108175 & 2002 & 67.70 & -0.21 & 6752.00 & 67511.28 & 1.00 & 1.00 & 1.00 \\
29824 & 105652 & 2002 & 359.60 & -0.29 & 34519.00 & 334948.49 & 1.04 & 0.93 & 0.97 \\
16172 & 102089 & 2002 & 233.20 & -0.12 & 21009.00 & 172454.95 & 1.11 & 0.74 & 0.82 \\
17754 & 102356 & 2002 & 3.20 & -0.18 & 289.00 & 2890.82 & 1.11 & 0.90 & 1.00 \\
37932 & 107167 & 2002 & 9.30 & -0.30 & 925.00 & 8944.13 & 1.01 & 0.96 & 0.97 \\
3143 & 100411 & 2002 & 2997.40 & -0.22 & 299159.00 & 2903136.40 & 1.00 & 0.97 & 0.97 \\
35475 & 106363 & 2002 & 203.30 & -0.30 & 19098.00 & 178048.40 & 1.06 & 0.88 & 0.93 \\
40125 & 108030 & 2002 & 25.50 & 0.05 & 3314.00 & 24371.05 & 0.77 & 0.96 & 0.74 \\
53854 & 359285 & 2002 & 138.10 & -0.14 & 13846.00 & 112709.47 & 1.00 & 0.82 & 0.81 \\
40982 & 108170 & 2002 & 244.80 & -0.43 & 24646.00 & 238325.82 & 0.99 & 0.97 & 0.97 \\
29766 & 105645 & 2002 & 3695.80 & -0.28 & 504116.00 & 3454233.77 & 0.73 & 0.93 & 0.69 \\
74895 & 601189 & 2002 & 6.80 & -0.46 & 654.00 & 6541.87 & 1.04 & 0.96 & 1.00 \\
25587 & 103497 & 2002 & 109.70 & -0.14 & 10979.00 & 88283.97 & 1.00 & 0.80 & 0.80 \\
23385 & 103174 & 2002 & 1220.50 & -0.22 & 122148.00 & 1213363.24 & 1.00 & 0.99 & 0.99 \\
42319 & 108951 & 2002 & 125.70 & -0.14 & 12132.00 & 121323.83 & 1.04 & 0.97 & 1.00 \\
35548 & 106372 & 2002 & 6.00 & -0.06 & 553.00 & 6123.92 & 1.08 & 1.02 & 1.11 \\
38268 & 107227 & 2002 & 58.40 & -0.34 & 5948.00 & 58193.21 & 0.98 & 1.00 & 0.98 \\
40154 & 108051 & 2002 & 326.70 & -0.14 & 33629.00 & 314354.88 & 0.97 & 0.96 & 0.93 \\
29705 & 105640 & 2002 & 113.70 & 0.01 & 7778.00 & 69152.21 & 1.46 & 0.61 & 0.89 \\
35521 & 106370 & 2002 & 36.90 & -0.40 & 3713.00 & 33871.41 & 0.99 & 0.92 & 0.91 \\
39239 & 107626 & 2002 & 515.20 & -0.23 & 43513.00 & 462025.34 & 1.18 & 0.90 & 1.06 \\
46508 & 200249 & 2002 & 58.40 & -0.39 & 5677.00 & 56647.81 & 1.03 & 0.97 & 1.00 \\
35556 & 106375 & 2002 & 19.80 & -0.18 & 1978.00 & 18200.43 & 1.00 & 0.92 & 0.92 \\
37928 & 107162 & 2002 & 5.70 & -0.22 & 576.00 & 5623.78 & 0.99 & 0.99 & 0.98 \\
29684 & 105635 & 2002 & 120.80 & -0.33 & 11987.00 & 117814.22 & 1.01 & 0.98 & 0.98 \\
17733 & 102350 & 2002 & 375.30 & -0.24 & 35709.00 & 353887.80 & 1.05 & 0.94 & 0.99 \\
35581 & 106376 & 2002 & 3.70 & -0.44 & 360.00 & 3583.70 & 1.03 & 0.97 & 1.00 \\
45382 & 200055 & 2002 & 74.90 & -0.16 & 7508.00 & 71786.67 & 1.00 & 0.96 & 0.96 \\
16203 & 102090 & 2002 & 4606.30 & -0.29 & 446123.00 & 4504182.07 & 1.03 & 0.98 & 1.01 \\
27015 & 103644 & 2002 & 32.80 & -0.17 & 4567.00 & 38033.53 & 0.72 & 1.16 & 0.83 \\
45377 & 200051 & 2002 & 1.70 & -0.23 & 136.00 & 1551.42 & 1.25 & 0.91 & 1.14 \\
19109 & 102549 & 2002 & 233.60 & -0.16 & 24732.00 & 202712.42 & 0.94 & 0.87 & 0.82 \\
21624 & 102901 & 2002 & 65.10 & -0.25 & 6512.00 & 64424.84 & 1.00 & 0.99 & 0.99 \\
29756 & 105644 & 2002 & 62.90 & -0.25 & 6210.00 & 56551.27 & 1.01 & 0.90 & 0.91 \\
46517 & 200250 & 2002 & 139.90 & -0.22 & 13939.00 & 139388.40 & 1.00 & 1.00 & 1.00 \\
10673 & 101307 & 2002 & 567.30 & 0.01 & 56732.00 & 542851.96 & 1.00 & 0.96 & 0.96 \\
24285 & 103304 & 2002 & 36.70 & -0.20 & 3609.00 & 31331.09 & 1.02 & 0.85 & 0.87 \\
19736 & 102650 & 2002 & 9837.90 & -0.13 & 987299.00 & 9496002.09 & 1.00 & 0.97 & 0.96 \\
13224 & 101708 & 2002 & 219.40 & -0.20 & 21693.00 & 206932.93 & 1.01 & 0.94 & 0.95 \\
17227 & 102271 & 2002 & 657.50 & -0.29 & 65785.00 & 653674.32 & 1.00 & 0.99 & 0.99 \\
53842 & 357762 & 2002 & 25.10 & -0.27 & 3539.00 & 20789.98 & 0.71 & 0.83 & 0.59 \\
32591 & 106042 & 2002 & 74.00 & -0.02 & 6177.00 & 51288.97 & 1.20 & 0.69 & 0.83 \\
55060 & 400050 & 2002 & 1094.10 & -0.10 & 143759.00 & 1028328.33 & 0.76 & 0.94 & 0.72 \\
29734 & 105643 & 2002 & 877.20 & -0.25 & 87798.00 & 826458.04 & 1.00 & 0.94 & 0.94 \\
6761 & 100950 & 2002 & 664.30 & -0.14 & 66694.00 & 648788.77 & 1.00 & 0.98 & 0.97 \\
21639 & 102937 & 2002 & 41.40 & -0.19 & 4808.00 & 39031.69 & 0.86 & 0.94 & 0.81 \\
35513 & 106369 & 2002 & 98.20 & -0.35 & 9643.00 & 95447.64 & 1.02 & 0.97 & 0.99 \\
60824 & 410731 & 2002 & 938.50 & -0.04 & 117122.00 & 901886.02 & 0.80 & 0.96 & 0.77 \\
4764 & 100671 & 2002 & 216.10 & -0.13 & 26048.00 & 176428.58 & 0.83 & 0.82 & 0.68 \\
46663 & 200274 & 2002 & 13.90 & -0.20 & 1387.00 & 13550.67 & 1.00 & 0.97 & 0.98 \\
64213 & 500592 & 2002 & 701.90 & -0.21 & 60021.00 & 588606.10 & 1.17 & 0.84 & 0.98 \\
34902 & 106292 & 2002 & 44.70 & 0.15 & 3807.00 & 38333.11 & 1.17 & 0.86 & 1.01 \\
39369 & 107673 & 2002 & 66.90 & -0.15 & 6789.00 & 61080.90 & 0.99 & 0.91 & 0.90 \\
30565 & 105770 & 2002 & 24.70 & -0.15 & 2452.00 & 21188.75 & 1.01 & 0.86 & 0.86 \\
19256 & 102575 & 2002 & 160.50 & -0.31 & 14977.00 & 149768.43 & 1.07 & 0.93 & 1.00 \\
39376 & 107677 & 2002 & 204.00 & -0.15 & 20359.00 & 197178.71 & 1.00 & 0.97 & 0.97 \\
96665 & 611002 & 2002 & 2845.70 & -0.27 & 484914.00 & 2875197.28 & 0.59 & 1.01 & 0.59 \\
59254 & 410448 & 2002 & 45.00 & -0.10 & 4152.00 & 41519.08 & 1.08 & 0.92 & 1.00 \\
39992 & 107968 & 2002 & 72.90 & -0.19 & 7429.00 & 70296.71 & 0.98 & 0.96 & 0.95 \\
21197 & 102837 & 2002 & 794.80 & -0.19 & 81766.00 & 726779.34 & 0.97 & 0.91 & 0.89 \\
3756 & 100480 & 2002 & 84.10 & -0.22 & 8554.00 & 81122.16 & 0.98 & 0.96 & 0.95 \\
37959 & 107174 & 2002 & 26.70 & -0.15 & 3663.00 & 30422.28 & 0.73 & 1.14 & 0.83 \\
9664 & 101161 & 2002 & 1012.80 & -0.21 & 100105.00 & 1001054.50 & 1.01 & 0.99 & 1.00 \\
58138 & 410100 & 2002 & 24.70 & -0.23 & 2491.00 & 23635.48 & 0.99 & 0.96 & 0.95 \\
34887 & 106284 & 2002 & 228.50 & -0.23 & 25923.00 & 213241.17 & 0.88 & 0.93 & 0.82 \\
46707 & 200286 & 2002 & 25.70 & -0.48 & 2584.00 & 25887.81 & 0.99 & 1.01 & 1.00 \\
42639 & 108990 & 2002 & 24.20 & -0.15 & 2322.00 & 22846.84 & 1.04 & 0.94 & 0.98 \\
43478 & 109126 & 2002 & 72.80 & -0.29 & 7261.00 & 69349.85 & 1.00 & 0.95 & 0.96 \\
34867 & 106283 & 2002 & 170.60 & -0.17 & 16995.00 & 156986.47 & 1.00 & 0.92 & 0.92 \\
42655 & 108992 & 2002 & 11.00 & -0.34 & 1875.00 & 18754.94 & 0.59 & 1.70 & 1.00 \\
33564 & 106149 & 2002 & 206.00 & -0.31 & 20609.00 & 189564.47 & 1.00 & 0.92 & 0.92 \\
44029 & 109259 & 2002 & 52.20 & -0.20 & 5190.00 & 45782.86 & 1.01 & 0.88 & 0.88 \\
39982 & 107967 & 2002 & 138.00 & -0.30 & 13798.00 & 135639.30 & 1.00 & 0.98 & 0.98 \\
30594 & 105775 & 2002 & 1354.80 & -0.17 & 135646.00 & 1345939.93 & 1.00 & 0.99 & 0.99 \\
38233 & 107224 & 2002 & 88.30 & -0.17 & 8807.00 & 87978.23 & 1.00 & 1.00 & 1.00 \\
46670 & 200276 & 2002 & 10.20 & -0.20 & 1018.00 & 8544.82 & 1.00 & 0.84 & 0.84 \\
42647 & 108991 & 2002 & 302.90 & -0.28 & 29409.00 & 287543.75 & 1.03 & 0.95 & 0.98 \\
42702 & 108996 & 2002 & 43.30 & -0.22 & 4351.00 & 42909.36 & 1.00 & 0.99 & 0.99 \\
6381 & 100856 & 2002 & 87.10 & -0.27 & 8748.00 & 86340.41 & 1.00 & 0.99 & 0.99 \\
25314 & 103466 & 2002 & 811.40 & -0.21 & 104714.00 & 792476.43 & 0.77 & 0.98 & 0.76 \\
10324 & 101279 & 2002 & 91.80 & -0.11 & 7036.00 & 64470.19 & 1.30 & 0.70 & 0.92 \\
53918 & 360123 & 2002 & 14.30 & -0.32 & 1608.00 & 14325.98 & 0.89 & 1.00 & 0.89 \\
37964 & 107175 & 2002 & 1096.10 & -0.24 & 146885.00 & 985182.54 & 0.75 & 0.90 & 0.67 \\
23449 & 103177 & 2002 & 143.30 & -0.29 & 14142.00 & 137730.36 & 1.01 & 0.96 & 0.97 \\
6750 & 100947 & 2002 & 459.30 & -0.23 & 45975.00 & 449961.60 & 1.00 & 0.98 & 0.98 \\
12392 & 101538 & 2002 & 77.50 & -0.21 & 7761.00 & 72639.10 & 1.00 & 0.94 & 0.94 \\
42633 & 108988 & 2002 & 242.50 & -0.36 & 42568.00 & 242042.96 & 0.57 & 1.00 & 0.57 \\
7835 & 101062 & 2002 & 1719.80 & 0.06 & 174995.00 & 1611627.21 & 0.98 & 0.94 & 0.92 \\
61980 & 500315 & 2002 & 39.80 & -0.15 & 3956.00 & 36122.70 & 1.01 & 0.91 & 0.91 \\
46650 & 200271 & 2002 & 3.00 & -0.39 & 241.00 & 2824.14 & 1.24 & 0.94 & 1.17 \\
43232 & 109086 & 2002 & 407.50 & -0.15 & 40941.00 & 399927.87 & 1.00 & 0.98 & 0.98 \\
46648 & 200270 & 2002 & 2.40 & -0.22 & 223.00 & 2381.24 & 1.08 & 0.99 & 1.07 \\
34973 & 106298 & 2002 & 14.60 & -0.37 & 1271.00 & 13987.29 & 1.15 & 0.96 & 1.10 \\
9352 & 101132 & 2002 & 268.70 & -0.06 & 33351.00 & 256433.90 & 0.81 & 0.95 & 0.77 \\
30499 & 105762 & 2002 & 253.00 & -0.20 & 32414.00 & 252617.96 & 0.78 & 1.00 & 0.78 \\
44037 & 109260 & 2002 & 27.30 & -0.29 & 2749.00 & 27249.11 & 0.99 & 1.00 & 0.99 \\
38243 & 107226 & 2002 & 155.40 & -0.12 & 15369.00 & 151204.03 & 1.01 & 0.97 & 0.98 \\
42710 & 109008 & 2002 & 24.20 & -0.13 & 4050.00 & 23969.58 & 0.60 & 0.99 & 0.59 \\
18132 & 102404 & 2002 & 2327.60 & -0.23 & 233367.00 & 2290843.98 & 1.00 & 0.98 & 0.98 \\
43601 & 109153 & 2002 & 283.30 & -0.20 & 26873.00 & 255653.66 & 1.05 & 0.90 & 0.95 \\
55356 & 400090 & 2002 & 4.60 & -0.31 & 473.00 & 4302.58 & 0.97 & 0.94 & 0.91 \\
42709 & 109006 & 2002 & 11.20 & -0.54 & 1115.00 & 10479.67 & 1.00 & 0.94 & 0.94 \\
46651 & 200273 & 2002 & 17.50 & -0.15 & 1752.00 & 14381.71 & 1.00 & 0.82 & 0.82 \\
32133 & 105984 & 2002 & 50.10 & -0.24 & 4954.00 & 48117.28 & 1.01 & 0.96 & 0.97 \\
15946 & 102061 & 2002 & 1184.90 & -0.13 & 102661.00 & 1063102.42 & 1.15 & 0.90 & 1.04 \\
34935 & 106294 & 2002 & 106.20 & 0.04 & 9450.00 & 85620.52 & 1.12 & 0.81 & 0.91 \\
42706 & 108997 & 2002 & 47.40 & -0.43 & 4760.00 & 42945.34 & 1.00 & 0.91 & 0.90 \\
24453 & 103327 & 2002 & 1058.10 & -0.21 & 141963.00 & 952867.30 & 0.75 & 0.90 & 0.67 \\
19304 & 102588 & 2002 & 294.90 & -0.14 & 29405.00 & 243842.38 & 1.00 & 0.83 & 0.83 \\
30754 & 105793 & 2002 & 2024.00 & -0.28 & 213394.00 & 1857430.64 & 0.95 & 0.92 & 0.87 \\
42624 & 108987 & 2002 & 63.30 & -0.26 & 10316.00 & 65696.24 & 0.61 & 1.04 & 0.64 \\
12127 & 101511 & 2002 & 190.50 & -0.35 & 17727.00 & 176517.44 & 1.07 & 0.93 & 1.00 \\
30527 & 105763 & 2002 & 233.60 & -0.16 & 30878.00 & 224159.89 & 0.76 & 0.96 & 0.73 \\
7496 & 101042 & 2002 & 13479.20 & -0.20 & 1566260.00 & 11386020.47 & 0.86 & 0.84 & 0.73 \\
44953 & 109404 & 2002 & 10.80 & -0.24 & 1034.00 & 11169.79 & 1.04 & 1.03 & 1.08 \\
42599 & 108985 & 2002 & 1447.70 & -0.15 & 125063.00 & 1136492.42 & 1.16 & 0.79 & 0.91 \\
61976 & 500313 & 2002 & 33.70 & -0.22 & 3613.00 & 26974.61 & 0.93 & 0.80 & 0.75 \\
34928 & 106293 & 2002 & 18.00 & -0.27 & 2927.00 & 29803.92 & 0.61 & 1.66 & 1.02 \\
32106 & 105983 & 2002 & 179.90 & -0.19 & 17993.00 & 175794.89 & 1.00 & 0.98 & 0.98 \\
51677 & 240539 & 2002 & 16.90 & -0.13 & NaN & 16695.53 & 1.00 & 0.99 & 1.00 \\
21111 & 102832 & 2002 & 88.70 & -0.37 & 8883.00 & 88755.26 & 1.00 & 1.00 & 1.00 \\
1421 & 100196 & 2002 & 2899.40 & -0.15 & 298621.00 & 2741006.50 & 0.97 & 0.95 & 0.92 \\
52808 & 330079 & 2002 & 28.30 & -0.18 & 2815.00 & 25700.90 & 1.01 & 0.91 & 0.91 \\
43599 & 109151 & 2002 & 76.60 & -0.30 & 7933.00 & 75822.39 & 0.97 & 0.99 & 0.96 \\
7085 & 100996 & 2002 & 1375.40 & -0.23 & 136256.00 & 1358432.42 & 1.01 & 0.99 & 1.00 \\
39974 & 107964 & 2002 & 278.10 & -0.27 & 27563.00 & 248328.90 & 1.01 & 0.89 & 0.90 \\
30685 & 105783 & 2002 & 2100.90 & 0.04 & 144398.00 & 1844264.70 & 1.45 & 0.88 & 1.28 \\
19284 & 102579 & 2002 & 222.50 & -0.48 & 22273.00 & 220166.08 & 1.00 & 0.99 & 0.99 \\
74440 & 601001 & 2002 & 55.60 & -0.15 & 5562.00 & 53132.78 & 1.00 & 0.96 & 0.96 \\
46694 & 200279 & 2002 & 62.80 & -0.32 & 5904.00 & 63003.13 & 1.06 & 1.00 & 1.07 \\
42694 & 108994 & 2002 & 185.80 & -0.26 & 17671.00 & 178579.69 & 1.05 & 0.96 & 1.01 \\
10295 & 101278 & 2002 & 80.10 & -0.12 & 7426.00 & 75118.49 & 1.08 & 0.94 & 1.01 \\
25283 & 103464 & 2002 & 679.80 & -0.32 & 74448.00 & 635942.83 & 0.91 & 0.94 & 0.85 \\
145 & 100010 & 2002 & 389.90 & -0.19 & 39065.00 & 378489.07 & 1.00 & 0.97 & 0.97 \\
18474 & 102462 & 2002 & 5.30 & -0.33 & 566.00 & 5653.41 & 0.94 & 1.07 & 1.00 \\
4178 & 100567 & 2002 & 796.20 & -0.26 & 79621.00 & 768424.28 & 1.00 & 0.97 & 0.97 \\
44964 & 109406 & 2002 & 46.80 & -0.30 & 5539.00 & 44673.14 & 0.84 & 0.95 & 0.81 \\
6086 & 100822 & 2002 & 7.40 & -0.14 & 736.00 & 7358.64 & 1.01 & 0.99 & 1.00 \\
34711 & 106272 & 2002 & 1588.10 & -0.23 & 155601.00 & 1635766.60 & 1.02 & 1.03 & 1.05 \\
13355 & 101729 & 2002 & 116.80 & -0.29 & 8807.00 & 88065.96 & 1.33 & 0.75 & 1.00 \\
45297 & 200015 & 2002 & 19.50 & -0.17 & 2162.00 & 17584.27 & 0.90 & 0.90 & 0.81 \\
49340 & 240284 & 2002 & 55.80 & -0.16 & 6167.00 & 51800.05 & 0.90 & 0.93 & 0.84 \\
33075 & 106089 & 2002 & 37.50 & -0.33 & 3772.00 & 33620.42 & 0.99 & 0.90 & 0.89 \\
30712 & 105788 & 2002 & 18.30 & -0.21 & 1834.00 & 17499.97 & 1.00 & 0.96 & 0.95 \\
52180 & 302627 & 2002 & 67.20 & -0.15 & 6747.00 & 66852.26 & 1.00 & 0.99 & 0.99 \\
61936 & 500310 & 2002 & 4.10 & -0.21 & 478.00 & 4406.83 & 0.86 & 1.07 & 0.92 \\
65064 & 500659 & 2002 & 11.60 & 0.18 & 951.00 & 9594.11 & 1.22 & 0.83 & 1.01 \\
61318 & 500028 & 2002 & 38.80 & 0.17 & 2382.00 & 23490.25 & 1.63 & 0.61 & 0.99 \\
54816 & 400015 & 2002 & 2.30 & -0.47 & 214.00 & 2069.91 & 1.07 & 0.90 & 0.97 \\
42679 & 108993 & 2002 & 40.70 & -0.10 & 4068.00 & 38262.53 & 1.00 & 0.94 & 0.94 \\
52854 & 330794 & 2002 & 40.60 & -0.20 & 4083.00 & 35169.64 & 0.99 & 0.87 & 0.86 \\
9185 & 101116 & 2002 & 819.70 & -0.27 & 80269.00 & 809099.07 & 1.02 & 0.99 & 1.01 \\
48258 & 240057 & 2002 & 100.60 & -0.35 & 10072.00 & 86917.07 & 1.00 & 0.86 & 0.86 \\
32857 & 106069 & 2002 & 16.50 & -0.24 & 2391.00 & 16024.50 & 0.69 & 0.97 & 0.67 \\
34829 & 106281 & 2002 & 18.10 & -0.13 & 1848.00 & 17121.88 & 0.98 & 0.95 & 0.93 \\
39394 & 107691 & 2002 & 51.80 & -0.38 & 6093.00 & 49865.78 & 0.85 & 0.96 & 0.82 \\
54790 & 400014 & 2002 & 144.00 & -0.27 & 14415.00 & 136400.97 & 1.00 & 0.95 & 0.95 \\
44958 & 109405 & 2002 & 23.00 & -0.37 & 1741.00 & 18003.03 & 1.32 & 0.78 & 1.03 \\
61961 & 500312 & 2002 & 31.20 & 0.05 & 1998.00 & 19280.80 & 1.56 & 0.62 & 0.97 \\
13332 & 101728 & 2002 & 28.20 & -0.38 & 2156.00 & 21556.55 & 1.31 & 0.76 & 1.00 \\
17845 & 102365 & 2002 & 307.70 & -0.31 & 31923.00 & 319710.07 & 0.96 & 1.04 & 1.00 \\
46680 & 200277 & 2002 & 5.30 & -0.27 & 472.00 & 4941.58 & 1.12 & 0.93 & 1.05 \\
26929 & 103628 & 2002 & 872.80 & -0.18 & 87373.00 & 854780.74 & 1.00 & 0.98 & 0.98 \\
34840 & 106282 & 2002 & 430.60 & -0.19 & 43063.00 & 393027.49 & 1.00 & 0.91 & 0.91 \\
21165 & 102835 & 2002 & 222.60 & -0.14 & 20436.00 & 180487.88 & 1.09 & 0.81 & 0.88 \\
49372 & 240286 & 2002 & 13.20 & -0.24 & 1739.00 & 12712.35 & 0.76 & 0.96 & 0.73 \\
30618 & 105779 & 2002 & 1598.60 & -0.23 & 162182.00 & 1531497.57 & 0.99 & 0.96 & 0.94 \\
52830 & 330728 & 2002 & 7.70 & -0.09 & 770.00 & 6972.50 & 1.00 & 0.91 & 0.91 \\
45779 & 200133 & 2002 & 16.00 & -0.22 & 1608.00 & 15525.79 & 1.00 & 0.97 & 0.97 \\
30654 & 105781 & 2002 & 317.50 & -0.18 & 24143.00 & 309421.72 & 1.32 & 0.97 & 1.28 \\
39419 & 107692 & 2002 & 8.60 & -0.23 & 878.00 & 7659.66 & 0.98 & 0.89 & 0.87 \\
14807 & 101914 & 2002 & 35.70 & -0.17 & 3558.00 & 34417.97 & 1.00 & 0.96 & 0.97 \\
4891 & 100691 & 2002 & 529.00 & -0.32 & 52932.00 & 521887.39 & 1.00 & 0.99 & 0.99 \\
11716 & 101457 & 2002 & 256.00 & -0.26 & 39767.00 & 245128.93 & 0.64 & 0.96 & 0.62 \\
30645 & 105780 & 2002 & 248.40 & -0.25 & 24703.00 & 250933.12 & 1.01 & 1.01 & 1.02 \\
19606 & 102636 & 2002 & 966.10 & -0.16 & 94173.00 & 941698.72 & 1.03 & 0.97 & 1.00 \\
51616 & 240535 & 2002 & 14.40 & -0.33 & 2235.00 & 14565.31 & 0.64 & 1.01 & 0.65 \\
19272 & 102578 & 2002 & 52.20 & -0.33 & 5177.00 & 51774.00 & 1.01 & 0.99 & 1.00 \\
33576 & 106150 & 2002 & 124.00 & -0.32 & 19137.00 & 107276.05 & 0.65 & 0.87 & 0.56 \\
21143 & 102833 & 2002 & 14.20 & -0.42 & 1482.00 & 14817.71 & 0.96 & 1.04 & 1.00 \\
15916 & 102059 & 2002 & 451.40 & -0.27 & 46110.00 & 426247.53 & 0.98 & 0.94 & 0.92 \\
34817 & 106278 & 2002 & 89.00 & -0.18 & 8883.00 & 87279.22 & 1.00 & 0.98 & 0.98 \\
11783 & 101461 & 2002 & 1953.10 & -0.25 & 273043.00 & 1975871.63 & 0.72 & 1.01 & 0.72 \\
25355 & 103478 & 2002 & 1682.30 & -0.23 & 170677.00 & 1676161.42 & 0.99 & 1.00 & 0.98 \\
745 & 100093 & 2002 & 130.70 & -0.28 & 13071.00 & 123539.71 & 1.00 & 0.95 & 0.95 \\
30296 & 105731 & 2002 & 1467.60 & -0.25 & 146363.00 & 1334695.64 & 1.00 & 0.91 & 0.91 \\
1937 & 100259 & 2002 & 110.20 & -0.18 & 10960.00 & 109493.49 & 1.01 & 0.99 & 1.00 \\
6055 & 100821 & 2002 & 59.20 & -0.17 & 5922.00 & 59075.04 & 1.00 & 1.00 & 1.00 \\
42532 & 108976 & 2002 & 39.30 & -0.24 & 3578.00 & 35781.14 & 1.10 & 0.91 & 1.00 \\
25412 & 103483 & 2002 & 490.70 & -0.21 & 48934.00 & 476237.21 & 1.00 & 0.97 & 0.97 \\
4229 & 100590 & 2002 & 158.80 & -0.28 & 15974.00 & 157004.31 & 0.99 & 0.99 & 0.98 \\
35118 & 106321 & 2002 & 3.20 & -0.25 & 314.00 & 3061.75 & 1.02 & 0.96 & 0.98 \\
42529 & 108975 & 2002 & 17.20 & -0.55 & 1634.00 & 16281.02 & 1.05 & 0.95 & 1.00 \\
33553 & 106148 & 2002 & 313.60 & -0.24 & 31257.00 & 309785.43 & 1.00 & 0.99 & 0.99 \\
15064 & 101955 & 2002 & 10418.90 & -0.20 & 1416135.00 & 10059356.44 & 0.74 & 0.97 & 0.71 \\
4911 & 100692 & 2002 & 381.60 & -0.30 & 38138.00 & 370013.26 & 1.00 & 0.97 & 0.97 \\
42513 & 108973 & 2002 & 51.40 & -0.10 & 5162.00 & 46935.89 & 1.00 & 0.91 & 0.91 \\
30270 & 105723 & 2002 & 548.00 & -0.17 & 56621.00 & 524031.19 & 0.97 & 0.96 & 0.93 \\
47212 & 200343 & 2002 & 3735.30 & -0.14 & NaN & 3090399.11 & 1.00 & 0.83 & 1.00 \\
44013 & 109258 & 2002 & 320.70 & -0.16 & 32139.00 & 270198.41 & 1.00 & 0.84 & 0.84 \\
35092 & 106320 & 2002 & 644.00 & -0.22 & 64300.00 & 622853.29 & 1.00 & 0.97 & 0.97 \\
54770 & 400008 & 2002 & 40.10 & -0.20 & 4002.00 & 34385.72 & 1.00 & 0.86 & 0.86 \\
15875 & 102050 & 2002 & 141.10 & -0.08 & 10051.00 & 101433.32 & 1.40 & 0.72 & 1.01 \\
2235 & 100298 & 2002 & 562.70 & -0.29 & 48059.00 & 499620.13 & 1.17 & 0.89 & 1.04 \\
12104 & 101503 & 2002 & 134.30 & -0.25 & 13456.00 & 126331.67 & 1.00 & 0.94 & 0.94 \\
35068 & 106317 & 2002 & 131.60 & -0.25 & 13114.00 & 124987.91 & 1.00 & 0.95 & 0.95 \\
61931 & 500309 & 2002 & 17.30 & -0.15 & 1740.00 & 17238.34 & 0.99 & 1.00 & 0.99 \\
24407 & 103319 & 2002 & 217.70 & -0.21 & 21750.00 & 205720.84 & 1.00 & 0.94 & 0.95 \\
51421 & 240519 & 2002 & 5.40 & -0.07 & 446.00 & 4080.95 & 1.21 & 0.76 & 0.92 \\
46609 & 200262 & 2002 & 3.20 & -0.34 & 243.00 & 2208.39 & 1.32 & 0.69 & 0.91 \\
54917 & 400025 & 2002 & 104.30 & -0.28 & 10215.00 & 103635.55 & 1.02 & 0.99 & 1.01 \\
3243 & 100417 & 2002 & 10.40 & -0.29 & 1214.00 & 9915.96 & 0.86 & 0.95 & 0.82 \\
35077 & 106318 & 2002 & 101.30 & -0.14 & 10129.00 & 97766.64 & 1.00 & 0.97 & 0.97 \\
58404 & 410154 & 2002 & 10.70 & -0.05 & 1259.00 & 10841.77 & 0.85 & 1.01 & 0.86 \\
40958 & 108168 & 2002 & 209.80 & -0.23 & 20972.00 & 208283.82 & 1.00 & 0.99 & 0.99 \\
30325 & 105737 & 2002 & 3.50 & -0.29 & 352.00 & 3514.56 & 0.99 & 1.00 & 1.00 \\
46644 & 200268 & 2002 & 41.70 & -0.22 & 3182.00 & 32738.32 & 1.31 & 0.79 & 1.03 \\
30268 & 105722 & 2002 & 11.70 & -0.25 & 1180.00 & 10202.41 & 0.99 & 0.87 & 0.86 \\
46709 & 200291 & 2002 & 8.60 & -0.20 & 858.00 & 8582.35 & 1.00 & 1.00 & 1.00 \\
44065 & 109264 & 2002 & 30.20 & -0.47 & 5228.00 & 28868.90 & 0.58 & 0.96 & 0.55 \\
10464 & 101286 & 2002 & 961.70 & -0.26 & 96170.00 & 876655.06 & 1.00 & 0.91 & 0.91 \\
35171 & 106330 & 2002 & 145.90 & -0.21 & 14550.00 & 130761.56 & 1.00 & 0.90 & 0.90 \\
40023 & 108009 & 2002 & 628.10 & -0.29 & 62795.00 & 588510.61 & 1.00 & 0.94 & 0.94 \\
44666 & 109358 & 2002 & 47.00 & -0.26 & 4134.00 & 44125.55 & 1.14 & 0.94 & 1.07 \\
42476 & 108970 & 2002 & 70.80 & -0.35 & 6211.00 & 67259.36 & 1.14 & 0.95 & 1.08 \\
30237 & 105720 & 2002 & 658.90 & -0.45 & 66630.00 & 594132.73 & 0.99 & 0.90 & 0.89 \\
42488 & 108971 & 2002 & 86.90 & -0.18 & 11001.00 & 81744.01 & 0.79 & 0.94 & 0.74 \\
58134 & 410095 & 2002 & 69.00 & -0.17 & 7236.00 & 71773.93 & 0.95 & 1.04 & 0.99 \\
25436 & 103487 & 2002 & 96.10 & -0.10 & 9535.00 & 94563.71 & 1.01 & 0.98 & 0.99 \\
15011 & 101933 & 2002 & 121.30 & -0.32 & 12096.00 & 112935.91 & 1.00 & 0.93 & 0.93 \\
13284 & 101717 & 2002 & 65.80 & -0.18 & 6271.00 & 65796.91 & 1.05 & 1.00 & 1.05 \\
56701 & 400241 & 2002 & 14.60 & -0.18 & NaN & 13413.90 & 1.00 & 0.92 & 1.00 \\
46604 & 200261 & 2002 & 4.50 & -0.46 & 357.00 & 3206.65 & 1.26 & 0.71 & 0.90 \\
49380 & 240287 & 2002 & 9.60 & -0.27 & 1223.00 & 9173.58 & 0.78 & 0.96 & 0.75 \\
61926 & 500308 & 2002 & 57.30 & -0.22 & 5724.00 & 56847.75 & 1.00 & 0.99 & 0.99 \\
13776 & 101763 & 2002 & 74.10 & -0.08 & 7411.00 & 68939.25 & 1.00 & 0.93 & 0.93 \\
59243 & 410447 & 2002 & 23.60 & -0.03 & 2358.00 & 22801.50 & 1.00 & 0.97 & 0.97 \\
39438 & 107693 & 2002 & 35.70 & 0.02 & 3573.00 & 28789.54 & 1.00 & 0.81 & 0.81 \\
8822 & 101100 & 2002 & 473.50 & -0.01 & 53253.00 & 433428.70 & 0.89 & 0.92 & 0.81 \\
65135 & 500664 & 2002 & 1439.00 & -0.19 & 185116.00 & 1425615.51 & 0.78 & 0.99 & 0.77 \\
40876 & 108160 & 2002 & 17.20 & -0.23 & 1717.00 & 16845.67 & 1.00 & 0.98 & 0.98 \\
57935 & 410003 & 2002 & 355.50 & -0.32 & 35747.00 & 353936.48 & 0.99 & 1.00 & 0.99 \\
33539 & 106147 & 2002 & 59.30 & -0.33 & 5950.00 & 56743.35 & 1.00 & 0.96 & 0.95 \\
13587 & 101744 & 2002 & 742.60 & -0.27 & 114161.00 & 753246.00 & 0.65 & 1.01 & 0.66 \\
47609 & 215687 & 2002 & 122.10 & -0.32 & 12218.00 & 119740.45 & 1.00 & 0.98 & 0.98 \\
16028 & 102073 & 2002 & 7033.00 & -0.17 & 645703.00 & 6945515.78 & 1.09 & 0.99 & 1.08 \\
35145 & 106329 & 2002 & 15.70 & -0.30 & 1511.00 & 15107.87 & 1.04 & 0.96 & 1.00 \\
7019 & 100985 & 2002 & 1986.20 & -0.19 & 202700.00 & 1901716.22 & 0.98 & 0.96 & 0.94 \\
21313 & 102847 & 2002 & 14.40 & -0.25 & 1468.00 & 14179.79 & 0.98 & 0.98 & 0.97 \\
53895 & 360021 & 2002 & 20.60 & -0.16 & 2562.00 & 19831.83 & 0.80 & 0.96 & 0.77 \\
30433 & 105758 & 2002 & 38.10 & -0.33 & 7149.00 & 37390.17 & 0.53 & 0.98 & 0.52 \\
53655 & 355965 & 2002 & 880.50 & -0.15 & 88295.00 & 877838.43 & 1.00 & 1.00 & 0.99 \\
10385 & 101284 & 2002 & 1284.90 & -0.25 & 128486.00 & 1153591.54 & 1.00 & 0.90 & 0.90 \\
21068 & 102827 & 2002 & 116.40 & -0.14 & 10683.00 & 103774.12 & 1.09 & 0.89 & 0.97 \\
26916 & 103621 & 2002 & 87.90 & -0.27 & 8784.00 & 87688.39 & 1.00 & 1.00 & 1.00 \\
32645 & 106045 & 2002 & 23.90 & -0.22 & 2206.00 & 23647.91 & 1.08 & 0.99 & 1.07 \\
47583 & 215413 & 2002 & 8.10 & -0.02 & 729.00 & 7852.90 & 1.11 & 0.97 & 1.08 \\
30764 & 105794 & 2002 & 28.70 & -0.19 & 2921.00 & 23946.90 & 0.98 & 0.83 & 0.82 \\
23628 & 103204 & 2002 & 151.80 & -0.20 & 15171.00 & 145797.70 & 1.00 & 0.96 & 0.96 \\
43255 & 109087 & 2002 & 228.80 & -0.28 & 23680.00 & 196356.19 & 0.97 & 0.86 & 0.83 \\
30443 & 105760 & 2002 & 451.00 & -0.31 & 57320.00 & 385644.13 & 0.79 & 0.86 & 0.67 \\
12412 & 101539 & 2002 & 1089.00 & -0.23 & 109538.00 & 1071124.43 & 0.99 & 0.98 & 0.98 \\
32147 & 105987 & 2002 & 208.50 & -0.29 & 20952.00 & 201251.68 & 1.00 & 0.97 & 0.96 \\
34983 & 106299 & 2002 & 59.80 & -0.31 & 4608.00 & 55685.11 & 1.30 & 0.93 & 1.21 \\
30471 & 105761 & 2002 & 772.20 & -0.17 & 96183.00 & 770694.11 & 0.80 & 1.00 & 0.80 \\
46640 & 200267 & 2002 & 8.70 & -0.17 & 745.00 & 7343.49 & 1.17 & 0.84 & 0.99 \\
40002 & 107992 & 2002 & 22.60 & -0.29 & 2190.00 & 22119.27 & 1.03 & 0.98 & 1.01 \\
34995 & 106305 & 2002 & 132.70 & -0.15 & 12649.00 & 127395.70 & 1.05 & 0.96 & 1.01 \\
44045 & 109261 & 2002 & 22.90 & -0.29 & 2291.00 & 22587.28 & 1.00 & 0.99 & 0.99 \\
43480 & 109127 & 2002 & 14.40 & -0.33 & 1419.00 & 13322.95 & 1.01 & 0.93 & 0.94 \\
35022 & 106306 & 2002 & 15.50 & -0.21 & 1418.00 & 15188.52 & 1.09 & 0.98 & 1.07 \\
10260 & 101276 & 2002 & 615.60 & -0.04 & 53460.00 & 448112.32 & 1.15 & 0.73 & 0.84 \\
21255 & 102843 & 2002 & 209.20 & -0.20 & 20988.00 & 206876.01 & 1.00 & 0.99 & 0.99 \\
44930 & 109402 & 2002 & 171.00 & -0.23 & 15865.00 & 158968.26 & 1.08 & 0.93 & 1.00 \\
26964 & 103638 & 2002 & 15.10 & -0.21 & 1443.00 & 15334.79 & 1.05 & 1.02 & 1.06 \\
30397 & 105753 & 2002 & 56.00 & -0.12 & 5006.00 & 48848.31 & 1.12 & 0.87 & 0.98 \\
21279 & 102844 & 2002 & 451.80 & -0.23 & 45082.00 & 448122.61 & 1.00 & 0.99 & 0.99 \\
30358 & 105741 & 2002 & 166.10 & -0.23 & 16611.00 & 161360.94 & 1.00 & 0.97 & 0.97 \\
48042 & 227155 & 2002 & 35.30 & -0.19 & 3559.00 & 34350.42 & 0.99 & 0.97 & 0.97 \\
17814 & 102364 & 2002 & 644.90 & -0.22 & 60686.00 & 612804.34 & 1.06 & 0.95 & 1.01 \\
43742 & 109217 & 2002 & 20.50 & -0.24 & 2049.00 & 20774.91 & 1.00 & 1.01 & 1.01 \\
55034 & 400049 & 2002 & 90.00 & 0.83 & 7163.00 & 63330.26 & 1.26 & 0.70 & 0.88 \\
15984 & 102062 & 2002 & 317.10 & -0.18 & 26875.00 & 263294.39 & 1.18 & 0.83 & 0.98 \\
21304 & 102846 & 2002 & 34.00 & -0.25 & 3385.00 & 32722.82 & 1.00 & 0.96 & 0.97 \\
1351 & 100190 & 2002 & 1604.20 & -0.30 & 161028.00 & 1558129.05 & 1.00 & 0.97 & 0.97 \\
7465 & 101040 & 2002 & 2752.10 & -0.19 & 312577.00 & 2480042.06 & 0.88 & 0.90 & 0.79 \\
3259 & 100419 & 2002 & 372.50 & -0.40 & 52728.00 & 344399.35 & 0.71 & 0.92 & 0.65 \\
4215 & 100575 & 2002 & 7.60 & -0.28 & 761.00 & 7553.59 & 1.00 & 0.99 & 0.99 \\
32780 & 106062 & 2002 & 56.30 & -0.14 & 2944.00 & 55421.61 & 1.91 & 0.98 & 1.88 \\
46613 & 200263 & 2002 & 37.70 & -0.11 & 3760.00 & 37475.70 & 1.00 & 0.99 & 1.00 \\
17254 & 102274 & 2002 & 4062.20 & -0.03 & 406228.00 & 3777441.36 & 1.00 & 0.93 & 0.93 \\
39332 & 107666 & 2002 & 22.90 & -0.43 & 1905.00 & 24029.35 & 1.20 & 1.05 & 1.26 \\
32153 & 105990 & 2002 & 196.90 & -0.33 & 25834.00 & 248356.00 & 0.76 & 1.26 & 0.96 \\
40953 & 108166 & 2002 & 126.70 & -0.36 & 12702.00 & 121624.30 & 1.00 & 0.96 & 0.96 \\
39336 & 107670 & 2002 & 386.50 & -0.25 & 38675.00 & 386158.85 & 1.00 & 1.00 & 1.00 \\
161 & 100016 & 2002 & 100.60 & -0.24 & 10246.00 & 99266.65 & 0.98 & 0.99 & 0.97 \\
49149 & 240234 & 2002 & 133.50 & 0.06 & 12488.00 & 124881.52 & 1.07 & 0.94 & 1.00 \\
19224 & 102570 & 2002 & 175.00 & -0.32 & 17534.00 & 161706.15 & 1.00 & 0.92 & 0.92 \\
30375 & 105746 & 2002 & 215.70 & -0.17 & 21545.00 & 206814.49 & 1.00 & 0.96 & 0.96 \\
46622 & 200266 & 2002 & 1.30 & -0.07 & 96.00 & 865.58 & 1.35 & 0.67 & 0.90 \\
24423 & 103326 & 2002 & 970.20 & -0.32 & 142104.00 & 995891.16 & 0.68 & 1.03 & 0.70 \\
42544 & 108977 & 2002 & 25.70 & -0.14 & 2292.00 & 22920.53 & 1.12 & 0.89 & 1.00 \\
58399 & 410153 & 2002 & 8.20 & -0.08 & 963.00 & 7706.55 & 0.85 & 0.94 & 0.80 \\
53878 & 360020 & 2002 & 47.80 & -0.26 & 5410.00 & 43985.51 & 0.88 & 0.92 & 0.81 \\
25382 & 103481 & 2002 & 172.50 & -0.19 & 17199.00 & 168611.80 & 1.00 & 0.98 & 0.98 \\
42563 & 108979 & 2002 & 100.70 & -0.17 & 9130.00 & 94456.61 & 1.10 & 0.94 & 1.03 \\
38283 & 107235 & 2003 & 79.00 & 0.27 & 7944.00 & 76688.64 & 0.99 & 0.97 & 0.97 \\
38392 & 107253 & 2003 & 255.90 & 0.22 & 25574.00 & 230461.25 & 1.00 & 0.90 & 0.90 \\
26653 & 103595 & 2003 & 151.90 & 0.30 & 15597.00 & 153374.82 & 0.97 & 1.01 & 0.98 \\
47923 & 222809 & 2003 & 21.80 & 0.33 & 1917.00 & 19525.22 & 1.14 & 0.90 & 1.02 \\
38275 & 107234 & 2003 & 10.70 & 0.22 & 1054.00 & 10541.25 & 1.02 & 0.99 & 1.00 \\
49577 & 240318 & 2003 & 150.50 & 0.57 & 14400.00 & 149807.71 & 1.05 & 1.00 & 1.04 \\
5137 & 100726 & 2003 & 5174.40 & 0.19 & 517542.00 & 4943894.29 & 1.00 & 0.96 & 0.96 \\
38159 & 107209 & 2003 & 252.80 & 0.20 & 22906.00 & 231185.11 & 1.10 & 0.91 & 1.01 \\
45678 & 200089 & 2003 & 57.10 & 0.58 & 4625.00 & 49806.68 & 1.23 & 0.87 & 1.08 \\
38244 & 107226 & 2003 & 314.80 & 0.41 & 31322.00 & 251372.94 & 1.01 & 0.80 & 0.80 \\
45647 & 200087 & 2003 & 8.40 & 0.04 & 801.00 & 8184.96 & 1.05 & 0.97 & 1.02 \\
14306 & 101843 & 2003 & 135.10 & 0.25 & 12512.00 & 135629.96 & 1.08 & 1.00 & 1.08 \\
26556 & 103591 & 2003 & 1454.70 & 0.38 & 132679.00 & 1343611.90 & 1.10 & 0.92 & 1.01 \\
32646 & 106045 & 2003 & 29.20 & 0.27 & 3147.00 & 28934.69 & 0.93 & 0.99 & 0.92 \\
38370 & 107246 & 2003 & 75.00 & 0.20 & 8164.00 & 69465.95 & 0.92 & 0.93 & 0.85 \\
38188 & 107215 & 2003 & 345.40 & 0.39 & 25152.00 & 289879.87 & 1.37 & 0.84 & 1.15 \\
26697 & 103600 & 2003 & 246.80 & 0.48 & 24676.00 & 203242.86 & 1.00 & 0.82 & 0.82 \\
26621 & 103593 & 2003 & 63798.80 & 0.25 & 6283830.00 & 59967045.39 & 1.02 & 0.94 & 0.95 \\
8280 & 101081 & 2003 & 528.00 & 0.31 & 47372.00 & 526650.56 & 1.11 & 1.00 & 1.11 \\
49568 & 240314 & 2003 & 21.10 & 0.24 & 2117.00 & 20999.19 & 1.00 & 1.00 & 0.99 \\
38320 & 107243 & 2003 & 1274.00 & 0.29 & 124251.00 & 1218380.07 & 1.03 & 0.96 & 0.98 \\
26524 & 103590 & 2003 & 1258.30 & 0.37 & 113407.00 & 1138455.19 & 1.11 & 0.90 & 1.00 \\
7020 & 100985 & 2003 & 3109.50 & 0.23 & 289931.00 & 3064681.23 & 1.07 & 0.99 & 1.06 \\
38295 & 107242 & 2003 & 1106.70 & 0.25 & 110450.00 & 1061592.08 & 1.00 & 0.96 & 0.96 \\
45703 & 200091 & 2003 & 9.10 & 0.25 & 909.00 & 8737.90 & 1.00 & 0.96 & 0.96 \\
15117 & 101958 & 2003 & 671.80 & 0.27 & 67233.00 & 660774.96 & 1.00 & 0.98 & 0.98 \\
45669 & 200088 & 2003 & 18.60 & 0.49 & 1842.00 & 17479.99 & 1.01 & 0.94 & 0.95 \\
38385 & 107250 & 2003 & 5.60 & 0.04 & 610.00 & 5654.06 & 0.92 & 1.01 & 0.93 \\
38269 & 107227 & 2003 & 100.10 & 0.49 & 10016.00 & 100005.64 & 1.00 & 1.00 & 1.00 \\
17343 & 102282 & 2003 & 170.50 & 0.34 & 17058.00 & 150397.43 & 1.00 & 0.88 & 0.88 \\
57970 & 410010 & 2003 & 556.10 & 0.22 & 41376.00 & 467290.81 & 1.34 & 0.84 & 1.13 \\
17387 & 102284 & 2003 & 285.00 & 0.25 & 28513.00 & 282261.92 & 1.00 & 0.99 & 0.99 \\
47383 & 210681 & 2003 & 36051.40 & 0.28 & 3027032.00 & 29685435.26 & 1.19 & 0.82 & 0.98 \\
38234 & 107224 & 2003 & 85.80 & 0.42 & 8441.00 & 76647.50 & 1.02 & 0.89 & 0.91 \\
836 & 100098 & 2003 & 595.70 & 0.45 & 54153.00 & 496341.96 & 1.10 & 0.83 & 0.92 \\
11815 & 101462 & 2003 & 628.60 & 0.24 & 50053.00 & 538736.65 & 1.26 & 0.86 & 1.08 \\
38213 & 107222 & 2003 & 942.90 & 0.30 & 73082.00 & 689860.55 & 1.29 & 0.73 & 0.94 \\
73394 & 600012 & 2003 & 93.10 & 0.56 & 8622.00 & 81386.07 & 1.08 & 0.87 & 0.94 \\
49585 & 240319 & 2003 & 149.70 & 0.68 & 11730.00 & 119886.66 & 1.28 & 0.80 & 1.02 \\
26587 & 103592 & 2003 & 595.10 & 0.45 & 54013.00 & 538057.96 & 1.10 & 0.90 & 1.00 \\
38345 & 107244 & 2003 & 288.30 & 0.23 & 27970.00 & 287120.62 & 1.03 & 1.00 & 1.03 \\
52541 & 303121 & 2003 & 151.40 & 0.22 & 15140.00 & 147036.61 & 1.00 & 0.97 & 0.97 \\
26487 & 103582 & 2003 & 17.40 & 0.50 & 1746.00 & 16399.44 & 1.00 & 0.94 & 0.94 \\
39370 & 107673 & 2003 & 165.70 & 0.34 & 14476.00 & 154549.99 & 1.14 & 0.93 & 1.07 \\
52364 & 302813 & 2003 & 21.40 & 0.29 & 2136.00 & 19314.19 & 1.00 & 0.90 & 0.90 \\
12128 & 101511 & 2003 & 685.30 & 0.44 & 64850.00 & 573186.03 & 1.06 & 0.84 & 0.88 \\
1371 & 100192 & 2003 & 66.40 & 0.26 & 5988.00 & 66488.99 & 1.11 & 1.00 & 1.11 \\
9353 & 101132 & 2003 & 289.90 & 0.22 & 23328.00 & 260137.63 & 1.24 & 0.90 & 1.12 \\
55357 & 400090 & 2003 & 10.80 & 0.31 & 1088.00 & 10728.93 & 0.99 & 0.99 & 0.99 \\
25356 & 103478 & 2003 & 2859.50 & 0.48 & 286078.00 & 2635272.93 & 1.00 & 0.92 & 0.92 \\
39377 & 107677 & 2003 & 345.10 & 0.44 & 28785.00 & 259774.04 & 1.20 & 0.75 & 0.90 \\
39337 & 107670 & 2003 & 426.90 & 0.32 & 42557.00 & 404982.94 & 1.00 & 0.95 & 0.95 \\
55379 & 400092 & 2003 & 9.00 & 0.31 & 768.00 & 7696.46 & 1.17 & 0.86 & 1.00 \\
53879 & 360020 & 2003 & 60.30 & 0.21 & 6015.00 & 59476.23 & 1.00 & 0.99 & 0.99 \\
25383 & 103481 & 2003 & 328.80 & 0.49 & 32861.00 & 321387.30 & 1.00 & 0.98 & 0.98 \\
17815 & 102364 & 2003 & 1081.10 & 0.28 & 107893.00 & 1002041.26 & 1.00 & 0.93 & 0.93 \\
1352 & 100190 & 2003 & 1794.90 & 0.25 & 176681.00 & 1621957.98 & 1.02 & 0.90 & 0.92 \\
55770 & 400150 & 2003 & 5.70 & 0.43 & 424.00 & 3679.19 & 1.34 & 0.65 & 0.87 \\
12105 & 101503 & 2003 & 164.20 & 0.35 & 16434.00 & 163234.25 & 1.00 & 0.99 & 0.99 \\
53896 & 360021 & 2003 & 29.60 & 0.27 & 2966.00 & 28472.30 & 1.00 & 0.96 & 0.96 \\
25315 & 103466 & 2003 & 1094.00 & 0.29 & 96187.00 & 1034297.00 & 1.14 & 0.95 & 1.08 \\
4892 & 100691 & 2003 & 532.80 & 0.25 & 53289.00 & 524229.16 & 1.00 & 0.98 & 0.98 \\
49373 & 240286 & 2003 & 16.40 & 0.32 & 1361.00 & 15651.13 & 1.20 & 0.95 & 1.15 \\
39479 & 107702 & 2003 & 1003.20 & 0.31 & 100413.00 & 988201.03 & 1.00 & 0.99 & 0.98 \\
7086 & 100996 & 2003 & 1720.10 & 0.27 & 168868.00 & 1743248.63 & 1.02 & 1.01 & 1.03 \\
12162 & 101513 & 2003 & 242.80 & 0.28 & 20885.00 & 217283.59 & 1.16 & 0.89 & 1.04 \\
1452 & 100200 & 2003 & 427.80 & 0.36 & 40294.00 & 414531.03 & 1.06 & 0.97 & 1.03 \\
17874 & 102367 & 2003 & 163.20 & 0.29 & 16353.00 & 152416.43 & 1.00 & 0.93 & 0.93 \\
14135 & 101805 & 2003 & 647.00 & 0.49 & 64688.00 & 641024.77 & 1.00 & 0.99 & 0.99 \\
39458 & 107694 & 2003 & 9.50 & 0.35 & 965.00 & 9198.45 & 0.98 & 0.97 & 0.95 \\
39439 & 107693 & 2003 & 61.60 & 0.29 & 6164.00 & 50140.60 & 1.00 & 0.81 & 0.81 \\
48043 & 227155 & 2003 & 153.80 & 0.44 & 16300.00 & 148324.42 & 0.94 & 0.96 & 0.91 \\
53919 & 360123 & 2003 & 17.60 & 0.26 & 1764.00 & 17764.23 & 1.00 & 1.01 & 1.01 \\
49341 & 240284 & 2003 & 76.60 & 0.17 & 6861.00 & 73267.12 & 1.12 & 0.96 & 1.07 \\
45298 & 200015 & 2003 & 29.70 & 0.29 & 2742.00 & 25483.07 & 1.08 & 0.86 & 0.93 \\
1422 & 100196 & 2003 & 4901.20 & 0.27 & 450496.00 & 4593213.50 & 1.09 & 0.94 & 1.02 \\
25284 & 103464 & 2003 & 841.90 & 0.25 & 78689.00 & 831203.69 & 1.07 & 0.99 & 1.06 \\
39420 & 107692 & 2003 & 7.70 & 0.20 & 808.00 & 7283.57 & 0.95 & 0.95 & 0.90 \\
17846 & 102365 & 2003 & 369.20 & 0.27 & 36906.00 & 366748.23 & 1.00 & 0.99 & 0.99 \\
32858 & 106069 & 2003 & 18.20 & 0.26 & 1734.00 & 18822.07 & 1.05 & 1.03 & 1.09 \\
58135 & 410095 & 2003 & 91.60 & 0.50 & 8262.00 & 81520.90 & 1.11 & 0.89 & 0.99 \\
32803 & 106064 & 2003 & 80.10 & 0.23 & 8019.00 & 76354.75 & 1.00 & 0.95 & 0.95 \\
14167 & 101819 & 2003 & 142.70 & 0.24 & 14637.00 & 124330.88 & 0.97 & 0.87 & 0.85 \\
4912 & 100692 & 2003 & 447.40 & 0.45 & 44763.00 & 437068.14 & 1.00 & 0.98 & 0.98 \\
65299 & 500684 & 2003 & 614.20 & 0.30 & 61424.00 & 611560.76 & 1.00 & 1.00 & 1.00 \\
65348 & 500689 & 2003 & 31.50 & 0.30 & 2445.00 & 28075.91 & 1.29 & 0.89 & 1.15 \\
17755 & 102356 & 2003 & 4.70 & 0.26 & 401.00 & 4008.96 & 1.17 & 0.85 & 1.00 \\
65371 & 500692 & 2003 & 113.30 & 0.48 & 7816.00 & 95739.07 & 1.45 & 0.85 & 1.22 \\
53855 & 359285 & 2003 & 181.00 & 0.32 & 18076.00 & 150199.64 & 1.00 & 0.83 & 0.83 \\
9390 & 101133 & 2003 & 1103.50 & 0.29 & 84825.00 & 916674.80 & 1.30 & 0.83 & 1.08 \\
25557 & 103496 & 2003 & 342.20 & 0.16 & 34649.00 & 344712.06 & 0.99 & 1.01 & 0.99 \\
39257 & 107627 & 2003 & 822.60 & 0.33 & 93752.00 & 780585.58 & 0.88 & 0.95 & 0.83 \\
45378 & 200051 & 2003 & 1.60 & 0.30 & 177.00 & 1815.27 & 0.90 & 1.13 & 1.03 \\
32875 & 106075 & 2003 & 239.40 & 0.24 & 22195.00 & 238223.78 & 1.08 & 1.00 & 1.07 \\
25588 & 103497 & 2003 & 213.70 & 0.34 & 17767.00 & 180567.29 & 1.20 & 0.84 & 1.02 \\
39240 & 107626 & 2003 & 764.20 & 0.30 & 83919.00 & 734005.41 & 0.91 & 0.96 & 0.87 \\
39231 & 107623 & 2003 & 3.80 & 0.38 & 394.00 & 3977.50 & 0.96 & 1.05 & 1.01 \\
17734 & 102350 & 2003 & 464.30 & 0.21 & 46504.00 & 464772.14 & 1.00 & 1.00 & 1.00 \\
53834 & 357756 & 2003 & 69.80 & 0.19 & 6913.00 & 66633.48 & 1.01 & 0.95 & 0.96 \\
55396 & 400093 & 2003 & 158.10 & 0.72 & 15823.00 & 143135.53 & 1.00 & 0.91 & 0.90 \\
53843 & 357762 & 2003 & 34.90 & 0.31 & 3414.00 & 32835.03 & 1.02 & 0.94 & 0.96 \\
45368 & 200050 & 2003 & 105.60 & 0.31 & 10501.00 & 100472.65 & 1.01 & 0.95 & 0.96 \\
39286 & 107648 & 2003 & 364.80 & 0.55 & 36947.00 & 382606.78 & 0.99 & 1.05 & 1.04 \\
12068 & 101494 & 2003 & 205.90 & 0.25 & 20587.00 & 201521.94 & 1.00 & 0.98 & 0.98 \\
49381 & 240287 & 2003 & 14.50 & 0.37 & 1197.00 & 14049.33 & 1.21 & 0.97 & 1.17 \\
25437 & 103487 & 2003 & 134.90 & 0.33 & 13460.00 & 121947.36 & 1.00 & 0.90 & 0.91 \\
15065 & 101955 & 2003 & 12906.30 & 0.26 & 1168455.00 & 12838438.35 & 1.10 & 0.99 & 1.10 \\
39314 & 107653 & 2003 & 26.10 & 0.16 & 2482.00 & 25865.67 & 1.05 & 0.99 & 1.04 \\
49389 & 240288 & 2003 & 7.00 & 0.39 & 598.00 & 7017.85 & 1.17 & 1.00 & 1.17 \\
65136 & 500664 & 2003 & 1778.70 & 0.26 & 151952.00 & 1717515.94 & 1.17 & 0.97 & 1.13 \\
12085 & 101497 & 2003 & 1345.30 & 0.19 & 134929.00 & 1265634.90 & 1.00 & 0.94 & 0.94 \\
47439 & 211051 & 2003 & 379.40 & 0.27 & 38917.00 & 357254.11 & 0.97 & 0.94 & 0.92 \\
65208 & 500670 & 2003 & 305.80 & 0.44 & 25748.00 & 233601.53 & 1.19 & 0.76 & 0.91 \\
17786 & 102357 & 2003 & 1258.00 & 0.26 & 125665.00 & 1256553.20 & 1.00 & 1.00 & 1.00 \\
39308 & 107652 & 2003 & 93.60 & 0.33 & 9316.00 & 91068.90 & 1.00 & 0.97 & 0.98 \\
1275 & 100171 & 2003 & 697.80 & 0.36 & 65546.00 & 639936.21 & 1.06 & 0.92 & 0.98 \\
48032 & 226946 & 2003 & 278.10 & 0.24 & 27767.00 & 255032.31 & 1.00 & 0.92 & 0.92 \\
53867 & 359485 & 2003 & 233.60 & 0.24 & 19162.00 & 203519.73 & 1.22 & 0.87 & 1.06 \\
1257 & 100167 & 2003 & 219.20 & 0.22 & 21419.00 & 228016.93 & 1.02 & 1.04 & 1.06 \\
45342 & 200047 & 2003 & 40.60 & 0.35 & 4061.00 & 39242.49 & 1.00 & 0.97 & 0.97 \\
39302 & 107650 & 2003 & 90.70 & 0.77 & 9066.00 & 87552.85 & 1.00 & 0.97 & 0.97 \\
25413 & 103483 & 2003 & 594.00 & 0.27 & 59246.00 & 580778.32 & 1.00 & 0.98 & 0.98 \\
1188 & 100159 & 2003 & 54.20 & 0.38 & 5245.00 & 53552.91 & 1.03 & 0.99 & 1.02 \\
17887 & 102371 & 2003 & 249.80 & 0.32 & 24970.00 & 231288.22 & 1.00 & 0.93 & 0.93 \\
45279 & 200011 & 2003 & 145.40 & 0.37 & 11874.00 & 139046.01 & 1.22 & 0.96 & 1.17 \\
24906 & 103394 & 2003 & 60.20 & 0.38 & 6016.00 & 52792.82 & 1.00 & 0.88 & 0.88 \\
49285 & 240264 & 2003 & 316.00 & 0.20 & 31585.00 & 264511.64 & 1.00 & 0.84 & 0.84 \\
45128 & 109431 & 2003 & 8.00 & 0.25 & 703.00 & 7619.92 & 1.14 & 0.95 & 1.08 \\
39666 & 107833 & 2003 & 272.70 & 0.21 & 27001.00 & 272645.95 & 1.01 & 1.00 & 1.01 \\
12248 & 101530 & 2003 & 2185.30 & 0.70 & 218555.00 & 1807124.01 & 1.00 & 0.83 & 0.83 \\
1687 & 100223 & 2003 & 2746.60 & 0.27 & 272733.00 & 2671695.36 & 1.01 & 0.97 & 0.98 \\
45149 & 109432 & 2003 & 7.10 & 0.26 & 728.00 & 6743.07 & 0.98 & 0.95 & 0.93 \\
45152 & 109433 & 2003 & 10.90 & 0.31 & 929.00 & 9089.91 & 1.17 & 0.83 & 0.98 \\
24944 & 103395 & 2003 & 139.80 & 0.24 & 13962.00 & 135032.38 & 1.00 & 0.97 & 0.97 \\
15048 & 101953 & 2003 & 398.70 & 0.32 & 38060.00 & 401804.96 & 1.05 & 1.01 & 1.06 \\
39639 & 107832 & 2003 & 197.20 & 0.32 & 19453.00 & 191509.49 & 1.01 & 0.97 & 0.98 \\
17971 & 102377 & 2003 & 243.60 & 0.68 & 24372.00 & 243717.04 & 1.00 & 1.00 & 1.00 \\
32839 & 106067 & 2003 & 6441.80 & 0.34 & 584405.00 & 5964235.53 & 1.10 & 0.93 & 1.02 \\
64867 & 500638 & 2003 & 103.10 & 0.27 & 10287.00 & 102831.19 & 1.00 & 1.00 & 1.00 \\
55334 & 400085 & 2003 & 33.30 & 0.28 & 3313.00 & 31618.81 & 1.01 & 0.95 & 0.95 \\
12233 & 101528 & 2003 & 72.50 & 0.22 & 7609.00 & 68200.94 & 0.95 & 0.94 & 0.90 \\
58216 & 410121 & 2003 & 41.10 & 0.17 & 4082.00 & 40813.82 & 1.01 & 0.99 & 1.00 \\
53961 & 362424 & 2003 & 144.00 & 0.36 & 14574.00 & 142473.75 & 0.99 & 0.99 & 0.98 \\
49270 & 240261 & 2003 & 113.40 & 0.21 & 10107.00 & 108636.17 & 1.12 & 0.96 & 1.07 \\
39683 & 107835 & 2003 & 390.90 & 0.35 & 39095.00 & 386629.56 & 1.00 & 0.99 & 0.99 \\
24885 & 103383 & 2003 & 1350.20 & 0.42 & 130494.00 & 1258956.52 & 1.03 & 0.93 & 0.96 \\
64775 & 500633 & 2003 & 49.60 & 0.48 & 4946.00 & 49218.26 & 1.00 & 0.99 & 1.00 \\
1750 & 100227 & 2003 & 119.10 & 0.27 & 11859.00 & 115367.28 & 1.00 & 0.97 & 0.97 \\
24804 & 103380 & 2003 & 4276.60 & 0.24 & 414239.00 & 3935615.96 & 1.03 & 0.92 & 0.95 \\
45087 & 109427 & 2003 & 61.00 & 0.26 & 4980.00 & 48606.44 & 1.22 & 0.80 & 0.98 \\
45109 & 109428 & 2003 & 6.00 & 0.42 & 596.00 & 5849.90 & 1.01 & 0.97 & 0.98 \\
39732 & 107858 & 2003 & 21.30 & 0.38 & 2276.00 & 20255.41 & 0.94 & 0.95 & 0.89 \\
49259 & 240256 & 2003 & 66.60 & 0.23 & 6610.00 & 65259.94 & 1.01 & 0.98 & 0.99 \\
45112 & 109429 & 2003 & 11.80 & 0.23 & 1353.00 & 14160.29 & 0.87 & 1.20 & 1.05 \\
64798 & 500634 & 2003 & 56.90 & 0.43 & 5657.00 & 56494.73 & 1.01 & 0.99 & 1.00 \\
24844 & 103381 & 2003 & 23293.90 & 0.26 & 2283003.00 & 22988276.69 & 1.02 & 0.99 & 1.01 \\
18011 & 102386 & 2003 & 230.30 & 0.42 & 22832.00 & 219330.04 & 1.01 & 0.95 & 0.96 \\
12276 & 101531 & 2003 & 219.00 & 0.54 & 22500.00 & 195761.65 & 0.97 & 0.89 & 0.87 \\
39711 & 107837 & 2003 & 36.70 & 0.31 & 3649.00 & 32106.94 & 1.01 & 0.87 & 0.88 \\
64821 & 500635 & 2003 & 35.00 & 0.29 & 3499.00 & 34959.15 & 1.00 & 1.00 & 1.00 \\
64844 & 500636 & 2003 & 86.50 & 0.26 & 8629.00 & 86283.63 & 1.00 & 1.00 & 1.00 \\
8071 & 101073 & 2003 & 7939.80 & 0.36 & 704103.00 & 7842291.65 & 1.13 & 0.99 & 1.11 \\
7117 & 100997 & 2003 & 613.00 & 0.35 & 56462.00 & 542027.37 & 1.09 & 0.88 & 0.96 \\
17997 & 102383 & 2003 & 6.00 & 0.26 & 496.00 & 4975.70 & 1.21 & 0.83 & 1.00 \\
24996 & 103406 & 2003 & 2093.80 & 0.19 & 196299.00 & 1951191.01 & 1.07 & 0.93 & 0.99 \\
25202 & 103460 & 2003 & 936.10 & 0.31 & 88615.00 & 938818.54 & 1.06 & 1.00 & 1.06 \\
53943 & 362337 & 2003 & 9.60 & 0.63 & 863.00 & 8093.40 & 1.11 & 0.84 & 0.94 \\
45174 & 109435 & 2003 & 4.30 & 0.24 & 347.00 & 3539.96 & 1.24 & 0.82 & 1.02 \\
39526 & 107719 & 2003 & 294.10 & 0.24 & 26202.00 & 276915.24 & 1.12 & 0.94 & 1.06 \\
12193 & 101518 & 2003 & 244.00 & 0.28 & 24418.00 & 241764.16 & 1.00 & 0.99 & 0.99 \\
1539 & 100213 & 2003 & 212.30 & 0.23 & 21484.00 & 203223.47 & 0.99 & 0.96 & 0.95 \\
58160 & 410110 & 2003 & 14.70 & 0.22 & 2890.00 & 24160.03 & 0.51 & 1.64 & 0.84 \\
55340 & 400087 & 2003 & 52.20 & 0.30 & 4831.00 & 48641.49 & 1.08 & 0.93 & 1.01 \\
25150 & 103439 & 2003 & 70.70 & 0.43 & 6241.00 & 68232.99 & 1.13 & 0.97 & 1.09 \\
39520 & 107716 & 2003 & 70.30 & 0.20 & 7062.00 & 70088.80 & 1.00 & 1.00 & 0.99 \\
17907 & 102372 & 2003 & 5538.10 & 0.25 & 551133.00 & 4936483.78 & 1.00 & 0.89 & 0.90 \\
32811 & 106066 & 2003 & 1960.70 & 0.39 & 183326.00 & 1691342.93 & 1.07 & 0.86 & 0.92 \\
39507 & 107711 & 2003 & 16.70 & 0.25 & 1665.00 & 16570.49 & 1.00 & 0.99 & 1.00 \\
58139 & 410100 & 2003 & 67.50 & 0.66 & 4644.00 & 45733.98 & 1.45 & 0.68 & 0.98 \\
64985 & 500653 & 2003 & 74.40 & 0.48 & 5227.00 & 44154.62 & 1.42 & 0.59 & 0.84 \\
1521 & 100209 & 2003 & 7349.90 & 0.27 & 748178.00 & 7255124.10 & 0.98 & 0.99 & 0.97 \\
55349 & 400088 & 2003 & 78.80 & 0.49 & 7246.00 & 74880.28 & 1.09 & 0.95 & 1.03 \\
49331 & 240270 & 2003 & 70.60 & 0.32 & 6860.00 & 67410.13 & 1.03 & 0.95 & 0.98 \\
52371 & 302819 & 2003 & 3.60 & 0.43 & 370.00 & 3614.03 & 0.97 & 1.00 & 0.98 \\
65024 & 500656 & 2003 & 708.00 & 0.46 & 70796.00 & 700930.34 & 1.00 & 0.99 & 0.99 \\
1483 & 100207 & 2003 & 1644.00 & 0.21 & 162045.00 & 1686568.31 & 1.01 & 1.03 & 1.04 \\
49320 & 240269 & 2003 & 199.70 & 0.34 & 18996.00 & 188002.94 & 1.05 & 0.94 & 0.99 \\
25115 & 103432 & 2003 & 1957.90 & 0.30 & 165694.00 & 1854551.14 & 1.18 & 0.95 & 1.12 \\
45235 & 109439 & 2003 & 374.80 & 0.39 & 38275.00 & 363593.59 & 0.98 & 0.97 & 0.95 \\
39602 & 107786 & 2003 & 1527.90 & 0.48 & 153044.00 & 1508182.50 & 1.00 & 0.99 & 0.99 \\
45181 & 109436 & 2003 & 37.60 & 0.35 & 3756.00 & 37110.91 & 1.00 & 0.99 & 0.99 \\
64936 & 500646 & 2003 & 2.20 & 0.46 & 215.00 & 2152.76 & 1.02 & 0.98 & 1.00 \\
1599 & 100217 & 2003 & 34.60 & 0.42 & 3373.00 & 30533.25 & 1.03 & 0.88 & 0.91 \\
53938 & 361995 & 2003 & 12.10 & 0.35 & 1294.00 & 12925.95 & 0.94 & 1.07 & 1.00 \\
52377 & 302825 & 2003 & 116.40 & -0.07 & 10720.00 & 115456.56 & 1.09 & 0.99 & 1.08 \\
25039 & 103426 & 2003 & 1098.20 & 0.33 & 98569.00 & 1032388.98 & 1.11 & 0.94 & 1.05 \\
39593 & 107781 & 2003 & 182.10 & 0.18 & 16816.00 & 167624.57 & 1.08 & 0.92 & 1.00 \\
39581 & 107726 & 2003 & 899.50 & 0.28 & 89992.00 & 890127.51 & 1.00 & 0.99 & 0.99 \\
45213 & 109438 & 2003 & 145.70 & 0.41 & 12835.00 & 130001.48 & 1.14 & 0.89 & 1.01 \\
53934 & 361852 & 2003 & 4.50 & 0.33 & 452.00 & 4457.71 & 1.00 & 0.99 & 0.99 \\
1570 & 100214 & 2003 & 155.20 & 0.17 & 15013.00 & 124089.83 & 1.03 & 0.80 & 0.83 \\
17937 & 102376 & 2003 & 111.30 & 0.33 & 11110.00 & 90866.24 & 1.00 & 0.82 & 0.82 \\
58191 & 410115 & 2003 & 45.60 & 0.15 & 4589.00 & 47997.10 & 0.99 & 1.05 & 1.05 \\
25077 & 103429 & 2003 & 2092.90 & 0.31 & 189354.00 & 2103396.36 & 1.11 & 1.01 & 1.11 \\
39565 & 107722 & 2003 & 244.50 & 0.34 & 16529.00 & 172757.26 & 1.48 & 0.71 & 1.05 \\
49294 & 240266 & 2003 & 242.30 & 0.34 & 26194.00 & 261645.93 & 0.93 & 1.08 & 1.00 \\
39551 & 107720 & 2003 & 150.80 & 0.45 & 15034.00 & 144489.45 & 1.00 & 0.96 & 0.96 \\
64958 & 500651 & 2003 & 8.30 & 0.41 & 829.00 & 7975.23 & 1.00 & 0.96 & 0.96 \\
38395 & 107254 & 2003 & 9.40 & 0.02 & 929.00 & 9034.41 & 1.01 & 0.96 & 0.97 \\
53811 & 357133 & 2003 & 47.40 & 0.37 & 4801.00 & 46855.37 & 0.99 & 0.99 & 0.98 \\
4942 & 100695 & 2003 & 173.90 & 0.25 & 17388.00 & 167794.48 & 1.00 & 0.96 & 0.97 \\
26177 & 103545 & 2003 & 26545.00 & 0.30 & 1705410.00 & 16584164.42 & 1.56 & 0.62 & 0.97 \\
38689 & 107308 & 2003 & 2538.10 & 0.36 & 253054.00 & 2395077.41 & 1.00 & 0.94 & 0.95 \\
15099 & 101956 & 2003 & 1678.90 & 0.26 & 168075.00 & 1675285.94 & 1.00 & 1.00 & 1.00 \\
49473 & 240302 & 2003 & 7.20 & 0.22 & 598.00 & 6390.73 & 1.20 & 0.89 & 1.07 \\
38666 & 107306 & 2003 & 267.00 & 0.31 & 26816.00 & 258812.44 & 1.00 & 0.97 & 0.97 \\
38658 & 107303 & 2003 & 52.80 & 0.27 & 4524.00 & 51907.00 & 1.17 & 0.98 & 1.15 \\
65688 & 500719 & 2003 & 154.00 & 0.26 & 13500.00 & 150583.12 & 1.14 & 0.98 & 1.12 \\
17509 & 102317 & 2003 & 20.10 & 0.22 & 1988.00 & 18733.77 & 1.01 & 0.93 & 0.94 \\
49481 & 240303 & 2003 & 15.30 & 0.41 & 1494.00 & 14905.48 & 1.02 & 0.97 & 1.00 \\
26217 & 103546 & 2003 & 18480.10 & 0.37 & 2425518.00 & 24667787.16 & 0.76 & 1.33 & 1.02 \\
38633 & 107302 & 2003 & 276.80 & 0.27 & 27712.00 & 273980.42 & 1.00 & 0.99 & 0.99 \\
38623 & 107300 & 2003 & 65.60 & 0.26 & 5867.00 & 64392.04 & 1.12 & 0.98 & 1.10 \\
65742 & 500729 & 2003 & 112.30 & 0.34 & 11228.00 & 112297.77 & 1.00 & 1.00 & 1.00 \\
32971 & 106084 & 2003 & 1208.40 & 0.58 & 103386.00 & 1150974.87 & 1.17 & 0.95 & 1.11 \\
11878 & 101464 & 2003 & 1215.70 & 0.43 & 95422.00 & 977825.26 & 1.27 & 0.80 & 1.02 \\
5054 & 100710 & 2003 & 1227.30 & 0.22 & 122592.00 & 1186996.19 & 1.00 & 0.97 & 0.97 \\
45579 & 200079 & 2003 & 15.10 & 0.93 & 1506.00 & 15373.87 & 1.00 & 1.02 & 1.02 \\
982 & 100113 & 2003 & 776.20 & 0.65 & 67492.00 & 775345.97 & 1.15 & 1.00 & 1.15 \\
32692 & 106049 & 2003 & 374.50 & 0.28 & 37436.00 & 369405.48 & 1.00 & 0.99 & 0.99 \\
49444 & 240300 & 2003 & 24.50 & 0.46 & 2452.00 & 23888.66 & 1.00 & 0.98 & 0.97 \\
38810 & 107328 & 2003 & 28.30 & 0.28 & 2672.00 & 27376.35 & 1.06 & 0.97 & 1.02 \\
17552 & 102318 & 2003 & 2452.20 & 0.29 & 245181.00 & 2178242.18 & 1.00 & 0.89 & 0.89 \\
38785 & 107323 & 2003 & 113.20 & 0.24 & 11202.00 & 108828.17 & 1.01 & 0.96 & 0.97 \\
32704 & 106050 & 2003 & 537.50 & 0.51 & 53735.00 & 525239.60 & 1.00 & 0.98 & 0.98 \\
38762 & 107322 & 2003 & 16.90 & 0.35 & 1605.00 & 16405.65 & 1.05 & 0.97 & 1.02 \\
47991 & 225687 & 2003 & 363.10 & 0.45 & 36385.00 & 345946.29 & 1.00 & 0.95 & 0.95 \\
65662 & 500712 & 2003 & 88.00 & 0.25 & 6676.00 & 67776.23 & 1.32 & 0.77 & 1.02 \\
65667 & 500713 & 2003 & 221.00 & 0.21 & 21676.00 & 201019.47 & 1.02 & 0.91 & 0.93 \\
49449 & 240301 & 2003 & 5.80 & 0.23 & 493.00 & 5401.64 & 1.18 & 0.93 & 1.10 \\
26109 & 103539 & 2003 & 783.00 & 0.29 & 64300.00 & 718679.38 & 1.22 & 0.92 & 1.12 \\
47978 & 225484 & 2003 & 46.90 & 0.25 & 4698.00 & 45679.95 & 1.00 & 0.97 & 0.97 \\
52352 & 302811 & 2003 & 41.00 & 0.37 & 4094.00 & 37516.34 & 1.00 & 0.92 & 0.92 \\
11908 & 101465 & 2003 & 124.60 & 0.28 & 12492.00 & 120474.69 & 1.00 & 0.97 & 0.96 \\
38726 & 107316 & 2003 & 824.00 & 0.45 & 82559.00 & 811933.15 & 1.00 & 0.99 & 0.98 \\
65672 & 500714 & 2003 & 63.80 & 0.35 & 6371.00 & 52457.84 & 1.00 & 0.82 & 0.82 \\
26143 & 103544 & 2003 & 42545.00 & 0.31 & 3187787.00 & 36506123.04 & 1.33 & 0.86 & 1.15 \\
32944 & 106083 & 2003 & 844.60 & 0.15 & 86465.00 & 883995.58 & 0.98 & 1.05 & 1.02 \\
38714 & 107309 & 2003 & 84.90 & 0.24 & 8297.00 & 86160.66 & 1.02 & 1.01 & 1.04 \\
14279 & 101842 & 2003 & 1497.40 & 0.23 & 149745.00 & 1493043.61 & 1.00 & 1.00 & 1.00 \\
26062 & 103536 & 2003 & 2314.00 & 0.29 & 197924.00 & 2260597.69 & 1.17 & 0.98 & 1.14 \\
38590 & 107294 & 2003 & 199.00 & 0.34 & 23714.00 & 191902.45 & 0.84 & 0.96 & 0.81 \\
938 & 100112 & 2003 & 7687.00 & 0.31 & 752753.00 & 7745090.75 & 1.02 & 1.01 & 1.03 \\
11847 & 101463 & 2003 & 708.50 & 0.30 & 65711.00 & 722222.25 & 1.08 & 1.02 & 1.10 \\
45617 & 200084 & 2003 & 1.20 & 0.22 & 117.00 & 1144.93 & 1.03 & 0.95 & 0.98 \\
26404 & 103579 & 2003 & 567.30 & 0.36 & 42068.00 & 433575.54 & 1.35 & 0.76 & 1.03 \\
38485 & 107263 & 2003 & 1436.90 & 0.23 & 123948.00 & 1266917.49 & 1.16 & 0.88 & 1.02 \\
17440 & 102306 & 2003 & 22060.22 & 0.32 & 1903035.00 & 18431669.97 & 1.16 & 0.84 & 0.97 \\
38460 & 107260 & 2003 & 735.10 & 0.28 & 48763.00 & 510180.58 & 1.51 & 0.69 & 1.05 \\
38435 & 107259 & 2003 & 483.40 & 0.29 & 47114.00 & 436289.83 & 1.03 & 0.90 & 0.93 \\
26378 & 103572 & 2003 & 28.70 & 0.23 & 2839.00 & 29865.58 & 1.01 & 1.04 & 1.05 \\
66295 & 500806 & 2003 & 388.60 & 0.25 & 38045.00 & 371544.94 & 1.02 & 0.96 & 0.98 \\
26436 & 103580 & 2003 & 310.50 & 0.17 & 26150.00 & 268310.96 & 1.19 & 0.86 & 1.03 \\
38426 & 107258 & 2003 & 76.00 & 0.21 & 10773.00 & 107307.33 & 0.71 & 1.41 & 1.00 \\
38401 & 107257 & 2003 & 297.00 & 0.19 & 29772.00 & 290890.39 & 1.00 & 0.98 & 0.98 \\
866 & 100099 & 2003 & 97.10 & 0.42 & 9337.00 & 89854.78 & 1.04 & 0.93 & 0.96 \\
49534 & 240311 & 2003 & 18.30 & 0.48 & 2087.00 & 16399.89 & 0.88 & 0.90 & 0.79 \\
49557 & 240312 & 2003 & 129.30 & 0.29 & 13116.00 & 128926.60 & 0.99 & 1.00 & 0.98 \\
49509 & 240308 & 2003 & 73.00 & 0.13 & 7787.00 & 61090.12 & 0.94 & 0.84 & 0.78 \\
49504 & 240307 & 2003 & 76.00 & 0.47 & 7017.00 & 60892.96 & 1.08 & 0.80 & 0.87 \\
896 & 100101 & 2003 & 483.50 & 0.35 & 52849.00 & 516243.67 & 0.91 & 1.07 & 0.98 \\
17477 & 102312 & 2003 & 39.90 & 0.26 & 4607.00 & 46096.80 & 0.87 & 1.16 & 1.00 \\
53702 & 355988 & 2003 & 75.60 & 0.48 & 4884.00 & 61877.11 & 1.55 & 0.82 & 1.27 \\
52343 & 302780 & 2003 & 57.50 & 0.27 & 6045.00 & 51735.46 & 0.95 & 0.90 & 0.86 \\
26295 & 103564 & 2003 & 480.50 & 0.36 & 48164.00 & 477029.45 & 1.00 & 0.99 & 0.99 \\
49488 & 240304 & 2003 & 266.70 & 0.31 & 25973.00 & 212520.09 & 1.03 & 0.80 & 0.82 \\
47950 & 225413 & 2003 & 163.30 & 0.33 & 16766.00 & 157639.21 & 0.97 & 0.97 & 0.94 \\
6596 & 100900 & 2003 & 87.30 & 0.29 & 7707.00 & 71111.45 & 1.13 & 0.81 & 0.92 \\
26313 & 103567 & 2003 & 693.90 & 0.26 & 112288.00 & 1117730.39 & 0.62 & 1.61 & 1.00 \\
38576 & 107290 & 2003 & 900.10 & 0.28 & 77451.00 & 838605.87 & 1.16 & 0.93 & 1.08 \\
5086 & 100723 & 2003 & 44.60 & 0.25 & 4476.00 & 44495.31 & 1.00 & 1.00 & 0.99 \\
58036 & 410055 & 2003 & 73.40 & 0.23 & 7322.00 & 73163.29 & 1.00 & 1.00 & 1.00 \\
38543 & 107281 & 2003 & 16.90 & 0.25 & 1682.00 & 16765.83 & 1.00 & 0.99 & 1.00 \\
49496 & 240305 & 2003 & 71.30 & 0.60 & 6387.00 & 57896.08 & 1.12 & 0.81 & 0.91 \\
26345 & 103570 & 2003 & 19.20 & 0.30 & 1920.00 & 18398.37 & 1.00 & 0.96 & 0.96 \\
32665 & 106047 & 2003 & 10.80 & 0.56 & 1064.00 & 10643.23 & 1.02 & 0.99 & 1.00 \\
38532 & 107274 & 2003 & 5.70 & 0.24 & 482.00 & 5196.42 & 1.18 & 0.91 & 1.08 \\
45594 & 200082 & 2003 & 19.10 & 0.50 & 1755.00 & 15733.44 & 1.09 & 0.82 & 0.90 \\
38509 & 107266 & 2003 & 301.90 & 0.28 & 25720.00 & 284736.29 & 1.17 & 0.94 & 1.11 \\
32757 & 106061 & 2003 & 424.80 & 0.27 & 41014.00 & 414227.24 & 1.04 & 0.98 & 1.01 \\
45538 & 200073 & 2003 & 441.90 & 0.39 & 41057.00 & 417785.82 & 1.08 & 0.95 & 1.02 \\
49435 & 240297 & 2003 & 719.20 & 0.46 & 68419.00 & 674466.35 & 1.05 & 0.94 & 0.99 \\
39122 & 107607 & 2003 & 76.70 & 0.29 & 7675.00 & 74302.45 & 1.00 & 0.97 & 0.97 \\
65487 & 500700 & 2003 & 102.00 & 1.05 & 10199.00 & 101817.13 & 1.00 & 1.00 & 1.00 \\
17680 & 102342 & 2003 & 141.50 & 0.28 & 15831.00 & 158310.06 & 0.89 & 1.12 & 1.00 \\
25762 & 103521 & 2003 & 4227.90 & 0.32 & 346606.00 & 3965451.87 & 1.22 & 0.94 & 1.14 \\
39060 & 107598 & 2003 & 104.90 & 0.28 & 9360.00 & 93641.28 & 1.12 & 0.89 & 1.00 \\
11993 & 101476 & 2003 & 2759.70 & 0.26 & 248551.00 & 2720455.63 & 1.11 & 0.99 & 1.09 \\
45420 & 200058 & 2003 & 2640.80 & 0.31 & 247500.00 & 2386217.44 & 1.07 & 0.90 & 0.96 \\
4968 & 100697 & 2003 & 34.10 & 0.32 & 3329.00 & 33289.94 & 1.02 & 0.98 & 1.00 \\
39031 & 107573 & 2003 & 218.60 & 0.29 & 21816.00 & 217990.37 & 1.00 & 1.00 & 1.00 \\
39025 & 107566 & 2003 & 3.70 & 0.43 & 350.00 & 3629.02 & 1.06 & 0.98 & 1.04 \\
48013 & 226438 & 2003 & 550.30 & 0.28 & 49906.00 & 549819.97 & 1.10 & 1.00 & 1.10 \\
8212 & 101079 & 2003 & 255.50 & 0.23 & 19233.00 & 214818.21 & 1.33 & 0.84 & 1.12 \\
65510 & 500701 & 2003 & 173.10 & 0.82 & 17310.00 & 172861.06 & 1.00 & 1.00 & 1.00 \\
49409 & 240293 & 2003 & 1705.70 & 0.28 & 154366.00 & 1508588.64 & 1.10 & 0.88 & 0.98 \\
25830 & 103524 & 2003 & 93440.30 & 0.29 & 7881412.00 & 89509810.72 & 1.19 & 0.96 & 1.14 \\
25796 & 103523 & 2003 & 6965.30 & 0.44 & 501217.00 & 5736245.18 & 1.39 & 0.82 & 1.14 \\
25730 & 103520 & 2003 & 9460.00 & 0.26 & 823822.00 & 9437830.37 & 1.15 & 1.00 & 1.15 \\
52388 & 302826 & 2003 & 328.40 & 0.26 & 29432.00 & 324826.71 & 1.12 & 0.99 & 1.10 \\
45383 & 200055 & 2003 & 112.50 & 0.25 & 11122.00 & 111183.09 & 1.01 & 0.99 & 1.00 \\
25632 & 103498 & 2003 & 309.80 & 0.30 & 26019.00 & 278435.01 & 1.19 & 0.90 & 1.07 \\
53791 & 357122 & 2003 & 533.60 & 0.30 & 47543.00 & 525719.90 & 1.12 & 0.99 & 1.11 \\
39198 & 107618 & 2003 & 2895.60 & 0.29 & 327038.00 & 2813009.94 & 0.89 & 0.97 & 0.86 \\
39173 & 107616 & 2003 & 236.70 & 0.35 & 23659.00 & 236179.91 & 1.00 & 1.00 & 1.00 \\
47414 & 210770 & 2003 & 2659.20 & 0.34 & 300270.00 & 2644626.25 & 0.89 & 0.99 & 0.88 \\
65400 & 500694 & 2003 & 284.10 & 0.36 & 28246.00 & 282456.49 & 1.01 & 0.99 & 1.00 \\
45409 & 200057 & 2003 & 882.90 & 0.24 & 89544.00 & 915964.59 & 0.99 & 1.04 & 1.02 \\
65464 & 500699 & 2003 & 147.30 & 0.54 & 14732.00 & 147131.01 & 1.00 & 1.00 & 1.00 \\
1156 & 100157 & 2003 & 1078.10 & 0.28 & 101851.00 & 1073771.69 & 1.06 & 1.00 & 1.05 \\
17711 & 102349 & 2003 & 743.90 & 0.25 & 74742.00 & 739892.63 & 1.00 & 0.99 & 0.99 \\
53778 & 357075 & 2003 & 209.60 & 0.44 & 16705.00 & 169513.07 & 1.25 & 0.81 & 1.01 \\
1141 & 100155 & 2003 & 1690.10 & 0.16 & 168967.00 & 1657548.10 & 1.00 & 0.98 & 0.98 \\
25698 & 103514 & 2003 & 3070.60 & 0.25 & 263664.00 & 2763416.41 & 1.16 & 0.90 & 1.05 \\
39148 & 107611 & 2003 & 1980.00 & 0.44 & 177049.00 & 1775480.74 & 1.12 & 0.90 & 1.00 \\
58102 & 410093 & 2003 & 88.70 & 0.80 & 8881.00 & 82528.84 & 1.00 & 0.93 & 0.93 \\
39132 & 107608 & 2003 & 76.30 & 0.43 & 7746.00 & 75114.52 & 0.99 & 0.98 & 0.97 \\
14200 & 101820 & 2003 & 218.40 & 0.23 & 21438.00 & 218265.15 & 1.02 & 1.00 & 1.02 \\
48004 & 225696 & 2003 & 84.80 & 0.46 & 8498.00 & 80102.58 & 1.00 & 0.94 & 0.94 \\
38826 & 107331 & 2003 & 16.40 & 0.39 & 1636.00 & 15978.14 & 1.00 & 0.97 & 0.98 \\
38994 & 107563 & 2003 & 1388.30 & 0.28 & 106579.00 & 1221203.25 & 1.30 & 0.88 & 1.15 \\
45446 & 200060 & 2003 & 1019.40 & 0.32 & 94762.00 & 940564.35 & 1.08 & 0.92 & 0.99 \\
14231 & 101834 & 2003 & 288.60 & 0.31 & 28630.00 & 280941.20 & 1.01 & 0.97 & 0.98 \\
65595 & 500707 & 2003 & 408.60 & 0.42 & 40862.00 & 408303.92 & 1.00 & 1.00 & 1.00 \\
65618 & 500708 & 2003 & 274.70 & 0.38 & 27466.00 & 273512.43 & 1.00 & 1.00 & 1.00 \\
65639 & 500710 & 2003 & 376.40 & 0.04 & 34850.00 & 356198.22 & 1.08 & 0.95 & 1.02 \\
1041 & 100128 & 2003 & 204.10 & 0.19 & 19660.00 & 169840.91 & 1.04 & 0.83 & 0.86 \\
25986 & 103532 & 2003 & 668.90 & 0.59 & 49470.00 & 659568.03 & 1.35 & 0.99 & 1.33 \\
38868 & 107338 & 2003 & 65.20 & 0.27 & 6130.00 & 52562.32 & 1.06 & 0.81 & 0.86 \\
65572 & 500706 & 2003 & 415.40 & 0.40 & 41541.00 & 414198.01 & 1.00 & 1.00 & 1.00 \\
17586 & 102319 & 2003 & 521.70 & 0.33 & 52141.00 & 488659.39 & 1.00 & 0.94 & 0.94 \\
1026 & 100127 & 2003 & 3301.70 & 0.41 & 320884.00 & 3282484.32 & 1.03 & 0.99 & 1.02 \\
53715 & 356500 & 2003 & 309.80 & 0.21 & 29120.00 & 310939.92 & 1.06 & 1.00 & 1.07 \\
6553 & 100890 & 2003 & 1502.30 & 0.35 & 151184.00 & 1456369.90 & 0.99 & 0.97 & 0.96 \\
9416 & 101134 & 2003 & 156.40 & 0.47 & 12701.00 & 137089.61 & 1.23 & 0.88 & 1.08 \\
14246 & 101835 & 2003 & 1356.40 & 0.28 & 135492.00 & 1347883.67 & 1.00 & 0.99 & 0.99 \\
45512 & 200072 & 2003 & 8.50 & 0.68 & 875.00 & 7839.73 & 0.97 & 0.92 & 0.90 \\
26032 & 103535 & 2003 & 2672.70 & 0.32 & 229319.00 & 2655788.91 & 1.17 & 0.99 & 1.16 \\
38847 & 107336 & 2003 & 592.20 & 0.37 & 50384.00 & 424330.38 & 1.18 & 0.72 & 0.84 \\
38860 & 107337 & 2003 & 128.20 & 0.38 & 11266.00 & 113375.28 & 1.14 & 0.88 & 1.01 \\
38878 & 107339 & 2003 & 62.40 & 0.25 & 6803.00 & 66988.59 & 0.92 & 1.07 & 0.98 \\
53772 & 357053 & 2003 & 59.30 & 0.30 & 5928.00 & 59063.94 & 1.00 & 1.00 & 1.00 \\
38884 & 107350 & 2003 & 4372.20 & 0.22 & 405461.00 & 4016182.68 & 1.08 & 0.92 & 0.99 \\
65533 & 500702 & 2003 & 193.40 & 0.50 & 19341.00 & 193182.95 & 1.00 & 1.00 & 1.00 \\
49417 & 240295 & 2003 & 1830.30 & 0.29 & 165798.00 & 1669674.30 & 1.10 & 0.91 & 1.01 \\
25863 & 103525 & 2003 & 38262.90 & 0.29 & 3379928.00 & 38398782.37 & 1.13 & 1.00 & 1.14 \\
45472 & 200061 & 2003 & 32.00 & 0.26 & 3198.00 & 31400.23 & 1.00 & 0.98 & 0.98 \\
7049 & 100992 & 2003 & 720.50 & 0.40 & 78340.00 & 725265.82 & 0.92 & 1.01 & 0.93 \\
32917 & 106082 & 2003 & 554.30 & 0.30 & 51984.00 & 580938.98 & 1.07 & 1.05 & 1.12 \\
65558 & 500704 & 2003 & 1.10 & 0.24 & 110.00 & 1134.03 & 1.00 & 1.03 & 1.03 \\
17623 & 102321 & 2003 & 146.00 & 0.40 & 14409.00 & 133172.02 & 1.01 & 0.91 & 0.92 \\
1076 & 100150 & 2003 & 23.80 & 0.28 & 2326.00 & 22427.14 & 1.02 & 0.94 & 0.96 \\
49425 & 240296 & 2003 & 959.70 & 0.37 & 87988.00 & 907005.49 & 1.09 & 0.95 & 1.03 \\
25897 & 103526 & 2003 & 4289.80 & 0.34 & 371955.00 & 3811245.80 & 1.15 & 0.89 & 1.02 \\
32723 & 106052 & 2003 & 133.00 & 0.30 & 13241.00 & 130701.07 & 1.00 & 0.98 & 0.99 \\
38925 & 107354 & 2003 & 36.30 & 0.36 & 3634.00 & 34919.23 & 1.00 & 0.96 & 0.96 \\
5001 & 100698 & 2003 & 52.50 & 0.50 & 4914.00 & 49123.60 & 1.07 & 0.94 & 1.00 \\
58047 & 410060 & 2003 & 11.20 & 0.44 & 1119.00 & 10977.59 & 1.00 & 0.98 & 0.98 \\
55812 & 400153 & 2003 & 37.90 & 0.33 & 3662.00 & 34322.20 & 1.03 & 0.91 & 0.94 \\
25926 & 103529 & 2003 & 6411.80 & 0.30 & 593174.00 & 6576128.28 & 1.08 & 1.03 & 1.11 \\
38908 & 107352 & 2003 & 247.50 & 0.39 & 19578.00 & 195768.52 & 1.26 & 0.79 & 1.00 \\
45478 & 200065 & 2003 & 12.50 & 0.23 & 1244.00 & 11672.62 & 1.00 & 0.93 & 0.94 \\
38988 & 107470 & 2003 & 2.00 & 0.32 & 197.00 & 1867.25 & 1.02 & 0.93 & 0.95 \\
32742 & 106057 & 2003 & 359.70 & 0.48 & 34854.00 & 348514.22 & 1.03 & 0.97 & 1.00 \\
16728 & 102182 & 2003 & 180.50 & 0.31 & 12845.00 & 145511.57 & 1.41 & 0.81 & 1.13 \\
38092 & 107201 & 2003 & 355.70 & 0.50 & 28278.00 & 312684.93 & 1.26 & 0.88 & 1.11 \\
10465 & 101286 & 2003 & 1546.40 & 0.29 & 154608.00 & 1431948.82 & 1.00 & 0.93 & 0.93 \\
51363 & 240507 & 2003 & 66.00 & 0.21 & 7120.00 & 71330.00 & 0.93 & 1.08 & 1.00 \\
52982 & 336508 & 2003 & 85.10 & 0.25 & 7394.00 & 78573.85 & 1.15 & 0.92 & 1.06 \\
30238 & 105720 & 2003 & 980.30 & 0.68 & 95257.00 & 870494.86 & 1.03 & 0.89 & 0.91 \\
30214 & 105716 & 2003 & 124.70 & 0.47 & 12419.00 & 118841.46 & 1.00 & 0.95 & 0.96 \\
35188 & 106333 & 2003 & 134.20 & 0.48 & 10959.00 & 122930.77 & 1.22 & 0.92 & 1.12 \\
6025 & 100820 & 2003 & 529.10 & 0.27 & 44015.00 & 483492.67 & 1.20 & 0.91 & 1.10 \\
180 & 100017 & 2003 & 72.80 & 0.30 & 7474.00 & 74412.83 & 0.97 & 1.02 & 1.00 \\
35215 & 106334 & 2003 & 43.00 & 0.32 & 4156.00 & 41564.49 & 1.03 & 0.97 & 1.00 \\
51318 & 240500 & 2003 & 10.20 & 0.23 & 1047.00 & 10152.08 & 0.97 & 1.00 & 0.97 \\
10481 & 101287 & 2003 & 782.00 & 0.24 & 78095.00 & 710178.40 & 1.00 & 0.91 & 0.91 \\
35172 & 106330 & 2003 & 243.60 & 0.36 & 24265.00 & 219324.70 & 1.00 & 0.90 & 0.90 \\
16029 & 102073 & 2003 & 8647.60 & 0.32 & 746636.00 & 8895632.10 & 1.16 & 1.03 & 1.19 \\
35146 & 106329 & 2003 & 18.40 & 0.24 & 1578.00 & 15784.16 & 1.17 & 0.86 & 1.00 \\
35078 & 106318 & 2003 & 107.90 & 0.37 & 10882.00 & 105861.70 & 0.99 & 0.98 & 0.97 \\
30326 & 105737 & 2003 & 4.70 & 0.33 & 472.00 & 4582.18 & 1.00 & 0.97 & 0.97 \\
51415 & 240517 & 2003 & 12.00 & 0.21 & 1242.00 & 11542.74 & 0.97 & 0.96 & 0.93 \\
35093 & 106320 & 2003 & 737.00 & 0.33 & 83417.00 & 714151.93 & 0.88 & 0.97 & 0.86 \\
30297 & 105731 & 2003 & 2012.50 & 0.35 & 178122.00 & 1727289.35 & 1.13 & 0.86 & 0.97 \\
33554 & 106148 & 2003 & 263.60 & 0.19 & 26367.00 & 251414.00 & 1.00 & 0.95 & 0.95 \\
52956 & 336065 & 2003 & 36.20 & 0.27 & 3604.00 & 34890.22 & 1.00 & 0.96 & 0.97 \\
35119 & 106321 & 2003 & 7.20 & 0.35 & 722.00 & 6776.89 & 1.00 & 0.94 & 0.94 \\
30271 & 105723 & 2003 & 725.40 & 0.32 & 72560.00 & 668221.47 & 1.00 & 0.92 & 0.92 \\
8823 & 101100 & 2003 & 369.40 & 0.08 & 40968.00 & 418364.33 & 0.90 & 1.13 & 1.02 \\
33540 & 106147 & 2003 & 81.30 & 0.47 & 8060.00 & 76228.38 & 1.01 & 0.94 & 0.95 \\
35223 & 106335 & 2003 & 139.40 & 0.33 & 13969.00 & 137966.42 & 1.00 & 0.99 & 0.99 \\
47610 & 215687 & 2003 & 123.40 & 0.39 & 12069.00 & 106941.80 & 1.02 & 0.87 & 0.89 \\
30141 & 105702 & 2003 & 25.50 & 0.31 & 2490.00 & 24401.65 & 1.02 & 0.96 & 0.98 \\
33482 & 106140 & 2003 & 1589.00 & 0.34 & 142934.00 & 1517204.33 & 1.11 & 0.95 & 1.06 \\
35281 & 106344 & 2003 & 956.00 & 0.31 & 87589.00 & 857877.96 & 1.09 & 0.90 & 0.98 \\
53037 & 337653 & 2003 & 96.00 & 0.05 & 7761.00 & 82922.60 & 1.24 & 0.86 & 1.07 \\
30049 & 105679 & 2003 & 405.70 & 0.50 & 40764.00 & 405768.49 & 1.00 & 1.00 & 1.00 \\
53058 & 337871 & 2003 & 140.00 & 0.45 & 16652.00 & 154940.03 & 0.84 & 1.11 & 0.93 \\
35308 & 106345 & 2003 & 898.00 & 0.67 & 89384.00 & 890873.25 & 1.00 & 0.99 & 1.00 \\
35323 & 106347 & 2003 & 93.60 & 0.40 & 7123.00 & 82404.76 & 1.31 & 0.88 & 1.16 \\
30021 & 105678 & 2003 & 39.60 & 0.29 & 3342.00 & 34707.76 & 1.18 & 0.88 & 1.04 \\
30012 & 105677 & 2003 & 58.50 & 0.33 & 5856.00 & 57888.93 & 1.00 & 0.99 & 0.99 \\
9634 & 101160 & 2003 & 417.90 & 0.33 & 41901.00 & 410357.23 & 1.00 & 0.98 & 0.98 \\
16111 & 102080 & 2003 & 2972.40 & 0.41 & 253976.00 & 2736206.54 & 1.17 & 0.92 & 1.08 \\
30004 & 105676 & 2003 & 378.60 & 0.27 & 38048.00 & 353164.64 & 1.00 & 0.93 & 0.93 \\
10570 & 101299 & 2003 & 2100.00 & 0.28 & 210124.00 & 2062307.53 & 1.00 & 0.98 & 0.98 \\
47625 & 215696 & 2003 & 26.60 & 0.50 & 2661.00 & 25948.65 & 1.00 & 0.98 & 0.98 \\
35069 & 106317 & 2003 & 158.00 & 0.26 & 15794.00 & 156407.40 & 1.00 & 0.99 & 0.99 \\
53020 & 337150 & 2003 & 21.90 & 0.29 & 2195.00 & 21941.66 & 1.00 & 1.00 & 1.00 \\
33508 & 106143 & 2003 & 113.30 & 0.38 & 13129.00 & 110163.80 & 0.86 & 0.97 & 0.84 \\
52995 & 336593 & 2003 & 30.00 & 0.27 & 3000.00 & 29400.52 & 1.00 & 0.98 & 0.98 \\
30128 & 105701 & 2003 & 755.00 & 0.25 & 75452.00 & 754120.22 & 1.00 & 1.00 & 1.00 \\
14776 & 101913 & 2003 & 64.40 & 0.32 & 6468.00 & 63888.63 & 1.00 & 0.99 & 0.99 \\
52999 & 336942 & 2003 & 45.30 & 0.03 & 4494.00 & 43681.03 & 1.01 & 0.96 & 0.97 \\
30118 & 105700 & 2003 & 197.90 & 0.27 & 19781.00 & 193632.70 & 1.00 & 0.98 & 0.98 \\
30110 & 105694 & 2003 & 51.80 & 0.42 & 5189.00 & 50020.71 & 1.00 & 0.97 & 0.96 \\
5986 & 100817 & 2003 & 184.10 & 0.23 & 18425.00 & 183491.62 & 1.00 & 1.00 & 1.00 \\
51312 & 240499 & 2003 & 45.70 & 0.93 & 2830.00 & 26152.30 & 1.61 & 0.57 & 0.92 \\
16080 & 102079 & 2003 & 430.40 & 0.34 & 38079.00 & 415047.17 & 1.13 & 0.96 & 1.09 \\
51290 & 240498 & 2003 & 74.00 & 0.26 & 7029.00 & 70190.75 & 1.05 & 0.95 & 1.00 \\
198 & 100018 & 2003 & 137.70 & 0.44 & 13837.00 & 132775.85 & 1.00 & 0.96 & 0.96 \\
10421 & 101285 & 2003 & 720.90 & 0.38 & 72034.00 & 692680.40 & 1.00 & 0.96 & 0.96 \\
46664 & 200274 & 2003 & 20.30 & 0.25 & 2032.00 & 19169.06 & 1.00 & 0.94 & 0.94 \\
34888 & 106284 & 2003 & 317.00 & 0.35 & 31760.00 & 297608.61 & 1.00 & 0.94 & 0.94 \\
6382 & 100856 & 2003 & 121.40 & 0.36 & 11450.00 & 111793.13 & 1.06 & 0.92 & 0.98 \\
30556 & 105769 & 2003 & 34.20 & 0.42 & 3840.00 & 40026.37 & 0.89 & 1.17 & 1.04 \\
9665 & 101161 & 2003 & 1196.70 & 0.25 & 120880.00 & 1193190.73 & 0.99 & 1.00 & 0.99 \\
51594 & 240533 & 2003 & 2.90 & 0.41 & 293.00 & 2660.49 & 0.99 & 0.92 & 0.91 \\
34929 & 106293 & 2003 & 16.80 & 0.22 & 2682.00 & 26699.07 & 0.63 & 1.59 & 1.00 \\
34936 & 106294 & 2003 & 167.10 & 0.37 & 16369.00 & 169749.84 & 1.02 & 1.02 & 1.04 \\
30528 & 105763 & 2003 & 327.80 & 0.38 & 27424.00 & 310481.05 & 1.20 & 0.95 & 1.13 \\
15947 & 102061 & 2003 & 1916.30 & 0.35 & 184235.00 & 1696664.38 & 1.04 & 0.89 & 0.92 \\
46652 & 200273 & 2003 & 31.80 & 0.38 & 3177.00 & 26280.46 & 1.00 & 0.83 & 0.83 \\
34868 & 106283 & 2003 & 269.80 & 0.44 & 26922.00 & 250560.88 & 1.00 & 0.93 & 0.93 \\
34963 & 106297 & 2003 & 28.50 & 0.47 & 2250.00 & 24518.24 & 1.27 & 0.86 & 1.09 \\
46671 & 200276 & 2003 & 16.20 & 0.38 & 1625.00 & 15752.21 & 1.00 & 0.97 & 0.97 \\
52855 & 330794 & 2003 & 56.40 & 0.25 & 5624.00 & 56452.51 & 1.00 & 1.00 & 1.00 \\
34795 & 106277 & 2003 & 217.50 & 0.31 & 21931.00 & 214823.82 & 0.99 & 0.99 & 0.98 \\
30655 & 105781 & 2003 & 519.00 & 0.49 & 55228.00 & 521858.23 & 0.94 & 1.01 & 0.94 \\
14808 & 101914 & 2003 & 40.70 & 0.26 & 4104.00 & 40593.06 & 0.99 & 1.00 & 0.99 \\
51617 & 240535 & 2003 & 16.30 & 0.31 & 1571.00 & 15704.59 & 1.04 & 0.96 & 1.00 \\
96666 & 611002 & 2003 & 3274.70 & 0.23 & 314606.00 & 3214830.61 & 1.04 & 0.98 & 1.02 \\
30646 & 105780 & 2003 & 288.80 & 0.26 & 28978.00 & 277965.46 & 1.00 & 0.96 & 0.96 \\
34818 & 106278 & 2003 & 160.20 & 0.58 & 15920.00 & 153489.75 & 1.01 & 0.96 & 0.96 \\
52831 & 330728 & 2003 & 9.30 & 0.20 & 956.00 & 8900.98 & 0.97 & 0.96 & 0.93 \\
15917 & 102059 & 2003 & 932.70 & 0.38 & 94889.00 & 902088.95 & 0.98 & 0.97 & 0.95 \\
52181 & 302627 & 2003 & 107.40 & 0.38 & 9064.00 & 98974.10 & 1.18 & 0.92 & 1.09 \\
34830 & 106281 & 2003 & 2.80 & 0.20 & 273.00 & 2731.11 & 1.03 & 0.98 & 1.00 \\
30619 & 105779 & 2003 & 1668.30 & 0.25 & 166957.00 & 1586284.62 & 1.00 & 0.95 & 0.95 \\
46681 & 200277 & 2003 & 5.70 & 0.35 & 573.00 & 5678.75 & 0.99 & 1.00 & 0.99 \\
33565 & 106149 & 2003 & 230.30 & 0.33 & 22982.00 & 218625.91 & 1.00 & 0.95 & 0.95 \\
34841 & 106282 & 2003 & 609.90 & 0.32 & 61398.00 & 562152.14 & 0.99 & 0.92 & 0.92 \\
30595 & 105775 & 2003 & 1354.10 & 0.26 & 135381.00 & 1289823.22 & 1.00 & 0.95 & 0.95 \\
29992 & 105665 & 2003 & 63.10 & 0.26 & 6314.00 & 61469.65 & 1.00 & 0.97 & 0.97 \\
32134 & 105984 & 2003 & 44.80 & 0.21 & 4234.00 & 42972.82 & 1.06 & 0.96 & 1.01 \\
9186 & 101116 & 2003 & 856.80 & 0.21 & 83729.00 & 898259.13 & 1.02 & 1.05 & 1.07 \\
35036 & 106309 & 2003 & 1298.40 & 0.35 & 101253.00 & 1006376.97 & 1.28 & 0.78 & 0.99 \\
30398 & 105753 & 2003 & 59.70 & 0.24 & 7220.00 & 70814.26 & 0.83 & 1.19 & 0.98 \\
15985 & 102062 & 2003 & 745.80 & 0.48 & 71551.00 & 698800.18 & 1.04 & 0.94 & 0.98 \\
30390 & 105748 & 2003 & 55.80 & 0.25 & 5490.00 & 43957.86 & 1.02 & 0.79 & 0.80 \\
46623 & 200266 & 2003 & 4.80 & 0.33 & 381.00 & 3369.60 & 1.26 & 0.70 & 0.88 \\
30376 & 105746 & 2003 & 260.40 & 0.30 & 25994.00 & 257946.62 & 1.00 & 0.99 & 0.99 \\
32154 & 105990 & 2003 & 337.00 & 0.38 & 38007.00 & 368922.95 & 0.89 & 1.09 & 0.97 \\
52952 & 335933 & 2003 & 32.80 & -0.02 & 3326.00 & 31480.73 & 0.99 & 0.96 & 0.95 \\
30359 & 105741 & 2003 & 250.40 & 0.27 & 24795.00 & 234696.15 & 1.01 & 0.94 & 0.95 \\
46615 & 200264 & 2003 & 10.90 & 0.13 & 1091.00 & 9738.40 & 1.00 & 0.89 & 0.89 \\
162 & 100016 & 2003 & 98.50 & 0.22 & 10069.00 & 97966.89 & 0.98 & 0.99 & 0.97 \\
46614 & 200263 & 2003 & 44.60 & 0.28 & 4480.00 & 44472.00 & 1.00 & 1.00 & 0.99 \\
47584 & 215413 & 2003 & 11.60 & 0.54 & 1117.00 & 10809.64 & 1.04 & 0.93 & 0.97 \\
30500 & 105762 & 2003 & 348.90 & 0.37 & 29651.00 & 338455.79 & 1.18 & 0.97 & 1.14 \\
6751 & 100947 & 2003 & 490.80 & 0.25 & 24901.00 & 246730.85 & 1.97 & 0.50 & 0.99 \\
10386 & 101284 & 2003 & 1870.90 & 0.40 & 186449.00 & 1765775.26 & 1.00 & 0.94 & 0.95 \\
46645 & 200268 & 2003 & 87.30 & 0.28 & 8719.00 & 86124.13 & 1.00 & 0.99 & 0.99 \\
10356 & 101283 & 2003 & 3243.50 & 0.29 & 283316.00 & 2626863.74 & 1.14 & 0.81 & 0.93 \\
15340 & 101987 & 2003 & 1261.80 & 0.42 & 108604.00 & 1184512.67 & 1.16 & 0.94 & 1.09 \\
46641 & 200267 & 2003 & 21.70 & 0.39 & 2159.00 & 18948.13 & 1.01 & 0.87 & 0.88 \\
30472 & 105761 & 2003 & 959.00 & 0.26 & 83853.00 & 935857.24 & 1.14 & 0.98 & 1.12 \\
32148 & 105987 & 2003 & 70.70 & 0.27 & 6580.00 & 68263.66 & 1.07 & 0.97 & 1.04 \\
34996 & 106305 & 2003 & 265.10 & 0.51 & 20263.00 & 208978.17 & 1.31 & 0.79 & 1.03 \\
35023 & 106306 & 2003 & 18.50 & 0.22 & 1694.00 & 17861.52 & 1.09 & 0.97 & 1.05 \\
30444 & 105760 & 2003 & 642.50 & 0.29 & 54606.00 & 610074.01 & 1.18 & 0.95 & 1.12 \\
56595 & 400230 & 2003 & 50.80 & 0.36 & 4562.00 & 45749.68 & 1.11 & 0.90 & 1.00 \\
30424 & 105757 & 2003 & 112.50 & 0.36 & 11314.00 & 94743.29 & 0.99 & 0.84 & 0.84 \\
51450 & 240521 & 2003 & 4.70 & 0.22 & 469.00 & 4093.55 & 1.00 & 0.87 & 0.87 \\
35336 & 106348 & 2003 & 103.70 & 0.34 & 8523.00 & 102750.80 & 1.22 & 0.99 & 1.21 \\
29974 & 105664 & 2003 & 331.20 & 0.63 & 28490.00 & 281663.69 & 1.16 & 0.85 & 0.99 \\
74912 & 601197 & 2003 & 33.30 & 0.33 & 3302.00 & 33002.62 & 1.01 & 0.99 & 1.00 \\
10771 & 101330 & 2003 & 2602.50 & 0.28 & 260219.00 & 2553508.59 & 1.00 & 0.98 & 0.98 \\
29490 & 105598 & 2003 & 364.60 & 0.28 & 33644.00 & 358179.31 & 1.08 & 0.98 & 1.06 \\
14710 & 101911 & 2003 & 2369.00 & 0.44 & 190072.00 & 1901173.81 & 1.25 & 0.80 & 1.00 \\
35738 & 106394 & 2003 & 66.20 & 0.23 & 6657.00 & 61920.28 & 0.99 & 0.94 & 0.93 \\
50822 & 240453 & 2003 & 51.40 & 0.32 & 5108.00 & 44824.57 & 1.01 & 0.87 & 0.88 \\
35749 & 106398 & 2003 & 5.50 & 0.30 & 537.00 & 5112.29 & 1.02 & 0.93 & 0.95 \\
16286 & 102121 & 2003 & 65.70 & 0.28 & 6535.00 & 62816.28 & 1.01 & 0.96 & 0.96 \\
29464 & 105597 & 2003 & 95.30 & 0.24 & 9533.00 & 92493.64 & 1.00 & 0.97 & 0.97 \\
46486 & 200248 & 2003 & 79.40 & 0.34 & 7003.00 & 70625.91 & 1.13 & 0.89 & 1.01 \\
35763 & 106401 & 2003 & 1425.70 & 0.39 & 121054.00 & 1191719.15 & 1.18 & 0.84 & 0.98 \\
29442 & 105595 & 2003 & 13.10 & 0.20 & 1025.00 & 10643.22 & 1.28 & 0.81 & 1.04 \\
50760 & 240448 & 2003 & 33.10 & 0.01 & 3305.00 & 31993.31 & 1.00 & 0.97 & 0.97 \\
46482 & 200247 & 2003 & 2.20 & 0.75 & 211.00 & 2082.71 & 1.04 & 0.95 & 0.99 \\
35789 & 106402 & 2003 & 54.10 & 0.29 & 4194.00 & 43908.65 & 1.29 & 0.81 & 1.05 \\
8682 & 101094 & 2003 & 823.10 & 0.35 & 77717.00 & 700260.10 & 1.06 & 0.85 & 0.90 \\
33422 & 106135 & 2003 & 57.70 & 0.26 & 5642.00 & 55777.73 & 1.02 & 0.97 & 0.99 \\
6412 & 100864 & 2003 & 458.10 & 0.25 & 39306.00 & 327154.36 & 1.17 & 0.71 & 0.83 \\
35706 & 106391 & 2003 & 75.30 & 0.40 & 7530.00 & 74535.62 & 1.00 & 0.99 & 0.99 \\
16260 & 102105 & 2003 & 135.90 & 0.35 & 10936.00 & 125417.90 & 1.24 & 0.92 & 1.15 \\
53133 & 340902 & 2003 & 4.40 & 0.04 & 444.00 & 4422.92 & 0.99 & 1.01 & 1.00 \\
35642 & 106381 & 2003 & 4.70 & 0.38 & 466.00 & 4354.32 & 1.01 & 0.93 & 0.93 \\
29579 & 105616 & 2003 & 29.40 & 0.34 & 2764.00 & 27639.85 & 1.06 & 0.94 & 1.00 \\
14956 & 101925 & 2003 & 6492.00 & 0.25 & 618171.00 & 6181708.59 & 1.05 & 0.95 & 1.00 \\
16245 & 102104 & 2003 & 512.00 & 0.35 & 41296.00 & 504626.94 & 1.24 & 0.99 & 1.22 \\
10735 & 101320 & 2003 & 71.30 & 0.25 & 7059.00 & 64165.81 & 1.01 & 0.90 & 0.91 \\
50832 & 240458 & 2003 & 437.20 & 0.35 & 28952.00 & 257242.18 & 1.51 & 0.59 & 0.89 \\
35676 & 106386 & 2003 & 528.90 & 0.29 & 60566.00 & 598699.33 & 0.87 & 1.13 & 0.99 \\
29550 & 105611 & 2003 & 2090.10 & 0.39 & 209773.00 & 1861686.86 & 1.00 & 0.89 & 0.89 \\
29538 & 105610 & 2003 & 358.50 & 0.31 & 35423.00 & 347581.62 & 1.01 & 0.97 & 0.98 \\
5874 & 100809 & 2003 & 1517.50 & 0.25 & 151322.00 & 1465227.03 & 1.00 & 0.97 & 0.97 \\
29531 & 105607 & 2003 & 3.60 & 0.31 & 365.00 & 3050.65 & 0.99 & 0.85 & 0.84 \\
32241 & 106007 & 2003 & 5253.00 & 0.40 & 483256.00 & 3899930.87 & 1.09 & 0.74 & 0.81 \\
47655 & 216504 & 2003 & 20.50 & 0.59 & 2771.00 & 17444.10 & 0.74 & 0.85 & 0.63 \\
9615 & 101158 & 2003 & 426.90 & 0.28 & 43418.00 & 381634.00 & 0.98 & 0.89 & 0.88 \\
53236 & 342127 & 2003 & 65.90 & 0.33 & 7355.00 & 72464.76 & 0.90 & 1.10 & 0.99 \\
50851 & 240459 & 2003 & 65.80 & 0.19 & 6680.00 & 62316.47 & 0.99 & 0.95 & 0.93 \\
35807 & 106413 & 2003 & 3149.70 & 0.32 & 407449.00 & 4073594.48 & 0.77 & 1.29 & 1.00 \\
57244 & 400323 & 2003 & 408.20 & 0.55 & 39166.00 & 356519.94 & 1.04 & 0.87 & 0.91 \\
35876 & 106421 & 2003 & 52.60 & 0.36 & 5241.00 & 51006.99 & 1.00 & 0.97 & 0.97 \\
29332 & 105587 & 2003 & 178.80 & 0.78 & 18596.00 & 166009.93 & 0.96 & 0.93 & 0.89 \\
50707 & 240441 & 2003 & 1090.60 & 0.27 & 98947.00 & 1059682.18 & 1.10 & 0.97 & 1.07 \\
35894 & 106422 & 2003 & 157.30 & 0.33 & 13721.00 & 150177.71 & 1.15 & 0.95 & 1.09 \\
35900 & 106424 & 2003 & 459.50 & 0.34 & 45505.00 & 448451.01 & 1.01 & 0.98 & 0.99 \\
5827 & 100804 & 2003 & 3994.10 & 0.20 & 398662.00 & 3741171.56 & 1.00 & 0.94 & 0.94 \\
32269 & 106008 & 2003 & 290.60 & 0.55 & 28091.00 & 252762.90 & 1.03 & 0.87 & 0.90 \\
10833 & 101334 & 2003 & 602.60 & 0.57 & 57368.00 & 514811.98 & 1.05 & 0.85 & 0.90 \\
50693 & 240440 & 2003 & 429.70 & 0.24 & 34392.00 & 365408.45 & 1.25 & 0.85 & 1.06 \\
35926 & 106434 & 2003 & 803.60 & 0.24 & 79068.00 & 778115.69 & 1.02 & 0.97 & 0.98 \\
29280 & 105581 & 2003 & 255.00 & 0.61 & 25342.00 & 237480.72 & 1.01 & 0.93 & 0.94 \\
29267 & 105574 & 2003 & 220.60 & 0.40 & 18744.00 & 209110.80 & 1.18 & 0.95 & 1.12 \\
35936 & 106441 & 2003 & 775.50 & 0.66 & 78258.00 & 772276.71 & 0.99 & 1.00 & 0.99 \\
46430 & 200244 & 2003 & 8.40 & 0.78 & 802.00 & 7992.18 & 1.05 & 0.95 & 1.00 \\
29416 & 105594 & 2003 & 4.00 & 0.34 & 326.00 & 3822.91 & 1.23 & 0.96 & 1.17 \\
35868 & 106420 & 2003 & 74.70 & 0.47 & 7488.00 & 70818.52 & 1.00 & 0.95 & 0.95 \\
47462 & 211485 & 2003 & 60.10 & 0.25 & 5457.00 & 59986.57 & 1.10 & 1.00 & 1.10 \\
33415 & 106133 & 2003 & 171.20 & 0.31 & 15332.00 & 169692.55 & 1.12 & 0.99 & 1.11 \\
29409 & 105593 & 2003 & 67.10 & 0.25 & 6665.00 & 66652.74 & 1.01 & 0.99 & 1.00 \\
6772 & 100953 & 2003 & 159.90 & 0.81 & 11799.00 & 110933.90 & 1.36 & 0.69 & 0.94 \\
29402 & 105592 & 2003 & 284.80 & 0.30 & 28549.00 & 279040.99 & 1.00 & 0.98 & 0.98 \\
16311 & 102124 & 2003 & 1989.80 & 0.27 & 194920.00 & 1934784.65 & 1.02 & 0.97 & 0.99 \\
52475 & 302944 & 2003 & 211.10 & 0.49 & 19267.00 & 197536.74 & 1.10 & 0.94 & 1.03 \\
35833 & 106415 & 2003 & 254.30 & 0.48 & 24670.00 & 237808.54 & 1.03 & 0.94 & 0.96 \\
10803 & 101331 & 2003 & 75.10 & 0.27 & 7505.00 & 74087.32 & 1.00 & 0.99 & 0.99 \\
46463 & 200246 & 2003 & 142.90 & 0.29 & 14305.00 & 136566.03 & 1.00 & 0.96 & 0.95 \\
35849 & 106418 & 2003 & 1360.20 & 0.28 & 137767.00 & 1330435.26 & 0.99 & 0.98 & 0.97 \\
50751 & 240447 & 2003 & 3.00 & 0.57 & 291.00 & 2827.55 & 1.03 & 0.94 & 0.97 \\
29364 & 105589 & 2003 & 271.20 & 0.41 & 27181.00 & 267480.21 & 1.00 & 0.99 & 0.98 \\
29358 & 105588 & 2003 & 7.70 & 0.17 & 775.00 & 6211.54 & 0.99 & 0.81 & 0.80 \\
46453 & 200245 & 2003 & 23.80 & 0.28 & 2355.00 & 23535.96 & 1.01 & 0.99 & 1.00 \\
50731 & 240444 & 2003 & 54.60 & 0.24 & 4437.00 & 41600.22 & 1.23 & 0.76 & 0.94 \\
10296 & 101278 & 2003 & 123.20 & 0.41 & 10964.00 & 103230.09 & 1.12 & 0.84 & 0.94 \\
50865 & 240462 & 2003 & 6.60 & 0.07 & 602.00 & 6344.19 & 1.10 & 0.96 & 1.05 \\
35615 & 106380 & 2003 & 324.20 & 0.38 & 32476.00 & 323272.18 & 1.00 & 1.00 & 1.00 \\
29868 & 105655 & 2003 & 967.50 & 0.37 & 79084.00 & 893884.65 & 1.22 & 0.92 & 1.13 \\
16154 & 102087 & 2003 & 697.40 & 0.22 & 68498.00 & 670869.51 & 1.02 & 0.96 & 0.98 \\
35427 & 106360 & 2003 & 394.10 & 0.40 & 28713.00 & 295285.69 & 1.37 & 0.75 & 1.03 \\
29851 & 105654 & 2003 & 73.60 & 0.31 & 6849.00 & 76304.95 & 1.07 & 1.04 & 1.11 \\
53102 & 338393 & 2003 & 12.70 & 0.24 & 1273.00 & 12726.33 & 1.00 & 1.00 & 1.00 \\
10644 & 101302 & 2003 & 1601.20 & 0.24 & 159369.00 & 1519801.79 & 1.00 & 0.95 & 0.95 \\
35454 & 106361 & 2003 & 163.30 & 0.34 & 16393.00 & 158216.17 & 1.00 & 0.97 & 0.97 \\
29825 & 105652 & 2003 & 397.70 & 0.26 & 38085.00 & 377334.55 & 1.04 & 0.95 & 0.99 \\
16173 & 102089 & 2003 & 330.40 & 0.26 & 28460.00 & 237051.56 & 1.16 & 0.72 & 0.83 \\
51070 & 240481 & 2003 & 31.20 & 0.25 & 3121.00 & 28556.62 & 1.00 & 0.92 & 0.91 \\
74898 & 601190 & 2003 & 13.10 & 0.24 & 1266.00 & 12434.88 & 1.03 & 0.95 & 0.98 \\
35476 & 106363 & 2003 & 176.30 & 0.29 & 17640.00 & 175159.60 & 1.00 & 0.99 & 0.99 \\
51059 & 240480 & 2003 & 21.20 & 0.25 & 2119.00 & 20674.10 & 1.00 & 0.98 & 0.98 \\
32213 & 106000 & 2003 & 729.00 & 0.44 & 72805.00 & 723063.12 & 1.00 & 0.99 & 0.99 \\
46547 & 200252 & 2003 & 79.00 & 0.37 & 7916.00 & 78182.42 & 1.00 & 0.99 & 0.99 \\
35419 & 106359 & 2003 & 186.20 & 0.26 & 18643.00 & 185151.21 & 1.00 & 0.99 & 0.99 \\
35484 & 106364 & 2003 & 11.50 & 0.21 & 1157.00 & 10759.14 & 0.99 & 0.94 & 0.93 \\
35412 & 106358 & 2003 & 2.70 & 0.33 & 266.00 & 2317.52 & 1.02 & 0.86 & 0.87 \\
51154 & 240487 & 2003 & 4.80 & 0.63 & 463.00 & 4596.62 & 1.04 & 0.96 & 0.99 \\
29958 & 105662 & 2003 & 111.20 & 0.18 & 11589.00 & 113836.76 & 0.96 & 1.02 & 0.98 \\
29946 & 105659 & 2003 & 211.80 & 0.23 & 20946.00 & 209361.80 & 1.01 & 0.99 & 1.00 \\
46562 & 200254 & 2003 & 257.40 & 0.15 & 29873.00 & 223482.01 & 0.86 & 0.87 & 0.75 \\
46555 & 200253 & 2003 & 5.50 & 0.54 & 546.00 & 5105.77 & 1.01 & 0.93 & 0.94 \\
29931 & 105658 & 2003 & 98.70 & 0.83 & 9938.00 & 97756.85 & 0.99 & 0.99 & 0.98 \\
5945 & 100812 & 2003 & 250.00 & 0.28 & 25498.00 & 251013.25 & 0.98 & 1.00 & 0.98 \\
233 & 100019 & 2003 & 11165.10 & 0.45 & 1117469.00 & 9061087.41 & 1.00 & 0.81 & 0.81 \\
51132 & 240486 & 2003 & 3.60 & 0.34 & 377.00 & 3451.80 & 0.95 & 0.96 & 0.92 \\
35352 & 106353 & 2003 & 454.80 & 0.24 & 43911.00 & 439206.08 & 1.04 & 0.97 & 1.00 \\
47633 & 215952 & 2003 & 706.30 & 0.23 & 71135.00 & 627062.79 & 0.99 & 0.89 & 0.88 \\
35386 & 106356 & 2003 & 16.80 & 0.37 & 1685.00 & 16328.77 & 1.00 & 0.97 & 0.97 \\
29897 & 105656 & 2003 & 145.00 & 0.44 & 13931.00 & 139335.15 & 1.04 & 0.96 & 1.00 \\
51094 & 240482 & 2003 & 2.80 & 0.23 & 231.00 & 2433.79 & 1.21 & 0.87 & 1.05 \\
29608 & 105623 & 2003 & 30.50 & 0.44 & 2799.00 & 28269.01 & 1.09 & 0.93 & 1.01 \\
29796 & 105647 & 2003 & 561.50 & 0.50 & 51849.00 & 509979.89 & 1.08 & 0.91 & 0.98 \\
51034 & 240477 & 2003 & 1.20 & 0.12 & 99.00 & 990.70 & 1.21 & 0.83 & 1.00 \\
46509 & 200249 & 2003 & 146.30 & 0.73 & 14147.00 & 140849.86 & 1.03 & 0.96 & 1.00 \\
35557 & 106375 & 2003 & 22.90 & 0.26 & 2291.00 & 21218.82 & 1.00 & 0.93 & 0.93 \\
29685 & 105635 & 2003 & 30.80 & 0.42 & 3095.00 & 30880.51 & 1.00 & 1.00 & 1.00 \\
33449 & 106136 & 2003 & 115.20 & 0.49 & 11671.00 & 109203.13 & 0.99 & 0.95 & 0.94 \\
29671 & 105632 & 2003 & 50.00 & 0.54 & 7091.00 & 73195.09 & 0.71 & 1.46 & 1.03 \\
35582 & 106376 & 2003 & 5.80 & 0.64 & 540.00 & 4884.64 & 1.07 & 0.84 & 0.90 \\
29661 & 105631 & 2003 & 5.00 & 0.32 & 590.00 & 6074.59 & 0.85 & 1.21 & 1.03 \\
5906 & 100811 & 2003 & 906.20 & 0.26 & 90624.00 & 868612.68 & 1.00 & 0.96 & 0.96 \\
47646 & 216438 & 2003 & 551.60 & 0.26 & 49020.00 & 536604.89 & 1.13 & 0.97 & 1.09 \\
10706 & 101312 & 2003 & 9005.60 & 0.37 & 896137.00 & 8571940.18 & 1.00 & 0.95 & 0.96 \\
29649 & 105630 & 2003 & 5.00 & 0.41 & 572.00 & 5845.97 & 0.87 & 1.17 & 1.02 \\
35605 & 106379 & 2003 & 178.40 & 0.27 & 17800.00 & 173067.11 & 1.00 & 0.97 & 0.97 \\
29620 & 105627 & 2003 & 993.60 & 0.39 & 98657.00 & 947707.43 & 1.01 & 0.95 & 0.96 \\
74896 & 601189 & 2003 & 16.30 & 0.43 & 1395.00 & 14153.61 & 1.17 & 0.87 & 1.01 \\
35549 & 106372 & 2003 & 5.20 & 0.18 & 716.00 & 6043.76 & 0.73 & 1.16 & 0.84 \\
50971 & 240474 & 2003 & 16.10 & 0.41 & 1352.00 & 13644.32 & 1.19 & 0.85 & 1.01 \\
46524 & 200251 & 2003 & 35.80 & 0.84 & 3674.00 & 33601.33 & 0.97 & 0.94 & 0.91 \\
51013 & 240476 & 2003 & 164.30 & 0.25 & 15817.00 & 167622.89 & 1.04 & 1.02 & 1.06 \\
35502 & 106367 & 2003 & 126.50 & 0.28 & 12537.00 & 123414.51 & 1.01 & 0.98 & 0.98 \\
29767 & 105645 & 2003 & 6398.00 & 0.50 & 595999.00 & 6085132.53 & 1.07 & 0.95 & 1.02 \\
53105 & 339611 & 2003 & 15.10 & 0.43 & 1468.00 & 14305.94 & 1.03 & 0.95 & 0.97 \\
46518 & 200250 & 2003 & 337.60 & 0.31 & 27510.00 & 304736.18 & 1.23 & 0.90 & 1.11 \\
29757 & 105644 & 2003 & 78.40 & 0.29 & 7876.00 & 77773.00 & 1.00 & 0.99 & 0.99 \\
47255 & 200344 & 2003 & 9658.80 & 0.30 & 855930.00 & 8572076.49 & 1.13 & 0.89 & 1.00 \\
10674 & 101307 & 2003 & 819.50 & 0.37 & 81482.00 & 728677.90 & 1.01 & 0.89 & 0.89 \\
35514 & 106369 & 2003 & 126.40 & 0.45 & 12467.00 & 119267.78 & 1.01 & 0.94 & 0.96 \\
29735 & 105643 & 2003 & 1078.70 & 0.29 & 108443.00 & 1068274.59 & 0.99 & 0.99 & 0.99 \\
52482 & 302964 & 2003 & 49.20 & 0.36 & 3899.00 & 41742.76 & 1.26 & 0.85 & 1.07 \\
50992 & 240475 & 2003 & 13.20 & 0.36 & 1213.00 & 11653.83 & 1.09 & 0.88 & 0.96 \\
15304 & 101982 & 2003 & 281.40 & 0.25 & 28161.00 & 276537.20 & 1.00 & 0.98 & 0.98 \\
35522 & 106370 & 2003 & 49.40 & 0.44 & 5166.00 & 47757.17 & 0.96 & 0.97 & 0.92 \\
16204 & 102090 & 2003 & 5495.40 & 0.28 & 512327.00 & 5157700.82 & 1.07 & 0.94 & 1.01 \\
34768 & 106276 & 2003 & 310.70 & 0.47 & 31251.00 & 312430.50 & 0.99 & 1.01 & 1.00 \\
46695 & 200279 & 2003 & 91.80 & 0.35 & 9539.00 & 91041.05 & 0.96 & 0.99 & 0.95 \\
30686 & 105783 & 2003 & 3794.60 & 0.34 & 289776.00 & 3121019.30 & 1.31 & 0.82 & 1.08 \\
33627 & 106157 & 2003 & 877.90 & 0.49 & 77312.00 & 870985.70 & 1.14 & 0.99 & 1.13 \\
46976 & 200324 & 2003 & 317.90 & 0.40 & 30679.00 & 306785.85 & 1.04 & 0.97 & 1.00 \\
52571 & 303130 & 2003 & 70.30 & 0.33 & 5762.00 & 63841.97 & 1.22 & 0.91 & 1.11 \\
34001 & 106197 & 2003 & 237.40 & 0.39 & 19678.00 & 236786.57 & 1.21 & 1.00 & 1.20 \\
34028 & 106198 & 2003 & 704.10 & 0.36 & 70444.00 & 657645.09 & 1.00 & 0.93 & 0.93 \\
31532 & 105905 & 2003 & 16.40 & 0.45 & 1653.00 & 15839.73 & 0.99 & 0.97 & 0.96 \\
15591 & 102007 & 2003 & 3186.90 & 0.24 & 299008.00 & 3181120.09 & 1.07 & 1.00 & 1.06 \\
9926 & 101212 & 2003 & 386.70 & 0.22 & 40929.00 & 429381.23 & 0.94 & 1.11 & 1.05 \\
34055 & 106199 & 2003 & 34.20 & 0.33 & 5563.00 & 52779.95 & 0.61 & 1.54 & 0.95 \\
46954 & 200322 & 2003 & 30.90 & 0.32 & 3077.00 & 28519.56 & 1.00 & 0.92 & 0.93 \\
34063 & 106200 & 2003 & 718.80 & 0.32 & 73118.00 & 764600.83 & 0.98 & 1.06 & 1.05 \\
52594 & 303140 & 2003 & 644.50 & 0.34 & 63509.00 & 633704.80 & 1.01 & 0.98 & 1.00 \\
52021 & 300777 & 2003 & 13.20 & 0.37 & 1340.00 & 12490.00 & 0.99 & 0.95 & 0.93 \\
31564 & 105909 & 2003 & 55.00 & 0.24 & 5505.00 & 54778.41 & 1.00 & 1.00 & 1.00 \\
46932 & 200319 & 2003 & 74.70 & 0.44 & 7457.00 & 71973.34 & 1.00 & 0.96 & 0.97 \\
15576 & 102005 & 2003 & 560.70 & 0.28 & 52484.00 & 510363.53 & 1.07 & 0.91 & 0.97 \\
47006 & 200327 & 2003 & 7.00 & 0.53 & 691.00 & 6199.90 & 1.01 & 0.89 & 0.90 \\
33924 & 106189 & 2003 & 357.90 & 0.21 & 35615.00 & 354451.10 & 1.00 & 0.99 & 1.00 \\
46984 & 200325 & 2003 & 6.50 & 0.04 & 647.00 & 5950.95 & 1.00 & 0.92 & 0.92 \\
31624 & 105918 & 2003 & 836.50 & 0.45 & 83706.00 & 828450.64 & 1.00 & 0.99 & 0.99 \\
33931 & 106192 & 2003 & 1854.50 & 0.22 & 178705.00 & 1640954.78 & 1.04 & 0.88 & 0.92 \\
31614 & 105917 & 2003 & 110.00 & 0.32 & 11033.00 & 103552.89 & 1.00 & 0.94 & 0.94 \\
6234 & 100831 & 2003 & 133.80 & 0.26 & 13378.00 & 131511.57 & 1.00 & 0.98 & 0.98 \\
9891 & 101200 & 2003 & 27.70 & 0.32 & 2774.00 & 26608.73 & 1.00 & 0.96 & 0.96 \\
96693 & 611006 & 2003 & 42.20 & 0.38 & 4585.00 & 45063.97 & 0.92 & 1.07 & 0.98 \\
32005 & 105973 & 2003 & 57.30 & 0.56 & 5911.00 & 55883.21 & 0.97 & 0.98 & 0.95 \\
33958 & 106193 & 2003 & 25.20 & 0.28 & 2453.00 & 24159.52 & 1.03 & 0.96 & 0.98 \\
56347 & 400197 & 2003 & 11.50 & 0.31 & 1105.00 & 11351.48 & 1.04 & 0.99 & 1.03 \\
14878 & 101919 & 2003 & 1193.90 & 0.40 & 115634.00 & 1156346.58 & 1.03 & 0.97 & 1.00 \\
9908 & 101211 & 2003 & 273.20 & 0.33 & 25725.00 & 268254.71 & 1.06 & 0.98 & 1.04 \\
31648 & 105920 & 2003 & 3484.80 & 0.29 & 343870.00 & 3367180.19 & 1.01 & 0.97 & 0.98 \\
51998 & 300695 & 2003 & 198.10 & 0.21 & 19817.00 & 198168.56 & 1.00 & 1.00 & 1.00 \\
46925 & 200318 & 2003 & 5.90 & 0.23 & 598.00 & 4805.03 & 0.99 & 0.81 & 0.80 \\
34140 & 106209 & 2003 & 612.20 & 0.41 & 50814.00 & 507586.06 & 1.20 & 0.83 & 1.00 \\
51966 & 300679 & 2003 & 1498.80 & 0.32 & 139648.00 & 1395339.71 & 1.07 & 0.93 & 1.00 \\
15642 & 102010 & 2003 & 5143.50 & 0.31 & 607049.00 & 4964959.45 & 0.85 & 0.97 & 0.82 \\
31393 & 105881 & 2003 & 1230.80 & 0.27 & 101671.00 & 1121613.41 & 1.21 & 0.91 & 1.10 \\
34194 & 106211 & 2003 & 80.00 & 0.22 & 7980.00 & 76507.93 & 1.00 & 0.96 & 0.96 \\
46897 & 200312 & 2003 & 238.70 & 0.36 & 23782.00 & 230470.86 & 1.00 & 0.97 & 0.97 \\
34206 & 106212 & 2003 & 87.50 & 0.34 & 8735.00 & 85589.24 & 1.00 & 0.98 & 0.98 \\
31366 & 105880 & 2003 & 1222.80 & 0.58 & 121792.00 & 1195563.36 & 1.00 & 0.98 & 0.98 \\
52620 & 303175 & 2003 & 640.90 & 0.28 & 57301.00 & 626744.08 & 1.12 & 0.98 & 1.09 \\
46882 & 200311 & 2003 & 237.30 & 0.36 & 23958.00 & 224499.24 & 0.99 & 0.95 & 0.94 \\
34224 & 106213 & 2003 & 144.30 & 0.41 & 11572.00 & 127620.89 & 1.25 & 0.88 & 1.10 \\
31348 & 105879 & 2003 & 1268.80 & 0.39 & 128320.00 & 1215377.65 & 0.99 & 0.96 & 0.95 \\
46873 & 200310 & 2003 & 62.60 & 0.40 & 6309.00 & 59769.41 & 0.99 & 0.95 & 0.95 \\
31420 & 105882 & 2003 & 170.30 & 0.31 & 15054.00 & 171942.14 & 1.13 & 1.01 & 1.14 \\
6200 & 100829 & 2003 & 745.30 & 0.31 & 74294.00 & 705755.47 & 1.00 & 0.95 & 0.95 \\
34097 & 106207 & 2003 & 17.40 & 0.60 & 1739.00 & 15984.19 & 1.00 & 0.92 & 0.92 \\
56369 & 400203 & 2003 & 7.20 & 0.20 & 722.00 & 7139.63 & 1.00 & 0.99 & 0.99 \\
9087 & 101111 & 2003 & 472.10 & 0.32 & 39455.00 & 445289.95 & 1.20 & 0.94 & 1.13 \\
6690 & 100910 & 2003 & 117.80 & 0.26 & 11774.00 & 115876.67 & 1.00 & 0.98 & 0.98 \\
34113 & 106208 & 2003 & 37.60 & 0.33 & 3768.00 & 36762.98 & 1.00 & 0.98 & 0.98 \\
33615 & 106156 & 2003 & 4.30 & 0.48 & 423.00 & 4182.67 & 1.02 & 0.97 & 0.99 \\
31442 & 105886 & 2003 & 42.50 & 0.29 & 3745.00 & 42161.85 & 1.13 & 0.99 & 1.13 \\
56411 & 400205 & 2003 & 12.50 & 0.30 & 1150.00 & 12089.35 & 1.09 & 0.97 & 1.05 \\
69 & 100004 & 2003 & 1321.40 & 0.26 & 123231.00 & 1309699.05 & 1.07 & 0.99 & 1.06 \\
51981 & 300684 & 2003 & 155.60 & 0.30 & 15622.00 & 150285.46 & 1.00 & 0.97 & 0.96 \\
32032 & 105974 & 2003 & 38.00 & 0.21 & 3796.00 & 35600.83 & 1.00 & 0.94 & 0.94 \\
33911 & 106182 & 2003 & 919.20 & 0.40 & 91902.00 & 879687.09 & 1.00 & 0.96 & 0.96 \\
33885 & 106180 & 2003 & 35.70 & 0.23 & 3194.00 & 33418.00 & 1.12 & 0.94 & 1.05 \\
33731 & 106164 & 2003 & 117.40 & 0.70 & 11970.00 & 118334.55 & 0.98 & 1.01 & 0.99 \\
52153 & 302206 & 2003 & 706.80 & 0.16 & 67832.00 & 592287.51 & 1.04 & 0.84 & 0.87 \\
14915 & 101922 & 2003 & 1042.00 & 0.31 & 99847.00 & 995393.15 & 1.04 & 0.96 & 1.00 \\
31867 & 105949 & 2003 & 693.00 & 0.46 & 69224.00 & 643509.29 & 1.00 & 0.93 & 0.93 \\
6305 & 100847 & 2003 & 3.70 & 0.23 & 368.00 & 3489.15 & 1.01 & 0.94 & 0.95 \\
33747 & 106165 & 2003 & 71.40 & 0.34 & 7226.00 & 69181.80 & 0.99 & 0.97 & 0.96 \\
31856 & 105948 & 2003 & 29.00 & 0.36 & 2611.00 & 28803.77 & 1.11 & 0.99 & 1.10 \\
52141 & 302067 & 2003 & 12.20 & 0.28 & 1213.00 & 12122.66 & 1.01 & 0.99 & 1.00 \\
31845 & 105947 & 2003 & 38.90 & 0.51 & 3381.00 & 36984.39 & 1.15 & 0.95 & 1.09 \\
15462 & 101992 & 2003 & 163.00 & 0.32 & 15859.00 & 166931.50 & 1.03 & 1.02 & 1.05 \\
31834 & 105946 & 2003 & 140.30 & 0.56 & 13029.00 & 145129.98 & 1.08 & 1.03 & 1.11 \\
9802 & 101193 & 2003 & 549.30 & 0.29 & 47626.00 & 515177.79 & 1.15 & 0.94 & 1.08 \\
96706 & 611008 & 2003 & 64.10 & 0.37 & 8699.00 & 77257.95 & 0.74 & 1.21 & 0.89 \\
96740 & 611010 & 2003 & 49.50 & 0.37 & 4741.00 & 42698.61 & 1.04 & 0.86 & 0.90 \\
33647 & 106158 & 2003 & 543.00 & 0.28 & 52272.00 & 522441.79 & 1.04 & 0.96 & 1.00 \\
31892 & 105951 & 2003 & 160.10 & 0.41 & 16040.00 & 158243.68 & 1.00 & 0.99 & 0.99 \\
9772 & 101192 & 2003 & 274.60 & 0.33 & 22662.00 & 254174.20 & 1.21 & 0.93 & 1.12 \\
31967 & 105964 & 2003 & 117.10 & 0.27 & 10941.00 & 103845.63 & 1.07 & 0.89 & 0.95 \\
47479 & 212351 & 2003 & 187.70 & 0.35 & 18861.00 & 185424.13 & 1.00 & 0.99 & 0.98 \\
33671 & 106160 & 2003 & 24.40 & 0.28 & 2421.00 & 23756.39 & 1.01 & 0.97 & 0.98 \\
47142 & 200338 & 2003 & 46.90 & 0.17 & 3497.00 & 38547.32 & 1.34 & 0.82 & 1.10 \\
31975 & 105965 & 2003 & 76.80 & 0.30 & 7204.00 & 72134.18 & 1.07 & 0.94 & 1.00 \\
9754 & 101186 & 2003 & 499.20 & 0.25 & 44534.00 & 483157.81 & 1.12 & 0.97 & 1.08 \\
15428 & 101990 & 2003 & 353.90 & 0.36 & 35250.00 & 332908.87 & 1.00 & 0.94 & 0.94 \\
33704 & 106163 & 2003 & 914.10 & 0.46 & 91339.00 & 877111.05 & 1.00 & 0.96 & 0.96 \\
31940 & 105963 & 2003 & 713.10 & 0.25 & 67400.00 & 659142.62 & 1.06 & 0.92 & 0.98 \\
52164 & 302545 & 2003 & 115.50 & 0.23 & 11548.00 & 107653.70 & 1.00 & 0.93 & 0.93 \\
31929 & 105961 & 2003 & 142.90 & 0.29 & 14470.00 & 138043.46 & 0.99 & 0.97 & 0.95 \\
52547 & 303123 & 2003 & 82.60 & 0.27 & 8270.00 & 82552.96 & 1.00 & 1.00 & 1.00 \\
31919 & 105960 & 2003 & 260.20 & 0.25 & 26142.00 & 250124.44 & 1.00 & 0.96 & 0.96 \\
47114 & 200334 & 2003 & 39.50 & 0.63 & 3848.00 & 31189.21 & 1.03 & 0.79 & 0.81 \\
31904 & 105957 & 2003 & 9.40 & 0.30 & 951.00 & 9419.78 & 0.99 & 1.00 & 0.99 \\
47092 & 200333 & 2003 & 1646.90 & 0.35 & 128997.00 & 1413785.53 & 1.28 & 0.86 & 1.10 \\
52062 & 301438 & 2003 & 485.40 & 0.40 & 39434.00 & 438055.53 & 1.23 & 0.90 & 1.11 \\
31817 & 105943 & 2003 & 24.00 & 0.27 & 2033.00 & 22345.16 & 1.18 & 0.93 & 1.10 \\
9729 & 101179 & 2003 & 646.80 & 0.30 & 64367.00 & 632989.58 & 1.00 & 0.98 & 0.98 \\
52092 & 301560 & 2003 & 1163.50 & 0.40 & 95798.00 & 1074206.43 & 1.21 & 0.92 & 1.12 \\
56338 & 400192 & 2003 & 8.50 & 0.27 & 776.00 & 6620.52 & 1.10 & 0.78 & 0.85 \\
33834 & 106173 & 2003 & 866.20 & 0.38 & 80381.00 & 849248.13 & 1.08 & 0.98 & 1.06 \\
9119 & 101112 & 2003 & 818.10 & 0.27 & 79333.00 & 781774.76 & 1.03 & 0.96 & 0.99 \\
6261 & 100833 & 2003 & 646.70 & 0.25 & 64723.00 & 632895.32 & 1.00 & 0.98 & 0.98 \\
33861 & 106176 & 2003 & 42.60 & 0.47 & 3890.00 & 41474.96 & 1.10 & 0.97 & 1.07 \\
31701 & 105931 & 2003 & 5987.50 & 0.24 & 509406.00 & 5240017.34 & 1.18 & 0.88 & 1.03 \\
47511 & 212408 & 2003 & 2282.60 & 0.27 & 229415.00 & 2207922.95 & 0.99 & 0.97 & 0.96 \\
15532 & 102000 & 2003 & 351.90 & 0.30 & 35048.00 & 349537.47 & 1.00 & 0.99 & 1.00 \\
33876 & 106179 & 2003 & 56.80 & 0.26 & 5678.00 & 56117.97 & 1.00 & 0.99 & 0.99 \\
31682 & 105930 & 2003 & 422.70 & 0.55 & 42344.00 & 418409.77 & 1.00 & 0.99 & 0.99 \\
9863 & 101198 & 2003 & 215.60 & 0.38 & 18936.00 & 201719.28 & 1.14 & 0.94 & 1.07 \\
15400 & 101989 & 2003 & 250.20 & 0.28 & 25207.00 & 237757.05 & 0.99 & 0.95 & 0.94 \\
47026 & 200329 & 2003 & 254.10 & 0.47 & 25724.00 & 243217.32 & 0.99 & 0.96 & 0.95 \\
31728 & 105932 & 2003 & 168.00 & 0.27 & 15604.00 & 159403.09 & 1.08 & 0.95 & 1.02 \\
6673 & 100908 & 2003 & 251.40 & 0.32 & 24900.00 & 233680.22 & 1.01 & 0.93 & 0.94 \\
33807 & 106172 & 2003 & 54.90 & 0.31 & 5521.00 & 52905.20 & 0.99 & 0.96 & 0.96 \\
47044 & 200330 & 2003 & 18.80 & 0.29 & 1395.00 & 15556.92 & 1.35 & 0.83 & 1.12 \\
52133 & 302060 & 2003 & 89.80 & 0.23 & 9462.00 & 98084.98 & 0.95 & 1.09 & 1.04 \\
31802 & 105938 & 2003 & 29.70 & 0.47 & 2357.00 & 28995.07 & 1.26 & 0.98 & 1.23 \\
31794 & 105936 & 2003 & 49.70 & 0.33 & 4374.00 & 44450.46 & 1.14 & 0.89 & 1.02 \\
52557 & 303124 & 2003 & 71.60 & 0.67 & 5642.00 & 66464.76 & 1.27 & 0.93 & 1.18 \\
33755 & 106167 & 2003 & 439.30 & 0.37 & 43925.00 & 402906.71 & 1.00 & 0.92 & 0.92 \\
9152 & 101115 & 2003 & 17852.30 & 0.29 & 1624083.00 & 16853421.66 & 1.10 & 0.94 & 1.04 \\
33767 & 106169 & 2003 & 810.80 & 0.55 & 80448.00 & 783172.65 & 1.01 & 0.97 & 0.97 \\
31766 & 105935 & 2003 & 256.00 & 0.30 & 24833.00 & 261765.00 & 1.03 & 1.02 & 1.05 \\
15501 & 101999 & 2003 & 1132.10 & 0.35 & 103416.00 & 1108767.35 & 1.09 & 0.98 & 1.07 \\
9833 & 101194 & 2003 & 170.70 & 0.25 & 14469.00 & 146790.06 & 1.18 & 0.86 & 1.01 \\
31755 & 105933 & 2003 & 1660.70 & 0.27 & 146039.00 & 1552641.26 & 1.14 & 0.93 & 1.06 \\
47065 & 200331 & 2003 & 18.70 & 0.22 & 1876.00 & 16921.54 & 1.00 & 0.90 & 0.90 \\
52105 & 301571 & 2003 & 551.30 & 0.59 & 54841.00 & 524583.47 & 1.01 & 0.95 & 0.96 \\
33794 & 106170 & 2003 & 319.10 & 0.43 & 31893.00 & 305853.77 & 1.00 & 0.96 & 0.96 \\
34236 & 106214 & 2003 & 238.10 & 0.42 & 23829.00 & 234010.11 & 1.00 & 0.98 & 0.98 \\
52522 & 302997 & 2003 & 313.70 & 0.49 & 24819.00 & 301137.30 & 1.26 & 0.96 & 1.21 \\
46720 & 200293 & 2003 & 30.10 & 0.63 & 3013.00 & 29073.75 & 1.00 & 0.97 & 0.96 \\
10200 & 101268 & 2003 & 530.40 & 0.31 & 51564.00 & 498342.55 & 1.03 & 0.94 & 0.97 \\
122 & 100009 & 2003 & 272.40 & 0.45 & 27065.00 & 268807.42 & 1.01 & 0.99 & 0.99 \\
46717 & 200292 & 2003 & 17.30 & 0.27 & 1504.00 & 14947.62 & 1.15 & 0.86 & 0.99 \\
34545 & 106251 & 2003 & 138.40 & 0.22 & 12377.00 & 129702.10 & 1.12 & 0.94 & 1.05 \\
30886 & 105806 & 2003 & 119.80 & 0.46 & 10672.00 & 117480.00 & 1.12 & 0.98 & 1.10 \\
32075 & 105980 & 2003 & 402.60 & 0.40 & 40175.00 & 381885.28 & 1.00 & 0.95 & 0.95 \\
10214 & 101274 & 2003 & 421.10 & 0.37 & 37840.00 & 421698.83 & 1.11 & 1.00 & 1.11 \\
33602 & 106152 & 2003 & 339.10 & 0.28 & 33979.00 & 329454.17 & 1.00 & 0.97 & 0.97 \\
34559 & 106255 & 2003 & 400.50 & 0.30 & 36810.00 & 400790.34 & 1.09 & 1.00 & 1.09 \\
30858 & 105805 & 2003 & 99.70 & 1.09 & 9748.00 & 96490.89 & 1.02 & 0.97 & 0.99 \\
51702 & 240543 & 2003 & 1.70 & 0.30 & 177.00 & 1583.62 & 0.96 & 0.93 & 0.89 \\
9684 & 101165 & 2003 & 2365.30 & 0.28 & 246749.00 & 2249241.21 & 0.96 & 0.95 & 0.91 \\
46743 & 200294 & 2003 & 68.70 & 0.27 & 6019.00 & 62728.91 & 1.14 & 0.91 & 1.04 \\
34585 & 106256 & 2003 & 79.70 & 0.33 & 7942.00 & 78669.65 & 1.00 & 0.99 & 0.99 \\
30919 & 105836 & 2003 & 210.40 & 0.34 & 22001.00 & 214011.87 & 0.96 & 1.02 & 0.97 \\
15792 & 102018 & 2003 & 464.00 & 0.26 & 45641.00 & 441456.61 & 1.02 & 0.95 & 0.97 \\
10149 & 101263 & 2003 & 236.50 & 0.30 & 21189.00 & 217877.81 & 1.12 & 0.92 & 1.03 \\
34536 & 106250 & 2003 & 137.20 & 0.18 & 12486.00 & 127292.82 & 1.10 & 0.93 & 1.02 \\
30963 & 105842 & 2003 & 550.90 & 0.49 & 55235.00 & 538514.69 & 1.00 & 0.98 & 0.97 \\
6118 & 100823 & 2003 & 23.60 & 0.37 & 2348.00 & 23481.39 & 1.01 & 0.99 & 1.00 \\
10170 & 101264 & 2003 & 819.00 & 0.36 & 68284.00 & 767729.96 & 1.20 & 0.94 & 1.12 \\
8971 & 101107 & 2003 & 867.50 & 0.28 & 78989.00 & 859523.70 & 1.10 & 0.99 & 1.09 \\
46756 & 200295 & 2003 & 315.20 & 0.42 & 25117.00 & 244936.68 & 1.25 & 0.78 & 0.98 \\
52730 & 307849 & 2003 & 106.10 & 0.15 & 10587.00 & 97561.02 & 1.00 & 0.92 & 0.92 \\
31001 & 105846 & 2003 & 1486.60 & 0.38 & 144431.00 & 1472196.38 & 1.03 & 0.99 & 1.02 \\
34592 & 106257 & 2003 & 256.10 & 0.35 & 26049.00 & 252562.20 & 0.98 & 0.99 & 0.97 \\
30830 & 105804 & 2003 & 635.50 & 0.24 & 59680.00 & 575809.35 & 1.06 & 0.91 & 0.96 \\
51678 & 240539 & 2003 & 21.00 & 0.28 & 1813.00 & 21053.76 & 1.16 & 1.00 & 1.16 \\
46705 & 200280 & 2003 & 3.70 & 0.31 & 376.00 & 3331.02 & 0.98 & 0.90 & 0.89 \\
34679 & 106270 & 2003 & 14.00 & 0.41 & 1406.00 & 13331.47 & 1.00 & 0.95 & 0.95 \\
30734 & 105791 & 2003 & 5.60 & 0.26 & 487.00 & 4759.27 & 1.15 & 0.85 & 0.98 \\
6087 & 100822 & 2003 & 11.40 & 0.37 & 1137.00 & 11372.83 & 1.00 & 1.00 & 1.00 \\
32107 & 105983 & 2003 & 256.00 & 0.28 & 25595.00 & 253949.40 & 1.00 & 0.99 & 0.99 \\
34712 & 106272 & 2003 & 2302.10 & 0.28 & 213512.00 & 2372133.45 & 1.08 & 1.03 & 1.11 \\
30713 & 105788 & 2003 & 34.70 & 0.28 & 3448.00 & 33378.15 & 1.01 & 0.96 & 0.97 \\
34741 & 106275 & 2003 & 52.00 & 0.37 & 5207.00 & 48939.48 & 1.00 & 0.94 & 0.94 \\
52809 & 330079 & 2003 & 37.50 & 0.32 & 3726.00 & 33429.43 & 1.01 & 0.89 & 0.90 \\
33577 & 106150 & 2003 & 146.40 & 0.34 & 14436.00 & 151435.73 & 1.01 & 1.03 & 1.05 \\
30755 & 105793 & 2003 & 2778.10 & 0.32 & 278788.00 & 2519252.28 & 1.00 & 0.91 & 0.90 \\
52745 & 320323 & 2003 & 1.40 & 0.15 & 191.00 & 1822.31 & 0.73 & 1.30 & 0.95 \\
30765 & 105794 & 2003 & 42.40 & 0.37 & 3383.00 & 36319.84 & 1.25 & 0.86 & 1.07 \\
10233 & 101275 & 2003 & 1161.60 & 0.31 & 106212.00 & 1149446.23 & 1.09 & 0.99 & 1.08 \\
34623 & 106261 & 2003 & 510.70 & 0.38 & 50301.00 & 480521.03 & 1.02 & 0.94 & 0.96 \\
33588 & 106151 & 2003 & 1263.20 & 0.37 & 130325.00 & 1244402.32 & 0.97 & 0.99 & 0.95 \\
34634 & 106262 & 2003 & 366.20 & 0.23 & 36588.00 & 354811.94 & 1.00 & 0.97 & 0.97 \\
30802 & 105803 & 2003 & 6396.90 & 0.44 & 601866.00 & 5617456.73 & 1.06 & 0.88 & 0.93 \\
52771 & 320640 & 2003 & 2.40 & 0.15 & 222.00 & 2220.93 & 1.08 & 0.93 & 1.00 \\
30787 & 105798 & 2003 & 1487.80 & 0.38 & 155724.00 & 1564170.25 & 0.96 & 1.05 & 1.00 \\
52778 & 322114 & 2003 & 4.00 & 0.22 & 386.00 & 3587.92 & 1.04 & 0.90 & 0.93 \\
46710 & 200291 & 2003 & 13.20 & 0.38 & 1330.00 & 12315.48 & 0.99 & 0.93 & 0.93 \\
55666 & 400136 & 2003 & 12.80 & 0.15 & 1266.00 & 12656.07 & 1.01 & 0.99 & 1.00 \\
46708 & 200286 & 2003 & 20.20 & 0.22 & 3316.00 & 32780.27 & 0.61 & 1.62 & 0.99 \\
15876 & 102050 & 2003 & 311.90 & 0.40 & 25532.00 & 267080.04 & 1.22 & 0.86 & 1.05 \\
56416 & 400206 & 2003 & 7.50 & 0.59 & 755.00 & 7180.27 & 0.99 & 0.96 & 0.95 \\
34527 & 106249 & 2003 & 168.80 & 0.27 & 15103.00 & 160873.02 & 1.12 & 0.95 & 1.07 \\
31017 & 105847 & 2003 & 37.60 & 0.33 & 3775.00 & 35029.25 & 1.00 & 0.93 & 0.93 \\
32038 & 105976 & 2003 & 9.80 & 0.25 & 979.00 & 9778.81 & 1.00 & 1.00 & 1.00 \\
34296 & 106221 & 2003 & 127.10 & 0.22 & 12302.00 & 128043.22 & 1.03 & 1.01 & 1.04 \\
34 & 100003 & 2003 & 850.20 & 0.31 & 84864.00 & 815465.80 & 1.00 & 0.96 & 0.96 \\
56162 & 400178 & 2003 & 58.10 & 0.55 & 5818.00 & 55433.14 & 1.00 & 0.95 & 0.95 \\
47542 & 212658 & 2003 & 10220.70 & 0.38 & 1022068.00 & 9577417.40 & 1.00 & 0.94 & 0.94 \\
31231 & 105869 & 2003 & 720.90 & 0.19 & 72209.00 & 705762.41 & 1.00 & 0.98 & 0.98 \\
6152 & 100825 & 2003 & 62.30 & 0.26 & 6238.00 & 61008.77 & 1.00 & 0.98 & 0.98 \\
47569 & 212809 & 2003 & 10.60 & 0.28 & 1090.00 & 9891.60 & 0.97 & 0.93 & 0.91 \\
10050 & 101256 & 2003 & 4.70 & 0.23 & 365.00 & 4224.64 & 1.29 & 0.90 & 1.16 \\
34323 & 106222 & 2003 & 170.40 & 0.50 & 17397.00 & 165447.16 & 0.98 & 0.97 & 0.95 \\
32047 & 105977 & 2003 & 1047.10 & 0.35 & 116341.00 & 1016522.48 & 0.90 & 0.97 & 0.87 \\
52643 & 305586 & 2003 & 119.50 & 0.46 & 7787.00 & 99946.26 & 1.53 & 0.84 & 1.28 \\
15727 & 102016 & 2003 & 11298.20 & 0.22 & 1139021.00 & 10837729.91 & 0.99 & 0.96 & 0.95 \\
56141 & 400176 & 2003 & 281.70 & 0.05 & 28721.00 & 280743.73 & 0.98 & 1.00 & 0.98 \\
34290 & 106220 & 2003 & 208.30 & 0.36 & 21820.00 & 215749.97 & 0.95 & 1.04 & 0.99 \\
55662 & 400135 & 2003 & 21.80 & -0.04 & 2246.00 & 22641.61 & 0.97 & 1.04 & 1.01 \\
15702 & 102015 & 2003 & 231.70 & 0.24 & 22019.00 & 240385.24 & 1.05 & 1.04 & 1.09 \\
52630 & 305184 & 2003 & 29.30 & 0.33 & 2430.00 & 24520.04 & 1.21 & 0.84 & 1.01 \\
31320 & 105878 & 2003 & 1786.00 & 0.37 & 175179.00 & 1542129.96 & 1.02 & 0.86 & 0.88 \\
56420 & 400207 & 2003 & 9.40 & 0.29 & 921.00 & 9106.19 & 1.02 & 0.97 & 0.99 \\
10008 & 101252 & 2003 & 142.80 & 0.31 & 13260.00 & 132627.47 & 1.08 & 0.93 & 1.00 \\
31312 & 105877 & 2003 & 6.50 & 0.25 & 651.00 & 6307.02 & 1.00 & 0.97 & 0.97 \\
6704 & 100913 & 2003 & 582.50 & 0.38 & 58025.00 & 538685.64 & 1.00 & 0.92 & 0.93 \\
51955 & 300673 & 2003 & 122.00 & 0.35 & 12231.00 & 119345.17 & 1.00 & 0.98 & 0.98 \\
15682 & 102013 & 2003 & 2715.20 & 0.27 & 310730.00 & 2774990.95 & 0.87 & 1.02 & 0.89 \\
31296 & 105875 & 2003 & 37.60 & 0.23 & 3756.00 & 36119.00 & 1.00 & 0.96 & 0.96 \\
34263 & 106216 & 2003 & 525.60 & 0.43 & 70122.00 & 513146.63 & 0.75 & 0.98 & 0.73 \\
31282 & 105874 & 2003 & 250.80 & 0.45 & 25360.00 & 223852.47 & 0.99 & 0.89 & 0.88 \\
31271 & 105873 & 2003 & 8.70 & 0.32 & 744.00 & 6307.68 & 1.17 & 0.73 & 0.85 \\
51921 & 300653 & 2003 & 185.10 & 0.49 & 19927.00 & 191465.46 & 0.93 & 1.03 & 0.96 \\
6165 & 100827 & 2003 & 229.20 & 0.87 & 22850.00 & 228567.80 & 1.00 & 1.00 & 1.00 \\
31258 & 105871 & 2003 & 10.00 & 0.48 & 912.00 & 7449.79 & 1.10 & 0.74 & 0.82 \\
52690 & 306690 & 2003 & 70.70 & 0.43 & 7046.00 & 68817.46 & 1.00 & 0.97 & 0.98 \\
34333 & 106223 & 2003 & 81.20 & 0.31 & 7629.00 & 81797.07 & 1.06 & 1.01 & 1.07 \\
52653 & 305590 & 2003 & 297.90 & 0.59 & 29089.00 & 278870.73 & 1.02 & 0.94 & 0.96 \\
96678 & 611003 & 2003 & 347.40 & 0.32 & 22933.00 & 231606.20 & 1.51 & 0.67 & 1.01 \\
32063 & 105978 & 2003 & 145.00 & 0.53 & 14538.00 & 144496.51 & 1.00 & 1.00 & 0.99 \\
31071 & 105854 & 2003 & 752.40 & 0.37 & 72677.00 & 779147.39 & 1.04 & 1.04 & 1.07 \\
46806 & 200300 & 2003 & 5.50 & 0.13 & 563.00 & 5601.05 & 0.98 & 1.02 & 0.99 \\
34474 & 106244 & 2003 & 2.00 & 0.52 & 202.00 & 1924.73 & 0.99 & 0.96 & 0.95 \\
46805 & 200299 & 2003 & 7.40 & 0.42 & 742.00 & 7417.99 & 1.00 & 1.00 & 1.00 \\
34500 & 106248 & 2003 & 601.90 & 0.39 & 51068.00 & 440050.60 & 1.18 & 0.73 & 0.86 \\
31043 & 105852 & 2003 & 58.20 & 0.40 & 5008.00 & 49136.10 & 1.16 & 0.84 & 0.98 \\
15370 & 101988 & 2003 & 379.40 & 0.22 & 37878.00 & 377319.54 & 1.00 & 0.99 & 1.00 \\
46779 & 200297 & 2003 & 99.60 & 0.37 & 9957.00 & 94844.39 & 1.00 & 0.95 & 0.95 \\
10132 & 101262 & 2003 & 15.50 & 0.28 & 1552.00 & 14987.90 & 1.00 & 0.97 & 0.97 \\
31025 & 105848 & 2003 & 58.50 & 0.35 & 5869.00 & 56866.82 & 1.00 & 0.97 & 0.97 \\
31192 & 105866 & 2003 & 7944.90 & 0.28 & 793081.00 & 7929178.43 & 1.00 & 1.00 & 1.00 \\
34447 & 106240 & 2003 & 202.20 & 0.42 & 15433.00 & 189620.25 & 1.31 & 0.94 & 1.23 \\
34431 & 106239 & 2003 & 59.10 & 0.40 & 5913.00 & 54020.46 & 1.00 & 0.91 & 0.91 \\
46851 & 200309 & 2003 & 59.70 & 0.34 & 5969.00 & 56222.46 & 1.00 & 0.94 & 0.94 \\
34353 & 106224 & 2003 & 313.10 & 0.29 & 25963.00 & 243397.39 & 1.21 & 0.78 & 0.94 \\
52676 & 305766 & 2003 & 75.60 & 0.28 & 7555.00 & 75364.28 & 1.00 & 1.00 & 1.00 \\
31164 & 105865 & 2003 & 113.80 & 0.43 & 11360.00 & 110626.77 & 1.00 & 0.97 & 0.97 \\
46829 & 200304 & 2003 & 15.20 & 0.34 & 1524.00 & 14902.53 & 1.00 & 0.98 & 0.98 \\
46823 & 200303 & 2003 & 12.90 & 0.44 & 1285.00 & 12483.56 & 1.00 & 0.97 & 0.97 \\
10080 & 101258 & 2003 & 4573.70 & 0.39 & 392099.00 & 4342477.64 & 1.17 & 0.95 & 1.11 \\
34371 & 106230 & 2003 & 207.20 & 0.36 & 21220.00 & 212081.22 & 0.98 & 1.02 & 1.00 \\
31141 & 105861 & 2003 & 293.30 & 0.40 & 24325.00 & 276465.78 & 1.21 & 0.94 & 1.14 \\
34398 & 106231 & 2003 & 272.40 & 0.48 & 27920.00 & 278136.96 & 0.98 & 1.02 & 1.00 \\
9002 & 101108 & 2003 & 949.00 & 0.35 & 93291.00 & 964345.45 & 1.02 & 1.02 & 1.03 \\
31113 & 105860 & 2003 & 1705.00 & 0.23 & 170909.00 & 1676013.61 & 1.00 & 0.98 & 0.98 \\
47190 & 200342 & 2003 & 6984.00 & 0.34 & 633268.00 & 6282040.74 & 1.10 & 0.90 & 0.99 \\
14834 & 101916 & 2003 & 548.40 & 0.34 & 54669.00 & 515668.24 & 1.00 & 0.94 & 0.94 \\
15761 & 102017 & 2003 & 6832.10 & 0.22 & 645662.00 & 6778860.06 & 1.06 & 0.99 & 1.05 \\
47663 & 216749 & 2003 & 97.30 & 0.26 & 7973.00 & 85594.08 & 1.22 & 0.88 & 1.07 \\
35960 & 106442 & 2003 & 4906.80 & 0.46 & 469534.00 & 4610289.00 & 1.05 & 0.94 & 0.98 \\
29240 & 105561 & 2003 & 320.80 & 0.37 & 32493.00 & 311680.76 & 0.99 & 0.97 & 0.96 \\
27564 & 105291 & 2003 & 2717.30 & 0.24 & 221414.00 & 1822348.43 & 1.23 & 0.67 & 0.82 \\
45922 & 200168 & 2003 & 184.90 & 0.15 & 20182.00 & 166249.04 & 0.92 & 0.90 & 0.82 \\
11487 & 101422 & 2003 & 55.60 & 0.27 & 5470.00 & 54587.46 & 1.02 & 0.98 & 1.00 \\
53546 & 351713 & 2003 & 275.30 & 0.40 & 29273.00 & 260824.03 & 0.94 & 0.95 & 0.89 \\
45917 & 200167 & 2003 & 129.10 & 0.03 & 11729.00 & 108932.63 & 1.10 & 0.84 & 0.93 \\
27545 & 105287 & 2003 & 130.50 & 0.55 & 13037.00 & 127905.77 & 1.00 & 0.98 & 0.98 \\
45912 & 200164 & 2003 & 24.30 & 0.32 & 2429.00 & 23017.66 & 1.00 & 0.95 & 0.95 \\
5405 & 100760 & 2003 & 520.80 & 0.39 & 47707.00 & 535041.26 & 1.09 & 1.03 & 1.12 \\
574 & 100076 & 2003 & 522.50 & 0.29 & 51703.00 & 526057.47 & 1.01 & 1.01 & 1.02 \\
53565 & 351891 & 2003 & 30.50 & 0.31 & 3026.00 & 30223.70 & 1.01 & 0.99 & 1.00 \\
37456 & 106919 & 2003 & 29.60 & 0.26 & 3013.00 & 30127.35 & 0.98 & 1.02 & 1.00 \\
32520 & 106038 & 2003 & 1966.00 & 0.39 & 196601.00 & 1699185.66 & 1.00 & 0.86 & 0.86 \\
27508 & 105283 & 2003 & 8.00 & 0.25 & 805.00 & 7870.42 & 0.99 & 0.98 & 0.98 \\
17007 & 102230 & 2003 & 22.70 & 0.30 & 2267.00 & 21810.30 & 1.00 & 0.96 & 0.96 \\
15178 & 101964 & 2003 & 1307.40 & 0.26 & 111150.00 & 1192813.16 & 1.18 & 0.91 & 1.07 \\
6489 & 100876 & 2003 & 1.60 & 0.53 & 142.00 & 1519.38 & 1.13 & 0.95 & 1.07 \\
27638 & 105306 & 2003 & 54.50 & 0.43 & 5416.00 & 54367.43 & 1.01 & 1.00 & 1.00 \\
45949 & 200172 & 2003 & 14.50 & 0.04 & 1451.00 & 13787.35 & 1.00 & 0.95 & 0.95 \\
37379 & 106742 & 2003 & 138.10 & 0.28 & 13909.00 & 136161.88 & 0.99 & 0.99 & 0.98 \\
37386 & 106746 & 2003 & 46.40 & 0.46 & 4644.00 & 45444.43 & 1.00 & 0.98 & 0.98 \\
45943 & 200171 & 2003 & 59.60 & 0.05 & 5953.00 & 59320.85 & 1.00 & 1.00 & 1.00 \\
37390 & 106747 & 2003 & 326.80 & 0.40 & 32705.00 & 304439.86 & 1.00 & 0.93 & 0.93 \\
27609 & 105303 & 2003 & 152.10 & 0.32 & 16876.00 & 166511.18 & 0.90 & 1.09 & 0.99 \\
548 & 100075 & 2003 & 2357.20 & 0.27 & 287718.00 & 2387118.76 & 0.82 & 1.01 & 0.83 \\
27592 & 105295 & 2003 & 470.20 & 0.24 & 47492.00 & 462463.40 & 0.99 & 0.98 & 0.97 \\
13 & 100001 & 2003 & 3038.90 & 0.21 & 303758.00 & 2824956.93 & 1.00 & 0.93 & 0.93 \\
37416 & 106869 & 2003 & 154.60 & 0.21 & 12960.00 & 135740.11 & 1.19 & 0.88 & 1.05 \\
14472 & 101861 & 2003 & 1700.10 & 0.31 & 139260.00 & 1460097.01 & 1.22 & 0.86 & 1.05 \\
37424 & 106896 & 2003 & 11.20 & 0.28 & 1119.00 & 11168.66 & 1.00 & 1.00 & 1.00 \\
37371 & 106740 & 2003 & 132.50 & 0.33 & 13319.00 & 132189.77 & 0.99 & 1.00 & 0.99 \\
11509 & 101425 & 2003 & 20.60 & 0.31 & 1844.00 & 18518.40 & 1.12 & 0.90 & 1.00 \\
6924 & 100969 & 2003 & 60.40 & 0.24 & 5434.00 & 58536.85 & 1.11 & 0.97 & 1.08 \\
27404 & 105276 & 2003 & 1555.30 & 0.30 & 155531.00 & 1442515.13 & 1.00 & 0.93 & 0.93 \\
5358 & 100757 & 2003 & 14.70 & 0.39 & 1394.00 & 14201.33 & 1.05 & 0.97 & 1.02 \\
11546 & 101427 & 2003 & 113.30 & 0.46 & 10170.00 & 85404.56 & 1.11 & 0.75 & 0.84 \\
74649 & 601147 & 2003 & 139.90 & 0.38 & 11198.00 & 105714.68 & 1.25 & 0.76 & 0.94 \\
37552 & 106968 & 2003 & 53.10 & 0.56 & 4364.00 & 49883.80 & 1.22 & 0.94 & 1.14 \\
14451 & 101858 & 2003 & 1588.50 & 0.43 & 128712.00 & 1126634.45 & 1.23 & 0.71 & 0.88 \\
5349 & 100754 & 2003 & 498.70 & 0.27 & 48513.00 & 495149.87 & 1.03 & 0.99 & 1.02 \\
27374 & 105275 & 2003 & 75.80 & 0.31 & 7583.00 & 74695.74 & 1.00 & 0.99 & 0.99 \\
52411 & 302881 & 2003 & 174.10 & 0.28 & 17443.00 & 173834.33 & 1.00 & 1.00 & 1.00 \\
17086 & 102255 & 2003 & 15.40 & 0.34 & 1542.00 & 15308.98 & 1.00 & 0.99 & 0.99 \\
53582 & 354336 & 2003 & 12.70 & 0.20 & 1275.00 & 12655.31 & 1.00 & 1.00 & 0.99 \\
37602 & 106972 & 2003 & 60.20 & 0.23 & 6100.00 & 59766.74 & 0.99 & 0.99 & 0.98 \\
45892 & 200156 & 2003 & 48.70 & 0.20 & 4552.00 & 47421.25 & 1.07 & 0.97 & 1.04 \\
37544 & 106962 & 2003 & 4.10 & 0.24 & 405.00 & 3990.53 & 1.01 & 0.97 & 0.99 \\
55495 & 400113 & 2003 & 45.10 & 0.17 & 4718.00 & 42940.18 & 0.96 & 0.95 & 0.91 \\
37481 & 106931 & 2003 & 576.20 & 0.78 & 57778.00 & 573070.56 & 1.00 & 0.99 & 0.99 \\
37490 & 106934 & 2003 & 900.60 & 0.52 & 90261.00 & 862916.43 & 1.00 & 0.96 & 0.96 \\
27478 & 105281 & 2003 & 412.00 & 0.37 & 37072.00 & 380782.40 & 1.11 & 0.92 & 1.03 \\
17043 & 102231 & 2003 & 530.50 & 0.30 & 53452.00 & 525766.85 & 0.99 & 0.99 & 0.98 \\
27469 & 105280 & 2003 & 40.40 & 0.33 & 4050.00 & 37572.08 & 1.00 & 0.93 & 0.93 \\
15001 & 101930 & 2003 & 871.20 & 0.21 & 87107.00 & 815883.33 & 1.00 & 0.94 & 0.94 \\
49775 & 240359 & 2003 & 9.20 & 0.37 & 870.00 & 8628.95 & 1.06 & 0.94 & 0.99 \\
599 & 100079 & 2003 & 2044.90 & 0.36 & 217876.00 & 1918775.77 & 0.94 & 0.94 & 0.88 \\
53574 & 354018 & 2003 & 92.20 & 0.28 & 9275.00 & 91585.48 & 0.99 & 0.99 & 0.99 \\
37516 & 106944 & 2003 & 45.30 & 0.23 & 4210.00 & 43370.43 & 1.08 & 0.96 & 1.03 \\
8400 & 101085 & 2003 & 207.40 & 0.26 & 19168.00 & 209577.31 & 1.08 & 1.01 & 1.09 \\
37525 & 106948 & 2003 & 315.20 & 0.28 & 31372.00 & 291047.62 & 1.00 & 0.92 & 0.93 \\
27433 & 105278 & 2003 & 695.60 & 0.38 & 68799.00 & 707267.35 & 1.01 & 1.02 & 1.03 \\
33149 & 106097 & 2003 & 532.50 & 0.18 & 48282.00 & 509872.53 & 1.10 & 0.96 & 1.06 \\
37537 & 106961 & 2003 & 88.20 & 0.27 & 8811.00 & 86262.23 & 1.00 & 0.98 & 0.98 \\
11563 & 101430 & 2003 & 134.70 & 0.30 & 13430.00 & 129966.86 & 1.00 & 0.96 & 0.97 \\
45955 & 200173 & 2003 & 34.30 & 0.03 & 3423.00 & 34071.50 & 1.00 & 0.99 & 1.00 \\
11375 & 101399 & 2003 & 89.40 & 0.19 & 9299.00 & 87828.95 & 0.96 & 0.98 & 0.94 \\
27864 & 105335 & 2003 & 241.00 & 0.22 & 24152.00 & 240069.64 & 1.00 & 1.00 & 0.99 \\
49973 & 240379 & 2003 & 14.00 & 0.23 & 1392.00 & 13919.21 & 1.01 & 0.99 & 1.00 \\
47339 & 210203 & 2003 & 6656.70 & 0.30 & 637945.00 & 6489525.98 & 1.04 & 0.97 & 1.02 \\
46072 & 200183 & 2003 & 23.40 & 0.30 & 2334.00 & 23338.06 & 1.00 & 1.00 & 1.00 \\
49951 & 240377 & 2003 & 23.00 & 0.32 & 2071.00 & 22177.67 & 1.11 & 0.96 & 1.07 \\
37159 & 106701 & 2003 & 50.40 & 0.31 & 4877.00 & 50639.09 & 1.03 & 1.00 & 1.04 \\
33192 & 106102 & 2003 & 242.10 & 0.31 & 23176.00 & 238883.14 & 1.04 & 0.99 & 1.03 \\
5471 & 100764 & 2003 & 438.10 & 0.41 & 43104.00 & 422695.26 & 1.02 & 0.96 & 0.98 \\
37166 & 106706 & 2003 & 55.70 & 0.35 & 5320.00 & 53296.30 & 1.05 & 0.96 & 1.00 \\
37180 & 106707 & 2003 & 147.50 & 0.31 & 13940.00 & 139699.59 & 1.06 & 0.95 & 1.00 \\
27827 & 105332 & 2003 & 82.20 & 0.22 & 8242.00 & 80296.62 & 1.00 & 0.98 & 0.97 \\
52417 & 302907 & 2003 & 827.70 & 0.49 & 77467.00 & 774645.57 & 1.07 & 0.94 & 1.00 \\
49928 & 240376 & 2003 & 54.30 & 0.29 & 5429.00 & 52730.05 & 1.00 & 0.97 & 0.97 \\
37130 & 106692 & 2003 & 755.10 & 0.28 & 112999.00 & 1030275.80 & 0.67 & 1.36 & 0.91 \\
46095 & 200184 & 2003 & 13.40 & 0.47 & 1294.00 & 12932.41 & 1.04 & 0.97 & 1.00 \\
27935 & 105353 & 2003 & 16.10 & 0.39 & 1587.00 & 15021.39 & 1.01 & 0.93 & 0.95 \\
14516 & 101871 & 2003 & 269.60 & 0.26 & 24143.00 & 273857.99 & 1.12 & 1.02 & 1.13 \\
8465 & 101087 & 2003 & 407.00 & 0.36 & 39259.00 & 418600.06 & 1.04 & 1.03 & 1.07 \\
52324 & 302763 & 2003 & 80.50 & 0.41 & 8158.00 & 76869.01 & 0.99 & 0.95 & 0.94 \\
74713 & 601155 & 2003 & 69.40 & 0.38 & 6136.00 & 64409.38 & 1.13 & 0.93 & 1.05 \\
52443 & 302941 & 2003 & 7.70 & 0.22 & 732.00 & 7314.22 & 1.05 & 0.95 & 1.00 \\
55528 & 400116 & 2003 & 53.60 & 0.31 & 5341.00 & 52881.68 & 1.00 & 0.99 & 0.99 \\
27912 & 105346 & 2003 & 1272.80 & 0.32 & 127134.00 & 1250806.84 & 1.00 & 0.98 & 0.98 \\
53515 & 351459 & 2003 & 829.90 & 0.29 & 70888.00 & 712155.78 & 1.17 & 0.86 & 1.00 \\
27903 & 105343 & 2003 & 114.90 & 0.28 & 11066.00 & 101398.15 & 1.04 & 0.88 & 0.92 \\
16880 & 102213 & 2003 & 790.40 & 0.26 & 79030.00 & 775095.44 & 1.00 & 0.98 & 0.98 \\
32476 & 106033 & 2003 & 1014.00 & 0.45 & 97683.00 & 973054.40 & 1.04 & 0.96 & 1.00 \\
6914 & 100968 & 2003 & 333.60 & 0.29 & 33357.00 & 302042.16 & 1.00 & 0.91 & 0.91 \\
37362 & 106737 & 2003 & 219.80 & 0.21 & 15873.00 & 159505.62 & 1.38 & 0.73 & 1.00 \\
47770 & 221210 & 2003 & 134.70 & 0.52 & 13647.00 & 132934.98 & 0.99 & 0.99 & 0.97 \\
37219 & 106710 & 2003 & 52.50 & 0.26 & 6130.00 & 60694.13 & 0.86 & 1.16 & 0.99 \\
55938 & 400160 & 2003 & 39.70 & 0.36 & 3750.00 & 41487.97 & 1.06 & 1.05 & 1.11 \\
27701 & 105311 & 2003 & 133.40 & 0.34 & 13261.00 & 112529.77 & 1.01 & 0.84 & 0.85 \\
49867 & 240369 & 2003 & 3.80 & 0.31 & 386.00 & 3687.22 & 0.98 & 0.97 & 0.96 \\
47801 & 221485 & 2003 & 477.80 & 0.23 & 47775.00 & 476417.28 & 1.00 & 1.00 & 1.00 \\
37289 & 106726 & 2003 & 1910.50 & 0.27 & 190843.00 & 1894126.51 & 1.00 & 0.99 & 0.99 \\
32511 & 106037 & 2003 & 37.50 & 0.28 & 3767.00 & 36131.16 & 1.00 & 0.96 & 0.96 \\
16971 & 102224 & 2003 & 7037.50 & 0.46 & 506296.00 & 6179145.85 & 1.39 & 0.88 & 1.22 \\
37316 & 106729 & 2003 & 1947.20 & 0.30 & 195816.00 & 1936109.22 & 0.99 & 0.99 & 0.99 \\
45982 & 200175 & 2003 & 1.10 & 0.06 & 112.00 & 1029.41 & 0.98 & 0.94 & 0.92 \\
27667 & 105309 & 2003 & 1042.10 & 0.33 & 87194.00 & 950770.80 & 1.20 & 0.91 & 1.09 \\
11460 & 101414 & 2003 & 22.80 & 0.26 & 2378.00 & 22412.71 & 0.96 & 0.98 & 0.94 \\
37206 & 106708 & 2003 & 493.50 & 0.25 & 46813.00 & 483817.49 & 1.05 & 0.98 & 1.03 \\
5449 & 100763 & 2003 & 1649.90 & 0.33 & 152517.00 & 1578765.03 & 1.08 & 0.96 & 1.04 \\
27730 & 105320 & 2003 & 247.50 & 0.40 & 20296.00 & 177949.44 & 1.22 & 0.72 & 0.88 \\
27798 & 105331 & 2003 & 13.00 & 0.40 & 1342.00 & 12443.52 & 0.97 & 0.96 & 0.93 \\
11403 & 101400 & 2003 & 315.90 & 0.35 & 36679.00 & 313620.85 & 0.86 & 0.99 & 0.86 \\
46035 & 200179 & 2003 & 11.10 & 0.25 & 1098.00 & 11237.56 & 1.01 & 1.01 & 1.02 \\
27774 & 105322 & 2003 & 31.90 & 0.26 & 2867.00 & 28667.60 & 1.11 & 0.90 & 1.00 \\
527 & 100072 & 2003 & 7933.30 & 0.29 & 786420.00 & 7999642.89 & 1.01 & 1.01 & 1.02 \\
46029 & 200178 & 2003 & 37.40 & 0.07 & 3734.00 & 36939.02 & 1.00 & 0.99 & 0.99 \\
27756 & 105321 & 2003 & 407.50 & 0.31 & 39182.00 & 381268.32 & 1.04 & 0.94 & 0.97 \\
49876 & 240370 & 2003 & 7.20 & 0.71 & 433.00 & 4088.08 & 1.66 & 0.57 & 0.94 \\
46023 & 200177 & 2003 & 38.90 & 0.02 & 3974.00 & 38178.60 & 0.98 & 0.98 & 0.96 \\
37251 & 106724 & 2003 & 1474.70 & 0.36 & 120716.00 & 1301909.26 & 1.22 & 0.88 & 1.08 \\
37277 & 106725 & 2003 & 5.60 & 0.31 & 558.00 & 5574.65 & 1.00 & 1.00 & 1.00 \\
8432 & 101086 & 2003 & 605.60 & 0.68 & 53297.00 & 494286.74 & 1.14 & 0.82 & 0.93 \\
27954 & 105358 & 2003 & 2521.20 & 0.32 & 218540.00 & 2044688.88 & 1.15 & 0.81 & 0.94 \\
6943 & 100973 & 2003 & 47.10 & 0.32 & 4586.00 & 45854.31 & 1.03 & 0.97 & 1.00 \\
27341 & 105269 & 2003 & 1081.80 & 0.38 & 109928.00 & 1020329.83 & 0.98 & 0.94 & 0.93 \\
17255 & 102274 & 2003 & 5987.50 & 0.32 & 598751.00 & 5600741.09 & 1.00 & 0.94 & 0.94 \\
33076 & 106089 & 2003 & 47.90 & 0.43 & 4774.00 & 38803.00 & 1.00 & 0.81 & 0.81 \\
57936 & 410003 & 2003 & 448.50 & 0.48 & 40432.00 & 464832.42 & 1.11 & 1.04 & 1.15 \\
746 & 100093 & 2003 & 173.80 & 0.42 & 17465.00 & 167562.60 & 1.00 & 0.96 & 0.96 \\
37942 & 107173 & 2003 & 117.90 & 0.32 & 11741.00 & 112477.92 & 1.00 & 0.95 & 0.96 \\
37960 & 107174 & 2003 & 24.40 & 0.33 & 2440.00 & 23370.65 & 1.00 & 0.96 & 0.96 \\
5218 & 100736 & 2003 & 563.40 & 0.28 & 56323.00 & 522570.00 & 1.00 & 0.93 & 0.93 \\
37965 & 107175 & 2003 & 1783.40 & 0.41 & 178138.00 & 1661600.53 & 1.00 & 0.93 & 0.93 \\
26930 & 103628 & 2003 & 1124.40 & 0.30 & 112571.00 & 1094273.41 & 1.00 & 0.97 & 0.97 \\
45780 & 200133 & 2003 & 15.90 & 0.18 & 1592.00 & 15788.92 & 1.00 & 0.99 & 0.99 \\
74441 & 601001 & 2003 & 107.60 & 0.35 & 8116.00 & 85108.41 & 1.33 & 0.79 & 1.05 \\
26917 & 103621 & 2003 & 88.00 & 0.24 & 8608.00 & 85927.88 & 1.02 & 0.98 & 1.00 \\
9464 & 101137 & 2003 & 21.60 & 0.36 & 2189.00 & 22921.62 & 0.99 & 1.06 & 1.05 \\
26965 & 103638 & 2003 & 15.90 & 0.29 & 1574.00 & 15735.63 & 1.01 & 0.99 & 1.00 \\
49660 & 240330 & 2003 & 97.50 & 0.91 & 8624.00 & 81210.69 & 1.13 & 0.83 & 0.94 \\
53645 & 355536 & 2003 & 12.90 & 0.32 & 1229.00 & 11781.23 & 1.05 & 0.91 & 0.96 \\
15149 & 101963 & 2003 & 726.40 & 0.27 & 72742.00 & 666854.93 & 1.00 & 0.92 & 0.92 \\
712 & 100092 & 2003 & 649.90 & 0.45 & 60898.00 & 595752.08 & 1.07 & 0.92 & 0.98 \\
37904 & 107160 & 2003 & 5523.50 & 0.34 & 402656.00 & 4798747.48 & 1.37 & 0.87 & 1.19 \\
47865 & 222408 & 2003 & 905.70 & 0.28 & 77344.00 & 809312.47 & 1.17 & 0.89 & 1.05 \\
27016 & 103644 & 2003 & 38.20 & 0.30 & 3411.00 & 38125.40 & 1.12 & 1.00 & 1.12 \\
17228 & 102271 & 2003 & 792.10 & 0.42 & 81362.00 & 787284.26 & 0.97 & 0.99 & 0.97 \\
11684 & 101456 & 2003 & 39.10 & 0.32 & 3910.00 & 38812.85 & 1.00 & 0.99 & 0.99 \\
5251 & 100741 & 2003 & 153.50 & 0.23 & 15300.00 & 144055.00 & 1.00 & 0.94 & 0.94 \\
49687 & 240332 & 2003 & 84.80 & 0.50 & 8476.00 & 75459.57 & 1.00 & 0.89 & 0.89 \\
32592 & 106042 & 2003 & 113.90 & 0.20 & 11304.00 & 94775.52 & 1.01 & 0.83 & 0.84 \\
45793 & 200140 & 2003 & 1329.00 & 0.31 & 131667.00 & 1195884.06 & 1.01 & 0.90 & 0.91 \\
33085 & 106090 & 2003 & 253.50 & 0.53 & 19655.00 & 205322.26 & 1.29 & 0.81 & 1.04 \\
49683 & 240331 & 2003 & 3.10 & 0.24 & 314.00 & 3042.14 & 0.99 & 0.98 & 0.97 \\
47891 & 222658 & 2003 & 382.50 & 0.40 & 38247.00 & 373330.34 & 1.00 & 0.98 & 0.98 \\
53651 & 355799 & 2003 & 7.30 & 0.37 & 742.00 & 6696.06 & 0.98 & 0.92 & 0.90 \\
8321 & 101082 & 2003 & 1957.30 & 0.16 & 188955.00 & 1993277.80 & 1.04 & 1.02 & 1.05 \\
37977 & 107178 & 2003 & 26.20 & 0.30 & 2561.00 & 25606.94 & 1.02 & 0.98 & 1.00 \\
52397 & 302879 & 2003 & 142.80 & 0.27 & 14261.00 & 141794.56 & 1.00 & 0.99 & 0.99 \\
17313 & 102280 & 2003 & 1666.80 & 0.36 & 166677.00 & 1621234.50 & 1.00 & 0.97 & 0.97 \\
38042 & 107196 & 2003 & 28.10 & 0.33 & 2815.00 & 27117.62 & 1.00 & 0.97 & 0.96 \\
26797 & 103607 & 2003 & 167.50 & 0.41 & 16674.00 & 158983.86 & 1.00 & 0.95 & 0.95 \\
5178 & 100730 & 2003 & 632.90 & 0.40 & 63450.00 & 629872.08 & 1.00 & 1.00 & 0.99 \\
11751 & 101460 & 2003 & 1681.30 & 0.35 & 171051.00 & 1572735.20 & 0.98 & 0.94 & 0.92 \\
45746 & 200097 & 2003 & 7.10 & 0.20 & 709.00 & 7014.74 & 1.00 & 0.99 & 0.99 \\
38067 & 107198 & 2003 & 132.20 & 0.37 & 12914.00 & 129081.37 & 1.02 & 0.98 & 1.00 \\
38083 & 107199 & 2003 & 8.10 & 0.32 & 763.00 & 7625.69 & 1.06 & 0.94 & 1.00 \\
26765 & 103606 & 2003 & 51.10 & 0.30 & 4959.00 & 48387.83 & 1.03 & 0.95 & 0.98 \\
45729 & 200094 & 2003 & 84.60 & 0.32 & 6163.00 & 54980.75 & 1.37 & 0.65 & 0.89 \\
49627 & 240326 & 2003 & 54.40 & 0.90 & 4915.00 & 53037.34 & 1.11 & 0.97 & 1.08 \\
45707 & 200092 & 2003 & 8.00 & 0.34 & 817.00 & 7707.59 & 0.98 & 0.96 & 0.94 \\
806 & 100097 & 2003 & 164.30 & 0.34 & 16063.00 & 163970.62 & 1.02 & 1.00 & 1.02 \\
14357 & 101851 & 2003 & 3385.30 & 0.55 & 289273.00 & 2599189.32 & 1.17 & 0.77 & 0.90 \\
14332 & 101850 & 2003 & 575.90 & 0.30 & 49946.00 & 531670.66 & 1.15 & 0.92 & 1.06 \\
38019 & 107192 & 2003 & 1126.50 & 0.33 & 109575.00 & 1063283.28 & 1.03 & 0.94 & 0.97 \\
26895 & 103620 & 2003 & 381.50 & 0.24 & 37855.00 & 371678.72 & 1.01 & 0.97 & 0.98 \\
49651 & 240328 & 2003 & 6.30 & 0.27 & 605.00 & 6013.16 & 1.04 & 0.95 & 0.99 \\
33049 & 106088 & 2003 & 142.80 & 0.27 & 15871.00 & 148867.79 & 0.90 & 1.04 & 0.94 \\
776 & 100096 & 2003 & 97.70 & 0.34 & 9172.00 & 95968.03 & 1.07 & 0.98 & 1.05 \\
37987 & 107179 & 2003 & 1093.50 & 0.27 & 110583.00 & 1076911.37 & 0.99 & 0.98 & 0.97 \\
37999 & 107181 & 2003 & 106.30 & 0.33 & 10641.00 & 100236.39 & 1.00 & 0.94 & 0.94 \\
26863 & 103614 & 2003 & 220.40 & 0.22 & 21997.00 & 219968.07 & 1.00 & 1.00 & 1.00 \\
5200 & 100731 & 2003 & 7913.50 & 0.27 & 791380.00 & 7718648.88 & 1.00 & 0.98 & 0.98 \\
15012 & 101933 & 2003 & 151.40 & 0.41 & 15062.00 & 146534.38 & 1.01 & 0.97 & 0.97 \\
49635 & 240327 & 2003 & 100.50 & 0.42 & 12147.00 & 98894.60 & 0.83 & 0.98 & 0.81 \\
26847 & 103609 & 2003 & 32.80 & 0.40 & 3270.00 & 31534.83 & 1.00 & 0.96 & 0.96 \\
17295 & 102278 & 2003 & 142.50 & 0.31 & 15715.00 & 137521.36 & 0.91 & 0.97 & 0.88 \\
53656 & 355965 & 2003 & 1381.40 & 0.35 & 122471.00 & 1325354.95 & 1.13 & 0.96 & 1.08 \\
33022 & 106086 & 2003 & 34.10 & 0.45 & 2991.00 & 32410.66 & 1.14 & 0.95 & 1.08 \\
26824 & 103608 & 2003 & 34.10 & 0.27 & 3403.00 & 33372.99 & 1.00 & 0.98 & 0.98 \\
45884 & 200153 & 2003 & 60.50 & 0.23 & 6079.00 & 60372.98 & 1.00 & 1.00 & 0.99 \\
5273 & 100745 & 2003 & 956.90 & 0.16 & 95169.00 & 897342.16 & 1.01 & 0.94 & 0.94 \\
37897 & 107159 & 2003 & 17.60 & 0.23 & 1765.00 & 17143.95 & 1.00 & 0.97 & 0.97 \\
5327 & 100753 & 2003 & 1080.80 & 0.27 & 107083.00 & 1077706.28 & 1.01 & 1.00 & 1.01 \\
37673 & 106993 & 2003 & 119.50 & 0.55 & 8959.00 & 100905.13 & 1.33 & 0.84 & 1.13 \\
47815 & 222027 & 2003 & 2131.80 & 0.37 & 187882.00 & 2061662.08 & 1.13 & 0.97 & 1.10 \\
37684 & 106995 & 2003 & 1769.80 & 0.39 & 174980.00 & 1617387.83 & 1.01 & 0.91 & 0.92 \\
27249 & 105259 & 2003 & 190.60 & 0.30 & 19041.00 & 189510.54 & 1.00 & 0.99 & 1.00 \\
6508 & 100878 & 2003 & 2377.50 & 0.23 & 237649.00 & 2296150.80 & 1.00 & 0.97 & 0.97 \\
634 & 100085 & 2003 & 9206.40 & 0.25 & 952945.00 & 8841207.83 & 0.97 & 0.96 & 0.93 \\
11603 & 101431 & 2003 & 320.30 & 0.33 & 31873.00 & 312560.58 & 1.00 & 0.98 & 0.98 \\
47840 & 222351 & 2003 & 383.50 & 0.42 & 39970.00 & 366431.09 & 0.96 & 0.96 & 0.92 \\
37714 & 107004 & 2003 & 107.40 & 0.27 & 10575.00 & 107979.48 & 1.02 & 1.01 & 1.02 \\
37721 & 107135 & 2003 & 234.90 & 0.36 & 23510.00 & 232573.24 & 1.00 & 0.99 & 0.99 \\
27222 & 105256 & 2003 & 192.40 & 0.41 & 19252.00 & 182856.74 & 1.00 & 0.95 & 0.95 \\
32564 & 106041 & 2003 & 364.30 & 0.47 & 34210.00 & 305880.67 & 1.06 & 0.84 & 0.89 \\
49741 & 240352 & 2003 & 2.40 & 0.33 & 228.00 & 2281.19 & 1.05 & 0.95 & 1.00 \\
33126 & 106092 & 2003 & 603.30 & 0.28 & 60625.00 & 593579.78 & 1.00 & 0.98 & 0.98 \\
8362 & 101084 & 2003 & 2522.90 & 0.59 & 223490.00 & 2356809.60 & 1.13 & 0.93 & 1.05 \\
27278 & 105260 & 2003 & 266.80 & 0.35 & 27092.00 & 258111.36 & 0.98 & 0.97 & 0.95 \\
49764 & 240358 & 2003 & 27.90 & 0.28 & 2788.00 & 27384.83 & 1.00 & 0.98 & 0.98 \\
49761 & 240356 & 2003 & 3.40 & 0.42 & 274.00 & 3399.82 & 1.24 & 1.00 & 1.24 \\
37643 & 106983 & 2003 & 204.70 & 0.34 & 20790.00 & 199499.54 & 0.98 & 0.97 & 0.96 \\
32548 & 106039 & 2003 & 1925.20 & 0.21 & 192427.00 & 1807985.67 & 1.00 & 0.94 & 0.94 \\
27316 & 105268 & 2003 & 587.30 & 0.26 & 58746.00 & 580686.31 & 1.00 & 0.99 & 0.99 \\
17109 & 102257 & 2003 & 1252.50 & 0.28 & 112333.00 & 1227789.19 & 1.11 & 0.98 & 1.09 \\
49758 & 240355 & 2003 & 4.00 & 0.47 & 324.00 & 4040.52 & 1.23 & 1.01 & 1.25 \\
49755 & 240354 & 2003 & 17.40 & 0.25 & 1742.00 & 16568.96 & 1.00 & 0.95 & 0.95 \\
37649 & 106984 & 2003 & 85.20 & 0.26 & 8562.00 & 83903.44 & 1.00 & 0.98 & 0.98 \\
55462 & 400100 & 2003 & 96.80 & 0.28 & 7985.00 & 88315.13 & 1.21 & 0.91 & 1.11 \\
37656 & 106985 & 2003 & 166.90 & 0.28 & 16956.00 & 163814.61 & 0.98 & 0.98 & 0.97 \\
55737 & 400144 & 2003 & 148.60 & 0.32 & 16828.00 & 175238.99 & 0.88 & 1.18 & 1.04 \\
27042 & 103645 & 2003 & 324.60 & 0.33 & 32097.00 & 267356.06 & 1.01 & 0.82 & 0.83 \\
49732 & 240345 & 2003 & 72.50 & 0.20 & 5726.00 & 50913.47 & 1.27 & 0.70 & 0.89 \\
660 & 100087 & 2003 & 2822.20 & 0.31 & 281045.00 & 2838475.00 & 1.00 & 1.01 & 1.01 \\
688 & 100090 & 2003 & 579.80 & 0.34 & 58341.00 & 567343.57 & 0.99 & 0.98 & 0.97 \\
17194 & 102270 & 2003 & 721.60 & 0.19 & 72234.00 & 709010.13 & 1.00 & 0.98 & 0.98 \\
27100 & 103652 & 2003 & 102.60 & 0.44 & 10428.00 & 99994.32 & 0.98 & 0.97 & 0.96 \\
74592 & 601139 & 2003 & 5280.90 & 0.28 & 483941.00 & 5016173.80 & 1.09 & 0.95 & 1.04 \\
45830 & 200147 & 2003 & 25.50 & 0.16 & 2576.00 & 26463.72 & 0.99 & 1.04 & 1.03 \\
27072 & 103647 & 2003 & 26.10 & 0.58 & 3223.00 & 30691.62 & 0.81 & 1.18 & 0.95 \\
49693 & 240333 & 2003 & 55.10 & 0.36 & 4830.00 & 52471.71 & 1.14 & 0.95 & 1.09 \\
37866 & 107152 & 2003 & 84.30 & 0.26 & 7362.00 & 84273.12 & 1.15 & 1.00 & 1.14 \\
37883 & 107156 & 2003 & 107.70 & -0.01 & 9422.00 & 94221.51 & 1.14 & 0.87 & 1.00 \\
49719 & 240337 & 2003 & 9.40 & 0.51 & 958.00 & 9486.25 & 0.98 & 1.01 & 0.99 \\
14407 & 101854 & 2003 & 13877.50 & 0.44 & 1175506.00 & 12460516.84 & 1.18 & 0.90 & 1.06 \\
27119 & 103658 & 2003 & 102.10 & 0.26 & 10240.00 & 101182.83 & 1.00 & 0.99 & 0.99 \\
55446 & 400097 & 2003 & 9.00 & 0.30 & 864.00 & 8635.82 & 1.04 & 0.96 & 1.00 \\
17165 & 102261 & 2003 & 1326.60 & 0.31 & 116493.00 & 1307962.00 & 1.14 & 0.99 & 1.12 \\
37752 & 107141 & 2003 & 1418.00 & 0.51 & 142165.00 & 1315960.79 & 1.00 & 0.93 & 0.93 \\
45847 & 200148 & 2003 & 100.00 & 0.17 & 10541.00 & 94661.30 & 0.95 & 0.95 & 0.90 \\
9283 & 101127 & 2003 & 162.50 & 0.36 & 13497.00 & 154098.36 & 1.20 & 0.95 & 1.14 \\
74617 & 601140 & 2003 & 17.40 & 0.33 & 1543.00 & 16502.22 & 1.13 & 0.95 & 1.07 \\
5294 & 100746 & 2003 & 1432.40 & 0.36 & 144086.00 & 1408815.30 & 0.99 & 0.98 & 0.98 \\
37780 & 107144 & 2003 & 61.80 & 0.33 & 6179.00 & 61144.27 & 1.00 & 0.99 & 0.99 \\
27135 & 105246 & 2003 & 5734.90 & 0.29 & 570975.00 & 5209583.74 & 1.00 & 0.91 & 0.91 \\
55448 & 400099 & 2003 & 8.70 & 0.58 & 889.00 & 8343.07 & 0.98 & 0.96 & 0.94 \\
38117 & 107202 & 2003 & 21.60 & 0.49 & 1876.00 & 21189.91 & 1.15 & 0.98 & 1.13 \\
49977 & 240380 & 2003 & 10.20 & 0.47 & 1022.00 & 10139.29 & 1.00 & 0.99 & 0.99 \\
28810 & 105478 & 2003 & 55.90 & 0.30 & 5587.00 & 46934.86 & 1.00 & 0.84 & 0.84 \\
342 & 100040 & 2003 & 2164.50 & 0.53 & 216574.00 & 2162970.58 & 1.00 & 1.00 & 1.00 \\
14662 & 101906 & 2003 & 8.00 & 0.42 & 834.00 & 8021.21 & 0.96 & 1.00 & 0.96 \\
36367 & 106519 & 2003 & 110.40 & 0.42 & 11023.00 & 109143.49 & 1.00 & 0.99 & 0.99 \\
16520 & 102152 & 2003 & 258.40 & 0.19 & 25836.00 & 244171.20 & 1.00 & 0.94 & 0.95 \\
11027 & 101360 & 2003 & 1959.90 & 0.31 & 195771.00 & 1745973.59 & 1.00 & 0.89 & 0.89 \\
32335 & 106011 & 2003 & 419.60 & 0.41 & 42185.00 & 418215.39 & 0.99 & 1.00 & 0.99 \\
28779 & 105476 & 2003 & 320.20 & 0.37 & 32288.00 & 315485.64 & 0.99 & 0.99 & 0.98 \\
46336 & 200223 & 2003 & 76.40 & 0.47 & 7589.00 & 72027.17 & 1.01 & 0.94 & 0.95 \\
36416 & 106524 & 2003 & 14.50 & 0.41 & 1285.00 & 11969.52 & 1.13 & 0.83 & 0.93 \\
28750 & 105475 & 2003 & 747.40 & 0.36 & 74857.00 & 745964.40 & 1.00 & 1.00 & 1.00 \\
36353 & 106487 & 2003 & 10.30 & 0.33 & 1081.00 & 10813.80 & 0.95 & 1.05 & 1.00 \\
50401 & 240415 & 2003 & 22.60 & 0.22 & 1951.00 & 19590.84 & 1.16 & 0.87 & 1.00 \\
36337 & 106485 & 2003 & 123.60 & 0.57 & 12287.00 & 122875.01 & 1.01 & 0.99 & 1.00 \\
36301 & 106482 & 2003 & 197.80 & 0.37 & 19776.00 & 187388.16 & 1.00 & 0.95 & 0.95 \\
28885 & 105502 & 2003 & 1456.20 & 0.29 & 139354.00 & 1211701.63 & 1.04 & 0.83 & 0.87 \\
46352 & 200225 & 2003 & 8.90 & 0.36 & 881.00 & 8676.23 & 1.01 & 0.97 & 0.98 \\
28875 & 105498 & 2003 & 73.20 & 0.27 & 7283.00 & 70974.30 & 1.01 & 0.97 & 0.97 \\
5755 & 100791 & 2003 & 5066.30 & 0.31 & 477718.00 & 4544096.81 & 1.06 & 0.90 & 0.95 \\
74809 & 601178 & 2003 & 53.90 & 0.22 & 4604.00 & 48910.50 & 1.17 & 0.91 & 1.06 \\
10987 & 101358 & 2003 & 411.70 & 0.28 & 41152.00 & 407485.90 & 1.00 & 0.99 & 0.99 \\
50473 & 240422 & 2003 & 30.30 & 0.03 & 3032.00 & 29754.35 & 1.00 & 0.98 & 0.98 \\
28857 & 105487 & 2003 & 146.60 & 0.25 & 14196.00 & 152363.22 & 1.03 & 1.04 & 1.07 \\
47303 & 200505 & 2003 & 484.30 & 0.28 & 40632.00 & 412255.40 & 1.19 & 0.85 & 1.01 \\
33351 & 106123 & 2003 & 236.70 & 0.24 & 23680.00 & 229910.13 & 1.00 & 0.97 & 0.97 \\
50452 & 240421 & 2003 & 273.30 & 0.39 & 21221.00 & 196977.03 & 1.29 & 0.72 & 0.93 \\
16498 & 102151 & 2003 & 8.40 & 0.46 & 835.00 & 8091.56 & 1.01 & 0.96 & 0.97 \\
36327 & 106483 & 2003 & 12.90 & 0.29 & 1318.00 & 11176.35 & 0.98 & 0.87 & 0.85 \\
46344 & 200224 & 2003 & 23.50 & 0.40 & 2345.00 & 23375.45 & 1.00 & 0.99 & 1.00 \\
36442 & 106528 & 2003 & 331.40 & 0.39 & 26172.00 & 312627.83 & 1.27 & 0.94 & 1.19 \\
6794 & 100954 & 2003 & 950.80 & 0.29 & 95323.00 & 950727.43 & 1.00 & 1.00 & 1.00 \\
28653 & 105458 & 2003 & 503.40 & 0.50 & 78097.00 & 749188.22 & 0.64 & 1.49 & 0.96 \\
46319 & 200211 & 2003 & 5.70 & 0.40 & 430.00 & 3979.64 & 1.33 & 0.70 & 0.93 \\
28638 & 105457 & 2003 & 2656.20 & 0.26 & 339511.00 & 3332615.48 & 0.78 & 1.25 & 0.98 \\
36472 & 106535 & 2003 & 289.70 & 0.45 & 29166.00 & 300738.38 & 0.99 & 1.04 & 1.03 \\
46304 & 200210 & 2003 & 12.50 & 0.21 & 1276.00 & 11934.57 & 0.98 & 0.95 & 0.94 \\
6640 & 100906 & 2003 & 1827.10 & 0.65 & 182713.00 & 1815427.99 & 1.00 & 0.99 & 0.99 \\
55614 & 400131 & 2003 & 68.10 & 0.38 & 5942.00 & 62031.90 & 1.15 & 0.91 & 1.04 \\
28617 & 105450 & 2003 & 21.00 & 0.16 & 2040.00 & 20589.81 & 1.03 & 0.98 & 1.01 \\
11091 & 101367 & 2003 & 621.50 & 0.38 & 61967.00 & 585503.18 & 1.00 & 0.94 & 0.94 \\
36500 & 106541 & 2003 & 480.20 & 0.46 & 48518.00 & 463122.55 & 0.99 & 0.96 & 0.95 \\
50284 & 240409 & 2003 & 27.30 & 0.49 & 1712.00 & 14677.37 & 1.59 & 0.54 & 0.86 \\
14642 & 101904 & 2003 & 8.40 & 0.39 & 835.00 & 7915.60 & 1.01 & 0.94 & 0.95 \\
36512 & 106545 & 2003 & 23.20 & 0.52 & 2310.00 & 21873.88 & 1.00 & 0.94 & 0.95 \\
28728 & 105472 & 2003 & 186.40 & 0.49 & 18183.00 & 181425.14 & 1.03 & 0.97 & 1.00 \\
36466 & 106533 & 2003 & 14.90 & 0.23 & 2017.00 & 20962.81 & 0.74 & 1.41 & 1.04 \\
50393 & 240414 & 2003 & 269.90 & 0.27 & 27055.00 & 258319.33 & 1.00 & 0.96 & 0.95 \\
74798 & 601172 & 2003 & 412.60 & 0.39 & 36445.00 & 383937.89 & 1.13 & 0.93 & 1.05 \\
28707 & 105469 & 2003 & 89.30 & 0.34 & 7360.00 & 81740.72 & 1.21 & 0.92 & 1.11 \\
16543 & 102154 & 2003 & 211.50 & 0.21 & 21211.00 & 208601.82 & 1.00 & 0.99 & 0.98 \\
5731 & 100790 & 2003 & 263.30 & 0.32 & 23361.00 & 228375.55 & 1.13 & 0.87 & 0.98 \\
46324 & 200213 & 2003 & 8.90 & 0.32 & 888.00 & 8724.11 & 1.00 & 0.98 & 0.98 \\
15255 & 101972 & 2003 & 3289.90 & 0.43 & 329001.00 & 2963986.61 & 1.00 & 0.90 & 0.90 \\
11059 & 101364 & 2003 & 301.30 & 0.28 & 30018.00 & 254663.50 & 1.00 & 0.85 & 0.85 \\
28682 & 105463 & 2003 & 1186.80 & 0.58 & 92459.00 & 1123676.27 & 1.28 & 0.95 & 1.22 \\
14652 & 101905 & 2003 & 10.80 & 0.04 & 931.00 & 9794.09 & 1.16 & 0.91 & 1.05 \\
33313 & 106114 & 2003 & 175.60 & 0.30 & 15175.00 & 170855.18 & 1.16 & 0.97 & 1.13 \\
36451 & 106529 & 2003 & 178.80 & 0.45 & 18289.00 & 172955.39 & 0.98 & 0.97 & 0.95 \\
50281 & 240408 & 2003 & 1.60 & 0.14 & 163.00 & 1629.98 & 0.98 & 1.02 & 1.00 \\
325 & 100036 & 2003 & 52.30 & 0.32 & 5302.00 & 51863.24 & 0.99 & 0.99 & 0.98 \\
29151 & 105533 & 2003 & 216.30 & 0.37 & 21880.00 & 207263.21 & 0.99 & 0.96 & 0.95 \\
56069 & 400171 & 2003 & 42.20 & 0.30 & 4588.00 & 41455.59 & 0.92 & 0.98 & 0.90 \\
36052 & 106451 & 2003 & 156.40 & 0.29 & 23231.00 & 216532.00 & 0.67 & 1.38 & 0.93 \\
29131 & 105531 & 2003 & 188.30 & 0.53 & 18911.00 & 187243.00 & 1.00 & 0.99 & 0.99 \\
47678 & 217585 & 2003 & 292.90 & 0.28 & 29291.00 & 290464.60 & 1.00 & 0.99 & 0.99 \\
10900 & 101345 & 2003 & 3081.60 & 0.27 & 281523.00 & 2453715.01 & 1.09 & 0.80 & 0.87 \\
50602 & 240430 & 2003 & 57.50 & 0.34 & 5718.00 & 58151.03 & 1.01 & 1.01 & 1.02 \\
29108 & 105527 & 2003 & 6.50 & 0.34 & 586.00 & 6082.00 & 1.11 & 0.94 & 1.04 \\
36087 & 106461 & 2003 & 407.50 & 0.33 & 40772.00 & 372893.56 & 1.00 & 0.92 & 0.91 \\
52270 & 302731 & 2003 & 2125.80 & 0.48 & 213509.00 & 2033199.55 & 1.00 & 0.96 & 0.95 \\
36106 & 106464 & 2003 & 358.50 & 0.53 & 31073.00 & 313746.03 & 1.15 & 0.88 & 1.01 \\
29080 & 105525 & 2003 & 2725.60 & 0.29 & 174685.00 & 1666131.02 & 1.56 & 0.61 & 0.95 \\
16417 & 102134 & 2003 & 48.80 & 0.26 & 7130.00 & 73437.79 & 0.68 & 1.50 & 1.03 \\
5787 & 100792 & 2003 & 582.50 & 0.42 & 55898.00 & 543103.95 & 1.04 & 0.93 & 0.97 \\
29070 & 105523 & 2003 & 91.90 & 0.32 & 8909.00 & 90417.59 & 1.03 & 0.98 & 1.01 \\
36037 & 106449 & 2003 & 156.30 & 0.62 & 14090.00 & 164006.76 & 1.11 & 1.05 & 1.16 \\
46417 & 200239 & 2003 & 2.80 & 0.39 & 266.00 & 2246.09 & 1.05 & 0.80 & 0.84 \\
52244 & 302698 & 2003 & 111.10 & 0.24 & 11100.00 & 101107.17 & 1.00 & 0.91 & 0.91 \\
277 & 100030 & 2003 & 178.70 & 0.33 & 18144.00 & 181485.09 & 0.98 & 1.02 & 1.00 \\
16365 & 102130 & 2003 & 358.20 & 0.32 & 33610.00 & 335421.84 & 1.07 & 0.94 & 1.00 \\
33403 & 106129 & 2003 & 509.40 & 0.25 & 50663.00 & 497196.91 & 1.01 & 0.98 & 0.98 \\
50657 & 240435 & 2003 & 21.30 & 0.15 & 2222.00 & 20334.25 & 0.96 & 0.95 & 0.92 \\
46424 & 200243 & 2003 & 1.30 & 0.42 & 133.00 & 1244.73 & 0.98 & 0.96 & 0.94 \\
35988 & 106444 & 2003 & 210.80 & 0.26 & 21091.00 & 202185.92 & 1.00 & 0.96 & 0.96 \\
293 & 100033 & 2003 & 281.00 & 0.39 & 30973.00 & 294716.47 & 0.91 & 1.05 & 0.95 \\
29201 & 105536 & 2003 & 253.80 & 0.30 & 25202.00 & 245826.44 & 1.01 & 0.97 & 0.98 \\
29185 & 105535 & 2003 & 194.10 & 0.30 & 19249.00 & 188668.60 & 1.01 & 0.97 & 0.98 \\
46419 & 200241 & 2003 & 1.60 & 0.34 & 155.00 & 1535.04 & 1.03 & 0.96 & 0.99 \\
46359 & 200227 & 2003 & 14.20 & 0.33 & 1424.00 & 13560.72 & 1.00 & 0.95 & 0.95 \\
36141 & 106470 & 2003 & 318.60 & 0.19 & 31953.00 & 314259.06 & 1.00 & 0.99 & 0.98 \\
36196 & 106477 & 2003 & 1970.30 & 0.31 & 169591.00 & 1707746.35 & 1.16 & 0.87 & 1.01 \\
28962 & 105508 & 2003 & 84.60 & 0.66 & 8452.00 & 83193.96 & 1.00 & 0.98 & 0.98 \\
46382 & 200228 & 2003 & 2.30 & 0.29 & 199.00 & 2025.98 & 1.16 & 0.88 & 1.02 \\
8633 & 101092 & 2003 & 425.90 & 0.37 & 35982.00 & 418910.17 & 1.18 & 0.98 & 1.16 \\
36219 & 106478 & 2003 & 218.60 & 0.38 & 18147.00 & 189962.29 & 1.20 & 0.87 & 1.05 \\
28943 & 105507 & 2003 & 955.30 & 0.28 & 96455.00 & 941798.50 & 0.99 & 0.99 & 0.98 \\
32310 & 106010 & 2003 & 2163.10 & 0.25 & 183725.00 & 1863855.21 & 1.18 & 0.86 & 1.01 \\
36242 & 106479 & 2003 & 73.60 & 0.42 & 6388.00 & 57078.55 & 1.15 & 0.78 & 0.89 \\
28927 & 105506 & 2003 & 808.50 & 0.24 & 81015.00 & 796535.60 & 1.00 & 0.99 & 0.98 \\
16467 & 102150 & 2003 & 64.10 & 0.30 & 6406.00 & 53510.77 & 1.00 & 0.83 & 0.84 \\
10961 & 101356 & 2003 & 412.30 & 0.42 & 41342.00 & 410304.10 & 1.00 & 1.00 & 0.99 \\
36268 & 106480 & 2003 & 924.30 & 0.43 & 74043.00 & 727494.35 & 1.25 & 0.79 & 0.98 \\
36170 & 106476 & 2003 & 11.50 & 0.37 & 1079.00 & 10792.27 & 1.07 & 0.94 & 1.00 \\
36132 & 106467 & 2003 & 190.30 & 0.29 & 16746.00 & 167372.36 & 1.14 & 0.88 & 1.00 \\
29041 & 105522 & 2003 & 371.20 & 0.34 & 32044.00 & 339392.86 & 1.16 & 0.91 & 1.06 \\
46399 & 200236 & 2003 & 24.30 & 0.53 & 1950.00 & 19372.74 & 1.25 & 0.80 & 0.99 \\
74813 & 601179 & 2003 & 53.20 & 0.28 & 4531.00 & 48376.06 & 1.17 & 0.91 & 1.07 \\
46396 & 200233 & 2003 & 4.80 & 0.24 & 480.00 & 4597.92 & 1.00 & 0.96 & 0.96 \\
10929 & 101354 & 2003 & 1020.50 & 0.22 & 102600.00 & 1017073.55 & 0.99 & 1.00 & 0.99 \\
32297 & 106009 & 2003 & 1824.40 & 0.25 & 173753.00 & 1806182.51 & 1.05 & 0.99 & 1.04 \\
46392 & 200232 & 2003 & 3.00 & 0.26 & 303.00 & 2974.68 & 0.99 & 0.99 & 0.98 \\
36152 & 106471 & 2003 & 141.30 & 0.28 & 13697.00 & 139408.24 & 1.03 & 0.99 & 1.02 \\
33384 & 106127 & 2003 & 465.30 & 0.33 & 35436.00 & 382160.60 & 1.31 & 0.82 & 1.08 \\
47689 & 220681 & 2003 & 654.70 & 0.20 & 65102.00 & 558870.34 & 1.01 & 0.85 & 0.86 \\
36160 & 106474 & 2003 & 234.30 & 0.28 & 19991.00 & 214528.19 & 1.17 & 0.92 & 1.07 \\
29006 & 105512 & 2003 & 11.80 & 0.30 & 1183.00 & 10846.48 & 1.00 & 0.92 & 0.92 \\
14676 & 101908 & 2003 & 224.80 & 0.45 & 22381.00 & 213777.59 & 1.00 & 0.95 & 0.96 \\
50537 & 240427 & 2003 & 42.60 & 1.28 & 3145.00 & 32028.10 & 1.35 & 0.75 & 1.02 \\
16445 & 102145 & 2003 & 168.50 & 0.18 & 14716.00 & 156365.39 & 1.15 & 0.93 & 1.06 \\
28988 & 105510 & 2003 & 19.70 & 0.37 & 1958.00 & 18692.42 & 1.01 & 0.95 & 0.95 \\
16850 & 102197 & 2003 & 774.60 & 0.38 & 77342.00 & 688429.72 & 1.00 & 0.89 & 0.89 \\
28588 & 105448 & 2003 & 552.90 & 0.39 & 43150.00 & 428333.82 & 1.28 & 0.77 & 0.99 \\
16594 & 102157 & 2003 & 4.20 & 0.41 & 426.00 & 3614.14 & 0.99 & 0.86 & 0.85 \\
50060 & 240388 & 2003 & 12.60 & 0.16 & 1256.00 & 11744.48 & 1.00 & 0.93 & 0.94 \\
36854 & 106605 & 2003 & 211.70 & 0.44 & 21614.00 & 201639.44 & 0.98 & 0.95 & 0.93 \\
11252 & 101379 & 2003 & 857.00 & 0.28 & 85618.00 & 851459.95 & 1.00 & 0.99 & 0.99 \\
36875 & 106619 & 2003 & 230.40 & 0.48 & 22443.00 & 220692.35 & 1.03 & 0.96 & 0.98 \\
28173 & 105390 & 2003 & 503.20 & 0.43 & 50227.00 & 497486.46 & 1.00 & 0.99 & 0.99 \\
15220 & 101968 & 2003 & 111.80 & 0.25 & 11177.00 & 105500.57 & 1.00 & 0.94 & 0.94 \\
11263 & 101380 & 2003 & 287.10 & 0.38 & 28691.00 & 285592.33 & 1.00 & 0.99 & 1.00 \\
465 & 100068 & 2003 & 70.90 & 0.29 & 7154.00 & 67945.68 & 0.99 & 0.96 & 0.95 \\
16778 & 102191 & 2003 & 60.30 & 0.17 & 5151.00 & 53747.45 & 1.17 & 0.89 & 1.04 \\
36881 & 106620 & 2003 & 230.70 & 0.33 & 22078.00 & 195538.41 & 1.04 & 0.85 & 0.89 \\
36907 & 106627 & 2003 & 1033.80 & 0.28 & 92074.00 & 931259.05 & 1.12 & 0.90 & 1.01 \\
33250 & 106108 & 2003 & 73.90 & 0.30 & 6284.00 & 62307.71 & 1.18 & 0.84 & 0.99 \\
28141 & 105384 & 2003 & 39.40 & 0.25 & 3498.00 & 37829.74 & 1.13 & 0.96 & 1.08 \\
55560 & 400125 & 2003 & 46.80 & 0.00 & 4688.00 & 46617.87 & 1.00 & 1.00 & 0.99 \\
6882 & 100967 & 2003 & 578.10 & 0.25 & 56868.00 & 480953.52 & 1.02 & 0.83 & 0.85 \\
53463 & 350572 & 2003 & 44.60 & 0.19 & 4082.00 & 45461.32 & 1.09 & 1.02 & 1.11 \\
28202 & 105391 & 2003 & 10.80 & 0.32 & 1018.00 & 10440.42 & 1.06 & 0.97 & 1.03 \\
74765 & 601163 & 2003 & 33.10 & 0.47 & 2973.00 & 30453.49 & 1.11 & 0.92 & 1.02 \\
14570 & 101885 & 2003 & 1412.60 & 0.24 & 136970.00 & 1288525.27 & 1.03 & 0.91 & 0.94 \\
33271 & 106109 & 2003 & 61.30 & 0.41 & 5159.00 & 46715.60 & 1.19 & 0.76 & 0.91 \\
36819 & 106597 & 2003 & 170.60 & 0.28 & 17085.00 & 168992.87 & 1.00 & 0.99 & 0.99 \\
11222 & 101376 & 2003 & 1405.60 & 0.12 & 144356.00 & 1442985.30 & 0.97 & 1.03 & 1.00 \\
28251 & 105399 & 2003 & 103.30 & 0.88 & 7745.00 & 87302.29 & 1.33 & 0.85 & 1.13 \\
50091 & 240392 & 2003 & 355.20 & 0.24 & 35521.00 & 351139.48 & 1.00 & 0.99 & 0.99 \\
28238 & 105397 & 2003 & 71.60 & 0.39 & 6302.00 & 67405.91 & 1.14 & 0.94 & 1.07 \\
5607 & 100773 & 2003 & 2684.50 & 0.61 & 273327.00 & 2647319.99 & 0.98 & 0.99 & 0.97 \\
74761 & 601162 & 2003 & 153.00 & 0.40 & 13203.00 & 132030.34 & 1.16 & 0.86 & 1.00 \\
50066 & 240391 & 2003 & 255.50 & 0.29 & 24973.00 & 249598.79 & 1.02 & 0.98 & 1.00 \\
53447 & 350408 & 2003 & 27.00 & 0.31 & 2272.00 & 25071.85 & 1.19 & 0.93 & 1.10 \\
28127 & 105383 & 2003 & 158.60 & 0.24 & 14710.00 & 151769.40 & 1.08 & 0.96 & 1.03 \\
8497 & 101088 & 2003 & 3932.00 & 0.72 & 344516.00 & 3391864.02 & 1.14 & 0.86 & 0.98 \\
50003 & 240382 & 2003 & 69.50 & 0.43 & 6954.00 & 61276.75 & 1.00 & 0.88 & 0.88 \\
28018 & 105369 & 2003 & 216.30 & 0.27 & 20749.00 & 220561.53 & 1.04 & 1.02 & 1.06 \\
46106 & 200190 & 2003 & 51.00 & 0.02 & 5113.00 & 45473.36 & 1.00 & 0.89 & 0.89 \\
9221 & 101119 & 2003 & 328.40 & 0.24 & 32794.00 & 292351.42 & 1.00 & 0.89 & 0.89 \\
37063 & 106655 & 2003 & 56.10 & 0.56 & 5581.00 & 53409.59 & 1.01 & 0.95 & 0.96 \\
11319 & 101393 & 2003 & 731.60 & 0.23 & 72401.00 & 723311.61 & 1.01 & 0.99 & 1.00 \\
47748 & 221051 & 2003 & 4726.20 & 0.24 & 462789.00 & 4539748.70 & 1.02 & 0.96 & 0.98 \\
37080 & 106667 & 2003 & 787.10 & 0.37 & 64799.00 & 525524.87 & 1.21 & 0.67 & 0.81 \\
27986 & 105364 & 2003 & 146.10 & 0.30 & 14660.00 & 145854.62 & 1.00 & 1.00 & 0.99 \\
5509 & 100769 & 2003 & 4725.10 & 0.60 & 480041.00 & 4649897.87 & 0.98 & 0.98 & 0.97 \\
49981 & 240381 & 2003 & 38.40 & 0.40 & 3842.00 & 35889.50 & 1.00 & 0.93 & 0.93 \\
491 & 100071 & 2003 & 3413.00 & 0.24 & 332628.00 & 3230035.30 & 1.03 & 0.95 & 0.97 \\
32448 & 106028 & 2003 & 1228.90 & 0.48 & 123220.00 & 1126842.53 & 1.00 & 0.92 & 0.91 \\
37089 & 106675 & 2003 & 68.20 & 0.29 & 6820.00 & 68128.44 & 1.00 & 1.00 & 1.00 \\
33240 & 106107 & 2003 & 38.50 & 0.19 & 3500.00 & 35855.83 & 1.10 & 0.93 & 1.02 \\
37037 & 106654 & 2003 & 512.60 & 0.28 & 43579.00 & 486075.32 & 1.18 & 0.95 & 1.12 \\
16792 & 102192 & 2003 & 1156.60 & 0.48 & 115847.00 & 1093014.77 & 1.00 & 0.95 & 0.94 \\
16821 & 102193 & 2003 & 117.20 & 0.25 & 11312.00 & 123019.65 & 1.04 & 1.05 & 1.09 \\
50014 & 240383 & 2003 & 116.40 & 0.54 & 21271.00 & 187228.17 & 0.55 & 1.61 & 0.88 \\
36925 & 106640 & 2003 & 106.70 & 0.23 & 9462.00 & 100582.35 & 1.13 & 0.94 & 1.06 \\
36937 & 106642 & 2003 & 990.70 & 0.41 & 94689.00 & 946862.42 & 1.05 & 0.96 & 1.00 \\
28098 & 105382 & 2003 & 117.80 & 0.23 & 13884.00 & 140800.88 & 0.85 & 1.20 & 1.01 \\
11287 & 101390 & 2003 & 3940.30 & 0.24 & 393823.00 & 3896445.04 & 1.00 & 0.99 & 0.99 \\
28079 & 105379 & 2003 & 629.70 & 0.27 & 56802.00 & 559318.22 & 1.11 & 0.89 & 0.98 \\
50042 & 240386 & 2003 & 10.40 & 0.19 & 1031.00 & 9489.14 & 1.01 & 0.91 & 0.92 \\
74720 & 601156 & 2003 & 101.70 & 0.36 & 10083.00 & 96058.42 & 1.01 & 0.94 & 0.95 \\
50039 & 240385 & 2003 & 1.60 & 0.26 & 154.00 & 1496.93 & 1.04 & 0.94 & 0.97 \\
5540 & 100771 & 2003 & 822.30 & 0.36 & 78585.00 & 781951.29 & 1.05 & 0.95 & 1.00 \\
36981 & 106644 & 2003 & 1.80 & 0.24 & 164.00 & 1597.62 & 1.10 & 0.89 & 0.97 \\
50036 & 240384 & 2003 & 5.50 & 0.57 & 563.00 & 5465.45 & 0.98 & 0.99 & 0.97 \\
37018 & 106650 & 2003 & 70.20 & 0.47 & 6998.00 & 69450.64 & 1.00 & 0.99 & 0.99 \\
28047 & 105370 & 2003 & 83.30 & 0.37 & 8035.00 & 83714.49 & 1.04 & 1.00 & 1.04 \\
55538 & 400117 & 2003 & 27.10 & 0.27 & 2367.00 & 24553.34 & 1.14 & 0.91 & 1.04 \\
32363 & 106014 & 2003 & 74.80 & 0.45 & 6801.00 & 71704.87 & 1.10 & 0.96 & 1.05 \\
28280 & 105400 & 2003 & 362.60 & 0.54 & 28138.00 & 352117.65 & 1.29 & 0.97 & 1.25 \\
36795 & 106595 & 2003 & 14.30 & 0.26 & 1433.00 & 11683.43 & 1.00 & 0.82 & 0.82 \\
15239 & 101970 & 2003 & 100.60 & 0.27 & 10061.00 & 97342.19 & 1.00 & 0.97 & 0.97 \\
36538 & 106560 & 2003 & 85.90 & 0.28 & 8232.00 & 87856.71 & 1.04 & 1.02 & 1.07 \\
16630 & 102166 & 2003 & 97.00 & 0.48 & 9715.00 & 95298.38 & 1.00 & 0.98 & 0.98 \\
36557 & 106561 & 2003 & 15.50 & 0.33 & 1397.00 & 13627.29 & 1.11 & 0.88 & 0.98 \\
50190 & 240402 & 2003 & 11.20 & 0.13 & 1058.00 & 10568.62 & 1.06 & 0.94 & 1.00 \\
28491 & 105427 & 2003 & 219.20 & 0.35 & 21840.00 & 214509.56 & 1.00 & 0.98 & 0.98 \\
46279 & 200207 & 2003 & 26.90 & 0.27 & 2335.00 & 25964.64 & 1.15 & 0.97 & 1.11 \\
420 & 100055 & 2003 & 16003.50 & 0.31 & 1515718.00 & 14900322.72 & 1.06 & 0.93 & 0.98 \\
14608 & 101902 & 2003 & 184.60 & 0.26 & 17842.00 & 157537.25 & 1.03 & 0.85 & 0.88 \\
36588 & 106567 & 2003 & 4.40 & 0.40 & 445.00 & 4283.69 & 0.99 & 0.97 & 0.96 \\
50162 & 240398 & 2003 & 888.70 & 0.38 & 78047.00 & 845633.12 & 1.14 & 0.95 & 1.08 \\
53409 & 348766 & 2003 & 708.50 & 0.23 & 56060.00 & 583489.00 & 1.26 & 0.82 & 1.04 \\
28520 & 105432 & 2003 & 19.90 & 0.24 & 1992.00 & 18863.93 & 1.00 & 0.95 & 0.95 \\
33286 & 106113 & 2003 & 404.40 & 0.28 & 41117.00 & 397025.32 & 0.98 & 0.98 & 0.97 \\
393 & 100048 & 2003 & 258.40 & 0.25 & 24683.00 & 257429.94 & 1.05 & 1.00 & 1.04 \\
28572 & 105444 & 2003 & 91.80 & 0.27 & 7213.00 & 71651.61 & 1.27 & 0.78 & 0.99 \\
5713 & 100789 & 2003 & 5.20 & 0.25 & 526.00 & 4188.30 & 0.99 & 0.81 & 0.80 \\
74778 & 601168 & 2003 & 33.50 & 0.27 & 3365.00 & 32854.08 & 1.00 & 0.98 & 0.98 \\
8570 & 101090 & 2003 & 959.80 & 0.63 & 87397.00 & 905756.79 & 1.10 & 0.94 & 1.04 \\
14632 & 101903 & 2003 & 98.80 & 0.28 & 9559.00 & 89036.67 & 1.03 & 0.90 & 0.93 \\
36532 & 106557 & 2003 & 64.60 & 0.38 & 7285.00 & 63797.42 & 0.89 & 0.99 & 0.88 \\
28542 & 105437 & 2003 & 1171.30 & 0.24 & 117278.00 & 1166301.89 & 1.00 & 1.00 & 0.99 \\
96782 & 611013 & 2003 & 26.90 & 0.28 & 2683.00 & 25700.02 & 1.00 & 0.96 & 0.96 \\
52291 & 302732 & 2003 & 258.80 & 0.27 & 25816.00 & 255278.15 & 1.00 & 0.99 & 0.99 \\
6838 & 100962 & 2003 & 6919.90 & 0.44 & 638285.00 & 5968210.49 & 1.08 & 0.86 & 0.94 \\
11125 & 101368 & 2003 & 1256.10 & 0.32 & 127733.00 & 1219581.09 & 0.98 & 0.97 & 0.95 \\
32407 & 106023 & 2003 & 195.80 & 0.32 & 19898.00 & 185614.08 & 0.98 & 0.95 & 0.93 \\
28462 & 105426 & 2003 & 920.90 & 0.43 & 92180.00 & 883171.18 & 1.00 & 0.96 & 0.96 \\
36595 & 106568 & 2003 & 841.70 & 0.30 & 85621.00 & 804190.64 & 0.98 & 0.96 & 0.94 \\
28352 & 105419 & 2003 & 65.60 & 0.31 & 6145.00 & 67344.10 & 1.07 & 1.03 & 1.10 \\
28340 & 105416 & 2003 & 3416.30 & 0.50 & 282711.00 & 3381257.52 & 1.21 & 0.99 & 1.20 \\
16694 & 102178 & 2003 & 782.40 & 0.26 & 68451.00 & 754545.50 & 1.14 & 0.96 & 1.10 \\
53441 & 349609 & 2003 & 17.00 & 0.17 & 1760.00 & 17584.25 & 0.97 & 1.03 & 1.00 \\
36747 & 106584 & 2003 & 264.70 & 0.42 & 26699.00 & 257127.43 & 0.99 & 0.97 & 0.96 \\
5651 & 100784 & 2003 & 29437.20 & 0.32 & 2948092.00 & 26098009.92 & 1.00 & 0.89 & 0.89 \\
8533 & 101089 & 2003 & 108.50 & 0.27 & 10054.00 & 105967.38 & 1.08 & 0.98 & 1.05 \\
28309 & 105401 & 2003 & 38.90 & 0.38 & 3963.00 & 38527.04 & 0.98 & 0.99 & 0.97 \\
6459 & 100875 & 2003 & 198.70 & 0.53 & 19643.00 & 186956.27 & 1.01 & 0.94 & 0.95 \\
5626 & 100775 & 2003 & 404.60 & 0.25 & 40642.00 & 401198.03 & 1.00 & 0.99 & 0.99 \\
36775 & 106590 & 2003 & 77.40 & 0.34 & 7737.00 & 73461.71 & 1.00 & 0.95 & 0.95 \\
55574 & 400126 & 2003 & 34.80 & 0.14 & 3480.00 & 34658.33 & 1.00 & 1.00 & 1.00 \\
36733 & 106583 & 2003 & 29.10 & 0.54 & 2478.00 & 25375.55 & 1.17 & 0.87 & 1.02 \\
5681 & 100785 & 2003 & 1448.60 & 0.28 & 131834.00 & 1470226.99 & 1.10 & 1.01 & 1.12 \\
16651 & 102173 & 2003 & 32.60 & 0.31 & 2900.00 & 28501.64 & 1.12 & 0.87 & 0.98 \\
52298 & 302760 & 2003 & 741.10 & 0.55 & 65234.00 & 703096.58 & 1.14 & 0.95 & 1.08 \\
28433 & 105424 & 2003 & 3932.20 & 0.25 & 394255.00 & 3843580.74 & 1.00 & 0.98 & 0.97 \\
16661 & 102175 & 2003 & 388.10 & 0.40 & 34824.00 & 395585.12 & 1.11 & 1.02 & 1.14 \\
11161 & 101369 & 2003 & 1883.80 & 0.31 & 188423.00 & 1865611.17 & 1.00 & 0.99 & 0.99 \\
36630 & 106571 & 2003 & 120.60 & 0.18 & 12035.00 & 116773.63 & 1.00 & 0.97 & 0.97 \\
36656 & 106573 & 2003 & 8.00 & 0.31 & 803.00 & 7642.17 & 1.00 & 0.96 & 0.95 \\
28404 & 105421 & 2003 & 24.30 & 0.35 & 2499.00 & 26005.06 & 0.97 & 1.07 & 1.04 \\
50156 & 240397 & 2003 & 14.40 & 0.37 & 1455.00 & 14227.80 & 0.99 & 0.99 & 0.98 \\
50152 & 240396 & 2003 & 20.70 & 0.49 & 2076.00 & 20429.42 & 1.00 & 0.99 & 0.98 \\
55600 & 400128 & 2003 & 60.90 & 0.15 & 5967.00 & 59642.17 & 1.02 & 0.98 & 1.00 \\
36691 & 106577 & 2003 & 2476.60 & 0.31 & 212974.00 & 2260150.23 & 1.16 & 0.91 & 1.06 \\
28378 & 105420 & 2003 & 44.50 & 0.24 & 4347.00 & 45510.89 & 1.02 & 1.02 & 1.05 \\
55579 & 400127 & 2003 & 77.20 & 0.47 & 7000.00 & 73543.58 & 1.10 & 0.95 & 1.05 \\
36717 & 106580 & 2003 & 92.20 & 0.41 & 9189.00 & 91889.46 & 1.00 & 1.00 & 1.00 \\
53984 & 362981 & 2003 & 265.20 & 0.23 & 26293.00 & 278341.14 & 1.01 & 1.05 & 1.06 \\
40534 & 108136 & 2003 & 19.20 & 0.31 & 1629.00 & 17391.36 & 1.18 & 0.91 & 1.07 \\
42884 & 109030 & 2003 & 51.20 & 0.24 & 4335.00 & 47536.12 & 1.18 & 0.93 & 1.10 \\
23964 & 103251 & 2003 & 348.70 & 0.40 & 30671.00 & 294023.28 & 1.14 & 0.84 & 0.96 \\
42600 & 108985 & 2003 & 2860.70 & 0.69 & 240864.00 & 2455083.81 & 1.19 & 0.86 & 1.02 \\
48906 & 240152 & 2003 & 231.90 & 0.23 & 23174.00 & 204545.52 & 1.00 & 0.88 & 0.88 \\
40457 & 108121 & 2003 & 1836.20 & 0.37 & 135031.00 & 1634762.09 & 1.36 & 0.89 & 1.21 \\
20305 & 102715 & 2003 & 2073.70 & 0.31 & 201370.00 & 1967608.68 & 1.03 & 0.95 & 0.98 \\
18778 & 102508 & 2003 & 114.90 & 0.31 & 11135.00 & 111271.66 & 1.03 & 0.97 & 1.00 \\
41646 & 108826 & 2003 & 319.00 & 0.35 & 31562.00 & 285754.04 & 1.01 & 0.90 & 0.91 \\
7497 & 101042 & 2003 & 18171.30 & 0.23 & 1573737.00 & 16536653.55 & 1.15 & 0.91 & 1.05 \\
61981 & 500315 & 2003 & 55.90 & 1.06 & 5277.00 & 50542.01 & 1.06 & 0.90 & 0.96 \\
55178 & 400066 & 2003 & 145.90 & 0.29 & 14581.00 & 145504.52 & 1.00 & 1.00 & 1.00 \\
19579 & 102633 & 2003 & 143.50 & 0.47 & 14238.00 & 140497.98 & 1.01 & 0.98 & 0.99 \\
41671 & 108827 & 2003 & 817.80 & 0.31 & 81990.00 & 793043.89 & 1.00 & 0.97 & 0.97 \\
54157 & 364633 & 2003 & 4.70 & 0.34 & 482.00 & 4712.70 & 0.98 & 1.00 & 0.98 \\
23985 & 103252 & 2003 & 316.70 & 0.36 & 29463.00 & 284261.67 & 1.07 & 0.90 & 0.96 \\
64398 & 500601 & 2003 & 128.50 & 0.31 & 12849.00 & 128128.02 & 1.00 & 1.00 & 1.00 \\
7149 & 100998 & 2003 & 106.40 & 0.45 & 9843.00 & 96577.74 & 1.08 & 0.91 & 0.98 \\
43228 & 109085 & 2003 & 16.10 & 0.31 & 1626.00 & 16055.59 & 0.99 & 1.00 & 0.99 \\
4403 & 100622 & 2003 & 536.70 & 0.39 & 53649.00 & 527131.09 & 1.00 & 0.98 & 0.98 \\
44502 & 109332 & 2003 & 40.40 & 0.57 & 4049.00 & 38000.78 & 1.00 & 0.94 & 0.94 \\
59052 & 410418 & 2003 & 924.10 & 0.30 & 153780.00 & 1579098.02 & 0.60 & 1.71 & 1.03 \\
2672 & 100351 & 2003 & 61.50 & 0.25 & 6143.00 & 61201.27 & 1.00 & 1.00 & 1.00 \\
3684 & 100468 & 2003 & 2507.90 & 0.31 & 252513.00 & 2423882.20 & 0.99 & 0.97 & 0.96 \\
18324 & 102425 & 2003 & 1624.20 & 0.16 & 158778.00 & 1615252.00 & 1.02 & 0.99 & 1.02 \\
23934 & 103242 & 2003 & 999.50 & 0.32 & 99746.00 & 995053.12 & 1.00 & 1.00 & 1.00 \\
63501 & 500512 & 2003 & 681.80 & 0.47 & 84330.00 & 849700.74 & 0.81 & 1.25 & 1.01 \\
12574 & 101554 & 2003 & 116.10 & 0.27 & 11619.00 & 115257.53 & 1.00 & 0.99 & 0.99 \\
42634 & 108988 & 2003 & 249.10 & 0.28 & 24618.00 & 241370.67 & 1.01 & 0.97 & 0.98 \\
40509 & 108134 & 2003 & 118.90 & 0.43 & 11888.00 & 110251.09 & 1.00 & 0.93 & 0.93 \\
44030 & 109259 & 2003 & 142.20 & 0.67 & 14128.00 & 130792.11 & 1.01 & 0.92 & 0.93 \\
54550 & 375746 & 2003 & 70.40 & 0.48 & 7024.00 & 62089.10 & 1.00 & 0.88 & 0.88 \\
43197 & 109080 & 2003 & 25.60 & 0.20 & 2561.00 & 25063.85 & 1.00 & 0.98 & 0.98 \\
41623 & 108782 & 2003 & 764.70 & 0.48 & 78047.00 & 748088.86 & 0.98 & 0.98 & 0.96 \\
43200 & 109083 & 2003 & 187.60 & 0.17 & 17686.00 & 176828.04 & 1.06 & 0.94 & 1.00 \\
40483 & 108122 & 2003 & 291.10 & 0.35 & 29107.00 & 271456.70 & 1.00 & 0.93 & 0.93 \\
42625 & 108987 & 2003 & 62.10 & 0.24 & 6230.00 & 61583.79 & 1.00 & 0.99 & 0.99 \\
43205 & 109084 & 2003 & 513.70 & 0.99 & 35726.00 & 334011.94 & 1.44 & 0.65 & 0.93 \\
54903 & 400021 & 2003 & 91.10 & 0.31 & 9064.00 & 92368.76 & 1.01 & 1.01 & 1.02 \\
61401 & 500048 & 2003 & 57.10 & 0.33 & 4588.00 & 51476.26 & 1.24 & 0.90 & 1.12 \\
13556 & 101743 & 2003 & 9886.40 & 0.25 & 989356.00 & 8755671.25 & 1.00 & 0.89 & 0.88 \\
24036 & 103255 & 2003 & 140.40 & 0.33 & 12788.00 & 116278.23 & 1.10 & 0.83 & 0.91 \\
4359 & 100611 & 2003 & 420.90 & 0.16 & 42126.00 & 379325.40 & 1.00 & 0.90 & 0.90 \\
63433 & 500508 & 2003 & 6778.20 & 0.28 & 598631.00 & 6383575.56 & 1.13 & 0.94 & 1.07 \\
44048 & 109263 & 2003 & 9.70 & 0.68 & 570.00 & 5479.64 & 1.70 & 0.56 & 0.96 \\
21256 & 102843 & 2003 & 331.30 & 0.46 & 33433.00 & 319616.52 & 0.99 & 0.96 & 0.96 \\
41704 & 108840 & 2003 & 128.10 & 0.28 & 10336.00 & 107049.05 & 1.24 & 0.84 & 1.04 \\
20258 & 102696 & 2003 & 237.40 & 0.39 & 23753.00 & 226477.13 & 1.00 & 0.95 & 0.95 \\
40371 & 108115 & 2003 & 1973.70 & 0.28 & 197555.00 & 1936229.87 & 1.00 & 0.98 & 0.98 \\
54574 & 375967 & 2003 & 116.80 & 0.23 & 11697.00 & 116049.27 & 1.00 & 0.99 & 0.99 \\
48475 & 240087 & 2003 & 223.60 & 0.17 & 22302.00 & 207182.53 & 1.00 & 0.93 & 0.93 \\
59074 & 410423 & 2003 & 4.10 & 0.21 & 297.00 & 2774.55 & 1.38 & 0.68 & 0.93 \\
24058 & 103259 & 2003 & 2639.90 & 0.28 & 228247.00 & 2261195.71 & 1.16 & 0.86 & 0.99 \\
48215 & 240051 & 2003 & 615.90 & 0.22 & 60765.00 & 613026.47 & 1.01 & 1.00 & 1.01 \\
12550 & 101553 & 2003 & 213.10 & 0.24 & 21320.00 & 209572.90 & 1.00 & 0.98 & 0.98 \\
64375 & 500600 & 2003 & 1295.80 & 0.25 & 129583.00 & 1293187.65 & 1.00 & 1.00 & 1.00 \\
2038 & 100286 & 2003 & 40.50 & 0.28 & 4043.00 & 39350.42 & 1.00 & 0.97 & 0.97 \\
40397 & 108117 & 2003 & 622.80 & 0.44 & 61944.00 & 535182.16 & 1.01 & 0.86 & 0.86 \\
44038 & 109260 & 2003 & 70.50 & 0.45 & 7044.00 & 68331.30 & 1.00 & 0.97 & 0.97 \\
22504 & 103016 & 2003 & 510.00 & 0.43 & 36295.00 & 372174.09 & 1.41 & 0.73 & 1.03 \\
4385 & 100614 & 2003 & 1654.10 & 0.27 & 165513.00 & 1570349.60 & 1.00 & 0.95 & 0.95 \\
63477 & 500511 & 2003 & 634.40 & 0.29 & 70223.00 & 707096.17 & 0.90 & 1.11 & 1.01 \\
40431 & 108119 & 2003 & 294.00 & 0.31 & 29397.00 & 283608.70 & 1.00 & 0.96 & 0.96 \\
43766 & 109218 & 2003 & 19.50 & 0.36 & 1952.00 & 19480.33 & 1.00 & 1.00 & 1.00 \\
24006 & 103253 & 2003 & 205.20 & 0.47 & 18554.00 & 178079.72 & 1.11 & 0.87 & 0.96 \\
22489 & 103015 & 2003 & 395.80 & 0.22 & 37737.00 & 400866.73 & 1.05 & 1.01 & 1.06 \\
44789 & 109380 & 2003 & 44.30 & 0.18 & 4402.00 & 36184.08 & 1.01 & 0.82 & 0.82 \\
7395 & 101038 & 2003 & 3459.40 & 0.25 & 319030.00 & 3295748.26 & 1.08 & 0.95 & 1.03 \\
62058 & 500326 & 2003 & 1.40 & 0.27 & 121.00 & 1297.99 & 1.16 & 0.93 & 1.07 \\
61375 & 500047 & 2003 & 1.80 & 0.32 & 262.00 & 2356.45 & 0.69 & 1.31 & 0.90 \\
7643 & 101050 & 2003 & 960.00 & 0.20 & 97484.00 & 931484.32 & 0.98 & 0.97 & 0.96 \\
40423 & 108118 & 2003 & 65.50 & 0.29 & 6550.00 & 64169.40 & 1.00 & 0.98 & 0.98 \\
55171 & 400065 & 2003 & 97.50 & 0.27 & 9749.00 & 97308.61 & 1.00 & 1.00 & 1.00 \\
4052 & 100543 & 2003 & 369.00 & 0.55 & 36939.00 & 356026.19 & 1.00 & 0.96 & 0.96 \\
3727 & 100475 & 2003 & 851.60 & 0.22 & 84241.00 & 816179.78 & 1.01 & 0.96 & 0.97 \\
41679 & 108839 & 2003 & 326.70 & 0.59 & 32611.00 & 323360.75 & 1.00 & 0.99 & 0.99 \\
22471 & 103014 & 2003 & 254.20 & 0.28 & 22341.00 & 244858.08 & 1.14 & 0.96 & 1.10 \\
22572 & 103021 & 2003 & 140.10 & 0.50 & 9212.00 & 114988.10 & 1.52 & 0.82 & 1.25 \\
54177 & 364809 & 2003 & 37.70 & 0.33 & 3791.00 & 37819.98 & 0.99 & 1.00 & 1.00 \\
54504 & 373198 & 2003 & 41.70 & 0.13 & 4193.00 & 42545.48 & 0.99 & 1.02 & 1.01 \\
42680 & 108993 & 2003 & 54.00 & 0.45 & 4712.00 & 46066.24 & 1.15 & 0.85 & 0.98 \\
2634 & 100348 & 2003 & 76.10 & 0.47 & 7684.00 & 75462.28 & 0.99 & 0.99 & 0.98 \\
43168 & 109072 & 2003 & 151.60 & 0.23 & 14137.00 & 146094.58 & 1.07 & 0.96 & 1.03 \\
40581 & 108140 & 2003 & 124.70 & 0.35 & 12307.00 & 123054.77 & 1.01 & 0.99 & 1.00 \\
59298 & 410460 & 2003 & 3.30 & 0.62 & 168.00 & 1715.34 & 1.96 & 0.52 & 1.02 \\
23829 & 103214 & 2003 & 1742.90 & 0.30 & 174950.00 & 1713628.17 & 1.00 & 0.98 & 0.98 \\
44541 & 109336 & 2003 & 37.90 & 0.28 & 3515.00 & 37074.06 & 1.08 & 0.98 & 1.05 \\
20390 & 102733 & 2003 & 2640.10 & 0.31 & 259748.00 & 2334027.77 & 1.02 & 0.88 & 0.90 \\
12599 & 101557 & 2003 & 41.60 & 0.15 & 3986.00 & 42611.85 & 1.04 & 1.02 & 1.07 \\
21144 & 102833 & 2003 & 31.40 & 0.77 & 3094.00 & 30907.88 & 1.01 & 0.98 & 1.00 \\
4066 & 100544 & 2003 & 644.40 & 0.90 & 64761.00 & 640650.36 & 1.00 & 0.99 & 0.99 \\
64329 & 500597 & 2003 & 7874.40 & 0.29 & 787439.00 & 7857772.65 & 1.00 & 1.00 & 1.00 \\
19273 & 102578 & 2003 & 42.70 & 0.25 & 4236.00 & 42272.23 & 1.01 & 0.99 & 1.00 \\
43162 & 109071 & 2003 & 14.80 & 0.40 & 1482.00 & 14753.01 & 1.00 & 1.00 & 1.00 \\
22684 & 103028 & 2003 & 5283.60 & 0.24 & 455046.00 & 4181701.33 & 1.16 & 0.79 & 0.92 \\
48259 & 240057 & 2003 & 128.90 & 0.28 & 11226.00 & 103147.17 & 1.15 & 0.80 & 0.92 \\
44769 & 109374 & 2003 & 112.80 & 0.43 & 10296.00 & 101333.78 & 1.10 & 0.90 & 0.98 \\
20422 & 102737 & 2003 & 2481.10 & 0.28 & 236105.00 & 2360037.57 & 1.05 & 0.95 & 1.00 \\
23799 & 103213 & 2003 & 644.10 & 0.27 & 64263.00 & 640648.70 & 1.00 & 0.99 & 1.00 \\
19285 & 102579 & 2003 & 430.20 & 0.57 & 42089.00 & 415982.31 & 1.02 & 0.97 & 0.99 \\
41487 & 108761 & 2003 & 467.10 & 0.53 & 44634.00 & 431391.80 & 1.05 & 0.92 & 0.97 \\
21112 & 102832 & 2003 & 125.00 & 0.42 & 12559.00 & 121012.97 & 1.00 & 0.97 & 0.96 \\
44775 & 109375 & 2003 & 57.50 & 0.47 & 5750.00 & 55069.87 & 1.00 & 0.96 & 0.96 \\
48527 & 240103 & 2003 & 349.30 & 0.44 & 34589.00 & 314461.37 & 1.01 & 0.90 & 0.91 \\
41509 & 108762 & 2003 & 74.00 & -0.02 & 7428.00 & 71818.06 & 1.00 & 0.97 & 0.97 \\
43153 & 109069 & 2003 & 228.90 & 0.44 & 23182.00 & 226352.29 & 0.99 & 0.99 & 0.98 \\
18358 & 102446 & 2003 & 16.20 & 0.36 & 1643.00 & 15736.20 & 0.99 & 0.97 & 0.96 \\
41534 & 108764 & 2003 & 384.00 & 0.27 & 43160.00 & 351157.92 & 0.89 & 0.91 & 0.81 \\
40590 & 108141 & 2003 & 201.90 & 0.19 & 20640.00 & 201848.90 & 0.98 & 1.00 & 0.98 \\
42695 & 108994 & 2003 & 182.00 & 0.21 & 17340.00 & 186385.11 & 1.05 & 1.02 & 1.07 \\
64306 & 500596 & 2003 & 1454.60 & 0.29 & 145463.00 & 1450480.05 & 1.00 & 1.00 & 1.00 \\
54791 & 400014 & 2003 & 190.60 & 0.28 & 19082.00 & 173667.41 & 1.00 & 0.91 & 0.91 \\
48107 & 240010 & 2003 & 326.70 & 0.38 & 31149.00 & 316240.51 & 1.05 & 0.97 & 1.02 \\
23862 & 103224 & 2003 & 106.60 & 0.27 & 11277.00 & 111090.42 & 0.95 & 1.04 & 0.99 \\
41559 & 108766 & 2003 & 196.40 & 0.29 & 19390.00 & 185677.76 & 1.01 & 0.95 & 0.96 \\
41574 & 108776 & 2003 & 432.00 & 0.37 & 36495.00 & 384386.31 & 1.18 & 0.89 & 1.05 \\
23898 & 103228 & 2003 & 130.40 & 0.39 & 14080.00 & 132593.04 & 0.93 & 1.02 & 0.94 \\
42640 & 108990 & 2003 & 41.00 & 0.36 & 4320.00 & 37471.71 & 0.95 & 0.91 & 0.87 \\
59036 & 410401 & 2003 & 152.00 & 0.31 & 11569.00 & 126835.27 & 1.31 & 0.83 & 1.10 \\
40541 & 108137 & 2003 & 1106.80 & 0.27 & 87777.00 & 918662.24 & 1.26 & 0.83 & 1.05 \\
22603 & 103024 & 2003 & 1321.20 & 0.38 & 132060.00 & 1211385.48 & 1.00 & 0.92 & 0.92 \\
64352 & 500598 & 2003 & 1986.10 & 0.39 & 198614.00 & 1983675.24 & 1.00 & 1.00 & 1.00 \\
41599 & 108777 & 2003 & 48.50 & 0.48 & 4961.00 & 42220.40 & 0.98 & 0.87 & 0.85 \\
19257 & 102575 & 2003 & 301.60 & 0.50 & 30640.00 & 290987.38 & 0.98 & 0.96 & 0.95 \\
12953 & 101616 & 2003 & 17610.20 & 0.24 & 1785295.00 & 17169736.08 & 0.99 & 0.97 & 0.96 \\
18760 & 102507 & 2003 & 216.80 & 0.40 & 21502.00 & 214965.99 & 1.01 & 0.99 & 1.00 \\
41609 & 108780 & 2003 & 56.60 & 0.28 & 5520.00 & 57768.64 & 1.03 & 1.02 & 1.05 \\
43194 & 109079 & 2003 & 45.10 & 0.27 & 4330.00 & 38885.08 & 1.04 & 0.86 & 0.90 \\
23914 & 103232 & 2003 & 209.80 & 0.22 & 20929.00 & 204545.35 & 1.00 & 0.97 & 0.98 \\
59255 & 410448 & 2003 & 124.90 & 0.35 & 9599.00 & 102352.80 & 1.30 & 0.82 & 1.07 \\
21198 & 102837 & 2003 & 960.60 & 0.36 & 84858.00 & 812856.54 & 1.13 & 0.85 & 0.96 \\
20339 & 102716 & 2003 & 885.00 & 0.27 & 86403.00 & 848560.59 & 1.02 & 0.96 & 0.98 \\
7911 & 101065 & 2003 & 1853.30 & 0.33 & 198812.00 & 1785589.92 & 0.93 & 0.96 & 0.90 \\
42648 & 108991 & 2003 & 320.50 & 0.27 & 34114.00 & 329929.26 & 0.94 & 1.03 & 0.97 \\
61483 & 500082 & 2003 & 336.70 & 0.33 & 34206.00 & 304679.82 & 0.98 & 0.90 & 0.89 \\
4645 & 100659 & 2003 & 694.90 & 0.40 & 110489.00 & 1051319.15 & 0.63 & 1.51 & 0.95 \\
61962 & 500312 & 2003 & 41.00 & 0.35 & 3829.00 & 38733.64 & 1.07 & 0.94 & 1.01 \\
44518 & 109334 & 2003 & 73.30 & 0.33 & 6481.00 & 66768.36 & 1.13 & 0.91 & 1.03 \\
22640 & 103027 & 2003 & 7587.60 & 0.36 & 660053.00 & 6806756.34 & 1.15 & 0.90 & 1.03 \\
2064 & 100287 & 2003 & 57.60 & 0.21 & 5783.00 & 54777.62 & 1.00 & 0.95 & 0.95 \\
21166 & 102835 & 2003 & 366.30 & 0.41 & 29782.00 & 270711.33 & 1.23 & 0.74 & 0.91 \\
19558 & 102624 & 2003 & 111.20 & 0.27 & 11210.00 & 110293.45 & 0.99 & 0.99 & 0.98 \\
7668 & 101054 & 2003 & 10388.40 & 0.27 & 1041185.00 & 10154574.78 & 1.00 & 0.98 & 0.98 \\
23880 & 103226 & 2003 & 130.10 & 0.26 & 13093.00 & 126960.18 & 0.99 & 0.98 & 0.97 \\
48496 & 240090 & 2003 & 24.10 & 0.37 & 1983.00 & 20465.14 & 1.22 & 0.85 & 1.03 \\
2653 & 100350 & 2003 & 232.50 & 0.20 & 23252.00 & 228795.58 & 1.00 & 0.98 & 0.98 \\
43173 & 109074 & 2003 & 106.20 & 0.48 & 9078.00 & 90282.02 & 1.17 & 0.85 & 0.99 \\
44507 & 109333 & 2003 & 55.10 & 0.92 & 5336.00 & 51308.32 & 1.03 & 0.93 & 0.96 \\
54817 & 400015 & 2003 & 4.20 & 0.74 & 275.00 & 3426.56 & 1.53 & 0.82 & 1.25 \\
22248 & 102997 & 2003 & 4161.20 & 0.42 & 415883.00 & 4077757.87 & 1.00 & 0.98 & 0.98 \\
64506 & 500606 & 2003 & 490.90 & 0.19 & 49085.00 & 490633.82 & 1.00 & 1.00 & 1.00 \\
19639 & 102639 & 2003 & 122.10 & 0.36 & 12253.00 & 112647.61 & 1.00 & 0.92 & 0.92 \\
21364 & 102854 & 2003 & 223.90 & 0.44 & 19194.00 & 223089.26 & 1.17 & 1.00 & 1.16 \\
54620 & 377074 & 2003 & 304.10 & 0.43 & 26582.00 & 243183.02 & 1.14 & 0.80 & 0.91 \\
13887 & 101785 & 2003 & 1653.00 & 0.48 & 162967.00 & 1664756.83 & 1.01 & 1.01 & 1.02 \\
4727 & 100669 & 2003 & 434.50 & 0.52 & 43642.00 & 433139.95 & 1.00 & 1.00 & 0.99 \\
12481 & 101542 & 2003 & 71.30 & 0.31 & 6816.00 & 62063.30 & 1.05 & 0.87 & 0.91 \\
24237 & 103299 & 2003 & 353.10 & 0.60 & 26224.00 & 331839.36 & 1.35 & 0.94 & 1.27 \\
18883 & 102525 & 2003 & 1228.40 & 0.35 & 122644.00 & 1206544.16 & 1.00 & 0.98 & 0.98 \\
42489 & 108971 & 2003 & 118.50 & 0.37 & 11261.00 & 111860.27 & 1.05 & 0.94 & 0.99 \\
43288 & 109090 & 2003 & 57.00 & 0.27 & 5696.00 & 56401.30 & 1.00 & 0.99 & 0.99 \\
42477 & 108970 & 2003 & 112.30 & 0.28 & 11080.00 & 110723.00 & 1.01 & 0.99 & 1.00 \\
44066 & 109264 & 2003 & 63.10 & 0.68 & 4750.00 & 50887.80 & 1.33 & 0.81 & 1.07 \\
40165 & 108070 & 2003 & 54.70 & 0.36 & 5252.00 & 52511.02 & 1.04 & 0.96 & 1.00 \\
43305 & 109092 & 2003 & 59.20 & 0.35 & 5441.00 & 54203.97 & 1.09 & 0.92 & 1.00 \\
13052 & 101623 & 2003 & 1255.30 & 0.32 & 125544.00 & 1242238.76 & 1.00 & 0.99 & 0.99 \\
59244 & 410447 & 2003 & 31.00 & 0.43 & 2609.00 & 25701.92 & 1.19 & 0.83 & 0.99 \\
7981 & 101068 & 2003 & 67995.10 & 0.25 & 6223184.00 & 66051675.51 & 1.09 & 0.97 & 1.06 \\
41861 & 108866 & 2003 & 326.70 & 0.46 & 33488.00 & 318012.46 & 0.98 & 0.97 & 0.95 \\
7613 & 101048 & 2003 & 10695.90 & 0.32 & 1042187.00 & 10227403.94 & 1.03 & 0.96 & 0.98 \\
41834 & 108859 & 2003 & 22.20 & 0.82 & 1623.00 & 21739.04 & 1.37 & 0.98 & 1.34 \\
49077 & 240218 & 2003 & 465.90 & 0.68 & 46649.00 & 403872.46 & 1.00 & 0.87 & 0.87 \\
44444 & 109324 & 2003 & 139.90 & 0.16 & 13942.00 & 114355.32 & 1.00 & 0.82 & 0.82 \\
20157 & 102671 & 2003 & 222.60 & 0.24 & 22223.00 & 221971.47 & 1.00 & 1.00 & 1.00 \\
22279 & 102999 & 2003 & 124.10 & 0.36 & 12809.00 & 127052.97 & 0.97 & 1.02 & 0.99 \\
48438 & 240083 & 2003 & 342.80 & 0.44 & 34289.00 & 338057.13 & 1.00 & 0.99 & 0.99 \\
44419 & 109307 & 2003 & 58.30 & 0.36 & 4783.00 & 45142.92 & 1.22 & 0.77 & 0.94 \\
41841 & 108860 & 2003 & 52.40 & 0.60 & 5028.00 & 51793.66 & 1.04 & 0.99 & 1.03 \\
40211 & 108073 & 2003 & 299.90 & 0.35 & 26380.00 & 278405.77 & 1.14 & 0.93 & 1.06 \\
42514 & 108973 & 2003 & 120.00 & 0.48 & 9544.00 & 102486.62 & 1.26 & 0.85 & 1.07 \\
18230 & 102417 & 2003 & 1083.70 & 0.39 & 104297.00 & 1066635.02 & 1.04 & 0.98 & 1.02 \\
48245 & 240056 & 2003 & 77.80 & 0.38 & 7798.00 & 62841.83 & 1.00 & 0.81 & 0.81 \\
41855 & 108861 & 2003 & 84.60 & 0.23 & 8458.00 & 84278.86 & 1.00 & 1.00 & 1.00 \\
40185 & 108071 & 2003 & 117.80 & 0.40 & 14990.00 & 149802.76 & 0.79 & 1.27 & 1.00 \\
13038 & 101622 & 2003 & 877.10 & 0.27 & 89909.00 & 852471.47 & 0.98 & 0.97 & 0.95 \\
13706 & 101758 & 2003 & 353.40 & 0.40 & 34696.00 & 342789.32 & 1.02 & 0.97 & 0.99 \\
22220 & 102996 & 2003 & 608.10 & 0.34 & 62049.00 & 607129.20 & 0.98 & 1.00 & 0.98 \\
1986 & 100278 & 2003 & 66.60 & 0.29 & 6661.00 & 64922.12 & 1.00 & 0.97 & 0.97 \\
13285 & 101717 & 2003 & 68.90 & 0.23 & 6889.00 & 68246.42 & 1.00 & 0.99 & 0.99 \\
43316 & 109094 & 2003 & 157.60 & 0.27 & 13497.00 & 150376.96 & 1.17 & 0.95 & 1.11 \\
7593 & 101047 & 2003 & 238.10 & 0.26 & 23784.00 & 235127.39 & 1.00 & 0.99 & 0.99 \\
40131 & 108037 & 2003 & 211.70 & 0.45 & 18430.00 & 190842.34 & 1.15 & 0.90 & 1.04 \\
4307 & 100603 & 2003 & 1681.30 & 0.38 & 149292.00 & 1617558.59 & 1.13 & 0.96 & 1.08 \\
20080 & 102665 & 2003 & 16.70 & 0.27 & 1755.00 & 15183.54 & 0.95 & 0.91 & 0.87 \\
13075 & 101626 & 2003 & 1090.90 & 0.16 & 111432.00 & 976615.72 & 0.98 & 0.90 & 0.88 \\
54024 & 363941 & 2003 & 7.30 & 0.31 & 640.00 & 6100.67 & 1.14 & 0.84 & 0.95 \\
44904 & 109399 & 2003 & 24.00 & 0.29 & 2402.00 & 21611.15 & 1.00 & 0.90 & 0.90 \\
41892 & 108868 & 2003 & 960.90 & 0.50 & 96165.00 & 950114.82 & 1.00 & 0.99 & 0.99 \\
13620 & 101748 & 2003 & 1902.00 & 0.16 & 190134.00 & 1882750.62 & 1.00 & 0.99 & 0.99 \\
61211 & 410907 & 2003 & 79.90 & 0.34 & 6353.00 & 61708.24 & 1.26 & 0.77 & 0.97 \\
3216 & 100415 & 2003 & 132.90 & 0.30 & 13283.00 & 128546.52 & 1.00 & 0.97 & 0.97 \\
59233 & 410446 & 2003 & 23.50 & 0.48 & 2075.00 & 20014.98 & 1.13 & 0.85 & 0.96 \\
24320 & 103308 & 2003 & 4651.20 & 0.17 & 438715.00 & 4406111.29 & 1.06 & 0.95 & 1.00 \\
41922 & 108870 & 2003 & 39.00 & 0.24 & 3949.00 & 38912.20 & 0.99 & 1.00 & 0.99 \\
43328 & 109095 & 2003 & 39.10 & 0.42 & 4782.00 & 40014.86 & 0.82 & 1.02 & 0.84 \\
4230 & 100590 & 2003 & 191.80 & 0.32 & 19281.00 & 189661.28 & 0.99 & 0.99 & 0.98 \\
54047 & 364291 & 2003 & 99.20 & 0.26 & 10152.00 & 106157.20 & 0.98 & 1.07 & 1.05 \\
24286 & 103304 & 2003 & 52.90 & 0.38 & 4101.00 & 48253.34 & 1.29 & 0.91 & 1.18 \\
55035 & 400049 & 2003 & 288.40 & 0.93 & 15829.00 & 130536.15 & 1.82 & 0.45 & 0.82 \\
20113 & 102667 & 2003 & 21726.50 & 0.38 & 1738859.00 & 19180365.97 & 1.25 & 0.88 & 1.10 \\
48881 & 240149 & 2003 & 162.50 & 0.34 & 16245.00 & 159726.36 & 1.00 & 0.98 & 0.98 \\
43705 & 109191 & 2003 & 1.10 & 0.60 & 109.00 & 1090.38 & 1.01 & 0.99 & 1.00 \\
44401 & 109300 & 2003 & 426.80 & 0.29 & 42622.00 & 414283.83 & 1.00 & 0.97 & 0.97 \\
54058 & 364292 & 2003 & 424.20 & 0.24 & 35475.00 & 385241.20 & 1.20 & 0.91 & 1.09 \\
4743 & 100670 & 2003 & 73.80 & 0.23 & 7379.00 & 73271.18 & 1.00 & 0.99 & 0.99 \\
2832 & 100362 & 2003 & 51.80 & 0.50 & 4486.00 & 44824.00 & 1.15 & 0.87 & 1.00 \\
24268 & 103301 & 2003 & 972.90 & 0.43 & 84052.00 & 960892.93 & 1.16 & 0.99 & 1.14 \\
21397 & 102861 & 2003 & 61.80 & 0.27 & 5939.00 & 63173.40 & 1.04 & 1.02 & 1.06 \\
41886 & 108867 & 2003 & 560.50 & 0.36 & 56063.00 & 551886.43 & 1.00 & 0.98 & 0.98 \\
40155 & 108051 & 2003 & 353.60 & 0.29 & 35353.00 & 348208.01 & 1.00 & 0.98 & 0.98 \\
58427 & 410157 & 2003 & 53.00 & 0.04 & 5280.00 & 50367.23 & 1.00 & 0.95 & 0.95 \\
22189 & 102994 & 2003 & 93.60 & 0.39 & 8300.00 & 84278.88 & 1.13 & 0.90 & 1.02 \\
43312 & 109093 & 2003 & 21.00 & 0.32 & 1927.00 & 17939.23 & 1.09 & 0.85 & 0.93 \\
42590 & 108984 & 2003 & 82.10 & 0.37 & 7127.00 & 73420.45 & 1.15 & 0.89 & 1.03 \\
44858 & 109395 & 2003 & 170.80 & 0.45 & 16710.00 & 136772.92 & 1.02 & 0.80 & 0.82 \\
43256 & 109087 & 2003 & 277.00 & 0.34 & 27728.00 & 269538.21 & 1.00 & 0.97 & 0.97 \\
61291 & 500027 & 2003 & 141.00 & 0.48 & 10914.00 & 118862.94 & 1.29 & 0.84 & 1.09 \\
41728 & 108849 & 2003 & 941.60 & 0.39 & 72633.00 & 885357.23 & 1.30 & 0.94 & 1.22 \\
64444 & 500603 & 2003 & 334.70 & 0.30 & 33470.00 & 333616.26 & 1.00 & 1.00 & 1.00 \\
42545 & 108977 & 2003 & 242.50 & 0.30 & 21950.00 & 197015.92 & 1.10 & 0.81 & 0.90 \\
49066 & 240212 & 2003 & 3376.20 & 0.18 & 336939.00 & 2795713.98 & 1.00 & 0.83 & 0.83 \\
41753 & 108852 & 2003 & 126.10 & 0.34 & 12958.00 & 122430.17 & 0.97 & 0.97 & 0.94 \\
44498 & 109330 & 2003 & 19.60 & 0.26 & 1976.00 & 19555.08 & 0.99 & 1.00 & 0.99 \\
22380 & 103008 & 2003 & 167.90 & 0.41 & 16737.00 & 167160.82 & 1.00 & 1.00 & 1.00 \\
24117 & 103267 & 2003 & 2910.30 & 0.50 & 286675.00 & 2863798.76 & 1.02 & 0.98 & 1.00 \\
62140 & 500348 & 2003 & 35.90 & 0.18 & 3584.00 & 35827.20 & 1.00 & 1.00 & 1.00 \\
44475 & 109327 & 2003 & 386.50 & 0.25 & 34676.00 & 385460.86 & 1.11 & 1.00 & 1.11 \\
12521 & 101545 & 2003 & 192.00 & 0.40 & 18388.00 & 183878.78 & 1.04 & 0.96 & 1.00 \\
44471 & 109326 & 2003 & 4.30 & 0.28 & NaN & 4366.20 & 1.00 & 1.02 & 1.00 \\
2755 & 100357 & 2003 & 144.80 & 0.24 & 13711.00 & 140002.65 & 1.06 & 0.97 & 1.02 \\
43233 & 109086 & 2003 & 797.30 & 0.31 & 79760.00 & 777080.53 & 1.00 & 0.97 & 0.97 \\
63352 & 500500 & 2003 & 264.80 & 0.32 & 25017.00 & 256189.69 & 1.06 & 0.97 & 1.02 \\
21280 & 102844 & 2003 & 497.80 & 0.29 & 49765.00 & 491071.93 & 1.00 & 0.99 & 0.99 \\
24101 & 103266 & 2003 & 201.60 & 0.44 & 20162.00 & 201185.82 & 1.00 & 1.00 & 1.00 \\
54578 & 376139 & 2003 & 7.80 & 0.38 & 782.00 & 7433.55 & 1.00 & 0.95 & 0.95 \\
40345 & 108112 & 2003 & 31.90 & 0.34 & 3143.00 & 30085.29 & 1.01 & 0.94 & 0.96 \\
18821 & 102523 & 2003 & 1323.50 & 0.39 & 89633.00 & 935993.68 & 1.48 & 0.71 & 1.04 \\
2724 & 100355 & 2003 & 4992.20 & 0.29 & 441143.00 & 5022218.31 & 1.13 & 1.01 & 1.14 \\
62127 & 500340 & 2003 & 40.60 & 0.03 & 4064.00 & 38845.67 & 1.00 & 0.96 & 0.96 \\
3757 & 100480 & 2003 & 109.00 & 0.28 & 11411.00 & 105013.81 & 0.96 & 0.96 & 0.92 \\
22416 & 103011 & 2003 & 59.30 & 0.26 & 5594.00 & 57325.04 & 1.06 & 0.97 & 1.02 \\
40319 & 108109 & 2003 & 60.00 & 0.32 & 4567.00 & 45967.27 & 1.31 & 0.77 & 1.01 \\
54089 & 364393 & 2003 & 113.70 & 0.32 & 9049.00 & 98044.11 & 1.26 & 0.86 & 1.08 \\
64421 & 500602 & 2003 & 147.70 & 0.31 & 14766.00 & 147054.49 & 1.00 & 1.00 & 1.00 \\
13003 & 101618 & 2003 & 148.00 & 0.29 & 14917.00 & 140896.11 & 0.99 & 0.95 & 0.94 \\
55249 & 400075 & 2003 & 1189.20 & 0.32 & 118948.00 & 1178089.84 & 1.00 & 0.99 & 0.99 \\
24089 & 103264 & 2003 & 744.70 & 0.27 & 74355.00 & 734822.32 & 1.00 & 0.99 & 0.99 \\
63370 & 500502 & 2003 & 16.30 & 0.22 & 1611.00 & 16354.68 & 1.01 & 1.00 & 1.02 \\
19607 & 102636 & 2003 & 1462.00 & 0.33 & 145129.00 & 1432243.65 & 1.01 & 0.98 & 0.99 \\
19225 & 102570 & 2003 & 214.40 & 0.39 & 21471.00 & 203361.01 & 1.00 & 0.95 & 0.95 \\
4706 & 100667 & 2003 & 14.60 & 0.21 & 1282.00 & 12303.96 & 1.14 & 0.84 & 0.96 \\
43743 & 109217 & 2003 & 28.90 & 0.37 & 2897.00 & 29012.51 & 1.00 & 1.00 & 1.00 \\
63316 & 500494 & 2003 & 1284.20 & 0.26 & 118194.00 & 1271937.41 & 1.09 & 0.99 & 1.08 \\
44835 & 109394 & 2003 & 316.40 & -0.49 & 21398.00 & 209157.64 & 1.48 & 0.66 & 0.98 \\
3244 & 100417 & 2003 & 13.50 & 0.31 & 1357.00 & 13472.89 & 0.99 & 1.00 & 0.99 \\
3787 & 100481 & 2003 & 118.40 & 0.35 & 12307.00 & 114410.56 & 0.96 & 0.97 & 0.93 \\
41786 & 108856 & 2003 & 371.60 & 0.38 & 33565.00 & 345557.74 & 1.11 & 0.93 & 1.03 \\
41808 & 108857 & 2003 & 36.70 & 0.29 & 3578.00 & 35231.79 & 1.03 & 0.96 & 0.98 \\
40236 & 108082 & 2003 & 50.40 & 0.26 & 5030.00 & 48350.44 & 1.00 & 0.96 & 0.96 \\
54599 & 377010 & 2003 & 52.80 & 0.35 & 5354.00 & 48963.80 & 0.99 & 0.93 & 0.91 \\
18852 & 102524 & 2003 & 2978.80 & 0.28 & 297537.00 & 2880742.07 & 1.00 & 0.97 & 0.97 \\
21333 & 102852 & 2003 & 1636.40 & 0.47 & 140769.00 & 1679884.89 & 1.16 & 1.03 & 1.19 \\
41827 & 108858 & 2003 & 48.00 & 0.25 & 4794.00 & 47396.46 & 1.00 & 0.99 & 0.99 \\
20178 & 102673 & 2003 & 497.20 & 0.32 & 48207.00 & 487553.82 & 1.03 & 0.98 & 1.01 \\
48450 & 240085 & 2003 & 191.10 & 0.38 & 19124.00 & 188883.03 & 1.00 & 0.99 & 0.99 \\
44448 & 109325 & 2003 & 378.20 & 0.26 & 37822.00 & 377361.84 & 1.00 & 1.00 & 1.00 \\
61262 & 500025 & 2003 & 3.30 & 0.30 & 300.00 & 2937.77 & 1.10 & 0.89 & 0.98 \\
43711 & 109199 & 2003 & 31.40 & 0.35 & 2449.00 & 20140.49 & 1.28 & 0.64 & 0.82 \\
22300 & 103005 & 2003 & 936.40 & 0.37 & 80972.00 & 809667.11 & 1.16 & 0.86 & 1.00 \\
40225 & 108074 & 2003 & 47.90 & 0.30 & 3822.00 & 41894.42 & 1.25 & 0.87 & 1.10 \\
42533 & 108976 & 2003 & 69.70 & 0.29 & 6402.00 & 66390.78 & 1.09 & 0.95 & 1.04 \\
2010 & 100280 & 2003 & 44.70 & 0.23 & 4448.00 & 45783.56 & 1.00 & 1.02 & 1.03 \\
13588 & 101744 & 2003 & 1407.80 & 0.56 & 141176.00 & 1402286.94 & 1.00 & 1.00 & 0.99 \\
58434 & 410158 & 2003 & 52.40 & 0.03 & 4032.00 & 46043.58 & 1.30 & 0.88 & 1.14 \\
54918 & 400025 & 2003 & 158.60 & 0.30 & 12662.00 & 154806.90 & 1.25 & 0.98 & 1.22 \\
41759 & 108853 & 2003 & 222.40 & 0.21 & 22340.00 & 204011.35 & 1.00 & 0.92 & 0.91 \\
40263 & 108083 & 2003 & 126.60 & 0.40 & 12630.00 & 121510.95 & 1.00 & 0.96 & 0.96 \\
3260 & 100419 & 2003 & 718.90 & 0.42 & 72083.00 & 662097.30 & 1.00 & 0.92 & 0.92 \\
4216 & 100575 & 2003 & 9.90 & 0.34 & 997.00 & 9807.54 & 0.99 & 0.99 & 0.98 \\
21314 & 102847 & 2003 & 22.00 & 0.26 & 2212.00 & 21391.60 & 0.99 & 0.97 & 0.97 \\
54080 & 364391 & 2003 & 22.30 & 0.27 & 2245.00 & 21403.63 & 0.99 & 0.96 & 0.95 \\
59111 & 410433 & 2003 & 1657.60 & 0.57 & 170414.00 & 1613170.48 & 0.97 & 0.97 & 0.95 \\
41767 & 108855 & 2003 & 409.90 & 0.31 & 46923.00 & 345412.97 & 0.87 & 0.84 & 0.74 \\
43718 & 109208 & 2003 & 36.60 & 0.45 & 2596.00 & 26100.02 & 1.41 & 0.71 & 1.01 \\
13017 & 101621 & 2003 & 2546.20 & 0.26 & 255511.00 & 2301939.59 & 1.00 & 0.90 & 0.90 \\
13807 & 101764 & 2003 & 596.30 & 0.28 & 57426.00 & 592673.89 & 1.04 & 0.99 & 1.03 \\
12509 & 101544 & 2003 & 326.70 & 0.27 & 32506.00 & 324749.96 & 1.01 & 0.99 & 1.00 \\
44828 & 109393 & 2003 & 10.60 & 0.27 & 992.00 & 10343.60 & 1.07 & 0.98 & 1.04 \\
20191 & 102676 & 2003 & 155.10 & 0.27 & 14805.00 & 155821.89 & 1.05 & 1.00 & 1.05 \\
64483 & 500605 & 2003 & 222.60 & 0.27 & 22258.00 & 222450.99 & 1.00 & 1.00 & 1.00 \\
22336 & 103007 & 2003 & 1117.30 & 0.28 & 108227.00 & 1071626.45 & 1.03 & 0.96 & 0.99 \\
43150 & 109067 & 2003 & 90.80 & 0.40 & 7000.00 & 76658.57 & 1.30 & 0.84 & 1.10 \\
64283 & 500595 & 2003 & 1859.80 & 0.30 & 185978.00 & 1853166.37 & 1.00 & 1.00 & 1.00 \\
61937 & 500310 & 2003 & 7.10 & 0.31 & 733.00 & 6841.43 & 0.97 & 0.96 & 0.93 \\
20888 & 102798 & 2003 & 246.60 & 0.28 & 22522.00 & 245887.78 & 1.09 & 1.00 & 1.09 \\
42967 & 109044 & 2003 & 159.00 & 0.26 & 15944.00 & 150882.13 & 1.00 & 0.95 & 0.95 \\
23105 & 103122 & 2003 & 267.10 & 0.35 & 27682.00 & 277005.67 & 0.96 & 1.04 & 1.00 \\
54431 & 367841 & 2003 & 432.30 & 0.29 & 43972.00 & 395292.36 & 0.98 & 0.91 & 0.90 \\
18475 & 102462 & 2003 & 10.30 & 0.83 & 1036.00 & 9578.94 & 0.99 & 0.93 & 0.92 \\
2401 & 100322 & 2003 & 406.90 & 0.25 & 37518.00 & 421557.61 & 1.08 & 1.04 & 1.12 \\
41188 & 108670 & 2003 & 287.20 & 0.37 & 28144.00 & 280091.23 & 1.02 & 0.98 & 1.00 \\
23450 & 103177 & 2003 & 209.40 & 0.28 & 20872.00 & 203800.57 & 1.00 & 0.97 & 0.98 \\
2236 & 100298 & 2003 & 566.00 & 0.25 & 55882.00 & 557878.92 & 1.01 & 0.99 & 1.00 \\
42816 & 109020 & 2003 & 369.60 & 0.39 & 33599.00 & 354254.42 & 1.10 & 0.96 & 1.05 \\
18593 & 102490 & 2003 & 57.30 & 0.24 & 5698.00 & 56913.72 & 1.01 & 0.99 & 1.00 \\
41213 & 108673 & 2003 & 154.70 & 0.45 & 16151.00 & 146974.10 & 0.96 & 0.95 & 0.91 \\
54288 & 366837 & 2003 & 11.40 & 0.39 & 948.00 & 11036.56 & 1.20 & 0.97 & 1.16 \\
43849 & 109225 & 2003 & 10.10 & 0.61 & 1326.00 & 10506.08 & 0.76 & 1.04 & 0.79 \\
4565 & 100639 & 2003 & 677.70 & 0.23 & 67693.00 & 668199.46 & 1.00 & 0.99 & 0.99 \\
42800 & 109019 & 2003 & 56.50 & 0.35 & 4881.00 & 51716.42 & 1.16 & 0.92 & 1.06 \\
54291 & 367042 & 2003 & 23.60 & 0.13 & 2368.00 & 23269.41 & 1.00 & 0.99 & 0.98 \\
55206 & 400072 & 2003 & 266.20 & 0.25 & 26893.00 & 268625.40 & 0.99 & 1.01 & 1.00 \\
43959 & 109238 & 2003 & 9.70 & 0.29 & 972.00 & 9296.37 & 1.00 & 0.96 & 0.96 \\
7836 & 101062 & 2003 & 2712.40 & 0.34 & 223856.00 & 2352316.34 & 1.21 & 0.87 & 1.05 \\
43941 & 109237 & 2003 & 66.20 & 0.57 & 6566.00 & 64819.42 & 1.01 & 0.98 & 0.99 \\
23416 & 103175 & 2003 & 825.20 & 0.23 & 82493.00 & 799792.97 & 1.00 & 0.97 & 0.97 \\
54427 & 367713 & 2003 & 1.40 & 0.45 & 148.00 & 1479.59 & 0.95 & 1.06 & 1.00 \\
40983 & 108170 & 2003 & 297.80 & 0.40 & 28169.00 & 297079.55 & 1.06 & 1.00 & 1.05 \\
42828 & 109023 & 2003 & 27.90 & 0.23 & 2617.00 & 28151.82 & 1.07 & 1.01 & 1.08 \\
40959 & 108168 & 2003 & 465.90 & 0.39 & 41860.00 & 423146.45 & 1.11 & 0.91 & 1.01 \\
44667 & 109358 & 2003 & 92.50 & 0.39 & 8409.00 & 90509.88 & 1.10 & 0.98 & 1.08 \\
40954 & 108166 & 2003 & 168.10 & 0.46 & 16651.00 & 166268.89 & 1.01 & 0.99 & 1.00 \\
43869 & 109226 & 2003 & 92.90 & 0.46 & 9657.00 & 86148.31 & 0.96 & 0.93 & 0.89 \\
41159 & 108211 & 2003 & 306.80 & 0.34 & 22434.00 & 247525.04 & 1.37 & 0.81 & 1.10 \\
54294 & 367116 & 2003 & 1.70 & 0.36 & 171.00 & 1677.20 & 0.99 & 0.99 & 0.98 \\
44684 & 109359 & 2003 & 109.00 & 0.23 & 16748.00 & 173163.66 & 0.65 & 1.59 & 1.03 \\
42823 & 109021 & 2003 & 22.30 & 0.46 & 2061.00 & 23118.63 & 1.08 & 1.04 & 1.12 \\
4492 & 100635 & 2003 & 1025.70 & 0.35 & 102219.00 & 975658.73 & 1.00 & 0.95 & 0.95 \\
13417 & 101738 & 2003 & 1883.80 & 0.36 & 188486.00 & 1836525.17 & 1.00 & 0.97 & 0.97 \\
42792 & 109018 & 2003 & 18.20 & 0.48 & 1580.00 & 14633.85 & 1.15 & 0.80 & 0.93 \\
54266 & 365483 & 2003 & 478.20 & 0.36 & 40739.00 & 441160.96 & 1.17 & 0.92 & 1.08 \\
23077 & 103110 & 2003 & 1016.50 & 0.32 & 101634.00 & 909386.84 & 1.00 & 0.89 & 0.89 \\
43973 & 109249 & 2003 & 204.20 & -0.85 & 21245.00 & 204946.44 & 0.96 & 1.00 & 0.96 \\
42981 & 109046 & 2003 & 60.60 & 0.37 & 6096.00 & 60849.96 & 0.99 & 1.00 & 1.00 \\
18458 & 102461 & 2003 & 1312.00 & 0.25 & 130491.00 & 1219564.60 & 1.01 & 0.93 & 0.93 \\
4133 & 100559 & 2003 & 20.70 & 0.43 & 1862.00 & 20790.18 & 1.11 & 1.00 & 1.12 \\
7317 & 101020 & 2003 & 4466.10 & 0.52 & 399828.00 & 3461809.22 & 1.12 & 0.78 & 0.87 \\
43005 & 109048 & 2003 & 249.20 & 0.42 & 24322.00 & 249469.46 & 1.02 & 1.00 & 1.03 \\
23034 & 103103 & 2003 & 343.90 & 0.22 & 34399.00 & 341195.57 & 1.00 & 0.99 & 0.99 \\
12841 & 101602 & 2003 & 2859.40 & 0.41 & 295058.00 & 2765553.09 & 0.97 & 0.97 & 0.94 \\
44643 & 109350 & 2003 & 40.30 & 0.29 & 4149.00 & 41783.48 & 0.97 & 1.04 & 1.01 \\
41268 & 108710 & 2003 & 534.40 & 0.58 & 53512.00 & 521263.12 & 1.00 & 0.98 & 0.97 \\
23523 & 103183 & 2003 & 855.30 & 0.87 & 85131.00 & 845126.45 & 1.00 & 0.99 & 0.99 \\
20622 & 102775 & 2003 & 3181.20 & 0.26 & 317196.00 & 2986789.22 & 1.00 & 0.94 & 0.94 \\
19472 & 102606 & 2003 & 3853.50 & 0.28 & 385328.00 & 3652509.45 & 1.00 & 0.95 & 0.95 \\
48586 & 240111 & 2003 & 1273.30 & 0.27 & 127534.00 & 1044084.72 & 1.00 & 0.82 & 0.82 \\
64145 & 500589 & 2003 & 1411.40 & 0.35 & 141136.00 & 1410863.12 & 1.00 & 1.00 & 1.00 \\
41280 & 108719 & 2003 & 174.10 & 0.37 & 17784.00 & 177336.25 & 0.98 & 1.02 & 1.00 \\
64168 & 500590 & 2003 & 625.10 & 0.37 & 62511.00 & 624797.98 & 1.00 & 1.00 & 1.00 \\
2170 & 100293 & 2003 & 161.50 & 0.34 & 16219.00 & 159186.91 & 1.00 & 0.99 & 0.98 \\
61712 & 500116 & 2003 & 722.50 & 0.53 & 77374.00 & 705530.32 & 0.93 & 0.98 & 0.91 \\
54446 & 367985 & 2003 & 199.00 & 0.20 & 19963.00 & 193029.46 & 1.00 & 0.97 & 0.97 \\
59364 & 410470 & 2003 & 4.60 & 0.06 & 248.00 & 2507.93 & 1.85 & 0.55 & 1.01 \\
64099 & 500587 & 2003 & 1316.20 & 0.31 & 131621.00 & 1311539.91 & 1.00 & 1.00 & 1.00 \\
54439 & 367842 & 2003 & 118.40 & 0.46 & 11763.00 & 112236.06 & 1.01 & 0.95 & 0.95 \\
20658 & 102777 & 2003 & 1924.60 & 0.34 & 192825.00 & 1854889.86 & 1.00 & 0.96 & 0.96 \\
54845 & 400018 & 2003 & 160.90 & 0.25 & 14817.00 & 149849.02 & 1.09 & 0.93 & 1.01 \\
2203 & 100295 & 2003 & 15.50 & 0.18 & 1415.00 & 14300.38 & 1.10 & 0.92 & 1.01 \\
44651 & 109351 & 2003 & 46.60 & 0.25 & 4803.00 & 48251.80 & 0.97 & 1.04 & 1.00 \\
13461 & 101740 & 2003 & 17122.70 & 0.32 & 1708919.00 & 16716311.85 & 1.00 & 0.98 & 0.98 \\
12828 & 101601 & 2003 & 172.80 & 0.51 & 17199.00 & 166865.17 & 1.00 & 0.97 & 0.97 \\
19370 & 102599 & 2003 & 1620.60 & 0.37 & 162056.00 & 1581668.91 & 1.00 & 0.98 & 0.98 \\
41240 & 108690 & 2003 & 67.10 & 0.44 & 7324.00 & 57647.83 & 0.92 & 0.86 & 0.79 \\
43843 & 109224 & 2003 & 7.90 & 0.33 & 792.00 & 7709.72 & 1.00 & 0.98 & 0.97 \\
2421 & 100323 & 2003 & 3090.90 & 0.25 & 295555.00 & 3017470.85 & 1.05 & 0.98 & 1.02 \\
54244 & 364993 & 2003 & 37.70 & 0.47 & 3440.00 & 35593.45 & 1.10 & 0.94 & 1.03 \\
64122 & 500588 & 2003 & 612.00 & 0.30 & 61204.00 & 612040.04 & 1.00 & 1.00 & 1.00 \\
18609 & 102491 & 2003 & 675.50 & 0.34 & 67624.00 & 658198.24 & 1.00 & 0.97 & 0.97 \\
44619 & 109348 & 2003 & 809.50 & 0.29 & 76343.00 & 770280.65 & 1.06 & 0.95 & 1.01 \\
12807 & 101600 & 2003 & 1629.50 & 0.27 & 166461.00 & 1592379.11 & 0.98 & 0.98 & 0.96 \\
63937 & 500568 & 2003 & 16.20 & 0.36 & 1467.00 & 14116.76 & 1.10 & 0.87 & 0.96 \\
48144 & 240027 & 2003 & 485.00 & 0.33 & 45726.00 & 433470.65 & 1.06 & 0.89 & 0.95 \\
42868 & 109028 & 2003 & 308.60 & 0.34 & 32789.00 & 288064.39 & 0.94 & 0.93 & 0.88 \\
19404 & 102600 & 2003 & 820.10 & 0.23 & 82009.00 & 799688.75 & 1.00 & 0.98 & 0.98 \\
41017 & 108180 & 2003 & 67.40 & 0.30 & 5803.00 & 56795.42 & 1.16 & 0.84 & 0.98 \\
2369 & 100320 & 2003 & 69.30 & 0.44 & 5322.00 & 67222.37 & 1.30 & 0.97 & 1.26 \\
54362 & 367231 & 2003 & 18.30 & 0.25 & 1835.00 & 18191.27 & 1.00 & 0.99 & 0.99 \\
54405 & 367600 & 2003 & 22.50 & 0.54 & 2176.00 & 19740.42 & 1.03 & 0.88 & 0.91 \\
58784 & 410217 & 2003 & 4.10 & 0.43 & 396.00 & 3915.55 & 1.04 & 0.96 & 0.99 \\
7804 & 101061 & 2003 & 3655.90 & 0.37 & 370584.00 & 3420510.81 & 0.99 & 0.94 & 0.92 \\
4114 & 100552 & 2003 & 60.60 & 0.43 & 6080.00 & 58981.75 & 1.00 & 0.97 & 0.97 \\
12786 & 101595 & 2003 & 1347.80 & 0.31 & 132417.00 & 1240761.22 & 1.02 & 0.92 & 0.94 \\
43910 & 109230 & 2003 & 161.00 & 0.47 & 14891.00 & 149023.47 & 1.08 & 0.93 & 1.00 \\
42902 & 109031 & 2003 & 25.50 & 0.39 & 2557.00 & 25326.06 & 1.00 & 0.99 & 0.99 \\
41040 & 108183 & 2003 & 99.00 & 0.38 & 9653.00 & 92601.56 & 1.03 & 0.94 & 0.96 \\
54387 & 367567 & 2003 & 376.60 & 0.42 & 29453.00 & 305190.40 & 1.28 & 0.81 & 1.04 \\
48596 & 240112 & 2003 & 18.00 & 0.40 & 1781.00 & 17618.42 & 1.01 & 0.98 & 0.99 \\
23285 & 103154 & 2003 & 388.00 & 0.41 & 34959.00 & 374423.02 & 1.11 & 0.97 & 1.07 \\
41061 & 108186 & 2003 & 10.90 & 0.28 & 1087.00 & 11000.67 & 1.00 & 1.01 & 1.01 \\
20771 & 102789 & 2003 & 999.00 & 0.27 & 94750.00 & 915419.12 & 1.05 & 0.92 & 0.97 \\
43907 & 109229 & 2003 & 32.80 & 0.23 & 3349.00 & 32377.55 & 0.98 & 0.99 & 0.97 \\
3499 & 100441 & 2003 & 312.90 & 0.22 & 31621.00 & 306496.43 & 0.99 & 0.98 & 0.97 \\
2337 & 100319 & 2003 & 305.10 & 0.51 & 20844.00 & 251459.05 & 1.46 & 0.82 & 1.21 \\
23253 & 103152 & 2003 & 2700.20 & 0.23 & 256099.00 & 2591807.79 & 1.05 & 0.96 & 1.01 \\
20810 & 102795 & 2003 & 665.70 & 0.45 & 59646.00 & 614164.48 & 1.12 & 0.92 & 1.03 \\
61755 & 500120 & 2003 & 7.70 & 0.26 & 704.00 & 7829.07 & 1.09 & 1.02 & 1.11 \\
43884 & 109228 & 2003 & 21.10 & 0.29 & 2250.00 & 21039.70 & 0.94 & 1.00 & 0.94 \\
4531 & 100637 & 2003 & 1054.40 & 0.29 & 105631.00 & 1046236.33 & 1.00 & 0.99 & 0.99 \\
23300 & 103158 & 2003 & 1543.10 & 0.30 & 141991.00 & 1615434.28 & 1.09 & 1.05 & 1.14 \\
41086 & 108192 & 2003 & 36.40 & 0.29 & 6873.00 & 67303.56 & 0.53 & 1.85 & 0.98 \\
42934 & 109037 & 2003 & 80.80 & 0.34 & 8073.00 & 75581.32 & 1.00 & 0.94 & 0.94 \\
55229 & 400074 & 2003 & 1622.30 & 0.23 & 162237.00 & 1613432.81 & 1.00 & 0.99 & 0.99 \\
64076 & 500586 & 2003 & 2326.30 & 0.30 & 232631.00 & 2316808.33 & 1.00 & 1.00 & 1.00 \\
48295 & 240060 & 2003 & 187.10 & 0.43 & 18741.00 & 152937.15 & 1.00 & 0.82 & 0.82 \\
2263 & 100303 & 2003 & 353.80 & 0.25 & 28578.00 & 309748.65 & 1.24 & 0.88 & 1.08 \\
54865 & 400019 & 2003 & 1159.40 & 0.37 & 108597.00 & 1109972.63 & 1.07 & 0.96 & 1.02 \\
54299 & 367166 & 2003 & 142.50 & 0.29 & 14503.00 & 134676.56 & 0.98 & 0.95 & 0.93 \\
3528 & 100453 & 2003 & 131.30 & 0.21 & 12595.00 & 124051.37 & 1.04 & 0.94 & 0.98 \\
44659 & 109357 & 2003 & 327.20 & 0.27 & 34650.00 & 299915.82 & 0.94 & 0.92 & 0.87 \\
63950 & 500571 & 2003 & 116.70 & 0.37 & 10775.00 & 105890.07 & 1.08 & 0.91 & 0.98 \\
23161 & 103136 & 2003 & 190.70 & 0.23 & 19096.00 & 187323.03 & 1.00 & 0.98 & 0.98 \\
23386 & 103174 & 2003 & 1488.90 & 0.18 & 149273.00 & 1477143.94 & 1.00 & 0.99 & 0.99 \\
13981 & 101794 & 2003 & 1296.40 & 0.39 & 128999.00 & 1264566.58 & 1.00 & 0.98 & 0.98 \\
43936 & 109233 & 2003 & 171.10 & 0.28 & 15072.00 & 127414.34 & 1.14 & 0.74 & 0.85 \\
48612 & 240114 & 2003 & 349.40 & 0.29 & 34992.00 & 358587.77 & 1.00 & 1.03 & 1.02 \\
63946 & 500570 & 2003 & 60.10 & 0.23 & 5319.00 & 58709.99 & 1.13 & 0.98 & 1.10 \\
41004 & 108175 & 2003 & 69.90 & 0.30 & 6297.00 & 70173.29 & 1.11 & 1.00 & 1.11 \\
18497 & 102465 & 2003 & 309.80 & 0.27 & 27617.00 & 300633.06 & 1.12 & 0.97 & 1.09 \\
20860 & 102797 & 2003 & 55.40 & 0.39 & 6090.00 & 56528.32 & 0.91 & 1.02 & 0.93 \\
23145 & 103134 & 2003 & 410.10 & 0.27 & 42167.00 & 398010.76 & 0.97 & 0.97 & 0.94 \\
40992 & 108172 & 2003 & 193.00 & 0.48 & 19573.00 & 172224.38 & 0.99 & 0.89 & 0.88 \\
18581 & 102489 & 2003 & 59.00 & 0.24 & 5867.00 & 57975.66 & 1.01 & 0.98 & 0.99 \\
54321 & 367168 & 2003 & 3.20 & 0.35 & 318.00 & 3135.60 & 1.01 & 0.98 & 0.99 \\
2276 & 100305 & 2003 & 77.20 & 0.22 & 6697.00 & 73971.35 & 1.15 & 0.96 & 1.10 \\
41092 & 108194 & 2003 & 32.00 & 0.36 & 3062.00 & 30618.33 & 1.05 & 0.96 & 1.00 \\
41095 & 108197 & 2003 & 30.70 & 0.29 & 3058.00 & 29454.24 & 1.00 & 0.96 & 0.96 \\
20732 & 102788 & 2003 & 306.10 & 0.28 & 28644.00 & 282315.17 & 1.07 & 0.92 & 0.99 \\
61790 & 500131 & 2003 & 29.20 & 0.04 & 2399.00 & 25459.72 & 1.22 & 0.87 & 1.06 \\
54340 & 367206 & 2003 & 206.90 & 0.49 & 20379.00 & 195006.82 & 1.02 & 0.94 & 0.96 \\
41101 & 108200 & 2003 & 125.60 & 0.27 & 11885.00 & 126424.56 & 1.06 & 1.01 & 1.06 \\
7433 & 101039 & 2003 & 5246.20 & 0.20 & 478675.00 & 4556176.70 & 1.10 & 0.87 & 0.95 \\
23354 & 103166 & 2003 & 1.40 & 0.12 & 137.00 & 1381.24 & 1.02 & 0.99 & 1.01 \\
23190 & 103144 & 2003 & 23.70 & 0.46 & 2185.00 & 24325.66 & 1.08 & 1.03 & 1.11 \\
42944 & 109038 & 2003 & 18.70 & 0.44 & 1872.00 & 17669.42 & 1.00 & 0.94 & 0.94 \\
7773 & 101057 & 2003 & 26725.20 & 0.41 & 1959925.00 & 18103439.39 & 1.36 & 0.68 & 0.92 \\
41011 & 108176 & 2003 & 21.00 & 0.30 & 2104.00 & 20938.57 & 1.00 & 1.00 & 1.00 \\
7168 & 101000 & 2003 & 1401.80 & 0.28 & 134521.00 & 1402319.14 & 1.04 & 1.00 & 1.04 \\
3465 & 100439 & 2003 & 18.00 & 0.05 & 2012.00 & 19730.22 & 0.89 & 1.10 & 0.98 \\
19438 & 102601 & 2003 & 6361.50 & 0.24 & 636139.00 & 6171650.78 & 1.00 & 0.97 & 0.97 \\
41113 & 108202 & 2003 & 101.70 & 0.63 & 9839.00 & 93808.38 & 1.03 & 0.92 & 0.95 \\
48605 & 240113 & 2003 & 51.40 & 0.32 & 4659.00 & 49027.76 & 1.10 & 0.95 & 1.05 \\
20713 & 102784 & 2003 & 17287.30 & 0.24 & 1732309.00 & 17033173.47 & 1.00 & 0.99 & 0.98 \\
23003 & 103101 & 2003 & 200.40 & 0.29 & 20101.00 & 199218.05 & 1.00 & 0.99 & 0.99 \\
64191 & 500591 & 2003 & 864.20 & 0.38 & 86421.00 & 861850.53 & 1.00 & 1.00 & 1.00 \\
7204 & 101013 & 2003 & 6004.00 & 0.29 & 589448.00 & 5767079.24 & 1.02 & 0.96 & 0.98 \\
61559 & 500094 & 2003 & 143.90 & 0.33 & 14468.00 & 140904.75 & 0.99 & 0.98 & 0.97 \\
64260 & 500594 & 2003 & 1607.30 & 0.46 & 160726.00 & 1602369.66 & 1.00 & 1.00 & 1.00 \\
48183 & 240040 & 2003 & 304.70 & 0.30 & 25258.00 & 307525.51 & 1.21 & 1.01 & 1.22 \\
2102 & 100291 & 2003 & 1213.20 & 0.44 & 120520.00 & 1193064.82 & 1.01 & 0.98 & 0.99 \\
41426 & 108752 & 2003 & 9.50 & 0.21 & 948.00 & 9362.45 & 1.00 & 0.99 & 0.99 \\
59325 & 410465 & 2003 & 11.90 & 0.17 & 1186.00 & 11803.67 & 1.00 & 0.99 & 1.00 \\
49001 & 240198 & 2003 & 306.20 & 0.25 & 30693.00 & 306368.10 & 1.00 & 1.00 & 1.00 \\
4599 & 100642 & 2003 & 1195.40 & 0.30 & 113473.00 & 1165496.91 & 1.05 & 0.97 & 1.03 \\
43081 & 109061 & 2003 & 162.20 & 0.51 & 15025.00 & 146763.88 & 1.08 & 0.90 & 0.98 \\
23727 & 103209 & 2003 & 166.80 & 0.38 & 16559.00 & 165215.14 & 1.01 & 0.99 & 1.00 \\
54480 & 372363 & 2003 & 12.70 & 0.29 & 1196.00 & 12642.14 & 1.06 & 1.00 & 1.06 \\
43086 & 109062 & 2003 & 27.60 & 0.23 & 2759.00 & 26786.83 & 1.00 & 0.97 & 0.97 \\
22797 & 103065 & 2003 & 192.30 & 0.39 & 18963.00 & 183511.16 & 1.01 & 0.95 & 0.97 \\
61927 & 500308 & 2003 & 88.70 & 0.27 & 8944.00 & 88887.78 & 0.99 & 1.00 & 0.99 \\
42711 & 109008 & 2003 & 40.20 & 0.32 & 5017.00 & 37250.14 & 0.80 & 0.93 & 0.74 \\
42717 & 109009 & 2003 & 160.70 & 0.27 & 12703.00 & 135861.12 & 1.27 & 0.85 & 1.07 \\
48539 & 240105 & 2003 & 229.80 & 0.72 & 23042.00 & 217961.71 & 1.00 & 0.95 & 0.95 \\
3625 & 100463 & 2003 & 6650.20 & 0.42 & 674452.00 & 6607189.46 & 0.99 & 0.99 & 0.98 \\
63773 & 500553 & 2003 & 193.40 & 0.61 & 19939.00 & 191106.35 & 0.97 & 0.99 & 0.96 \\
40824 & 108155 & 2003 & 79.40 & 0.29 & 7940.00 & 79270.76 & 1.00 & 1.00 & 1.00 \\
44750 & 109371 & 2003 & 395.00 & 0.31 & 39583.00 & 378092.38 & 1.00 & 0.96 & 0.96 \\
41374 & 108736 & 2003 & 108.70 & 0.33 & 9417.00 & 91510.76 & 1.15 & 0.84 & 0.97 \\
40819 & 108154 & 2003 & 34.90 & 0.28 & 3373.00 & 33733.26 & 1.03 & 0.97 & 1.00 \\
18402 & 102447 & 2003 & 5183.30 & 0.32 & 518324.00 & 4784601.88 & 1.00 & 0.92 & 0.92 \\
64237 & 500593 & 2003 & 995.70 & 0.38 & 99566.00 & 995353.50 & 1.00 & 1.00 & 1.00 \\
42740 & 109010 & 2003 & 15.80 & 0.33 & 1577.00 & 15404.12 & 1.00 & 0.97 & 0.98 \\
14025 & 101800 & 2003 & 500.90 & 0.27 & 50361.00 & 481196.92 & 0.99 & 0.96 & 0.96 \\
23693 & 103208 & 2003 & 1591.00 & 0.38 & 157935.00 & 1449200.90 & 1.01 & 0.91 & 0.92 \\
20503 & 102760 & 2003 & 1556.90 & 0.32 & 156246.00 & 1532574.96 & 1.00 & 0.98 & 0.98 \\
4441 & 100625 & 2003 & 2008.50 & 0.19 & 201126.00 & 1969084.30 & 1.00 & 0.98 & 0.98 \\
41392 & 108745 & 2003 & 55.10 & 0.35 & 5504.00 & 49657.79 & 1.00 & 0.90 & 0.90 \\
40794 & 108153 & 2003 & 29.60 & 0.48 & 2959.00 & 24482.63 & 1.00 & 0.83 & 0.83 \\
22832 & 103067 & 2003 & 61.90 & 0.26 & 6209.00 & 61633.59 & 1.00 & 1.00 & 0.99 \\
48931 & 240153 & 2003 & 85.90 & 0.31 & 8587.00 & 84833.43 & 1.00 & 0.99 & 0.99 \\
13777 & 101763 & 2003 & 113.60 & 0.39 & 11412.00 & 110641.74 & 1.00 & 0.97 & 0.97 \\
20474 & 102757 & 2003 & 8953.70 & 0.32 & 874618.00 & 8742583.33 & 1.02 & 0.98 & 1.00 \\
7706 & 101055 & 2003 & 22608.70 & 0.41 & 2262691.00 & 21523712.10 & 1.00 & 0.95 & 0.95 \\
3357 & 100425 & 2003 & 2560.30 & 0.28 & 237285.00 & 2582365.58 & 1.08 & 1.01 & 1.09 \\
4179 & 100567 & 2003 & 934.10 & 0.25 & 93286.00 & 882112.77 & 1.00 & 0.94 & 0.95 \\
23763 & 103212 & 2003 & 2850.10 & 0.37 & 285423.00 & 2700944.90 & 1.00 & 0.95 & 0.95 \\
13517 & 101742 & 2003 & 5366.30 & 0.32 & 535887.00 & 4790311.77 & 1.00 & 0.89 & 0.89 \\
44572 & 109341 & 2003 & 312.10 & 0.44 & 31297.00 & 302707.28 & 1.00 & 0.97 & 0.97 \\
54485 & 372487 & 2003 & 2.20 & 0.39 & 214.00 & 2047.86 & 1.03 & 0.93 & 0.96 \\
43102 & 109064 & 2003 & 116.90 & 0.74 & 10930.00 & 105932.43 & 1.07 & 0.91 & 0.97 \\
59303 & 410463 & 2003 & 288.60 & 0.27 & 21640.00 & 193716.70 & 1.33 & 0.67 & 0.90 \\
43125 & 109065 & 2003 & 43.70 & 0.29 & 4328.00 & 43131.33 & 1.01 & 0.99 & 1.00 \\
63651 & 500539 & 2003 & 7.40 & 0.26 & 592.00 & 7248.42 & 1.25 & 0.98 & 1.22 \\
44549 & 109338 & 2003 & 258.10 & 0.50 & 20457.00 & 198365.39 & 1.26 & 0.77 & 0.97 \\
12918 & 101606 & 2003 & 3592.30 & 0.20 & 367644.00 & 3278983.11 & 0.98 & 0.91 & 0.89 \\
44014 & 109258 & 2003 & 423.20 & 0.24 & 35502.00 & 370925.65 & 1.19 & 0.88 & 1.04 \\
2615 & 100347 & 2003 & 752.40 & 0.28 & 76195.00 & 741488.61 & 0.99 & 0.99 & 0.97 \\
12613 & 101560 & 2003 & 37.30 & 0.80 & 3728.00 & 36398.94 & 1.00 & 0.98 & 0.98 \\
44757 & 109373 & 2003 & 188.60 & 0.52 & 17966.00 & 179636.25 & 1.05 & 0.95 & 1.00 \\
4147 & 100561 & 2003 & 77.20 & 0.32 & 6768.00 & 72436.84 & 1.14 & 0.94 & 1.07 \\
49014 & 240199 & 2003 & 398.40 & 0.54 & 39759.00 & 387952.55 & 1.00 & 0.97 & 0.98 \\
20452 & 102744 & 2003 & 858.20 & 0.26 & 86156.00 & 859625.32 & 1.00 & 1.00 & 1.00 \\
61932 & 500309 & 2003 & 30.50 & 0.37 & 3111.00 & 30531.31 & 0.98 & 1.00 & 0.98 \\
7466 & 101040 & 2003 & 3987.00 & 0.27 & 376422.00 & 3808291.89 & 1.06 & 0.96 & 1.01 \\
2543 & 100343 & 2003 & 598.70 & 0.23 & 52815.00 & 576132.20 & 1.13 & 0.96 & 1.09 \\
54890 & 400020 & 2003 & 70.80 & 0.29 & 7008.00 & 70742.95 & 1.01 & 1.00 & 1.01 \\
43781 & 109219 & 2003 & 47.50 & 0.59 & 4736.00 & 47414.13 & 1.00 & 1.00 & 1.00 \\
63733 & 500550 & 2003 & 43815.40 & 0.28 & 3603201.00 & 40242782.23 & 1.22 & 0.92 & 1.12 \\
63729 & 500549 & 2003 & 1.70 & 0.32 & 168.00 & 1508.94 & 1.01 & 0.89 & 0.90 \\
21069 & 102827 & 2003 & 210.80 & 0.25 & 21060.00 & 203723.45 & 1.00 & 0.97 & 0.97 \\
44581 & 109343 & 2003 & 128.60 & 0.25 & 12756.00 & 107022.09 & 1.01 & 0.83 & 0.84 \\
41438 & 108759 & 2003 & 45.90 & 0.62 & 4565.00 & 45662.82 & 1.01 & 0.99 & 1.00 \\
61493 & 500083 & 2003 & 23.00 & 0.34 & 1988.00 & 22347.67 & 1.16 & 0.97 & 1.12 \\
43098 & 109063 & 2003 & 13.20 & 0.50 & 1282.00 & 11596.64 & 1.03 & 0.88 & 0.90 \\
19305 & 102588 & 2003 & 336.50 & 0.32 & 32756.00 & 331862.40 & 1.03 & 0.99 & 1.01 \\
41462 & 108760 & 2003 & 133.00 & 0.29 & 13847.00 & 138316.62 & 0.96 & 1.04 & 1.00 \\
42703 & 108996 & 2003 & 47.70 & 0.28 & 4773.00 & 46343.35 & 1.00 & 0.97 & 0.97 \\
21027 & 102824 & 2003 & 45.70 & 0.27 & 4391.00 & 46863.69 & 1.04 & 1.03 & 1.07 \\
19512 & 102608 & 2003 & 118.50 & 0.33 & 11844.00 & 105511.30 & 1.00 & 0.89 & 0.89 \\
48565 & 240107 & 2003 & 81.30 & 0.33 & 8315.00 & 71250.49 & 0.98 & 0.88 & 0.86 \\
13486 & 101741 & 2003 & 5977.10 & 0.25 & 596109.00 & 5488395.83 & 1.00 & 0.92 & 0.92 \\
43028 & 109052 & 2003 & 39.90 & 0.27 & 4048.00 & 39816.69 & 0.99 & 1.00 & 0.98 \\
19495 & 102607 & 2003 & 625.20 & 0.38 & 62525.00 & 612307.10 & 1.00 & 0.98 & 0.98 \\
42787 & 109017 & 2003 & 73.00 & 0.28 & 6319.00 & 68950.43 & 1.16 & 0.94 & 1.09 \\
40905 & 108163 & 2003 & 133.00 & 0.54 & 9081.00 & 86354.61 & 1.46 & 0.65 & 0.95 \\
54235 & 364950 & 2003 & 5.90 & 0.20 & 590.00 & 5774.42 & 1.00 & 0.98 & 0.98 \\
22951 & 103090 & 2003 & 751.60 & 0.30 & 66390.00 & 705567.82 & 1.13 & 0.94 & 1.06 \\
18643 & 102493 & 2003 & 1978.40 & 0.25 & 198220.00 & 1966230.14 & 1.00 & 0.99 & 0.99 \\
43034 & 109056 & 2003 & 616.30 & 0.35 & 63767.00 & 518460.64 & 0.97 & 0.84 & 0.81 \\
42750 & 109015 & 2003 & 169.30 & 0.29 & 15211.00 & 165620.01 & 1.11 & 0.98 & 1.09 \\
63867 & 500562 & 2003 & 22.50 & 0.27 & 2244.00 & 22441.25 & 1.00 & 1.00 & 1.00 \\
63857 & 500561 & 2003 & 20.40 & 0.34 & 2060.00 & 20400.76 & 0.99 & 1.00 & 0.99 \\
20562 & 102767 & 2003 & 5210.90 & 0.27 & 517247.00 & 5083613.51 & 1.01 & 0.98 & 0.98 \\
40894 & 108161 & 2003 & 258.40 & 0.29 & 25805.00 & 253334.95 & 1.00 & 0.98 & 0.98 \\
43980 & 109250 & 2003 & 169.60 & 0.29 & 16659.00 & 160103.40 & 1.02 & 0.94 & 0.96 \\
18432 & 102452 & 2003 & 120.60 & 0.37 & 12026.00 & 119106.67 & 1.00 & 0.99 & 0.99 \\
48987 & 240197 & 2003 & 125.20 & 0.25 & 12655.00 & 124109.80 & 0.99 & 0.99 & 0.98 \\
20981 & 102814 & 2003 & 183.40 & 0.34 & 18309.00 & 181275.56 & 1.00 & 0.99 & 0.99 \\
48937 & 240154 & 2003 & 37.70 & 0.39 & 3759.00 & 37649.13 & 1.00 & 1.00 & 1.00 \\
23605 & 103202 & 2003 & 54.00 & 0.27 & 5358.00 & 50825.45 & 1.01 & 0.94 & 0.95 \\
23575 & 103193 & 2003 & 56.70 & 0.42 & 5653.00 & 53727.40 & 1.00 & 0.95 & 0.95 \\
54237 & 364951 & 2003 & 598.10 & 0.35 & 59829.00 & 479502.74 & 1.00 & 0.80 & 0.80 \\
61900 & 500290 & 2003 & 77.20 & 0.22 & 6828.00 & 70278.36 & 1.13 & 0.91 & 1.03 \\
2444 & 100330 & 2003 & 1948.60 & 0.27 & 195718.00 & 2040739.66 & 1.00 & 1.05 & 1.04 \\
43024 & 109051 & 2003 & 19.20 & 0.42 & 1915.00 & 19248.90 & 1.00 & 1.00 & 1.01 \\
41305 & 108723 & 2003 & 184.70 & 0.28 & 19161.00 & 196765.04 & 0.96 & 1.07 & 1.03 \\
20602 & 102774 & 2003 & 3657.80 & 0.33 & 366396.00 & 3753480.42 & 1.00 & 1.03 & 1.02 \\
12674 & 101562 & 2003 & 346.00 & 0.26 & 30527.00 & 342547.92 & 1.13 & 0.99 & 1.12 \\
7740 & 101056 & 2003 & 30571.70 & 0.19 & 2826341.00 & 28329959.05 & 1.08 & 0.93 & 1.00 \\
22985 & 103100 & 2003 & 244.20 & 0.27 & 24540.00 & 237340.76 & 1.00 & 0.97 & 0.97 \\
43820 & 109223 & 2003 & 9.00 & 0.64 & 834.00 & 7344.93 & 1.08 & 0.82 & 0.88 \\
23556 & 103186 & 2003 & 855.10 & 0.59 & 85698.00 & 852888.00 & 1.00 & 1.00 & 1.00 \\
2139 & 100292 & 2003 & 2380.20 & 0.28 & 237708.00 & 2083120.76 & 1.00 & 0.88 & 0.88 \\
61675 & 500114 & 2003 & 26.10 & 0.65 & 2613.00 & 26108.39 & 1.00 & 1.00 & 1.00 \\
20951 & 102813 & 2003 & 1191.70 & 0.35 & 119114.00 & 1103127.87 & 1.00 & 0.93 & 0.93 \\
63878 & 500563 & 2003 & 21.90 & 0.37 & 2189.00 & 21801.65 & 1.00 & 1.00 & 1.00 \\
4481 & 100634 & 2003 & 958.90 & 0.26 & 96211.00 & 957779.52 & 1.00 & 1.00 & 1.00 \\
41315 & 108726 & 2003 & 15.60 & 0.35 & 1569.00 & 15988.41 & 0.99 & 1.02 & 1.02 \\
3581 & 100457 & 2003 & 202.30 & 0.29 & 20271.00 & 197230.47 & 1.00 & 0.97 & 0.97 \\
22966 & 103099 & 2003 & 115.40 & 0.28 & 11533.00 & 101205.85 & 1.00 & 0.88 & 0.88 \\
54455 & 367992 & 2003 & 72.00 & 0.28 & 7251.00 & 68018.29 & 0.99 & 0.94 & 0.94 \\
3593 & 100460 & 2003 & 360.70 & 0.39 & 36107.00 & 350887.90 & 1.00 & 0.97 & 0.97 \\
18658 & 102500 & 2003 & 99.70 & 0.33 & 10015.00 & 96853.76 & 1.00 & 0.97 & 0.97 \\
40844 & 108158 & 2003 & 30.00 & 0.29 & 2900.00 & 29560.29 & 1.03 & 0.99 & 1.02 \\
19336 & 102597 & 2003 & 237.60 & 0.66 & 23780.00 & 236925.01 & 1.00 & 1.00 & 1.00 \\
54227 & 364947 & 2003 & 29.60 & 0.41 & 2120.00 & 20529.24 & 1.40 & 0.69 & 0.97 \\
2509 & 100336 & 2003 & 124.90 & 0.22 & 11485.00 & 114775.23 & 1.09 & 0.92 & 1.00 \\
61909 & 500306 & 2003 & 4.90 & 0.51 & 384.00 & 3505.41 & 1.28 & 0.72 & 0.91 \\
40834 & 108156 & 2003 & 16.20 & 0.26 & 1430.00 & 15511.34 & 1.13 & 0.96 & 1.08 \\
44596 & 109347 & 2003 & 831.40 & 0.41 & 78634.00 & 798248.51 & 1.06 & 0.96 & 1.02 \\
63806 & 500556 & 2003 & 112.60 & 0.49 & 7585.00 & 92824.46 & 1.48 & 0.82 & 1.22 \\
20525 & 102761 & 2003 & 19593.60 & 0.27 & 2038881.00 & 21382885.46 & 0.96 & 1.09 & 1.05 \\
44735 & 109370 & 2003 & 964.70 & 0.39 & 98406.00 & 848662.03 & 0.98 & 0.88 & 0.86 \\
12643 & 101561 & 2003 & 113.90 & 0.50 & 8299.00 & 105574.93 & 1.37 & 0.93 & 1.27 \\
54827 & 400017 & 2003 & 145.30 & 0.29 & 13457.00 & 135733.06 & 1.08 & 0.93 & 1.01 \\
63778 & 500554 & 2003 & 230.90 & 0.30 & 23023.00 & 219522.54 & 1.00 & 0.95 & 0.95 \\
23660 & 103205 & 2003 & 62.20 & 0.50 & 6028.00 & 60290.20 & 1.03 & 0.97 & 1.00 \\
22864 & 103073 & 2003 & 549.10 & 0.28 & 54903.00 & 540562.31 & 1.00 & 0.98 & 0.98 \\
12885 & 101603 & 2003 & 2158.20 & 0.23 & 218699.00 & 2117160.15 & 0.99 & 0.98 & 0.97 \\
21008 & 102821 & 2003 & 113.40 & 0.38 & 11321.00 & 107637.51 & 1.00 & 0.95 & 0.95 \\
63846 & 500560 & 2003 & 19.00 & 0.23 & 1907.00 & 18944.91 & 1.00 & 1.00 & 0.99 \\
44725 & 109368 & 2003 & 470.00 & 0.29 & 47097.00 & 426055.76 & 1.00 & 0.91 & 0.90 \\
22895 & 103084 & 2003 & 526.50 & 0.41 & 49143.00 & 513870.88 & 1.07 & 0.98 & 1.05 \\
42745 & 109011 & 2003 & 44.10 & 0.39 & 4405.00 & 43479.81 & 1.00 & 0.99 & 0.99 \\
3399 & 100431 & 2003 & 362.50 & 0.21 & 34341.00 & 358877.38 & 1.06 & 0.99 & 1.05 \\
43989 & 109255 & 2003 & 457.60 & 0.34 & 45778.00 & 456367.46 & 1.00 & 1.00 & 1.00 \\
63827 & 500559 & 2003 & 32.30 & 0.06 & 2737.00 & 29778.71 & 1.18 & 0.92 & 1.09 \\
61636 & 500109 & 2003 & 4514.80 & 0.47 & 451475.00 & 4502068.78 & 1.00 & 1.00 & 1.00 \\
41326 & 108728 & 2003 & 47.00 & 0.27 & 4218.00 & 44491.98 & 1.11 & 0.95 & 1.05 \\
40877 & 108160 & 2003 & 19.70 & 0.24 & 1972.00 & 19490.36 & 1.00 & 0.99 & 0.99 \\
48647 & 240116 & 2003 & 138.00 & 0.31 & 12182.00 & 132452.92 & 1.13 & 0.96 & 1.09 \\
54462 & 368366 & 2003 & 126.40 & 0.59 & 11476.00 & 115688.15 & 1.10 & 0.92 & 1.01 \\
61616 & 500107 & 2003 & 127.50 & 0.38 & 13481.00 & 135240.94 & 0.95 & 1.06 & 1.00 \\
4456 & 100633 & 2003 & 419.70 & 0.26 & 42096.00 & 413411.34 & 1.00 & 0.99 & 0.98 \\
41351 & 108732 & 2003 & 231.70 & 0.40 & 23233.00 & 191527.21 & 1.00 & 0.83 & 0.82 \\
20997 & 102818 & 2003 & 20.60 & 0.21 & 2041.00 & 19593.52 & 1.01 & 0.95 & 0.96 \\
64214 & 500592 & 2003 & 1477.00 & 0.32 & 147700.00 & 1475278.30 & 1.00 & 1.00 & 1.00 \\
41364 & 108733 & 2003 & 29.50 & -0.08 & 3562.00 & 33930.20 & 0.83 & 1.15 & 0.95 \\
43055 & 109058 & 2003 & 40.70 & 0.33 & 4071.00 & 39526.47 & 1.00 & 0.97 & 0.97 \\
23629 & 103204 & 2003 & 145.90 & 0.25 & 14604.00 & 143193.52 & 1.00 & 0.98 & 0.98 \\
13385 & 101736 & 2003 & 48.90 & 0.24 & 4904.00 & 48615.22 & 1.00 & 0.99 & 0.99 \\
40853 & 108159 & 2003 & 48.30 & 0.44 & 4827.00 & 48265.26 & 1.00 & 1.00 & 1.00 \\
40099 & 108029 & 2003 & 339.40 & 0.28 & 32222.00 & 319008.53 & 1.05 & 0.94 & 0.99 \\
64053 & 500585 & 2003 & 1514.70 & 0.48 & 151469.00 & 1513163.92 & 1.00 & 1.00 & 1.00 \\
44994 & 109410 & 2003 & 106.50 & 0.36 & 9352.00 & 94180.03 & 1.14 & 0.88 & 1.01 \\
24538 & 103339 & 2003 & 852.30 & 0.30 & 85256.00 & 832115.97 & 1.00 & 0.98 & 0.98 \\
21772 & 102951 & 2003 & 5322.70 & 0.26 & 488735.00 & 5294014.47 & 1.09 & 0.99 & 1.08 \\
42028 & 108910 & 2003 & 32.60 & 0.37 & 3283.00 & 32584.30 & 0.99 & 1.00 & 0.99 \\
42152 & 108930 & 2003 & 173.90 & 0.26 & 17316.00 & 173066.01 & 1.00 & 1.00 & 1.00 \\
24424 & 103326 & 2003 & 1214.10 & 0.38 & 115198.00 & 1223836.25 & 1.05 & 1.01 & 1.06 \\
18949 & 102529 & 2003 & 16.20 & -0.06 & 1627.00 & 15774.72 & 1.00 & 0.97 & 0.97 \\
2957 & 100389 & 2003 & 405.00 & 0.25 & 31374.00 & 325547.69 & 1.29 & 0.80 & 1.04 \\
19026 & 102544 & 2003 & 867.60 & 0.26 & 72966.00 & 824746.67 & 1.19 & 0.95 & 1.13 \\
43446 & 109122 & 2003 & 4.40 & 0.26 & 452.00 & 4515.85 & 0.97 & 1.03 & 1.00 \\
64695 & 500620 & 2003 & 485.60 & 0.24 & 48563.00 & 485073.77 & 1.00 & 1.00 & 1.00 \\
19802 & 102652 & 2003 & 1963.80 & 0.28 & 196375.00 & 1901120.62 & 1.00 & 0.97 & 0.97 \\
19737 & 102650 & 2003 & 13756.10 & 0.26 & 1375608.00 & 13071348.71 & 1.00 & 0.95 & 0.95 \\
58400 & 410153 & 2003 & 12.60 & 0.41 & 1196.00 & 11946.28 & 1.05 & 0.95 & 1.00 \\
42008 & 108907 & 2003 & 864.20 & 0.34 & 75472.00 & 744053.45 & 1.15 & 0.86 & 0.99 \\
40011 & 107994 & 2003 & 922.70 & 0.75 & 81784.00 & 850651.71 & 1.13 & 0.92 & 1.04 \\
64766 & 500628 & 2003 & 64.10 & 0.55 & 6257.00 & 58408.90 & 1.02 & 0.91 & 0.93 \\
12413 & 101539 & 2003 & 1496.50 & 0.29 & 150800.00 & 1426509.27 & 0.99 & 0.95 & 0.95 \\
43504 & 109129 & 2003 & 134.40 & 0.27 & 13446.00 & 130638.28 & 1.00 & 0.97 & 0.97 \\
39993 & 107968 & 2003 & 90.50 & 0.24 & 9223.00 & 85201.27 & 0.98 & 0.94 & 0.92 \\
18962 & 102531 & 2003 & 23.80 & 0.37 & 2360.00 & 23595.84 & 1.01 & 0.99 & 1.00 \\
42196 & 108934 & 2003 & 126.20 & 1.01 & 7022.00 & 106503.09 & 1.80 & 0.84 & 1.52 \\
21857 & 102957 & 2003 & 563.20 & 0.25 & 52736.00 & 567690.29 & 1.07 & 1.01 & 1.08 \\
43386 & 109110 & 2003 & 34.60 & 0.37 & 3450.00 & 32955.75 & 1.00 & 0.95 & 0.96 \\
59147 & 410442 & 2003 & 53.60 & 0.32 & 5546.00 & 52511.93 & 0.97 & 0.98 & 0.95 \\
43561 & 109144 & 2003 & 136.70 & 0.35 & 10239.00 & 105542.73 & 1.34 & 0.77 & 1.03 \\
64718 & 500621 & 2003 & 381.90 & 0.36 & 38193.00 & 375803.23 & 1.00 & 0.98 & 0.98 \\
48783 & 240140 & 2003 & 49.10 & 0.18 & 4663.00 & 38086.09 & 1.05 & 0.78 & 0.82 \\
22050 & 102988 & 2003 & 48.40 & 0.41 & 4833.00 & 47989.08 & 1.00 & 0.99 & 0.99 \\
54951 & 400030 & 2003 & 34.70 & 0.27 & 2558.00 & 26476.10 & 1.36 & 0.76 & 1.04 \\
42320 & 108951 & 2003 & 135.80 & 0.29 & 13461.00 & 134595.76 & 1.01 & 0.99 & 1.00 \\
21599 & 102895 & 2003 & 742.50 & 0.24 & 74327.00 & 731466.16 & 1.00 & 0.99 & 0.98 \\
44931 & 109402 & 2003 & 185.60 & 0.35 & 16420.00 & 178160.83 & 1.13 & 0.96 & 1.09 \\
60888 & 410737 & 2003 & 14.40 & 0.25 & 1438.00 & 13448.71 & 1.00 & 0.93 & 0.94 \\
44185 & 109271 & 2003 & 105.80 & 0.44 & 10029.00 & 100919.48 & 1.05 & 0.95 & 1.01 \\
40003 & 107992 & 2003 & 22.90 & 0.26 & 2316.00 & 22118.43 & 0.99 & 0.97 & 0.96 \\
7351 & 101023 & 2003 & 23520.00 & 0.30 & 2009332.00 & 21634342.88 & 1.17 & 0.92 & 1.08 \\
13128 & 101668 & 2003 & 107.90 & 0.17 & 10786.00 & 96607.60 & 1.00 & 0.90 & 0.90 \\
45048 & 109415 & 2003 & 21.00 & 0.27 & 1852.00 & 20668.76 & 1.13 & 0.98 & 1.12 \\
21535 & 102893 & 2003 & 235.10 & 0.42 & 17335.00 & 213745.04 & 1.36 & 0.91 & 1.23 \\
42419 & 108964 & 2003 & 642.60 & 0.29 & 64200.00 & 640465.49 & 1.00 & 1.00 & 1.00 \\
60906 & 410748 & 2003 & 2.50 & 0.32 & 244.00 & 2516.30 & 1.02 & 1.01 & 1.03 \\
8032 & 101071 & 2003 & 8049.10 & 0.33 & 755575.00 & 7129748.55 & 1.07 & 0.89 & 0.94 \\
60825 & 410731 & 2003 & 1151.90 & 0.31 & 106255.00 & 1181844.88 & 1.08 & 1.03 & 1.11 \\
48830 & 240144 & 2003 & 44.30 & 0.30 & 4415.00 & 35372.90 & 1.00 & 0.80 & 0.80 \\
20018 & 102663 & 2003 & 5036.00 & 0.51 & 503598.00 & 4567211.75 & 1.00 & 0.91 & 0.91 \\
19703 & 102649 & 2003 & 695.10 & 0.25 & 69506.00 & 653403.27 & 1.00 & 0.94 & 0.94 \\
39801 & 107874 & 2003 & 221.60 & 0.39 & 22092.00 & 220396.18 & 1.00 & 0.99 & 1.00 \\
44135 & 109268 & 2003 & 768.70 & 0.21 & 62471.00 & 640341.90 & 1.23 & 0.83 & 1.03 \\
48718 & 240130 & 2003 & 651.00 & 0.57 & 61574.00 & 625942.61 & 1.06 & 0.96 & 1.02 \\
42386 & 108960 & 2003 & 29.30 & 0.48 & 2340.00 & 24052.84 & 1.25 & 0.82 & 1.03 \\
42113 & 108923 & 2003 & 23.90 & 0.45 & 2177.00 & 20933.64 & 1.10 & 0.88 & 0.96 \\
7530 & 101043 & 2003 & 5278.50 & 0.33 & 535775.00 & 5107489.05 & 0.99 & 0.97 & 0.95 \\
3109 & 100409 & 2003 & 379.30 & 0.29 & 38086.00 & 358154.74 & 1.00 & 0.94 & 0.94 \\
64657 & 500617 & 2003 & 6427.00 & 0.35 & NaN & 7564407.28 & 1.00 & 1.18 & 1.00 \\
44362 & 109290 & 2003 & 207.10 & 0.25 & 20816.00 & 201808.18 & 0.99 & 0.97 & 0.97 \\
13843 & 101769 & 2003 & 1020.00 & 0.24 & 97083.00 & 1037782.64 & 1.05 & 1.02 & 1.07 \\
19881 & 102654 & 2003 & 915.00 & 0.35 & 91503.00 & 882357.16 & 1.00 & 0.96 & 0.96 \\
53990 & 363013 & 2003 & 9.80 & 0.34 & 984.00 & 9554.46 & 1.00 & 0.97 & 0.97 \\
42141 & 108929 & 2003 & 465.90 & 0.28 & 40808.00 & 451113.20 & 1.14 & 0.97 & 1.11 \\
59192 & 410444 & 2003 & 42.70 & 0.35 & 3587.00 & 36056.87 & 1.19 & 0.84 & 1.01 \\
13676 & 101757 & 2003 & 471.20 & 0.32 & 47052.00 & 457673.51 & 1.00 & 0.97 & 0.97 \\
64691 & 500619 & 2003 & 1.30 & 0.57 & 94.00 & 928.13 & 1.38 & 0.71 & 0.99 \\
59143 & 410439 & 2003 & 15.30 & 0.21 & 1404.00 & 15576.41 & 1.09 & 1.02 & 1.11 \\
1890 & 100247 & 2003 & 688.90 & 0.43 & 70779.00 & 643501.08 & 0.97 & 0.93 & 0.91 \\
60845 & 410732 & 2003 & 1285.30 & 0.30 & 121007.00 & 1320889.58 & 1.06 & 1.03 & 1.09 \\
13870 & 101781 & 2003 & 281.40 & 0.44 & 27471.00 & 285728.85 & 1.02 & 1.02 & 1.04 \\
44359 & 109289 & 2003 & 51.00 & 0.26 & 5228.00 & 48840.30 & 0.98 & 0.96 & 0.93 \\
55272 & 400076 & 2003 & 700.10 & 0.40 & 69914.00 & 680700.64 & 1.00 & 0.97 & 0.97 \\
48415 & 240080 & 2003 & 115.80 & 0.31 & 9306.00 & 94878.04 & 1.24 & 0.82 & 1.02 \\
22078 & 102989 & 2003 & 1566.60 & 0.28 & 156846.00 & 1549124.42 & 1.00 & 0.99 & 0.99 \\
24728 & 103376 & 2003 & 5179.00 & 0.24 & 502818.00 & 4914771.34 & 1.03 & 0.95 & 0.98 \\
24408 & 103319 & 2003 & 266.60 & 0.23 & 26640.00 & 266201.35 & 1.00 & 1.00 & 1.00 \\
58405 & 410154 & 2003 & 16.10 & 0.42 & 1494.00 & 15493.64 & 1.08 & 0.96 & 1.04 \\
3144 & 100411 & 2003 & 3796.20 & 0.31 & 381319.00 & 3734949.18 & 1.00 & 0.98 & 0.98 \\
19673 & 102645 & 2003 & 320.60 & 0.24 & 28843.00 & 322178.01 & 1.11 & 1.00 & 1.12 \\
19925 & 102655 & 2003 & 1331.70 & 0.33 & 133171.00 & 1187554.00 & 1.00 & 0.89 & 0.89 \\
21754 & 102949 & 2003 & 1857.10 & 0.29 & 167017.00 & 1786840.99 & 1.11 & 0.96 & 1.07 \\
1938 & 100259 & 2003 & 135.10 & 0.28 & 12924.00 & 135452.84 & 1.05 & 1.00 & 1.05 \\
39807 & 107875 & 2003 & 193.70 & 0.18 & 19429.00 & 176295.92 & 1.00 & 0.91 & 0.91 \\
3087 & 100408 & 2003 & 246.90 & 0.20 & 24724.00 & 241708.06 & 1.00 & 0.98 & 0.98 \\
8013 & 101069 & 2003 & 9824.50 & 0.28 & 920898.00 & 9445593.77 & 1.07 & 0.96 & 1.03 \\
21640 & 102937 & 2003 & 49.50 & 0.37 & 4939.00 & 44757.02 & 1.00 & 0.90 & 0.91 \\
21816 & 102952 & 2003 & 263.20 & 0.35 & 24022.00 & 261771.36 & 1.10 & 0.99 & 1.09 \\
7281 & 101018 & 2003 & 22383.80 & 0.25 & 1993552.00 & 21489179.87 & 1.12 & 0.96 & 1.08 \\
24506 & 103329 & 2003 & 454.70 & 0.42 & 42376.00 & 471083.62 & 1.07 & 1.04 & 1.11 \\
59169 & 410443 & 2003 & 18.80 & 0.32 & 1665.00 & 16928.07 & 1.13 & 0.90 & 1.02 \\
21500 & 102875 & 2003 & 12.50 & 0.27 & 1253.00 & 12476.73 & 1.00 & 1.00 & 1.00 \\
43526 & 109133 & 2003 & 46.20 & 0.62 & 4619.00 & 42923.96 & 1.00 & 0.93 & 0.93 \\
24476 & 103328 & 2003 & 510.60 & 0.47 & 49658.00 & 521074.19 & 1.03 & 1.02 & 1.05 \\
13225 & 101708 & 2003 & 349.20 & 0.30 & 39521.00 & 334365.03 & 0.88 & 0.96 & 0.85 \\
48072 & 235413 & 2003 & 146.20 & 0.57 & 14552.00 & 135846.58 & 1.00 & 0.93 & 0.93 \\
21511 & 102876 & 2003 & 12.40 & 0.28 & 1244.00 & 12433.39 & 1.00 & 1.00 & 1.00 \\
44328 & 109284 & 2003 & 8.50 & 0.48 & 851.00 & 8315.35 & 1.00 & 0.98 & 0.98 \\
60801 & 410727 & 2003 & 11.70 & 0.55 & 890.00 & 8851.44 & 1.31 & 0.76 & 0.99 \\
1827 & 100244 & 2003 & 200.90 & 0.49 & 19881.00 & 191708.88 & 1.01 & 0.95 & 0.96 \\
19983 & 102660 & 2003 & 7071.30 & 0.30 & 707163.00 & 7019257.69 & 1.00 & 0.99 & 0.99 \\
48377 & 240067 & 2003 & 298.80 & 0.26 & 29923.00 & 287912.56 & 1.00 & 0.96 & 0.96 \\
42165 & 108932 & 2003 & 930.40 & 0.32 & 82744.00 & 829545.51 & 1.12 & 0.89 & 1.00 \\
54693 & 378134 & 2003 & 71.70 & 0.25 & 7253.00 & 71297.69 & 0.99 & 0.99 & 0.98 \\
44236 & 109278 & 2003 & 25.00 & 0.37 & 2504.00 & 25211.31 & 1.00 & 1.01 & 1.01 \\
19110 & 102549 & 2003 & 417.60 & 0.55 & 33422.00 & 411836.47 & 1.25 & 0.99 & 1.23 \\
19953 & 102659 & 2003 & 4955.10 & 0.49 & 495508.00 & 4899809.74 & 1.00 & 0.99 & 0.99 \\
42098 & 108919 & 2003 & 146.10 & 0.56 & 14362.00 & 145370.79 & 1.02 & 1.00 & 1.01 \\
43641 & 109176 & 2003 & 23.90 & 0.55 & 2336.00 & 22761.82 & 1.02 & 0.95 & 0.97 \\
44959 & 109405 & 2003 & 53.60 & 0.49 & 5054.00 & 50466.89 & 1.06 & 0.94 & 1.00 \\
12393 & 101538 & 2003 & 160.80 & 0.65 & 16219.00 & 155330.35 & 0.99 & 0.97 & 0.96 \\
54645 & 377385 & 2003 & 117.00 & 0.44 & 11656.00 & 113867.50 & 1.00 & 0.97 & 0.98 \\
4287 & 100600 & 2003 & 53.90 & 0.33 & 5393.00 & 53114.54 & 1.00 & 0.99 & 0.98 \\
2918 & 100379 & 2003 & 457.90 & 0.25 & 35706.00 & 366835.63 & 1.28 & 0.80 & 1.03 \\
39959 & 107960 & 2003 & 189.70 & 0.30 & 14727.00 & 152751.72 & 1.29 & 0.81 & 1.04 \\
63007 & 500466 & 2003 & 1350.20 & 0.23 & 135599.00 & 1340847.60 & 1.00 & 0.99 & 0.99 \\
44305 & 109283 & 2003 & 1343.90 & 0.40 & 95426.00 & 1094890.45 & 1.41 & 0.81 & 1.15 \\
39975 & 107964 & 2003 & 241.60 & 0.21 & 24222.00 & 228699.40 & 1.00 & 0.95 & 0.94 \\
54761 & 378620 & 2003 & 13.80 & 0.22 & 1408.00 & 13001.21 & 0.98 & 0.94 & 0.92 \\
19843 & 102653 & 2003 & 5755.00 & 0.31 & 575491.00 & 4985717.64 & 1.00 & 0.87 & 0.87 \\
43428 & 109118 & 2003 & 433.60 & 0.30 & 43335.00 & 413417.56 & 1.00 & 0.95 & 0.95 \\
4811 & 100682 & 2003 & 67.20 & 0.14 & 5532.00 & 60079.76 & 1.21 & 0.89 & 1.09 \\
43442 & 109121 & 2003 & 3.60 & 0.62 & 360.00 & 3421.74 & 1.00 & 0.95 & 0.95 \\
39849 & 107882 & 2003 & 225.50 & 0.28 & 37776.00 & 369287.87 & 0.60 & 1.64 & 0.98 \\
54641 & 377379 & 2003 & 18.70 & 0.17 & 1940.00 & 16483.17 & 0.96 & 0.88 & 0.85 \\
7243 & 101015 & 2003 & 1946.80 & 0.60 & 196411.00 & 1781473.30 & 0.99 & 0.92 & 0.91 \\
42073 & 108918 & 2003 & 54.30 & 0.24 & 5430.00 & 54294.62 & 1.00 & 1.00 & 1.00 \\
18133 & 102404 & 2003 & 2876.00 & 0.27 & 290006.00 & 2867283.71 & 0.99 & 1.00 & 0.99 \\
18084 & 102396 & 2003 & 3303.40 & 0.37 & 330512.00 & 2877663.91 & 1.00 & 0.87 & 0.87 \\
21625 & 102901 & 2003 & 89.70 & 0.44 & 8968.00 & 89095.12 & 1.00 & 0.99 & 0.99 \\
64572 & 500610 & 2003 & 362.50 & 0.50 & 36252.00 & 361851.49 & 1.00 & 1.00 & 1.00 \\
60865 & 410733 & 2003 & 359.10 & 0.46 & 29828.00 & 348793.49 & 1.20 & 0.97 & 1.17 \\
44336 & 109286 & 2003 & 215.80 & 0.32 & 10814.00 & 175689.97 & 2.00 & 0.81 & 1.62 \\
43417 & 109112 & 2003 & 49.70 & 0.52 & 4987.00 & 47840.58 & 1.00 & 0.96 & 0.96 \\
21960 & 102981 & 2003 & 214.20 & 0.55 & 17195.00 & 188731.14 & 1.25 & 0.88 & 1.10 \\
44211 & 109275 & 2003 & 82.20 & 0.36 & 8287.00 & 81546.10 & 0.99 & 0.99 & 0.98 \\
64634 & 500614 & 2003 & 123.60 & 0.34 & 12363.00 & 123578.93 & 1.00 & 1.00 & 1.00 \\
24648 & 103373 & 2003 & 58.30 & 0.45 & 5827.00 & 56755.92 & 1.00 & 0.97 & 0.97 \\
55061 & 400050 & 2003 & 1422.10 & 0.35 & 133243.00 & 1334876.28 & 1.07 & 0.94 & 1.00 \\
42065 & 108915 & 2003 & 47.70 & 1.22 & 5499.00 & 54932.20 & 0.87 & 1.15 & 1.00 \\
24454 & 103327 & 2003 & 1227.20 & 0.28 & 111223.00 & 1186586.73 & 1.10 & 0.97 & 1.07 \\
42190 & 108933 & 2003 & 237.30 & 0.41 & 21889.00 & 220735.32 & 1.08 & 0.93 & 1.01 \\
44954 & 109404 & 2003 & 15.70 & 0.29 & 1570.00 & 14602.41 & 1.00 & 0.93 & 0.93 \\
48688 & 240121 & 2003 & 526.70 & 0.24 & 52632.00 & 526047.63 & 1.00 & 1.00 & 1.00 \\
55303 & 400081 & 2003 & 1316.00 & 0.31 & 180756.00 & 1502637.35 & 0.73 & 1.14 & 0.83 \\
43394 & 109111 & 2003 & 45.20 & 0.29 & 4487.00 & 44096.92 & 1.01 & 0.98 & 0.98 \\
42041 & 108914 & 2003 & 699.50 & 0.38 & 39320.00 & 383005.34 & 1.78 & 0.55 & 0.97 \\
39814 & 107880 & 2003 & 19.80 & 0.41 & 1972.00 & 16297.94 & 1.00 & 0.82 & 0.83 \\
64742 & 500625 & 2003 & 186.70 & 0.36 & 22774.00 & 173593.43 & 0.82 & 0.93 & 0.76 \\
42106 & 108920 & 2003 & 32.40 & 0.54 & 2976.00 & 32165.79 & 1.09 & 0.99 & 1.08 \\
3186 & 100413 & 2003 & 62.60 & 0.36 & 6255.00 & 56695.17 & 1.00 & 0.91 & 0.91 \\
64595 & 500612 & 2003 & 323.10 & 0.35 & 32309.00 & 322687.74 & 1.00 & 1.00 & 1.00 \\
21613 & 102897 & 2003 & 153.50 & 0.24 & 15306.00 & 151569.58 & 1.00 & 0.99 & 0.99 \\
49215 & 240250 & 2003 & 449.40 & 0.37 & 44931.00 & 397780.62 & 1.00 & 0.89 & 0.89 \\
54753 & 378596 & 2003 & 54.70 & 0.30 & 4452.00 & 44952.42 & 1.23 & 0.82 & 1.01 \\
3975 & 100535 & 2003 & 363.90 & 0.25 & 36392.00 & 325641.78 & 1.00 & 0.89 & 0.89 \\
39983 & 107967 & 2003 & 165.50 & 0.24 & 16627.00 & 163294.35 & 1.00 & 0.99 & 0.98 \\
4018 & 100538 & 2003 & 528.90 & 0.45 & 50435.00 & 504652.98 & 1.05 & 0.95 & 1.00 \\
54672 & 378072 & 2003 & 35.20 & 0.04 & 3504.00 & 34204.27 & 1.00 & 0.97 & 0.98 \\
48806 & 240143 & 2003 & 168.10 & 0.44 & 16690.00 & 159100.12 & 1.01 & 0.95 & 0.95 \\
22019 & 102987 & 2003 & 977.10 & 0.56 & 98668.00 & 952761.35 & 0.99 & 0.98 & 0.97 \\
12329 & 101536 & 2003 & 2554.80 & 0.26 & 255464.00 & 2530318.41 & 1.00 & 0.99 & 0.99 \\
18994 & 102540 & 2003 & 170.10 & 0.63 & 17073.00 & 142635.31 & 1.00 & 0.84 & 0.84 \\
3057 & 100401 & 2003 & 1625.00 & 0.26 & 162267.00 & 1560241.76 & 1.00 & 0.96 & 0.96 \\
24688 & 103375 & 2003 & 905.30 & 0.20 & 88765.00 & 869696.63 & 1.02 & 0.96 & 0.98 \\
58378 & 410151 & 2003 & 7.00 & 0.20 & 700.00 & 6973.23 & 1.00 & 1.00 & 1.00 \\
42159 & 108931 & 2003 & 40.50 & 0.21 & 4062.00 & 37630.71 & 1.00 & 0.93 & 0.93 \\
49150 & 240234 & 2003 & 204.90 & 0.32 & 19709.00 & 197094.09 & 1.04 & 0.96 & 1.00 \\
48350 & 240065 & 2003 & 443.40 & 0.22 & 44226.00 & 433764.60 & 1.00 & 0.98 & 0.98 \\
4765 & 100671 & 2003 & 404.60 & 0.32 & 31854.00 & 332018.06 & 1.27 & 0.82 & 1.04 \\
24765 & 103377 & 2003 & 1349.20 & 0.21 & 131104.00 & 1293538.88 & 1.03 & 0.96 & 0.99 \\
19768 & 102651 & 2003 & 3885.40 & 0.28 & 388544.00 & 3760725.74 & 1.00 & 0.97 & 0.97 \\
21710 & 102940 & 2003 & 1724.20 & 0.43 & 172099.00 & 1719311.33 & 1.00 & 1.00 & 1.00 \\
1861 & 100245 & 2003 & 536.80 & 0.32 & 49069.00 & 475837.78 & 1.09 & 0.89 & 0.97 \\
55123 & 400062 & 2003 & 53.80 & 0.46 & 4537.00 & 51211.94 & 1.19 & 0.95 & 1.13 \\
42367 & 108954 & 2003 & 56.70 & 0.27 & 5182.00 & 52364.59 & 1.09 & 0.92 & 1.01 \\
55311 & 400084 & 2003 & 24.10 & 0.28 & 1598.00 & 22063.90 & 1.51 & 0.92 & 1.38 \\
44270 & 109281 & 2003 & 143.40 & 0.31 & 14347.00 & 135872.07 & 1.00 & 0.95 & 0.95 \\
7561 & 101045 & 2003 & 12255.50 & 0.33 & 1159366.00 & 11992935.91 & 1.06 & 0.98 & 1.03 \\
43675 & 109189 & 2003 & 665.10 & 0.23 & 67373.00 & 699923.98 & 0.99 & 1.05 & 1.04 \\
21565 & 102894 & 2003 & 1001.50 & 0.45 & 84842.00 & 942997.78 & 1.18 & 0.94 & 1.11 \\
12447 & 101541 & 2003 & 439.10 & 0.26 & 41861.00 & 418454.62 & 1.05 & 0.95 & 1.00 \\
42361 & 108953 & 2003 & 177.30 & 0.46 & 17664.00 & 176560.43 & 1.00 & 1.00 & 1.00 \\
49109 & 240222 & 2003 & 516.10 & 0.20 & 49854.00 & 531197.31 & 1.04 & 1.03 & 1.07 \\
24596 & 103370 & 2003 & 228.10 & 0.59 & 19238.00 & 220561.00 & 1.19 & 0.97 & 1.15 \\
22111 & 102990 & 2003 & 4003.50 & 0.29 & 398057.00 & 3948749.15 & 1.01 & 0.99 & 0.99 \\
43481 & 109127 & 2003 & 18.60 & 0.47 & 1941.00 & 17417.82 & 0.96 & 0.94 & 0.90 \\
42206 & 108939 & 2003 & 37.80 & 0.34 & 3681.00 & 36811.45 & 1.03 & 0.97 & 1.00 \\
44984 & 109407 & 2003 & 239.40 & 0.32 & 22073.00 & 216106.83 & 1.08 & 0.90 & 0.98 \\
42379 & 108956 & 2003 & 51.10 & 0.26 & 5575.00 & 54519.88 & 0.92 & 1.07 & 0.98 \\
43602 & 109153 & 2003 & 523.10 & 0.46 & 46202.00 & 449823.71 & 1.13 & 0.86 & 0.97 \\
64668 & 500618 & 2003 & 144.20 & 0.13 & 14416.00 & 143997.39 & 1.00 & 1.00 & 1.00 \\
42130 & 108925 & 2003 & 459.40 & 0.23 & 45847.00 & 437346.42 & 1.00 & 0.95 & 0.95 \\
48729 & 240134 & 2003 & 236.80 & 0.33 & 24353.00 & 240011.99 & 0.97 & 1.01 & 0.99 \\
43359 & 109100 & 2003 & 6.70 & 0.27 & 447.00 & 5578.39 & 1.50 & 0.83 & 1.25 \\
39923 & 107938 & 2003 & 187.50 & 0.32 & 18764.00 & 172854.89 & 1.00 & 0.92 & 0.92 \\
1770 & 100228 & 2003 & 117.90 & 0.25 & 11623.00 & 115534.84 & 1.01 & 0.98 & 0.99 \\
53993 & 363121 & 2003 & 1042.20 & 0.32 & 94050.00 & 1070023.32 & 1.11 & 1.03 & 1.14 \\
24374 & 103318 & 2003 & 1654.90 & 0.25 & 148578.00 & 1632113.41 & 1.11 & 0.99 & 1.10 \\
43632 & 109175 & 2003 & 11.80 & 0.30 & 1179.00 & 11334.81 & 1.00 & 0.96 & 0.96 \\
12362 & 101537 & 2003 & 475.80 & 0.22 & 47512.00 & 451331.72 & 1.00 & 0.95 & 0.95 \\
41949 & 108874 & 2003 & 41.90 & 0.30 & 4238.00 & 41204.38 & 0.99 & 0.98 & 0.97 \\
19082 & 102548 & 2003 & 716.90 & 0.30 & 61962.00 & 707127.25 & 1.16 & 0.99 & 1.14 \\
21912 & 102979 & 2003 & 70.20 & 0.27 & 5967.00 & 63559.05 & 1.18 & 0.91 & 1.07 \\
24576 & 103369 & 2003 & 846.20 & 0.39 & 72439.00 & 803859.57 & 1.17 & 0.95 & 1.11 \\
43475 & 109125 & 2003 & 149.50 & 0.40 & 13989.00 & 136279.94 & 1.07 & 0.91 & 0.97 \\
64549 & 500609 & 2003 & 229.40 & 0.31 & 22943.00 & 229055.50 & 1.00 & 1.00 & 1.00 \\
44202 & 109274 & 2003 & 45.40 & 0.19 & 4080.00 & 41372.47 & 1.11 & 0.91 & 1.01 \\
42373 & 108955 & 2003 & 52.10 & 0.53 & 5146.00 & 51384.89 & 1.01 & 0.99 & 1.00 \\
39791 & 107872 & 2003 & 54.90 & 0.27 & 5026.00 & 55597.40 & 1.09 & 1.01 & 1.11 \\
39897 & 107928 & 2003 & 3277.20 & 0.17 & 327423.00 & 3217058.52 & 1.00 & 0.98 & 0.98 \\
3921 & 100514 & 2003 & 49.20 & 0.22 & 4359.00 & 42990.69 & 1.13 & 0.87 & 0.99 \\
18183 & 102414 & 2003 & 8229.20 & 0.29 & 720619.00 & 6891705.13 & 1.14 & 0.84 & 0.96 \\
42454 & 108968 & 2003 & 23.10 & 0.17 & 2299.00 & 22759.32 & 1.00 & 0.99 & 0.99 \\
40047 & 108018 & 2003 & 371.60 & 0.35 & 37160.00 & 354896.88 & 1.00 & 0.96 & 0.96 \\
20052 & 102664 & 2003 & 3594.80 & 0.36 & 359480.00 & 3312346.03 & 1.00 & 0.92 & 0.92 \\
40024 & 108009 & 2003 & 789.60 & 0.21 & 74109.00 & 798717.78 & 1.07 & 1.01 & 1.08 \\
24346 & 103315 & 2003 & 138.20 & 0.43 & 13850.00 & 117792.29 & 1.00 & 0.85 & 0.85 \\
44180 & 109270 & 2003 & 75.60 & 0.16 & 7366.00 & 76908.93 & 1.03 & 1.02 & 1.04 \\
41977 & 108886 & 2003 & 76.20 & 0.40 & 7587.00 & 74956.67 & 1.00 & 0.98 & 0.99 \\
42336 & 108952 & 2003 & 307.70 & 0.42 & 28469.00 & 284663.07 & 1.08 & 0.93 & 1.00 \\
44365 & 109291 & 2003 & 48.90 & 0.36 & 3727.00 & 43143.06 & 1.31 & 0.88 & 1.16 \\
44369 & 109295 & 2003 & 162.00 & 0.49 & 16008.00 & 147830.26 & 1.01 & 0.91 & 0.92 \\
39775 & 107870 & 2003 & 323.80 & 0.29 & 31807.00 & 304729.20 & 1.02 & 0.94 & 0.96 \\
48095 & 235952 & 2003 & 50.40 & 0.23 & 5087.00 & 43960.86 & 0.99 & 0.87 & 0.86 \\
43698 & 109190 & 2003 & 39.00 & 0.63 & 3746.00 & 37488.26 & 1.04 & 0.96 & 1.00 \\
48679 & 240118 & 2003 & 59.70 & 0.46 & 5083.00 & 53716.91 & 1.17 & 0.90 & 1.06 \\
59215 & 410445 & 2003 & 56.40 & 0.27 & 4977.00 & 49610.45 & 1.13 & 0.88 & 1.00 \\
21680 & 102939 & 2003 & 6310.30 & 0.29 & 629430.00 & 6244347.57 & 1.00 & 0.99 & 0.99 \\
49242 & 240254 & 2003 & 517.00 & 0.39 & 51648.00 & 500913.84 & 1.00 & 0.97 & 0.97 \\
43370 & 109102 & 2003 & 13.60 & 0.28 & 1368.00 & 13679.50 & 0.99 & 1.01 & 1.00 \\
43373 & 109104 & 2003 & 37.40 & 0.42 & 3751.00 & 36483.58 & 1.00 & 0.98 & 0.97 \\
48713 & 240123 & 2003 & 1.80 & 0.27 & 183.00 & 1775.12 & 0.98 & 0.99 & 0.97 \\
42262 & 108946 & 2003 & 24.50 & 0.23 & 2462.00 & 20088.65 & 1.00 & 0.82 & 0.82 \\
55099 & 400061 & 2003 & 589.50 & 0.52 & 45201.00 & 428802.25 & 1.30 & 0.73 & 0.95 \\
44157 & 109269 & 2003 & 47.20 & 0.56 & 4243.00 & 44930.61 & 1.11 & 0.95 & 1.06 \\
18930 & 102528 & 2003 & 87.60 & 0.32 & 8756.00 & 87088.40 & 1.00 & 0.99 & 0.99 \\
41986 & 108889 & 2003 & 4.20 & 0.33 & 429.00 & 4179.40 & 0.98 & 1.00 & 0.97 \\
13159 & 101681 & 2003 & 1003.60 & 0.33 & 99379.00 & 1005508.02 & 1.01 & 1.00 & 1.01 \\
42118 & 108924 & 2003 & 191.20 & 0.37 & 18434.00 & 184290.18 & 1.04 & 0.96 & 1.00 \\
42237 & 108944 & 2003 & 137.80 & 0.26 & 10049.00 & 96337.20 & 1.37 & 0.70 & 0.96 \\
61187 & 410904 & 2003 & 222.90 & 0.37 & 19123.00 & 202832.04 & 1.17 & 0.91 & 1.06 \\
43378 & 109108 & 2003 & 39.30 & 0.34 & 3927.00 & 37666.28 & 1.00 & 0.96 & 0.96 \\
44089 & 109265 & 2003 & 47.70 & 0.33 & 4480.00 & 36442.48 & 1.06 & 0.76 & 0.81 \\
18914 & 102527 & 2003 & 98.60 & 0.20 & 10183.00 & 101340.93 & 0.97 & 1.03 & 1.00 \\
44259 & 109279 & 2003 & 42.20 & 0.35 & 4217.00 & 41818.39 & 1.00 & 0.99 & 0.99 \\
2997 & 100395 & 2003 & 806.70 & 0.28 & 80290.00 & 779956.70 & 1.00 & 0.97 & 0.97 \\
45004 & 109413 & 2003 & 24.20 & 0.26 & 1992.00 & 21849.34 & 1.21 & 0.90 & 1.10 \\
24619 & 103372 & 2003 & 751.00 & 0.23 & 67006.00 & 717008.75 & 1.12 & 0.95 & 1.07 \\
42137 & 108926 & 2003 & 84.20 & 0.19 & 8355.00 & 81858.89 & 1.01 & 0.97 & 0.98 \\
43575 & 109147 & 2003 & 58.30 & 1.01 & 5597.00 & 55200.67 & 1.04 & 0.95 & 0.99 \\
58249 & 410133 & 2003 & 23.60 & 0.39 & 2360.00 & 23118.62 & 1.00 & 0.98 & 0.98 \\
40075 & 108021 & 2003 & 1027.40 & 0.37 & 102737.00 & 986949.78 & 1.00 & 0.96 & 0.96 \\
12311 & 101534 & 2003 & 1844.40 & 0.32 & 184361.00 & 1775567.78 & 1.00 & 0.96 & 0.96 \\
42436 & 108966 & 2003 & 273.20 & 0.20 & 27366.00 & 271475.18 & 1.00 & 0.99 & 0.99 \\
18046 & 102387 & 2003 & 499.90 & 0.62 & 49996.00 & 461517.18 & 1.00 & 0.92 & 0.92 \\
44965 & 109406 & 2003 & 93.80 & 0.39 & 8255.00 & 80363.62 & 1.14 & 0.86 & 0.97 \\
22145 & 102993 & 2003 & 5891.20 & 0.31 & 510422.00 & 4747865.68 & 1.15 & 0.81 & 0.93 \\
44914 & 109401 & 2003 & 14.40 & 0.28 & 1108.00 & 12298.21 & 1.30 & 0.85 & 1.11 \\
54987 & 400040 & 2003 & 130.50 & 0.24 & 12196.00 & 132992.72 & 1.07 & 1.02 & 1.09 \\
40039 & 108013 & 2003 & 74.50 & 0.25 & 7444.00 & 72516.32 & 1.00 & 0.97 & 0.97 \\
64529 & 500607 & 2003 & 186.50 & 0.35 & 20059.00 & 205063.08 & 0.93 & 1.10 & 1.02 \\
44191 & 109273 & 2003 & 24.90 & 0.67 & 2692.00 & 20012.22 & 0.92 & 0.80 & 0.74 \\
54666 & 377933 & 2003 & 9.20 & -0.01 & 1459.00 & 14248.97 & 0.63 & 1.55 & 0.98 \\
46744 & 200294 & 2004 & 122.40 & 0.10 & 10085.00 & 101739.57 & 1.21 & 0.83 & 1.01 \\
51725 & 240546 & 2004 & 19.70 & 0.03 & 1977.00 & 19580.99 & 1.00 & 0.99 & 0.99 \\
22798 & 103065 & 2004 & 211.50 & 0.17 & 21134.00 & 193240.50 & 1.00 & 0.91 & 0.91 \\
19306 & 102588 & 2004 & 325.40 & 0.13 & 31605.00 & 314461.49 & 1.03 & 0.97 & 0.99 \\
45956 & 200173 & 2004 & 47.30 & 0.23 & 4389.00 & 48732.40 & 1.08 & 1.03 & 1.11 \\
47802 & 221485 & 2004 & 410.90 & 0.09 & 41120.00 & 407331.24 & 1.00 & 0.99 & 0.99 \\
35428 & 106360 & 2004 & 842.40 & 0.27 & 84141.00 & 801630.21 & 1.00 & 0.95 & 0.95 \\
59304 & 410463 & 2004 & 413.80 & 0.12 & 36674.00 & 318569.71 & 1.13 & 0.77 & 0.87 \\
42766 & 109016 & 2004 & 1245.20 & 0.14 & 127160.00 & 1138080.25 & 0.98 & 0.91 & 0.89 \\
29768 & 105645 & 2004 & 8548.10 & 0.22 & 776403.00 & 7563514.09 & 1.10 & 0.88 & 0.97 \\
46525 & 200251 & 2004 & 47.80 & 0.21 & 3527.00 & 30953.29 & 1.36 & 0.65 & 0.88 \\
45950 & 200172 & 2004 & 24.70 & 0.16 & 2290.00 & 23994.77 & 1.08 & 0.97 & 1.05 \\
49858 & 240368 & 2004 & 1010.00 & 0.07 & 100752.00 & 1007526.33 & 1.00 & 1.00 & 1.00 \\
53516 & 351459 & 2004 & 887.10 & 0.10 & 88515.00 & 757846.56 & 1.00 & 0.85 & 0.86 \\
46711 & 200291 & 2004 & 8.30 & 0.08 & 840.00 & 8193.71 & 0.99 & 0.99 & 0.98 \\
54481 & 372363 & 2004 & 13.10 & 0.09 & 1267.00 & 12820.98 & 1.03 & 0.98 & 1.01 \\
10201 & 101268 & 2004 & 404.20 & 0.07 & 42958.00 & 418202.43 & 0.94 & 1.03 & 0.97 \\
37290 & 106726 & 2004 & 1820.10 & 0.10 & 178561.00 & 1842404.28 & 1.02 & 1.01 & 1.03 \\
42160 & 108931 & 2004 & 37.30 & 0.10 & 4186.00 & 35032.12 & 0.89 & 0.94 & 0.84 \\
48184 & 240040 & 2004 & 394.10 & 0.23 & 39042.00 & 349333.09 & 1.01 & 0.89 & 0.89 \\
29852 & 105654 & 2004 & 61.30 & 0.07 & 5995.00 & 62285.10 & 1.02 & 1.02 & 1.04 \\
2998 & 100395 & 2004 & 832.80 & 0.11 & 83069.00 & 784803.13 & 1.00 & 0.94 & 0.94 \\
50680 & 240439 & 2004 & 28.80 & 0.02 & 2863.00 & 25125.51 & 1.01 & 0.87 & 0.88 \\
48689 & 240121 & 2004 & 402.90 & 0.03 & 47817.00 & 426389.05 & 0.84 & 1.06 & 0.89 \\
10804 & 101331 & 2004 & 77.60 & 0.14 & 7602.00 & 77545.56 & 1.02 & 1.00 & 1.02 \\
52718 & 307603 & 2004 & 57.60 & 0.16 & 6143.00 & 55975.96 & 0.94 & 0.97 & 0.91 \\
44237 & 109278 & 2004 & 31.10 & 0.14 & 3117.00 & 30773.99 & 1.00 & 0.99 & 0.99 \\
42362 & 108953 & 2004 & 159.60 & 0.09 & 22492.00 & 162700.59 & 0.71 & 1.02 & 0.72 \\
20982 & 102814 & 2004 & 223.90 & 0.20 & 22403.00 & 214695.27 & 1.00 & 0.96 & 0.96 \\
54646 & 377385 & 2004 & 259.80 & 0.19 & 26059.00 & 238675.83 & 1.00 & 0.92 & 0.92 \\
22967 & 103099 & 2004 & 100.00 & 0.10 & 11201.00 & 88689.10 & 0.89 & 0.89 & 0.79 \\
49868 & 240369 & 2004 & 5.60 & 0.07 & 522.00 & 4704.68 & 1.07 & 0.84 & 0.90 \\
63847 & 500560 & 2004 & 20.60 & 0.06 & 2069.00 & 20581.20 & 1.00 & 1.00 & 0.99 \\
37131 & 106692 & 2004 & 2093.20 & 0.10 & 188170.00 & 1860351.02 & 1.11 & 0.89 & 0.99 \\
13959 & 101789 & 2004 & 110.20 & 0.10 & 12043.00 & 112305.19 & 0.92 & 1.02 & 0.93 \\
63730 & 500549 & 2004 & 1.70 & 0.06 & 149.00 & 1251.36 & 1.14 & 0.74 & 0.84 \\
54456 & 367992 & 2004 & 108.50 & 0.11 & 12989.00 & 123498.88 & 0.84 & 1.14 & 0.95 \\
74714 & 601155 & 2004 & 92.50 & 0.33 & 8145.00 & 87017.98 & 1.14 & 0.94 & 1.07 \\
7467 & 101040 & 2004 & 4951.20 & 0.12 & 447804.00 & 4677779.02 & 1.11 & 0.94 & 1.04 \\
15877 & 102050 & 2004 & 391.80 & 0.15 & 38853.00 & 327172.42 & 1.01 & 0.84 & 0.84 \\
21070 & 102827 & 2004 & 247.70 & 0.15 & 24760.00 & 221620.16 & 1.00 & 0.89 & 0.90 \\
48938 & 240154 & 2004 & 45.00 & 0.23 & 4000.00 & 44140.83 & 1.12 & 0.98 & 1.10 \\
55529 & 400116 & 2004 & 59.00 & 0.09 & 6033.00 & 58225.78 & 0.98 & 0.99 & 0.97 \\
7244 & 101015 & 2004 & 2283.10 & 0.14 & 234608.00 & 2091022.98 & 0.97 & 0.92 & 0.89 \\
27668 & 105309 & 2004 & 1643.10 & 0.14 & 153692.00 & 1494266.42 & 1.07 & 0.91 & 0.97 \\
22952 & 103090 & 2004 & 787.00 & 0.10 & 74721.00 & 686969.27 & 1.05 & 0.87 & 0.92 \\
63868 & 500562 & 2004 & 25.00 & 0.13 & 2503.00 & 24909.35 & 1.00 & 1.00 & 1.00 \\
8972 & 101107 & 2004 & 883.30 & 0.11 & 81996.00 & 830115.36 & 1.08 & 0.94 & 1.01 \\
42788 & 109017 & 2004 & 48.20 & 0.11 & 7422.00 & 71795.71 & 0.65 & 1.49 & 0.97 \\
16881 & 102213 & 2004 & 704.90 & 0.06 & 66413.00 & 662529.63 & 1.06 & 0.94 & 1.00 \\
42751 & 109015 & 2004 & 184.70 & 0.09 & 18384.00 & 178285.49 & 1.00 & 0.97 & 0.97 \\
37372 & 106740 & 2004 & 116.40 & 0.12 & 11262.00 & 116112.30 & 1.03 & 1.00 & 1.03 \\
45962 & 200174 & 2004 & 325.20 & 0.09 & 27166.00 & 275583.70 & 1.20 & 0.85 & 1.01 \\
30920 & 105836 & 2004 & 196.30 & 0.13 & 18827.00 & 182717.41 & 1.04 & 0.93 & 0.97 \\
29869 & 105655 & 2004 & 1251.00 & 0.37 & 108074.00 & 1142098.84 & 1.16 & 0.91 & 1.06 \\
46757 & 200295 & 2004 & 444.40 & 0.15 & 41083.00 & 412242.66 & 1.08 & 0.93 & 1.00 \\
35850 & 106418 & 2004 & 1561.00 & 0.07 & 151737.00 & 1511670.40 & 1.03 & 0.97 & 1.00 \\
16155 & 102087 & 2004 & 539.30 & 0.10 & 64105.00 & 518038.00 & 0.84 & 0.96 & 0.81 \\
54447 & 367985 & 2004 & 237.60 & 0.20 & 20836.00 & 221151.17 & 1.14 & 0.93 & 1.06 \\
21566 & 102894 & 2004 & 1116.80 & 0.09 & 111196.00 & 1064323.31 & 1.00 & 0.95 & 0.96 \\
63734 & 500550 & 2004 & 55943.70 & 0.11 & 4568695.00 & 49521046.07 & 1.22 & 0.89 & 1.08 \\
11461 & 101414 & 2004 & 24.30 & 0.17 & 2562.00 & 21807.46 & 0.95 & 0.90 & 0.85 \\
2544 & 100343 & 2004 & 707.60 & 0.12 & 89019.00 & 675229.40 & 0.79 & 0.95 & 0.76 \\
35503 & 106367 & 2004 & 106.60 & 0.10 & 10576.00 & 105269.55 & 1.01 & 0.99 & 1.00 \\
34664 & 106268 & 2004 & 232.20 & 0.15 & 23183.00 & 221687.29 & 1.00 & 0.95 & 0.96 \\
29268 & 105574 & 2004 & 270.00 & 0.22 & 23679.00 & 261792.52 & 1.14 & 0.97 & 1.11 \\
18644 & 102493 & 2004 & 1830.60 & 0.07 & 184247.00 & 1842541.20 & 0.99 & 1.01 & 1.00 \\
27904 & 105343 & 2004 & 109.40 & 0.07 & 10940.00 & 101146.99 & 1.00 & 0.92 & 0.92 \\
46548 & 200252 & 2004 & 52.80 & 0.11 & 4712.00 & 49547.32 & 1.12 & 0.94 & 1.05 \\
29365 & 105589 & 2004 & 274.50 & 0.08 & 27474.00 & 272243.42 & 1.00 & 0.99 & 0.99 \\
43981 & 109250 & 2004 & 164.40 & 0.09 & 15366.00 & 152643.03 & 1.07 & 0.93 & 0.99 \\
46464 & 200246 & 2004 & 154.70 & 0.10 & 15521.00 & 151613.47 & 1.00 & 0.98 & 0.98 \\
30766 & 105794 & 2004 & 54.70 & 0.20 & 4708.00 & 49467.78 & 1.16 & 0.90 & 1.05 \\
50752 & 240447 & 2004 & 16.30 & 0.21 & 1593.00 & 15296.93 & 1.02 & 0.94 & 0.96 \\
7707 & 101055 & 2004 & 25180.30 & 0.12 & 2481692.00 & 24519845.20 & 1.01 & 0.97 & 0.99 \\
19111 & 102549 & 2004 & 671.00 & 0.12 & 59652.00 & 606518.46 & 1.12 & 0.90 & 1.02 \\
27913 & 105346 & 2004 & 1634.00 & 0.15 & 164219.00 & 1600051.21 & 1.00 & 0.98 & 0.97 \\
63858 & 500561 & 2004 & 25.00 & 0.16 & 2505.00 & 24925.70 & 1.00 & 1.00 & 1.00 \\
16972 & 102224 & 2004 & 9531.80 & 0.20 & 878914.00 & 9376366.88 & 1.08 & 0.98 & 1.07 \\
35937 & 106441 & 2004 & 918.80 & 0.12 & 85083.00 & 730212.56 & 1.08 & 0.79 & 0.86 \\
46721 & 200293 & 2004 & 68.60 & 0.16 & 5613.00 & 52014.93 & 1.22 & 0.76 & 0.93 \\
37317 & 106729 & 2004 & 1653.00 & 0.12 & 165686.00 & 1642779.13 & 1.00 & 0.99 & 0.99 \\
50660 & 240437 & 2004 & 1218.30 & 0.03 & 118030.00 & 1118563.36 & 1.03 & 0.92 & 0.95 \\
52731 & 307849 & 2004 & 111.00 & 0.14 & 10778.00 & 90211.16 & 1.03 & 0.81 & 0.84 \\
51071 & 240481 & 2004 & 60.80 & 0.26 & 6145.00 & 54889.00 & 0.99 & 0.90 & 0.89 \\
37363 & 106737 & 2004 & 217.40 & -0.02 & 21401.00 & 214024.81 & 1.02 & 0.98 & 1.00 \\
10234 & 101275 & 2004 & 1041.50 & -0.01 & 117967.00 & 1140446.17 & 0.88 & 1.10 & 0.97 \\
6915 & 100968 & 2004 & 479.10 & 0.12 & 49862.00 & 401855.70 & 0.96 & 0.84 & 0.81 \\
63779 & 500554 & 2004 & 345.80 & 0.10 & 29914.00 & 313003.45 & 1.16 & 0.91 & 1.05 \\
51733 & 240548 & 2004 & 3.70 & 0.03 & 213.00 & 2210.50 & 1.74 & 0.60 & 1.04 \\
34593 & 106257 & 2004 & 259.60 & 0.02 & 25905.00 & 230862.58 & 1.00 & 0.89 & 0.89 \\
22833 & 103067 & 2004 & 64.60 & 0.13 & 6429.00 & 59393.41 & 1.00 & 0.92 & 0.92 \\
10834 & 101334 & 2004 & 543.70 & 0.10 & 49053.00 & 500662.09 & 1.11 & 0.92 & 1.02 \\
74899 & 601190 & 2004 & 31.00 & 0.12 & 3985.00 & 41117.37 & 0.78 & 1.33 & 1.03 \\
5472 & 100764 & 2004 & 488.70 & 0.21 & 51341.00 & 475742.70 & 0.95 & 0.97 & 0.93 \\
34635 & 106262 & 2004 & 347.80 & 0.04 & 34828.00 & 336166.75 & 1.00 & 0.97 & 0.97 \\
42142 & 108929 & 2004 & 383.40 & 0.10 & 35928.00 & 368742.26 & 1.07 & 0.96 & 1.03 \\
29826 & 105652 & 2004 & 364.40 & 0.07 & 32253.00 & 336727.73 & 1.13 & 0.92 & 1.04 \\
11404 & 101400 & 2004 & 277.40 & 0.21 & 32136.00 & 269549.40 & 0.86 & 0.97 & 0.84 \\
56558 & 400226 & 2004 & 169.60 & 0.03 & 17019.00 & 169612.50 & 1.00 & 1.00 & 1.00 \\
22865 & 103073 & 2004 & 479.40 & 0.07 & 54206.00 & 510971.73 & 0.88 & 1.07 & 0.94 \\
55100 & 400061 & 2004 & 1895.80 & 0.23 & 158745.00 & 1341068.79 & 1.19 & 0.71 & 0.84 \\
44181 & 109270 & 2004 & 69.00 & 0.05 & 6607.00 & 69014.54 & 1.04 & 1.00 & 1.04 \\
6773 & 100953 & 2004 & 301.90 & 0.39 & 30066.00 & 295811.01 & 1.00 & 0.98 & 0.98 \\
528 & 100072 & 2004 & 6742.30 & 0.06 & 754898.00 & 6859812.65 & 0.89 & 1.02 & 0.91 \\
49898 & 240371 & 2004 & 9.70 & 0.22 & 997.00 & 7941.22 & 0.97 & 0.82 & 0.80 \\
51703 & 240543 & 2004 & 2.10 & 0.10 & 209.00 & 1820.90 & 1.00 & 0.87 & 0.87 \\
30803 & 105803 & 2004 & 8766.40 & 0.16 & 854941.00 & 7358631.00 & 1.03 & 0.84 & 0.86 \\
49952 & 240377 & 2004 & 30.00 & 0.07 & 2838.00 & 28656.13 & 1.06 & 0.96 & 1.01 \\
63807 & 500556 & 2004 & 172.20 & 0.16 & 14426.00 & 162043.45 & 1.19 & 0.94 & 1.12 \\
46036 & 200179 & 2004 & 50.30 & 0.18 & 5035.00 & 46623.70 & 1.00 & 0.93 & 0.93 \\
27775 & 105322 & 2004 & 37.20 & 0.12 & 3365.00 & 33420.52 & 1.11 & 0.90 & 0.99 \\
48932 & 240153 & 2004 & 117.80 & 0.15 & 10274.00 & 108537.98 & 1.15 & 0.92 & 1.06 \\
37160 & 106701 & 2004 & 49.50 & 0.11 & 4678.00 & 47452.09 & 1.06 & 0.96 & 1.01 \\
56553 & 400225 & 2004 & 41.70 & 0.04 & 4168.00 & 41667.92 & 1.00 & 1.00 & 1.00 \\
51060 & 240480 & 2004 & 25.20 & 0.12 & 2525.00 & 24659.42 & 1.00 & 0.98 & 0.98 \\
16174 & 102089 & 2004 & 422.40 & 0.15 & 38008.00 & 311357.51 & 1.11 & 0.74 & 0.82 \\
56548 & 400224 & 2004 & 27.30 & 0.05 & 2725.00 & 27246.02 & 1.00 & 1.00 & 1.00 \\
37167 & 106706 & 2004 & 106.20 & 0.21 & 8918.00 & 104100.55 & 1.19 & 0.98 & 1.17 \\
12886 & 101603 & 2004 & 2182.90 & 0.09 & 216174.00 & 2162021.27 & 1.01 & 0.99 & 1.00 \\
5828 & 100804 & 2004 & 4435.70 & 0.06 & 443026.00 & 3900470.69 & 1.00 & 0.88 & 0.88 \\
3145 & 100411 & 2004 & 3730.50 & 0.13 & 377181.00 & 3425006.45 & 0.99 & 0.92 & 0.91 \\
62483 & 500394 & 2004 & 32.60 & 0.04 & 3253.00 & 32518.79 & 1.00 & 1.00 & 1.00 \\
13160 & 101681 & 2004 & 624.90 & 0.13 & 60768.00 & 572243.39 & 1.03 & 0.92 & 0.94 \\
48680 & 240118 & 2004 & 96.70 & 0.39 & 7256.00 & 80021.87 & 1.33 & 0.83 & 1.10 \\
47771 & 221210 & 2004 & 134.50 & 0.08 & 12740.00 & 132207.13 & 1.06 & 0.98 & 1.04 \\
41386 & 108742 & 2004 & 38.40 & 0.06 & 3496.00 & 34390.80 & 1.10 & 0.90 & 0.98 \\
42741 & 109010 & 2004 & 15.40 & 0.01 & 1538.00 & 14634.89 & 1.00 & 0.95 & 0.95 \\
49929 & 240376 & 2004 & 48.00 & 0.11 & 4797.00 & 47964.82 & 1.00 & 1.00 & 1.00 \\
41375 & 108736 & 2004 & 118.80 & 0.12 & 10763.00 & 110350.77 & 1.10 & 0.93 & 1.03 \\
37181 & 106707 & 2004 & 246.70 & 0.16 & 22766.00 & 235933.35 & 1.08 & 0.96 & 1.04 \\
55179 & 400066 & 2004 & 164.60 & 0.14 & 16523.00 & 164884.43 & 1.00 & 1.00 & 1.00 \\
50694 & 240440 & 2004 & 657.20 & 0.10 & 58215.00 & 618701.33 & 1.13 & 0.94 & 1.06 \\
35901 & 106424 & 2004 & 484.10 & 0.18 & 48600.00 & 461018.07 & 1.00 & 0.95 & 0.95 \\
37207 & 106708 & 2004 & 472.90 & 0.10 & 48393.00 & 484607.98 & 0.98 & 1.02 & 1.00 \\
42718 & 109009 & 2004 & 244.90 & 0.19 & 21815.00 & 224878.74 & 1.12 & 0.92 & 1.03 \\
4442 & 100625 & 2004 & 1959.60 & 0.13 & 188546.00 & 1906596.89 & 1.04 & 0.97 & 1.01 \\
27799 & 105331 & 2004 & 17.50 & 0.12 & 1711.00 & 15505.73 & 1.02 & 0.89 & 0.91 \\
10645 & 101302 & 2004 & 2332.90 & 0.10 & 208959.00 & 2146282.56 & 1.12 & 0.92 & 1.03 \\
51691 & 240542 & 2004 & 206.50 & 0.06 & 13328.00 & 119759.29 & 1.55 & 0.58 & 0.90 \\
37220 & 106710 & 2004 & 54.80 & 0.15 & 7183.00 & 70929.43 & 0.76 & 1.29 & 0.99 \\
63774 & 500553 & 2004 & 175.00 & -0.03 & 16254.00 & 158670.27 & 1.08 & 0.91 & 0.98 \\
30831 & 105804 & 2004 & 754.60 & 0.05 & 72473.00 & 693725.87 & 1.04 & 0.92 & 0.96 \\
6088 & 100822 & 2004 & 14.10 & 0.17 & 1365.00 & 13938.21 & 1.03 & 0.99 & 1.02 \\
13940 & 101788 & 2004 & 477.60 & 0.10 & 61873.00 & 565900.52 & 0.77 & 1.18 & 0.91 \\
42138 & 108926 & 2004 & 76.80 & 0.12 & 7311.00 & 73896.61 & 1.05 & 0.96 & 1.01 \\
4148 & 100561 & 2004 & 81.90 & 0.11 & 7712.00 & 72894.85 & 1.06 & 0.89 & 0.95 \\
34624 & 106261 & 2004 & 740.40 & 0.06 & 74024.00 & 726780.05 & 1.00 & 0.98 & 0.98 \\
27828 & 105332 & 2004 & 71.70 & 0.17 & 7169.00 & 67569.30 & 1.00 & 0.94 & 0.94 \\
21028 & 102824 & 2004 & 47.50 & 0.14 & 4879.00 & 47179.65 & 0.97 & 0.99 & 0.97 \\
35477 & 106363 & 2004 & 152.20 & 0.13 & 14417.00 & 147668.21 & 1.06 & 0.97 & 1.02 \\
19027 & 102544 & 2004 & 897.10 & 0.12 & 82797.00 & 859926.91 & 1.08 & 0.96 & 1.04 \\
46073 & 200183 & 2004 & 21.50 & 0.12 & 2146.00 & 18707.12 & 1.00 & 0.87 & 0.87 \\
11376 & 101399 & 2004 & 69.90 & 0.11 & 7279.00 & 68001.55 & 0.96 & 0.97 & 0.93 \\
42712 & 109008 & 2004 & 14.20 & 0.04 & 1350.00 & 12924.67 & 1.05 & 0.91 & 0.96 \\
51055 & 240478 & 2004 & 2.50 & 0.14 & 242.00 & 2419.00 & 1.03 & 0.97 & 1.00 \\
30788 & 105798 & 2004 & 1475.40 & 0.15 & 155552.00 & 1523569.00 & 0.95 & 1.03 & 0.98 \\
29797 & 105647 & 2004 & 544.30 & 0.13 & 51950.00 & 520203.35 & 1.05 & 0.96 & 1.00 \\
5450 & 100763 & 2004 & 1671.10 & 0.12 & 188866.00 & 1710073.05 & 0.88 & 1.02 & 0.91 \\
52772 & 320640 & 2004 & 2.90 & 0.17 & 292.00 & 2593.50 & 0.99 & 0.89 & 0.89 \\
48648 & 240116 & 2004 & 193.80 & 0.16 & 15655.00 & 147497.84 & 1.24 & 0.76 & 0.94 \\
8433 & 101086 & 2004 & 782.20 & 0.17 & 78407.00 & 664306.36 & 1.00 & 0.85 & 0.85 \\
35869 & 106420 & 2004 & 120.60 & 0.10 & 10781.00 & 106675.21 & 1.12 & 0.88 & 0.99 \\
35485 & 106364 & 2004 & 16.60 & 0.26 & 1630.00 & 14060.89 & 1.02 & 0.85 & 0.86 \\
41365 & 108733 & 2004 & 25.50 & 0.04 & 2511.00 & 25618.67 & 1.02 & 1.00 & 1.02 \\
63008 & 500466 & 2004 & 1409.40 & 0.08 & 141007.00 & 1370043.38 & 1.00 & 0.97 & 0.97 \\
27714 & 105317 & 2004 & 305.60 & 0.14 & 38624.00 & 404234.89 & 0.79 & 1.32 & 1.05 \\
46454 & 200245 & 2004 & 24.80 & 0.12 & 2181.00 & 23700.81 & 1.14 & 0.96 & 1.09 \\
11438 & 101402 & 2004 & 115.50 & 0.19 & 11581.00 & 103340.37 & 1.00 & 0.89 & 0.89 \\
53103 & 338393 & 2004 & 8.00 & 0.07 & 779.00 & 7562.89 & 1.03 & 0.95 & 0.97 \\
62503 & 500395 & 2004 & 43.60 & 0.04 & 4355.00 & 43349.33 & 1.00 & 0.99 & 1.00 \\
34546 & 106251 & 2004 & 147.80 & 0.07 & 14788.00 & 134728.68 & 1.00 & 0.91 & 0.91 \\
63828 & 500559 & 2004 & 47.30 & 0.15 & 4435.00 & 44916.98 & 1.07 & 0.95 & 1.01 \\
30887 & 105806 & 2004 & 134.60 & 0.16 & 12593.00 & 127299.10 & 1.07 & 0.95 & 1.01 \\
74897 & 601189 & 2004 & 21.40 & 0.10 & 1871.00 & 20639.64 & 1.14 & 0.96 & 1.10 \\
4457 & 100633 & 2004 & 290.20 & 0.08 & 28923.00 & 286230.94 & 1.00 & 0.99 & 0.99 \\
51035 & 240477 & 2004 & 2.30 & 0.11 & 210.00 & 2101.20 & 1.10 & 0.91 & 1.00 \\
54667 & 377933 & 2004 & 24.00 & 0.16 & 3002.00 & 30028.91 & 0.80 & 1.25 & 1.00 \\
41427 & 108752 & 2004 & 82.20 & 0.07 & 8140.00 & 81863.86 & 1.01 & 1.00 & 1.01 \\
35927 & 106434 & 2004 & 739.30 & 0.11 & 73507.00 & 730466.47 & 1.01 & 0.99 & 0.99 \\
29281 & 105581 & 2004 & 176.40 & 0.09 & 17689.00 & 174293.49 & 1.00 & 0.99 & 0.99 \\
27731 & 105320 & 2004 & 344.50 & 0.21 & 34382.00 & 282245.83 & 1.00 & 0.82 & 0.82 \\
37278 & 106725 & 2004 & 4.00 & 0.04 & 397.00 & 3972.22 & 1.01 & 0.99 & 1.00 \\
44186 & 109271 & 2004 & 86.20 & 0.03 & 8130.00 & 86806.37 & 1.06 & 1.01 & 1.07 \\
41393 & 108745 & 2004 & 74.20 & 0.29 & 7406.00 & 71095.32 & 1.00 & 0.96 & 0.96 \\
35877 & 106421 & 2004 & 79.80 & 0.33 & 7946.00 & 78944.95 & 1.00 & 0.99 & 0.99 \\
49877 & 240370 & 2004 & 14.20 & -0.03 & 755.00 & 6322.27 & 1.88 & 0.45 & 0.84 \\
18659 & 102500 & 2004 & 42.10 & 0.05 & 4250.00 & 40179.04 & 0.99 & 0.95 & 0.95 \\
27757 & 105321 & 2004 & 481.40 & 0.12 & 41820.00 & 442465.50 & 1.15 & 0.92 & 1.06 \\
49974 & 240379 & 2004 & 13.80 & 0.04 & 1360.00 & 12827.09 & 1.01 & 0.93 & 0.94 \\
29333 & 105587 & 2004 & 259.10 & 0.16 & 25881.00 & 251103.64 & 1.00 & 0.97 & 0.97 \\
42153 & 108930 & 2004 & 156.40 & 0.09 & 16261.00 & 158251.08 & 0.96 & 1.01 & 0.97 \\
55635 & 400132 & 2004 & 133.30 & 0.20 & 11736.00 & 130482.23 & 1.14 & 0.98 & 1.11 \\
35455 & 106361 & 2004 & 328.80 & 0.10 & 32749.00 & 311310.92 & 1.00 & 0.95 & 0.95 \\
46030 & 200178 & 2004 & 52.30 & 0.14 & 4973.00 & 50003.44 & 1.05 & 0.96 & 1.01 \\
2510 & 100336 & 2004 & 164.10 & 0.11 & 16412.00 & 145770.35 & 1.00 & 0.89 & 0.89 \\
34560 & 106255 & 2004 & 351.00 & 0.10 & 32001.00 & 326273.56 & 1.10 & 0.93 & 1.02 \\
46024 & 200177 & 2004 & 45.10 & 0.08 & 4270.00 & 43604.19 & 1.06 & 0.97 & 1.02 \\
56506 & 400221 & 2004 & 26.30 & 0.09 & 1704.00 & 17373.85 & 1.54 & 0.66 & 1.02 \\
50708 & 240441 & 2004 & 1487.20 & 0.13 & 131462.00 & 1384194.32 & 1.13 & 0.93 & 1.05 \\
22896 & 103084 & 2004 & 566.80 & 0.17 & 52046.00 & 565150.83 & 1.09 & 1.00 & 1.09 \\
46003 & 200176 & 2004 & 59.10 & 0.12 & 5919.00 & 58932.16 & 1.00 & 1.00 & 1.00 \\
13386 & 101736 & 2004 & 46.30 & 0.09 & 4652.00 & 44194.66 & 1.00 & 0.95 & 0.95 \\
27865 & 105335 & 2004 & 174.20 & 0.07 & 17424.00 & 169111.93 & 1.00 & 0.97 & 0.97 \\
10215 & 101274 & 2004 & 434.60 & 0.08 & 46075.00 & 422843.36 & 0.94 & 0.97 & 0.92 \\
37252 & 106724 & 2004 & 2165.70 & 0.22 & 216137.00 & 2115022.50 & 1.00 & 0.98 & 0.98 \\
46431 & 200244 & 2004 & 6.70 & -0.05 & 614.00 & 6638.61 & 1.09 & 0.99 & 1.08 \\
54463 & 368366 & 2004 & 138.80 & 0.15 & 12955.00 & 143690.66 & 1.07 & 1.04 & 1.11 \\
21858 & 102957 & 2004 & 481.10 & 0.04 & 45293.00 & 441321.30 & 1.06 & 0.92 & 0.97 \\
56528 & 400222 & 2004 & 39.50 & 0.08 & 3458.00 & 33565.22 & 1.14 & 0.85 & 0.97 \\
19337 & 102597 & 2004 & 202.20 & 0.08 & 20239.00 & 201651.57 & 1.00 & 1.00 & 1.00 \\
123 & 100009 & 2004 & 239.50 & 0.05 & 23958.00 & 236786.37 & 1.00 & 0.99 & 0.99 \\
54828 & 400017 & 2004 & 144.70 & 0.08 & 13945.00 & 142567.68 & 1.04 & 0.99 & 1.02 \\
6752 & 100947 & 2004 & 440.40 & 0.07 & 43679.00 & 445505.59 & 1.01 & 1.01 & 1.02 \\
37380 & 106742 & 2004 & 154.70 & 0.13 & 15418.00 & 150770.10 & 1.00 & 0.97 & 0.98 \\
54712 & 378591 & 2004 & 3.00 & 0.26 & 168.00 & 1606.44 & 1.79 & 0.54 & 0.96 \\
41114 & 108202 & 2004 & 114.20 & 0.05 & 10346.00 & 107269.14 & 1.10 & 0.94 & 1.04 \\
48606 & 240113 & 2004 & 62.10 & 0.12 & 5519.00 & 52707.62 & 1.13 & 0.85 & 0.96 \\
6944 & 100973 & 2004 & 62.40 & 0.11 & 6363.00 & 56964.54 & 0.98 & 0.91 & 0.90 \\
29662 & 105631 & 2004 & 4.10 & -0.03 & 477.00 & 4865.84 & 0.86 & 1.19 & 1.02 \\
96679 & 611003 & 2004 & 393.40 & 0.09 & 36470.00 & 357179.74 & 1.08 & 0.91 & 0.98 \\
49765 & 240358 & 2004 & 40.00 & 0.19 & 4008.00 & 39111.96 & 1.00 & 0.98 & 0.98 \\
53113 & 339977 & 2004 & 109.40 & 0.10 & 10399.00 & 102778.83 & 1.05 & 0.94 & 0.99 \\
13982 & 101794 & 2004 & 1201.60 & 0.06 & 119864.00 & 1185589.09 & 1.00 & 0.99 & 0.99 \\
3088 & 100408 & 2004 & 202.80 & 0.04 & 20248.00 & 192127.39 & 1.00 & 0.95 & 0.95 \\
34399 & 106231 & 2004 & 365.40 & 0.20 & 37616.00 & 375142.80 & 0.97 & 1.03 & 1.00 \\
27342 & 105269 & 2004 & 1168.50 & 0.21 & 116588.00 & 1127287.52 & 1.00 & 0.96 & 0.97 \\
58988 & 410242 & 2004 & 12.70 & 0.24 & 1079.00 & 11644.06 & 1.18 & 0.92 & 1.08 \\
63951 & 500571 & 2004 & 95.30 & 0.03 & 9361.00 & 98271.81 & 1.02 & 1.03 & 1.05 \\
45885 & 200153 & 2004 & 72.70 & 0.01 & 5021.00 & 51674.16 & 1.45 & 0.71 & 1.03 \\
14835 & 101916 & 2004 & 656.10 & 0.22 & 58181.00 & 620033.51 & 1.13 & 0.95 & 1.07 \\
51912 & 300102 & 2004 & 16.80 & 0.04 & 1704.00 & 14745.22 & 0.99 & 0.88 & 0.87 \\
37644 & 106983 & 2004 & 259.10 & 0.14 & 28059.00 & 251161.72 & 0.92 & 0.97 & 0.90 \\
61791 & 500131 & 2004 & 101.00 & 0.36 & 7776.00 & 84124.37 & 1.30 & 0.83 & 1.08 \\
53193 & 341117 & 2004 & 41.80 & 0.04 & 3398.00 & 38282.22 & 1.23 & 0.92 & 1.13 \\
58924 & 410235 & 2004 & 2.40 & 0.12 & 223.00 & 2337.06 & 1.08 & 0.97 & 1.05 \\
29539 & 105610 & 2004 & 341.30 & 0.11 & 33813.00 & 326480.66 & 1.01 & 0.96 & 0.97 \\
41102 & 108200 & 2004 & 110.50 & 0.02 & 10947.00 & 115553.08 & 1.01 & 1.05 & 1.06 \\
55062 & 400050 & 2004 & 2299.60 & 0.20 & 198698.00 & 1939413.36 & 1.16 & 0.84 & 0.98 \\
10081 & 101258 & 2004 & 4922.10 & 0.07 & 462058.00 & 4280899.82 & 1.07 & 0.87 & 0.93 \\
23191 & 103144 & 2004 & 21.70 & 0.18 & 2063.00 & 20465.72 & 1.05 & 0.94 & 0.99 \\
42207 & 108939 & 2004 & 36.30 & 0.03 & 3626.00 & 35056.94 & 1.00 & 0.97 & 0.97 \\
34372 & 106230 & 2004 & 215.50 & 0.15 & 22648.00 & 226606.48 & 0.95 & 1.05 & 1.00 \\
17110 & 102257 & 2004 & 1386.30 & 0.11 & 128245.00 & 1348671.90 & 1.08 & 0.97 & 1.05 \\
44192 & 109273 & 2004 & 35.50 & 0.16 & 3824.00 & 29102.87 & 0.93 & 0.82 & 0.76 \\
27317 & 105268 & 2004 & 622.80 & 0.13 & 62233.00 & 604520.27 & 1.00 & 0.97 & 0.97 \\
31142 & 105861 & 2004 & 367.70 & 0.23 & 37945.00 & 355126.67 & 0.97 & 0.97 & 0.94 \\
3466 & 100439 & 2004 & 18.70 & 0.11 & 1869.00 & 18307.63 & 1.00 & 0.98 & 0.98 \\
29532 & 105607 & 2004 & 3.50 & 0.16 & 350.00 & 2978.02 & 1.00 & 0.85 & 0.85 \\
16261 & 102105 & 2004 & 164.00 & 0.13 & 16036.00 & 147542.92 & 1.02 & 0.90 & 0.92 \\
11564 & 101430 & 2004 & 161.00 & 0.25 & 16142.00 & 158230.92 & 1.00 & 0.98 & 0.98 \\
3110 & 100409 & 2004 & 663.10 & 0.19 & 68194.00 & 579254.07 & 0.97 & 0.87 & 0.85 \\
56447 & 400211 & 2004 & 388.90 & 0.03 & 38930.00 & 374790.74 & 1.00 & 0.96 & 0.96 \\
10772 & 101330 & 2004 & 2555.70 & 0.14 & 255560.00 & 2496941.85 & 1.00 & 0.98 & 0.98 \\
34432 & 106239 & 2004 & 56.00 & 0.19 & 5168.00 & 52470.53 & 1.08 & 0.94 & 1.02 \\
29686 & 105635 & 2004 & 29.30 & 0.11 & 2962.00 & 24527.76 & 0.99 & 0.84 & 0.83 \\
37578 & 106969 & 2004 & 43.80 & 0.17 & 4118.00 & 39631.33 & 1.06 & 0.90 & 0.96 \\
54428 & 367713 & 2004 & 4.70 & 0.15 & 465.00 & 4464.65 & 1.01 & 0.95 & 0.96 \\
50943 & 240469 & 2004 & 99.10 & 0.13 & 8943.00 & 94909.00 & 1.11 & 0.96 & 1.06 \\
42829 & 109023 & 2004 & 15.20 & 0.02 & 1472.00 & 14724.46 & 1.03 & 0.97 & 1.00 \\
37553 & 106968 & 2004 & 110.80 & -0.00 & 11065.00 & 110059.01 & 1.00 & 0.99 & 0.99 \\
43942 & 109237 & 2004 & 143.30 & 0.12 & 14137.00 & 136789.26 & 1.01 & 0.95 & 0.97 \\
29491 & 105598 & 2004 & 370.60 & 0.08 & 34606.00 & 363744.40 & 1.07 & 0.98 & 1.05 \\
34448 & 106240 & 2004 & 226.80 & 0.14 & 20590.00 & 226942.72 & 1.10 & 1.00 & 1.10 \\
35558 & 106375 & 2004 & 23.30 & 0.10 & 2336.00 & 21763.86 & 1.00 & 0.93 & 0.93 \\
31087 & 105857 & 2004 & 641.70 & 0.25 & 54348.00 & 568679.06 & 1.18 & 0.89 & 1.05 \\
4493 & 100635 & 2004 & 1002.80 & 0.30 & 97182.00 & 998886.21 & 1.03 & 1.00 & 1.03 \\
42272 & 108947 & 2004 & 413.10 & 0.12 & 42693.00 & 434491.31 & 0.97 & 1.05 & 1.02 \\
35583 & 106376 & 2004 & 6.00 & 0.10 & 612.00 & 6001.00 & 0.98 & 1.00 & 0.98 \\
63938 & 500568 & 2004 & 15.50 & 0.13 & 1458.00 & 14967.40 & 1.06 & 0.97 & 1.03 \\
23162 & 103136 & 2004 & 168.40 & 0.08 & 17823.00 & 165249.12 & 0.94 & 0.98 & 0.93 \\
47656 & 216504 & 2004 & 12.90 & 0.11 & 1957.00 & 16216.31 & 0.66 & 1.26 & 0.83 \\
29672 & 105632 & 2004 & 39.50 & 0.05 & 6476.00 & 66964.63 & 0.61 & 1.70 & 1.03 \\
8683 & 101094 & 2004 & 825.70 & 0.08 & 82403.00 & 756792.98 & 1.00 & 0.92 & 0.92 \\
54406 & 367600 & 2004 & 22.80 & 0.24 & 2566.00 & 21041.77 & 0.89 & 0.92 & 0.82 \\
5907 & 100811 & 2004 & 874.30 & 0.07 & 87051.00 & 836122.13 & 1.00 & 0.96 & 0.96 \\
45893 & 200156 & 2004 & 50.20 & 0.07 & 4651.00 & 47129.84 & 1.08 & 0.94 & 1.01 \\
31114 & 105860 & 2004 & 1170.90 & 0.01 & 117509.00 & 1123174.71 & 1.00 & 0.96 & 0.96 \\
17087 & 102255 & 2004 & 13.50 & 0.15 & 1345.00 & 13177.94 & 1.00 & 0.98 & 0.98 \\
48719 & 240130 & 2004 & 416.20 & -0.02 & 50112.00 & 457578.91 & 0.83 & 1.10 & 0.91 \\
37603 & 106972 & 2004 & 77.20 & 0.09 & 7639.00 & 76394.09 & 1.01 & 0.99 & 1.00 \\
53583 & 354336 & 2004 & 8.90 & 0.13 & 896.00 & 8669.51 & 0.99 & 0.97 & 0.97 \\
35707 & 106391 & 2004 & 82.60 & 0.10 & 8278.00 & 82367.87 & 1.00 & 1.00 & 1.00 \\
56450 & 400212 & 2004 & 31.00 & 0.05 & 4023.00 & 36089.41 & 0.77 & 1.16 & 0.90 \\
20861 & 102797 & 2004 & 76.50 & 0.23 & 7667.00 & 71262.78 & 1.00 & 0.93 & 0.93 \\
23146 & 103134 & 2004 & 412.90 & 0.10 & 44505.00 & 421135.21 & 0.93 & 1.02 & 0.95 \\
59193 & 410444 & 2004 & 68.10 & 0.16 & 5868.00 & 63065.71 & 1.16 & 0.93 & 1.07 \\
27375 & 105275 & 2004 & 92.50 & 0.11 & 9220.00 & 90798.83 & 1.00 & 0.98 & 0.98 \\
59216 & 410445 & 2004 & 91.90 & 0.16 & 8140.00 & 87699.65 & 1.13 & 0.95 & 1.08 \\
46824 & 200303 & 2004 & 18.50 & 0.17 & 1798.00 & 17600.04 & 1.03 & 0.95 & 0.98 \\
9003 & 101108 & 2004 & 1092.90 & 0.15 & 95190.00 & 950010.44 & 1.15 & 0.87 & 1.00 \\
23254 & 103152 & 2004 & 2951.60 & 0.07 & 287731.00 & 2695504.01 & 1.03 & 0.91 & 0.94 \\
29621 & 105627 & 2004 & 1089.10 & 0.19 & 109326.00 & 1075991.33 & 1.00 & 0.99 & 0.98 \\
27223 & 105256 & 2004 & 486.20 & 0.19 & 48845.00 & 413320.81 & 1.00 & 0.85 & 0.85 \\
29580 & 105616 & 2004 & 43.10 & 0.23 & 4016.00 & 32454.47 & 1.07 & 0.75 & 0.81 \\
37722 & 107135 & 2004 & 593.40 & 0.18 & 58605.00 & 551665.99 & 1.01 & 0.93 & 0.94 \\
34334 & 106223 & 2004 & 95.70 & 0.18 & 8406.00 & 81969.60 & 1.14 & 0.86 & 0.98 \\
11604 & 101431 & 2004 & 329.50 & 0.16 & 32926.00 & 304985.36 & 1.00 & 0.93 & 0.93 \\
15728 & 102016 & 2004 & 8972.00 & 0.09 & 909721.00 & 8735918.27 & 0.99 & 0.97 & 0.96 \\
64015 & 500577 & 2004 & 815.70 & 0.12 & 74667.00 & 746687.04 & 1.09 & 0.92 & 1.00 \\
47841 & 222351 & 2004 & 484.30 & 0.23 & 48416.00 & 467545.24 & 1.00 & 0.97 & 0.97 \\
48145 & 240027 & 2004 & 601.50 & 0.31 & 58800.00 & 563179.42 & 1.02 & 0.94 & 0.96 \\
31193 & 105866 & 2004 & 8813.70 & 0.11 & 810402.00 & 8770671.28 & 1.09 & 1.00 & 1.08 \\
20811 & 102795 & 2004 & 1116.40 & 0.28 & 90373.00 & 963363.35 & 1.24 & 0.86 & 1.07 \\
635 & 100085 & 2004 & 6879.90 & 0.05 & 711982.00 & 6700710.22 & 0.97 & 0.97 & 0.94 \\
52644 & 305586 & 2004 & 136.70 & 0.15 & 13201.00 & 128155.59 & 1.04 & 0.94 & 0.97 \\
50866 & 240462 & 2004 & 26.30 & 0.14 & 2306.00 & 23882.75 & 1.14 & 0.91 & 1.04 \\
49742 & 240352 & 2004 & 2.90 & 0.05 & 284.00 & 2732.14 & 1.02 & 0.94 & 0.96 \\
44203 & 109274 & 2004 & 69.00 & 0.02 & 5583.00 & 56605.17 & 1.24 & 0.82 & 1.01 \\
42238 & 108944 & 2004 & 157.60 & 0.08 & 10469.00 & 86441.31 & 1.51 & 0.55 & 0.83 \\
43911 & 109230 & 2004 & 347.60 & 0.12 & 32764.00 & 326210.32 & 1.06 & 0.94 & 1.00 \\
42885 & 109030 & 2004 & 65.70 & 0.11 & 5919.00 & 56451.56 & 1.11 & 0.86 & 0.95 \\
50852 & 240459 & 2004 & 125.40 & 0.13 & 11660.00 & 113071.67 & 1.08 & 0.90 & 0.97 \\
661 & 100087 & 2004 & 2420.10 & 0.08 & 237353.00 & 2478023.18 & 1.02 & 1.02 & 1.04 \\
29609 & 105623 & 2004 & 47.60 & 0.15 & 4740.00 & 47307.75 & 1.00 & 0.99 & 1.00 \\
41041 & 108183 & 2004 & 117.50 & 0.21 & 11730.00 & 112509.89 & 1.00 & 0.96 & 0.96 \\
42869 & 109028 & 2004 & 383.50 & 0.15 & 39718.00 & 368710.22 & 0.97 & 0.96 & 0.93 \\
14408 & 101854 & 2004 & 17222.80 & 0.13 & 1472384.00 & 15635274.10 & 1.17 & 0.91 & 1.06 \\
10051 & 101256 & 2004 & 4.50 & 0.07 & 438.00 & 3883.62 & 1.03 & 0.86 & 0.89 \\
48597 & 240112 & 2004 & 17.40 & 0.07 & 1744.00 & 15772.60 & 1.00 & 0.91 & 0.90 \\
34324 & 106222 & 2004 & 195.80 & 0.13 & 17670.00 & 191967.07 & 1.11 & 0.98 & 1.09 \\
17152 & 102259 & 2004 & 504.00 & 0.14 & 44728.00 & 483902.29 & 1.13 & 0.96 & 1.08 \\
41062 & 108186 & 2004 & 8.90 & 0.14 & 992.00 & 9913.89 & 0.90 & 1.11 & 1.00 \\
4532 & 100637 & 2004 & 899.10 & 0.06 & 92874.00 & 923813.50 & 0.97 & 1.03 & 0.99 \\
64005 & 500576 & 2004 & 185.50 & 0.03 & 17522.00 & 155090.47 & 1.06 & 0.84 & 0.89 \\
21711 & 102940 & 2004 & 1781.20 & 0.17 & 161214.00 & 1823353.83 & 1.10 & 1.02 & 1.13 \\
10736 & 101320 & 2004 & 199.50 & 0.31 & 16747.00 & 147263.93 & 1.19 & 0.74 & 0.88 \\
16246 & 102104 & 2004 & 758.70 & 0.23 & 71731.00 & 632020.48 & 1.06 & 0.83 & 0.88 \\
35677 & 106386 & 2004 & 446.60 & 0.11 & 52499.00 & 521064.56 & 0.85 & 1.17 & 0.99 \\
8363 & 101084 & 2004 & 2205.40 & 0.00 & 222232.00 & 2133456.92 & 0.99 & 0.97 & 0.96 \\
47647 & 216438 & 2004 & 538.40 & 0.10 & 53795.00 & 535990.22 & 1.00 & 1.00 & 1.00 \\
52677 & 305766 & 2004 & 83.30 & 0.11 & 8368.00 & 83451.01 & 1.00 & 1.00 & 1.00 \\
29551 & 105611 & 2004 & 3978.50 & 0.18 & 399399.00 & 3342029.74 & 1.00 & 0.84 & 0.84 \\
42213 & 108943 & 2004 & 2606.60 & 0.19 & 219393.00 & 2415927.71 & 1.19 & 0.93 & 1.10 \\
27279 & 105260 & 2004 & 343.90 & 0.22 & 35320.00 & 336618.49 & 0.97 & 0.98 & 0.95 \\
46830 & 200304 & 2004 & 32.90 & 0.18 & 2554.00 & 25705.17 & 1.29 & 0.78 & 1.01 \\
21681 & 102939 & 2004 & 6783.20 & 0.11 & 631434.00 & 6817437.51 & 1.07 & 1.01 & 1.08 \\
37663 & 106992 & 2004 & 189.80 & 0.12 & 18535.00 & 195923.52 & 1.02 & 1.03 & 1.06 \\
31165 & 105865 & 2004 & 148.10 & 0.17 & 14797.00 & 141567.16 & 1.00 & 0.96 & 0.96 \\
5875 & 100809 & 2004 & 1507.50 & 0.08 & 150387.00 & 1434240.84 & 1.00 & 0.95 & 0.95 \\
29650 & 105630 & 2004 & 5.10 & 0.10 & 610.00 & 6322.81 & 0.84 & 1.24 & 1.04 \\
37657 & 106985 & 2004 & 195.00 & 0.10 & 18565.00 & 194158.60 & 1.05 & 1.00 & 1.05 \\
53176 & 341116 & 2004 & 53.50 & 0.05 & 4643.00 & 52792.05 & 1.15 & 0.99 & 1.14 \\
41096 & 108197 & 2004 & 40.50 & 0.09 & 3922.00 & 40880.68 & 1.03 & 1.01 & 1.04 \\
55463 & 400100 & 2004 & 144.70 & 0.19 & 12880.00 & 126037.18 & 1.12 & 0.87 & 0.98 \\
10707 & 101312 & 2004 & 8902.60 & 0.13 & 878413.00 & 8646880.67 & 1.01 & 0.97 & 0.98 \\
37650 & 106984 & 2004 & 139.70 & 0.10 & 13010.00 & 137846.23 & 1.07 & 0.99 & 1.06 \\
2370 & 100320 & 2004 & 83.10 & 0.19 & 7637.00 & 81780.63 & 1.09 & 0.98 & 1.07 \\
42843 & 109025 & 2004 & 115.40 & 0.21 & 12502.00 & 100424.43 & 0.92 & 0.87 & 0.80 \\
35606 & 106379 & 2004 & 194.70 & 0.13 & 18020.00 & 191243.18 & 1.08 & 0.98 & 1.06 \\
19083 & 102548 & 2004 & 795.60 & 0.18 & 68809.00 & 785144.43 & 1.16 & 0.99 & 1.14 \\
53134 & 340902 & 2004 & 7.10 & 0.06 & 639.00 & 6713.09 & 1.11 & 0.95 & 1.05 \\
27250 & 105259 & 2004 & 222.90 & 0.11 & 22282.00 & 221738.85 & 1.00 & 0.99 & 1.00 \\
46852 & 200309 & 2004 & 136.80 & 0.18 & 13950.00 & 131444.30 & 0.98 & 0.96 & 0.94 \\
37685 & 106995 & 2004 & 2683.60 & 0.23 & 265639.00 & 2336700.16 & 1.01 & 0.87 & 0.88 \\
19405 & 102600 & 2004 & 950.60 & 0.16 & 88358.00 & 931943.99 & 1.08 & 0.98 & 1.05 \\
34354 & 106224 & 2004 & 461.10 & 0.09 & 45729.00 & 412607.51 & 1.01 & 0.89 & 0.90 \\
47816 & 222027 & 2004 & 2089.50 & 0.14 & 239950.00 & 2144428.95 & 0.87 & 1.03 & 0.89 \\
52654 & 305590 & 2004 & 335.40 & 0.11 & 33112.00 & 320698.66 & 1.01 & 0.96 & 0.97 \\
37674 & 106993 & 2004 & 160.20 & 0.23 & 14576.00 & 136856.28 & 1.10 & 0.85 & 0.94 \\
23221 & 103145 & 2004 & 57.80 & 0.11 & 5547.00 & 59996.53 & 1.04 & 1.04 & 1.08 \\
53156 & 341114 & 2004 & 25.70 & 0.03 & 2284.00 & 25181.31 & 1.13 & 0.98 & 1.10 \\
12787 & 101595 & 2004 & 1200.60 & 0.15 & 108504.00 & 1061870.78 & 1.11 & 0.88 & 0.98 \\
5328 & 100753 & 2004 & 1565.20 & 0.10 & 161742.00 & 1496078.48 & 0.97 & 0.96 & 0.92 \\
41160 & 108211 & 2004 & 557.30 & 0.21 & 48685.00 & 470671.62 & 1.14 & 0.84 & 0.97 \\
27405 & 105276 & 2004 & 2543.20 & 0.15 & 214063.00 & 2025946.97 & 1.19 & 0.80 & 0.95 \\
5359 & 100757 & 2004 & 19.70 & 0.12 & 1772.00 & 17808.79 & 1.11 & 0.90 & 1.01 \\
27546 & 105287 & 2004 & 98.00 & 0.04 & 9807.00 & 89631.42 & 1.00 & 0.91 & 0.91 \\
56485 & 400217 & 2004 & 3.00 & 0.11 & 228.00 & 2560.04 & 1.32 & 0.85 & 1.12 \\
4134 & 100559 & 2004 & 21.50 & 0.19 & 1961.00 & 21831.32 & 1.10 & 1.02 & 1.11 \\
53547 & 351713 & 2004 & 345.50 & 0.14 & 35310.00 & 342698.02 & 0.98 & 0.99 & 0.97 \\
10150 & 101263 & 2004 & 237.90 & 0.08 & 21351.00 & 215207.07 & 1.11 & 0.90 & 1.01 \\
19051 & 102545 & 2004 & 3545.90 & 0.17 & 317894.00 & 2836635.78 & 1.12 & 0.80 & 0.89 \\
57245 & 400323 & 2004 & 403.20 & 0.10 & 35801.00 & 361103.50 & 1.13 & 0.90 & 1.01 \\
45918 & 200167 & 2004 & 209.00 & 0.12 & 20164.00 & 184712.64 & 1.04 & 0.88 & 0.92 \\
51754 & 240549 & 2004 & 37.60 & 0.16 & 3667.00 & 36213.14 & 1.03 & 0.96 & 0.99 \\
29410 & 105593 & 2004 & 64.70 & 0.13 & 6357.00 & 63570.44 & 1.02 & 0.98 & 1.00 \\
59326 & 410465 & 2004 & 14.70 & 0.18 & 1456.00 & 14445.23 & 1.01 & 0.98 & 0.99 \\
44620 & 109348 & 2004 & 1066.00 & 0.13 & 106475.00 & 1005946.19 & 1.00 & 0.94 & 0.94 \\
54694 & 378134 & 2004 & 104.10 & 0.18 & 7881.00 & 76091.74 & 1.32 & 0.73 & 0.97 \\
41281 & 108719 & 2004 & 424.70 & 0.10 & 43409.00 & 430374.27 & 0.98 & 1.01 & 0.99 \\
41269 & 108710 & 2004 & 414.40 & -0.06 & 41542.00 & 373803.93 & 1.00 & 0.90 & 0.90 \\
30991 & 105845 & 2004 & 2.40 & 0.04 & 456.00 & 4278.58 & 0.53 & 1.78 & 0.94 \\
45913 & 200164 & 2004 & 29.20 & 0.10 & 2931.00 & 26145.68 & 1.00 & 0.90 & 0.89 \\
34528 & 106249 & 2004 & 184.00 & 0.10 & 18403.00 & 178100.54 & 1.00 & 0.97 & 0.97 \\
10675 & 101307 & 2004 & 1005.50 & 0.32 & 93876.00 & 956974.96 & 1.07 & 0.95 & 1.02 \\
42166 & 108932 & 2004 & 1435.70 & 0.16 & 143177.00 & 1317054.44 & 1.00 & 0.92 & 0.92 \\
37457 & 106919 & 2004 & 29.80 & 0.12 & 2973.00 & 29508.20 & 1.00 & 0.99 & 0.99 \\
31002 & 105846 & 2004 & 1350.50 & 0.15 & 135287.00 & 1352896.80 & 1.00 & 1.00 & 1.00 \\
575 & 100076 & 2004 & 471.10 & 0.13 & 46084.00 & 476842.22 & 1.02 & 1.01 & 1.03 \\
35515 & 106369 & 2004 & 82.00 & -0.07 & 9344.00 & 93859.79 & 0.88 & 1.14 & 1.00 \\
3058 & 100401 & 2004 & 1741.40 & 0.09 & 173910.00 & 1659939.38 & 1.00 & 0.95 & 0.95 \\
15793 & 102018 & 2004 & 383.30 & 0.07 & 46123.00 & 390538.09 & 0.83 & 1.02 & 0.85 \\
12842 & 101602 & 2004 & 4051.70 & 0.18 & 356446.00 & 3276319.34 & 1.14 & 0.81 & 0.92 \\
96667 & 611002 & 2004 & 3018.60 & 0.02 & 301490.00 & 3164565.73 & 1.00 & 1.05 & 1.05 \\
44644 & 109350 & 2004 & 47.70 & 0.10 & 6545.00 & 65447.22 & 0.73 & 1.37 & 1.00 \\
5406 & 100760 & 2004 & 567.80 & 0.12 & 60751.00 & 545472.18 & 0.93 & 0.96 & 0.90 \\
44212 & 109275 & 2004 & 97.60 & 0.16 & 8313.00 & 79566.99 & 1.17 & 0.82 & 0.96 \\
35808 & 106413 & 2004 & 4762.50 & 0.15 & 584198.00 & 5842305.16 & 0.82 & 1.23 & 1.00 \\
29417 & 105594 & 2004 & 7.80 & 0.15 & 666.00 & 7043.22 & 1.17 & 0.90 & 1.06 \\
17008 & 102230 & 2004 & 22.70 & 0.09 & 1336.00 & 13789.36 & 1.70 & 0.61 & 1.03 \\
23004 & 103101 & 2004 & 121.10 & 0.17 & 12119.00 & 113964.50 & 1.00 & 0.94 & 0.94 \\
37425 & 106896 & 2004 & 6.60 & 0.19 & 663.00 & 6591.76 & 1.00 & 1.00 & 0.99 \\
549 & 100075 & 2004 & 2035.80 & 0.12 & 222684.00 & 2039454.10 & 0.91 & 1.00 & 0.92 \\
55186 & 400069 & 2004 & 80.50 & 0.11 & 8096.00 & 80794.44 & 0.99 & 1.00 & 1.00 \\
13226 & 101708 & 2004 & 346.90 & 0.09 & 36209.00 & 321022.24 & 0.96 & 0.93 & 0.89 \\
49826 & 240365 & 2004 & 39.20 & 0.07 & 3868.00 & 37495.41 & 1.01 & 0.96 & 0.97 \\
48270 & 240058 & 2004 & 865.70 & 0.17 & 82136.00 & 808504.30 & 1.05 & 0.93 & 0.98 \\
53106 & 339611 & 2004 & 20.10 & 0.17 & 1887.00 & 18775.34 & 1.07 & 0.93 & 0.99 \\
43974 & 109249 & 2004 & 174.30 & -0.01 & 22255.00 & 180833.18 & 0.78 & 1.04 & 0.81 \\
47570 & 212809 & 2004 & 27.80 & 0.12 & 2536.00 & 21772.20 & 1.10 & 0.78 & 0.86 \\
63879 & 500563 & 2004 & 26.70 & 0.21 & 2680.00 & 26687.47 & 1.00 & 1.00 & 1.00 \\
20952 & 102813 & 2004 & 1521.40 & 0.21 & 152139.00 & 1448792.13 & 1.00 & 0.95 & 0.95 \\
37391 & 106747 & 2004 & 227.60 & 0.13 & 23085.00 & 211244.98 & 0.99 & 0.93 & 0.92 \\
3032 & 100399 & 2004 & 376.90 & 0.09 & 37724.00 & 312096.60 & 1.00 & 0.83 & 0.83 \\
4482 & 100634 & 2004 & 808.70 & 0.07 & 79668.00 & 784959.60 & 1.02 & 0.97 & 0.99 \\
35834 & 106415 & 2004 & 249.20 & 0.12 & 24275.00 & 217309.33 & 1.03 & 0.87 & 0.90 \\
37387 & 106746 & 2004 & 16.10 & 0.05 & 1616.00 & 15498.55 & 1.00 & 0.96 & 0.96 \\
46519 & 200250 & 2004 & 356.70 & 0.12 & 32481.00 & 355551.32 & 1.10 & 1.00 & 1.09 \\
41316 & 108726 & 2004 & 36.50 & 0.24 & 2855.00 & 30531.62 & 1.28 & 0.84 & 1.07 \\
45923 & 200168 & 2004 & 335.60 & 0.09 & 32329.00 & 309527.72 & 1.04 & 0.92 & 0.96 \\
35790 & 106402 & 2004 & 86.80 & 0.25 & 8075.00 & 80428.27 & 1.07 & 0.93 & 1.00 \\
6119 & 100823 & 2004 & 40.80 & 0.23 & 4051.00 & 39316.89 & 1.01 & 0.96 & 0.97 \\
29758 & 105644 & 2004 & 20.60 & 0.11 & 2151.00 & 19403.02 & 0.96 & 0.94 & 0.90 \\
7741 & 101056 & 2004 & 32106.10 & 0.10 & 3029157.00 & 29162547.66 & 1.06 & 0.91 & 0.96 \\
29403 & 105592 & 2004 & 262.20 & 0.04 & 26275.00 & 251938.23 & 1.00 & 0.96 & 0.96 \\
2445 & 100330 & 2004 & 2376.20 & 0.27 & 201459.00 & 2257304.34 & 1.18 & 0.95 & 1.12 \\
34537 & 106250 & 2004 & 145.10 & 0.05 & 14514.00 & 127959.04 & 1.00 & 0.88 & 0.88 \\
37417 & 106869 & 2004 & 163.10 & 0.14 & 13351.00 & 138199.94 & 1.22 & 0.85 & 1.04 \\
41306 & 108723 & 2004 & 240.40 & 0.10 & 28126.00 & 241398.61 & 0.85 & 1.00 & 0.86 \\
14473 & 101861 & 2004 & 1910.50 & 0.10 & 180480.00 & 1688682.41 & 1.06 & 0.88 & 0.94 \\
61901 & 500290 & 2004 & 151.80 & 0.32 & 14775.00 & 149513.57 & 1.03 & 0.98 & 1.01 \\
30964 & 105842 & 2004 & 497.40 & 0.13 & 49773.00 & 492633.08 & 1.00 & 0.99 & 0.99 \\
21817 & 102952 & 2004 & 227.40 & 0.03 & 22779.00 & 216780.53 & 1.00 & 0.95 & 0.95 \\
22986 & 103100 & 2004 & 237.70 & 0.12 & 23768.00 & 231031.12 & 1.00 & 0.97 & 0.97 \\
54673 & 378072 & 2004 & 53.00 & 0.14 & 5342.00 & 51997.83 & 0.99 & 0.98 & 0.97 \\
27593 & 105295 & 2004 & 473.50 & 0.09 & 48269.00 & 445790.40 & 0.98 & 0.94 & 0.92 \\
44597 & 109347 & 2004 & 1169.30 & 0.19 & 116481.00 & 1036266.97 & 1.00 & 0.89 & 0.89 \\
16312 & 102124 & 2004 & 1504.40 & 0.09 & 170144.00 & 1721856.78 & 0.88 & 1.14 & 1.01 \\
18610 & 102491 & 2004 & 793.50 & 0.20 & 79660.00 & 768201.71 & 1.00 & 0.97 & 0.96 \\
29736 & 105643 & 2004 & 948.00 & 0.11 & 95101.00 & 857786.82 & 1.00 & 0.90 & 0.90 \\
37526 & 106948 & 2004 & 385.90 & 0.15 & 38596.00 & 324186.66 & 1.00 & 0.84 & 0.84 \\
54432 & 367841 & 2004 & 750.00 & 0.13 & 75996.00 & 671053.65 & 0.99 & 0.89 & 0.88 \\
35750 & 106398 & 2004 & 6.20 & 0.09 & 626.00 & 6288.83 & 0.99 & 1.01 & 1.00 \\
5373 & 100758 & 2004 & 16.30 & 0.10 & 1561.00 & 15586.76 & 1.04 & 0.96 & 1.00 \\
50951 & 240473 & 2004 & 40.80 & 0.04 & 4090.00 & 40240.26 & 1.00 & 0.99 & 0.98 \\
51761 & 240550 & 2004 & 1.50 & 0.07 & 117.00 & 994.05 & 1.28 & 0.66 & 0.85 \\
29465 & 105597 & 2004 & 98.00 & 0.09 & 9797.00 & 97996.42 & 1.00 & 1.00 & 1.00 \\
16205 & 102090 & 2004 & 5628.70 & 0.06 & 612200.00 & 5291730.53 & 0.92 & 0.94 & 0.86 \\
41189 & 108670 & 2004 & 282.70 & 0.15 & 25778.00 & 274096.47 & 1.10 & 0.97 & 1.06 \\
8401 & 101085 & 2004 & 345.40 & 0.24 & 36520.00 & 353616.34 & 0.95 & 1.02 & 0.97 \\
37517 & 106944 & 2004 & 41.30 & 0.09 & 4180.00 & 37568.04 & 0.99 & 0.91 & 0.90 \\
16287 & 102121 & 2004 & 251.30 & 0.16 & 29796.00 & 255003.59 & 0.84 & 1.01 & 0.86 \\
42817 & 109020 & 2004 & 368.70 & 0.07 & 33340.00 & 372952.89 & 1.11 & 1.01 & 1.12 \\
34501 & 106248 & 2004 & 1134.10 & 0.21 & 90490.00 & 758975.75 & 1.25 & 0.67 & 0.84 \\
600 & 100079 & 2004 & 2251.30 & 0.11 & 254319.00 & 2110232.02 & 0.89 & 0.94 & 0.83 \\
42321 & 108951 & 2004 & 126.80 & 0.16 & 14202.00 & 123690.64 & 0.89 & 0.98 & 0.87 \\
50801 & 240451 & 2004 & 283.60 & 0.03 & 28437.00 & 282623.79 & 1.00 & 1.00 & 0.99 \\
34475 & 106244 & 2004 & 1.70 & -0.03 & 166.00 & 1420.40 & 1.02 & 0.84 & 0.86 \\
49776 & 240359 & 2004 & 13.60 & 0.08 & 1614.00 & 13187.84 & 0.84 & 0.97 & 0.82 \\
27434 & 105278 & 2004 & 590.90 & 0.08 & 56769.00 & 581212.73 & 1.04 & 0.98 & 1.02 \\
29707 & 105640 & 2004 & 2045.20 & 0.32 & 165798.00 & 1613673.38 & 1.23 & 0.79 & 0.97 \\
42296 & 108950 & 2004 & 771.00 & 0.27 & 61525.00 & 663933.18 & 1.25 & 0.86 & 1.08 \\
15762 & 102017 & 2004 & 6110.70 & 0.11 & 586567.00 & 5821284.91 & 1.04 & 0.95 & 0.99 \\
37545 & 106962 & 2004 & 4.40 & 0.09 & 408.00 & 4007.77 & 1.08 & 0.91 & 0.98 \\
37538 & 106961 & 2004 & 67.90 & 0.01 & 7099.00 & 70560.03 & 0.96 & 1.04 & 0.99 \\
55207 & 400072 & 2004 & 302.00 & 0.13 & 32843.00 & 311846.64 & 0.92 & 1.03 & 0.95 \\
13418 & 101738 & 2004 & 1834.80 & 0.21 & 183581.00 & 1735025.64 & 1.00 & 0.95 & 0.95 \\
55496 & 400113 & 2004 & 61.00 & 0.12 & 6335.00 & 57888.22 & 0.96 & 0.95 & 0.91 \\
46807 & 200300 & 2004 & 6.60 & 0.07 & 667.00 & 6609.63 & 0.99 & 1.00 & 0.99 \\
35739 & 106394 & 2004 & 61.80 & 0.12 & 5951.00 & 59806.18 & 1.04 & 0.97 & 1.00 \\
74650 & 601147 & 2004 & 169.40 & 0.20 & 14871.00 & 146791.39 & 1.14 & 0.87 & 0.99 \\
50949 & 240470 & 2004 & 48.90 & 0.18 & 4217.00 & 45914.96 & 1.16 & 0.94 & 1.09 \\
46510 & 200249 & 2004 & 315.20 & 0.01 & 32597.00 & 306126.64 & 0.97 & 0.97 & 0.94 \\
53575 & 354018 & 2004 & 121.10 & 0.10 & 12093.00 & 118678.69 & 1.00 & 0.98 & 0.98 \\
43960 & 109238 & 2004 & 20.80 & 0.13 & 2136.00 & 18397.85 & 0.97 & 0.88 & 0.86 \\
50823 & 240453 & 2004 & 106.10 & 0.33 & 10450.00 & 98690.18 & 1.02 & 0.93 & 0.94 \\
13844 & 101769 & 2004 & 885.90 & 0.04 & 92223.00 & 900747.91 & 0.96 & 1.02 & 0.98 \\
23106 & 103122 & 2004 & 221.00 & 0.06 & 21585.00 & 213337.81 & 1.02 & 0.97 & 0.99 \\
21773 & 102951 & 2004 & 5194.90 & 0.09 & 519650.00 & 5070325.59 & 1.00 & 0.98 & 0.98 \\
45944 & 200171 & 2004 & 112.70 & 0.11 & 10402.00 & 111161.08 & 1.08 & 0.99 & 1.07 \\
41214 & 108673 & 2004 & 203.20 & 0.17 & 21313.00 & 192032.06 & 0.95 & 0.95 & 0.90 \\
50972 & 240474 & 2004 & 31.40 & 0.18 & 2623.00 & 23383.75 & 1.20 & 0.74 & 0.89 \\
50993 & 240475 & 2004 & 16.70 & 0.15 & 1471.00 & 16008.74 & 1.14 & 0.96 & 1.09 \\
29443 & 105595 & 2004 & 15.80 & 0.04 & 1466.00 & 15420.15 & 1.08 & 0.98 & 1.05 \\
12829 & 101601 & 2004 & 174.50 & 0.19 & 15861.00 & 176111.86 & 1.10 & 1.01 & 1.11 \\
52683 & 306482 & 2004 & 112.90 & 0.23 & 11259.00 & 94855.15 & 1.00 & 0.84 & 0.84 \\
19371 & 102599 & 2004 & 2145.90 & 0.21 & 194848.00 & 2033407.04 & 1.10 & 0.95 & 1.04 \\
42191 & 108933 & 2004 & 218.60 & 0.06 & 20180.00 & 208158.31 & 1.08 & 0.95 & 1.03 \\
11510 & 101425 & 2004 & 20.50 & 0.05 & 2051.00 & 18199.15 & 1.00 & 0.89 & 0.89 \\
59170 & 410443 & 2004 & 24.10 & 0.10 & 2159.00 & 23325.25 & 1.12 & 0.97 & 1.08 \\
48613 & 240114 & 2004 & 324.20 & 0.10 & 32617.00 & 305599.37 & 0.99 & 0.94 & 0.94 \\
52691 & 306690 & 2004 & 144.60 & 0.23 & 10499.00 & 116749.18 & 1.38 & 0.81 & 1.11 \\
50761 & 240448 & 2004 & 83.50 & 0.00 & 8341.00 & 79585.60 & 1.00 & 0.95 & 0.95 \\
2422 & 100323 & 2004 & 2598.60 & 0.04 & 260893.00 & 2675733.86 & 1.00 & 1.03 & 1.03 \\
56480 & 400216 & 2004 & 172.70 & 0.12 & 14451.00 & 128926.13 & 1.20 & 0.75 & 0.89 \\
27509 & 105283 & 2004 & 7.80 & 0.06 & 743.00 & 6397.83 & 1.05 & 0.82 & 0.86 \\
46483 & 200247 & 2004 & 1.50 & -0.05 & 130.00 & 1343.27 & 1.15 & 0.90 & 1.03 \\
53566 & 351891 & 2004 & 30.60 & 0.10 & 3021.00 & 29436.18 & 1.01 & 0.96 & 0.97 \\
37482 & 106931 & 2004 & 484.40 & -0.05 & 48478.00 & 480746.59 & 1.00 & 0.99 & 0.99 \\
46802 & 200298 & 2004 & 100.50 & 0.11 & 10880.00 & 106136.68 & 0.92 & 1.06 & 0.98 \\
54440 & 367842 & 2004 & 169.40 & 0.18 & 16470.00 & 168014.29 & 1.03 & 0.99 & 1.02 \\
6925 & 100969 & 2004 & 62.40 & 0.05 & 5877.00 & 60388.57 & 1.06 & 0.97 & 1.03 \\
31044 & 105852 & 2004 & 120.40 & 0.17 & 8840.00 & 76964.18 & 1.36 & 0.64 & 0.87 \\
42197 & 108934 & 2004 & 106.70 & 0.09 & 9518.00 & 105784.86 & 1.12 & 0.99 & 1.11 \\
2402 & 100322 & 2004 & 406.60 & 0.16 & 37928.00 & 387745.70 & 1.07 & 0.95 & 1.02 \\
17044 & 102231 & 2004 & 490.90 & 0.05 & 49967.00 & 478461.20 & 0.98 & 0.97 & 0.96 \\
42801 & 109019 & 2004 & 67.10 & 0.15 & 6054.00 & 58801.80 & 1.11 & 0.88 & 0.97 \\
35523 & 106370 & 2004 & 66.00 & 0.11 & 6115.00 & 59289.72 & 1.08 & 0.90 & 0.97 \\
46780 & 200297 & 2004 & 720.40 & 0.26 & 68103.00 & 681003.90 & 1.06 & 0.95 & 1.00 \\
49785 & 240360 & 2004 & 1148.70 & 0.12 & 106679.00 & 913585.66 & 1.08 & 0.80 & 0.86 \\
42793 & 109018 & 2004 & 18.90 & 0.10 & 1696.00 & 18835.23 & 1.11 & 1.00 & 1.11 \\
23078 & 103110 & 2004 & 1697.70 & 0.15 & 139858.00 & 1339563.16 & 1.21 & 0.79 & 0.96 \\
46487 & 200248 & 2004 & 121.40 & 0.15 & 10404.00 & 104425.31 & 1.17 & 0.86 & 1.00 \\
27479 & 105281 & 2004 & 594.80 & 0.33 & 50511.00 & 539682.65 & 1.18 & 0.91 & 1.07 \\
21641 & 102937 & 2004 & 103.00 & 0.25 & 10346.00 & 89856.03 & 1.00 & 0.87 & 0.87 \\
10133 & 101262 & 2004 & 15.80 & 0.07 & 1590.00 & 15338.35 & 0.99 & 0.97 & 0.96 \\
37491 & 106934 & 2004 & 1978.50 & -0.00 & 198167.00 & 1768245.11 & 1.00 & 0.89 & 0.89 \\
44652 & 109351 & 2004 & 54.00 & 0.06 & 7120.00 & 71204.43 & 0.76 & 1.32 & 1.00 \\
35764 & 106401 & 2004 & 2252.00 & 0.22 & 177651.00 & 1872212.65 & 1.27 & 0.83 & 1.05 \\
74877 & 601188 & 2004 & 62.10 & 0.21 & 3721.00 & 35682.06 & 1.67 & 0.57 & 0.96 \\
54846 & 400018 & 2004 & 179.40 & 0.12 & 16939.00 & 169618.00 & 1.06 & 0.95 & 1.00 \\
7562 & 101045 & 2004 & 12915.40 & 0.10 & 1288933.00 & 12655404.37 & 1.00 & 0.98 & 0.98 \\
46107 & 200190 & 2004 & 161.90 & 0.21 & 16020.00 & 134239.22 & 1.01 & 0.83 & 0.84 \\
8466 & 101087 & 2004 & 436.30 & 0.13 & 41232.00 & 368609.72 & 1.06 & 0.84 & 0.89 \\
42029 & 108910 & 2004 & 11.00 & 0.05 & 1097.00 & 10667.74 & 1.00 & 0.97 & 0.97 \\
6839 & 100962 & 2004 & 9576.30 & 0.14 & 869963.00 & 8613195.01 & 1.10 & 0.90 & 0.99 \\
44472 & 109326 & 2004 & 4.90 & 0.09 & NaN & 4422.29 & 1.00 & 0.90 & 1.00 \\
56765 & 400252 & 2004 & 38.40 & 0.10 & 3764.00 & 36516.68 & 1.02 & 0.95 & 0.97 \\
28989 & 105510 & 2004 & 14.30 & 0.04 & 1451.00 & 13275.65 & 0.99 & 0.93 & 0.91 \\
2756 & 100357 & 2004 & 196.10 & 0.24 & 15581.00 & 154465.85 & 1.26 & 0.79 & 0.99 \\
30360 & 105741 & 2004 & 260.30 & 0.08 & 24634.00 & 236794.56 & 1.06 & 0.91 & 0.96 \\
63317 & 500494 & 2004 & 1686.30 & 0.23 & 127863.00 & 1411531.61 & 1.32 & 0.84 & 1.10 \\
18950 & 102529 & 2004 & 14.50 & 0.22 & 1452.00 & 13218.54 & 1.00 & 0.91 & 0.91 \\
28543 & 105437 & 2004 & 1012.30 & 0.09 & 101310.00 & 1000759.25 & 1.00 & 0.99 & 0.99 \\
16446 & 102145 & 2004 & 324.30 & 0.07 & 30963.00 & 302152.14 & 1.05 & 0.93 & 0.98 \\
63294 & 500493 & 2004 & 1235.60 & 0.05 & 93661.00 & 1039357.58 & 1.32 & 0.84 & 1.11 \\
53021 & 337150 & 2004 & 58.40 & 0.13 & 5843.00 & 58428.54 & 1.00 & 1.00 & 1.00 \\
44449 & 109325 & 2004 & 608.00 & 0.06 & 60719.00 & 607106.04 & 1.00 & 1.00 & 1.00 \\
8571 & 101090 & 2004 & 737.50 & -0.02 & 74570.00 & 718530.55 & 0.99 & 0.97 & 0.96 \\
52953 & 335933 & 2004 & 33.80 & 0.06 & 3379.00 & 32595.10 & 1.00 & 0.96 & 0.96 \\
59245 & 410447 & 2004 & 73.40 & 0.19 & 5954.00 & 62892.77 & 1.23 & 0.86 & 1.06 \\
3261 & 100419 & 2004 & 646.50 & 0.00 & 65425.00 & 632606.19 & 0.99 & 0.98 & 0.97 \\
44476 & 109327 & 2004 & 651.40 & 0.19 & 62771.00 & 628680.21 & 1.04 & 0.97 & 1.00 \\
11126 & 101368 & 2004 & 1299.10 & 0.12 & 119417.00 & 1244956.52 & 1.09 & 0.96 & 1.04 \\
16631 & 102166 & 2004 & 99.20 & 0.13 & 9594.00 & 87979.72 & 1.03 & 0.89 & 0.92 \\
41729 & 108849 & 2004 & 1296.70 & 0.23 & 112808.00 & 1218333.14 & 1.15 & 0.94 & 1.08 \\
22051 & 102988 & 2004 & 53.90 & 0.23 & 5392.00 & 52788.24 & 1.00 & 0.98 & 0.98 \\
30377 & 105746 & 2004 & 361.30 & 0.14 & 32961.00 & 317113.61 & 1.10 & 0.88 & 0.96 \\
36539 & 106560 & 2004 & 99.90 & 0.11 & 9188.00 & 95766.33 & 1.09 & 0.96 & 1.04 \\
42546 & 108977 & 2004 & 359.70 & 0.14 & 40145.00 & 348493.03 & 0.90 & 0.97 & 0.87 \\
53038 & 337653 & 2004 & 121.20 & 0.03 & 10666.00 & 118405.75 & 1.14 & 0.98 & 1.11 \\
59144 & 410439 & 2004 & 16.30 & 0.09 & 1404.00 & 16824.12 & 1.16 & 1.03 & 1.20 \\
52920 & 335108 & 2004 & 179.10 & 0.02 & 17306.00 & 167620.28 & 1.03 & 0.94 & 0.97 \\
41754 & 108852 & 2004 & 220.00 & 0.16 & 20195.00 & 203140.46 & 1.09 & 0.92 & 1.01 \\
22381 & 103008 & 2004 & 180.80 & 0.11 & 18075.00 & 180741.55 & 1.00 & 1.00 & 1.00 \\
28521 & 105432 & 2004 & 29.30 & 0.11 & 2772.00 & 25659.63 & 1.06 & 0.88 & 0.93 \\
52932 & 335811 & 2004 & 902.00 & 0.10 & 88957.00 & 876242.12 & 1.01 & 0.97 & 0.99 \\
35282 & 106344 & 2004 & 1022.20 & 0.11 & 89258.00 & 899968.22 & 1.15 & 0.88 & 1.01 \\
55610 & 400130 & 2004 & 14.60 & 0.03 & 1144.00 & 12397.88 & 1.28 & 0.85 & 1.08 \\
42387 & 108960 & 2004 & 51.70 & 0.12 & 4429.00 & 49370.96 & 1.17 & 0.95 & 1.11 \\
62141 & 500348 & 2004 & 230.00 & 0.19 & 22985.00 & 229166.72 & 1.00 & 1.00 & 1.00 \\
50499 & 240425 & 2004 & 6.90 & 0.17 & 1098.00 & 8828.04 & 0.63 & 1.28 & 0.80 \\
51241 & 240493 & 2004 & 89.40 & 0.21 & 8938.00 & 88062.67 & 1.00 & 0.99 & 0.99 \\
74779 & 601168 & 2004 & 32.80 & 0.07 & 3313.00 & 30706.17 & 0.99 & 0.94 & 0.93 \\
13018 & 101621 & 2004 & 2088.10 & 0.06 & 219334.00 & 1977302.77 & 0.95 & 0.95 & 0.90 \\
3245 & 100417 & 2004 & 13.90 & 0.01 & 1396.00 & 13462.93 & 1.00 & 0.97 & 0.96 \\
16081 & 102079 & 2004 & 418.50 & 0.12 & 39837.00 & 392226.06 & 1.05 & 0.94 & 0.98 \\
28589 & 105448 & 2004 & 1161.20 & 0.36 & 96400.00 & 940859.12 & 1.20 & 0.81 & 0.98 \\
63255 & 500491 & 2004 & 1139.80 & 0.10 & 99980.00 & 1065408.17 & 1.14 & 0.93 & 1.07 \\
8634 & 101092 & 2004 & 433.80 & 0.13 & 42433.00 & 409372.36 & 1.02 & 0.94 & 0.96 \\
30327 & 105737 & 2004 & 5.40 & 0.14 & 539.00 & 5159.59 & 1.00 & 0.96 & 0.96 \\
36513 & 106545 & 2004 & 22.20 & 0.13 & 1967.00 & 20734.65 & 1.13 & 0.93 & 1.05 \\
41787 & 108856 & 2004 & 393.60 & 0.16 & 39451.00 & 383907.53 & 1.00 & 0.98 & 0.97 \\
50285 & 240409 & 2004 & 58.50 & 0.42 & 3100.00 & 27396.29 & 1.89 & 0.47 & 0.88 \\
51416 & 240517 & 2004 & 37.40 & 0.29 & 3727.00 & 34406.49 & 1.00 & 0.92 & 0.92 \\
48246 & 240056 & 2004 & 72.80 & 0.13 & 6552.00 & 54497.29 & 1.11 & 0.75 & 0.83 \\
36501 & 106541 & 2004 & 591.60 & 0.20 & 58935.00 & 579232.23 & 1.00 & 0.98 & 0.98 \\
22079 & 102989 & 2004 & 1900.10 & 0.10 & 189835.00 & 1892588.11 & 1.00 & 1.00 & 1.00 \\
41809 & 108857 & 2004 & 48.30 & 0.29 & 4790.00 & 46800.45 & 1.01 & 0.97 & 0.98 \\
35079 & 106318 & 2004 & 102.70 & 0.17 & 10275.00 & 100914.07 & 1.00 & 0.98 & 0.98 \\
11092 & 101367 & 2004 & 703.70 & 0.22 & 62122.00 & 680302.19 & 1.13 & 0.97 & 1.10 \\
46280 & 200207 & 2004 & 33.10 & 0.11 & 3175.00 & 32030.01 & 1.04 & 0.97 & 1.01 \\
50261 & 240407 & 2004 & 135.20 & 0.14 & 7135.00 & 58842.64 & 1.89 & 0.44 & 0.82 \\
35070 & 106317 & 2004 & 171.40 & 0.04 & 18298.00 & 165982.07 & 0.94 & 0.97 & 0.91 \\
22337 & 103007 & 2004 & 1126.90 & 0.09 & 110916.00 & 1074327.61 & 1.02 & 0.95 & 0.97 \\
28963 & 105508 & 2004 & 84.50 & 0.01 & 7530.00 & 85177.57 & 1.12 & 1.01 & 1.13 \\
42009 & 108907 & 2004 & 923.10 & 0.13 & 80429.00 & 841606.42 & 1.15 & 0.91 & 1.05 \\
4217 & 100575 & 2004 & 8.80 & 0.09 & 884.00 & 8788.85 & 1.00 & 1.00 & 0.99 \\
41768 & 108855 & 2004 & 607.70 & 0.27 & 83123.00 & 767982.05 & 0.73 & 1.26 & 0.92 \\
163 & 100016 & 2004 & 115.90 & 0.11 & 11140.00 & 117549.45 & 1.04 & 1.01 & 1.06 \\
394 & 100048 & 2004 & 115.00 & 0.11 & 10446.00 & 109323.12 & 1.10 & 0.95 & 1.05 \\
50474 & 240422 & 2004 & 42.50 & 0.10 & 3976.00 & 42165.20 & 1.07 & 0.99 & 1.06 \\
199 & 100018 & 2004 & 167.10 & 0.19 & 15244.00 & 164473.14 & 1.10 & 0.98 & 1.08 \\
5714 & 100789 & 2004 & 5.70 & -0.02 & 488.00 & 4510.85 & 1.17 & 0.79 & 0.92 \\
51248 & 240495 & 2004 & 1261.20 & 0.04 & 110193.00 & 1061877.75 & 1.14 & 0.84 & 0.96 \\
28573 & 105444 & 2004 & 84.60 & 0.07 & 8694.00 & 78225.35 & 0.97 & 0.92 & 0.90 \\
36197 & 106477 & 2004 & 2774.20 & 0.12 & 268095.00 & 2623172.40 & 1.03 & 0.95 & 0.98 \\
19149 & 102551 & 2004 & 3081.50 & 0.23 & 270321.00 & 2780910.07 & 1.14 & 0.90 & 1.03 \\
16595 & 102157 & 2004 & 2.90 & -0.07 & 283.00 & 2464.51 & 1.02 & 0.85 & 0.87 \\
13808 & 101764 & 2004 & 597.10 & 0.15 & 63400.00 & 650886.48 & 0.94 & 1.09 & 1.03 \\
10422 & 101285 & 2004 & 772.20 & 0.07 & 73438.00 & 756107.32 & 1.05 & 0.98 & 1.03 \\
14677 & 101908 & 2004 & 293.70 & 0.04 & 28509.00 & 280959.14 & 1.03 & 0.96 & 0.99 \\
36558 & 106561 & 2004 & 19.50 & 0.26 & 1577.00 & 17707.73 & 1.24 & 0.91 & 1.12 \\
52899 & 333181 & 2004 & 51.20 & 0.03 & 5134.00 & 50285.15 & 1.00 & 0.98 & 0.98 \\
46400 & 200236 & 2004 & 43.90 & 0.12 & 3824.00 & 35579.28 & 1.15 & 0.81 & 0.93 \\
30435 & 105758 & 2004 & 123.00 & 0.12 & 13813.00 & 123993.42 & 0.89 & 1.01 & 0.90 \\
36631 & 106571 & 2004 & 220.70 & 0.17 & 22579.00 & 195367.03 & 0.98 & 0.89 & 0.87 \\
59075 & 410423 & 2004 & 5.00 & 0.11 & 297.00 & 3071.80 & 1.68 & 0.61 & 1.03 \\
42591 & 108984 & 2004 & 97.60 & 0.09 & 9298.00 & 91810.21 & 1.05 & 0.94 & 0.99 \\
56596 & 400230 & 2004 & 41.50 & -0.06 & 4163.00 & 37921.49 & 1.00 & 0.91 & 0.91 \\
53310 & 343540 & 2004 & 23.60 & 0.03 & 1511.00 & 14906.23 & 1.56 & 0.63 & 0.99 \\
52879 & 333058 & 2004 & 764.90 & 0.13 & 76422.00 & 764206.80 & 1.00 & 1.00 & 1.00 \\
16662 & 102175 & 2004 & 440.10 & 0.14 & 43957.00 & 418326.45 & 1.00 & 0.95 & 0.95 \\
54575 & 375967 & 2004 & 96.00 & 0.10 & 9391.00 & 99978.56 & 1.02 & 1.04 & 1.06 \\
28434 & 105424 & 2004 & 4120.30 & 0.07 & 412035.00 & 4026164.94 & 1.00 & 0.98 & 0.98 \\
30425 & 105757 & 2004 & 68.70 & -0.00 & 9823.00 & 83747.95 & 0.70 & 1.22 & 0.85 \\
44337 & 109286 & 2004 & 346.20 & 0.08 & 32637.00 & 337093.47 & 1.06 & 0.97 & 1.03 \\
74814 & 601179 & 2004 & 66.40 & 0.13 & 5960.00 & 61773.05 & 1.11 & 0.93 & 1.04 \\
48807 & 240143 & 2004 & 177.60 & 0.17 & 15979.00 & 159232.07 & 1.11 & 0.90 & 1.00 \\
56599 & 400231 & 2004 & 148.40 & 0.03 & 14881.00 & 147544.49 & 1.00 & 0.99 & 0.99 \\
10387 & 101284 & 2004 & 2134.90 & 0.12 & 199945.00 & 1977745.48 & 1.07 & 0.93 & 0.99 \\
56790 & 400255 & 2004 & 26.50 & 0.06 & 2657.00 & 23019.44 & 1.00 & 0.87 & 0.87 \\
22020 & 102987 & 2004 & 1135.30 & 0.09 & 113718.00 & 1090489.37 & 1.00 & 0.96 & 0.96 \\
55601 & 400128 & 2004 & 95.70 & 0.22 & 9503.00 & 93653.86 & 1.01 & 0.98 & 0.99 \\
46624 & 200266 & 2004 & 7.40 & 0.11 & 740.00 & 6982.81 & 1.00 & 0.94 & 0.94 \\
36133 & 106467 & 2004 & 219.80 & 0.11 & 19152.00 & 206793.83 & 1.15 & 0.94 & 1.08 \\
62343 & 500387 & 2004 & 26.80 & 0.03 & 2675.00 & 26746.32 & 1.00 & 1.00 & 1.00 \\
8763 & 101097 & 2004 & 376.90 & 0.18 & 37729.00 & 348865.45 & 1.00 & 0.93 & 0.92 \\
30022 & 105678 & 2004 & 47.80 & 0.10 & 4458.00 & 46260.44 & 1.07 & 0.97 & 1.04 \\
50153 & 240396 & 2004 & 21.80 & 0.10 & 2187.00 & 20376.25 & 1.00 & 0.93 & 0.93 \\
36142 & 106470 & 2004 & 248.50 & 0.03 & 29072.00 & 288053.01 & 0.85 & 1.16 & 0.99 \\
35024 & 106306 & 2004 & 17.60 & 0.04 & 1748.00 & 16473.15 & 1.01 & 0.94 & 0.94 \\
35324 & 106347 & 2004 & 99.30 & 0.15 & 8477.00 & 86780.32 & 1.17 & 0.87 & 1.02 \\
41705 & 108840 & 2004 & 360.00 & 0.14 & 28801.00 & 282032.49 & 1.25 & 0.78 & 0.98 \\
30445 & 105760 & 2004 & 813.50 & 0.08 & 93922.00 & 740155.14 & 0.87 & 0.91 & 0.79 \\
28405 & 105421 & 2004 & 24.70 & 0.20 & 2486.00 & 25407.74 & 0.99 & 1.03 & 1.02 \\
29042 & 105522 & 2004 & 444.80 & 0.17 & 43369.00 & 413313.08 & 1.03 & 0.93 & 0.95 \\
21257 & 102843 & 2004 & 739.70 & 0.18 & 57005.00 & 625107.43 & 1.30 & 0.85 & 1.10 \\
50157 & 240397 & 2004 & 14.10 & 0.11 & 1410.00 & 13701.68 & 1.00 & 0.97 & 0.97 \\
44049 & 109263 & 2004 & 17.70 & 0.15 & 1252.00 & 11779.98 & 1.41 & 0.67 & 0.94 \\
36657 & 106573 & 2004 & 8.30 & 0.14 & 830.00 & 7665.75 & 1.00 & 0.92 & 0.92 \\
46271 & 200205 & 2004 & 489.50 & 0.27 & 36634.00 & 342481.31 & 1.34 & 0.70 & 0.93 \\
42420 & 108964 & 2004 & 881.40 & 0.11 & 87646.00 & 864987.95 & 1.01 & 0.98 & 0.99 \\
53073 & 338387 & 2004 & 5008.60 & 0.15 & 500618.00 & 5002211.10 & 1.00 & 1.00 & 1.00 \\
42066 & 108915 & 2004 & 39.60 & -0.02 & 5414.00 & 52709.36 & 0.73 & 1.33 & 0.97 \\
59148 & 410442 & 2004 & 107.70 & 0.17 & 8614.00 & 90942.49 & 1.25 & 0.84 & 1.06 \\
55606 & 400129 & 2004 & 298.00 & 0.10 & 24724.00 & 261949.52 & 1.21 & 0.88 & 1.06 \\
19226 & 102570 & 2004 & 180.40 & 0.08 & 18047.00 & 188308.43 & 1.00 & 1.04 & 1.04 \\
44499 & 109330 & 2004 & 21.20 & 0.03 & 2127.00 & 20365.38 & 1.00 & 0.96 & 0.96 \\
421 & 100055 & 2004 & 19500.80 & 0.12 & 1824243.00 & 18173305.06 & 1.07 & 0.93 & 1.00 \\
36161 & 106474 & 2004 & 238.30 & 0.10 & 24188.00 & 247215.73 & 0.99 & 1.04 & 1.02 \\
50186 & 240401 & 2004 & 37.70 & -0.10 & 3635.00 & 33719.14 & 1.04 & 0.89 & 0.93 \\
56770 & 400254 & 2004 & 28.80 & 0.02 & 2878.00 & 26221.60 & 1.00 & 0.91 & 0.91 \\
30050 & 105679 & 2004 & 446.80 & 0.10 & 46591.00 & 453464.66 & 0.96 & 1.01 & 0.97 \\
51423 & 240519 & 2004 & 18.00 & 0.09 & 1614.00 & 18018.59 & 1.12 & 1.00 & 1.12 \\
29007 & 105512 & 2004 & 12.60 & 0.12 & 1265.00 & 11831.69 & 1.00 & 0.94 & 0.94 \\
28492 & 105427 & 2004 & 192.40 & 0.07 & 19244.00 & 180567.65 & 1.00 & 0.94 & 0.94 \\
21281 & 102844 & 2004 & 547.00 & 0.09 & 52531.00 & 531203.23 & 1.04 & 0.97 & 1.01 \\
15986 & 102062 & 2004 & 1632.10 & 0.14 & 128383.00 & 1400472.44 & 1.27 & 0.86 & 1.09 \\
18963 & 102531 & 2004 & 26.30 & 0.13 & 2641.00 & 26411.74 & 1.00 & 1.00 & 1.00 \\
62294 & 500377 & 2004 & 1.90 & 0.04 & 147.00 & 1529.98 & 1.29 & 0.81 & 1.04 \\
53059 & 337871 & 2004 & 206.00 & 0.17 & 16652.00 & 177302.65 & 1.24 & 0.86 & 1.06 \\
51451 & 240521 & 2004 & 8.40 & 0.05 & 843.00 & 8030.35 & 1.00 & 0.96 & 0.95 \\
36596 & 106568 & 2004 & 3663.10 & 0.20 & 363897.00 & 3481230.60 & 1.01 & 0.95 & 0.96 \\
2725 & 100355 & 2004 & 5275.00 & 0.13 & 488285.00 & 5298373.80 & 1.08 & 1.00 & 1.09 \\
10930 & 101354 & 2004 & 969.00 & 0.07 & 91015.00 & 959137.91 & 1.06 & 0.99 & 1.05 \\
53410 & 348766 & 2004 & 848.70 & 0.09 & 71528.00 & 766446.93 & 1.19 & 0.90 & 1.07 \\
5682 & 100785 & 2004 & 1642.70 & 0.10 & 166627.00 & 1666151.60 & 0.99 & 1.01 & 1.00 \\
48216 & 240051 & 2004 & 690.30 & 0.16 & 70753.00 & 665560.12 & 0.98 & 0.96 & 0.94 \\
62128 & 500340 & 2004 & 46.80 & 0.10 & 4688.00 & 46466.50 & 1.00 & 0.99 & 0.99 \\
18822 & 102523 & 2004 & 1995.70 & 0.20 & 190859.00 & 1692832.70 & 1.05 & 0.85 & 0.89 \\
42042 & 108914 & 2004 & 1284.00 & 0.31 & 102510.00 & 1142375.33 & 1.25 & 0.89 & 1.11 \\
28463 & 105426 & 2004 & 1075.80 & 0.23 & 107523.00 & 1025556.90 & 1.00 & 0.95 & 0.95 \\
35037 & 106309 & 2004 & 1805.00 & 0.25 & 157018.00 & 1612298.82 & 1.15 & 0.89 & 1.03 \\
36153 & 106471 & 2004 & 125.50 & 0.10 & 12416.00 & 125439.48 & 1.01 & 1.00 & 1.01 \\
36589 & 106567 & 2004 & 5.90 & 0.17 & 607.00 & 5850.39 & 0.97 & 0.99 & 0.96 \\
54579 & 376139 & 2004 & 15.30 & 0.16 & 1531.00 & 14630.93 & 1.00 & 0.96 & 0.96 \\
35309 & 106345 & 2004 & 649.60 & 0.03 & 65058.00 & 645061.87 & 1.00 & 0.99 & 0.99 \\
4360 & 100611 & 2004 & 413.60 & 0.21 & 41087.00 & 387216.81 & 1.01 & 0.94 & 0.94 \\
18853 & 102524 & 2004 & 3326.70 & 0.10 & 332236.00 & 3307471.47 & 1.00 & 0.99 & 1.00 \\
36220 & 106478 & 2004 & 496.00 & 0.33 & 41887.00 & 441617.01 & 1.18 & 0.89 & 1.05 \\
46337 & 200223 & 2004 & 76.20 & -0.02 & 7783.00 & 67560.77 & 0.98 & 0.89 & 0.87 \\
10482 & 101287 & 2004 & 925.00 & 0.10 & 86203.00 & 745032.65 & 1.07 & 0.81 & 0.86 \\
343 & 100040 & 2004 & 1545.20 & -0.03 & 154627.00 & 1545060.89 & 1.00 & 1.00 & 1.00 \\
11028 & 101360 & 2004 & 2218.50 & 0.16 & 200036.00 & 1962234.12 & 1.11 & 0.88 & 0.98 \\
16521 & 102152 & 2004 & 270.00 & 0.07 & 26997.00 & 257557.42 & 1.00 & 0.95 & 0.95 \\
53330 & 344017 & 2004 & 6.20 & 0.00 & 533.00 & 4385.67 & 1.16 & 0.71 & 0.82 \\
35251 & 106336 & 2004 & 45.80 & 0.18 & 4589.00 & 40891.21 & 1.00 & 0.89 & 0.89 \\
41887 & 108867 & 2004 & 473.40 & 0.07 & 47672.00 & 462118.43 & 0.99 & 0.98 & 0.97 \\
28780 & 105476 & 2004 & 606.00 & 0.32 & 60276.00 & 600619.76 & 1.01 & 0.99 & 1.00 \\
42455 & 108968 & 2004 & 51.50 & 0.05 & 5161.00 & 49993.24 & 1.00 & 0.97 & 0.97 \\
51351 & 240505 & 2004 & 90.50 & 0.09 & 9309.00 & 89310.59 & 0.97 & 0.99 & 0.96 \\
21398 & 102861 & 2004 & 44.20 & 0.02 & 4422.00 & 44198.00 & 1.00 & 1.00 & 1.00 \\
28876 & 105498 & 2004 & 60.30 & 0.11 & 6259.00 & 60623.73 & 0.96 & 1.01 & 0.97 \\
51345 & 240504 & 2004 & 188.40 & 0.05 & 18909.00 & 185863.67 & 1.00 & 0.99 & 0.98 \\
36368 & 106519 & 2004 & 356.80 & 0.11 & 35691.00 & 337585.71 & 1.00 & 0.95 & 0.95 \\
35189 & 106333 & 2004 & 178.40 & 0.16 & 15578.00 & 170450.16 & 1.15 & 0.96 & 1.09 \\
51338 & 240502 & 2004 & 65.20 & 0.04 & 5934.00 & 59006.24 & 1.10 & 0.91 & 0.99 \\
53374 & 344889 & 2004 & 38.80 & 0.04 & 3867.00 & 38322.11 & 1.00 & 0.99 & 0.99 \\
59234 & 410446 & 2004 & 30.30 & 0.14 & 2616.00 & 28075.06 & 1.16 & 0.93 & 1.07 \\
63232 & 500490 & 2004 & 1571.50 & 0.33 & 146849.00 & 1515574.13 & 1.07 & 0.96 & 1.03 \\
50402 & 240415 & 2004 & 67.50 & 0.16 & 6819.00 & 58052.38 & 0.99 & 0.86 & 0.85 \\
48857 & 240148 & 2004 & 506.80 & 0.14 & 47563.00 & 463645.03 & 1.07 & 0.91 & 0.97 \\
52962 & 336226 & 2004 & 105.60 & 0.14 & 8074.00 & 83412.75 & 1.31 & 0.79 & 1.03 \\
48882 & 240149 & 2004 & 204.60 & 0.24 & 20297.00 & 202776.11 & 1.01 & 0.99 & 1.00 \\
28751 & 105475 & 2004 & 356.60 & 0.26 & 35637.00 & 343546.78 & 1.00 & 0.96 & 0.96 \\
46353 & 200225 & 2004 & 10.10 & 0.11 & 1023.00 & 9946.43 & 0.99 & 0.98 & 0.97 \\
44402 & 109300 & 2004 & 463.10 & 0.14 & 46277.00 & 433167.50 & 1.00 & 0.94 & 0.94 \\
30129 & 105701 & 2004 & 539.30 & 0.09 & 53602.00 & 558122.22 & 1.01 & 1.03 & 1.04 \\
36302 & 106482 & 2004 & 184.90 & 0.32 & 18339.00 & 178015.19 & 1.01 & 0.96 & 0.97 \\
28886 & 105502 & 2004 & 1457.00 & 0.10 & 136940.00 & 1250630.09 & 1.06 & 0.86 & 0.91 \\
2833 & 100362 & 2004 & 114.80 & 0.23 & 9434.00 & 94104.40 & 1.22 & 0.82 & 1.00 \\
30215 & 105716 & 2004 & 85.30 & -0.16 & 8523.00 & 83704.80 & 1.00 & 0.98 & 0.98 \\
41950 & 108874 & 2004 & 46.10 & 0.05 & 4430.00 & 42131.55 & 1.04 & 0.91 & 0.95 \\
7282 & 101018 & 2004 & 23578.10 & 0.06 & 2240630.00 & 22872541.42 & 1.05 & 0.97 & 1.02 \\
36397 & 106523 & 2004 & 266.10 & 0.23 & 24575.00 & 267310.63 & 1.08 & 1.00 & 1.09 \\
63209 & 500489 & 2004 & 1053.20 & -0.01 & 81120.00 & 845754.73 & 1.30 & 0.80 & 1.04 \\
30182 & 105705 & 2004 & 132.70 & 0.07 & 11828.00 & 134732.24 & 1.12 & 1.02 & 1.14 \\
44090 & 109265 & 2004 & 56.90 & 0.11 & 5396.00 & 51903.78 & 1.05 & 0.91 & 0.96 \\
41893 & 108868 & 2004 & 1015.90 & 0.25 & 101588.00 & 945827.75 & 1.00 & 0.93 & 0.93 \\
3217 & 100415 & 2004 & 207.50 & 0.13 & 17498.00 & 185491.46 & 1.19 & 0.89 & 1.06 \\
28858 & 105487 & 2004 & 106.00 & 0.10 & 13175.00 & 116146.07 & 0.80 & 1.10 & 0.88 \\
46345 & 200224 & 2004 & 26.40 & 0.17 & 2696.00 & 27963.16 & 0.98 & 1.06 & 1.04 \\
35224 & 106335 & 2004 & 124.70 & 0.15 & 12567.00 & 124111.18 & 0.99 & 1.00 & 0.99 \\
22146 & 102993 & 2004 & 7343.90 & 0.12 & 674424.00 & 5862950.14 & 1.09 & 0.80 & 0.87 \\
51313 & 240499 & 2004 & 166.30 & 0.31 & 12735.00 & 119873.71 & 1.31 & 0.72 & 0.94 \\
52983 & 336508 & 2004 & 147.90 & 0.11 & 11614.00 & 109805.65 & 1.27 & 0.74 & 0.95 \\
36328 & 106483 & 2004 & 16.30 & 0.15 & 1650.00 & 16135.79 & 0.99 & 0.99 & 0.98 \\
41923 & 108870 & 2004 & 46.00 & 0.11 & 4641.00 & 43493.24 & 0.99 & 0.95 & 0.94 \\
16499 & 102151 & 2004 & 7.20 & 0.00 & 724.00 & 7095.10 & 0.99 & 0.99 & 0.98 \\
53348 & 344278 & 2004 & 142.90 & 0.18 & 15564.00 & 135091.47 & 0.92 & 0.95 & 0.87 \\
44067 & 109264 & 2004 & 46.20 & -0.12 & 4466.00 & 40932.61 & 1.03 & 0.89 & 0.92 \\
13076 & 101626 & 2004 & 1057.30 & 0.13 & 102213.00 & 899866.47 & 1.03 & 0.85 & 0.88 \\
36338 & 106485 & 2004 & 156.10 & 0.10 & 15622.00 & 150118.92 & 1.00 & 0.96 & 0.96 \\
62198 & 500364 & 2004 & 137.30 & 0.10 & 14041.00 & 134778.25 & 0.98 & 0.98 & 0.96 \\
35216 & 106334 & 2004 & 33.10 & 0.23 & 3287.00 & 32373.12 & 1.01 & 0.98 & 0.98 \\
10988 & 101358 & 2004 & 453.80 & 0.13 & 42950.00 & 461952.99 & 1.06 & 1.02 & 1.08 \\
44370 & 109295 & 2004 & 322.70 & 0.22 & 30070.00 & 289539.18 & 1.07 & 0.90 & 0.96 \\
63189 & 500488 & 2004 & 707.70 & 0.13 & 66171.00 & 712508.10 & 1.07 & 1.01 & 1.08 \\
6026 & 100820 & 2004 & 593.80 & 0.07 & 59343.00 & 589825.66 & 1.00 & 0.99 & 0.99 \\
53336 & 344277 & 2004 & 599.00 & 0.12 & 59628.00 & 580345.11 & 1.00 & 0.97 & 0.97 \\
53985 & 362981 & 2004 & 259.50 & 0.13 & 24558.00 & 260372.29 & 1.06 & 1.00 & 1.06 \\
5756 & 100791 & 2004 & 6122.30 & 0.13 & 555800.00 & 5347755.98 & 1.10 & 0.87 & 0.96 \\
28811 & 105478 & 2004 & 59.20 & 0.08 & 5531.00 & 52149.53 & 1.07 & 0.88 & 0.94 \\
4308 & 100603 & 2004 & 2264.70 & 0.38 & 189294.00 & 2087870.54 & 1.20 & 0.92 & 1.10 \\
51334 & 240501 & 2004 & 19.50 & 0.25 & 1653.00 & 17831.49 & 1.18 & 0.91 & 1.08 \\
51319 & 240500 & 2004 & 358.60 & 0.13 & 36472.00 & 350911.33 & 0.98 & 0.98 & 0.96 \\
21426 & 102871 & 2004 & 659.00 & 0.15 & 62826.00 & 563648.81 & 1.05 & 0.86 & 0.90 \\
18915 & 102527 & 2004 & 95.70 & 0.11 & 9864.00 & 97240.28 & 0.97 & 1.02 & 0.99 \\
181 & 100017 & 2004 & 96.70 & 0.14 & 9378.00 & 97149.48 & 1.03 & 1.00 & 1.04 \\
50448 & 240419 & 2004 & 60.40 & 0.03 & 5965.00 & 58270.44 & 1.01 & 0.96 & 0.98 \\
74810 & 601178 & 2004 & 68.10 & 0.09 & 6036.00 & 63954.88 & 1.13 & 0.94 & 1.06 \\
51364 & 240507 & 2004 & 76.00 & 0.09 & 6976.00 & 63077.30 & 1.09 & 0.83 & 0.90 \\
10466 & 101286 & 2004 & 1806.90 & 0.06 & 166267.00 & 1669622.79 & 1.09 & 0.92 & 1.00 \\
28654 & 105458 & 2004 & 553.00 & 0.13 & 85166.00 & 882100.09 & 0.65 & 1.60 & 1.04 \\
35120 & 106321 & 2004 & 12.80 & 0.13 & 764.00 & 7466.85 & 1.68 & 0.58 & 0.98 \\
18931 & 102528 & 2004 & 90.50 & 0.15 & 9065.00 & 88900.39 & 1.00 & 0.98 & 0.98 \\
22280 & 102999 & 2004 & 111.30 & 0.12 & 11604.00 & 115317.56 & 0.96 & 1.04 & 0.99 \\
50329 & 240411 & 2004 & 188.50 & 0.20 & 14243.00 & 156800.47 & 1.32 & 0.83 & 1.10 \\
52957 & 336065 & 2004 & 26.10 & -0.14 & 2652.00 & 24536.32 & 0.98 & 0.94 & 0.93 \\
36243 & 106479 & 2004 & 119.90 & 0.16 & 10458.00 & 103995.85 & 1.15 & 0.87 & 0.99 \\
50351 & 240412 & 2004 & 209.40 & 0.16 & 12432.00 & 102512.33 & 1.68 & 0.49 & 0.82 \\
47585 & 215413 & 2004 & 16.40 & 0.26 & 1648.00 & 15374.80 & 1.00 & 0.94 & 0.93 \\
36452 & 106529 & 2004 & 272.80 & 0.23 & 27093.00 & 275142.79 & 1.01 & 1.01 & 1.02 \\
42515 & 108973 & 2004 & 309.40 & 0.22 & 19860.00 & 204177.85 & 1.56 & 0.66 & 1.03 \\
28928 & 105506 & 2004 & 1077.10 & 0.09 & 108753.00 & 1068265.45 & 0.99 & 0.99 & 0.98 \\
41842 & 108860 & 2004 & 36.90 & 0.01 & 3689.00 & 34900.26 & 1.00 & 0.95 & 0.95 \\
30272 & 105723 & 2004 & 816.20 & 0.15 & 81684.00 & 777836.43 & 1.00 & 0.95 & 0.95 \\
41835 & 108859 & 2004 & 18.20 & 0.09 & 1819.00 & 17474.99 & 1.00 & 0.96 & 0.96 \\
48831 & 240144 & 2004 & 102.10 & 0.15 & 7192.00 & 57602.45 & 1.42 & 0.56 & 0.80 \\
4231 & 100590 & 2004 & 174.90 & 0.09 & 16339.00 & 183515.81 & 1.07 & 1.05 & 1.12 \\
41828 & 108858 & 2004 & 36.60 & 0.11 & 3634.00 & 35498.54 & 1.01 & 0.97 & 0.98 \\
51291 & 240498 & 2004 & 86.60 & 0.08 & 7830.00 & 81476.44 & 1.11 & 0.94 & 1.04 \\
35094 & 106320 & 2004 & 700.30 & 0.13 & 78977.00 & 673376.05 & 0.89 & 0.96 & 0.85 \\
54600 & 377010 & 2004 & 172.80 & 0.15 & 17147.00 & 153117.20 & 1.01 & 0.89 & 0.89 \\
36473 & 106535 & 2004 & 417.00 & 0.15 & 43552.00 & 409171.58 & 0.96 & 0.98 & 0.94 \\
21457 & 102872 & 2004 & 3169.90 & 0.22 & 317462.00 & 2705895.91 & 1.00 & 0.85 & 0.85 \\
6057 & 100821 & 2004 & 59.90 & 0.13 & 4179.00 & 41270.49 & 1.43 & 0.69 & 0.99 \\
46305 & 200210 & 2004 & 9.90 & 0.07 & 985.00 & 9273.19 & 1.01 & 0.94 & 0.94 \\
30298 & 105731 & 2004 & 2152.30 & 0.07 & 215102.00 & 1913759.35 & 1.00 & 0.89 & 0.89 \\
44113 & 109266 & 2004 & 2142.00 & 0.24 & 162876.00 & 1592333.43 & 1.32 & 0.74 & 0.98 \\
55615 & 400131 & 2004 & 173.40 & 0.22 & 13358.00 & 146402.12 & 1.30 & 0.84 & 1.10 \\
28944 & 105507 & 2004 & 1196.60 & 0.11 & 109922.00 & 1165632.94 & 1.09 & 0.97 & 1.06 \\
42000 & 108901 & 2004 & 293.20 & 0.12 & 29460.00 & 277780.57 & 1.00 & 0.95 & 0.94 \\
42534 & 108976 & 2004 & 299.00 & 0.14 & 26173.00 & 261139.50 & 1.14 & 0.87 & 1.00 \\
28639 & 105457 & 2004 & 2878.70 & 0.03 & 363214.00 & 3589979.49 & 0.79 & 1.25 & 0.99 \\
7531 & 101043 & 2004 & 5053.90 & 0.11 & 520821.00 & 4911478.23 & 0.97 & 0.97 & 0.94 \\
7614 & 101048 & 2004 & 10091.40 & 0.03 & 1031821.00 & 9666273.79 & 0.98 & 0.96 & 0.94 \\
44445 & 109324 & 2004 & 92.00 & 0.06 & 9145.00 & 87027.60 & 1.01 & 0.95 & 0.95 \\
51376 & 240509 & 2004 & 9.20 & 0.20 & 655.00 & 6059.16 & 1.40 & 0.66 & 0.93 \\
22301 & 103005 & 2004 & 1394.70 & 0.14 & 130679.00 & 1306800.41 & 1.07 & 0.94 & 1.00 \\
10962 & 101356 & 2004 & 417.00 & 0.11 & 38355.00 & 422432.09 & 1.09 & 1.01 & 1.10 \\
74799 & 601172 & 2004 & 655.10 & 0.36 & 58615.00 & 619439.09 & 1.12 & 0.95 & 1.06 \\
13053 & 101623 & 2004 & 1215.70 & 0.14 & 119353.00 & 1098594.56 & 1.02 & 0.90 & 0.92 \\
42437 & 108966 & 2004 & 187.80 & 0.06 & 18880.00 & 181478.02 & 0.99 & 0.97 & 0.96 \\
50394 & 240414 & 2004 & 1813.90 & 0.05 & 178999.00 & 1701630.39 & 1.01 & 0.94 & 0.95 \\
46360 & 200227 & 2004 & 16.50 & 0.13 & 1657.00 & 15436.71 & 1.00 & 0.94 & 0.93 \\
35147 & 106329 & 2004 & 13.60 & 0.14 & 1195.00 & 13648.21 & 1.14 & 1.00 & 1.14 \\
16030 & 102073 & 2004 & 8887.80 & 0.10 & 867859.00 & 7780070.17 & 1.02 & 0.88 & 0.90 \\
6795 & 100954 & 2004 & 878.10 & 0.08 & 86847.00 & 819414.44 & 1.01 & 0.93 & 0.94 \\
42478 & 108970 & 2004 & 129.20 & 0.03 & 12120.00 & 126464.23 & 1.07 & 0.98 & 1.04 \\
326 & 100036 & 2004 & 26.30 & 0.04 & 2642.00 & 25430.16 & 1.00 & 0.97 & 0.96 \\
35173 & 106330 & 2004 & 292.90 & 0.08 & 29577.00 & 287776.47 & 0.99 & 0.98 & 0.97 \\
18884 & 102525 & 2004 & 1708.70 & 0.16 & 170517.00 & 1676404.98 & 1.00 & 0.98 & 0.98 \\
28729 & 105472 & 2004 & 139.70 & 0.09 & 9914.00 & 98249.22 & 1.41 & 0.70 & 0.99 \\
30239 & 105720 & 2004 & 873.50 & 0.14 & 97294.00 & 869941.02 & 0.90 & 1.00 & 0.89 \\
30119 & 105700 & 2004 & 200.10 & 0.13 & 20108.00 & 198661.65 & 1.00 & 0.99 & 0.99 \\
41978 & 108886 & 2004 & 73.10 & 0.14 & 7965.00 & 72644.78 & 0.92 & 0.99 & 0.91 \\
22221 & 102996 & 2004 & 561.80 & 0.07 & 58065.00 & 562951.50 & 0.97 & 1.00 & 0.97 \\
22112 & 102990 & 2004 & 4394.60 & 0.15 & 438906.00 & 4330414.23 & 1.00 & 0.99 & 0.99 \\
42490 & 108971 & 2004 & 149.40 & 0.16 & 16556.00 & 134254.71 & 0.90 & 0.90 & 0.81 \\
28708 & 105469 & 2004 & 71.00 & 0.11 & 7943.00 & 71538.01 & 0.89 & 1.01 & 0.90 \\
13888 & 101785 & 2004 & 1429.70 & 0.02 & 140829.00 & 1386390.44 & 1.02 & 0.97 & 0.98 \\
41856 & 108861 & 2004 & 90.20 & 0.06 & 9013.00 & 90064.79 & 1.00 & 1.00 & 1.00 \\
28683 & 105463 & 2004 & 815.90 & -0.06 & 96275.00 & 804583.42 & 0.85 & 0.99 & 0.84 \\
11060 & 101364 & 2004 & 397.00 & 0.13 & 35448.00 & 292512.99 & 1.12 & 0.74 & 0.83 \\
59112 & 410433 & 2004 & 2159.00 & 0.22 & 219269.00 & 2123254.54 & 0.98 & 0.98 & 0.97 \\
41862 & 108866 & 2004 & 387.10 & 0.17 & 38678.00 & 380135.97 & 1.00 & 0.98 & 0.98 \\
8824 & 101100 & 2004 & 588.30 & 0.35 & 62489.00 & 577258.89 & 0.94 & 0.98 & 0.92 \\
16468 & 102150 & 2004 & 58.20 & 0.01 & 5858.00 & 53160.63 & 0.99 & 0.91 & 0.91 \\
22249 & 102997 & 2004 & 3908.80 & 0.10 & 347900.00 & 3562795.90 & 1.12 & 0.91 & 1.02 \\
28692 & 105464 & 2004 & 32.00 & 0.28 & 2719.00 & 30972.32 & 1.18 & 0.97 & 1.14 \\
5987 & 100817 & 2004 & 102.90 & 0.08 & 10272.00 & 100826.59 & 1.00 & 0.98 & 0.98 \\
56659 & 400237 & 2004 & 22.80 & 0.02 & 2356.00 & 21993.40 & 0.97 & 0.96 & 0.93 \\
13039 & 101622 & 2004 & 703.50 & 0.09 & 70420.00 & 735645.52 & 1.00 & 1.05 & 1.04 \\
5732 & 100790 & 2004 & 530.30 & 0.09 & 37675.00 & 349019.13 & 1.41 & 0.66 & 0.93 \\
13286 & 101717 & 2004 & 62.50 & 0.09 & 5859.00 & 62344.35 & 1.07 & 1.00 & 1.06 \\
16544 & 102154 & 2004 & 219.10 & 0.13 & 21882.00 & 211997.17 & 1.00 & 0.97 & 0.97 \\
54621 & 377074 & 2004 & 512.20 & 0.11 & 46011.00 & 490301.69 & 1.11 & 0.96 & 1.07 \\
21365 & 102854 & 2004 & 218.00 & 0.12 & 21037.00 & 221796.39 & 1.04 & 1.02 & 1.05 \\
36269 & 106480 & 2004 & 1282.50 & 0.18 & 120659.00 & 1173169.30 & 1.06 & 0.91 & 0.97 \\
30013 & 105677 & 2004 & 59.10 & 0.05 & 5916.00 & 58565.30 & 1.00 & 0.99 & 0.99 \\
51485 & 240524 & 2004 & 12.20 & 0.04 & 1219.00 & 11799.76 & 1.00 & 0.97 & 0.97 \\
2635 & 100348 & 2004 & 79.60 & 0.18 & 7958.00 & 77603.29 & 1.00 & 0.97 & 0.98 \\
29186 & 105535 & 2004 & 175.10 & 0.08 & 17429.00 & 161234.12 & 1.00 & 0.92 & 0.93 \\
74721 & 601156 & 2004 & 120.20 & 0.19 & 11916.00 & 119508.89 & 1.01 & 0.99 & 1.00 \\
59299 & 410460 & 2004 & 4.10 & 0.13 & 335.00 & 2930.54 & 1.22 & 0.71 & 0.87 \\
11288 & 101390 & 2004 & 4153.40 & 0.09 & 380133.00 & 3994580.94 & 1.09 & 0.96 & 1.05 \\
5541 & 100771 & 2004 & 1381.90 & 0.32 & 133077.00 & 1334956.24 & 1.04 & 0.97 & 1.00 \\
34796 & 106277 & 2004 & 341.10 & 0.18 & 34204.00 & 330187.54 & 1.00 & 0.97 & 0.97 \\
51622 & 240536 & 2004 & 62.80 & 0.05 & 4430.00 & 43996.74 & 1.42 & 0.70 & 0.99 \\
28080 & 105379 & 2004 & 360.10 & 0.09 & 35483.00 & 352750.98 & 1.01 & 0.98 & 0.99 \\
30656 & 105781 & 2004 & 755.20 & 0.15 & 79137.00 & 756608.36 & 0.95 & 1.00 & 0.96 \\
36956 & 106643 & 2004 & 104.10 & 0.18 & 10425.00 & 103490.36 & 1.00 & 0.99 & 0.99 \\
51618 & 240535 & 2004 & 13.10 & 0.02 & 1309.00 & 11567.02 & 1.00 & 0.88 & 0.88 \\
46128 & 200192 & 2004 & 1.80 & 0.05 & 160.00 & 1689.15 & 1.12 & 0.94 & 1.06 \\
54505 & 373198 & 2004 & 45.10 & 0.10 & 5275.00 & 53209.51 & 0.85 & 1.18 & 1.01 \\
30647 & 105780 & 2004 & 271.70 & 0.04 & 28735.00 & 278033.87 & 0.95 & 1.02 & 0.97 \\
19274 & 102578 & 2004 & 9.40 & 0.04 & 904.00 & 8882.29 & 1.04 & 0.94 & 0.98 \\
46134 & 200193 & 2004 & 58.80 & 0.10 & 5878.00 & 58744.65 & 1.00 & 1.00 & 1.00 \\
21536 & 102893 & 2004 & 164.30 & -0.15 & 16285.00 & 132369.31 & 1.01 & 0.81 & 0.81 \\
36012 & 106447 & 2004 & 29.60 & 0.04 & 2416.00 & 27254.52 & 1.23 & 0.92 & 1.13 \\
28099 & 105382 & 2004 & 83.10 & 0.03 & 8342.00 & 77362.96 & 1.00 & 0.93 & 0.93 \\
50040 & 240385 & 2004 & 3.00 & 0.18 & 297.00 & 3009.49 & 1.01 & 1.00 & 1.01 \\
35353 & 106353 & 2004 & 779.40 & 0.17 & 76224.00 & 762212.55 & 1.02 & 0.98 & 1.00 \\
41535 & 108764 & 2004 & 418.90 & 0.10 & 42950.00 & 406756.09 & 0.98 & 0.97 & 0.95 \\
28048 & 105370 & 2004 & 82.70 & 0.09 & 7820.00 & 85086.15 & 1.06 & 1.03 & 1.09 \\
42696 & 108994 & 2004 & 120.90 & 0.01 & 12370.00 & 116250.45 & 0.98 & 0.96 & 0.94 \\
44271 & 109281 & 2004 & 148.30 & 0.17 & 14835.00 & 141292.46 & 1.00 & 0.95 & 0.95 \\
42380 & 108956 & 2004 & 44.10 & 0.07 & 4879.00 & 48515.26 & 0.90 & 1.10 & 0.99 \\
48260 & 240057 & 2004 & 126.50 & 0.10 & 11544.00 & 109107.56 & 1.10 & 0.86 & 0.95 \\
37019 & 106650 & 2004 & 256.30 & 0.25 & 25986.00 & 247622.52 & 0.99 & 0.97 & 0.95 \\
22685 & 103028 & 2004 & 5904.60 & 0.09 & 549204.00 & 4770529.59 & 1.08 & 0.81 & 0.87 \\
294 & 100033 & 2004 & 219.60 & 0.09 & 27006.00 & 238462.07 & 0.81 & 1.09 & 0.88 \\
34769 & 106276 & 2004 & 352.00 & 0.04 & 35378.00 & 341564.80 & 0.99 & 0.97 & 0.97 \\
46420 & 200241 & 2004 & 1.90 & 0.07 & 159.00 & 1667.46 & 1.19 & 0.88 & 1.05 \\
42681 & 108993 & 2004 & 82.10 & 0.23 & 7175.00 & 73595.08 & 1.14 & 0.90 & 1.03 \\
36982 & 106644 & 2004 & 1.70 & 0.05 & 147.00 & 1401.05 & 1.16 & 0.82 & 0.95 \\
54495 & 372855 & 2004 & 11.30 & 0.01 & 2247.00 & 22735.77 & 0.50 & 2.01 & 1.01 \\
14809 & 101914 & 2004 & 35.40 & 0.15 & 3359.00 & 34616.55 & 1.05 & 0.98 & 1.03 \\
50037 & 240384 & 2004 & 5.20 & 0.01 & 479.00 & 5304.09 & 1.09 & 1.02 & 1.11 \\
44542 & 109336 & 2004 & 386.70 & 0.13 & 20745.00 & 180362.10 & 1.86 & 0.47 & 0.87 \\
46154 & 200194 & 2004 & 8.70 & 0.04 & 616.00 & 6281.10 & 1.41 & 0.72 & 1.02 \\
234 & 100019 & 2004 & 13858.70 & 0.21 & 1214363.00 & 10976782.15 & 1.14 & 0.79 & 0.90 \\
21145 & 102833 & 2004 & 19.40 & -0.04 & 1931.00 & 18336.19 & 1.00 & 0.95 & 0.95 \\
44519 & 109334 & 2004 & 104.70 & 0.10 & 9959.00 & 96309.03 & 1.05 & 0.92 & 0.97 \\
5946 & 100812 & 2004 & 238.20 & 0.06 & 23867.00 & 234548.89 & 1.00 & 0.98 & 0.98 \\
466 & 100068 & 2004 & 89.30 & 0.13 & 8484.00 & 81885.65 & 1.05 & 0.92 & 0.97 \\
46682 & 200277 & 2004 & 5.70 & 0.11 & 567.00 & 5522.49 & 1.01 & 0.97 & 0.97 \\
52832 & 330728 & 2004 & 68.40 & 0.11 & 8805.00 & 84022.27 & 0.78 & 1.23 & 0.95 \\
22641 & 103027 & 2004 & 8906.90 & 0.10 & 856204.00 & 8695162.56 & 1.04 & 0.98 & 1.02 \\
36038 & 106449 & 2004 & 115.00 & -0.03 & 10965.00 & 103720.49 & 1.05 & 0.90 & 0.95 \\
28142 & 105384 & 2004 & 44.80 & 0.13 & 4049.00 & 43484.27 & 1.11 & 0.97 & 1.07 \\
29932 & 105658 & 2004 & 116.60 & 0.21 & 12691.00 & 105234.14 & 0.92 & 0.90 & 0.83 \\
53247 & 342547 & 2004 & 93.80 & 0.02 & 9525.00 & 94530.92 & 0.98 & 1.01 & 0.99 \\
36908 & 106627 & 2004 & 1858.60 & 0.15 & 149274.00 & 1531928.76 & 1.25 & 0.82 & 1.03 \\
5788 & 100792 & 2004 & 861.90 & 0.25 & 67579.00 & 749072.95 & 1.28 & 0.87 & 1.11 \\
34842 & 106282 & 2004 & 855.20 & 0.14 & 85489.00 & 730058.43 & 1.00 & 0.85 & 0.85 \\
62443 & 500392 & 2004 & 22.40 & 0.04 & 2235.00 & 22300.65 & 1.00 & 1.00 & 1.00 \\
46189 & 200197 & 2004 & 25.70 & 0.07 & 2179.00 & 22522.58 & 1.18 & 0.88 & 1.03 \\
52856 & 330794 & 2004 & 104.80 & 0.09 & 10190.00 & 104313.80 & 1.03 & 1.00 & 1.02 \\
36882 & 106620 & 2004 & 244.90 & 0.09 & 24393.00 & 198554.39 & 1.00 & 0.81 & 0.81 \\
30620 & 105779 & 2004 & 1329.30 & 0.07 & 133334.00 & 1315533.45 & 1.00 & 0.99 & 0.99 \\
50603 & 240430 & 2004 & 48.60 & 0.17 & 4878.00 & 46682.10 & 1.00 & 0.96 & 0.96 \\
28128 & 105383 & 2004 & 139.30 & 0.07 & 14088.00 & 136997.37 & 0.99 & 0.98 & 0.97 \\
36938 & 106642 & 2004 & 1399.90 & 0.21 & 129194.00 & 1291939.72 & 1.08 & 0.92 & 1.00 \\
42657 & 108992 & 2004 & 14.30 & 0.03 & 2769.00 & 26551.40 & 0.52 & 1.86 & 0.96 \\
54508 & 373584 & 2004 & 25.90 & 0.10 & 2602.00 & 25177.59 & 1.00 & 0.97 & 0.97 \\
34819 & 106278 & 2004 & 170.40 & 0.11 & 16869.00 & 167345.66 & 1.01 & 0.98 & 0.99 \\
46169 & 200196 & 2004 & 57.10 & 0.05 & 5258.00 & 52492.17 & 1.09 & 0.92 & 1.00 \\
10610 & 101300 & 2004 & 2141.40 & 0.23 & 191399.00 & 1613630.85 & 1.12 & 0.75 & 0.84 \\
3335 & 100424 & 2004 & 575.10 & 0.09 & 67647.00 & 679762.30 & 0.85 & 1.18 & 1.00 \\
36926 & 106640 & 2004 & 122.10 & 0.06 & 12259.00 & 115797.44 & 1.00 & 0.95 & 0.94 \\
46696 & 200279 & 2004 & 68.60 & 0.05 & 6531.00 & 63174.66 & 1.05 & 0.92 & 0.97 \\
74843 & 601186 & 2004 & 7.40 & 0.09 & 673.00 & 6592.43 & 1.10 & 0.89 & 0.98 \\
54792 & 400014 & 2004 & 188.20 & 0.12 & 18493.00 & 151602.01 & 1.02 & 0.81 & 0.82 \\
8498 & 101088 & 2004 & 2921.40 & -0.10 & 301815.00 & 2546662.63 & 0.97 & 0.87 & 0.84 \\
34831 & 106281 & 2004 & 6.90 & 0.10 & 523.00 & 5386.58 & 1.32 & 0.78 & 1.03 \\
16793 & 102192 & 2004 & 1985.20 & 0.25 & 197422.00 & 1821715.89 & 1.01 & 0.92 & 0.92 \\
41560 & 108766 & 2004 & 155.60 & 0.07 & 15567.00 & 140639.65 & 1.00 & 0.90 & 0.90 \\
54733 & 378592 & 2004 & 1.10 & 0.11 & 59.00 & 558.95 & 1.86 & 0.51 & 0.95 \\
15918 & 102059 & 2004 & 1068.50 & 0.13 & 91478.00 & 1028674.38 & 1.17 & 0.96 & 1.12 \\
52810 & 330079 & 2004 & 43.80 & 0.14 & 4354.00 & 37642.28 & 1.01 & 0.86 & 0.86 \\
50015 & 240383 & 2004 & 159.90 & 0.25 & 21454.00 & 209643.57 & 0.75 & 1.31 & 0.98 \\
18726 & 102504 & 2004 & 894.10 & 0.09 & 104632.00 & 800766.75 & 0.85 & 0.90 & 0.77 \\
16851 & 102197 & 2004 & 1121.90 & 0.16 & 113460.00 & 990302.25 & 0.99 & 0.88 & 0.87 \\
53242 & 342448 & 2004 & 151.20 & 0.11 & 14074.00 & 127789.25 & 1.07 & 0.85 & 0.91 \\
54486 & 372487 & 2004 & 2.90 & 0.17 & 294.00 & 2765.66 & 0.99 & 0.95 & 0.94 \\
42368 & 108954 & 2004 & 70.60 & 0.10 & 6722.00 & 68625.59 & 1.05 & 0.97 & 1.02 \\
4180 & 100567 & 2004 & 888.60 & 0.04 & 86264.00 & 835691.47 & 1.03 & 0.94 & 0.97 \\
37090 & 106675 & 2004 & 104.00 & 0.17 & 10386.00 & 98967.89 & 1.00 & 0.95 & 0.95 \\
44136 & 109268 & 2004 & 1465.80 & 0.05 & 146855.00 & 1340712.21 & 1.00 & 0.91 & 0.91 \\
16366 & 102130 & 2004 & 268.10 & 0.08 & 26555.00 & 269653.46 & 1.01 & 1.01 & 1.02 \\
51657 & 240538 & 2004 & 91.20 & 0.07 & 8406.00 & 78130.55 & 1.08 & 0.86 & 0.93 \\
51095 & 240482 & 2004 & 3.10 & 0.09 & 288.00 & 2943.39 & 1.08 & 0.95 & 1.02 \\
53464 & 350572 & 2004 & 46.80 & 0.09 & 4286.00 & 46522.33 & 1.09 & 0.99 & 1.09 \\
49978 & 240380 & 2004 & 10.10 & 0.06 & 1009.00 & 9808.57 & 1.00 & 0.97 & 0.97 \\
492 & 100071 & 2004 & 2923.70 & 0.08 & 290392.00 & 2640596.89 & 1.01 & 0.90 & 0.91 \\
37100 & 106678 & 2004 & 52.90 & 0.18 & 5152.00 & 51118.70 & 1.03 & 0.97 & 0.99 \\
27955 & 105358 & 2004 & 4049.00 & 0.27 & 311358.00 & 3338025.14 & 1.30 & 0.82 & 1.07 \\
47664 & 216749 & 2004 & 106.70 & 0.10 & 10517.00 & 107448.23 & 1.01 & 1.01 & 1.02 \\
30756 & 105793 & 2004 & 2464.80 & 0.09 & 246926.00 & 2158107.21 & 1.00 & 0.88 & 0.87 \\
44260 & 109279 & 2004 & 63.00 & 0.21 & 6297.00 & 59730.22 & 1.00 & 0.95 & 0.95 \\
14517 & 101871 & 2004 & 305.30 & 0.17 & 21692.00 & 286002.69 & 1.41 & 0.94 & 1.32 \\
41439 & 108759 & 2004 & 93.60 & 0.21 & 9423.00 & 90298.22 & 0.99 & 0.96 & 0.96 \\
35961 & 106442 & 2004 & 5168.80 & -0.00 & 482327.00 & 4863477.46 & 1.07 & 0.94 & 1.01 \\
27936 & 105353 & 2004 & 51.00 & 0.21 & 5057.00 & 48919.32 & 1.01 & 0.96 & 0.97 \\
53492 & 351048 & 2004 & 624.90 & 0.29 & 33008.00 & 312546.53 & 1.89 & 0.50 & 0.95 \\
42704 & 108996 & 2004 & 40.50 & 0.04 & 3921.00 & 40652.09 & 1.03 & 1.00 & 1.04 \\
35420 & 106359 & 2004 & 129.00 & 0.07 & 12910.00 & 128926.77 & 1.00 & 1.00 & 1.00 \\
62463 & 500393 & 2004 & 24.10 & 0.02 & 2406.00 & 23946.98 & 1.00 & 0.99 & 1.00 \\
41463 & 108760 & 2004 & 163.80 & 0.11 & 16690.00 & 166821.61 & 0.98 & 1.02 & 1.00 \\
51679 & 240539 & 2004 & 22.70 & 0.13 & 2032.00 & 22270.81 & 1.12 & 0.98 & 1.10 \\
29241 & 105561 & 2004 & 624.70 & 0.33 & 62384.00 & 607054.56 & 1.00 & 0.97 & 0.97 \\
13778 & 101763 & 2004 & 313.30 & 0.33 & 31497.00 & 302889.15 & 0.99 & 0.97 & 0.96 \\
48730 & 240134 & 2004 & 311.50 & 0.29 & 27925.00 & 309586.20 & 1.12 & 0.99 & 1.11 \\
34680 & 106270 & 2004 & 28.70 & 0.09 & 2855.00 & 27944.31 & 1.01 & 0.97 & 0.98 \\
51681 & 240540 & 2004 & 68.30 & 0.04 & 5660.00 & 61082.56 & 1.21 & 0.89 & 1.08 \\
7669 & 101054 & 2004 & 10812.20 & 0.08 & 1009370.00 & 10317179.32 & 1.07 & 0.95 & 1.02 \\
5510 & 100769 & 2004 & 6561.90 & 0.26 & 655938.00 & 6388945.57 & 1.00 & 0.97 & 0.97 \\
21913 & 102979 & 2004 & 59.80 & 0.10 & 5521.00 & 57543.03 & 1.08 & 0.96 & 1.04 \\
35989 & 106444 & 2004 & 236.50 & 0.07 & 23762.00 & 230034.57 & 1.00 & 0.97 & 0.97 \\
42374 & 108955 & 2004 & 55.90 & 0.02 & 4987.00 & 55700.10 & 1.12 & 1.00 & 1.12 \\
34742 & 106275 & 2004 & 121.50 & 0.29 & 12090.00 & 112791.18 & 1.00 & 0.93 & 0.93 \\
16822 & 102193 & 2004 & 104.80 & 0.12 & 10550.00 & 109674.42 & 0.99 & 1.05 & 1.04 \\
41488 & 108761 & 2004 & 493.60 & 0.17 & 46884.00 & 455510.61 & 1.05 & 0.92 & 0.97 \\
21113 & 102832 & 2004 & 90.10 & -0.01 & 9023.00 & 86145.04 & 1.00 & 0.96 & 0.95 \\
29898 & 105656 & 2004 & 425.00 & 0.22 & 41898.00 & 418964.91 & 1.01 & 0.99 & 1.00 \\
37038 & 106654 & 2004 & 513.70 & 0.10 & 47812.00 & 501593.17 & 1.07 & 0.98 & 1.05 \\
35387 & 106356 & 2004 & 17.40 & 0.17 & 1742.00 & 17331.41 & 1.00 & 1.00 & 0.99 \\
41510 & 108762 & 2004 & 76.40 & 0.08 & 7576.00 & 74335.21 & 1.01 & 0.97 & 0.98 \\
50004 & 240382 & 2004 & 57.80 & -0.04 & 5771.00 & 48601.93 & 1.00 & 0.84 & 0.84 \\
30687 & 105783 & 2004 & 5998.40 & 0.31 & 553288.00 & 5796914.24 & 1.08 & 0.97 & 1.05 \\
55539 & 400117 & 2004 & 36.80 & 0.10 & 3120.00 & 33357.64 & 1.18 & 0.91 & 1.07 \\
42119 & 108924 & 2004 & 221.00 & 0.37 & 21249.00 & 214866.68 & 1.04 & 0.97 & 1.01 \\
47634 & 215952 & 2004 & 659.90 & 0.14 & 65987.00 & 546474.29 & 1.00 & 0.83 & 0.83 \\
29202 & 105536 & 2004 & 204.10 & 0.06 & 20300.00 & 190257.73 & 1.01 & 0.93 & 0.94 \\
28019 & 105369 & 2004 & 182.20 & 0.04 & 17545.00 & 187656.65 & 1.04 & 1.03 & 1.07 \\
19286 & 102579 & 2004 & 221.40 & 0.00 & 22037.00 & 218888.40 & 1.00 & 0.99 & 0.99 \\
48907 & 240152 & 2004 & 201.20 & 0.06 & 21166.00 & 190343.40 & 0.95 & 0.95 & 0.90 \\
50453 & 240421 & 2004 & 550.40 & 0.23 & 55664.00 & 483253.04 & 0.99 & 0.88 & 0.87 \\
27987 & 105364 & 2004 & 154.10 & 0.08 & 15911.00 & 152424.38 & 0.97 & 0.99 & 0.96 \\
7205 & 101013 & 2004 & 4906.70 & 0.07 & 491310.00 & 4913358.87 & 1.00 & 1.00 & 1.00 \\
44015 & 109258 & 2004 & 581.70 & 0.14 & 49869.00 & 521632.22 & 1.17 & 0.90 & 1.05 \\
49982 & 240381 & 2004 & 44.80 & 0.32 & 5162.00 & 43315.77 & 0.87 & 0.97 & 0.84 \\
47749 & 221051 & 2004 & 4704.80 & 0.12 & 468536.00 & 3929926.62 & 1.00 & 0.84 & 0.84 \\
30714 & 105788 & 2004 & 42.30 & 0.11 & 4252.00 & 39141.09 & 0.99 & 0.93 & 0.92 \\
54818 & 400015 & 2004 & 3.90 & -0.05 & 291.00 & 3121.84 & 1.34 & 0.80 & 1.07 \\
11320 & 101393 & 2004 & 747.40 & 0.08 & 70198.00 & 740734.47 & 1.06 & 0.99 & 1.06 \\
44550 & 109338 & 2004 & 436.20 & 0.23 & 44811.00 & 422350.23 & 0.97 & 0.97 & 0.94 \\
2616 & 100347 & 2004 & 766.80 & 0.10 & 76521.00 & 757271.22 & 1.00 & 0.99 & 0.99 \\
51102 & 240483 & 2004 & 1.70 & 0.18 & 160.00 & 1587.64 & 1.06 & 0.93 & 0.99 \\
37064 & 106655 & 2004 & 37.70 & -0.05 & 3774.00 & 37552.36 & 1.00 & 1.00 & 1.00 \\
34713 & 106272 & 2004 & 2602.60 & 0.10 & 317661.00 & 2904395.96 & 0.82 & 1.12 & 0.91 \\
52779 & 322114 & 2004 & 7.70 & 0.20 & 729.00 & 6422.37 & 1.06 & 0.83 & 0.88 \\
46425 & 200243 & 2004 & 1.80 & 0.15 & 144.00 & 1677.04 & 1.25 & 0.93 & 1.16 \\
13921 & 101787 & 2004 & 270.20 & 0.05 & 28685.00 & 292176.73 & 0.94 & 1.08 & 1.02 \\
46100 & 200189 & 2004 & 79.80 & 0.15 & 8424.00 & 67076.20 & 0.95 & 0.84 & 0.80 \\
61938 & 500310 & 2004 & 7.30 & 0.06 & 759.00 & 6809.26 & 0.96 & 0.93 & 0.90 \\
42131 & 108925 & 2004 & 369.00 & 0.11 & 35797.00 & 377283.22 & 1.03 & 1.02 & 1.05 \\
12919 & 101606 & 2004 & 3085.60 & 0.08 & 305337.00 & 2828508.21 & 1.01 & 0.92 & 0.93 \\
16779 & 102191 & 2004 & 80.00 & 0.05 & 7354.00 & 70026.32 & 1.09 & 0.88 & 0.95 \\
46672 & 200276 & 2004 & 36.10 & 0.13 & 3650.00 & 34974.23 & 0.99 & 0.97 & 0.96 \\
46556 & 200253 & 2004 & 6.90 & 0.00 & 692.00 & 6457.57 & 1.00 & 0.94 & 0.93 \\
52863 & 332404 & 2004 & 19.20 & 0.15 & 1512.00 & 15327.72 & 1.27 & 0.80 & 1.01 \\
4288 & 100600 & 2004 & 67.50 & 0.18 & 6719.00 & 62626.54 & 1.00 & 0.93 & 0.93 \\
28310 & 105401 & 2004 & 218.80 & 0.16 & 22755.00 & 212894.74 & 0.96 & 0.97 & 0.94 \\
56825 & 400257 & 2004 & 31.90 & -0.03 & 3288.00 & 31940.14 & 0.97 & 1.00 & 0.97 \\
41647 & 108826 & 2004 & 342.70 & 0.21 & 35735.00 & 347014.44 & 0.96 & 1.01 & 0.97 \\
61982 & 500315 & 2004 & 56.20 & 0.27 & 5250.00 & 48956.89 & 1.07 & 0.87 & 0.93 \\
8534 & 101089 & 2004 & 171.40 & 0.13 & 15786.00 & 158224.69 & 1.09 & 0.92 & 1.00 \\
29993 & 105665 & 2004 & 19.10 & 0.05 & 1952.00 & 18992.60 & 0.98 & 0.99 & 0.97 \\
54642 & 377379 & 2004 & 31.20 & 0.11 & 3138.00 & 28531.14 & 0.99 & 0.91 & 0.91 \\
74769 & 601165 & 2004 & 18.30 & 0.15 & 1848.00 & 17378.67 & 0.99 & 0.95 & 0.94 \\
36107 & 106464 & 2004 & 357.20 & 0.03 & 35878.00 & 314310.62 & 1.00 & 0.88 & 0.88 \\
7498 & 101042 & 2004 & 19422.30 & 0.06 & 1885531.00 & 19188087.81 & 1.03 & 0.99 & 1.02 \\
5652 & 100784 & 2004 & 43266.30 & 0.19 & 4272128.00 & 34605711.20 & 1.01 & 0.80 & 0.81 \\
53442 & 349609 & 2004 & 12.00 & 0.04 & 1165.00 & 11648.05 & 1.03 & 0.97 & 1.00 \\
30501 & 105762 & 2004 & 403.90 & 0.11 & 47756.00 & 384991.39 & 0.85 & 0.95 & 0.81 \\
41672 & 108827 & 2004 & 776.10 & 0.13 & 73856.00 & 784052.60 & 1.05 & 1.01 & 1.06 \\
28327 & 105412 & 2004 & 146.70 & 0.09 & 14088.00 & 148820.99 & 1.04 & 1.01 & 1.06 \\
54762 & 378620 & 2004 & 39.90 & 0.18 & 3988.00 & 33677.49 & 1.00 & 0.84 & 0.84 \\
2693 & 100352 & 2004 & 1818.40 & 0.14 & 160271.00 & 1760263.68 & 1.13 & 0.97 & 1.10 \\
5627 & 100775 & 2004 & 366.40 & 0.05 & 39297.00 & 393551.37 & 0.93 & 1.07 & 1.00 \\
50580 & 240429 & 2004 & 19.10 & 0.31 & 2279.00 & 19178.81 & 0.84 & 1.00 & 0.84 \\
4404 & 100622 & 2004 & 719.30 & 0.22 & 62121.00 & 695015.04 & 1.16 & 0.97 & 1.12 \\
30529 & 105763 & 2004 & 413.10 & 0.17 & 44503.00 & 386895.81 & 0.93 & 0.94 & 0.87 \\
44031 & 109259 & 2004 & 206.20 & 0.22 & 20426.00 & 201590.78 & 1.01 & 0.98 & 0.99 \\
28281 & 105400 & 2004 & 365.80 & 0.12 & 32376.00 & 358536.21 & 1.13 & 0.98 & 1.11 \\
51201 & 240491 & 2004 & 55.60 & 0.04 & 5537.00 & 54911.66 & 1.00 & 0.99 & 0.99 \\
15948 & 102061 & 2004 & 2405.70 & 0.15 & 212188.00 & 2081984.32 & 1.13 & 0.87 & 0.98 \\
2919 & 100379 & 2004 & 494.70 & 0.12 & 34564.00 & 354903.36 & 1.43 & 0.72 & 1.03 \\
51556 & 240529 & 2004 & 89.60 & 0.16 & 8622.00 & 87515.68 & 1.04 & 0.98 & 1.02 \\
29081 & 105525 & 2004 & 4625.10 & 0.17 & 439348.00 & 3531614.92 & 1.05 & 0.76 & 0.80 \\
22532 & 103017 & 2004 & 2118.90 & 0.11 & 282347.00 & 2739413.48 & 0.75 & 1.29 & 0.97 \\
36088 & 106461 & 2004 & 455.30 & 0.06 & 45520.00 & 395300.33 & 1.00 & 0.87 & 0.87 \\
42601 & 108985 & 2004 & 3757.70 & 0.14 & 324546.00 & 3121227.95 & 1.16 & 0.83 & 0.96 \\
36776 & 106590 & 2004 & 89.50 & 0.09 & 8943.00 & 79732.85 & 1.00 & 0.89 & 0.89 \\
18779 & 102508 & 2004 & 92.40 & 0.05 & 8998.00 & 95815.45 & 1.03 & 1.04 & 1.06 \\
74913 & 601197 & 2004 & 33.10 & 0.07 & 3316.00 & 33155.07 & 1.00 & 1.00 & 1.00 \\
51221 & 240492 & 2004 & 160.90 & 0.04 & 15892.00 & 158396.10 & 1.01 & 0.98 & 1.00 \\
47718 & 220770 & 2004 & 2071.20 & 0.11 & 386349.00 & 3517724.36 & 0.54 & 1.70 & 0.91 \\
46646 & 200268 & 2004 & 145.10 & 0.10 & 14489.00 & 143031.85 & 1.00 & 0.99 & 0.99 \\
44306 & 109283 & 2004 & 2406.50 & 0.12 & 219121.00 & 2325452.42 & 1.10 & 0.97 & 1.06 \\
53434 & 349198 & 2004 & 847.00 & 0.05 & 85108.00 & 851108.88 & 1.00 & 1.00 & 1.00 \\
7644 & 101050 & 2004 & 891.70 & 0.07 & 83792.00 & 785747.37 & 1.06 & 0.88 & 0.94 \\
30473 & 105761 & 2004 & 1131.70 & 0.13 & 125592.00 & 1083486.79 & 0.90 & 0.96 & 0.86 \\
36718 & 106580 & 2004 & 109.70 & 0.23 & 12411.00 & 105543.96 & 0.88 & 0.96 & 0.85 \\
44329 & 109284 & 2004 & 7.40 & 0.20 & 740.00 & 7211.17 & 1.00 & 0.97 & 0.97 \\
3187 & 100413 & 2004 & 74.10 & 0.08 & 7429.00 & 64229.73 & 1.00 & 0.87 & 0.86 \\
62079 & 500327 & 2004 & 1.10 & 0.09 & 108.00 & 991.19 & 1.02 & 0.90 & 0.92 \\
51509 & 240526 & 2004 & 71.10 & 0.02 & 7110.00 & 70818.83 & 1.00 & 1.00 & 1.00 \\
55580 & 400127 & 2004 & 95.70 & 0.14 & 8198.00 & 91161.10 & 1.17 & 0.95 & 1.11 \\
41680 & 108839 & 2004 & 250.80 & 0.03 & 25102.00 & 249834.82 & 1.00 & 1.00 & 1.00 \\
28379 & 105420 & 2004 & 37.90 & 0.01 & 3771.00 & 37977.01 & 1.01 & 1.00 & 1.01 \\
6153 & 100825 & 2004 & 65.60 & 0.12 & 6560.00 & 63310.46 & 1.00 & 0.97 & 0.97 \\
50130 & 240395 & 2004 & 127.50 & 0.33 & 10567.00 & 109442.66 & 1.21 & 0.86 & 1.04 \\
34997 & 106305 & 2004 & 423.60 & 0.26 & 42236.00 & 388070.06 & 1.00 & 0.92 & 0.92 \\
36692 & 106577 & 2004 & 2891.90 & 0.10 & 258095.00 & 2771131.28 & 1.12 & 0.96 & 1.07 \\
53290 & 342993 & 2004 & 27.70 & 0.11 & 2194.00 & 22619.93 & 1.26 & 0.82 & 1.03 \\
22472 & 103014 & 2004 & 268.40 & 0.10 & 27454.00 & 274619.35 & 0.98 & 1.02 & 1.00 \\
62363 & 500388 & 2004 & 36.20 & 0.03 & 3617.00 & 36152.78 & 1.00 & 1.00 & 1.00 \\
62059 & 500326 & 2004 & 2.10 & 0.16 & 204.00 & 1971.70 & 1.03 & 0.94 & 0.97 \\
46642 & 200267 & 2004 & 55.20 & 0.18 & 5506.00 & 52028.43 & 1.00 & 0.94 & 0.94 \\
47690 & 220681 & 2004 & 551.90 & 0.11 & 53576.00 & 461588.16 & 1.03 & 0.84 & 0.86 \\
51514 & 240527 & 2004 & 260.80 & 0.03 & 26155.00 & 261180.81 & 1.00 & 1.00 & 1.00 \\
22505 & 103016 & 2004 & 747.30 & 0.20 & 80082.00 & 641752.92 & 0.93 & 0.86 & 0.80 \\
10571 & 101299 & 2004 & 2267.30 & 0.11 & 207796.00 & 2230644.22 & 1.09 & 0.98 & 1.07 \\
44039 & 109260 & 2004 & 101.70 & 0.21 & 10173.00 & 101472.61 & 1.00 & 1.00 & 1.00 \\
4386 & 100614 & 2004 & 1858.30 & 0.15 & 173414.00 & 1806310.14 & 1.07 & 0.97 & 1.04 \\
63478 & 500511 & 2004 & 787.20 & 0.11 & 82098.00 & 801809.73 & 0.96 & 1.02 & 0.98 \\
28341 & 105416 & 2004 & 3753.30 & 0.15 & 436996.00 & 3790921.64 & 0.86 & 1.01 & 0.87 \\
10357 & 101283 & 2004 & 4241.50 & 0.25 & 396998.00 & 3687374.52 & 1.07 & 0.87 & 0.93 \\
36748 & 106584 & 2004 & 972.10 & 0.25 & 50210.00 & 505131.86 & 1.94 & 0.52 & 1.01 \\
16112 & 102080 & 2004 & 3310.10 & 0.14 & 336351.00 & 3310304.75 & 0.98 & 1.00 & 0.98 \\
55575 & 400126 & 2004 & 35.60 & 0.06 & 3550.00 & 34160.48 & 1.00 & 0.96 & 0.96 \\
22490 & 103015 & 2004 & 322.50 & 0.05 & 31627.00 & 314112.06 & 1.02 & 0.97 & 0.99 \\
29071 & 105523 & 2004 & 86.80 & 0.10 & 8512.00 & 85480.60 & 1.02 & 0.98 & 1.00 \\
28353 & 105419 & 2004 & 60.30 & 0.08 & 6069.00 & 57351.60 & 0.99 & 0.95 & 0.94 \\
34985 & 106299 & 2004 & 145.10 & 0.24 & 13748.00 & 142627.30 & 1.06 & 0.98 & 1.04 \\
36734 & 106583 & 2004 & 40.70 & 0.16 & 3700.00 & 37008.04 & 1.10 & 0.91 & 1.00 \\
55172 & 400065 & 2004 & 116.60 & 0.08 & 11751.00 & 116729.40 & 0.99 & 1.00 & 0.99 \\
34975 & 106298 & 2004 & 52.20 & 0.15 & 5083.00 & 50273.44 & 1.03 & 0.96 & 0.99 \\
44582 & 109343 & 2004 & 236.20 & 0.13 & 21064.00 & 195499.69 & 1.12 & 0.83 & 0.93 \\
54553 & 375941 & 2004 & 32.10 & 0.10 & 3179.00 & 29911.69 & 1.01 & 0.93 & 0.94 \\
46665 & 200274 & 2004 & 28.90 & 0.04 & 2887.00 & 26469.28 & 1.00 & 0.92 & 0.92 \\
36053 & 106451 & 2004 & 119.00 & 0.03 & 18278.00 & 175084.79 & 0.65 & 1.47 & 0.96 \\
62383 & 500389 & 2004 & 30.00 & 0.04 & 3005.00 & 30043.96 & 1.00 & 1.00 & 1.00 \\
36855 & 106605 & 2004 & 367.10 & 0.18 & 36563.00 & 338905.22 & 1.00 & 0.92 & 0.93 \\
19258 & 102575 & 2004 & 276.40 & 0.07 & 27634.00 & 266323.83 & 1.00 & 0.96 & 0.96 \\
34889 & 106284 & 2004 & 372.20 & 0.17 & 37763.00 & 342766.84 & 0.99 & 0.92 & 0.91 \\
53448 & 350408 & 2004 & 37.80 & 0.17 & 3531.00 & 34854.57 & 1.07 & 0.92 & 0.99 \\
59037 & 410401 & 2004 & 206.90 & 0.30 & 19692.00 & 175973.97 & 1.05 & 0.85 & 0.89 \\
46563 & 200254 & 2004 & 280.10 & 0.01 & 40265.00 & 265656.56 & 0.70 & 0.95 & 0.66 \\
50061 & 240388 & 2004 & 18.70 & 0.10 & 1820.00 & 16586.30 & 1.03 & 0.89 & 0.91 \\
22604 & 103024 & 2004 & 1715.90 & 0.16 & 158370.00 & 1586313.41 & 1.08 & 0.92 & 1.00 \\
51607 & 240534 & 2004 & 120.40 & 0.03 & 10232.00 & 112938.28 & 1.18 & 0.94 & 1.10 \\
61963 & 500312 & 2004 & 59.10 & 0.30 & 5534.00 & 54015.89 & 1.07 & 0.91 & 0.98 \\
21961 & 102981 & 2004 & 223.60 & 0.09 & 19032.00 & 222069.19 & 1.17 & 0.99 & 1.17 \\
29132 & 105531 & 2004 & 177.40 & 0.03 & 17577.00 & 161270.46 & 1.01 & 0.91 & 0.92 \\
41600 & 108777 & 2004 & 28.20 & 0.25 & 3077.00 & 27112.33 & 0.92 & 0.96 & 0.88 \\
30567 & 105770 & 2004 & 37.20 & 0.08 & 3722.00 & 34424.78 & 1.00 & 0.93 & 0.92 \\
42641 & 108990 & 2004 & 32.00 & 0.13 & 3749.00 & 34231.28 & 0.85 & 1.07 & 0.91 \\
48784 & 240140 & 2004 & 90.40 & 0.07 & 5738.00 & 47294.87 & 1.58 & 0.52 & 0.82 \\
11264 & 101380 & 2004 & 350.10 & 0.21 & 30993.00 & 345304.40 & 1.13 & 0.99 & 1.11 \\
29152 & 105533 & 2004 & 172.40 & 0.05 & 16065.00 & 170314.73 & 1.07 & 0.99 & 1.06 \\
2654 & 100350 & 2004 & 204.80 & 0.05 & 20491.00 & 200461.51 & 1.00 & 0.98 & 0.98 \\
55561 & 400125 & 2004 & 79.00 & 0.10 & 7893.00 & 78566.44 & 1.00 & 0.99 & 1.00 \\
44508 & 109333 & 2004 & 48.30 & 0.12 & 3972.00 & 45116.24 & 1.22 & 0.93 & 1.14 \\
62423 & 500391 & 2004 & 37.60 & 0.02 & 3761.00 & 37536.52 & 1.00 & 1.00 & 1.00 \\
62403 & 500390 & 2004 & 23.80 & 0.03 & 2377.00 & 23672.22 & 1.00 & 0.99 & 1.00 \\
30596 & 105775 & 2004 & 1118.50 & 0.09 & 112008.00 & 1062055.33 & 1.00 & 0.95 & 0.95 \\
28212 & 105393 & 2004 & 4769.60 & 0.14 & 453733.00 & 3931063.07 & 1.05 & 0.82 & 0.87 \\
42649 & 108991 & 2004 & 262.10 & 0.09 & 26348.00 & 237076.70 & 0.99 & 0.90 & 0.90 \\
53267 & 342548 & 2004 & 137.60 & 0.02 & 13695.00 & 136628.45 & 1.00 & 0.99 & 1.00 \\
41568 & 108773 & 2004 & 257.90 & 0.24 & 26165.00 & 250996.71 & 0.99 & 0.97 & 0.96 \\
28174 & 105390 & 2004 & 454.00 & 0.19 & 45391.00 & 448016.96 & 1.00 & 0.99 & 0.99 \\
34869 & 106283 & 2004 & 371.60 & 0.13 & 36363.00 & 349109.55 & 1.02 & 0.94 & 0.96 \\
51133 & 240486 & 2004 & 5.00 & 0.06 & 502.00 & 4563.44 & 1.00 & 0.91 & 0.91 \\
13129 & 101668 & 2004 & 91.10 & 0.05 & 9083.00 & 84400.82 & 1.00 & 0.93 & 0.93 \\
5569 & 100772 & 2004 & 2252.20 & 0.18 & 249816.00 & 2365550.85 & 0.90 & 1.05 & 0.95 \\
11253 & 101379 & 2004 & 899.20 & 0.10 & 84971.00 & 902937.22 & 1.06 & 1.00 & 1.06 \\
41575 & 108776 & 2004 & 721.50 & 0.20 & 54628.00 & 579669.15 & 1.32 & 0.80 & 1.06 \\
29947 & 105659 & 2004 & 190.30 & 0.07 & 18771.00 & 181671.87 & 1.01 & 0.95 & 0.97 \\
74766 & 601163 & 2004 & 35.40 & 0.10 & 3220.00 & 35898.06 & 1.10 & 1.01 & 1.11 \\
29959 & 105662 & 2004 & 153.00 & 0.05 & 14180.00 & 147712.60 & 1.08 & 0.97 & 1.04 \\
51181 & 240490 & 2004 & 76.60 & 0.03 & 7578.00 & 75699.95 & 1.01 & 0.99 & 1.00 \\
50092 & 240392 & 2004 & 443.90 & 0.08 & 39318.00 & 369490.00 & 1.13 & 0.83 & 0.94 \\
34904 & 106292 & 2004 & 289.50 & 0.20 & 23295.00 & 231594.46 & 1.24 & 0.80 & 0.99 \\
63502 & 500512 & 2004 & 794.70 & 0.14 & 105753.00 & 948787.19 & 0.75 & 1.19 & 0.90 \\
28252 & 105399 & 2004 & 117.30 & 0.23 & 9899.00 & 114652.53 & 1.18 & 0.98 & 1.16 \\
51582 & 240531 & 2004 & 15.40 & 0.03 & 1323.00 & 13017.80 & 1.16 & 0.85 & 0.98 \\
42635 & 108988 & 2004 & 212.10 & 0.05 & 23662.00 & 221197.86 & 0.90 & 1.04 & 0.93 \\
11223 & 101376 & 2004 & 2103.90 & 0.07 & 210163.00 & 1998053.84 & 1.00 & 0.95 & 0.95 \\
54754 & 378596 & 2004 & 79.20 & 0.14 & 9040.00 & 83151.05 & 0.88 & 1.05 & 0.92 \\
36827 & 106602 & 2004 & 25.40 & -0.16 & 2411.00 & 24599.58 & 1.05 & 0.97 & 1.02 \\
59053 & 410418 & 2004 & 1134.90 & 0.11 & 140097.00 & 1429932.91 & 0.81 & 1.26 & 1.02 \\
16729 & 102182 & 2004 & 166.10 & 0.14 & 17867.00 & 160509.80 & 0.93 & 0.97 & 0.90 \\
29975 & 105664 & 2004 & 384.60 & 0.08 & 35328.00 & 330102.35 & 1.09 & 0.86 & 0.93 \\
29109 & 105527 & 2004 & 7.10 & 0.13 & 704.00 & 6803.80 & 1.01 & 0.96 & 0.97 \\
36820 & 106597 & 2004 & 179.30 & 0.05 & 17872.00 & 178121.42 & 1.00 & 0.99 & 1.00 \\
41624 & 108782 & 2004 & 954.60 & 0.18 & 88477.00 & 963471.55 & 1.08 & 1.01 & 1.09 \\
42074 & 108918 & 2004 & 86.20 & 0.19 & 8570.00 & 85691.33 & 1.01 & 0.99 & 1.00 \\
42626 & 108987 & 2004 & 51.60 & 0.08 & 5149.00 & 50471.48 & 1.00 & 0.98 & 0.98 \\
42099 & 108919 & 2004 & 116.90 & 0.05 & 11755.00 & 98182.15 & 0.99 & 0.84 & 0.84 \\
47679 & 217585 & 2004 & 417.50 & 0.11 & 32142.00 & 338237.60 & 1.30 & 0.81 & 1.05 \\
2673 & 100351 & 2004 & 43.70 & 0.08 & 4367.00 & 41246.47 & 1.00 & 0.94 & 0.94 \\
10901 & 101345 & 2004 & 3700.80 & 0.14 & 376339.00 & 3073397.76 & 0.98 & 0.83 & 0.82 \\
41610 & 108780 & 2004 & 53.20 & 0.11 & 4985.00 & 52761.62 & 1.07 & 0.99 & 1.06 \\
51595 & 240533 & 2004 & 3.20 & 0.12 & 325.00 & 2937.64 & 0.98 & 0.92 & 0.90 \\
59256 & 410448 & 2004 & 339.80 & 0.22 & 25821.00 & 289970.15 & 1.32 & 0.85 & 1.12 \\
50067 & 240391 & 2004 & 497.40 & 0.07 & 39670.00 & 451439.84 & 1.25 & 0.91 & 1.14 \\
16746 & 102183 & 2004 & 1372.70 & 0.19 & 104231.00 & 1125771.71 & 1.32 & 0.82 & 1.08 \\
18995 & 102540 & 2004 & 258.20 & 0.17 & 25879.00 & 220184.77 & 1.00 & 0.85 & 0.85 \\
12954 & 101616 & 2004 & 14814.40 & 0.03 & 1475521.00 & 14198539.57 & 1.00 & 0.96 & 0.96 \\
74762 & 601162 & 2004 & 145.70 & 0.13 & 13960.00 & 137504.40 & 1.04 & 0.94 & 0.98 \\
18761 & 102507 & 2004 & 193.90 & 0.09 & 18647.00 & 186027.27 & 1.04 & 0.96 & 1.00 \\
56589 & 400229 & 2004 & 22.10 & 0.02 & 2119.00 & 21250.28 & 1.04 & 0.96 & 1.00 \\
21199 & 102837 & 2004 & 975.20 & 0.12 & 91266.00 & 846442.63 & 1.07 & 0.87 & 0.93 \\
22573 & 103021 & 2004 & 243.80 & 0.24 & 22191.00 & 237173.06 & 1.10 & 0.97 & 1.07 \\
51155 & 240487 & 2004 & 6.50 & 0.11 & 653.00 & 5541.57 & 1.00 & 0.85 & 0.85 \\
5608 & 100773 & 2004 & 3379.70 & 0.18 & 337930.00 & 3292604.27 & 1.00 & 0.97 & 0.97 \\
44503 & 109332 & 2004 & 22.20 & 0.09 & 2219.00 & 21095.49 & 1.00 & 0.95 & 0.95 \\
51161 & 240489 & 2004 & 60.40 & 0.03 & 6149.00 & 61468.94 & 0.98 & 1.02 & 1.00 \\
28239 & 105397 & 2004 & 87.10 & 0.30 & 7610.00 & 81567.29 & 1.14 & 0.94 & 1.07 \\
54529 & 373714 & 2004 & 209.10 & 0.28 & 20200.00 & 167967.91 & 1.04 & 0.80 & 0.83 \\
35337 & 106348 & 2004 & 132.30 & 0.16 & 11923.00 & 121691.21 & 1.11 & 0.92 & 1.02 \\
51489 & 240525 & 2004 & 57.90 & 0.04 & 5817.00 & 58166.55 & 1.00 & 1.00 & 1.00 \\
7594 & 101047 & 2004 & 238.70 & 0.11 & 22376.00 & 238858.51 & 1.07 & 1.00 & 1.07 \\
47543 & 212658 & 2004 & 13219.50 & 0.18 & 1320173.00 & 12492417.99 & 1.00 & 0.94 & 0.95 \\
58076 & 410075 & 2004 & 803.40 & 0.47 & 82203.00 & 709956.45 & 0.98 & 0.88 & 0.86 \\
43719 & 109208 & 2004 & 54.90 & 0.09 & 5700.00 & 52777.51 & 0.96 & 0.96 & 0.93 \\
56097 & 400173 & 2004 & 32.10 & 0.06 & 3199.00 & 31201.55 & 1.00 & 0.97 & 0.98 \\
39032 & 107573 & 2004 & 211.60 & 0.11 & 25378.00 & 214518.84 & 0.83 & 1.01 & 0.85 \\
12510 & 101544 & 2004 & 189.30 & -0.03 & 18921.00 & 186960.55 & 1.00 & 0.99 & 0.99 \\
54081 & 364391 & 2004 & 42.20 & 0.12 & 3827.00 & 40263.57 & 1.10 & 0.95 & 1.05 \\
39026 & 107566 & 2004 & 4.50 & 0.21 & 404.00 & 4225.01 & 1.11 & 0.94 & 1.05 \\
33541 & 106147 & 2004 & 60.20 & 0.11 & 5988.00 & 58723.56 & 1.01 & 0.98 & 0.98 \\
64445 & 500603 & 2004 & 434.50 & 0.12 & 43412.00 & 432022.18 & 1.00 & 0.99 & 1.00 \\
56117 & 400174 & 2004 & 145.90 & 0.04 & 12788.00 & 140017.94 & 1.14 & 0.96 & 1.09 \\
52193 & 302676 & 2004 & 806.00 & 0.19 & 71281.00 & 787291.68 & 1.13 & 0.98 & 1.10 \\
40264 & 108083 & 2004 & 207.40 & 0.18 & 20746.00 & 202174.20 & 1.00 & 0.97 & 0.97 \\
55687 & 400138 & 2004 & 108.30 & 0.03 & 10747.00 & 93359.63 & 1.01 & 0.86 & 0.87 \\
54919 & 400025 & 2004 & 136.90 & 0.15 & 13720.00 & 135621.88 & 1.00 & 0.99 & 0.99 \\
65511 & 500701 & 2004 & 307.00 & -0.14 & 30712.00 & 282999.72 & 1.00 & 0.92 & 0.92 \\
58435 & 410158 & 2004 & 129.10 & 0.27 & 9697.00 & 99741.52 & 1.33 & 0.77 & 1.03 \\
49410 & 240293 & 2004 & 2291.50 & 0.11 & 213511.00 & 2184491.26 & 1.07 & 0.95 & 1.02 \\
25831 & 103524 & 2004 & 106578.70 & 0.11 & 9830973.00 & 105370839.35 & 1.08 & 0.99 & 1.07 \\
52483 & 302964 & 2004 & 172.80 & 0.22 & 12317.00 & 125833.76 & 1.40 & 0.73 & 1.02 \\
48005 & 225696 & 2004 & 97.20 & 0.11 & 9363.00 & 93212.62 & 1.04 & 0.96 & 1.00 \\
8213 & 101079 & 2004 & 569.70 & 0.32 & 59458.00 & 469579.41 & 0.96 & 0.82 & 0.79 \\
25797 & 103523 & 2004 & 9203.70 & 0.19 & 827707.00 & 8676482.13 & 1.11 & 0.94 & 1.05 \\
20192 & 102676 & 2004 & 158.20 & 0.10 & 15058.00 & 156171.89 & 1.05 & 0.99 & 1.04 \\
33483 & 106140 & 2004 & 1258.70 & 0.06 & 153854.00 & 1255467.40 & 0.82 & 1.00 & 0.82 \\
65488 & 500700 & 2004 & 179.00 & 0.10 & 17905.00 & 178864.77 & 1.00 & 1.00 & 1.00 \\
9635 & 101160 & 2004 & 559.50 & 0.22 & 55638.00 & 548858.53 & 1.01 & 0.98 & 0.99 \\
40237 & 108082 & 2004 & 68.00 & 0.16 & 6823.00 & 59784.23 & 1.00 & 0.88 & 0.88 \\
25763 & 103521 & 2004 & 5879.80 & 0.15 & 495499.00 & 5372300.75 & 1.19 & 0.91 & 1.08 \\
3788 & 100481 & 2004 & 149.30 & 0.16 & 14607.00 & 147246.03 & 1.02 & 0.99 & 1.01 \\
33509 & 106143 & 2004 & 113.40 & 0.15 & 13033.00 & 111740.32 & 0.87 & 0.99 & 0.86 \\
39086 & 107604 & 2004 & 803.80 & 0.09 & 74382.00 & 761100.91 & 1.08 & 0.95 & 1.02 \\
24168 & 103294 & 2004 & 6694.80 & 0.15 & 624073.00 & 6012689.95 & 1.07 & 0.90 & 0.96 \\
52219 & 302677 & 2004 & 39.40 & 0.05 & 3840.00 & 40635.47 & 1.03 & 1.03 & 1.06 \\
11994 & 101476 & 2004 & 2685.70 & 0.10 & 257531.00 & 2605318.19 & 1.04 & 0.97 & 1.01 \\
39061 & 107598 & 2004 & 134.70 & 0.10 & 13060.00 & 132303.37 & 1.03 & 0.98 & 1.01 \\
32186 & 105999 & 2004 & 2065.10 & 0.14 & 161366.00 & 1394108.60 & 1.28 & 0.68 & 0.86 \\
48014 & 226438 & 2004 & 567.70 & 0.06 & 53116.00 & 550887.69 & 1.07 & 0.97 & 1.04 \\
44829 & 109393 & 2004 & 12.10 & 0.10 & 1111.00 & 11892.72 & 1.09 & 0.98 & 1.07 \\
2011 & 100280 & 2004 & 57.20 & 0.10 & 5701.00 & 56851.75 & 1.00 & 0.99 & 1.00 \\
13589 & 101744 & 2004 & 1476.70 & 0.18 & 140703.00 & 1431830.14 & 1.05 & 0.97 & 1.02 \\
1107 & 100153 & 2004 & 90.80 & 0.14 & 9205.00 & 89681.37 & 0.99 & 0.99 & 0.97 \\
4969 & 100697 & 2004 & 353.80 & 0.17 & 33403.00 & 318902.13 & 1.06 & 0.90 & 0.95 \\
43234 & 109086 & 2004 & 1156.60 & 0.11 & 115330.00 & 1069625.42 & 1.00 & 0.92 & 0.93 \\
15341 & 101987 & 2004 & 1475.20 & 0.22 & 123088.00 & 1375912.57 & 1.20 & 0.93 & 1.12 \\
38995 & 107563 & 2004 & 1773.90 & 0.10 & 175369.00 & 1753747.28 & 1.01 & 0.99 & 1.00 \\
49426 & 240296 & 2004 & 1382.50 & 0.35 & 130042.00 & 1338355.09 & 1.06 & 0.97 & 1.03 \\
24090 & 103264 & 2004 & 993.70 & 0.07 & 96578.00 & 1023288.76 & 1.03 & 1.03 & 1.06 \\
33578 & 106150 & 2004 & 123.70 & 0.03 & 12980.00 & 128241.95 & 0.95 & 1.04 & 0.99 \\
25898 & 103526 & 2004 & 5170.70 & 0.11 & 467978.00 & 4513310.35 & 1.10 & 0.87 & 0.96 \\
64422 & 500602 & 2004 & 324.60 & 0.17 & 32414.00 & 323718.78 & 1.00 & 1.00 & 1.00 \\
32108 & 105983 & 2004 & 293.20 & 0.13 & 25992.00 & 283679.37 & 1.13 & 0.97 & 1.09 \\
61320 & 500028 & 2004 & 626.10 & 0.34 & 42656.00 & 349148.46 & 1.47 & 0.56 & 0.82 \\
38926 & 107354 & 2004 & 79.50 & 0.22 & 6430.00 & 60912.83 & 1.24 & 0.77 & 0.95 \\
40320 & 108109 & 2004 & 92.00 & 0.20 & 7257.00 & 62352.15 & 1.27 & 0.68 & 0.86 \\
3758 & 100480 & 2004 & 110.60 & 0.11 & 11109.00 & 114416.08 & 1.00 & 1.03 & 1.03 \\
11960 & 101473 & 2004 & 1127.50 & 0.05 & 110550.00 & 1180975.22 & 1.02 & 1.05 & 1.07 \\
54090 & 364393 & 2004 & 173.40 & 0.12 & 15169.00 & 153353.15 & 1.14 & 0.88 & 1.01 \\
5002 & 100698 & 2004 & 73.70 & 0.14 & 7162.00 & 71619.32 & 1.03 & 0.97 & 1.00 \\
47191 & 200342 & 2004 & 10279.00 & 0.21 & 891521.00 & 8885959.73 & 1.15 & 0.86 & 1.00 \\
56121 & 400175 & 2004 & 358.70 & 0.19 & 29700.00 & 325681.73 & 1.21 & 0.91 & 1.10 \\
40346 & 108112 & 2004 & 31.10 & 0.33 & 3018.00 & 30733.05 & 1.03 & 0.99 & 1.02 \\
33589 & 106151 & 2004 & 1345.80 & 0.21 & 131046.00 & 1341673.70 & 1.03 & 1.00 & 1.02 \\
25927 & 103529 & 2004 & 7222.20 & 0.11 & 662776.00 & 6872384.43 & 1.09 & 0.95 & 1.04 \\
38909 & 107352 & 2004 & 296.40 & 0.08 & 29715.00 & 273274.32 & 1.00 & 0.92 & 0.92 \\
24059 & 103259 & 2004 & 2857.10 & 0.14 & 273040.00 & 2697752.36 & 1.05 & 0.94 & 0.99 \\
17681 & 102342 & 2004 & 136.30 & 0.09 & 13726.00 & 135467.20 & 0.99 & 0.99 & 0.99 \\
1077 & 100150 & 2004 & 35.30 & 0.21 & 3288.00 & 26690.59 & 1.07 & 0.76 & 0.81 \\
55250 & 400075 & 2004 & 1368.40 & 0.17 & 136721.00 & 1348415.57 & 1.00 & 0.99 & 0.99 \\
40294 & 108087 & 2004 & 28.60 & 0.05 & 2961.00 & 29069.82 & 0.97 & 1.02 & 0.98 \\
38989 & 107470 & 2004 & 1.70 & -0.01 & 165.00 & 1523.91 & 1.03 & 0.90 & 0.92 \\
9187 & 101116 & 2004 & 961.60 & 0.21 & 88451.00 & 894240.50 & 1.09 & 0.93 & 1.01 \\
24118 & 103267 & 2004 & 4192.90 & 0.13 & 412874.00 & 4129056.95 & 1.02 & 0.98 & 1.00 \\
33555 & 106148 & 2004 & 149.90 & 0.07 & 14993.00 & 136978.53 & 1.00 & 0.91 & 0.91 \\
45447 & 200060 & 2004 & 1636.60 & 0.14 & 157849.00 & 1563934.23 & 1.04 & 0.96 & 0.99 \\
12522 & 101545 & 2004 & 335.70 & 0.16 & 31501.00 & 315019.40 & 1.07 & 0.94 & 1.00 \\
65534 & 500702 & 2004 & 179.10 & 0.13 & 17925.00 & 179144.45 & 1.00 & 1.00 & 1.00 \\
49418 & 240295 & 2004 & 2410.70 & 0.10 & 225879.00 & 2255583.74 & 1.07 & 0.94 & 1.00 \\
43744 & 109217 & 2004 & 76.90 & 0.31 & 5371.00 & 62032.67 & 1.43 & 0.81 & 1.15 \\
25864 & 103525 & 2004 & 48483.30 & 0.11 & 4324897.00 & 46552629.71 & 1.12 & 0.96 & 1.08 \\
6383 & 100856 & 2004 & 135.10 & 0.16 & 13519.00 & 130514.00 & 1.00 & 0.97 & 0.97 \\
19608 & 102636 & 2004 & 1445.50 & 0.21 & 144798.00 & 1394543.38 & 1.00 & 0.96 & 0.96 \\
9666 & 101161 & 2004 & 1235.60 & 0.13 & 123350.00 & 1196067.77 & 1.00 & 0.97 & 0.97 \\
4707 & 100667 & 2004 & 16.10 & 0.10 & 1615.00 & 14550.53 & 1.00 & 0.90 & 0.90 \\
7050 & 100992 & 2004 & 720.10 & 0.15 & 72981.00 & 674988.59 & 0.99 & 0.94 & 0.92 \\
17624 & 102321 & 2004 & 158.70 & 0.14 & 13852.00 & 142540.30 & 1.15 & 0.90 & 1.03 \\
24102 & 103266 & 2004 & 197.70 & 0.19 & 19536.00 & 192480.79 & 1.01 & 0.97 & 0.99 \\
33566 & 106149 & 2004 & 193.30 & 0.07 & 19354.00 & 172698.08 & 1.00 & 0.89 & 0.89 \\
38951 & 107358 & 2004 & 506.10 & 0.31 & 50396.00 & 418528.60 & 1.00 & 0.83 & 0.83 \\
52182 & 302627 & 2004 & 128.50 & 0.22 & 14410.00 & 129088.70 & 0.89 & 1.00 & 0.90 \\
48451 & 240085 & 2004 & 197.70 & 0.10 & 19879.00 & 194314.17 & 0.99 & 0.98 & 0.98 \\
20179 & 102673 & 2004 & 572.60 & 0.17 & 54237.00 & 571297.58 & 1.06 & 1.00 & 1.05 \\
39098 & 107605 & 2004 & 1571.30 & 0.28 & 114986.00 & 1329039.72 & 1.37 & 0.85 & 1.16 \\
33385 & 106127 & 2004 & 670.20 & 0.13 & 72402.00 & 655435.29 & 0.93 & 0.98 & 0.91 \\
39213 & 107619 & 2004 & 357.90 & 0.11 & 34819.00 & 321015.59 & 1.03 & 0.90 & 0.92 \\
4744 & 100670 & 2004 & 69.60 & 0.06 & 6829.00 & 72578.41 & 1.02 & 1.04 & 1.06 \\
32298 & 106009 & 2004 & 1661.20 & 0.03 & 164466.00 & 1706458.69 & 1.01 & 1.03 & 1.04 \\
1189 & 100159 & 2004 & 35.00 & 0.11 & 3990.00 & 32780.33 & 0.88 & 0.94 & 0.82 \\
54059 & 364292 & 2004 & 424.00 & 0.06 & 41767.00 & 414912.01 & 1.02 & 0.98 & 0.99 \\
55397 & 400093 & 2004 & 166.10 & 0.17 & 17922.00 & 151571.63 & 0.93 & 0.91 & 0.85 \\
6641 & 100906 & 2004 & 1658.60 & 0.05 & 167661.00 & 1420900.28 & 0.99 & 0.86 & 0.85 \\
53812 & 357133 & 2004 & 82.60 & 0.19 & 13243.00 & 121788.38 & 0.62 & 1.47 & 0.92 \\
20114 & 102667 & 2004 & 27295.90 & 0.18 & 2501024.00 & 23337824.31 & 1.09 & 0.85 & 0.93 \\
4943 & 100695 & 2004 & 191.40 & 0.07 & 19133.00 & 168347.99 & 1.00 & 0.88 & 0.88 \\
52271 & 302731 & 2004 & 2269.60 & 0.19 & 196627.00 & 1962034.65 & 1.15 & 0.86 & 1.00 \\
53792 & 357122 & 2004 & 881.30 & 0.14 & 78903.00 & 800040.24 & 1.12 & 0.91 & 1.01 \\
45384 & 200055 & 2004 & 138.20 & 0.11 & 13870.00 & 126182.10 & 1.00 & 0.91 & 0.91 \\
58428 & 410157 & 2004 & 69.40 & 0.23 & 8167.00 & 70172.11 & 0.85 & 1.01 & 0.86 \\
25633 & 103498 & 2004 & 384.00 & 0.18 & 33770.00 & 329459.98 & 1.14 & 0.86 & 0.98 \\
39199 & 107618 & 2004 & 2999.80 & 0.03 & 331920.00 & 2835901.82 & 0.90 & 0.95 & 0.85 \\
40166 & 108070 & 2004 & 54.00 & 0.21 & 5168.00 & 51676.46 & 1.04 & 0.96 & 1.00 \\
43289 & 109090 & 2004 & 81.30 & 0.23 & 8128.00 & 80780.27 & 1.00 & 0.99 & 0.99 \\
33404 & 106129 & 2004 & 438.80 & 0.07 & 43483.00 & 434834.55 & 1.01 & 0.99 & 1.00 \\
39174 & 107616 & 2004 & 402.80 & 0.19 & 40229.00 & 398500.98 & 1.00 & 0.99 & 0.99 \\
24269 & 103301 & 2004 & 953.90 & 0.14 & 95111.00 & 940424.21 & 1.00 & 0.99 & 0.99 \\
52245 & 302698 & 2004 & 193.60 & 0.15 & 15431.00 & 146485.49 & 1.25 & 0.76 & 0.95 \\
53835 & 357756 & 2004 & 75.80 & 0.07 & 7478.00 & 67476.15 & 1.01 & 0.89 & 0.90 \\
32336 & 106011 & 2004 & 395.30 & 0.22 & 38635.00 & 387834.94 & 1.02 & 0.98 & 1.00 \\
44882 & 109397 & 2004 & 693.30 & 0.15 & 68021.00 & 589295.75 & 1.02 & 0.85 & 0.87 \\
25558 & 103496 & 2004 & 318.60 & 0.09 & 30995.00 & 330227.64 & 1.03 & 1.04 & 1.07 \\
40132 & 108037 & 2004 & 173.30 & 0.02 & 18088.00 & 162793.76 & 0.96 & 0.94 & 0.90 \\
43317 & 109094 & 2004 & 178.50 & 0.10 & 16351.00 & 171908.64 & 1.09 & 0.96 & 1.05 \\
53844 & 357762 & 2004 & 42.70 & 0.09 & 4074.00 & 39098.36 & 1.05 & 0.92 & 0.96 \\
47304 & 200505 & 2004 & 579.80 & 0.10 & 51929.00 & 535526.25 & 1.12 & 0.92 & 1.03 \\
39258 & 107627 & 2004 & 1087.70 & 0.15 & 123553.00 & 896668.28 & 0.88 & 0.82 & 0.73 \\
43313 & 109093 & 2004 & 21.80 & 0.14 & 1980.00 & 21149.05 & 1.10 & 0.97 & 1.07 \\
24287 & 103304 & 2004 & 57.70 & 0.11 & 5838.00 & 59939.65 & 0.99 & 1.04 & 1.03 \\
39241 & 107626 & 2004 & 911.50 & 0.09 & 100958.00 & 844025.42 & 0.90 & 0.93 & 0.84 \\
43306 & 109092 & 2004 & 69.40 & 0.17 & 6348.00 & 67766.42 & 1.09 & 0.98 & 1.07 \\
1227 & 100166 & 2004 & 18384.00 & 0.09 & 1692277.00 & 13645109.99 & 1.09 & 0.74 & 0.81 \\
54048 & 364291 & 2004 & 87.70 & 0.13 & 8743.00 & 85738.73 & 1.00 & 0.98 & 0.98 \\
33359 & 106124 & 2004 & 2101.20 & 0.17 & 200534.00 & 1869145.37 & 1.05 & 0.89 & 0.93 \\
9587 & 101151 & 2004 & 234.20 & 0.25 & 22353.00 & 196079.70 & 1.05 & 0.84 & 0.88 \\
25589 & 103497 & 2004 & 372.50 & 0.17 & 27515.00 & 310853.02 & 1.35 & 0.83 & 1.13 \\
40156 & 108051 & 2004 & 279.90 & 0.10 & 28207.00 & 287242.09 & 0.99 & 1.03 & 1.02 \\
32311 & 106010 & 2004 & 2738.30 & 0.12 & 254124.00 & 2379612.56 & 1.08 & 0.87 & 0.94 \\
60742 & 410722 & 2004 & 199.60 & 0.15 & 17108.00 & 189660.10 & 1.17 & 0.95 & 1.11 \\
39232 & 107623 & 2004 & 3.60 & 0.08 & 330.00 & 2863.14 & 1.09 & 0.80 & 0.87 \\
17735 & 102350 & 2004 & 349.30 & 0.04 & 34840.00 & 342526.65 & 1.00 & 0.98 & 0.98 \\
9685 & 101165 & 2004 & 2536.20 & 0.10 & 256590.00 & 2498902.39 & 0.99 & 0.99 & 0.97 \\
24238 & 103299 & 2004 & 297.80 & 0.12 & 31893.00 & 300534.22 & 0.93 & 1.01 & 0.94 \\
4728 & 100669 & 2004 & 425.30 & 0.05 & 43559.00 & 423250.84 & 0.98 & 1.00 & 0.97 \\
24203 & 103296 & 2004 & 718.40 & 0.09 & 80588.00 & 714401.24 & 0.89 & 0.99 & 0.89 \\
14957 & 101925 & 2004 & 9482.20 & 0.24 & 873839.00 & 8649841.34 & 1.09 & 0.91 & 0.99 \\
18231 & 102417 & 2004 & 981.80 & 0.04 & 94066.00 & 899996.54 & 1.04 & 0.92 & 0.96 \\
58103 & 410093 & 2004 & 106.50 & 0.20 & 10665.00 & 97172.88 & 1.00 & 0.91 & 0.91 \\
9616 & 101158 & 2004 & 414.30 & 0.00 & 41431.00 & 394681.38 & 1.00 & 0.95 & 0.95 \\
20158 & 102671 & 2004 & 203.30 & 0.05 & 20335.00 & 203343.17 & 1.00 & 1.00 & 1.00 \\
39133 & 107608 & 2004 & 74.80 & 0.14 & 7572.00 & 72519.73 & 0.99 & 0.97 & 0.96 \\
49078 & 240218 & 2004 & 629.90 & 0.17 & 48564.00 & 498154.85 & 1.30 & 0.79 & 1.03 \\
64484 & 500605 & 2004 & 498.30 & 0.32 & 49854.00 & 498213.29 & 1.00 & 1.00 & 1.00 \\
33450 & 106136 & 2004 & 159.80 & 0.17 & 15919.00 & 154568.87 & 1.00 & 0.97 & 0.97 \\
14201 & 101820 & 2004 & 197.70 & 0.01 & 19357.00 & 188483.78 & 1.02 & 0.95 & 0.97 \\
43257 & 109087 & 2004 & 116.20 & 0.20 & 11618.00 & 109135.29 & 1.00 & 0.94 & 0.94 \\
15305 & 101982 & 2004 & 266.20 & 0.09 & 26630.00 & 265300.09 & 1.00 & 1.00 & 1.00 \\
49400 & 240291 & 2004 & 13.50 & 0.05 & 1266.00 & 13840.83 & 1.07 & 1.03 & 1.09 \\
44859 & 109395 & 2004 & 322.40 & 0.18 & 34945.00 & 261840.24 & 0.92 & 0.81 & 0.75 \\
25731 & 103520 & 2004 & 10214.60 & 0.07 & 959608.00 & 10356003.79 & 1.06 & 1.01 & 1.08 \\
13707 & 101758 & 2004 & 211.00 & 0.14 & 21091.00 & 206318.32 & 1.00 & 0.98 & 0.98 \\
39123 & 107607 & 2004 & 100.70 & 0.17 & 10126.00 & 96192.06 & 0.99 & 0.96 & 0.95 \\
40226 & 108074 & 2004 & 72.40 & 0.05 & 6200.00 & 65247.58 & 1.17 & 0.90 & 1.05 \\
32214 & 106000 & 2004 & 317.10 & 0.10 & 37526.00 & 386173.72 & 0.85 & 1.22 & 1.03 \\
45421 & 200058 & 2004 & 3236.90 & 0.12 & 311065.00 & 3110712.00 & 1.04 & 0.96 & 1.00 \\
61263 & 500025 & 2004 & 4.00 & 0.12 & 368.00 & 3987.99 & 1.09 & 1.00 & 1.08 \\
39149 & 107611 & 2004 & 3746.40 & 0.20 & 300859.00 & 3350113.57 & 1.25 & 0.89 & 1.11 \\
25699 & 103514 & 2004 & 3843.20 & 0.14 & 346313.00 & 3375670.07 & 1.11 & 0.88 & 0.97 \\
65401 & 500694 & 2004 & 535.00 & 0.33 & 53243.00 & 532427.19 & 1.00 & 1.00 & 1.00 \\
32270 & 106008 & 2004 & 313.60 & 0.11 & 34221.00 & 281915.33 & 0.92 & 0.90 & 0.82 \\
1987 & 100278 & 2004 & 62.60 & 0.09 & 6253.00 & 61441.10 & 1.00 & 0.98 & 0.98 \\
65440 & 500697 & 2004 & 17.00 & 0.12 & 1707.00 & 15093.65 & 1.00 & 0.89 & 0.88 \\
19640 & 102639 & 2004 & 179.60 & 0.35 & 18049.00 & 176629.68 & 1.00 & 0.98 & 0.98 \\
47463 & 211485 & 2004 & 25.60 & 0.14 & 2556.00 & 24707.67 & 1.00 & 0.97 & 0.97 \\
53779 & 357075 & 2004 & 336.20 & 0.11 & 28789.00 & 317775.86 & 1.17 & 0.95 & 1.10 \\
45410 & 200057 & 2004 & 1081.20 & 0.19 & 97078.00 & 1019786.73 & 1.11 & 0.94 & 1.05 \\
17712 & 102349 & 2004 & 782.00 & 0.08 & 78418.00 & 740802.82 & 1.00 & 0.95 & 0.94 \\
64507 & 500606 & 2004 & 933.90 & 0.04 & 93399.00 & 933542.28 & 1.00 & 1.00 & 1.00 \\
65465 & 500699 & 2004 & 194.00 & 0.02 & 19408.00 & 191645.21 & 1.00 & 0.99 & 0.99 \\
56070 & 400171 & 2004 & 75.70 & 0.16 & 7553.00 & 74395.86 & 1.00 & 0.98 & 0.98 \\
1157 & 100157 & 2004 & 1036.50 & 0.10 & 97261.00 & 1032138.79 & 1.07 & 1.00 & 1.06 \\
7982 & 101068 & 2004 & 63295.80 & 0.07 & 6201241.00 & 63947449.93 & 1.02 & 1.01 & 1.03 \\
52476 & 302944 & 2004 & 281.30 & 0.14 & 31004.00 & 267654.68 & 0.91 & 0.95 & 0.86 \\
1142 & 100155 & 2004 & 1625.90 & 0.04 & 161760.00 & 1589593.21 & 1.01 & 0.98 & 0.98 \\
40186 & 108071 & 2004 & 117.90 & 0.08 & 11846.00 & 116079.10 & 1.00 & 0.98 & 0.98 \\
33423 & 106135 & 2004 & 40.00 & 0.16 & 3779.00 & 35200.52 & 1.06 & 0.88 & 0.93 \\
40212 & 108073 & 2004 & 359.10 & 0.37 & 30340.00 & 345811.42 & 1.18 & 0.96 & 1.14 \\
55663 & 400135 & 2004 & 17.70 & -0.00 & 1974.00 & 14706.26 & 0.90 & 0.83 & 0.74 \\
38885 & 107350 & 2004 & 5218.20 & 0.05 & 511342.00 & 4962847.00 & 1.02 & 0.95 & 0.97 \\
23899 & 103228 & 2004 & 104.00 & 0.02 & 10437.00 & 98150.11 & 1.00 & 0.94 & 0.94 \\
43182 & 109076 & 2004 & 175.30 & 0.05 & 18707.00 & 175302.76 & 0.94 & 1.00 & 0.94 \\
38667 & 107306 & 2004 & 239.90 & 0.10 & 23871.00 & 241279.72 & 1.00 & 1.01 & 1.01 \\
47115 & 200334 & 2004 & 134.80 & 0.27 & 10119.00 & 98508.05 & 1.33 & 0.73 & 0.97 \\
96741 & 611010 & 2004 & 123.20 & 0.16 & 12209.00 & 114998.39 & 1.01 & 0.93 & 0.94 \\
7912 & 101065 & 2004 & 1849.70 & 0.12 & 192889.00 & 1692528.75 & 0.96 & 0.92 & 0.88 \\
49474 & 240302 & 2004 & 6.20 & 0.07 & 553.00 & 5665.75 & 1.12 & 0.91 & 1.02 \\
38659 & 107303 & 2004 & 49.70 & 0.05 & 4839.00 & 51632.81 & 1.03 & 1.04 & 1.07 \\
58560 & 410167 & 2004 & 18.40 & 0.13 & 1782.00 & 18367.65 & 1.03 & 1.00 & 1.03 \\
55418 & 400094 & 2004 & 921.00 & 0.18 & 90231.00 & 883063.74 & 1.02 & 0.96 & 0.98 \\
31905 & 105957 & 2004 & 8.70 & 0.10 & 880.00 & 8752.39 & 0.99 & 1.01 & 0.99 \\
43174 & 109074 & 2004 & 138.30 & 0.07 & 12871.00 & 127084.85 & 1.07 & 0.92 & 0.99 \\
65689 & 500719 & 2004 & 131.50 & 0.16 & 12422.00 & 120389.87 & 1.06 & 0.92 & 0.97 \\
47093 & 200333 & 2004 & 2391.50 & 0.20 & 207060.00 & 2231282.17 & 1.15 & 0.93 & 1.08 \\
19559 & 102624 & 2004 & 97.70 & 0.14 & 9984.00 & 96498.11 & 0.98 & 0.99 & 0.97 \\
9773 & 101192 & 2004 & 375.90 & 0.16 & 32391.00 & 346416.45 & 1.16 & 0.92 & 1.07 \\
23881 & 103226 & 2004 & 101.40 & 0.05 & 10003.00 & 95161.06 & 1.01 & 0.94 & 0.95 \\
49482 & 240303 & 2004 & 10.60 & 0.03 & 1063.00 & 9721.78 & 1.00 & 0.92 & 0.91 \\
5055 & 100710 & 2004 & 1404.60 & 0.05 & 140365.00 & 1387086.90 & 1.00 & 0.99 & 0.99 \\
26218 & 103546 & 2004 & 21013.90 & 0.13 & 2937814.00 & 29090982.18 & 0.72 & 1.38 & 0.99 \\
31893 & 105951 & 2004 & 141.10 & 0.14 & 16111.00 & 151896.79 & 0.88 & 1.08 & 0.94 \\
31920 & 105960 & 2004 & 184.10 & 0.03 & 18082.00 & 190987.52 & 1.02 & 1.04 & 1.06 \\
47088 & 200332 & 2004 & 188.50 & 0.15 & 16162.00 & 169252.69 & 1.17 & 0.90 & 1.05 \\
26178 & 103545 & 2004 & 34570.30 & 0.18 & 1953652.00 & 18444603.70 & 1.77 & 0.53 & 0.94 \\
40542 & 108137 & 2004 & 1562.10 & 0.17 & 139811.00 & 1390205.65 & 1.12 & 0.89 & 0.99 \\
40535 & 108136 & 2004 & 31.70 & 0.12 & 2374.00 & 24240.85 & 1.34 & 0.76 & 1.02 \\
45548 & 200074 & 2004 & 10.30 & 0.10 & 1468.00 & 12632.61 & 0.70 & 1.23 & 0.86 \\
54178 & 364809 & 2004 & 44.30 & 0.15 & 4082.00 & 40690.24 & 1.09 & 0.92 & 1.00 \\
20340 & 102716 & 2004 & 912.80 & 0.15 & 108795.00 & 839797.33 & 0.84 & 0.92 & 0.77 \\
26144 & 103544 & 2004 & 57014.10 & 0.13 & 5094904.00 & 54422451.55 & 1.12 & 0.95 & 1.07 \\
44776 & 109375 & 2004 & 94.00 & 0.21 & 9518.00 & 90762.68 & 0.99 & 0.97 & 0.95 \\
43195 & 109079 & 2004 & 39.60 & 0.10 & 4594.00 & 35935.12 & 0.86 & 0.91 & 0.78 \\
23915 & 103232 & 2004 & 174.30 & 0.09 & 17406.00 & 168617.75 & 1.00 & 0.97 & 0.97 \\
45555 & 200075 & 2004 & 358.10 & 0.21 & 35779.00 & 338353.37 & 1.00 & 0.94 & 0.95 \\
31941 & 105963 & 2004 & 733.20 & 0.06 & 69947.00 & 695295.98 & 1.05 & 0.95 & 0.99 \\
15429 & 101990 & 2004 & 554.30 & 0.17 & 55708.00 & 506148.46 & 1.00 & 0.91 & 0.91 \\
38715 & 107309 & 2004 & 50.80 & 0.11 & 7225.00 & 50533.29 & 0.70 & 0.99 & 0.70 \\
58555 & 410166 & 2004 & 75.40 & 0.04 & 7523.00 & 74993.00 & 1.00 & 0.99 & 1.00 \\
52165 & 302545 & 2004 & 116.00 & 0.10 & 10869.00 & 120144.93 & 1.07 & 1.04 & 1.11 \\
43189 & 109077 & 2004 & 6.80 & 0.18 & 647.00 & 6369.28 & 1.05 & 0.94 & 0.98 \\
33705 & 106163 & 2004 & 995.60 & 0.17 & 99544.00 & 950621.71 & 1.00 & 0.95 & 0.95 \\
38690 & 107308 & 2004 & 2802.00 & 0.16 & 284157.00 & 2674873.26 & 0.99 & 0.95 & 0.94 \\
983 & 100113 & 2004 & 883.50 & 0.19 & 88650.00 & 868145.30 & 1.00 & 0.98 & 0.98 \\
64353 & 500598 & 2004 & 4197.50 & 0.22 & 419580.00 & 4187081.73 & 1.00 & 1.00 & 1.00 \\
45580 & 200079 & 2004 & 16.60 & 0.12 & 1593.00 & 16238.61 & 1.04 & 0.98 & 1.02 \\
47135 & 200335 & 2004 & 1.70 & 0.03 & 146.00 & 1608.58 & 1.16 & 0.95 & 1.10 \\
17510 & 102317 & 2004 & 27.10 & 0.20 & 2741.00 & 25445.99 & 0.99 & 0.94 & 0.93 \\
52548 & 303123 & 2004 & 93.90 & 0.11 & 9436.00 & 83055.65 & 1.00 & 0.88 & 0.88 \\
18325 & 102425 & 2004 & 1500.60 & 0.04 & 147359.00 & 1442970.35 & 1.02 & 0.96 & 0.98 \\
4646 & 100659 & 2004 & 849.60 & 0.13 & 143327.00 & 1393020.63 & 0.59 & 1.64 & 0.97 \\
52142 & 302067 & 2004 & 23.20 & 0.01 & 1993.00 & 21184.14 & 1.16 & 0.91 & 1.06 \\
47951 & 225413 & 2004 & 174.90 & 0.09 & 17543.00 & 174828.24 & 1.00 & 1.00 & 1.00 \\
17478 & 102312 & 2004 & 12.20 & 0.14 & 1125.00 & 12209.85 & 1.08 & 1.00 & 1.09 \\
33748 & 106165 & 2004 & 67.70 & 0.18 & 6016.00 & 65501.64 & 1.13 & 0.97 & 1.09 \\
31846 & 105947 & 2004 & 41.30 & 0.17 & 4177.00 & 39915.03 & 0.99 & 0.97 & 0.96 \\
40582 & 108140 & 2004 & 102.20 & 0.10 & 10117.00 & 96347.99 & 1.01 & 0.94 & 0.95 \\
53703 & 355988 & 2004 & 75.00 & 0.10 & 6933.00 & 68350.37 & 1.08 & 0.91 & 0.99 \\
43169 & 109072 & 2004 & 124.00 & 0.00 & 12613.00 & 130431.50 & 0.98 & 1.05 & 1.03 \\
26296 & 103564 & 2004 & 508.90 & 0.17 & 50917.00 & 495691.62 & 1.00 & 0.97 & 0.97 \\
15463 & 101992 & 2004 & 207.80 & 0.14 & 17658.00 & 157486.91 & 1.18 & 0.76 & 0.89 \\
23830 & 103214 & 2004 & 1525.50 & 0.13 & 144210.00 & 1570877.21 & 1.06 & 1.03 & 1.09 \\
96707 & 611008 & 2004 & 70.60 & 0.14 & 12141.00 & 119809.80 & 0.58 & 1.70 & 0.99 \\
31835 & 105946 & 2004 & 162.80 & 0.19 & 16343.00 & 155164.02 & 1.00 & 0.95 & 0.95 \\
43163 & 109071 & 2004 & 15.00 & 0.09 & 1556.00 & 16049.90 & 0.96 & 1.07 & 1.03 \\
54183 & 364818 & 2004 & 659.90 & 0.23 & 61444.00 & 509594.46 & 1.07 & 0.77 & 0.83 \\
64307 & 500596 & 2004 & 3070.70 & 0.06 & 306948.00 & 3068174.81 & 1.00 & 1.00 & 1.00 \\
26314 & 103567 & 2004 & 671.90 & 0.14 & 68766.00 & 649863.22 & 0.98 & 0.97 & 0.95 \\
12600 & 101557 & 2004 & 21.00 & 0.04 & 2084.00 & 20339.08 & 1.01 & 0.97 & 0.98 \\
38634 & 107302 & 2004 & 363.10 & 0.10 & 35531.00 & 364438.82 & 1.02 & 1.00 & 1.03 \\
939 & 100112 & 2004 & 7346.10 & 0.12 & 734713.00 & 6952941.26 & 1.00 & 0.95 & 0.95 \\
58623 & 410180 & 2004 & 53.60 & 0.32 & 5365.00 & 51955.28 & 1.00 & 0.97 & 0.97 \\
52154 & 302206 & 2004 & 852.40 & 0.09 & 77422.00 & 755659.21 & 1.10 & 0.89 & 0.98 \\
14916 & 101922 & 2004 & 2153.70 & 0.15 & 184330.00 & 1870501.07 & 1.17 & 0.87 & 1.01 \\
38624 & 107300 & 2004 & 62.00 & 0.05 & 6024.00 & 64216.84 & 1.03 & 1.04 & 1.07 \\
33732 & 106164 & 2004 & 60.90 & -0.03 & 6132.00 & 59549.80 & 0.99 & 0.98 & 0.97 \\
65743 & 500729 & 2004 & 374.10 & 0.24 & 37416.00 & 317879.23 & 1.00 & 0.85 & 0.85 \\
56200 & 400181 & 2004 & 19.50 & 0.19 & 2203.00 & 22025.07 & 0.89 & 1.13 & 1.00 \\
11879 & 101464 & 2004 & 1912.40 & 0.19 & 201336.00 & 1857612.09 & 0.95 & 0.97 & 0.92 \\
23863 & 103224 & 2004 & 102.50 & 0.15 & 10273.00 & 99931.73 & 1.00 & 0.97 & 0.97 \\
26249 & 103547 & 2004 & 6246.10 & 0.07 & 624846.00 & 5873731.43 & 1.00 & 0.94 & 0.94 \\
61484 & 500082 & 2004 & 531.00 & 0.05 & 53288.00 & 512678.58 & 1.00 & 0.97 & 0.96 \\
48108 & 240010 & 2004 & 437.80 & 0.34 & 43750.00 & 366810.58 & 1.00 & 0.84 & 0.84 \\
64330 & 500597 & 2004 & 11784.40 & 0.11 & 1177214.00 & 11768474.91 & 1.00 & 1.00 & 1.00 \\
31868 & 105949 & 2004 & 869.20 & 0.20 & 75078.00 & 803173.64 & 1.16 & 0.92 & 1.07 \\
14280 & 101842 & 2004 & 1268.40 & 0.08 & 125412.00 & 1237827.52 & 1.01 & 0.98 & 0.99 \\
40553 & 108138 & 2004 & 24.40 & 0.13 & 2103.00 & 17165.62 & 1.16 & 0.70 & 0.82 \\
4067 & 100544 & 2004 & 785.40 & 0.36 & 67604.00 & 800640.09 & 1.16 & 1.02 & 1.18 \\
31857 & 105948 & 2004 & 34.90 & 0.22 & 3480.00 & 34105.61 & 1.00 & 0.98 & 0.98 \\
20391 & 102733 & 2004 & 3085.90 & 0.31 & 274684.00 & 2363140.88 & 1.12 & 0.77 & 0.86 \\
54025 & 363941 & 2004 & 19.00 & 0.21 & 1942.00 & 16854.67 & 0.98 & 0.89 & 0.87 \\
9755 & 101186 & 2004 & 504.70 & 0.06 & 47206.00 & 487916.56 & 1.07 & 0.97 & 1.03 \\
12551 & 101553 & 2004 & 162.60 & 0.02 & 16136.00 & 159103.01 & 1.01 & 0.98 & 0.99 \\
32048 & 105977 & 2004 & 1188.70 & 0.15 & 120804.00 & 1124944.35 & 0.98 & 0.95 & 0.93 \\
17587 & 102319 & 2004 & 556.90 & 0.12 & 55695.00 & 529375.80 & 1.00 & 0.95 & 0.95 \\
56142 & 400176 & 2004 & 636.20 & 0.17 & 49250.00 & 522648.51 & 1.29 & 0.82 & 1.06 \\
40424 & 108118 & 2004 & 82.70 & 0.09 & 8281.00 & 77752.58 & 1.00 & 0.94 & 0.94 \\
4053 & 100543 & 2004 & 400.30 & 0.18 & 37248.00 & 325104.43 & 1.07 & 0.81 & 0.87 \\
56163 & 400178 & 2004 & 67.80 & 0.06 & 11732.00 & 114005.85 & 0.58 & 1.68 & 0.97 \\
32039 & 105976 & 2004 & 9.80 & 0.08 & 971.00 & 9708.65 & 1.01 & 0.99 & 1.00 \\
65640 & 500710 & 2004 & 662.30 & 0.11 & 56598.00 & 631786.36 & 1.17 & 0.95 & 1.12 \\
61376 & 500047 & 2004 & 2.80 & 0.13 & 284.00 & 2622.76 & 0.99 & 0.94 & 0.92 \\
53716 & 356500 & 2004 & 299.40 & 0.08 & 28893.00 & 273199.23 & 1.04 & 0.91 & 0.95 \\
44790 & 109380 & 2004 & 52.20 & 0.10 & 4654.00 & 39921.91 & 1.12 & 0.76 & 0.86 \\
52523 & 302997 & 2004 & 433.40 & 0.23 & 43280.00 & 388777.34 & 1.00 & 0.90 & 0.90 \\
24007 & 103253 & 2004 & 246.10 & 0.24 & 22279.00 & 241775.95 & 1.10 & 0.98 & 1.09 \\
54136 & 364519 & 2004 & 82.90 & 0.17 & 7338.00 & 81071.16 & 1.13 & 0.98 & 1.10 \\
56178 & 400180 & 2004 & 35.70 & 0.05 & 3549.00 & 35493.51 & 1.01 & 0.99 & 1.00 \\
7396 & 101038 & 2004 & 3341.10 & 0.06 & 314639.00 & 3132107.58 & 1.06 & 0.94 & 1.00 \\
45513 & 200072 & 2004 & 8.70 & 0.23 & 887.00 & 8021.60 & 0.98 & 0.92 & 0.90 \\
54904 & 400021 & 2004 & 142.20 & 0.18 & 11121.00 & 111402.26 & 1.28 & 0.78 & 1.00 \\
14247 & 101835 & 2004 & 1453.90 & 0.11 & 144899.00 & 1371736.53 & 1.00 & 0.94 & 0.95 \\
38869 & 107338 & 2004 & 107.80 & 0.11 & 7092.00 & 70718.26 & 1.52 & 0.66 & 1.00 \\
40432 & 108119 & 2004 & 356.00 & 0.12 & 31933.00 & 340225.88 & 1.11 & 0.96 & 1.07 \\
3728 & 100475 & 2004 & 1078.50 & 0.06 & 107679.00 & 1052042.73 & 1.00 & 0.98 & 0.98 \\
40398 & 108117 & 2004 & 624.30 & 0.12 & 63065.00 & 619929.12 & 0.99 & 0.99 & 0.98 \\
45479 & 200065 & 2004 & 14.00 & 0.10 & 1397.00 & 13244.22 & 1.00 & 0.95 & 0.95 \\
53773 & 357053 & 2004 & 66.60 & 0.12 & 6669.00 & 66469.96 & 1.00 & 1.00 & 1.00 \\
33603 & 106152 & 2004 & 282.60 & 0.04 & 27674.00 & 279938.04 & 1.02 & 0.99 & 1.01 \\
65573 & 500706 & 2004 & 1131.20 & 0.20 & 112997.00 & 1128641.99 & 1.00 & 1.00 & 1.00 \\
32076 & 105980 & 2004 & 477.60 & 0.20 & 47748.00 & 456192.69 & 1.00 & 0.96 & 0.96 \\
40372 & 108115 & 2004 & 2140.40 & 0.12 & 213968.00 & 2111516.05 & 1.00 & 0.99 & 0.99 \\
65596 & 500707 & 2004 & 819.90 & 0.15 & 81981.00 & 818876.75 & 1.00 & 1.00 & 1.00 \\
48476 & 240087 & 2004 & 230.60 & 0.04 & 23083.00 & 211110.02 & 1.00 & 0.92 & 0.91 \\
13557 & 101743 & 2004 & 10081.90 & 0.24 & 1008794.00 & 8978581.12 & 1.00 & 0.89 & 0.89 \\
14232 & 101834 & 2004 & 252.80 & 0.14 & 25297.00 & 252345.33 & 1.00 & 1.00 & 1.00 \\
15371 & 101988 & 2004 & 368.20 & 0.10 & 36839.00 & 366179.55 & 1.00 & 0.99 & 0.99 \\
32064 & 105978 & 2004 & 92.30 & 0.07 & 9211.00 & 92119.74 & 1.00 & 1.00 & 1.00 \\
20259 & 102696 & 2004 & 231.00 & 0.16 & 19610.00 & 192617.42 & 1.18 & 0.83 & 0.98 \\
24037 & 103255 & 2004 & 158.80 & 0.15 & 14144.00 & 154044.36 & 1.12 & 0.97 & 1.09 \\
64054 & 500585 & 2004 & 2203.20 & 0.14 & 219936.00 & 2193701.64 & 1.00 & 1.00 & 1.00 \\
61334 & 500037 & 2004 & 2048.50 & 0.15 & 205395.00 & 1964328.80 & 1.00 & 0.96 & 0.96 \\
44800 & 109389 & 2004 & 75.80 & 0.24 & 6652.00 & 64941.08 & 1.14 & 0.86 & 0.98 \\
2039 & 100286 & 2004 & 42.80 & 0.16 & 4362.00 & 40258.69 & 0.98 & 0.94 & 0.92 \\
64376 & 500600 & 2004 & 2156.90 & 0.06 & 215476.00 & 2153854.86 & 1.00 & 1.00 & 1.00 \\
33616 & 106156 & 2004 & 7.00 & 0.10 & 703.00 & 6604.04 & 1.00 & 0.94 & 0.94 \\
38848 & 107336 & 2004 & 688.00 & 0.07 & 67400.00 & 609919.33 & 1.02 & 0.89 & 0.90 \\
38763 & 107322 & 2004 & 20.10 & 0.20 & 1822.00 & 19149.38 & 1.10 & 0.95 & 1.05 \\
31976 & 105965 & 2004 & 121.50 & 0.10 & 11551.00 & 100535.24 & 1.05 & 0.83 & 0.87 \\
43206 & 109084 & 2004 & 737.60 & 0.14 & 69185.00 & 567190.06 & 1.07 & 0.77 & 0.82 \\
65663 & 500712 & 2004 & 251.30 & 0.09 & 20628.00 & 214296.23 & 1.22 & 0.85 & 1.04 \\
47143 & 200338 & 2004 & 77.20 & 0.15 & 6496.00 & 66943.56 & 1.19 & 0.87 & 1.03 \\
65668 & 500713 & 2004 & 246.10 & 0.06 & 24480.00 & 238838.73 & 1.01 & 0.97 & 0.98 \\
43201 & 109083 & 2004 & 174.70 & 0.15 & 21210.00 & 178825.13 & 0.82 & 1.02 & 0.84 \\
96761 & 611011 & 2004 & 26.60 & 0.04 & 2662.00 & 25713.39 & 1.00 & 0.97 & 0.97 \\
40484 & 108122 & 2004 & 641.30 & 0.37 & 45909.00 & 521610.52 & 1.40 & 0.81 & 1.14 \\
43198 & 109080 & 2004 & 22.20 & 0.01 & 2756.00 & 20277.37 & 0.81 & 0.91 & 0.74 \\
58550 & 410165 & 2004 & 55.60 & 0.04 & 5554.00 & 55527.11 & 1.00 & 1.00 & 1.00 \\
26110 & 103539 & 2004 & 983.80 & 0.11 & 89663.00 & 884051.89 & 1.10 & 0.90 & 0.99 \\
47480 & 212351 & 2004 & 92.70 & 0.07 & 9324.00 & 91941.00 & 0.99 & 0.99 & 0.99 \\
49450 & 240301 & 2004 & 6.70 & 0.05 & 608.00 & 6117.17 & 1.10 & 0.91 & 1.01 \\
40510 & 108134 & 2004 & 161.10 & 0.18 & 16083.00 & 151515.67 & 1.00 & 0.94 & 0.94 \\
47979 & 225484 & 2004 & 42.40 & 0.06 & 4202.00 & 40500.22 & 1.01 & 0.96 & 0.96 \\
33672 & 106160 & 2004 & 204.80 & 0.12 & 20476.00 & 181343.67 & 1.00 & 0.89 & 0.89 \\
11909 & 101465 & 2004 & 137.20 & 0.09 & 13773.00 & 130955.90 & 1.00 & 0.95 & 0.95 \\
3685 & 100468 & 2004 & 3264.80 & 0.13 & 298913.00 & 3118976.96 & 1.09 & 0.96 & 1.04 \\
23935 & 103242 & 2004 & 1525.60 & 0.04 & 152822.00 & 1513822.90 & 1.00 & 0.99 & 0.99 \\
43767 & 109218 & 2004 & 21.20 & 0.17 & 1918.00 & 21095.02 & 1.11 & 1.00 & 1.10 \\
38727 & 107316 & 2004 & 973.20 & 0.19 & 125619.00 & 930877.03 & 0.77 & 0.96 & 0.74 \\
12575 & 101554 & 2004 & 151.20 & 0.18 & 14952.00 & 148216.10 & 1.01 & 0.98 & 0.99 \\
33648 & 106158 & 2004 & 727.10 & 0.06 & 60129.00 & 632574.46 & 1.21 & 0.87 & 1.05 \\
26033 & 103535 & 2004 & 3095.00 & 0.12 & 281846.00 & 3085531.98 & 1.10 & 1.00 & 1.09 \\
43229 & 109085 & 2004 & 10.50 & 0.08 & 1045.00 & 9921.25 & 1.00 & 0.94 & 0.95 \\
23965 & 103251 & 2004 & 344.00 & 0.04 & 33040.00 & 354905.13 & 1.04 & 1.03 & 1.07 \\
61402 & 500048 & 2004 & 61.80 & 0.08 & 5785.00 & 59896.31 & 1.07 & 0.97 & 1.04 \\
33628 & 106157 & 2004 & 999.40 & 0.11 & 100085.00 & 994654.69 & 1.00 & 1.00 & 0.99 \\
38827 & 107331 & 2004 & 24.00 & 0.21 & 1766.00 & 20790.31 & 1.36 & 0.87 & 1.18 \\
32006 & 105973 & 2004 & 80.90 & 0.13 & 8092.00 & 79937.79 & 1.00 & 0.99 & 0.99 \\
23986 & 103252 & 2004 & 385.60 & 0.19 & 35803.00 & 388239.91 & 1.08 & 1.01 & 1.08 \\
49436 & 240297 & 2004 & 887.20 & 0.14 & 91220.00 & 810839.85 & 0.97 & 0.91 & 0.89 \\
19580 & 102633 & 2004 & 132.90 & 0.20 & 13956.00 & 139562.68 & 0.95 & 1.05 & 1.00 \\
45539 & 200073 & 2004 & 383.00 & 0.00 & 37327.00 & 328773.84 & 1.03 & 0.86 & 0.88 \\
15401 & 101989 & 2004 & 192.60 & 0.12 & 19465.00 & 188564.15 & 0.99 & 0.98 & 0.97 \\
54158 & 364633 & 2004 & 13.60 & 0.19 & 1330.00 & 13002.86 & 1.02 & 0.96 & 0.98 \\
64399 & 500601 & 2004 & 131.60 & 0.07 & 13155.00 & 131473.17 & 1.00 & 1.00 & 1.00 \\
26063 & 103536 & 2004 & 2615.10 & 0.11 & 241808.00 & 2605287.50 & 1.08 & 1.00 & 1.08 \\
20306 & 102715 & 2004 & 1839.50 & 0.09 & 172853.00 & 1704122.11 & 1.06 & 0.93 & 0.99 \\
40458 & 108121 & 2004 & 3147.40 & 0.24 & 250364.00 & 2699881.25 & 1.26 & 0.86 & 1.08 \\
35 & 100003 & 2004 & 942.20 & 0.16 & 93773.00 & 906507.45 & 1.00 & 0.96 & 0.97 \\
9730 & 101179 & 2004 & 517.00 & 0.05 & 53937.00 & 482695.88 & 0.96 & 0.93 & 0.89 \\
38811 & 107328 & 2004 & 28.60 & 0.15 & 2687.00 & 28572.84 & 1.06 & 1.00 & 1.06 \\
9153 & 101115 & 2004 & 21084.40 & 0.11 & 2016000.00 & 20408488.05 & 1.05 & 0.97 & 1.01 \\
17553 & 102318 & 2004 & 2503.60 & 0.12 & 250633.00 & 2251047.13 & 1.00 & 0.90 & 0.90 \\
49445 & 240300 & 2004 & 11.30 & 0.00 & 1623.00 & 16903.29 & 0.70 & 1.50 & 1.04 \\
38786 & 107323 & 2004 & 100.40 & 0.15 & 9138.00 & 100705.39 & 1.10 & 1.00 & 1.10 \\
6335 & 100849 & 2004 & 49.00 & 0.03 & 4774.00 & 45786.40 & 1.03 & 0.93 & 0.96 \\
47992 & 225687 & 2004 & 292.50 & 0.05 & 29086.00 & 276037.44 & 1.01 & 0.94 & 0.95 \\
56252 & 400185 & 2004 & 53.60 & 0.21 & 4784.00 & 48918.95 & 1.12 & 0.91 & 1.02 \\
65372 & 500692 & 2004 & 133.50 & 0.10 & 11850.00 & 132198.63 & 1.13 & 0.99 & 1.12 \\
32972 & 106084 & 2004 & 1007.90 & -0.01 & 97043.00 & 978293.70 & 1.04 & 0.97 & 1.01 \\
55335 & 400085 & 2004 & 33.20 & 0.12 & 4034.00 & 27024.79 & 0.82 & 0.81 & 0.67 \\
64719 & 500621 & 2004 & 1294.90 & 0.30 & 129167.00 & 1289221.76 & 1.00 & 1.00 & 1.00 \\
18085 & 102396 & 2004 & 2989.80 & 0.14 & 298607.00 & 2457830.90 & 1.00 & 0.82 & 0.82 \\
43482 & 109127 & 2004 & 20.90 & 0.03 & 2108.00 & 20278.09 & 0.99 & 0.97 & 0.96 \\
25040 & 103426 & 2004 & 1296.90 & 0.19 & 114687.00 & 1255788.91 & 1.13 & 0.97 & 1.09 \\
6597 & 100900 & 2004 & 102.10 & 0.08 & 9627.00 & 79531.04 & 1.06 & 0.78 & 0.83 \\
39594 & 107781 & 2004 & 185.90 & 0.06 & 18373.00 & 193385.40 & 1.01 & 1.04 & 1.05 \\
1600 & 100217 & 2004 & 31.20 & -0.03 & 3124.00 & 29550.89 & 1.00 & 0.95 & 0.95 \\
60889 & 410737 & 2004 & 22.50 & 0.11 & 2179.00 & 19895.41 & 1.03 & 0.88 & 0.91 \\
39582 & 107726 & 2004 & 1009.70 & 0.14 & 100950.00 & 926128.35 & 1.00 & 0.92 & 0.92 \\
64696 & 500620 & 2004 & 559.60 & 0.23 & 55970.00 & 559213.37 & 1.00 & 1.00 & 1.00 \\
45214 & 109438 & 2004 & 199.40 & 0.23 & 17916.00 & 188168.08 & 1.11 & 0.94 & 1.05 \\
53935 & 361852 & 2004 & 20.40 & 0.14 & 2226.00 & 21987.35 & 0.92 & 1.08 & 0.99 \\
64943 & 500647 & 2004 & 97.10 & 0.16 & 8519.00 & 93191.16 & 1.14 & 0.96 & 1.09 \\
55751 & 400149 & 2004 & 56.60 & 0.45 & 6768.00 & 53682.53 & 0.84 & 0.95 & 0.79 \\
1571 & 100214 & 2004 & 164.40 & 0.19 & 16500.00 & 137112.52 & 1.00 & 0.83 & 0.83 \\
32666 & 106047 & 2004 & 10.40 & 0.05 & 960.00 & 10325.56 & 1.08 & 0.99 & 1.08 \\
52344 & 302780 & 2004 & 99.30 & 0.11 & 9536.00 & 90165.73 & 1.04 & 0.91 & 0.95 \\
45182 & 109436 & 2004 & 54.10 & 0.15 & 5408.00 & 52999.58 & 1.00 & 0.98 & 0.98 \\
64937 & 500646 & 2004 & 2.50 & 0.01 & 248.00 & 2387.16 & 1.01 & 0.95 & 0.96 \\
48378 & 240067 & 2004 & 289.10 & 0.11 & 28984.00 & 255232.28 & 1.00 & 0.88 & 0.88 \\
58217 & 410121 & 2004 & 51.90 & 0.13 & 5660.00 & 48845.19 & 0.92 & 0.94 & 0.86 \\
39850 & 107882 & 2004 & 242.20 & 0.12 & 38125.00 & 386063.37 & 0.64 & 1.59 & 1.01 \\
19844 & 102653 & 2004 & 8787.60 & 0.10 & 878762.00 & 7700996.63 & 1.00 & 0.88 & 0.88 \\
32705 & 106050 & 2004 & 708.40 & 0.24 & 71796.00 & 688061.24 & 0.99 & 0.97 & 0.96 \\
53944 & 362337 & 2004 & 18.30 & 0.20 & 1592.00 & 13934.30 & 1.15 & 0.76 & 0.88 \\
43642 & 109176 & 2004 & 19.80 & 0.11 & 1972.00 & 17919.80 & 1.00 & 0.91 & 0.91 \\
52353 & 302811 & 2004 & 44.90 & 0.06 & 4460.00 & 41430.95 & 1.01 & 0.92 & 0.93 \\
24997 & 103406 & 2004 & 2051.10 & 0.07 & 198884.00 & 1944083.32 & 1.03 & 0.95 & 0.98 \\
43527 & 109133 & 2004 & 48.40 & 0.03 & 4842.00 & 45183.91 & 1.00 & 0.93 & 0.93 \\
49216 & 240250 & 2004 & 657.90 & 0.26 & 61627.00 & 563179.21 & 1.07 & 0.86 & 0.91 \\
32945 & 106083 & 2004 & 783.20 & 0.08 & 75350.00 & 808378.85 & 1.04 & 1.03 & 1.07 \\
32693 & 106049 & 2004 & 328.90 & 0.10 & 33601.00 & 321908.73 & 0.98 & 0.98 & 0.96 \\
45175 & 109435 & 2004 & 6.10 & 0.18 & 464.00 & 4742.29 & 1.31 & 0.78 & 1.02 \\
43505 & 109129 & 2004 & 115.30 & 0.05 & 11455.00 & 107329.54 & 1.01 & 0.93 & 0.94 \\
39603 & 107786 & 2004 & 1499.10 & 0.11 & 140527.00 & 1445978.30 & 1.07 & 0.96 & 1.03 \\
55304 & 400081 & 2004 & 1907.60 & 0.15 & 184131.00 & 1939304.46 & 1.04 & 1.02 & 1.05 \\
15100 & 101956 & 2004 & 1475.30 & 0.03 & 147754.00 & 1451431.88 & 1.00 & 0.98 & 0.98 \\
24649 & 103373 & 2004 & 57.80 & 0.19 & 5653.00 & 54215.93 & 1.02 & 0.94 & 0.96 \\
15118 & 101958 & 2004 & 635.90 & 0.10 & 63616.00 & 611881.18 & 1.00 & 0.96 & 0.96 \\
12194 & 101518 & 2004 & 258.20 & 0.18 & 25762.00 & 249251.42 & 1.00 & 0.97 & 0.97 \\
60910 & 410749 & 2004 & 1.50 & -0.22 & 231.00 & 1562.42 & 0.65 & 1.04 & 0.68 \\
39527 & 107719 & 2004 & 379.50 & 0.06 & 36914.00 & 381730.55 & 1.03 & 1.01 & 1.03 \\
55855 & 400156 & 2004 & 228.80 & 0.12 & 15643.00 & 158669.20 & 1.46 & 0.69 & 1.01 \\
45258 & 109444 & 2004 & 15.10 & 0.18 & 1115.00 & 12113.37 & 1.35 & 0.80 & 1.09 \\
44985 & 109407 & 2004 & 333.10 & 0.16 & 31666.00 & 298086.22 & 1.05 & 0.89 & 0.94 \\
9465 & 101137 & 2004 & 20.90 & 0.13 & 2099.00 & 20686.28 & 1.00 & 0.99 & 0.99 \\
1540 & 100213 & 2004 & 239.20 & 0.23 & 23291.00 & 210282.12 & 1.03 & 0.88 & 0.90 \\
39924 & 107938 & 2004 & 103.10 & -0.00 & 10347.00 & 93536.54 & 1.00 & 0.91 & 0.90 \\
58161 & 410110 & 2004 & 18.40 & 0.08 & 2533.00 & 18143.44 & 0.73 & 0.99 & 0.72 \\
55341 & 400087 & 2004 & 65.40 & 0.15 & 6358.00 & 62710.59 & 1.03 & 0.96 & 0.99 \\
24577 & 103369 & 2004 & 786.70 & 0.05 & 79928.00 & 834617.95 & 0.98 & 1.06 & 1.04 \\
8102 & 101074 & 2004 & 614.60 & 0.40 & 63771.00 & 546541.30 & 0.96 & 0.89 & 0.86 \\
55876 & 400157 & 2004 & 210.50 & 0.13 & 14590.00 & 137136.89 & 1.44 & 0.65 & 0.94 \\
17908 & 102372 & 2004 & 7213.70 & 0.12 & 722598.00 & 6292396.14 & 1.00 & 0.87 & 0.87 \\
3922 & 100514 & 2004 & 53.20 & 0.06 & 4593.00 & 48176.18 & 1.16 & 0.91 & 1.05 \\
25151 & 103439 & 2004 & 92.10 & 0.23 & 7969.00 & 84170.11 & 1.16 & 0.91 & 1.06 \\
39521 & 107716 & 2004 & 52.20 & 0.04 & 5214.00 & 51254.19 & 1.00 & 0.98 & 0.98 \\
12363 & 101537 & 2004 & 483.00 & 0.11 & 48268.00 & 447702.08 & 1.00 & 0.93 & 0.93 \\
43699 & 109190 & 2004 & 39.20 & 0.05 & 3852.00 & 38527.54 & 1.02 & 0.98 & 1.00 \\
32593 & 106042 & 2004 & 148.80 & 0.14 & 15185.00 & 121352.69 & 0.98 & 0.82 & 0.80 \\
33050 & 106088 & 2004 & 188.80 & 0.14 & 18538.00 & 186929.46 & 1.02 & 0.99 & 1.01 \\
1828 & 100244 & 2004 & 256.80 & 0.15 & 24146.00 & 244421.18 & 1.06 & 0.95 & 1.01 \\
44995 & 109410 & 2004 & 213.50 & 0.16 & 17301.00 & 169512.92 & 1.23 & 0.79 & 0.98 \\
43447 & 109122 & 2004 & 3.40 & 0.10 & 341.00 & 3384.50 & 1.00 & 1.00 & 0.99 \\
32996 & 106085 & 2004 & 212.30 & 0.22 & 20319.00 & 203878.40 & 1.04 & 0.96 & 1.00 \\
58192 & 410115 & 2004 & 85.70 & 0.17 & 6173.00 & 68871.59 & 1.39 & 0.80 & 1.12 \\
19882 & 102654 & 2004 & 934.80 & 0.11 & 88216.00 & 899389.03 & 1.06 & 0.96 & 1.02 \\
25078 & 103429 & 2004 & 2286.40 & 0.16 & 202414.00 & 1846918.04 & 1.13 & 0.81 & 0.91 \\
32647 & 106045 & 2004 & 32.30 & 0.11 & 3328.00 & 32568.00 & 0.97 & 1.01 & 0.98 \\
39566 & 107722 & 2004 & 365.80 & 0.14 & NaN & 191542.30 & 1.00 & 0.52 & 1.00 \\
53994 & 363121 & 2004 & 1239.90 & 0.13 & 97286.00 & 1050656.37 & 1.27 & 0.85 & 1.08 \\
43676 & 109189 & 2004 & 771.00 & 0.14 & 85765.00 & 845911.71 & 0.90 & 1.10 & 0.99 \\
39552 & 107720 & 2004 & 190.70 & 0.18 & 19001.00 & 188609.39 & 1.00 & 0.99 & 0.99 \\
64949 & 500648 & 2004 & 335.90 & 0.13 & 27817.00 & 280386.50 & 1.21 & 0.83 & 1.01 \\
52398 & 302879 & 2004 & 252.90 & 0.07 & 25171.00 & 218060.19 & 1.00 & 0.86 & 0.87 \\
39898 & 107928 & 2004 & 3188.40 & 0.04 & 318507.00 & 3084836.65 & 1.00 & 0.97 & 0.97 \\
64959 & 500651 & 2004 & 9.80 & 0.18 & 830.00 & 9191.01 & 1.18 & 0.94 & 1.11 \\
43453 & 109124 & 2004 & 94.40 & 0.18 & 8488.00 & 93208.65 & 1.11 & 0.99 & 1.10 \\
24597 & 103370 & 2004 & 264.00 & 0.12 & 26503.00 & 258001.80 & 1.00 & 0.98 & 0.97 \\
33023 & 106086 & 2004 & 35.60 & 0.15 & 3445.00 & 34569.27 & 1.03 & 0.97 & 1.00 \\
45236 & 109439 & 2004 & 712.80 & 0.11 & 64612.00 & 639765.20 & 1.10 & 0.90 & 0.99 \\
55834 & 400155 & 2004 & 38.20 & 0.31 & 3874.00 & 31559.40 & 0.99 & 0.83 & 0.81 \\
25116 & 103432 & 2004 & 1843.50 & 0.05 & 175252.00 & 1834520.90 & 1.05 & 1.00 & 1.05 \\
4812 & 100682 & 2004 & 70.70 & 0.13 & 7024.00 & 66426.86 & 1.01 & 0.94 & 0.95 \\
6554 & 100890 & 2004 & 2365.10 & 0.10 & 234932.00 & 2119876.40 & 1.01 & 0.90 & 0.90 \\
19769 & 102651 & 2004 & 4788.70 & 0.11 & 478870.00 & 4643177.23 & 1.00 & 0.97 & 0.97 \\
43633 & 109175 & 2004 & 13.40 & 0.08 & 1339.00 & 13135.36 & 1.00 & 0.98 & 0.98 \\
49243 & 240254 & 2004 & 441.30 & 0.15 & 44125.00 & 413502.34 & 1.00 & 0.94 & 0.94 \\
1771 & 100228 & 2004 & 120.10 & 0.09 & 11317.00 & 110639.12 & 1.06 & 0.92 & 0.98 \\
24845 & 103381 & 2004 & 24050.40 & 0.10 & 2340491.00 & 23227102.95 & 1.03 & 0.97 & 0.99 \\
52365 & 302813 & 2004 & 21.90 & 0.07 & 2201.00 & 19458.49 & 1.00 & 0.89 & 0.88 \\
18012 & 102386 & 2004 & 2017.20 & 0.21 & 200096.00 & 1934540.68 & 1.01 & 0.96 & 0.97 \\
12277 & 101531 & 2004 & 416.50 & 0.17 & 35259.00 & 311939.09 & 1.18 & 0.75 & 0.88 \\
64767 & 500628 & 2004 & 55.60 & 0.12 & 4791.00 & 47622.29 & 1.16 & 0.86 & 0.99 \\
39712 & 107837 & 2004 & 32.30 & 0.28 & 3191.00 & 29184.49 & 1.01 & 0.90 & 0.91 \\
9354 & 101132 & 2004 & 448.10 & 0.36 & 42541.00 & 395473.64 & 1.05 & 0.88 & 0.93 \\
55312 & 400084 & 2004 & 27.20 & 0.07 & 2723.00 & 25583.98 & 1.00 & 0.94 & 0.94 \\
7118 & 100997 & 2004 & 793.50 & 0.16 & 79199.00 & 733103.75 & 1.00 & 0.92 & 0.93 \\
18047 & 102387 & 2004 & 721.80 & 0.27 & 72154.00 & 688471.45 & 1.00 & 0.95 & 0.95 \\
32782 & 106062 & 2004 & 109.50 & 0.17 & 9132.00 & 81555.04 & 1.20 & 0.74 & 0.89 \\
64822 & 500635 & 2004 & 82.10 & 0.05 & 8187.00 & 81876.17 & 1.00 & 1.00 & 1.00 \\
39802 & 107874 & 2004 & 215.20 & 0.17 & 21922.00 & 217481.15 & 0.98 & 1.01 & 0.99 \\
1707 & 100226 & 2004 & 14966.10 & 0.10 & 1430523.00 & 12176906.18 & 1.05 & 0.81 & 0.85 \\
15066 & 101955 & 2004 & 12724.30 & 0.09 & 1230752.00 & 12650815.35 & 1.03 & 0.99 & 1.03 \\
13677 & 101757 & 2004 & 504.30 & 0.11 & 50562.00 & 490985.22 & 1.00 & 0.97 & 0.97 \\
64799 & 500634 & 2004 & 99.50 & 0.18 & 9925.00 & 99253.59 & 1.00 & 1.00 & 1.00 \\
45113 & 109429 & 2004 & 47.80 & 0.04 & 4831.00 & 42564.83 & 0.99 & 0.89 & 0.88 \\
58226 & 410130 & 2004 & 226.30 & 0.05 & 22961.00 & 217952.60 & 0.99 & 0.96 & 0.95 \\
1751 & 100227 & 2004 & 119.20 & 0.08 & 11396.00 & 112309.61 & 1.05 & 0.94 & 0.99 \\
43603 & 109153 & 2004 & 482.10 & 0.09 & 49821.00 & 473089.33 & 0.97 & 0.98 & 0.95 \\
45071 & 109421 & 2004 & 578.50 & 0.17 & 34903.00 & 340437.39 & 1.66 & 0.59 & 0.98 \\
32840 & 106067 & 2004 & 7295.80 & 0.11 & 698880.00 & 6906780.89 & 1.04 & 0.95 & 0.99 \\
12312 & 101534 & 2004 & 1829.50 & 0.15 & 183167.00 & 1795125.72 & 1.00 & 0.98 & 0.98 \\
58250 & 410133 & 2004 & 57.80 & 0.11 & 5772.00 & 56474.38 & 1.00 & 0.98 & 0.98 \\
24805 & 103380 & 2004 & 4109.90 & 0.07 & 390026.00 & 3680743.63 & 1.05 & 0.90 & 0.94 \\
53962 & 362424 & 2004 & 248.10 & 0.20 & 24851.00 & 233203.01 & 1.00 & 0.94 & 0.94 \\
52378 & 302825 & 2004 & 92.80 & 0.03 & 9573.00 & 94489.77 & 0.97 & 1.02 & 0.99 \\
39776 & 107870 & 2004 & 287.40 & 0.06 & 27765.00 & 294417.49 & 1.04 & 1.02 & 1.06 \\
45088 & 109427 & 2004 & 153.70 & 0.13 & 13013.00 & 118772.60 & 1.18 & 0.77 & 0.91 \\
60806 & 410730 & 2004 & 418.70 & 0.11 & 46151.00 & 457634.97 & 0.91 & 1.09 & 0.99 \\
15049 & 101953 & 2004 & 379.20 & 0.09 & 37908.00 & 344417.46 & 1.00 & 0.91 & 0.91 \\
32812 & 106066 & 2004 & 1825.20 & 0.07 & 206595.00 & 1882841.41 & 0.88 & 1.03 & 0.91 \\
39733 & 107858 & 2004 & 24.60 & 0.22 & 2779.00 & 26102.48 & 0.89 & 1.06 & 0.94 \\
49260 & 240256 & 2004 & 64.00 & 0.17 & 6399.00 & 61792.48 & 1.00 & 0.97 & 0.97 \\
24766 & 103377 & 2004 & 1314.60 & 0.07 & 123275.00 & 1326724.20 & 1.07 & 1.01 & 1.08 \\
64776 & 500633 & 2004 & 56.80 & 0.15 & 5673.00 & 56725.27 & 1.00 & 1.00 & 1.00 \\
47440 & 211051 & 2004 & 281.10 & 0.01 & 29989.00 & 272536.47 & 0.94 & 0.97 & 0.91 \\
52372 & 302819 & 2004 & 2.60 & 0.06 & 263.00 & 2327.44 & 0.99 & 0.90 & 0.88 \\
13658 & 101754 & 2004 & 110.00 & 0.19 & 11013.00 & 106477.74 & 1.00 & 0.97 & 0.97 \\
39808 & 107875 & 2004 & 150.50 & 0.07 & 19064.00 & 155576.42 & 0.79 & 1.03 & 0.82 \\
60826 & 410731 & 2004 & 1306.50 & 0.18 & 142602.00 & 1352642.29 & 0.92 & 1.04 & 0.95 \\
1688 & 100223 & 2004 & 2695.30 & 0.09 & 269566.00 & 2524958.00 & 1.00 & 0.94 & 0.94 \\
32743 & 106057 & 2004 & 344.70 & 0.08 & 28815.00 & 297102.32 & 1.20 & 0.86 & 1.03 \\
43534 & 109142 & 2004 & 102.60 & 0.10 & 10295.00 & 106383.88 & 1.00 & 1.04 & 1.03 \\
8072 & 101073 & 2004 & 8143.50 & 0.08 & 747728.00 & 7425308.70 & 1.09 & 0.91 & 0.99 \\
39828 & 107881 & 2004 & 15.00 & 0.15 & 1647.00 & 13081.38 & 0.91 & 0.87 & 0.79 \\
24945 & 103395 & 2004 & 121.50 & 0.05 & 12145.00 & 116924.57 & 1.00 & 0.96 & 0.96 \\
9327 & 101131 & 2004 & 2803.90 & 0.12 & 290398.00 & 2917208.22 & 0.97 & 1.04 & 1.00 \\
39640 & 107832 & 2004 & 226.50 & 0.19 & 22627.00 & 221512.00 & 1.00 & 0.98 & 0.98 \\
60866 & 410733 & 2004 & 432.70 & 0.15 & 54131.00 & 456767.30 & 0.80 & 1.06 & 0.84 \\
17972 & 102377 & 2004 & 175.70 & 0.06 & 13823.00 & 154992.23 & 1.27 & 0.88 & 1.12 \\
24689 & 103375 & 2004 & 948.00 & 0.07 & 105489.00 & 924468.85 & 0.90 & 0.98 & 0.88 \\
32918 & 106082 & 2004 & 495.20 & 0.11 & 47750.00 & 486362.97 & 1.04 & 0.98 & 1.02 \\
3976 & 100535 & 2004 & 329.90 & 0.13 & 31658.00 & 302062.60 & 1.04 & 0.92 & 0.95 \\
12330 & 101536 & 2004 & 2437.90 & 0.12 & 243116.00 & 2378693.13 & 1.00 & 0.98 & 0.98 \\
32724 & 106052 & 2004 & 133.30 & 0.13 & 13180.00 & 131074.17 & 1.01 & 0.98 & 0.99 \\
60802 & 410727 & 2004 & 18.00 & 0.13 & 1736.00 & 16915.89 & 1.04 & 0.94 & 0.97 \\
12234 & 101528 & 2004 & 83.00 & 0.13 & 9068.00 & 67970.71 & 0.92 & 0.82 & 0.75 \\
64868 & 500638 & 2004 & 119.90 & 0.10 & 11967.00 & 119674.63 & 1.00 & 1.00 & 1.00 \\
55813 & 400153 & 2004 & 111.30 & 0.13 & 11018.00 & 94166.12 & 1.01 & 0.85 & 0.85 \\
52389 & 302826 & 2004 & 398.20 & 0.06 & 51053.00 & 375588.53 & 0.78 & 0.94 & 0.74 \\
45060 & 109416 & 2004 & 77.30 & 0.16 & 6334.00 & 69051.27 & 1.22 & 0.89 & 1.09 \\
48351 & 240065 & 2004 & 390.40 & 0.05 & 38517.00 & 384946.46 & 1.01 & 0.99 & 1.00 \\
12249 & 101530 & 2004 & 1876.30 & 0.09 & 176851.00 & 1709523.66 & 1.06 & 0.91 & 0.97 \\
24729 & 103376 & 2004 & 5151.10 & 0.10 & 497608.00 & 5078707.96 & 1.04 & 0.99 & 1.02 \\
24886 & 103383 & 2004 & 1728.20 & 0.15 & 156290.00 & 1636214.20 & 1.11 & 0.95 & 1.05 \\
39684 & 107835 & 2004 & 842.90 & 0.22 & 84396.00 & 793533.30 & 1.00 & 0.94 & 0.94 \\
60846 & 410732 & 2004 & 1404.20 & 0.12 & 147039.00 & 1422295.61 & 0.95 & 1.01 & 0.97 \\
45049 & 109415 & 2004 & 44.50 & 0.14 & 3895.00 & 41462.10 & 1.14 & 0.93 & 1.06 \\
53991 & 363013 & 2004 & 10.70 & 0.13 & 1012.00 & 10108.58 & 1.06 & 0.94 & 1.00 \\
19738 & 102650 & 2004 & 18375.80 & 0.15 & 1837580.00 & 16421034.72 & 1.00 & 0.89 & 0.89 \\
64845 & 500636 & 2004 & 122.90 & 0.13 & 12233.00 & 122333.71 & 1.00 & 1.00 & 1.00 \\
19803 & 102652 & 2004 & 1881.00 & 0.11 & 188095.00 & 1776558.13 & 1.00 & 0.94 & 0.94 \\
32876 & 106075 & 2004 & 208.10 & 0.06 & 20806.00 & 196442.64 & 1.00 & 0.94 & 0.94 \\
4844 & 100685 & 2004 & 21.10 & 0.15 & 1846.00 & 19127.06 & 1.14 & 0.91 & 1.04 \\
1790 & 100237 & 2004 & 29.60 & 0.15 & 3142.00 & 32489.07 & 0.94 & 1.10 & 1.03 \\
24907 & 103394 & 2004 & 195.90 & 0.25 & 19880.00 & 176080.36 & 0.99 & 0.90 & 0.89 \\
32758 & 106061 & 2004 & 603.20 & 0.20 & 60793.00 & 570597.21 & 0.99 & 0.95 & 0.94 \\
49286 & 240264 & 2004 & 289.20 & 0.21 & 28443.00 & 255380.54 & 1.02 & 0.88 & 0.90 \\
43562 & 109144 & 2004 & 271.90 & 0.06 & 25579.00 & 248334.17 & 1.06 & 0.91 & 0.97 \\
39667 & 107833 & 2004 & 197.90 & 0.04 & 19595.00 & 194732.08 & 1.01 & 0.98 & 0.99 \\
47415 & 210770 & 2004 & 3957.10 & 0.18 & 438609.00 & 4043739.28 & 0.90 & 1.02 & 0.92 \\
45129 & 109431 & 2004 & 13.30 & 0.11 & 1241.00 & 11178.22 & 1.07 & 0.84 & 0.90 \\
43557 & 109143 & 2004 & 239.00 & 0.13 & 24156.00 & 225200.85 & 0.99 & 0.94 & 0.93 \\
64743 & 500625 & 2004 & 141.20 & 0.05 & 14396.00 & 150195.67 & 0.98 & 1.06 & 1.04 \\
64669 & 500618 & 2004 & 351.90 & 0.18 & 35157.00 & 335030.19 & 1.00 & 0.95 & 0.95 \\
54952 & 400030 & 2004 & 48.70 & 0.08 & 4427.00 & 41999.81 & 1.10 & 0.86 & 0.95 \\
44966 & 109406 & 2004 & 147.40 & 0.06 & 12913.00 & 122591.41 & 1.14 & 0.83 & 0.95 \\
1939 & 100259 & 2004 & 115.20 & 0.15 & 11680.00 & 113516.39 & 0.99 & 0.99 & 0.97 \\
25414 & 103483 & 2004 & 556.90 & 0.10 & 53194.00 & 549763.69 & 1.05 & 0.99 & 1.03 \\
33251 & 106108 & 2004 & 98.50 & 0.17 & 8727.00 & 88560.24 & 1.13 & 0.90 & 1.01 \\
15221 & 101968 & 2004 & 99.50 & 0.05 & 9953.00 & 92010.55 & 1.00 & 0.92 & 0.92 \\
58406 & 410154 & 2004 & 16.80 & 0.15 & 1575.00 & 16576.98 & 1.07 & 0.99 & 1.05 \\
20019 & 102663 & 2004 & 4507.10 & 0.02 & 413008.00 & 4191632.75 & 1.09 & 0.93 & 1.01 \\
4766 & 100671 & 2004 & 716.50 & 0.15 & 61160.00 & 608801.80 & 1.17 & 0.85 & 1.00 \\
4913 & 100692 & 2004 & 388.30 & 0.13 & 39551.00 & 374444.97 & 0.98 & 0.96 & 0.95 \\
49382 & 240287 & 2004 & 19.30 & 0.15 & 1197.00 & 16231.75 & 1.61 & 0.84 & 1.36 \\
43379 & 109108 & 2004 & 55.00 & 0.15 & 4789.00 & 51662.36 & 1.15 & 0.94 & 1.08 \\
40025 & 108009 & 2004 & 670.60 & 0.10 & 68081.00 & 633338.68 & 0.99 & 0.94 & 0.93 \\
9549 & 101149 & 2004 & 1593.70 & 0.03 & 150843.00 & 1526844.31 & 1.06 & 0.96 & 1.01 \\
25438 & 103487 & 2004 & 210.20 & 0.34 & 18842.00 & 180138.79 & 1.12 & 0.86 & 0.96 \\
6460 & 100875 & 2004 & 299.30 & 0.25 & 31211.00 & 301835.60 & 0.96 & 1.01 & 0.97 \\
43374 & 109104 & 2004 & 32.90 & 0.01 & 3299.00 & 32112.87 & 1.00 & 0.98 & 0.97 \\
32408 & 106023 & 2004 & 210.00 & 0.15 & 21045.00 & 207155.04 & 1.00 & 0.99 & 0.98 \\
39315 & 107653 & 2004 & 23.40 & 0.10 & 2284.00 & 22844.77 & 1.02 & 0.98 & 1.00 \\
65137 & 500664 & 2004 & 1994.30 & 0.12 & 199598.00 & 1960566.48 & 1.00 & 0.98 & 0.98 \\
12086 & 101497 & 2004 & 1197.90 & 0.05 & 119919.00 & 1121759.15 & 1.00 & 0.94 & 0.94 \\
49136 & 240225 & 2004 & 453.20 & 0.04 & 40085.00 & 338705.53 & 1.13 & 0.75 & 0.84 \\
24375 & 103318 & 2004 & 1735.00 & 0.07 & 173220.00 & 1732186.95 & 1.00 & 1.00 & 1.00 \\
24409 & 103319 & 2004 & 263.20 & 0.09 & 26378.00 & 260190.80 & 1.00 & 0.99 & 0.99 \\
39350 & 107672 & 2004 & 1.60 & 0.15 & 158.00 & 1269.28 & 1.01 & 0.79 & 0.80 \\
44932 & 109402 & 2004 & 170.40 & 0.11 & 16955.00 & 152885.93 & 1.01 & 0.90 & 0.90 \\
53897 & 360021 & 2004 & 90.30 & 0.12 & 9408.00 & 88766.77 & 0.96 & 0.98 & 0.94 \\
56017 & 400166 & 2004 & 305.80 & 0.04 & 30646.00 & 298580.70 & 1.00 & 0.98 & 0.97 \\
39338 & 107670 & 2004 & 489.40 & 0.17 & 44373.00 & 400621.33 & 1.10 & 0.82 & 0.90 \\
12414 & 101539 & 2004 & 1656.00 & 0.15 & 166319.00 & 1626399.30 & 1.00 & 0.98 & 0.98 \\
58379 & 410151 & 2004 & 7.80 & 0.10 & 787.00 & 7775.18 & 0.99 & 1.00 & 0.99 \\
24425 & 103326 & 2004 & 1374.40 & 0.10 & 132947.00 & 1285856.18 & 1.03 & 0.94 & 0.97 \\
53880 & 360020 & 2004 & 66.60 & 0.05 & 6669.00 & 61140.61 & 1.00 & 0.92 & 0.92 \\
32449 & 106028 & 2004 & 1156.10 & 0.07 & 115338.00 & 1006052.59 & 1.00 & 0.87 & 0.87 \\
25384 & 103481 & 2004 & 530.90 & 0.23 & 45175.00 & 487820.80 & 1.18 & 0.92 & 1.08 \\
56037 & 400167 & 2004 & 148.40 & 0.05 & 14835.00 & 147900.75 & 1.00 & 1.00 & 1.00 \\
9222 & 101119 & 2004 & 435.40 & 0.19 & 43632.00 & 398704.26 & 1.00 & 0.92 & 0.91 \\
17816 & 102364 & 2004 & 1425.10 & 0.10 & 142438.00 & 1308825.45 & 1.00 & 0.92 & 0.92 \\
58401 & 410153 & 2004 & 13.20 & 0.12 & 1250.00 & 13476.32 & 1.06 & 1.02 & 1.08 \\
1353 & 100190 & 2004 & 1653.90 & 0.03 & 170783.00 & 1569393.31 & 0.97 & 0.95 & 0.92 \\
40012 & 107994 & 2004 & 558.60 & -0.07 & 59017.00 & 616377.94 & 0.95 & 1.10 & 1.04 \\
33241 & 106107 & 2004 & 51.00 & 0.10 & 4591.00 & 48933.08 & 1.11 & 0.96 & 1.07 \\
12106 & 101503 & 2004 & 146.30 & 0.05 & 14659.00 & 143766.79 & 1.00 & 0.98 & 0.98 \\
19674 & 102645 & 2004 & 345.30 & 0.10 & 32521.00 & 330092.21 & 1.06 & 0.96 & 1.02 \\
52325 & 302763 & 2004 & 280.50 & 0.18 & 28575.00 & 252146.58 & 0.98 & 0.90 & 0.88 \\
64530 & 500607 & 2004 & 139.30 & 0.06 & 14949.00 & 143899.33 & 0.93 & 1.03 & 0.96 \\
96783 & 611013 & 2004 & 31.80 & 0.10 & 3176.00 & 31398.34 & 1.00 & 0.99 & 0.99 \\
52292 & 302732 & 2004 & 149.70 & 0.13 & 14227.00 & 149389.54 & 1.05 & 1.00 & 1.05 \\
40076 & 108021 & 2004 & 1329.40 & 0.16 & 132939.00 & 1240415.93 & 1.00 & 0.93 & 0.93 \\
56058 & 400170 & 2004 & 403.30 & 0.05 & 36973.00 & 365148.26 & 1.09 & 0.91 & 0.99 \\
43337 & 109099 & 2004 & 153.80 & 0.05 & 14372.00 & 147244.77 & 1.07 & 0.96 & 1.02 \\
39287 & 107648 & 2004 & 326.80 & 0.10 & 32767.00 & 318645.45 & 1.00 & 0.98 & 0.97 \\
40100 & 108029 & 2004 & 393.50 & 0.12 & 36406.00 & 357891.66 & 1.08 & 0.91 & 0.98 \\
45369 & 200050 & 2004 & 136.00 & 0.30 & 11655.00 & 126386.40 & 1.17 & 0.93 & 1.08 \\
43329 & 109095 & 2004 & 31.30 & 0.17 & 4059.00 & 34597.37 & 0.77 & 1.11 & 0.85 \\
52450 & 302942 & 2004 & 1565.40 & 0.16 & 132860.00 & 1464035.74 & 1.18 & 0.94 & 1.10 \\
19661 & 102641 & 2004 & 426.30 & 0.09 & 39604.00 & 388253.37 & 1.08 & 0.91 & 0.98 \\
24321 & 103308 & 2004 & 4260.00 & 0.09 & 410871.00 & 3986711.00 & 1.04 & 0.94 & 0.97 \\
33314 & 106114 & 2004 & 216.40 & 0.19 & 25840.00 & 214383.11 & 0.84 & 0.99 & 0.83 \\
17756 & 102356 & 2004 & 5.40 & 0.16 & 540.00 & 5252.50 & 1.00 & 0.97 & 0.97 \\
12448 & 101541 & 2004 & 518.00 & 0.12 & 50927.00 & 509289.49 & 1.02 & 0.98 & 1.00 \\
65300 & 500684 & 2004 & 1297.80 & 0.29 & 129259.00 & 1292586.88 & 1.00 & 1.00 & 1.00 \\
61212 & 410907 & 2004 & 185.10 & 0.13 & 16346.00 & 157483.09 & 1.13 & 0.85 & 0.96 \\
15256 & 101972 & 2004 & 3428.20 & 0.15 & 370022.00 & 3400912.95 & 0.93 & 0.99 & 0.92 \\
65349 & 500689 & 2004 & 42.80 & 0.13 & 3888.00 & 35758.99 & 1.10 & 0.84 & 0.92 \\
44915 & 109401 & 2004 & 41.40 & 0.20 & 4043.00 & 36972.00 & 1.02 & 0.89 & 0.91 \\
49110 & 240222 & 2004 & 458.90 & 0.07 & 45833.00 & 452549.77 & 1.00 & 0.99 & 0.99 \\
39303 & 107650 & 2004 & 75.90 & -0.01 & 7472.00 & 72577.24 & 1.02 & 0.96 & 0.97 \\
25501 & 103494 & 2004 & 215.00 & 0.12 & 19588.00 & 216020.93 & 1.10 & 1.00 & 1.10 \\
61188 & 410904 & 2004 & 293.80 & 0.17 & 25310.00 & 278370.90 & 1.16 & 0.95 & 1.10 \\
17787 & 102357 & 2004 & 1126.60 & 0.11 & 113189.00 & 1174530.06 & 1.00 & 1.04 & 1.04 \\
64550 & 500609 & 2004 & 302.70 & 0.13 & 30263.00 & 302026.93 & 1.00 & 1.00 & 1.00 \\
40040 & 108013 & 2004 & 28.70 & 0.06 & 2785.00 & 27850.23 & 1.03 & 0.97 & 1.00 \\
52299 & 302760 & 2004 & 780.50 & 0.22 & 73436.00 & 799966.39 & 1.06 & 1.02 & 1.09 \\
13621 & 101748 & 2004 & 2834.40 & 0.06 & 282851.00 & 2803058.86 & 1.00 & 0.99 & 0.99 \\
39309 & 107652 & 2004 & 85.00 & 0.07 & 8179.00 & 81134.93 & 1.04 & 0.95 & 0.99 \\
48096 & 235952 & 2004 & 42.30 & 0.13 & 4219.00 & 35131.38 & 1.00 & 0.83 & 0.83 \\
48033 & 226946 & 2004 & 250.10 & 0.11 & 25024.00 & 212576.12 & 1.00 & 0.85 & 0.85 \\
1276 & 100171 & 2004 & 1024.20 & 0.19 & 83676.00 & 780496.20 & 1.22 & 0.76 & 0.93 \\
40048 & 108018 & 2004 & 411.20 & 0.13 & 41121.00 & 375090.37 & 1.00 & 0.91 & 0.91 \\
20053 & 102664 & 2004 & 5101.50 & 0.26 & 436250.00 & 4379873.41 & 1.17 & 0.86 & 1.00 \\
33287 & 106113 & 2004 & 397.90 & 0.10 & 39699.00 & 393796.32 & 1.00 & 0.99 & 0.99 \\
18184 & 102414 & 2004 & 9519.60 & 0.06 & 920687.00 & 8563854.31 & 1.03 & 0.90 & 0.93 \\
1258 & 100167 & 2004 & 127.00 & 0.03 & 15784.00 & 161287.02 & 0.80 & 1.27 & 1.02 \\
24347 & 103315 & 2004 & 156.40 & 0.13 & 14513.00 & 136066.42 & 1.08 & 0.87 & 0.94 \\
43360 & 109100 & 2004 & 12.20 & 0.12 & 1036.00 & 9248.16 & 1.18 & 0.76 & 0.89 \\
49390 & 240288 & 2004 & 9.90 & 0.15 & 598.00 & 8120.59 & 1.66 & 0.82 & 1.36 \\
15240 & 101970 & 2004 & 119.60 & 0.14 & 11960.00 & 110779.26 & 1.00 & 0.93 & 0.93 \\
45343 & 200047 & 2004 & 64.90 & 0.22 & 5312.00 & 50411.21 & 1.22 & 0.78 & 0.95 \\
12069 & 101494 & 2004 & 174.80 & -0.01 & 17495.00 & 159562.00 & 1.00 & 0.91 & 0.91 \\
44905 & 109399 & 2004 & 50.40 & 0.09 & 4904.00 & 41910.52 & 1.03 & 0.83 & 0.85 \\
25357 & 103478 & 2004 & 4432.30 & 0.20 & 363224.00 & 3575033.54 & 1.22 & 0.81 & 0.98 \\
55358 & 400090 & 2004 & 26.10 & 0.16 & 2575.00 & 24822.15 & 1.01 & 0.95 & 0.96 \\
54937 & 400029 & 2004 & 5.50 & 0.13 & 304.00 & 2507.98 & 1.81 & 0.46 & 0.82 \\
7087 & 100996 & 2004 & 1473.90 & 0.10 & 144533.00 & 1434004.74 & 1.02 & 0.97 & 0.99 \\
33127 & 106092 & 2004 & 549.10 & 0.11 & 53881.00 & 504517.80 & 1.02 & 0.92 & 0.94 \\
64635 & 500614 & 2004 & 118.60 & 0.12 & 11864.00 & 118550.73 & 1.00 & 1.00 & 1.00 \\
65025 & 500656 & 2004 & 1492.90 & 0.04 & 149389.00 & 1424504.72 & 1.00 & 0.95 & 0.95 \\
32549 & 106039 & 2004 & 1710.80 & 0.06 & 144552.00 & 1490599.53 & 1.18 & 0.87 & 1.03 \\
1453 & 100200 & 2004 & 550.20 & 0.36 & 55022.00 & 525131.27 & 1.00 & 0.95 & 0.95 \\
55918 & 400159 & 2004 & 47.50 & 0.26 & 3272.00 & 32643.48 & 1.45 & 0.69 & 1.00 \\
17875 & 102367 & 2004 & 143.00 & 0.11 & 14373.00 & 129744.90 & 0.99 & 0.91 & 0.90 \\
48389 & 240074 & 2004 & 179.40 & 0.24 & 18780.00 & 152951.45 & 0.96 & 0.85 & 0.81 \\
64596 & 500612 & 2004 & 400.00 & 0.12 & 39983.00 & 399579.80 & 1.00 & 1.00 & 1.00 \\
39935 & 107958 & 2004 & 38.40 & 0.13 & 3450.00 & 32056.77 & 1.11 & 0.83 & 0.93 \\
14136 & 101805 & 2004 & 706.70 & 0.10 & 68693.00 & 656054.16 & 1.03 & 0.93 & 0.96 \\
6490 & 100876 & 2004 & 1.30 & -0.15 & 120.00 & 1234.84 & 1.08 & 0.95 & 1.03 \\
4019 & 100538 & 2004 & 513.90 & 0.19 & 46855.00 & 513953.18 & 1.10 & 1.00 & 1.10 \\
39459 & 107694 & 2004 & 11.90 & 0.12 & 1172.00 & 11507.62 & 1.02 & 0.97 & 0.98 \\
33150 & 106097 & 2004 & 536.00 & 0.06 & 50784.00 & 501055.63 & 1.06 & 0.93 & 0.99 \\
48044 & 227155 & 2004 & 218.50 & 0.10 & 19041.00 & 205390.64 & 1.15 & 0.94 & 1.08 \\
53920 & 360123 & 2004 & 12.80 & 0.02 & 1340.00 & 13356.80 & 0.96 & 1.04 & 1.00 \\
19954 & 102659 & 2004 & 5740.90 & 0.18 & 508923.00 & 5660946.37 & 1.13 & 0.99 & 1.11 \\
12163 & 101513 & 2004 & 423.10 & 0.15 & 39219.00 & 404424.15 & 1.08 & 0.96 & 1.03 \\
24507 & 103329 & 2004 & 489.80 & 0.24 & 47580.00 & 458632.75 & 1.03 & 0.94 & 0.96 \\
55350 & 400088 & 2004 & 82.70 & 0.24 & 8471.00 & 74459.49 & 0.98 & 0.90 & 0.88 \\
7352 & 101023 & 2004 & 26371.70 & 0.11 & 2431658.00 & 23200866.66 & 1.08 & 0.88 & 0.95 \\
58140 & 410100 & 2004 & 147.70 & 0.14 & 14687.00 & 118507.34 & 1.01 & 0.80 & 0.81 \\
19704 & 102649 & 2004 & 753.80 & 0.12 & 75089.00 & 702506.58 & 1.00 & 0.93 & 0.94 \\
55273 & 400076 & 2004 & 896.00 & 0.18 & 89511.00 & 891027.19 & 1.00 & 0.99 & 1.00 \\
64986 & 500653 & 2004 & 70.70 & -0.04 & 7054.00 & 59443.87 & 1.00 & 0.84 & 0.84 \\
15150 & 101963 & 2004 & 731.70 & 0.10 & 73574.00 & 637542.95 & 0.99 & 0.87 & 0.87 \\
1522 & 100209 & 2004 & 6224.60 & 0.07 & 627654.00 & 5780946.64 & 0.99 & 0.93 & 0.92 \\
33086 & 106090 & 2004 & 673.20 & 0.24 & 57485.00 & 507722.34 & 1.17 & 0.75 & 0.88 \\
64658 & 500617 & 2004 & 7894.70 & 0.16 & 527999.00 & 6762795.50 & 1.50 & 0.86 & 1.28 \\
49321 & 240269 & 2004 & 244.20 & 0.30 & 24196.00 & 235603.95 & 1.01 & 0.96 & 0.97 \\
8033 & 101071 & 2004 & 9093.30 & 0.09 & 843971.00 & 7804151.92 & 1.08 & 0.86 & 0.92 \\
9284 & 101127 & 2004 & 123.20 & 0.18 & 10880.00 & 115868.43 & 1.13 & 0.94 & 1.06 \\
1484 & 100207 & 2004 & 1514.80 & 0.06 & 151832.00 & 1456059.57 & 1.00 & 0.96 & 0.96 \\
33100 & 106091 & 2004 & 371.20 & 0.22 & 33659.00 & 303057.51 & 1.10 & 0.82 & 0.90 \\
1891 & 100247 & 2004 & 774.10 & 0.16 & 74895.00 & 641503.25 & 1.03 & 0.83 & 0.86 \\
45280 & 200011 & 2004 & 174.60 & 0.14 & 15595.00 & 170752.23 & 1.12 & 0.98 & 1.09 \\
17888 & 102371 & 2004 & 262.40 & 0.08 & 26258.00 & 214213.06 & 1.00 & 0.82 & 0.82 \\
24539 & 103339 & 2004 & 831.60 & 0.14 & 83299.00 & 774365.42 & 1.00 & 0.93 & 0.93 \\
25203 & 103460 & 2004 & 1075.80 & 0.13 & 97171.00 & 1028002.96 & 1.11 & 0.96 & 1.06 \\
32565 & 106041 & 2004 & 287.70 & 0.06 & 27411.00 & 268926.91 & 1.05 & 0.93 & 0.98 \\
6509 & 100878 & 2004 & 2445.30 & 0.08 & 220672.00 & 2308755.28 & 1.11 & 0.94 & 1.05 \\
39480 & 107702 & 2004 & 1232.90 & 0.12 & 111763.00 & 1207752.74 & 1.10 & 0.98 & 1.08 \\
43387 & 109110 & 2004 & 41.00 & 0.21 & 4095.00 & 39352.36 & 1.00 & 0.96 & 0.96 \\
48073 & 235413 & 2004 & 194.50 & 0.17 & 19468.00 & 180103.86 & 1.00 & 0.93 & 0.93 \\
43443 & 109121 & 2004 & 2.30 & 0.02 & 231.00 & 2346.77 & 1.00 & 1.02 & 1.02 \\
39984 & 107967 & 2004 & 153.60 & 0.07 & 14919.00 & 147788.67 & 1.03 & 0.96 & 0.99 \\
14 & 100001 & 2004 & 3875.30 & 0.08 & 387380.00 & 3655898.55 & 1.00 & 0.94 & 0.94 \\
25316 & 103466 & 2004 & 1256.10 & 0.11 & 119302.00 & 1194199.82 & 1.05 & 0.95 & 1.00 \\
52418 & 302907 & 2004 & 1217.70 & 0.11 & 118227.00 & 1182269.75 & 1.03 & 0.97 & 1.00 \\
39378 & 107677 & 2004 & 504.90 & 0.19 & 37787.00 & 338522.24 & 1.34 & 0.67 & 0.90 \\
43418 & 109112 & 2004 & 53.50 & 0.04 & 5399.00 & 49560.49 & 0.99 & 0.93 & 0.92 \\
49151 & 240234 & 2004 & 211.40 & 0.22 & 20168.00 & 201680.96 & 1.05 & 0.95 & 1.00 \\
4893 & 100691 & 2004 & 474.00 & 0.09 & 46752.00 & 480980.45 & 1.01 & 1.01 & 1.03 \\
33193 & 106102 & 2004 & 398.30 & 0.17 & 38992.00 & 384388.87 & 1.02 & 0.97 & 0.99 \\
39371 & 107673 & 2004 & 198.00 & 0.15 & 19785.00 & 176829.50 & 1.00 & 0.89 & 0.89 \\
47340 & 210203 & 2004 & 7054.60 & 0.09 & 688391.00 & 6363836.58 & 1.02 & 0.90 & 0.92 \\
44955 & 109404 & 2004 & 20.50 & 0.11 & 1959.00 & 18754.18 & 1.05 & 0.91 & 0.96 \\
24455 & 103327 & 2004 & 1448.70 & 0.12 & 140313.00 & 1307359.40 & 1.03 & 0.90 & 0.93 \\
32477 & 106033 & 2004 & 1830.80 & 0.22 & 145996.00 & 1574746.48 & 1.25 & 0.86 & 1.08 \\
12129 & 101511 & 2004 & 730.70 & 0.11 & 74134.00 & 712648.43 & 0.99 & 0.98 & 0.96 \\
43395 & 109111 & 2004 & 121.20 & 1.70 & 12221.00 & 117372.77 & 0.99 & 0.97 & 0.96 \\
1372 & 100192 & 2004 & 63.60 & 0.05 & 6358.00 & 60267.80 & 1.00 & 0.95 & 0.95 \\
55952 & 400161 & 2004 & 175.90 & 0.10 & 17643.00 & 170719.25 & 1.00 & 0.97 & 0.97 \\
8014 & 101069 & 2004 & 12112.90 & 0.11 & 1080563.00 & 11339660.45 & 1.12 & 0.94 & 1.05 \\
39994 & 107968 & 2004 & 108.30 & 0.13 & 10543.00 & 94573.35 & 1.03 & 0.87 & 0.90 \\
9513 & 101141 & 2004 & 2021.70 & 0.03 & 202462.00 & 2068201.79 & 1.00 & 1.02 & 1.02 \\
18134 & 102404 & 2004 & 2673.20 & 0.10 & 268967.00 & 2656753.08 & 0.99 & 0.99 & 0.99 \\
49374 & 240286 & 2004 & 21.70 & 0.12 & 1361.00 & 17485.59 & 1.59 & 0.81 & 1.28 \\
19984 & 102660 & 2004 & 8263.30 & 0.09 & 832098.00 & 8036688.71 & 0.99 & 0.97 & 0.97 \\
49342 & 240284 & 2004 & 84.00 & 0.18 & 7329.00 & 84038.37 & 1.15 & 1.00 & 1.15 \\
39960 & 107960 & 2004 & 291.30 & 0.10 & 24726.00 & 242414.58 & 1.18 & 0.83 & 0.98 \\
32521 & 106038 & 2004 & 5070.80 & 0.20 & 505406.00 & 4442590.35 & 1.00 & 0.88 & 0.88 \\
45299 & 200015 & 2004 & 36.70 & 0.17 & 3459.00 & 36746.76 & 1.06 & 1.00 & 1.06 \\
15179 & 101964 & 2004 & 2055.60 & 0.18 & 174016.00 & 1869126.47 & 1.18 & 0.91 & 1.07 \\
39976 & 107964 & 2004 & 189.50 & 0.10 & 18141.00 & 182368.65 & 1.04 & 0.96 & 1.01 \\
43429 & 109118 & 2004 & 290.40 & 0.04 & 29049.00 & 262675.61 & 1.00 & 0.90 & 0.90 \\
25285 & 103464 & 2004 & 911.90 & 0.13 & 82223.00 & 899631.06 & 1.11 & 0.99 & 1.09 \\
12394 & 101538 & 2004 & 329.90 & 0.17 & 33219.00 & 303173.70 & 0.99 & 0.92 & 0.91 \\
32512 & 106037 & 2004 & 18.70 & 0.22 & 1869.00 & 18631.55 & 1.00 & 1.00 & 1.00 \\
39421 & 107692 & 2004 & 8.40 & 0.18 & 753.00 & 8168.81 & 1.12 & 0.97 & 1.08 \\
60936 & 410755 & 2004 & 3.50 & -0.01 & 278.00 & 2393.81 & 1.26 & 0.68 & 0.86 \\
44960 & 109405 & 2004 & 58.60 & 0.10 & 5625.00 & 55126.08 & 1.04 & 0.94 & 0.98 \\
24477 & 103328 & 2004 & 351.50 & 0.11 & 33875.00 & 335809.62 & 1.04 & 0.96 & 0.99 \\
48409 & 240076 & 2004 & 28.30 & 0.02 & 2971.00 & 30549.64 & 0.95 & 1.08 & 1.03 \\
17847 & 102365 & 2004 & 289.70 & 0.06 & 28882.00 & 286319.04 & 1.00 & 0.99 & 0.99 \\
64573 & 500610 & 2004 & 468.20 & -0.00 & 46856.00 & 468319.59 & 1.00 & 1.00 & 1.00 \\
38577 & 107290 & 2004 & 1078.50 & 0.10 & 107728.00 & 1046580.26 & 1.00 & 0.97 & 0.97 \\
65619 & 500708 & 2004 & 1055.00 & 0.23 & 102053.00 & 1050234.78 & 1.03 & 1.00 & 1.03 \\
35616 & 106380 & 2004 & 568.80 & 0.22 & 56775.00 & 563540.52 & 1.00 & 0.99 & 0.99 \\
17344 & 102282 & 2004 & 332.00 & 0.13 & 39475.00 & 270590.79 & 0.84 & 0.82 & 0.69 \\
20504 & 102760 & 2004 & 1243.60 & 0.07 & 124882.00 & 1211544.60 & 1.00 & 0.97 & 0.97 \\
52063 & 301438 & 2004 & 700.40 & 0.15 & 62114.00 & 687134.15 & 1.13 & 0.98 & 1.11 \\
14358 & 101851 & 2004 & 3841.00 & 0.14 & 341987.00 & 3754438.91 & 1.12 & 0.98 & 1.10 \\
713 & 100092 & 2004 & 686.50 & 0.12 & 63174.00 & 628802.00 & 1.09 & 0.92 & 1.00 \\
31533 & 105905 & 2004 & 16.70 & 0.14 & 1673.00 & 16567.23 & 1.00 & 0.99 & 0.99 \\
31321 & 105878 & 2004 & 2072.80 & 0.11 & 190651.00 & 1990433.49 & 1.09 & 0.96 & 1.04 \\
14307 & 101843 & 2004 & 138.10 & 0.12 & 15835.00 & 130916.91 & 0.87 & 0.95 & 0.83 \\
14026 & 101800 & 2004 & 471.40 & 0.04 & 46713.00 & 456847.26 & 1.01 & 0.97 & 0.98 \\
38371 & 107246 & 2004 & 48.10 & 0.06 & 5015.00 & 49149.11 & 0.96 & 1.02 & 0.98 \\
19439 & 102601 & 2004 & 6651.70 & 0.11 & 612692.00 & 6244677.41 & 1.09 & 0.94 & 1.02 \\
8322 & 101082 & 2004 & 1969.90 & 0.09 & 189833.00 & 1965811.40 & 1.04 & 1.00 & 1.04 \\
54322 & 367168 & 2004 & 7.90 & 0.25 & 788.00 & 7619.51 & 1.00 & 0.96 & 0.97 \\
23694 & 103208 & 2004 & 1325.00 & 0.09 & 129011.00 & 1305354.03 & 1.03 & 0.99 & 1.01 \\
20714 & 102784 & 2004 & 13588.00 & 0.04 & 1381140.00 & 12731222.69 & 0.98 & 0.94 & 0.92 \\
56421 & 400207 & 2004 & 14.10 & 0.10 & 1408.00 & 13947.38 & 1.00 & 0.99 & 0.99 \\
40795 & 108153 & 2004 & 68.90 & 0.19 & 6892.00 & 56075.59 & 1.00 & 0.81 & 0.81 \\
7318 & 101020 & 2004 & 6452.30 & 0.11 & 599116.00 & 5279938.48 & 1.08 & 0.82 & 0.88 \\
26525 & 103590 & 2004 & 1287.30 & 0.13 & 119642.00 & 1295488.81 & 1.08 & 1.01 & 1.08 \\
37978 & 107178 & 2004 & 50.40 & 0.05 & 3985.00 & 41872.85 & 1.26 & 0.83 & 1.05 \\
9864 & 101198 & 2004 & 240.00 & 0.11 & 20967.00 & 235596.59 & 1.14 & 0.98 & 1.12 \\
7150 & 100998 & 2004 & 112.20 & 0.17 & 11225.00 & 108403.71 & 1.00 & 0.97 & 0.97 \\
47027 & 200329 & 2004 & 625.40 & 0.23 & 49114.00 & 479445.85 & 1.27 & 0.77 & 0.98 \\
10009 & 101252 & 2004 & 189.70 & 0.10 & 18421.00 & 182865.00 & 1.03 & 0.96 & 0.99 \\
49688 & 240332 & 2004 & 91.80 & 0.10 & 9178.00 & 75443.78 & 1.00 & 0.82 & 0.82 \\
54289 & 366837 & 2004 & 12.80 & 0.06 & 1181.00 & 10954.25 & 1.08 & 0.86 & 0.93 \\
2264 & 100303 & 2004 & 399.90 & 0.10 & 39760.00 & 370044.80 & 1.01 & 0.93 & 0.93 \\
38346 & 107244 & 2004 & 264.10 & 0.04 & 26471.00 & 260157.78 & 1.00 & 0.99 & 0.98 \\
33886 & 106180 & 2004 & 43.50 & 0.16 & 3996.00 & 39696.26 & 1.09 & 0.91 & 0.99 \\
34168 & 106210 & 2004 & 1012.40 & 0.31 & 92816.00 & 747848.49 & 1.09 & 0.74 & 0.81 \\
49661 & 240330 & 2004 & 266.60 & 0.23 & 25076.00 & 216936.60 & 1.06 & 0.81 & 0.87 \\
31649 & 105920 & 2004 & 3258.90 & 0.09 & 312531.00 & 3007855.29 & 1.04 & 0.92 & 0.96 \\
38321 & 107243 & 2004 & 1261.60 & 0.10 & 123123.00 & 1266494.30 & 1.02 & 1.00 & 1.03 \\
38134 & 107204 & 2004 & 585.10 & 0.16 & 31873.00 & 304711.42 & 1.84 & 0.52 & 0.96 \\
18498 & 102465 & 2004 & 262.70 & 0.02 & 25560.00 & 272173.34 & 1.03 & 1.04 & 1.06 \\
33912 & 106182 & 2004 & 1098.20 & 0.17 & 100470.00 & 1082799.73 & 1.09 & 0.99 & 1.08 \\
40906 & 108163 & 2004 & 207.70 & 0.17 & 13002.00 & 121489.50 & 1.60 & 0.58 & 0.93 \\
40825 & 108155 & 2004 & 78.00 & 0.04 & 7645.00 & 77073.67 & 1.02 & 0.99 & 1.01 \\
26557 & 103591 & 2004 & 1662.00 & 0.22 & 152400.00 & 1638782.76 & 1.09 & 0.99 & 1.08 \\
3626 & 100463 & 2004 & 13853.60 & 0.18 & 1387608.00 & 13678394.20 & 1.00 & 0.99 & 0.99 \\
23387 & 103174 & 2004 & 1822.40 & 0.06 & 178915.00 & 1830626.74 & 1.02 & 1.00 & 1.02 \\
23576 & 103193 & 2004 & 110.20 & 0.24 & 10606.00 & 106055.43 & 1.04 & 0.96 & 1.00 \\
27017 & 103644 & 2004 & 43.50 & 0.16 & 3870.00 & 43812.56 & 1.12 & 1.01 & 1.13 \\
3529 & 100453 & 2004 & 123.00 & 0.07 & 12251.00 & 112192.73 & 1.00 & 0.91 & 0.92 \\
43850 & 109225 & 2004 & 31.80 & 0.16 & 2218.00 & 18106.41 & 1.43 & 0.57 & 0.82 \\
34237 & 106214 & 2004 & 276.70 & 0.22 & 28068.00 & 273397.45 & 0.99 & 0.99 & 0.97 \\
18403 & 102447 & 2004 & 6820.30 & 0.19 & 587781.00 & 6047283.12 & 1.16 & 0.89 & 1.03 \\
19496 & 102607 & 2004 & 590.80 & 0.07 & 57463.00 & 594952.10 & 1.03 & 1.01 & 1.04 \\
56417 & 400206 & 2004 & 6.20 & 0.05 & 716.00 & 6897.78 & 0.87 & 1.11 & 0.96 \\
5138 & 100726 & 2004 & 4015.20 & 0.08 & 400047.00 & 4071449.93 & 1.00 & 1.01 & 1.02 \\
37905 & 107160 & 2004 & 8543.00 & 0.20 & 714324.00 & 7963619.31 & 1.20 & 0.93 & 1.11 \\
54300 & 367166 & 2004 & 207.50 & 0.20 & 21016.00 & 180137.32 & 0.99 & 0.87 & 0.86 \\
64238 & 500593 & 2004 & 1726.50 & 0.17 & 172565.00 & 1724625.27 & 1.00 & 1.00 & 1.00 \\
54866 & 400019 & 2004 & 1787.90 & 0.18 & 158952.00 & 1738769.95 & 1.12 & 0.97 & 1.09 \\
19513 & 102608 & 2004 & 163.70 & 0.35 & 13590.00 & 151964.09 & 1.20 & 0.93 & 1.12 \\
37898 & 107159 & 2004 & 14.50 & 0.10 & 1512.00 & 13497.80 & 0.96 & 0.93 & 0.89 \\
48540 & 240105 & 2004 & 221.70 & 0.09 & 21378.00 & 195093.36 & 1.04 & 0.88 & 0.91 \\
777 & 100096 & 2004 & 100.10 & 0.04 & 9205.00 & 94998.78 & 1.09 & 0.95 & 1.03 \\
49558 & 240312 & 2004 & 235.50 & 0.09 & 23752.00 & 219941.55 & 0.99 & 0.93 & 0.93 \\
26488 & 103582 & 2004 & 17.50 & 0.07 & 1744.00 & 15683.40 & 1.00 & 0.90 & 0.90 \\
6691 & 100910 & 2004 & 169.10 & 0.18 & 14612.00 & 135893.24 & 1.16 & 0.80 & 0.93 \\
47924 & 222809 & 2004 & 185.10 & 0.12 & 10523.00 & 88748.21 & 1.76 & 0.48 & 0.84 \\
46955 & 200322 & 2004 & 34.90 & 0.11 & 3368.00 & 30504.40 & 1.04 & 0.87 & 0.91 \\
49694 & 240333 & 2004 & 91.50 & 0.34 & 10573.00 & 97337.02 & 0.87 & 1.06 & 0.92 \\
55230 & 400074 & 2004 & 1648.40 & 0.07 & 164833.00 & 1619546.07 & 1.00 & 0.98 & 0.98 \\
4600 & 100642 & 2004 & 1186.70 & 0.08 & 112015.00 & 1076050.95 & 1.06 & 0.91 & 0.96 \\
38093 & 107201 & 2004 & 424.30 & 0.15 & 35036.00 & 367390.71 & 1.21 & 0.87 & 1.05 \\
43082 & 109061 & 2004 & 144.80 & 0.09 & 15755.00 & 147647.10 & 0.92 & 1.02 & 0.94 \\
43087 & 109062 & 2004 & 35.20 & 0.06 & 3530.00 & 34104.64 & 1.00 & 0.97 & 0.97 \\
27073 & 103647 & 2004 & 31.20 & 0.18 & 3992.00 & 32645.99 & 0.78 & 1.05 & 0.82 \\
31702 & 105931 & 2004 & 6941.10 & 0.15 & 685094.00 & 6226384.88 & 1.01 & 0.90 & 0.91 \\
49535 & 240311 & 2004 & 16.30 & 0.10 & 1796.00 & 17875.57 & 0.91 & 1.10 & 1.00 \\
40591 & 108141 & 2004 & 244.80 & 0.09 & 24442.00 & 243394.78 & 1.00 & 0.99 & 1.00 \\
23728 & 103209 & 2004 & 190.70 & 0.16 & 16291.00 & 176376.24 & 1.17 & 0.92 & 1.08 \\
15683 & 102013 & 2004 & 2244.80 & 0.12 & 256382.00 & 2152946.71 & 0.88 & 0.96 & 0.84 \\
31421 & 105882 & 2004 & 205.70 & 0.15 & 16446.00 & 177059.59 & 1.25 & 0.86 & 1.08 \\
34056 & 106199 & 2004 & 29.10 & 0.29 & 2916.00 & 29082.27 & 1.00 & 1.00 & 1.00 \\
807 & 100097 & 2004 & 134.70 & 0.08 & 12545.00 & 126623.22 & 1.07 & 0.94 & 1.01 \\
20475 & 102757 & 2004 & 7136.00 & 0.10 & 720672.00 & 7025394.19 & 0.99 & 0.98 & 0.97 \\
2103 & 100291 & 2004 & 914.50 & 0.15 & 91221.00 & 835630.60 & 1.00 & 0.91 & 0.92 \\
38393 & 107253 & 2004 & 412.50 & 0.09 & 30157.00 & 285219.52 & 1.37 & 0.69 & 0.95 \\
43800 & 109222 & 2004 & 36.80 & -0.35 & 2775.00 & 22695.35 & 1.33 & 0.62 & 0.82 \\
34141 & 106209 & 2004 & 661.60 & 0.16 & 66719.00 & 532862.31 & 0.99 & 0.81 & 0.80 \\
26896 & 103620 & 2004 & 360.20 & 0.11 & 35824.00 & 359350.11 & 1.01 & 1.00 & 1.00 \\
59412 & 410479 & 2004 & 191.10 & 0.19 & 15583.00 & 174188.83 & 1.23 & 0.91 & 1.12 \\
54267 & 365483 & 2004 & 577.20 & 0.20 & 52417.00 & 554302.33 & 1.10 & 0.96 & 1.06 \\
73367 & 600006 & 2004 & 214.10 & 0.15 & 18478.00 & 186668.03 & 1.16 & 0.87 & 1.01 \\
37884 & 107156 & 2004 & 111.70 & 0.04 & 10865.00 & 113219.27 & 1.03 & 1.01 & 1.04 \\
40770 & 108149 & 2004 & 127.60 & 0.32 & 12516.00 & 103606.88 & 1.02 & 0.81 & 0.83 \\
15592 & 102007 & 2004 & 3111.50 & 0.15 & 289588.00 & 2899749.20 & 1.07 & 0.93 & 1.00 \\
31683 & 105930 & 2004 & 324.60 & 0.03 & 31967.00 & 287489.22 & 1.02 & 0.89 & 0.90 \\
11816 & 101462 & 2004 & 893.50 & 0.18 & 85604.00 & 840554.23 & 1.04 & 0.94 & 0.98 \\
4566 & 100639 & 2004 & 548.70 & 0.04 & 55826.00 & 476809.34 & 0.98 & 0.87 & 0.85 \\
5274 & 100745 & 2004 & 915.90 & 0.20 & 86081.00 & 876822.56 & 1.06 & 0.96 & 1.02 \\
2277 & 100305 & 2004 & 84.30 & 0.10 & 8257.00 & 82564.98 & 1.02 & 0.98 & 1.00 \\
41012 & 108176 & 2004 & 18.50 & -0.01 & 1854.00 & 16679.39 & 1.00 & 0.90 & 0.90 \\
20659 & 102777 & 2004 & 1507.80 & 0.08 & 152573.00 & 1456124.57 & 0.99 & 0.97 & 0.95 \\
33877 & 106179 & 2004 & 63.60 & 0.08 & 6360.00 & 60136.77 & 1.00 & 0.95 & 0.95 \\
38118 & 107202 & 2004 & 24.90 & 0.19 & 2451.00 & 22593.38 & 1.02 & 0.91 & 0.92 \\
37867 & 107152 & 2004 & 90.60 & 0.11 & 8440.00 & 78249.00 & 1.07 & 0.86 & 0.93 \\
49652 & 240328 & 2004 & 9.40 & 0.22 & 955.00 & 9142.46 & 0.98 & 0.97 & 0.96 \\
61560 & 500094 & 2004 & 199.70 & 0.31 & 20177.00 & 196989.01 & 0.99 & 0.99 & 0.98 \\
15533 & 102000 & 2004 & 362.10 & 0.16 & 44620.00 & 333025.77 & 0.81 & 0.92 & 0.75 \\
7169 & 101000 & 2004 & 1415.10 & 0.10 & 140417.00 & 1354779.63 & 1.01 & 0.96 & 0.96 \\
64100 & 500587 & 2004 & 1754.40 & 0.12 & 175263.00 & 1750494.90 & 1.00 & 1.00 & 1.00 \\
40745 & 108148 & 2004 & 65.60 & 0.06 & 5697.00 & 55026.70 & 1.15 & 0.84 & 0.97 \\
6705 & 100913 & 2004 & 643.20 & 0.19 & 58966.00 & 619326.12 & 1.09 & 0.96 & 1.05 \\
64261 & 500594 & 2004 & 2907.80 & 0.22 & 290297.00 & 2895718.55 & 1.00 & 1.00 & 1.00 \\
42945 & 109038 & 2004 & 27.70 & 0.13 & 2766.00 & 25936.03 & 1.00 & 0.94 & 0.94 \\
5252 & 100741 & 2004 & 171.40 & 0.26 & 17660.00 & 154515.12 & 0.97 & 0.90 & 0.87 \\
45648 & 200087 & 2004 & 13.20 & 0.15 & 1123.00 & 11438.25 & 1.18 & 0.87 & 1.02 \\
49569 & 240316 & 2004 & 41.60 & 0.31 & 4076.00 & 38221.82 & 1.02 & 0.92 & 0.94 \\
38235 & 107224 & 2004 & 86.00 & 0.18 & 8756.00 & 82784.45 & 0.98 & 0.96 & 0.95 \\
31367 & 105880 & 2004 & 1344.00 & 0.18 & 134804.00 & 1317133.16 & 1.00 & 0.98 & 0.98 \\
96694 & 611006 & 2004 & 18.00 & 0.11 & 2887.00 & 25815.73 & 0.62 & 1.43 & 0.89 \\
9892 & 101200 & 2004 & 24.10 & 0.06 & 2209.00 & 22410.79 & 1.09 & 0.93 & 1.01 \\
17256 & 102274 & 2004 & 8156.70 & 0.19 & 849580.00 & 7358234.10 & 0.96 & 0.90 & 0.87 \\
9909 & 101211 & 2004 & 306.40 & 0.14 & 29906.00 & 304125.52 & 1.02 & 0.99 & 1.02 \\
43870 & 109226 & 2004 & 119.30 & 0.21 & 11699.00 & 108477.09 & 1.02 & 0.91 & 0.93 \\
40955 & 108166 & 2004 & 109.20 & 0.04 & 10578.00 & 100062.74 & 1.03 & 0.92 & 0.95 \\
42968 & 109044 & 2004 & 71.10 & 0.09 & 7133.00 & 69496.98 & 1.00 & 0.98 & 0.97 \\
45781 & 200133 & 2004 & 17.60 & 0.11 & 1761.00 & 17026.86 & 1.00 & 0.97 & 0.97 \\
14315 & 101849 & 2004 & 14.00 & 0.03 & 1405.00 & 12218.06 & 1.00 & 0.87 & 0.87 \\
54228 & 364947 & 2004 & 44.40 & 0.11 & 3634.00 & 37513.25 & 1.22 & 0.84 & 1.03 \\
38245 & 107226 & 2004 & 675.30 & 0.21 & 71223.00 & 632385.46 & 0.95 & 0.94 & 0.89 \\
26931 & 103628 & 2004 & 961.30 & 0.14 & 94588.00 & 938900.48 & 1.02 & 0.98 & 0.99 \\
43056 & 109058 & 2004 & 49.90 & 0.13 & 5002.00 & 48836.90 & 1.00 & 0.98 & 0.98 \\
38189 & 107215 & 2004 & 552.80 & 0.34 & 47804.00 & 504128.72 & 1.16 & 0.91 & 1.05 \\
53652 & 355799 & 2004 & 3.60 & 0.01 & 575.00 & 4981.32 & 0.63 & 1.38 & 0.87 \\
46883 & 200311 & 2004 & 478.50 & 0.12 & 47835.00 & 442035.10 & 1.00 & 0.92 & 0.92 \\
6235 & 100831 & 2004 & 134.60 & 0.12 & 13466.00 & 130614.68 & 1.00 & 0.97 & 0.97 \\
11785 & 101461 & 2004 & 2041.70 & 0.09 & 204623.00 & 1774646.07 & 1.00 & 0.87 & 0.87 \\
53646 & 355536 & 2004 & 8.60 & 0.01 & 912.00 & 8895.66 & 0.94 & 1.03 & 0.98 \\
23630 & 103204 & 2004 & 127.20 & 0.14 & 11851.00 & 119600.58 & 1.07 & 0.94 & 1.01 \\
31565 & 105909 & 2004 & 60.40 & 0.12 & 5986.00 & 57897.02 & 1.01 & 0.96 & 0.97 \\
31615 & 105917 & 2004 & 225.80 & 0.11 & 22612.00 & 218513.16 & 1.00 & 0.97 & 0.97 \\
26622 & 103593 & 2004 & 75870.30 & 0.09 & 7481055.00 & 71782412.18 & 1.01 & 0.95 & 0.96 \\
40960 & 108168 & 2004 & 526.80 & 0.08 & 52712.00 & 480436.42 & 1.00 & 0.91 & 0.91 \\
70 & 100004 & 2004 & 1239.70 & 0.07 & 121044.00 & 1228746.17 & 1.02 & 0.99 & 1.02 \\
48988 & 240197 & 2004 & 76.10 & 0.08 & 7769.00 & 76300.01 & 0.98 & 1.00 & 0.98 \\
54295 & 367116 & 2004 & 2.50 & 0.14 & 218.00 & 2076.80 & 1.15 & 0.83 & 0.95 \\
45679 & 200089 & 2004 & 186.90 & 0.19 & 14137.00 & 161494.88 & 1.32 & 0.86 & 1.14 \\
54292 & 367042 & 2004 & 28.90 & 0.16 & 2598.00 & 25818.11 & 1.11 & 0.89 & 0.99 \\
45670 & 200088 & 2004 & 38.70 & 0.15 & 4022.00 & 36683.18 & 0.96 & 0.95 & 0.91 \\
46898 & 200312 & 2004 & 443.30 & 0.08 & 44346.00 & 433980.78 & 1.00 & 0.98 & 0.98 \\
57937 & 410003 & 2004 & 416.90 & 0.14 & 41772.00 & 413386.17 & 1.00 & 0.99 & 0.99 \\
61637 & 500109 & 2004 & 6987.90 & 0.18 & 698789.00 & 6946043.35 & 1.00 & 0.99 & 0.99 \\
3594 & 100460 & 2004 & 187.70 & 0.13 & 20451.00 & 186283.89 & 0.92 & 0.99 & 0.91 \\
15577 & 102005 & 2004 & 489.60 & 0.11 & 46728.00 & 459674.13 & 1.05 & 0.94 & 0.98 \\
14879 & 101919 & 2004 & 1064.60 & 0.06 & 126195.00 & 1077762.52 & 0.84 & 1.01 & 0.85 \\
42959 & 109042 & 2004 & 102.00 & 0.14 & 6791.00 & 69616.23 & 1.50 & 0.68 & 1.03 \\
37943 & 107173 & 2004 & 105.10 & 0.35 & 10430.00 & 101143.81 & 1.01 & 0.96 & 0.97 \\
26654 & 103595 & 2004 & 175.10 & 0.14 & 17467.00 & 172467.80 & 1.00 & 0.98 & 0.99 \\
40878 & 108160 & 2004 & 21.00 & 0.11 & 1917.00 & 19531.13 & 1.10 & 0.93 & 1.02 \\
44685 & 109359 & 2004 & 142.00 & 0.08 & 23659.00 & 209806.63 & 0.60 & 1.48 & 0.89 \\
47892 & 222658 & 2004 & 402.40 & 0.19 & 40248.00 & 392523.78 & 1.00 & 0.98 & 0.98 \\
23606 & 103202 & 2004 & 61.90 & 0.14 & 5765.00 & 56779.98 & 1.07 & 0.92 & 0.98 \\
33959 & 106193 & 2004 & 23.50 & 0.02 & 2444.00 & 23499.85 & 0.96 & 1.00 & 0.96 \\
13487 & 101741 & 2004 & 5616.30 & 0.12 & 562140.00 & 4784163.93 & 1.00 & 0.85 & 0.85 \\
61617 & 500107 & 2004 & 98.20 & 0.07 & 10000.00 & 88586.81 & 0.98 & 0.90 & 0.89 \\
38214 & 107222 & 2004 & 1249.20 & 0.18 & 100621.00 & 1011980.60 & 1.24 & 0.81 & 1.01 \\
34195 & 106211 & 2004 & 79.00 & 0.10 & 7852.00 & 76799.83 & 1.01 & 0.97 & 0.98 \\
59416 & 410481 & 2004 & 11.80 & 0.05 & 753.00 & 7874.19 & 1.57 & 0.67 & 1.05 \\
34207 & 106212 & 2004 & 92.90 & 0.18 & 9221.00 & 91739.44 & 1.01 & 0.99 & 0.99 \\
23451 & 103177 & 2004 & 230.90 & 0.16 & 23373.00 & 224939.45 & 0.99 & 0.97 & 0.96 \\
37966 & 107175 & 2004 & 1843.10 & 0.01 & 183956.00 & 1816669.90 & 1.00 & 0.99 & 0.99 \\
747 & 100093 & 2004 & 173.00 & 0.15 & 15947.00 & 167422.82 & 1.08 & 0.97 & 1.05 \\
18476 & 102462 & 2004 & 21.20 & 0.36 & 2189.00 & 21641.53 & 0.97 & 1.02 & 0.99 \\
20563 & 102767 & 2004 & 4382.90 & 0.12 & 439185.00 & 4256730.80 & 1.00 & 0.97 & 0.97 \\
56348 & 400197 & 2004 & 11.30 & 0.06 & 1078.00 & 11306.20 & 1.05 & 1.00 & 1.05 \\
33932 & 106192 & 2004 & 1864.60 & 0.05 & 185418.00 & 1678129.44 & 1.01 & 0.90 & 0.91 \\
38284 & 107235 & 2004 & 69.50 & 0.07 & 7143.00 & 66465.70 & 0.97 & 0.96 & 0.93 \\
20526 & 102761 & 2004 & 18273.80 & 0.11 & 1836909.00 & 18041563.59 & 0.99 & 0.99 & 0.98 \\
40993 & 108172 & 2004 & 262.50 & 0.08 & 24694.00 & 204866.24 & 1.06 & 0.78 & 0.83 \\
26588 & 103592 & 2004 & 420.40 & -0.01 & 42387.00 & 431188.35 & 0.99 & 1.03 & 1.02 \\
2237 & 100298 & 2004 & 527.20 & 0.15 & 55241.00 & 546197.22 & 0.95 & 1.04 & 0.99 \\
12644 & 101561 & 2004 & 129.70 & 0.04 & 11299.00 & 124780.99 & 1.15 & 0.96 & 1.10 \\
45794 & 200140 & 2004 & 1395.10 & 0.10 & 139691.00 & 1228435.07 & 1.00 & 0.88 & 0.88 \\
31394 & 105881 & 2004 & 2448.30 & 0.25 & 237531.00 & 2215475.73 & 1.03 & 0.90 & 0.93 \\
52572 & 303130 & 2004 & 81.40 & 0.18 & 7247.00 & 81501.51 & 1.12 & 1.00 & 1.12 \\
44736 & 109370 & 2004 & 1216.00 & 0.20 & 121032.00 & 1120297.97 & 1.00 & 0.92 & 0.93 \\
48566 & 240107 & 2004 & 425.50 & 0.38 & 42279.00 & 342029.26 & 1.01 & 0.80 & 0.81 \\
34002 & 106197 & 2004 & 246.50 & 0.12 & 24555.00 & 248015.01 & 1.00 & 1.01 & 1.01 \\
46874 & 200310 & 2004 & 118.50 & 0.11 & 12131.00 & 112536.34 & 0.98 & 0.95 & 0.93 \\
18433 & 102452 & 2004 & 136.50 & 0.17 & 12827.00 & 141813.58 & 1.06 & 1.04 & 1.11 \\
52040 & 301299 & 2004 & 4889.10 & 0.08 & 487469.00 & 4693205.39 & 1.00 & 0.96 & 0.96 \\
43035 & 109056 & 2004 & 653.00 & 0.15 & 65402.00 & 550200.94 & 1.00 & 0.84 & 0.84 \\
38160 & 107209 & 2004 & 306.00 & 0.06 & 29010.00 & 292456.68 & 1.05 & 0.96 & 1.01 \\
41005 & 108175 & 2004 & 119.10 & 0.12 & 11952.00 & 119071.89 & 1.00 & 1.00 & 1.00 \\
17388 & 102284 & 2004 & 305.70 & 0.22 & 29967.00 & 300752.06 & 1.02 & 0.98 & 1.00 \\
44702 & 109366 & 2004 & 21.40 & 0.16 & 2034.00 & 20340.94 & 1.05 & 0.95 & 1.00 \\
34029 & 106198 & 2004 & 800.50 & 0.25 & 100454.00 & 704874.63 & 0.80 & 0.88 & 0.70 \\
23661 & 103205 & 2004 & 115.50 & 0.26 & 10973.00 & 107374.03 & 1.05 & 0.93 & 0.98 \\
45704 & 200091 & 2004 & 5.10 & 0.10 & 501.00 & 5083.19 & 1.02 & 1.00 & 1.01 \\
11685 & 101456 & 2004 & 43.70 & 0.17 & 4367.00 & 43589.21 & 1.00 & 1.00 & 1.00 \\
56412 & 400205 & 2004 & 12.60 & 0.15 & 1240.00 & 12277.74 & 1.02 & 0.97 & 0.99 \\
38296 & 107242 & 2004 & 1286.00 & 0.06 & 129594.00 & 1244460.25 & 0.99 & 0.97 & 0.96 \\
15643 & 102010 & 2004 & 4393.20 & 0.12 & 498527.00 & 4136152.84 & 0.88 & 0.94 & 0.83 \\
837 & 100098 & 2004 & 972.20 & 0.12 & 85779.00 & 849648.50 & 1.13 & 0.87 & 0.99 \\
7837 & 101062 & 2004 & 4556.60 & 0.34 & 389807.00 & 3878711.21 & 1.17 & 0.85 & 1.00 \\
40895 & 108161 & 2004 & 258.00 & 0.12 & 24711.00 & 258794.53 & 1.04 & 1.00 & 1.05 \\
49586 & 240319 & 2004 & 488.70 & 0.43 & 49184.00 & 395029.05 & 0.99 & 0.81 & 0.80 \\
40845 & 108158 & 2004 & 32.90 & 0.10 & 3271.00 & 31661.51 & 1.01 & 0.96 & 0.97 \\
73395 & 600012 & 2004 & 52.60 & -0.13 & 5277.00 & 50485.00 & 1.00 & 0.96 & 0.96 \\
52621 & 303175 & 2004 & 643.60 & 0.16 & 75722.00 & 625177.82 & 0.85 & 0.97 & 0.83 \\
51956 & 300673 & 2004 & 144.90 & 0.18 & 14463.00 & 136097.97 & 1.00 & 0.94 & 0.94 \\
44726 & 109368 & 2004 & 794.80 & 0.11 & 78047.00 & 651700.94 & 1.02 & 0.82 & 0.84 \\
40984 & 108170 & 2004 & 217.80 & -0.01 & 22290.00 & 214576.98 & 0.98 & 0.99 & 0.96 \\
45682 & 200090 & 2004 & 26.10 & 0.04 & 2988.00 & 29741.47 & 0.87 & 1.14 & 1.00 \\
23417 & 103175 & 2004 & 911.50 & 0.15 & 95950.00 & 903138.80 & 0.95 & 0.99 & 0.94 \\
56339 & 400192 & 2004 & 6.70 & 0.12 & 867.00 & 6307.45 & 0.77 & 0.94 & 0.73 \\
11718 & 101457 & 2004 & 276.00 & 0.07 & 26237.00 & 265455.76 & 1.05 & 0.96 & 1.01 \\
57971 & 410010 & 2004 & 987.80 & 0.15 & 105735.00 & 1044308.75 & 0.93 & 1.06 & 0.99 \\
49684 & 240331 & 2004 & 2.60 & 0.18 & 221.00 & 2415.19 & 1.18 & 0.93 & 1.09 \\
64215 & 500592 & 2004 & 2931.40 & 0.16 & 292892.00 & 2927635.26 & 1.00 & 1.00 & 1.00 \\
47512 & 212408 & 2004 & 2273.20 & 0.06 & 227164.00 & 2257607.47 & 1.00 & 0.99 & 0.99 \\
49578 & 240318 & 2004 & 158.20 & 0.00 & 15085.00 & 143562.58 & 1.05 & 0.91 & 0.95 \\
61571 & 500096 & 2004 & 150.00 & 0.37 & 13595.00 & 135772.60 & 1.10 & 0.91 & 1.00 \\
31625 & 105918 & 2004 & 622.70 & 0.01 & 58576.00 & 619820.02 & 1.06 & 1.00 & 1.06 \\
40835 & 108156 & 2004 & 16.70 & 0.08 & 1655.00 & 16502.26 & 1.01 & 0.99 & 1.00 \\
38276 & 107234 & 2004 & 9.60 & 0.07 & 955.00 & 8469.76 & 1.01 & 0.88 & 0.89 \\
44660 & 109357 & 2004 & 363.10 & 0.10 & 38755.00 & 328854.67 & 0.94 & 0.91 & 0.85 \\
26991 & 103643 & 2004 & 65.50 & 0.12 & 6045.00 & 65600.09 & 1.08 & 1.00 & 1.09 \\
74442 & 601001 & 2004 & 125.70 & 0.16 & 14061.00 & 132030.42 & 0.89 & 1.05 & 0.94 \\
31349 & 105879 & 2004 & 1053.80 & 0.14 & 105424.00 & 1038377.28 & 1.00 & 0.99 & 0.98 \\
7021 & 100985 & 2004 & 3528.20 & 0.07 & 352618.00 & 3479877.85 & 1.00 & 0.99 & 0.99 \\
46977 & 200324 & 2004 & 341.60 & 0.14 & 33261.00 & 325964.88 & 1.03 & 0.95 & 0.98 \\
867 & 100099 & 2004 & 81.50 & 0.16 & 7034.00 & 77970.33 & 1.16 & 0.96 & 1.11 \\
49612 & 240322 & 2004 & 1179.80 & 0.17 & 118062.00 & 1120072.19 & 1.00 & 0.95 & 0.95 \\
52558 & 303124 & 2004 & 61.90 & 0.09 & 5419.00 & 61800.18 & 1.14 & 1.00 & 1.14 \\
6201 & 100829 & 2004 & 911.50 & 0.15 & 90493.00 & 828904.16 & 1.01 & 0.91 & 0.92 \\
3569 & 100456 & 2004 & 1.50 & 0.16 & 118.00 & 1335.84 & 1.27 & 0.89 & 1.13 \\
64169 & 500590 & 2004 & 1063.60 & 0.22 & 106339.00 & 1063041.00 & 1.00 & 1.00 & 1.00 \\
33756 & 106167 & 2004 & 601.70 & 0.21 & 62949.00 & 527630.27 & 0.96 & 0.88 & 0.84 \\
40666 & 108144 & 2004 & 131.80 & 0.01 & 12329.00 & 120000.24 & 1.07 & 0.91 & 0.97 \\
23301 & 103158 & 2004 & 1445.70 & 0.10 & 148266.00 & 1417884.79 & 0.98 & 0.98 & 0.96 \\
11848 & 101463 & 2004 & 786.30 & 0.14 & 79363.00 & 771416.48 & 0.99 & 0.98 & 0.97 \\
43126 & 109065 & 2004 & 55.40 & 0.07 & 5491.00 & 54810.61 & 1.01 & 0.99 & 1.00 \\
6674 & 100908 & 2004 & 304.70 & 0.13 & 26280.00 & 276964.85 & 1.16 & 0.91 & 1.05 \\
26798 & 103607 & 2004 & 194.40 & 0.12 & 16757.00 & 186509.36 & 1.16 & 0.96 & 1.11 \\
12614 & 101560 & 2004 & 73.00 & 0.24 & 7103.00 & 67224.47 & 1.03 & 0.92 & 0.95 \\
52631 & 305184 & 2004 & 43.80 & 0.13 & 4044.00 & 38005.55 & 1.08 & 0.87 & 0.94 \\
56292 & 400187 & 2004 & 24.20 & 0.19 & 2424.00 & 24474.85 & 1.00 & 1.01 & 1.01 \\
34098 & 106207 & 2004 & 17.10 & 0.16 & 1614.00 & 16828.57 & 1.06 & 0.98 & 1.04 \\
54245 & 364993 & 2004 & 77.30 & 0.10 & 6212.00 & 66016.85 & 1.24 & 0.85 & 1.06 \\
52106 & 301571 & 2004 & 332.50 & 0.00 & 33086.00 & 327458.86 & 1.00 & 0.98 & 0.99 \\
37781 & 107144 & 2004 & 52.40 & 0.16 & 5682.00 & 53397.04 & 0.92 & 1.02 & 0.94 \\
64123 & 500588 & 2004 & 1082.70 & 0.13 & 108261.00 & 1082116.38 & 1.00 & 1.00 & 1.00 \\
43885 & 109228 & 2004 & 23.00 & 0.07 & 2528.00 & 23212.34 & 0.91 & 1.01 & 0.92 \\
5295 & 100746 & 2004 & 1987.90 & 0.16 & 173644.00 & 1876884.75 & 1.14 & 0.94 & 1.08 \\
45595 & 200082 & 2004 & 70.20 & 0.29 & 6578.00 & 52689.51 & 1.07 & 0.75 & 0.80 \\
48587 & 240111 & 2004 & 1264.80 & 0.15 & 125766.00 & 1013404.90 & 1.01 & 0.80 & 0.81 \\
40641 & 108143 & 2004 & 11.40 & 0.13 & 1417.00 & 13794.80 & 0.80 & 1.21 & 0.97 \\
26848 & 103609 & 2004 & 24.90 & 0.22 & 2458.00 & 22830.88 & 1.01 & 0.92 & 0.93 \\
17314 & 102280 & 2004 & 1838.60 & 0.19 & 222581.00 & 1621123.03 & 0.83 & 0.88 & 0.73 \\
897 & 100101 & 2004 & 408.90 & 0.07 & 39295.00 & 386428.34 & 1.04 & 0.95 & 0.98 \\
5179 & 100730 & 2004 & 595.10 & 0.21 & 55347.00 & 584391.06 & 1.08 & 0.98 & 1.06 \\
38068 & 107198 & 2004 & 197.50 & 0.23 & 19244.00 & 192457.31 & 1.03 & 0.97 & 1.00 \\
12675 & 101562 & 2004 & 390.00 & 0.15 & 36963.00 & 367217.04 & 1.06 & 0.94 & 0.99 \\
26405 & 103579 & 2004 & 533.80 & 0.13 & 47161.00 & 507019.53 & 1.13 & 0.95 & 1.08 \\
51999 & 300695 & 2004 & 305.70 & 0.14 & 21510.00 & 209892.01 & 1.42 & 0.69 & 0.98 \\
42911 & 109033 & 2004 & 30.70 & 0.12 & 2555.00 & 25393.13 & 1.20 & 0.83 & 0.99 \\
13518 & 101742 & 2004 & 4693.80 & 0.11 & 469652.00 & 4081046.27 & 1.00 & 0.87 & 0.87 \\
40691 & 108145 & 2004 & 52.50 & 0.34 & 5060.00 & 50086.55 & 1.04 & 0.95 & 0.99 \\
18459 & 102461 & 2004 & 848.00 & 0.07 & 84538.00 & 816337.39 & 1.00 & 0.96 & 0.97 \\
23313 & 103160 & 2004 & 420.80 & 0.02 & 56526.00 & 479248.07 & 0.74 & 1.14 & 0.85 \\
15502 & 101999 & 2004 & 1525.20 & 0.18 & 139270.00 & 1294592.51 & 1.10 & 0.85 & 0.93 \\
4115 & 100552 & 2004 & 41.50 & 0.04 & 4098.00 & 39855.89 & 1.01 & 0.96 & 0.97 \\
46926 & 200318 & 2004 & 7.90 & 0.11 & 702.00 & 6351.62 & 1.13 & 0.80 & 0.90 \\
54363 & 367231 & 2004 & 16.00 & 0.01 & 1604.00 & 14604.99 & 1.00 & 0.91 & 0.91 \\
43103 & 109064 & 2004 & 168.20 & 0.26 & 13242.00 & 148022.56 & 1.27 & 0.88 & 1.12 \\
45747 & 200097 & 2004 & 18.90 & 0.16 & 1897.00 & 18707.57 & 1.00 & 0.99 & 0.99 \\
54366 & 367500 & 2004 & 127.00 & 0.10 & 12701.00 & 113222.62 & 1.00 & 0.89 & 0.89 \\
31767 & 105935 & 2004 & 333.80 & 0.10 & 26775.00 & 271257.63 & 1.25 & 0.81 & 1.01 \\
49636 & 240327 & 2004 & 71.90 & 0.06 & 8824.00 & 69031.02 & 0.81 & 0.96 & 0.78 \\
64192 & 500591 & 2004 & 1617.50 & 0.21 & 161673.00 & 1616292.58 & 1.00 & 1.00 & 1.00 \\
6166 & 100827 & 2004 & 199.50 & 0.13 & 19924.00 & 198879.66 & 1.00 & 1.00 & 1.00 \\
34072 & 106203 & 2004 & 60.80 & 0.10 & 5810.00 & 60669.62 & 1.05 & 1.00 & 1.04 \\
27136 & 105246 & 2004 & 5811.70 & 0.10 & 580989.00 & 5352244.92 & 1.00 & 0.92 & 0.92 \\
7434 & 101039 & 2004 & 5658.30 & 0.10 & 527367.00 & 4562640.06 & 1.07 & 0.81 & 0.87 \\
33768 & 106169 & 2004 & 3001.40 & 0.25 & 299716.00 & 2872482.10 & 1.00 & 0.96 & 0.96 \\
6262 & 100833 & 2004 & 484.00 & 0.05 & 48334.00 & 468985.56 & 1.00 & 0.97 & 0.97 \\
44758 & 109373 & 2004 & 446.90 & 0.21 & 33455.00 & 362602.32 & 1.34 & 0.81 & 1.08 \\
56272 & 400186 & 2004 & 19.30 & 0.04 & 1895.00 & 19526.26 & 1.02 & 1.01 & 1.03 \\
2338 & 100319 & 2004 & 402.20 & 0.15 & 34946.00 & 365541.13 & 1.15 & 0.91 & 1.05 \\
20772 & 102789 & 2004 & 1034.80 & 0.12 & 96560.00 & 1001551.47 & 1.07 & 0.97 & 1.04 \\
55449 & 400099 & 2004 & 17.70 & 0.19 & 1422.00 & 14897.54 & 1.24 & 0.84 & 1.05 \\
53610 & 354931 & 2004 & 18.60 & 0.16 & 1539.00 & 13375.61 & 1.21 & 0.72 & 0.87 \\
56370 & 400203 & 2004 & 15.50 & 0.18 & 1564.00 & 15319.30 & 0.99 & 0.99 & 0.98 \\
8281 & 101081 & 2004 & 529.30 & 0.11 & 51303.00 & 487204.68 & 1.03 & 0.92 & 0.95 \\
31453 & 105890 & 2004 & 76.00 & 0.06 & 8576.00 & 74155.72 & 0.89 & 0.98 & 0.86 \\
31818 & 105943 & 2004 & 24.70 & 0.14 & 2267.00 & 23234.24 & 1.09 & 0.94 & 1.02 \\
38544 & 107281 & 2004 & 18.70 & 0.06 & 1771.00 & 18526.00 & 1.06 & 0.99 & 1.05 \\
20623 & 102775 & 2004 & 3599.90 & 0.12 & 361499.00 & 3328784.36 & 1.00 & 0.92 & 0.92 \\
14333 & 101850 & 2004 & 669.20 & 0.14 & 59725.00 & 594237.54 & 1.12 & 0.89 & 0.99 \\
49497 & 240305 & 2004 & 61.60 & -0.00 & 6037.00 & 58943.99 & 1.02 & 0.96 & 0.98 \\
42903 & 109031 & 2004 & 29.30 & 0.07 & 2798.00 & 30050.27 & 1.05 & 1.03 & 1.07 \\
18359 & 102446 & 2004 & 102.80 & 0.22 & 8543.00 & 81009.09 & 1.20 & 0.79 & 0.95 \\
58037 & 410055 & 2004 & 78.30 & 0.04 & 7818.00 & 78188.85 & 1.00 & 1.00 & 1.00 \\
17166 & 102261 & 2004 & 1684.30 & 0.13 & 148025.00 & 1538967.07 & 1.14 & 0.91 & 1.04 \\
43154 & 109069 & 2004 & 288.60 & 0.08 & 28941.00 & 277024.87 & 1.00 & 0.96 & 0.96 \\
54388 & 367567 & 2004 & 448.80 & 0.15 & 42652.00 & 386494.10 & 1.05 & 0.86 & 0.91 \\
23524 & 103183 & 2004 & 700.90 & -0.03 & 68719.00 & 655804.58 & 1.02 & 0.94 & 0.95 \\
26825 & 103608 & 2004 & 61.90 & 0.13 & 3414.00 & 59609.85 & 1.81 & 0.96 & 1.75 \\
38020 & 107192 & 2004 & 1097.80 & 0.18 & 105217.00 & 1024610.49 & 1.04 & 0.93 & 0.97 \\
31232 & 105869 & 2004 & 1354.20 & 0.08 & 135451.00 & 1321309.89 & 1.00 & 0.98 & 0.98 \\
43006 & 109048 & 2004 & 245.80 & 0.07 & 23303.00 & 244988.75 & 1.05 & 1.00 & 1.05 \\
5087 & 100723 & 2004 & 40.60 & 0.07 & 4123.00 & 43049.67 & 0.98 & 1.06 & 1.04 \\
34114 & 106208 & 2004 & 32.10 & 0.12 & 3127.00 & 30529.96 & 1.03 & 0.95 & 0.98 \\
9803 & 101193 & 2004 & 636.10 & 0.11 & 58058.00 & 604794.44 & 1.10 & 0.95 & 1.04 \\
40616 & 108142 & 2004 & 58.10 & 0.20 & 5386.00 & 51077.62 & 1.08 & 0.88 & 0.95 \\
48528 & 240103 & 2004 & 604.10 & 0.21 & 60516.00 & 558951.71 & 1.00 & 0.93 & 0.92 \\
19473 & 102606 & 2004 & 4006.50 & 0.11 & 380864.00 & 3624355.54 & 1.05 & 0.90 & 0.95 \\
38043 & 107196 & 2004 & 21.80 & 0.14 & 2171.00 & 19524.15 & 1.00 & 0.90 & 0.90 \\
2171 & 100293 & 2004 & 105.50 & 0.05 & 10587.00 & 104769.45 & 1.00 & 0.99 & 0.99 \\
17296 & 102278 & 2004 & 136.30 & 0.11 & 14834.00 & 130325.84 & 0.92 & 0.96 & 0.88 \\
3500 & 100441 & 2004 & 299.40 & 0.11 & 30365.00 & 289707.31 & 0.99 & 0.97 & 0.95 \\
15703 & 102015 & 2004 & 203.70 & 0.06 & 24015.00 & 205824.51 & 0.85 & 1.01 & 0.86 \\
31795 & 105936 & 2004 & 39.10 & 0.00 & 4187.00 & 39580.62 & 0.93 & 1.01 & 0.95 \\
38510 & 107266 & 2004 & 377.50 & 0.19 & 35319.00 & 286524.82 & 1.07 & 0.76 & 0.81 \\
47066 & 200331 & 2004 & 26.00 & 0.15 & 2601.00 & 22142.89 & 1.00 & 0.85 & 0.85 \\
64284 & 500595 & 2004 & 1869.20 & 0.11 & 186727.00 & 1866101.06 & 1.00 & 1.00 & 1.00 \\
38533 & 107274 & 2004 & 8.50 & 0.15 & 885.00 & 8013.32 & 0.96 & 0.94 & 0.91 \\
9088 & 101111 & 2004 & 507.40 & 0.12 & 52004.00 & 473829.45 & 0.98 & 0.93 & 0.91 \\
64146 & 500589 & 2004 & 2322.60 & 0.16 & 232155.00 & 2319186.47 & 1.00 & 1.00 & 1.00 \\
9040 & 101109 & 2004 & 447.40 & 0.45 & 29880.00 & 289720.33 & 1.50 & 0.65 & 0.97 \\
45848 & 200148 & 2004 & 235.20 & 0.18 & 24995.00 & 220525.81 & 0.94 & 0.94 & 0.88 \\
53657 & 355965 & 2004 & 1670.50 & 0.16 & 150273.00 & 1576801.69 & 1.11 & 0.94 & 1.05 \\
42982 & 109046 & 2004 & 57.50 & 0.17 & 5757.00 & 56666.03 & 1.00 & 0.99 & 0.98 \\
44770 & 109374 & 2004 & 106.40 & 0.18 & 9928.00 & 108811.26 & 1.07 & 1.02 & 1.10 \\
23286 & 103154 & 2004 & 379.20 & 0.12 & 44607.00 & 330082.45 & 0.85 & 0.87 & 0.74 \\
15483 & 101998 & 2004 & 392.10 & 0.20 & 61056.00 & 629601.80 & 0.64 & 1.61 & 1.03 \\
52134 & 302060 & 2004 & 64.50 & 0.09 & 7091.00 & 64206.80 & 0.91 & 1.00 & 0.91 \\
31470 & 105895 & 2004 & 955.90 & 0.01 & 88735.00 & 991993.37 & 1.08 & 1.04 & 1.12 \\
34297 & 106221 & 2004 & 118.20 & 0.13 & 10919.00 & 117316.12 & 1.08 & 0.99 & 1.07 \\
26346 & 103570 & 2004 & 23.20 & 0.14 & 2236.00 & 22358.34 & 1.04 & 0.96 & 1.00 \\
23800 & 103213 & 2004 & 522.50 & 0.09 & 51342.00 & 533465.47 & 1.02 & 1.02 & 1.04 \\
56390 & 400204 & 2004 & 8.60 & 0.11 & 883.00 & 7485.89 & 0.97 & 0.87 & 0.85 \\
37753 & 107141 & 2004 & 1970.90 & 0.29 & 162374.00 & 1708467.05 & 1.21 & 0.87 & 1.05 \\
31272 & 105873 & 2004 & 9.80 & 0.07 & 922.00 & 7469.31 & 1.06 & 0.76 & 0.81 \\
61676 & 500114 & 2004 & 30.30 & 0.07 & 3031.00 & 27751.96 & 1.00 & 0.92 & 0.92 \\
5219 & 100736 & 2004 & 508.10 & 0.08 & 50256.00 & 471977.84 & 1.01 & 0.93 & 0.94 \\
74593 & 601139 & 2004 & 6662.00 & 0.12 & 648542.00 & 6318334.66 & 1.03 & 0.95 & 0.97 \\
689 & 100090 & 2004 & 496.20 & 0.05 & 49898.00 & 518346.12 & 0.99 & 1.04 & 1.04 \\
38000 & 107181 & 2004 & 121.20 & 0.13 & 12112.00 & 109738.43 & 1.00 & 0.91 & 0.91 \\
42935 & 109037 & 2004 & 58.70 & 0.02 & 5875.00 & 49760.34 & 1.00 & 0.85 & 0.85 \\
45730 & 200094 & 2004 & 122.00 & 0.15 & 12200.00 & 116066.70 & 1.00 & 0.95 & 0.95 \\
64077 & 500586 & 2004 & 4709.70 & 0.16 & 470299.00 & 4698281.49 & 1.00 & 1.00 & 1.00 \\
34264 & 106216 & 2004 & 572.20 & 0.19 & 74483.00 & 571478.32 & 0.77 & 1.00 & 0.77 \\
31729 & 105932 & 2004 & 193.20 & 0.14 & 18622.00 & 170164.05 & 1.04 & 0.88 & 0.91 \\
47045 & 200330 & 2004 & 29.00 & 0.13 & 2478.00 & 25149.24 & 1.17 & 0.87 & 1.01 \\
59365 & 410470 & 2004 & 26.20 & 0.12 & 1501.00 & 15555.69 & 1.75 & 0.59 & 1.04 \\
38402 & 107257 & 2004 & 252.30 & 0.03 & 25267.00 & 232055.97 & 1.00 & 0.92 & 0.92 \\
52093 & 301560 & 2004 & 1397.20 & 0.20 & 123686.00 & 1386158.06 & 1.13 & 0.99 & 1.12 \\
45708 & 200092 & 2004 & 22.40 & 0.24 & 2246.00 & 19667.43 & 1.00 & 0.88 & 0.88 \\
54341 & 367206 & 2004 & 244.80 & 0.17 & 24459.00 & 233795.75 & 1.00 & 0.96 & 0.96 \\
40716 & 108146 & 2004 & 56.80 & -0.21 & 6189.00 & 60624.47 & 0.92 & 1.07 & 0.98 \\
37988 & 107179 & 2004 & 1079.10 & 0.10 & 110372.00 & 1087636.39 & 0.98 & 1.01 & 0.99 \\
26437 & 103580 & 2004 & 270.70 & 0.09 & 26212.00 & 260727.51 & 1.03 & 0.96 & 0.99 \\
5201 & 100731 & 2004 & 6881.20 & 0.08 & 687419.00 & 6531187.73 & 1.00 & 0.95 & 0.95 \\
51982 & 300684 & 2004 & 147.00 & 0.14 & 14260.00 & 143184.84 & 1.03 & 0.97 & 1.00 \\
61494 & 500083 & 2004 & 19.40 & 0.32 & 1770.00 & 19596.21 & 1.10 & 1.01 & 1.11 \\
52595 & 303140 & 2004 & 848.20 & 0.11 & 80968.00 & 834061.35 & 1.05 & 0.98 & 1.03 \\
38427 & 107258 & 2004 & 80.80 & 0.07 & 8165.00 & 73818.52 & 0.99 & 0.91 & 0.90 \\
45831 & 200147 & 2004 & 52.60 & 0.08 & 5313.00 & 53529.56 & 0.99 & 1.02 & 1.01 \\
7805 & 101061 & 2004 & 3618.40 & 0.04 & 368746.00 & 3211140.34 & 0.98 & 0.89 & 0.87 \\
9120 & 101112 & 2004 & 821.80 & 0.10 & 79585.00 & 803225.61 & 1.03 & 0.98 & 1.01 \\
33808 & 106172 & 2004 & 59.90 & 0.12 & 6036.00 & 57030.97 & 0.99 & 0.95 & 0.94 \\
37829 & 107147 & 2004 & 25.90 & -0.03 & 2598.00 & 25524.59 & 1.00 & 0.99 & 0.98 \\
31283 & 105874 & 2004 & 351.80 & 0.15 & 30170.00 & 304096.25 & 1.17 & 0.86 & 1.01 \\
23557 & 103186 & 2004 & 822.40 & 0.08 & 77440.00 & 807854.33 & 1.06 & 0.98 & 1.04 \\
26766 & 103606 & 2004 & 131.10 & 0.11 & 11103.00 & 105929.25 & 1.18 & 0.81 & 0.95 \\
13462 & 101740 & 2004 & 13604.10 & 0.05 & 1361838.00 & 12559088.94 & 1.00 & 0.92 & 0.92 \\
17195 & 102270 & 2004 & 700.00 & 0.03 & 70021.00 & 665980.03 & 1.00 & 0.95 & 0.95 \\
33795 & 106170 & 2004 & 413.10 & 0.28 & 41248.00 & 398734.17 & 1.00 & 0.97 & 0.97 \\
58785 & 410217 & 2004 & 5.00 & 0.21 & 411.00 & 4875.04 & 1.22 & 0.98 & 1.19 \\
43844 & 109224 & 2004 & 11.00 & 0.14 & 1107.00 & 10862.50 & 0.99 & 0.99 & 0.98 \\
58643 & 410181 & 2004 & 4.30 & -0.01 & 342.00 & 3179.87 & 1.26 & 0.74 & 0.93 \\
58015 & 410018 & 2004 & 12.50 & 0.18 & 1227.00 & 12144.44 & 1.02 & 0.97 & 0.99 \\
2140 & 100292 & 2004 & 4196.60 & 0.22 & 419197.00 & 3455640.26 & 1.00 & 0.82 & 0.82 \\
66296 & 500806 & 2004 & 404.30 & 0.08 & 39535.00 & 386437.86 & 1.02 & 0.96 & 0.98 \\
9834 & 101194 & 2004 & 238.80 & 0.15 & 20669.00 & 185978.24 & 1.16 & 0.78 & 0.90 \\
54891 & 400020 & 2004 & 116.10 & 0.13 & 9223.00 & 90894.95 & 1.26 & 0.78 & 0.99 \\
20603 & 102774 & 2004 & 3015.70 & 0.19 & 303813.00 & 2991296.23 & 0.99 & 0.99 & 0.98 \\
49015 & 240199 & 2004 & 534.60 & 0.10 & 53508.00 & 506600.44 & 1.00 & 0.95 & 0.95 \\
20453 & 102744 & 2004 & 787.10 & 0.15 & 78816.00 & 774052.62 & 1.00 & 0.98 & 0.98 \\
51922 & 300653 & 2004 & 310.90 & 0.16 & 31180.00 & 300904.66 & 1.00 & 0.97 & 0.97 \\
11752 & 101460 & 2004 & 1656.00 & 0.16 & 193703.00 & 1422430.46 & 0.85 & 0.86 & 0.73 \\
33835 & 106173 & 2004 & 1104.70 & 0.22 & 96680.00 & 907054.00 & 1.14 & 0.82 & 0.94 \\
41018 & 108180 & 2004 & 94.80 & 0.18 & 8088.00 & 88395.41 & 1.17 & 0.93 & 1.09 \\
20733 & 102788 & 2004 & 403.40 & 0.16 & 36521.00 & 348288.33 & 1.10 & 0.86 & 0.95 \\
31756 & 105933 & 2004 & 1662.20 & 0.06 & 170465.00 & 1543376.21 & 0.98 & 0.93 & 0.91 \\
49720 & 240337 & 2004 & 9.40 & 0.04 & 885.00 & 7769.79 & 1.06 & 0.83 & 0.88 \\
53681 & 355987 & 2004 & 500.10 & 0.19 & 48757.00 & 467954.63 & 1.03 & 0.94 & 0.96 \\
17441 & 102306 & 2004 & 26953.80 & 0.12 & 2476467.00 & 21846285.12 & 1.09 & 0.81 & 0.88 \\
59383 & 410472 & 2004 & 871.30 & 0.12 & 84779.00 & 857195.15 & 1.03 & 0.98 & 1.01 \\
31297 & 105875 & 2004 & 36.90 & 0.11 & 3507.00 & 35413.71 & 1.05 & 0.96 & 1.01 \\
23764 & 103212 & 2004 & 2547.10 & 0.18 & 239539.00 & 2393092.90 & 1.06 & 0.94 & 1.00 \\
2204 & 100295 & 2004 & 17.00 & 0.08 & 1626.00 & 13623.91 & 1.05 & 0.80 & 0.84 \\
46933 & 200319 & 2004 & 166.80 & 0.17 & 13669.00 & 137883.62 & 1.22 & 0.83 & 1.01 \\
38461 & 107260 & 2004 & 1021.00 & 0.13 & 100811.00 & 889101.57 & 1.01 & 0.87 & 0.88 \\
49002 & 240198 & 2004 & 231.10 & 0.16 & 23456.00 & 228023.52 & 0.99 & 0.99 & 0.97 \\
38084 & 107199 & 2004 & 20.10 & 0.16 & 1956.00 & 19557.52 & 1.03 & 0.97 & 1.00 \\
56444 & 400210 & 2004 & 292.00 & 0.04 & 26761.00 & 259588.04 & 1.09 & 0.89 & 0.97 \\
44751 & 109371 & 2004 & 459.60 & 0.12 & 49531.00 & 469853.14 & 0.93 & 1.02 & 0.95 \\
200 & 100018 & 2005 & 159.10 & 0.13 & 15320.00 & 146742.12 & 1.04 & 0.92 & 0.96 \\
53022 & 337150 & 2005 & 187.00 & 0.10 & 18727.00 & 187195.14 & 1.00 & 1.00 & 1.00 \\
32450 & 106028 & 2005 & 1025.30 & 0.05 & 102611.00 & 956459.02 & 1.00 & 0.93 & 0.93 \\
6692 & 100910 & 2005 & 187.70 & 0.06 & 18211.00 & 178145.90 & 1.03 & 0.95 & 0.98 \\
7353 & 101023 & 2005 & 26057.70 & 0.06 & 2533704.00 & 25015871.25 & 1.03 & 0.96 & 0.99 \\
52419 & 302907 & 2005 & 1083.70 & 0.06 & 116073.00 & 1160725.84 & 0.93 & 1.07 & 1.00 \\
29948 & 105659 & 2005 & 176.60 & 0.07 & 17657.00 & 175652.07 & 1.00 & 0.99 & 0.99 \\
60939 & 410756 & 2005 & 93.40 & 0.07 & 12809.00 & 109749.35 & 0.73 & 1.18 & 0.86 \\
55940 & 400160 & 2005 & 55.70 & 0.08 & 5565.00 & 54867.62 & 1.00 & 0.99 & 0.99 \\
10134 & 101262 & 2005 & 13.60 & 0.01 & 1402.00 & 14239.16 & 0.97 & 1.05 & 1.02 \\
31454 & 105890 & 2005 & 67.10 & 0.10 & 7343.00 & 73423.43 & 0.91 & 1.09 & 1.00 \\
32566 & 106041 & 2005 & 205.10 & 0.15 & 22909.00 & 196945.23 & 0.90 & 0.96 & 0.86 \\
62384 & 500389 & 2005 & 131.40 & 0.16 & 13128.00 & 131250.18 & 1.00 & 1.00 & 1.00 \\
34196 & 106211 & 2005 & 64.80 & 0.03 & 7274.00 & 75773.93 & 0.89 & 1.17 & 1.04 \\
5947 & 100812 & 2005 & 238.20 & 0.06 & 22875.00 & 229620.38 & 1.04 & 0.96 & 1.00 \\
7532 & 101043 & 2005 & 4329.10 & 0.07 & 435059.00 & 4359241.46 & 1.00 & 1.01 & 1.00 \\
51242 & 240493 & 2005 & 81.40 & 0.05 & 8122.00 & 79833.49 & 1.00 & 0.98 & 0.98 \\
56018 & 400166 & 2005 & 582.40 & 0.04 & 54682.00 & 532360.64 & 1.07 & 0.91 & 0.97 \\
31003 & 105846 & 2005 & 1291.70 & 0.10 & 126572.00 & 1269452.05 & 1.02 & 0.98 & 1.00 \\
51249 & 240495 & 2005 & 2768.00 & -0.01 & 283566.00 & 2574240.58 & 0.98 & 0.93 & 0.91 \\
35338 & 106348 & 2005 & 152.10 & 0.14 & 11778.00 & 127792.97 & 1.29 & 0.84 & 1.09 \\
62424 & 500391 & 2005 & 50.40 & 0.06 & 5033.00 & 50310.35 & 1.00 & 1.00 & 1.00 \\
43845 & 109224 & 2005 & 11.90 & 0.03 & 1192.00 & 11653.90 & 1.00 & 0.98 & 0.98 \\
51222 & 240492 & 2005 & 74.70 & 0.08 & 7292.00 & 72827.28 & 1.02 & 0.97 & 1.00 \\
42983 & 109046 & 2005 & 56.20 & 0.10 & 5632.00 & 54674.81 & 1.00 & 0.97 & 0.97 \\
62444 & 500392 & 2005 & 47.20 & 0.05 & 4720.00 & 47188.80 & 1.00 & 1.00 & 1.00 \\
56038 & 400167 & 2005 & 249.60 & 0.10 & 21497.00 & 217124.49 & 1.16 & 0.87 & 1.01 \\
43419 & 109112 & 2005 & 50.60 & 0.03 & 5061.00 & 48441.67 & 1.00 & 0.96 & 0.96 \\
19372 & 102599 & 2005 & 2861.70 & 0.07 & 278585.00 & 2785770.40 & 1.03 & 0.97 & 1.00 \\
3188 & 100413 & 2005 & 80.60 & 0.06 & 7865.00 & 74311.62 & 1.02 & 0.92 & 0.94 \\
56481 & 400216 & 2005 & 210.10 & -0.01 & 22354.00 & 203992.85 & 0.94 & 0.97 & 0.91 \\
30023 & 105678 & 2005 & 45.40 & 0.04 & 5727.00 & 45215.69 & 0.79 & 1.00 & 0.79 \\
62404 & 500390 & 2005 & 77.60 & 0.09 & 7760.00 & 77508.42 & 1.00 & 1.00 & 1.00 \\
56810 & 400256 & 2005 & 20.60 & 0.08 & 2127.00 & 19409.64 & 0.97 & 0.94 & 0.91 \\
62344 & 500387 & 2005 & 149.70 & 0.03 & 14968.00 & 149619.76 & 1.00 & 1.00 & 1.00 \\
42969 & 109044 & 2005 & 57.70 & 0.07 & 6579.00 & 64527.48 & 0.88 & 1.12 & 0.98 \\
6461 & 100875 & 2005 & 747.90 & 0.10 & 74792.00 & 745583.00 & 1.00 & 1.00 & 1.00 \\
29933 & 105658 & 2005 & 119.10 & 0.05 & 17165.00 & 171316.27 & 0.69 & 1.44 & 1.00 \\
46758 & 200295 & 2005 & 466.90 & 0.09 & 46664.00 & 449650.90 & 1.00 & 0.96 & 0.96 \\
54972 & 400034 & 2005 & 6.00 & 0.07 & 565.00 & 5738.78 & 1.06 & 0.96 & 1.02 \\
19985 & 102660 & 2005 & 8703.20 & 0.06 & 870321.00 & 8539862.83 & 1.00 & 0.98 & 0.98 \\
33128 & 106092 & 2005 & 514.80 & 0.09 & 53102.00 & 518924.55 & 0.97 & 1.01 & 0.98 \\
15 & 100001 & 2005 & 3618.70 & 0.05 & 361783.00 & 3448154.13 & 1.00 & 0.95 & 0.95 \\
46899 & 200312 & 2005 & 670.20 & 0.04 & 67052.00 & 656857.80 & 1.00 & 0.98 & 0.98 \\
46557 & 200253 & 2005 & 4.70 & -0.01 & 470.00 & 4695.16 & 1.00 & 1.00 & 1.00 \\
56391 & 400204 & 2005 & 19.30 & 0.09 & 1928.00 & 18674.10 & 1.00 & 0.97 & 0.97 \\
34115 & 106208 & 2005 & 28.80 & -0.00 & 2904.00 & 26667.62 & 0.99 & 0.93 & 0.92 \\
55739 & 400144 & 2005 & 17.70 & 0.11 & 3229.00 & 30668.38 & 0.55 & 1.73 & 0.95 \\
34476 & 106244 & 2005 & 1.50 & 0.03 & 156.00 & 1520.53 & 0.96 & 1.01 & 0.97 \\
51202 & 240491 & 2005 & 161.60 & 0.11 & 15392.00 & 153905.34 & 1.05 & 0.95 & 1.00 \\
51134 & 240486 & 2005 & 6.50 & 0.08 & 648.00 & 6329.40 & 1.00 & 0.97 & 0.98 \\
34169 & 106210 & 2005 & 810.40 & 0.02 & 80796.00 & 789745.36 & 1.00 & 0.97 & 0.98 \\
48271 & 240058 & 2005 & 1184.90 & 0.07 & 120723.00 & 1146339.71 & 0.98 & 0.97 & 0.95 \\
56470 & 400213 & 2005 & 3.60 & 0.13 & 349.00 & 3045.06 & 1.03 & 0.85 & 0.87 \\
33221 & 106103 & 2005 & 8.60 & 0.03 & 1179.00 & 11774.42 & 0.73 & 1.37 & 1.00 \\
53074 & 338387 & 2005 & 7251.80 & 0.03 & 723239.00 & 7227728.50 & 1.00 & 1.00 & 1.00 \\
34142 & 106209 & 2005 & 432.20 & -0.02 & 43452.00 & 392991.20 & 0.99 & 0.91 & 0.90 \\
52326 & 302763 & 2005 & 667.30 & 0.09 & 66774.00 & 658050.29 & 1.00 & 0.99 & 0.99 \\
29976 & 105664 & 2005 & 457.90 & 0.02 & 73036.00 & 709319.12 & 0.63 & 1.55 & 0.97 \\
19955 & 102659 & 2005 & 6539.40 & 0.12 & 653944.00 & 6272527.29 & 1.00 & 0.96 & 0.96 \\
15644 & 102010 & 2005 & 3670.80 & 0.08 & 359379.00 & 3489634.76 & 1.02 & 0.95 & 0.97 \\
43961 & 109238 & 2005 & 22.20 & 0.05 & 2179.00 & 21607.09 & 1.02 & 0.97 & 0.99 \\
13463 & 101740 & 2005 & 11239.00 & 0.04 & 1150280.00 & 10347956.95 & 0.98 & 0.92 & 0.90 \\
48588 & 240111 & 2005 & 1163.60 & 0.11 & 131687.00 & 1084534.04 & 0.88 & 0.93 & 0.82 \\
31395 & 105881 & 2005 & 5115.60 & 0.12 & 487991.00 & 4518914.83 & 1.05 & 0.88 & 0.93 \\
9962 & 101215 & 2005 & 17.10 & 0.07 & 1737.00 & 15010.93 & 0.98 & 0.88 & 0.86 \\
32478 & 106033 & 2005 & 2894.60 & 0.13 & 253867.00 & 2415099.87 & 1.14 & 0.83 & 0.95 \\
54755 & 378596 & 2005 & 319.00 & 0.15 & 21721.00 & 195342.75 & 1.47 & 0.61 & 0.90 \\
53060 & 337871 & 2005 & 235.30 & 0.03 & 16652.00 & 168484.19 & 1.41 & 0.72 & 1.01 \\
32522 & 106038 & 2005 & 6107.50 & 0.07 & 610753.00 & 5245261.26 & 1.00 & 0.86 & 0.86 \\
62295 & 500377 & 2005 & 2.70 & 0.04 & 261.00 & 2679.37 & 1.03 & 0.99 & 1.03 \\
31045 & 105852 & 2005 & 249.30 & 0.07 & 24822.00 & 233931.78 & 1.00 & 0.94 & 0.94 \\
56826 & 400257 & 2005 & 274.30 & 0.39 & 27580.00 & 274261.17 & 0.99 & 1.00 & 0.99 \\
46803 & 200298 & 2005 & 119.40 & 0.13 & 13601.00 & 112529.65 & 0.88 & 0.94 & 0.83 \\
33151 & 106097 & 2005 & 556.70 & 0.10 & 54879.00 & 548760.21 & 1.01 & 0.99 & 1.00 \\
55953 & 400161 & 2005 & 170.80 & 0.08 & 16605.00 & 171312.93 & 1.03 & 1.00 & 1.03 \\
42802 & 109019 & 2005 & 68.30 & 0.07 & 6632.00 & 61612.69 & 1.03 & 0.90 & 0.93 \\
56771 & 400254 & 2005 & 44.10 & 0.14 & 4417.00 & 41089.53 & 1.00 & 0.93 & 0.93 \\
51182 & 240490 & 2005 & 44.60 & 0.03 & 4895.00 & 48616.61 & 0.91 & 1.09 & 0.99 \\
43388 & 109110 & 2005 & 48.10 & 0.07 & 4805.00 & 46473.20 & 1.00 & 0.97 & 0.97 \\
15180 & 101964 & 2005 & 2522.60 & 0.04 & 241620.00 & 2450945.46 & 1.04 & 0.97 & 1.01 \\
43430 & 109118 & 2005 & 196.20 & 0.04 & 20527.00 & 188101.30 & 0.96 & 0.96 & 0.92 \\
53039 & 337653 & 2005 & 158.80 & 0.03 & 14780.00 & 148430.70 & 1.07 & 0.93 & 1.00 \\
96668 & 611002 & 2005 & 2664.40 & -0.01 & 284868.00 & 2401975.37 & 0.94 & 0.90 & 0.84 \\
43396 & 109111 & 2005 & 199.00 & 0.06 & 20146.00 & 196661.88 & 0.99 & 0.99 & 0.98 \\
46781 & 200297 & 2005 & 2706.30 & 0.15 & 275868.00 & 2565076.05 & 0.98 & 0.95 & 0.93 \\
6202 & 100829 & 2005 & 1105.70 & 0.14 & 105591.00 & 1067463.42 & 1.05 & 0.97 & 1.01 \\
20660 & 102777 & 2005 & 1145.20 & 0.07 & 113450.00 & 1121552.70 & 1.01 & 0.98 & 0.99 \\
16113 & 102080 & 2005 & 3005.90 & 0.09 & 336351.00 & 3515497.38 & 0.89 & 1.17 & 1.05 \\
62364 & 500388 & 2005 & 123.50 & 0.07 & 12341.00 & 121925.69 & 1.00 & 0.99 & 0.99 \\
48390 & 240074 & 2005 & 683.50 & 0.46 & 62467.00 & 617058.15 & 1.09 & 0.90 & 0.99 \\
35310 & 106345 & 2005 & 587.20 & 0.11 & 59630.00 & 586011.13 & 0.98 & 1.00 & 0.98 \\
47341 & 210203 & 2005 & 8321.30 & 0.14 & 800575.00 & 7499777.92 & 1.04 & 0.90 & 0.94 \\
55919 & 400159 & 2005 & 95.60 & 0.07 & 8655.00 & 80863.69 & 1.10 & 0.85 & 0.93 \\
42388 & 108960 & 2005 & 377.70 & 0.12 & 39983.00 & 399828.50 & 0.94 & 1.06 & 1.00 \\
54979 & 400037 & 2005 & 476.20 & 0.04 & 67795.00 & 676307.68 & 0.70 & 1.42 & 1.00 \\
51957 & 300673 & 2005 & 124.50 & -0.02 & 14638.00 & 144887.94 & 0.85 & 1.16 & 0.99 \\
56766 & 400252 & 2005 & 72.90 & 0.03 & 6462.00 & 66895.11 & 1.13 & 0.92 & 1.04 \\
34529 & 106249 & 2005 & 184.30 & 0.04 & 18275.00 & 177816.03 & 1.01 & 0.96 & 0.97 \\
9223 & 101119 & 2005 & 445.50 & 0.09 & 44555.00 & 426869.60 & 1.00 & 0.96 & 0.96 \\
51968 & 300679 & 2005 & 1248.90 & 0.09 & 125332.00 & 1211921.60 & 1.00 & 0.97 & 0.97 \\
32550 & 106039 & 2005 & 817.40 & -0.03 & 97775.00 & 955939.41 & 0.84 & 1.17 & 0.98 \\
33194 & 106102 & 2005 & 697.60 & 0.14 & 72389.00 & 633150.11 & 0.96 & 0.91 & 0.87 \\
56791 & 400255 & 2005 & 38.10 & 0.09 & 3810.00 & 34939.63 & 1.00 & 0.92 & 0.92 \\
35283 & 106344 & 2005 & 815.20 & 0.05 & 79093.00 & 716371.72 & 1.03 & 0.88 & 0.91 \\
29994 & 105665 & 2005 & 42.20 & 0.09 & 3982.00 & 39338.41 & 1.06 & 0.93 & 0.99 \\
51162 & 240489 & 2005 & 75.10 & 0.01 & 7495.00 & 74294.65 & 1.00 & 0.99 & 0.99 \\
35325 & 106347 & 2005 & 111.80 & 0.16 & 8989.00 & 98372.10 & 1.24 & 0.88 & 1.09 \\
43700 & 109190 & 2005 & 25.40 & -0.02 & 2689.00 & 26797.01 & 0.94 & 1.06 & 1.00 \\
42818 & 109020 & 2005 & 258.90 & 0.03 & 26377.00 & 252899.99 & 0.98 & 0.98 & 0.96 \\
30051 & 105679 & 2005 & 422.20 & 0.06 & 45012.00 & 454861.20 & 0.94 & 1.08 & 1.01 \\
20890 & 102798 & 2005 & 136.50 & 0.09 & 13403.00 & 131912.35 & 1.02 & 0.97 & 0.98 \\
4020 & 100538 & 2005 & 592.10 & 0.14 & 58924.00 & 566341.48 & 1.00 & 0.96 & 0.96 \\
8764 & 101097 & 2005 & 348.00 & 0.05 & 35452.00 & 344290.52 & 0.98 & 0.99 & 0.97 \\
51156 & 240487 & 2005 & 6.80 & 0.06 & 651.00 & 6451.86 & 1.04 & 0.95 & 0.99 \\
42794 & 109018 & 2005 & 17.70 & 0.01 & 1730.00 & 15912.32 & 1.02 & 0.90 & 0.92 \\
60937 & 410755 & 2005 & 3.60 & 0.01 & 289.00 & 3125.97 & 1.25 & 0.87 & 1.08 \\
48614 & 240114 & 2005 & 276.00 & 0.08 & 27613.00 & 273724.91 & 1.00 & 0.99 & 0.99 \\
10572 & 101299 & 2005 & 2319.20 & 0.08 & 232312.00 & 2284676.43 & 1.00 & 0.99 & 0.98 \\
29960 & 105662 & 2005 & 103.00 & -0.02 & 10870.00 & 110325.13 & 0.95 & 1.07 & 1.01 \\
51983 & 300684 & 2005 & 135.80 & 0.09 & 13464.00 & 137841.51 & 1.01 & 1.02 & 1.02 \\
52684 & 306482 & 2005 & 186.40 & 0.00 & 18119.00 & 172549.72 & 1.03 & 0.93 & 0.95 \\
48410 & 240076 & 2005 & 10.70 & 0.05 & 1276.00 & 12974.71 & 0.84 & 1.21 & 1.02 \\
52399 & 302879 & 2005 & 684.00 & 0.15 & 52439.00 & 522787.96 & 1.30 & 0.76 & 1.00 \\
54734 & 378592 & 2005 & 3.10 & 0.06 & 168.00 & 1560.05 & 1.85 & 0.50 & 0.93 \\
59217 & 410445 & 2005 & 125.70 & 0.13 & 12265.00 & 122223.26 & 1.02 & 0.97 & 1.00 \\
43563 & 109144 & 2005 & 230.40 & -0.90 & 23023.00 & 228545.46 & 1.00 & 0.99 & 0.99 \\
47416 & 210770 & 2005 & 5601.70 & 0.10 & 571314.00 & 5425471.66 & 0.98 & 0.97 & 0.95 \\
50952 & 240473 & 2005 & 66.00 & 0.10 & 6601.00 & 65148.31 & 1.00 & 0.99 & 0.99 \\
52655 & 305590 & 2005 & 330.30 & 0.17 & 32614.00 & 323714.27 & 1.01 & 0.98 & 0.99 \\
43558 & 109143 & 2005 & 170.90 & 0.03 & 19319.00 & 191799.66 & 0.88 & 1.12 & 0.99 \\
42322 & 108951 & 2005 & 105.40 & 0.04 & 10338.00 & 90770.39 & 1.02 & 0.86 & 0.88 \\
42912 & 109033 & 2005 & 24.50 & -0.02 & 2144.00 & 20936.83 & 1.14 & 0.85 & 0.98 \\
42844 & 109025 & 2005 & 477.80 & 0.16 & 52783.00 & 415526.98 & 0.91 & 0.87 & 0.79 \\
21642 & 102937 & 2005 & 147.30 & 0.06 & 14368.00 & 132987.07 & 1.03 & 0.90 & 0.93 \\
50973 & 240474 & 2005 & 145.10 & 0.18 & 9067.00 & 78796.15 & 1.60 & 0.54 & 0.87 \\
35524 & 106370 & 2005 & 73.80 & 0.11 & 7382.00 & 70627.85 & 1.00 & 0.96 & 0.96 \\
32759 & 106061 & 2005 & 822.40 & 0.17 & 82424.00 & 802273.02 & 1.00 & 0.98 & 0.97 \\
50950 & 240470 & 2005 & 44.70 & 0.03 & 4384.00 & 42674.57 & 1.02 & 0.95 & 0.97 \\
51923 & 300653 & 2005 & 403.50 & 0.08 & 39449.00 & 373840.25 & 1.02 & 0.93 & 0.95 \\
29687 & 105635 & 2005 & 22.90 & 0.07 & 2301.00 & 22496.58 & 1.00 & 0.98 & 0.98 \\
4116 & 100552 & 2005 & 7.60 & -0.09 & 725.00 & 7248.29 & 1.05 & 0.95 & 1.00 \\
35559 & 106375 & 2005 & 21.60 & 0.04 & 2228.00 & 21587.61 & 0.97 & 1.00 & 0.97 \\
46853 & 200309 & 2005 & 171.70 & 0.04 & 16641.00 & 163910.84 & 1.03 & 0.95 & 0.98 \\
6167 & 100827 & 2005 & 165.50 & 0.07 & 19809.00 & 178047.53 & 0.84 & 1.08 & 0.90 \\
56445 & 400210 & 2005 & 288.70 & 0.04 & 29383.00 & 279942.38 & 0.98 & 0.97 & 0.95 \\
50944 & 240469 & 2005 & 93.60 & 0.09 & 9022.00 & 89943.30 & 1.04 & 0.96 & 1.00 \\
74858 & 601187 & 2005 & 80.00 & 0.03 & 11547.00 & 108284.54 & 0.69 & 1.35 & 0.94 \\
19406 & 102600 & 2005 & 978.20 & 0.05 & 97574.00 & 970077.69 & 1.00 & 0.99 & 0.99 \\
42297 & 108950 & 2005 & 820.40 & 0.01 & 78790.00 & 756179.83 & 1.04 & 0.92 & 0.96 \\
19804 & 102652 & 2005 & 1722.00 & 0.03 & 172201.00 & 1642246.98 & 1.00 & 0.95 & 0.95 \\
16206 & 102090 & 2005 & 5184.30 & 0.09 & 572095.00 & 5264252.03 & 0.91 & 1.02 & 0.92 \\
31273 & 105873 & 2005 & 9.10 & 0.00 & 948.00 & 9721.96 & 0.96 & 1.07 & 1.03 \\
29708 & 105640 & 2005 & 2162.50 & 0.12 & 210925.00 & 1696003.20 & 1.03 & 0.78 & 0.80 \\
9392 & 101133 & 2005 & 741.20 & 0.05 & 70299.00 & 740975.51 & 1.05 & 1.00 & 1.05 \\
32744 & 106057 & 2005 & 234.80 & 0.09 & 31089.00 & 288498.77 & 0.76 & 1.23 & 0.93 \\
15729 & 102016 & 2005 & 7396.70 & 0.03 & 710561.00 & 7000662.13 & 1.04 & 0.95 & 0.99 \\
43535 & 109142 & 2005 & 94.80 & 0.12 & 9608.00 & 95011.88 & 0.99 & 1.00 & 0.99 \\
62618 & 500412 & 2005 & 22.80 & 0.02 & 2340.00 & 22771.40 & 0.97 & 1.00 & 0.97 \\
54953 & 400030 & 2005 & 54.40 & 0.19 & 6248.00 & 62124.27 & 0.87 & 1.14 & 0.99 \\
15684 & 102013 & 2005 & 1601.80 & 0.01 & 152882.00 & 1520565.10 & 1.05 & 0.95 & 0.99 \\
6555 & 100890 & 2005 & 3312.80 & 0.09 & 353620.00 & 3037274.13 & 0.94 & 0.92 & 0.86 \\
51949 & 300657 & 2005 & 6.30 & 0.01 & 541.00 & 4910.82 & 1.16 & 0.78 & 0.91 \\
9004 & 101108 & 2005 & 1854.80 & 0.11 & 166179.00 & 1600566.14 & 1.12 & 0.86 & 0.96 \\
46825 & 200303 & 2005 & 19.10 & 0.06 & 1890.00 & 16847.58 & 1.01 & 0.88 & 0.89 \\
48379 & 240067 & 2005 & 288.50 & 0.09 & 29153.00 & 266943.06 & 0.99 & 0.93 & 0.92 \\
9418 & 101134 & 2005 & 172.40 & 0.32 & 14518.00 & 136045.24 & 1.19 & 0.79 & 0.94 \\
52354 & 302811 & 2005 & 44.30 & 0.04 & 4660.00 & 44306.45 & 0.95 & 1.00 & 0.95 \\
13227 & 101708 & 2005 & 282.30 & -0.01 & 29127.00 & 292662.36 & 0.97 & 1.04 & 1.00 \\
32706 & 106050 & 2005 & 607.00 & 0.04 & 60774.00 & 599833.93 & 1.00 & 0.99 & 0.99 \\
50994 & 240475 & 2005 & 24.00 & 0.19 & 1957.00 & 19958.66 & 1.23 & 0.83 & 1.02 \\
13753 & 101762 & 2005 & 8408.00 & 0.10 & 824655.00 & 6792207.77 & 1.02 & 0.81 & 0.82 \\
32725 & 106052 & 2005 & 143.50 & 0.09 & 14340.00 & 141145.14 & 1.00 & 0.98 & 0.98 \\
20734 & 102788 & 2005 & 497.40 & 0.09 & 46886.00 & 482267.65 & 1.06 & 0.97 & 1.03 \\
53107 & 339611 & 2005 & 23.20 & 0.06 & 2298.00 & 22755.44 & 1.01 & 0.98 & 0.99 \\
34265 & 106216 & 2005 & 609.30 & 0.17 & 61030.00 & 576252.89 & 1.00 & 0.95 & 0.94 \\
52390 & 302826 & 2005 & 355.10 & 0.01 & 44751.00 & 344561.02 & 0.79 & 0.97 & 0.77 \\
9328 & 101131 & 2005 & 1871.90 & 0.10 & 184162.00 & 1746431.42 & 1.02 & 0.93 & 0.95 \\
52678 & 305766 & 2005 & 84.60 & 0.01 & 8460.00 & 84570.94 & 1.00 & 1.00 & 1.00 \\
29737 & 105643 & 2005 & 799.30 & 0.11 & 80254.00 & 792478.14 & 1.00 & 0.99 & 0.99 \\
46831 & 200304 & 2005 & 54.30 & 0.12 & 4556.00 & 44962.28 & 1.19 & 0.83 & 0.99 \\
10676 & 101307 & 2005 & 1082.90 & 0.13 & 108562.00 & 977217.88 & 1.00 & 0.90 & 0.90 \\
31166 & 105865 & 2005 & 231.20 & 0.08 & 21468.00 & 223727.29 & 1.08 & 0.97 & 1.04 \\
3977 & 100535 & 2005 & 322.80 & 0.06 & 29971.00 & 279097.42 & 1.08 & 0.86 & 0.93 \\
32919 & 106082 & 2005 & 418.00 & 0.07 & 40189.00 & 402329.16 & 1.04 & 0.96 & 1.00 \\
42936 & 109037 & 2005 & 35.40 & 0.03 & 3536.00 & 34487.04 & 1.00 & 0.97 & 0.98 \\
62675 & 500417 & 2005 & 116.90 & 0.05 & 11692.00 & 116800.40 & 1.00 & 1.00 & 1.00 \\
62656 & 500416 & 2005 & 491.90 & 0.03 & 49447.00 & 492041.20 & 0.99 & 1.00 & 1.00 \\
74878 & 601188 & 2005 & 224.10 & 0.13 & 18218.00 & 166660.20 & 1.23 & 0.74 & 0.91 \\
46511 & 200249 & 2005 & 297.70 & 0.02 & 30227.00 & 293799.64 & 0.98 & 0.99 & 0.97 \\
31284 & 105874 & 2005 & 308.90 & 0.05 & 30970.00 & 282862.67 & 1.00 & 0.92 & 0.91 \\
42273 & 108947 & 2005 & 316.60 & 0.02 & 34823.00 & 334411.94 & 0.91 & 1.06 & 0.96 \\
15067 & 101955 & 2005 & 11635.20 & 0.05 & 1161325.00 & 10737491.48 & 1.00 & 0.92 & 0.92 \\
20812 & 102795 & 2005 & 2283.10 & 0.11 & 196651.00 & 1995824.29 & 1.16 & 0.87 & 1.01 \\
20773 & 102789 & 2005 & 928.40 & 0.06 & 92257.00 & 888392.93 & 1.01 & 0.96 & 0.96 \\
35607 & 106379 & 2005 & 195.60 & 0.08 & 18919.00 & 195762.17 & 1.03 & 1.00 & 1.03 \\
15050 & 101953 & 2005 & 370.90 & 0.06 & 37623.00 & 364561.07 & 0.99 & 0.98 & 0.97 \\
32813 & 106066 & 2005 & 1416.60 & 0.08 & 150370.00 & 1523363.60 & 0.94 & 1.08 & 1.01 \\
10052 & 101256 & 2005 & 4.00 & 0.02 & 384.00 & 3889.04 & 1.04 & 0.97 & 1.01 \\
48352 & 240065 & 2005 & 501.20 & 0.11 & 50227.00 & 497414.94 & 1.00 & 0.99 & 0.99 \\
50903 & 240464 & 2005 & 61.10 & -0.01 & 8444.00 & 57174.76 & 0.72 & 0.94 & 0.68 \\
34298 & 106221 & 2005 & 111.70 & 0.07 & 11237.00 & 115972.97 & 0.99 & 1.04 & 1.03 \\
13678 & 101757 & 2005 & 1759.80 & 0.02 & 175514.00 & 1706947.82 & 1.00 & 0.97 & 0.97 \\
57107 & 400286 & 2005 & 42.00 & 0.08 & 4206.00 & 40985.14 & 1.00 & 0.98 & 0.97 \\
10708 & 101312 & 2005 & 8416.60 & 0.10 & 830575.00 & 8129029.67 & 1.01 & 0.97 & 0.98 \\
54847 & 400018 & 2005 & 178.90 & 0.05 & 18467.00 & 179342.01 & 0.97 & 1.00 & 0.97 \\
47648 & 216438 & 2005 & 515.30 & 0.06 & 51449.00 & 517367.72 & 1.00 & 1.00 & 1.01 \\
50906 & 240467 & 2005 & 22.20 & 0.34 & 2059.00 & 18295.52 & 1.08 & 0.82 & 0.89 \\
14836 & 101916 & 2005 & 724.40 & 0.06 & 70656.00 & 679964.05 & 1.03 & 0.94 & 0.96 \\
42870 & 109028 & 2005 & 364.40 & 0.12 & 36689.00 & 343879.63 & 0.99 & 0.94 & 0.94 \\
57144 & 400291 & 2005 & 10.60 & 0.07 & 1056.00 & 10210.21 & 1.00 & 0.96 & 0.97 \\
50853 & 240459 & 2005 & 138.00 & 0.06 & 13307.00 & 134289.93 & 1.04 & 0.97 & 1.01 \\
44204 & 109274 & 2005 & 56.80 & -0.90 & 5156.00 & 51555.39 & 1.10 & 0.91 & 1.00 \\
52379 & 302825 & 2005 & 218.50 & 0.06 & 19011.00 & 176961.88 & 1.15 & 0.81 & 0.93 \\
42886 & 109030 & 2005 & 73.50 & 0.10 & 6943.00 & 70480.00 & 1.06 & 0.96 & 1.02 \\
29610 & 105623 & 2005 & 59.20 & 0.04 & 5983.00 & 53676.99 & 0.99 & 0.91 & 0.90 \\
43604 & 109153 & 2005 & 423.50 & 0.06 & 43326.00 & 440601.16 & 0.98 & 1.04 & 1.02 \\
31233 & 105869 & 2005 & 1791.90 & 0.02 & 179262.00 & 1784087.92 & 1.00 & 1.00 & 1.00 \\
47544 & 212658 & 2005 & 15658.20 & 0.15 & 1565815.00 & 15254620.78 & 1.00 & 0.97 & 0.97 \\
50867 & 240462 & 2005 & 39.80 & 0.10 & 3862.00 & 39614.43 & 1.03 & 1.00 & 1.03 \\
50887 & 240463 & 2005 & 19.60 & 0.15 & 1751.00 & 16270.46 & 1.12 & 0.83 & 0.93 \\
42264 & 108946 & 2005 & 14.60 & -0.01 & 1455.00 & 13262.63 & 1.00 & 0.91 & 0.91 \\
42904 & 109031 & 2005 & 32.50 & 0.08 & 3117.00 & 31184.80 & 1.04 & 0.96 & 1.00 \\
29622 & 105627 & 2005 & 792.40 & 0.06 & 84914.00 & 816688.64 & 0.93 & 1.03 & 0.96 \\
43912 & 109230 & 2005 & 1040.10 & 0.04 & 95424.00 & 954276.80 & 1.09 & 0.92 & 1.00 \\
32841 & 106067 & 2005 & 6259.70 & 0.05 & 618489.00 & 5979823.04 & 1.01 & 0.96 & 0.97 \\
21682 & 102939 & 2005 & 6637.70 & 0.06 & 648907.00 & 6638301.14 & 1.02 & 1.00 & 1.02 \\
52373 & 302819 & 2005 & 3.40 & 0.15 & 338.00 & 3331.55 & 1.01 & 0.98 & 0.99 \\
9041 & 101109 & 2005 & 1705.80 & 0.29 & 165508.00 & 1581029.61 & 1.03 & 0.93 & 0.96 \\
52645 & 305586 & 2005 & 127.10 & 0.01 & 17202.00 & 125151.47 & 0.74 & 0.98 & 0.73 \\
53112 & 339958 & 2005 & 22.10 & 0.01 & 2010.00 & 18005.34 & 1.10 & 0.81 & 0.90 \\
55636 & 400132 & 2005 & 211.60 & 0.06 & 24456.00 & 243949.07 & 0.87 & 1.15 & 1.00 \\
32783 & 106062 & 2005 & 154.70 & 0.13 & 13842.00 & 134551.54 & 1.12 & 0.87 & 0.97 \\
31194 & 105866 & 2005 & 7799.90 & 0.05 & 781590.00 & 8072017.05 & 1.00 & 1.03 & 1.03 \\
3111 & 100409 & 2005 & 1300.00 & 0.08 & 134604.00 & 1321837.38 & 0.97 & 1.02 & 0.98 \\
62728 & 500423 & 2005 & 36.80 & 0.07 & 3636.00 & 31603.96 & 1.01 & 0.86 & 0.87 \\
19739 & 102650 & 2005 & 20392.40 & 0.04 & 2039234.00 & 19185744.61 & 1.00 & 0.94 & 0.94 \\
29673 & 105632 & 2005 & 32.90 & 0.16 & 5109.00 & 51502.36 & 0.64 & 1.57 & 1.01 \\
43570 & 109145 & 2005 & 49.30 & 0.07 & 5577.00 & 49658.58 & 0.88 & 1.01 & 0.89 \\
7435 & 101039 & 2005 & 5287.20 & 0.02 & 532680.00 & 4752465.74 & 0.99 & 0.90 & 0.89 \\
5908 & 100811 & 2005 & 791.90 & 0.03 & 79369.00 & 802290.30 & 1.00 & 1.01 & 1.01 \\
62710 & 500420 & 2005 & 21.30 & 0.14 & 2128.00 & 20631.56 & 1.00 & 0.97 & 0.97 \\
31260 & 105871 & 2005 & 10.20 & 0.07 & 1090.00 & 9009.98 & 0.94 & 0.88 & 0.83 \\
19845 & 102653 & 2005 & 12774.60 & 0.06 & 1277456.00 & 11176231.59 & 1.00 & 0.87 & 0.87 \\
53114 & 339977 & 2005 & 115.90 & 0.02 & 11013.00 & 110360.10 & 1.05 & 0.95 & 1.00 \\
29663 & 105631 & 2005 & 3.30 & -0.01 & 432.00 & 4230.56 & 0.76 & 1.28 & 0.98 \\
43886 & 109228 & 2005 & 34.30 & 0.06 & 3944.00 & 34677.35 & 0.87 & 1.01 & 0.88 \\
34325 & 106222 & 2005 & 163.70 & 0.03 & 16472.00 & 169904.98 & 0.99 & 1.04 & 1.03 \\
29651 & 105630 & 2005 & 4.30 & 0.05 & 578.00 & 5803.67 & 0.74 & 1.35 & 1.00 \\
48598 & 240112 & 2005 & 14.60 & 0.04 & 1456.00 & 14151.24 & 1.00 & 0.97 & 0.97 \\
19770 & 102651 & 2005 & 5757.90 & 0.05 & 556508.00 & 5454018.47 & 1.03 & 0.95 & 0.98 \\
52366 & 302813 & 2005 & 21.80 & 0.01 & 2120.00 & 21583.63 & 1.03 & 0.99 & 1.02 \\
52632 & 305184 & 2005 & 49.60 & 0.08 & 5154.00 & 47253.74 & 0.96 & 0.95 & 0.92 \\
43634 & 109175 & 2005 & 27.50 & 0.03 & 2689.00 & 27291.96 & 1.02 & 0.99 & 1.01 \\
3501 & 100441 & 2005 & 251.20 & 0.01 & 25528.00 & 244110.41 & 0.98 & 0.97 & 0.96 \\
43577 & 109147 & 2005 & 82.60 & 0.11 & 7627.00 & 63408.59 & 1.08 & 0.77 & 0.83 \\
54695 & 378134 & 2005 & 525.40 & 0.10 & 59919.00 & 574524.74 & 0.88 & 1.09 & 0.96 \\
9355 & 101132 & 2005 & 1244.30 & 0.53 & 111286.00 & 1057317.46 & 1.12 & 0.85 & 0.95 \\
34335 & 106223 & 2005 & 97.00 & 0.06 & 9389.00 & 92615.14 & 1.03 & 0.95 & 0.99 \\
54713 & 378591 & 2005 & 7.20 & 0.02 & 496.00 & 4744.62 & 1.45 & 0.66 & 0.96 \\
51773 & 240551 & 2005 & 26.00 & 0.08 & 2506.00 & 23965.11 & 1.04 & 0.92 & 0.96 \\
46520 & 200250 & 2005 & 365.70 & 0.06 & 36028.00 & 364134.89 & 1.02 & 1.00 & 1.01 \\
61792 & 500131 & 2005 & 249.70 & 0.14 & 20912.00 & 210113.21 & 1.19 & 0.84 & 1.00 \\
60922 & 410753 & 2005 & 6.50 & 0.08 & 617.00 & 6219.81 & 1.05 & 0.96 & 1.01 \\
61700 & 500115 & 2005 & 173.70 & 0.10 & 24939.00 & 220840.80 & 0.70 & 1.27 & 0.89 \\
33051 & 106088 & 2005 & 134.60 & 0.02 & 13532.00 & 139736.37 & 0.99 & 1.04 & 1.03 \\
62464 & 500393 & 2005 & 53.00 & 0.03 & 5305.00 & 52828.67 & 1.00 & 1.00 & 1.00 \\
34433 & 106239 & 2005 & 45.90 & 0.15 & 6142.00 & 54620.30 & 0.75 & 1.19 & 0.89 \\
51800 & 240557 & 2005 & 21.70 & 0.00 & 2307.00 & 23068.96 & 0.94 & 1.06 & 1.00 \\
43871 & 109226 & 2005 & 99.00 & -0.03 & 12746.00 & 104711.89 & 0.78 & 1.06 & 0.82 \\
52622 & 303175 & 2005 & 933.90 & 0.21 & 93143.00 & 901777.74 & 1.00 & 0.97 & 0.97 \\
55856 & 400156 & 2005 & 509.40 & 0.06 & 43871.00 & 409120.27 & 1.16 & 0.80 & 0.93 \\
60929 & 410754 & 2005 & 1.20 & 0.12 & 116.00 & 1139.50 & 1.03 & 0.95 & 0.98 \\
44137 & 109268 & 2005 & 2295.20 & 0.06 & 236782.00 & 2190625.01 & 0.97 & 0.95 & 0.93 \\
33078 & 106089 & 2005 & 27.80 & 0.05 & 2279.00 & 20655.14 & 1.22 & 0.74 & 0.91 \\
3923 & 100514 & 2005 & 42.20 & 0.03 & 4461.00 & 44639.49 & 0.95 & 1.06 & 1.00 \\
42830 & 109023 & 2005 & 16.90 & 0.02 & 1750.00 & 16398.49 & 0.97 & 0.97 & 0.94 \\
9466 & 101137 & 2005 & 23.10 & 0.15 & 4012.00 & 37260.13 & 0.58 & 1.61 & 0.93 \\
29870 & 105655 & 2005 & 1366.20 & 0.18 & 128470.00 & 1296968.77 & 1.06 & 0.95 & 1.01 \\
31350 & 105879 & 2005 & 751.40 & 0.09 & 83755.00 & 797227.77 & 0.90 & 1.06 & 0.95 \\
46875 & 200310 & 2005 & 188.50 & 0.04 & 18470.00 & 182565.37 & 1.02 & 0.97 & 0.99 \\
29853 & 105654 & 2005 & 66.40 & 0.17 & 6007.00 & 63964.35 & 1.11 & 0.96 & 1.06 \\
56451 & 400212 & 2005 & 98.10 & 0.10 & 10001.00 & 100073.79 & 0.98 & 1.02 & 1.00 \\
44159 & 109269 & 2005 & 375.70 & 0.22 & 21266.00 & 174277.19 & 1.77 & 0.46 & 0.82 \\
21567 & 102894 & 2005 & 1018.10 & 0.01 & 101821.00 & 961205.26 & 1.00 & 0.94 & 0.94 \\
61714 & 500116 & 2005 & 1258.50 & 0.08 & 123203.00 & 1172732.69 & 1.02 & 0.93 & 0.95 \\
56949 & 400268 & 2005 & 289.50 & 0.14 & 28232.00 & 274562.06 & 1.03 & 0.95 & 0.97 \\
33024 & 106086 & 2005 & 36.70 & 0.08 & 3678.00 & 35153.97 & 1.00 & 0.96 & 0.96 \\
16156 & 102087 & 2005 & 543.30 & 0.14 & 60062.00 & 446321.90 & 0.90 & 0.82 & 0.74 \\
20862 & 102797 & 2005 & 69.50 & 0.04 & 7029.00 & 70259.52 & 0.99 & 1.01 & 1.00 \\
51072 & 240481 & 2005 & 87.70 & 0.16 & 8712.00 & 70594.27 & 1.01 & 0.80 & 0.81 \\
42836 & 109024 & 2005 & 3.70 & 0.03 & 389.00 & 3037.33 & 0.95 & 0.82 & 0.78 \\
43448 & 109122 & 2005 & 2.90 & 0.10 & 291.00 & 2905.92 & 1.00 & 1.00 & 1.00 \\
14746 & 101912 & 2005 & 8779.00 & 0.04 & 873259.00 & 7007920.08 & 1.01 & 0.80 & 0.80 \\
43454 & 109124 & 2005 & 142.30 & 0.09 & 11516.00 & 115213.30 & 1.24 & 0.81 & 1.00 \\
51096 & 240482 & 2005 & 3.60 & 0.08 & 369.00 & 3381.86 & 0.98 & 0.94 & 0.92 \\
55877 & 400157 & 2005 & 470.80 & 0.05 & 41947.00 & 389636.72 & 1.12 & 0.83 & 0.93 \\
33087 & 106090 & 2005 & 1156.80 & 0.08 & 112993.00 & 1112773.67 & 1.02 & 0.96 & 0.98 \\
9285 & 101127 & 2005 & 114.90 & 0.12 & 10980.00 & 109422.88 & 1.05 & 0.95 & 1.00 \\
51789 & 240554 & 2005 & 232.70 & 0.05 & 23577.00 & 235763.20 & 0.99 & 1.01 & 1.00 \\
51785 & 240553 & 2005 & 26.10 & 0.04 & 2611.00 & 25935.65 & 1.00 & 0.99 & 0.99 \\
21537 & 102893 & 2005 & 231.10 & 0.21 & 23614.00 & 233285.79 & 0.98 & 1.01 & 0.99 \\
51795 & 240556 & 2005 & 22.20 & 0.05 & 2260.00 & 22605.10 & 0.98 & 1.02 & 1.00 \\
46808 & 200300 & 2005 & 6.90 & 0.02 & 689.00 & 6759.97 & 1.00 & 0.98 & 0.98 \\
42960 & 109042 & 2005 & 94.10 & 0.08 & 6838.00 & 63459.17 & 1.38 & 0.67 & 0.93 \\
33101 & 106091 & 2005 & 574.00 & 0.11 & 54887.00 & 540282.51 & 1.05 & 0.94 & 0.98 \\
19927 & 102655 & 2005 & 3763.50 & 0.30 & 376353.00 & 3220649.93 & 1.00 & 0.86 & 0.86 \\
55898 & 400158 & 2005 & 100.20 & 0.05 & 9046.00 & 85957.03 & 1.11 & 0.86 & 0.95 \\
6510 & 100878 & 2005 & 2274.90 & 0.01 & 235686.00 & 2261304.30 & 0.97 & 0.99 & 0.96 \\
10611 & 101300 & 2005 & 2506.30 & 0.10 & 250074.00 & 2166806.47 & 1.00 & 0.86 & 0.87 \\
35354 & 106353 & 2005 & 1047.70 & 0.09 & 97473.00 & 1014568.68 & 1.07 & 0.97 & 1.04 \\
15763 & 102017 & 2005 & 5287.60 & 0.07 & 527413.00 & 5135837.83 & 1.00 & 0.97 & 0.97 \\
51113 & 240485 & 2005 & 426.20 & 0.04 & 42649.00 & 426114.69 & 1.00 & 1.00 & 1.00 \\
235 & 100019 & 2005 & 16379.70 & 0.12 & 1635572.00 & 13272970.39 & 1.00 & 0.81 & 0.81 \\
46884 & 200311 & 2005 & 652.30 & 0.02 & 64272.00 & 634057.70 & 1.01 & 0.97 & 0.99 \\
34449 & 106240 & 2005 & 167.70 & -0.01 & 18163.00 & 175527.35 & 0.92 & 1.05 & 0.97 \\
32594 & 106042 & 2005 & 182.10 & 0.06 & 18514.00 & 171884.83 & 0.98 & 0.94 & 0.93 \\
47635 & 215952 & 2005 & 851.50 & 0.16 & 79350.00 & 673946.73 & 1.07 & 0.79 & 0.85 \\
31088 & 105857 & 2005 & 882.00 & 0.09 & 88351.00 & 782860.55 & 1.00 & 0.89 & 0.89 \\
42375 & 108955 & 2005 & 47.90 & 0.02 & 4828.00 & 48122.58 & 0.99 & 1.00 & 1.00 \\
13659 & 101754 & 2005 & 203.40 & 0.03 & 21219.00 & 196203.88 & 0.96 & 0.96 & 0.92 \\
29899 & 105656 & 2005 & 1172.20 & 0.07 & 111755.00 & 1117540.14 & 1.05 & 0.95 & 1.00 \\
19705 & 102649 & 2005 & 773.40 & 0.03 & 76965.00 & 741153.31 & 1.00 & 0.96 & 0.96 \\
43943 & 109237 & 2005 & 125.60 & 0.02 & 15240.00 & 152404.39 & 0.82 & 1.21 & 1.00 \\
31368 & 105880 & 2005 & 1344.00 & 0.08 & 153144.00 & 1474280.74 & 0.88 & 1.10 & 0.96 \\
51103 & 240483 & 2005 & 2.60 & 0.04 & 257.00 & 2122.42 & 1.01 & 0.82 & 0.83 \\
34208 & 106212 & 2005 & 86.00 & 0.07 & 8550.00 & 88244.36 & 1.01 & 1.03 & 1.03 \\
15151 & 101963 & 2005 & 720.30 & 0.05 & 72002.00 & 698170.60 & 1.00 & 0.97 & 0.97 \\
35388 & 106356 & 2005 & 17.80 & 0.06 & 1783.00 & 17830.56 & 1.00 & 1.00 & 1.00 \\
42363 & 108953 & 2005 & 144.50 & 0.11 & 14473.00 & 144442.67 & 1.00 & 1.00 & 1.00 \\
56968 & 400269 & 2005 & 687.50 & 0.15 & 72396.00 & 704558.88 & 0.95 & 1.02 & 0.97 \\
55814 & 400153 & 2005 & 118.60 & 0.04 & 11858.00 & 110033.35 & 1.00 & 0.93 & 0.93 \\
42946 & 109038 & 2005 & 37.20 & 0.05 & 3715.00 & 34045.36 & 1.00 & 0.92 & 0.92 \\
10010 & 101252 & 2005 & 228.50 & 0.07 & 20635.00 & 201597.15 & 1.11 & 0.88 & 0.98 \\
32946 & 106083 & 2005 & 811.30 & 0.16 & 78639.00 & 804977.05 & 1.03 & 0.99 & 1.02 \\
52345 & 302780 & 2005 & 140.10 & 0.02 & 14223.00 & 138451.45 & 0.99 & 0.99 & 0.97 \\
48297 & 240060 & 2005 & 187.30 & 0.02 & 19524.00 & 151065.77 & 0.96 & 0.81 & 0.77 \\
51036 & 240477 & 2005 & 5.10 & 0.05 & 513.00 & 4570.59 & 0.99 & 0.90 & 0.89 \\
46526 & 200251 & 2005 & 96.50 & 0.16 & 8570.00 & 86331.27 & 1.13 & 0.89 & 1.01 \\
62484 & 500394 & 2005 & 89.80 & 0.04 & 8965.00 & 89597.03 & 1.00 & 1.00 & 1.00 \\
29798 & 105647 & 2005 & 574.50 & 0.11 & 57536.00 & 475787.73 & 1.00 & 0.83 & 0.83 \\
32973 & 106084 & 2005 & 880.60 & 0.10 & 84855.00 & 876669.34 & 1.04 & 1.00 & 1.03 \\
43483 & 109127 & 2005 & 27.70 & 0.12 & 2776.00 & 26112.64 & 1.00 & 0.94 & 0.94 \\
44187 & 109271 & 2005 & 65.80 & -0.04 & 7438.00 & 70223.76 & 0.88 & 1.07 & 0.94 \\
34373 & 106230 & 2005 & 227.30 & 0.09 & 23997.00 & 240270.11 & 0.95 & 1.06 & 1.00 \\
20715 & 102784 & 2005 & 10332.00 & 0.05 & 1031315.00 & 10238809.08 & 1.00 & 0.99 & 0.99 \\
51913 & 300102 & 2005 & 22.50 & 0.04 & 1389.00 & 14509.81 & 1.62 & 0.64 & 1.04 \\
47385 & 210681 & 2005 & 60093.80 & 0.16 & 5634569.00 & 47473843.71 & 1.07 & 0.79 & 0.84 \\
15101 & 101956 & 2005 & 1250.10 & 0.08 & 124997.00 & 1248280.15 & 1.00 & 1.00 & 1.00 \\
48690 & 240121 & 2005 & 294.80 & 0.02 & 31956.00 & 311370.65 & 0.92 & 1.06 & 0.97 \\
6706 & 100913 & 2005 & 577.40 & 0.03 & 59120.00 & 590567.99 & 0.98 & 1.02 & 1.00 \\
10082 & 101258 & 2005 & 4437.10 & 0.04 & 482797.00 & 4075006.66 & 0.92 & 0.92 & 0.84 \\
29769 & 105645 & 2005 & 8408.00 & 0.08 & 841384.00 & 8230565.01 & 1.00 & 0.98 & 0.98 \\
35504 & 106367 & 2005 & 85.00 & 0.04 & 10136.00 & 102886.46 & 0.84 & 1.21 & 1.02 \\
43528 & 109133 & 2005 & 41.90 & 0.04 & 4072.00 & 40900.03 & 1.03 & 0.98 & 1.00 \\
96680 & 611003 & 2005 & 406.90 & 0.06 & 40170.00 & 386179.30 & 1.01 & 0.95 & 0.96 \\
62504 & 500395 & 2005 & 102.00 & 0.09 & 10199.00 & 101207.47 & 1.00 & 0.99 & 0.99 \\
3467 & 100439 & 2005 & 16.60 & -0.01 & 1952.00 & 19529.42 & 0.85 & 1.18 & 1.00 \\
32694 & 106049 & 2005 & 91.30 & 0.05 & 9619.00 & 86252.55 & 0.95 & 0.94 & 0.90 \\
19112 & 102549 & 2005 & 658.50 & 0.07 & 77257.00 & 636966.44 & 0.85 & 0.97 & 0.82 \\
31143 & 105861 & 2005 & 328.50 & 0.01 & 35975.00 & 345196.76 & 0.91 & 1.05 & 0.96 \\
43506 & 109129 & 2005 & 110.70 & 0.06 & 11061.00 & 106834.28 & 1.00 & 0.97 & 0.97 \\
51015 & 240476 & 2005 & 1402.80 & 0.14 & 141030.00 & 1473579.32 & 0.99 & 1.05 & 1.04 \\
56422 & 400207 & 2005 & 9.80 & -0.08 & 1043.00 & 9063.42 & 0.94 & 0.92 & 0.87 \\
6598 & 100900 & 2005 & 97.50 & 0.03 & 9052.00 & 76687.09 & 1.08 & 0.79 & 0.85 \\
51056 & 240478 & 2005 & 2.70 & 0.06 & 274.00 & 2699.12 & 0.99 & 1.00 & 0.99 \\
32667 & 106047 & 2005 & 8.80 & 0.05 & 874.00 & 9197.88 & 1.01 & 1.05 & 1.05 \\
59327 & 410465 & 2005 & 22.60 & 0.11 & 2149.00 & 22239.57 & 1.05 & 0.98 & 1.03 \\
31115 & 105860 & 2005 & 826.30 & 0.03 & 82037.00 & 820442.58 & 1.01 & 0.99 & 1.00 \\
29827 & 105652 & 2005 & 377.90 & 0.10 & 33935.00 & 331718.10 & 1.11 & 0.88 & 0.98 \\
35456 & 106361 & 2005 & 331.30 & 0.10 & 33319.00 & 315126.66 & 0.99 & 0.95 & 0.95 \\
56413 & 400205 & 2005 & 39.40 & 0.05 & 3852.00 & 37987.43 & 1.02 & 0.96 & 0.99 \\
43677 & 109189 & 2005 & 893.20 & 0.10 & 92437.00 & 942256.95 & 0.97 & 1.05 & 1.02 \\
42338 & 108952 & 2005 & 955.70 & 0.22 & 96942.00 & 905857.23 & 0.99 & 0.95 & 0.93 \\
34226 & 106213 & 2005 & 99.90 & 0.08 & 11320.00 & 98989.25 & 0.88 & 0.99 & 0.87 \\
35429 & 106360 & 2005 & 1218.70 & 0.16 & 115725.00 & 1117189.64 & 1.05 & 0.92 & 0.97 \\
57008 & 400277 & 2005 & 1220.10 & 0.10 & 107337.00 & 1073868.96 & 1.14 & 0.88 & 1.00 \\
13419 & 101738 & 2005 & 1792.00 & 0.11 & 201946.00 & 1724445.88 & 0.89 & 0.96 & 0.85 \\
71 & 100004 & 2005 & 1967.50 & 0.08 & 200212.00 & 1998871.16 & 0.98 & 1.02 & 1.00 \\
33242 & 106107 & 2005 & 58.20 & 0.03 & 5595.00 & 53526.30 & 1.04 & 0.92 & 0.96 \\
19883 & 102654 & 2005 & 864.50 & 0.04 & 86476.00 & 854352.30 & 1.00 & 0.99 & 0.99 \\
32648 & 106045 & 2005 & 32.80 & 0.05 & 3723.00 & 33727.01 & 0.88 & 1.03 & 0.91 \\
32997 & 106085 & 2005 & 185.40 & 0.09 & 18918.00 & 189112.40 & 0.98 & 1.02 & 1.00 \\
48681 & 240118 & 2005 & 137.30 & 0.15 & 12639.00 & 119230.91 & 1.09 & 0.87 & 0.94 \\
44182 & 109270 & 2005 & 66.90 & 0.11 & 7065.00 & 71558.41 & 0.95 & 1.07 & 1.01 \\
16175 & 102089 & 2005 & 621.50 & 0.12 & 58100.00 & 495559.93 & 1.07 & 0.80 & 0.85 \\
31322 & 105878 & 2005 & 1868.10 & 0.03 & 187870.00 & 1858784.76 & 0.99 & 1.00 & 0.99 \\
19440 & 102601 & 2005 & 7255.50 & 0.06 & 724623.00 & 7073560.30 & 1.00 & 0.97 & 0.98 \\
51061 & 240480 & 2005 & 31.10 & 0.09 & 2995.00 & 28372.81 & 1.04 & 0.91 & 0.95 \\
56418 & 400206 & 2005 & 45.80 & 0.09 & 6895.00 & 69391.24 & 0.66 & 1.52 & 1.01 \\
34400 & 106231 & 2005 & 323.50 & -0.04 & 33467.00 & 336319.44 & 0.97 & 1.04 & 1.00 \\
10646 & 101302 & 2005 & 2361.70 & 0.02 & 238532.00 & 2344598.25 & 0.99 & 0.99 & 0.98 \\
60894 & 410739 & 2005 & 45.80 & 0.05 & 4989.00 & 50387.76 & 0.92 & 1.10 & 1.01 \\
34238 & 106214 & 2005 & 91.00 & 0.02 & 9128.00 & 87678.66 & 1.00 & 0.96 & 0.96 \\
3146 & 100411 & 2005 & 4100.00 & 0.14 & 399709.00 & 3888978.34 & 1.03 & 0.95 & 0.97 \\
15119 & 101958 & 2005 & 544.80 & 0.04 & 54484.00 & 551206.91 & 1.00 & 1.01 & 1.01 \\
3530 & 100453 & 2005 & 113.50 & 0.05 & 12326.00 & 114012.27 & 0.92 & 1.00 & 0.92 \\
56448 & 400211 & 2005 & 365.30 & 0.07 & 36597.00 & 357172.01 & 1.00 & 0.98 & 0.98 \\
56253 & 400185 & 2005 & 146.90 & 0.15 & 14758.00 & 146870.75 & 1.00 & 1.00 & 1.00 \\
51515 & 240527 & 2005 & 703.30 & 0.05 & 63325.00 & 630227.59 & 1.11 & 0.90 & 1.00 \\
61377 & 500047 & 2005 & 3.30 & 0.11 & 315.00 & 3091.95 & 1.05 & 0.94 & 0.98 \\
10358 & 101283 & 2005 & 4767.10 & 0.11 & 416730.00 & 3986314.02 & 1.14 & 0.84 & 0.96 \\
34976 & 106298 & 2005 & 62.00 & 0.06 & 5857.00 & 56673.15 & 1.06 & 0.91 & 0.97 \\
52524 & 302997 & 2005 & 521.60 & 0.11 & 52216.00 & 466127.56 & 1.00 & 0.89 & 0.89 \\
52773 & 320640 & 2005 & 4.00 & 0.09 & 326.00 & 3404.48 & 1.23 & 0.85 & 1.04 \\
7499 & 101042 & 2005 & 16508.80 & 0.03 & 1618058.00 & 16502897.99 & 1.02 & 1.00 & 1.02 \\
59305 & 410463 & 2005 & 482.60 & 0.07 & 45037.00 & 383702.27 & 1.07 & 0.80 & 0.85 \\
46712 & 200291 & 2005 & 7.50 & 0.14 & 750.00 & 7588.29 & 1.00 & 1.01 & 1.01 \\
7397 & 101038 & 2005 & 3172.70 & 0.12 & 303454.00 & 3113207.34 & 1.05 & 0.98 & 1.03 \\
43088 & 109062 & 2005 & 45.10 & 0.03 & 4506.00 & 43782.63 & 1.00 & 0.97 & 0.97 \\
33863 & 106176 & 2005 & 25.40 & 0.04 & 2583.00 & 23750.43 & 0.98 & 0.94 & 0.92 \\
4054 & 100543 & 2005 & 460.80 & 0.25 & 48028.00 & 481240.51 & 0.96 & 1.04 & 1.00 \\
30789 & 105798 & 2005 & 1067.50 & 0.10 & 146774.00 & 1383159.50 & 0.73 & 1.30 & 0.94 \\
48529 & 240103 & 2005 & 468.20 & 0.00 & 52077.00 & 478465.33 & 0.90 & 1.02 & 0.92 \\
51486 & 240524 & 2005 & 12.00 & 0.01 & 1116.00 & 10424.38 & 1.08 & 0.87 & 0.93 \\
34998 & 106305 & 2005 & 604.70 & 0.15 & 61158.00 & 571136.80 & 0.99 & 0.94 & 0.93 \\
3729 & 100475 & 2005 & 1235.10 & 0.04 & 123366.00 & 1215439.67 & 1.00 & 0.98 & 0.99 \\
31684 & 105930 & 2005 & 220.20 & -0.02 & 21865.00 & 218636.39 & 1.01 & 0.99 & 1.00 \\
46625 & 200266 & 2005 & 9.00 & 0.09 & 603.00 & 5217.16 & 1.49 & 0.58 & 0.87 \\
56122 & 400175 & 2005 & 559.90 & 0.12 & 55834.00 & 553987.99 & 1.00 & 0.99 & 0.99 \\
51490 & 240525 & 2005 & 527.30 & 0.05 & 48720.00 & 497033.81 & 1.08 & 0.94 & 1.02 \\
33617 & 106156 & 2005 & 8.80 & 0.08 & 899.00 & 8760.24 & 0.98 & 1.00 & 0.97 \\
52752 & 320584 & 2005 & 10.90 & 0.06 & 1241.00 & 11012.02 & 0.88 & 1.01 & 0.89 \\
51510 & 240526 & 2005 & 72.90 & 0.04 & 7366.00 & 72425.06 & 0.99 & 0.99 & 0.98 \\
62080 & 500327 & 2005 & 1.60 & 0.11 & 123.00 & 1147.32 & 1.30 & 0.72 & 0.93 \\
43790 & 109221 & 2005 & 493.30 & 0.07 & 38265.00 & 346941.93 & 1.29 & 0.70 & 0.91 \\
42713 & 109008 & 2005 & 9.00 & -0.01 & 750.00 & 6696.74 & 1.20 & 0.74 & 0.89 \\
62060 & 500326 & 2005 & 2.70 & 0.11 & 266.00 & 2662.14 & 1.02 & 0.99 & 1.00 \\
56143 & 400176 & 2005 & 1065.90 & 0.05 & 103328.00 & 1055069.25 & 1.03 & 0.99 & 1.02 \\
30474 & 105761 & 2005 & 1309.20 & 0.10 & 132921.00 & 1242842.30 & 0.98 & 0.95 & 0.94 \\
32049 & 105977 & 2005 & 1299.40 & 0.09 & 121562.00 & 1169541.94 & 1.07 & 0.90 & 0.96 \\
19581 & 102633 & 2005 & 110.20 & 0.05 & 11513.00 & 112290.55 & 0.96 & 1.02 & 0.98 \\
9154 & 101115 & 2005 & 20238.10 & 0.05 & 2321539.00 & 22676596.15 & 0.87 & 1.12 & 0.98 \\
31703 & 105931 & 2005 & 7774.40 & 0.06 & 782690.00 & 7313604.68 & 0.99 & 0.94 & 0.93 \\
6336 & 100849 & 2005 & 43.60 & -0.13 & 4260.00 & 45293.98 & 1.02 & 1.04 & 1.06 \\
52857 & 330794 & 2005 & 91.50 & 0.05 & 9160.00 & 89826.07 & 1.00 & 0.98 & 0.98 \\
34665 & 106268 & 2005 & 242.70 & 0.04 & 25269.00 & 248820.04 & 0.96 & 1.03 & 0.98 \\
43230 & 109085 & 2005 & 7.10 & 0.06 & 709.00 & 7093.16 & 1.00 & 1.00 & 1.00 \\
31730 & 105932 & 2005 & 250.70 & 0.06 & 24845.00 & 248418.33 & 1.01 & 0.99 & 1.00 \\
34938 & 106294 & 2005 & 144.90 & 0.13 & 25345.00 & 250651.35 & 0.57 & 1.73 & 0.99 \\
20307 & 102715 & 2005 & 1855.20 & 0.17 & 187495.00 & 1761934.39 & 0.99 & 0.95 & 0.94 \\
15949 & 102061 & 2005 & 1902.70 & 0.03 & 210539.00 & 2077083.73 & 0.90 & 1.09 & 0.99 \\
42602 & 108985 & 2005 & 4228.30 & 0.07 & 533107.00 & 4755386.03 & 0.79 & 1.12 & 0.89 \\
33809 & 106172 & 2005 & 67.30 & 0.08 & 6731.00 & 65966.38 & 1.00 & 0.98 & 0.98 \\
61495 & 500083 & 2005 & 17.60 & 0.11 & 1804.00 & 18068.84 & 0.98 & 1.03 & 1.00 \\
33649 & 106158 & 2005 & 631.50 & 0.01 & 69429.00 & 651496.79 & 0.91 & 1.03 & 0.94 \\
43207 & 109084 & 2005 & 628.20 & -0.03 & 63181.00 & 562770.65 & 0.99 & 0.90 & 0.89 \\
44032 & 109259 & 2005 & 259.70 & 0.08 & 27523.00 & 224828.44 & 0.94 & 0.87 & 0.82 \\
30767 & 105794 & 2005 & 61.50 & 0.08 & 5729.00 & 49887.53 & 1.07 & 0.81 & 0.87 \\
52747 & 320323 & 2005 & 5.30 & 0.09 & 489.00 & 4732.74 & 1.08 & 0.89 & 0.97 \\
52094 & 301560 & 2005 & 1191.80 & 0.03 & 135646.00 & 1237759.37 & 0.88 & 1.04 & 0.91 \\
9731 & 101179 & 2005 & 352.40 & -0.03 & 37364.00 & 366646.17 & 0.94 & 1.04 & 0.98 \\
61403 & 500048 & 2005 & 51.20 & 0.04 & 5423.00 & 52163.87 & 0.94 & 1.02 & 0.96 \\
20476 & 102757 & 2005 & 5871.00 & 0.10 & 587996.00 & 5858749.70 & 1.00 & 1.00 & 1.00 \\
33629 & 106157 & 2005 & 1023.50 & 0.12 & 101967.00 & 995040.04 & 1.00 & 0.97 & 0.98 \\
32007 & 105973 & 2005 & 113.30 & 0.10 & 11406.00 & 105166.51 & 0.99 & 0.93 & 0.92 \\
30502 & 105762 & 2005 & 431.90 & 0.03 & 45064.00 & 405681.70 & 0.96 & 0.94 & 0.90 \\
43783 & 109219 & 2005 & 62.70 & 0.04 & 5331.00 & 55538.36 & 1.18 & 0.89 & 1.04 \\
3359 & 100425 & 2005 & 2585.30 & 0.05 & 327172.00 & 3298081.80 & 0.79 & 1.28 & 1.01 \\
36 & 100003 & 2005 & 1055.90 & 0.16 & 104889.00 & 1008574.86 & 1.01 & 0.96 & 0.96 \\
6263 & 100833 & 2005 & 391.60 & 0.03 & 41797.00 & 418299.13 & 0.94 & 1.07 & 1.00 \\
33836 & 106173 & 2005 & 1470.80 & 0.07 & 150403.00 & 1471293.13 & 0.98 & 1.00 & 0.98 \\
15402 & 101989 & 2005 & 177.60 & 0.04 & 17771.00 & 172347.63 & 1.00 & 0.97 & 0.97 \\
55002 & 400041 & 2005 & 265.50 & 0.08 & 26539.00 & 262773.64 & 1.00 & 0.99 & 0.99 \\
56336 & 400191 & 2005 & 38.10 & 0.03 & 4570.00 & 32865.57 & 0.83 & 0.86 & 0.72 \\
54892 & 400020 & 2005 & 131.60 & 0.07 & 13102.00 & 118327.61 & 1.00 & 0.90 & 0.90 \\
34965 & 106297 & 2005 & 89.30 & 0.14 & 11705.00 & 111337.99 & 0.76 & 1.25 & 0.95 \\
61983 & 500315 & 2005 & 52.90 & 0.10 & 5539.00 & 50733.24 & 0.96 & 0.96 & 0.92 \\
19307 & 102588 & 2005 & 289.70 & 0.07 & 29639.00 & 292265.80 & 0.98 & 1.01 & 0.99 \\
52484 & 302964 & 2005 & 1048.00 & 0.12 & 112086.00 & 1107310.67 & 0.93 & 1.06 & 0.99 \\
61561 & 500094 & 2005 & 200.50 & 0.10 & 31066.00 & 198421.66 & 0.65 & 0.99 & 0.64 \\
35038 & 106309 & 2005 & 1489.90 & 0.06 & 162772.00 & 1427598.55 & 0.92 & 0.96 & 0.88 \\
47008 & 200327 & 2005 & 19.30 & 0.12 & 1954.00 & 18156.70 & 0.99 & 0.94 & 0.93 \\
3759 & 100480 & 2005 & 108.10 & 0.05 & 10728.00 & 105857.52 & 1.01 & 0.98 & 0.99 \\
54781 & 400012 & 2005 & 28.10 & 0.04 & 2382.00 & 21881.20 & 1.18 & 0.78 & 0.92 \\
62129 & 500340 & 2005 & 51.80 & 0.07 & 5174.00 & 49100.73 & 1.00 & 0.95 & 0.95 \\
3627 & 100463 & 2005 & 18138.10 & 0.08 & 1821714.00 & 18120082.21 & 1.00 & 1.00 & 0.99 \\
10235 & 101275 & 2005 & 841.20 & 0.06 & 101817.00 & 918817.41 & 0.83 & 1.09 & 0.90 \\
47215 & 200343 & 2005 & 12264.90 & 0.08 & 1146309.00 & 9558197.16 & 1.07 & 0.78 & 0.83 \\
33887 & 106180 & 2005 & 135.00 & 0.18 & 10983.00 & 109510.30 & 1.23 & 0.81 & 1.00 \\
51692 & 240542 & 2005 & 318.00 & 0.05 & 31970.00 & 275329.24 & 0.99 & 0.87 & 0.86 \\
51444 & 240520 & 2005 & 103.10 & 0.15 & 14642.00 & 149435.37 & 0.70 & 1.45 & 1.02 \\
47192 & 200342 & 2005 & 14327.80 & 0.14 & 1310742.00 & 13027147.43 & 1.09 & 0.91 & 0.99 \\
42566 & 108979 & 2005 & 83.50 & 0.04 & 9087.00 & 86229.40 & 0.92 & 1.03 & 0.95 \\
30400 & 105753 & 2005 & 529.10 & 0.02 & 60977.00 & 561953.05 & 0.87 & 1.06 & 0.92 \\
32109 & 105983 & 2005 & 643.00 & 0.09 & 54925.00 & 562655.64 & 1.17 & 0.88 & 1.02 \\
13708 & 101758 & 2005 & 158.60 & 0.04 & 15606.00 & 154134.28 & 1.02 & 0.97 & 0.99 \\
31650 & 105920 & 2005 & 3296.40 & 0.14 & 328883.00 & 3215444.31 & 1.00 & 0.98 & 0.98 \\
19609 & 102636 & 2005 & 1611.40 & 0.12 & 161197.00 & 1586139.31 & 1.00 & 0.98 & 0.98 \\
96695 & 611006 & 2005 & 9.60 & 0.13 & 957.00 & 9501.37 & 1.00 & 0.99 & 0.99 \\
52880 & 333058 & 2005 & 1313.20 & 0.07 & 101322.00 & 1025438.93 & 1.30 & 0.78 & 1.01 \\
15987 & 102062 & 2005 & 1635.60 & 0.00 & 160901.00 & 1642155.10 & 1.02 & 1.00 & 1.02 \\
21282 & 102844 & 2005 & 568.30 & 0.12 & 55212.00 & 561530.99 & 1.03 & 0.99 & 1.02 \\
9667 & 101161 & 2005 & 878.90 & 0.07 & 87823.00 & 863383.23 & 1.00 & 0.98 & 0.98 \\
52041 & 301299 & 2005 & 3667.20 & 0.14 & 367879.00 & 3617566.02 & 1.00 & 0.99 & 0.98 \\
51452 & 240521 & 2005 & 11.00 & 0.09 & 1090.00 & 10574.66 & 1.01 & 0.96 & 0.97 \\
30832 & 105804 & 2005 & 884.80 & 0.11 & 87858.00 & 828899.17 & 1.01 & 0.94 & 0.94 \\
51424 & 240519 & 2005 & 19.60 & 0.08 & 1963.00 & 19683.98 & 1.00 & 1.00 & 1.00 \\
19514 & 102608 & 2005 & 186.90 & 0.14 & 18235.00 & 182351.57 & 1.02 & 0.98 & 1.00 \\
33567 & 106149 & 2005 & 147.60 & 0.05 & 14781.00 & 142087.68 & 1.00 & 0.96 & 0.96 \\
52183 & 302627 & 2005 & 119.50 & 0.07 & 15324.00 & 119348.84 & 0.78 & 1.00 & 0.78 \\
61293 & 500027 & 2005 & 218.70 & 0.10 & 21131.00 & 206864.45 & 1.03 & 0.95 & 0.98 \\
48452 & 240085 & 2005 & 182.30 & 0.05 & 18261.00 & 177300.97 & 1.00 & 0.97 & 0.97 \\
10388 & 101284 & 2005 & 2531.20 & 0.06 & 270176.00 & 2605590.25 & 0.94 & 1.03 & 0.96 \\
56619 & 400232 & 2005 & 38.50 & 0.16 & 3963.00 & 38079.97 & 0.97 & 0.99 & 0.96 \\
52864 & 332404 & 2005 & 59.30 & 0.09 & 4578.00 & 43852.87 & 1.30 & 0.74 & 0.96 \\
51468 & 240522 & 2005 & 1.80 & 0.11 & 159.00 & 1505.21 & 1.13 & 0.84 & 0.95 \\
44050 & 109263 & 2005 & 11.60 & 0.02 & 1111.00 & 11391.67 & 1.04 & 0.98 & 1.03 \\
48477 & 240087 & 2005 & 219.30 & 0.03 & 21845.00 & 219686.96 & 1.00 & 1.00 & 1.01 \\
9686 & 101165 & 2005 & 2190.80 & 0.04 & 219321.00 & 2160280.70 & 1.00 & 0.99 & 0.98 \\
21258 & 102843 & 2005 & 2130.60 & 0.18 & 173285.00 & 1732437.71 & 1.23 & 0.81 & 1.00 \\
13558 & 101743 & 2005 & 9754.50 & 0.09 & 978930.00 & 8996554.23 & 1.00 & 0.92 & 0.92 \\
30436 & 105758 & 2005 & 115.40 & 0.06 & 14228.00 & 115326.77 & 0.81 & 1.00 & 0.81 \\
47028 & 200329 & 2005 & 1237.00 & 0.16 & 105406.00 & 1047126.41 & 1.17 & 0.85 & 0.99 \\
30446 & 105760 & 2005 & 990.30 & 0.05 & 97869.00 & 930309.48 & 1.01 & 0.94 & 0.95 \\
32065 & 105978 & 2005 & 40.50 & 0.03 & 4050.00 & 40501.20 & 1.00 & 1.00 & 1.00 \\
34636 & 106262 & 2005 & 213.90 & -0.00 & 21525.00 & 214973.65 & 0.99 & 1.01 & 1.00 \\
35025 & 106306 & 2005 & 17.00 & 0.04 & 1651.00 & 16507.77 & 1.03 & 0.97 & 1.00 \\
15534 & 102000 & 2005 & 359.80 & 0.13 & 36019.00 & 360117.24 & 1.00 & 1.00 & 1.00 \\
30804 & 105803 & 2005 & 7890.20 & 0.08 & 786161.00 & 7548966.83 & 1.00 & 0.96 & 0.96 \\
15372 & 101988 & 2005 & 330.70 & 0.06 & 32922.00 & 328372.07 & 1.00 & 0.99 & 1.00 \\
48541 & 240105 & 2005 & 172.00 & 0.05 & 17268.00 & 165671.36 & 1.00 & 0.96 & 0.96 \\
61335 & 500037 & 2005 & 1857.10 & 0.11 & 186325.00 & 1854343.24 & 1.00 & 1.00 & 1.00 \\
9865 & 101198 & 2005 & 242.10 & 0.04 & 24571.00 & 230211.79 & 0.99 & 0.95 & 0.94 \\
32077 & 105980 & 2005 & 503.80 & 0.08 & 50345.00 & 475881.78 & 1.00 & 0.94 & 0.95 \\
20505 & 102760 & 2005 & 976.80 & 0.06 & 96458.00 & 962512.16 & 1.01 & 0.99 & 1.00 \\
33590 & 106151 & 2005 & 1171.60 & 0.07 & 116923.00 & 1168967.34 & 1.00 & 1.00 & 1.00 \\
19227 & 102570 & 2005 & 162.10 & 0.09 & 15571.00 & 158353.80 & 1.04 & 0.98 & 1.02 \\
56600 & 400231 & 2005 & 161.70 & 0.04 & 15682.00 & 142786.25 & 1.03 & 0.88 & 0.91 \\
52064 & 301438 & 2005 & 852.80 & 0.04 & 88146.00 & 790141.21 & 0.97 & 0.93 & 0.90 \\
34625 & 106261 & 2005 & 629.70 & -0.03 & 63221.00 & 618743.54 & 1.00 & 0.98 & 0.98 \\
42592 & 108984 & 2005 & 90.20 & 0.09 & 9025.00 & 89652.05 & 1.00 & 0.99 & 0.99 \\
56597 & 400230 & 2005 & 30.90 & -0.02 & 3301.00 & 30187.40 & 0.94 & 0.98 & 0.91 \\
42719 & 109009 & 2005 & 330.30 & 0.08 & 31774.00 & 313517.45 & 1.04 & 0.95 & 0.99 \\
47464 & 211485 & 2005 & 27.10 & 0.08 & 2713.00 & 26620.50 & 1.00 & 0.98 & 0.98 \\
33878 & 106179 & 2005 & 67.90 & 0.03 & 6941.00 & 69522.76 & 0.98 & 1.02 & 1.00 \\
33604 & 106152 & 2005 & 254.20 & 0.05 & 25477.00 & 250932.74 & 1.00 & 0.99 & 0.98 \\
51688 & 240541 & 2005 & 99.20 & 0.64 & 11296.00 & 82521.01 & 0.88 & 0.83 & 0.73 \\
30530 & 105763 & 2005 & 438.90 & 0.03 & 48507.00 & 425814.92 & 0.90 & 0.97 & 0.88 \\
47144 & 200338 & 2005 & 292.60 & 0.19 & 22764.00 & 231787.06 & 1.29 & 0.79 & 1.02 \\
56313 & 400188 & 2005 & 336.40 & 0.08 & 33832.00 & 336749.06 & 0.99 & 1.00 & 1.00 \\
30621 & 105779 & 2005 & 1019.40 & 0.05 & 102365.00 & 1017959.09 & 1.00 & 1.00 & 0.99 \\
34832 & 106281 & 2005 & 8.90 & 0.04 & 923.00 & 9364.29 & 0.96 & 1.05 & 1.01 \\
51623 & 240536 & 2005 & 136.80 & 0.07 & 14020.00 & 134882.57 & 0.98 & 0.99 & 0.96 \\
34714 & 106272 & 2005 & 2488.70 & 0.05 & 340483.00 & 3054629.34 & 0.73 & 1.23 & 0.90 \\
47067 & 200331 & 2005 & 33.80 & 0.09 & 3173.00 & 29230.64 & 1.07 & 0.86 & 0.92 \\
52780 & 322114 & 2005 & 11.80 & 0.10 & 1152.00 & 10735.61 & 1.02 & 0.91 & 0.93 \\
31869 & 105949 & 2005 & 1216.00 & 0.12 & 120022.00 & 1089592.66 & 1.01 & 0.90 & 0.91 \\
14810 & 101914 & 2005 & 38.00 & 0.13 & 3791.00 & 36979.76 & 1.00 & 0.97 & 0.98 \\
20424 & 102737 & 2005 & 1883.10 & 0.04 & 189782.00 & 1827906.83 & 0.99 & 0.97 & 0.96 \\
61485 & 500082 & 2005 & 561.10 & 0.09 & 54690.00 & 547146.41 & 1.03 & 0.98 & 1.00 \\
6307 & 100847 & 2005 & 3.70 & 0.05 & 366.00 & 3229.43 & 1.01 & 0.87 & 0.88 \\
34820 & 106278 & 2005 & 177.10 & 0.04 & 17860.00 & 173212.95 & 0.99 & 0.98 & 0.97 \\
15484 & 101998 & 2005 & 719.70 & 0.08 & 38262.00 & 369250.82 & 1.88 & 0.51 & 0.97 \\
33733 & 106164 & 2005 & 50.00 & -0.02 & 5032.00 & 49186.52 & 0.99 & 0.98 & 0.98 \\
19287 & 102579 & 2005 & 171.00 & 0.06 & 17131.00 & 167424.98 & 1.00 & 0.98 & 0.98 \\
44016 & 109258 & 2005 & 582.00 & 0.04 & 61221.00 & 607204.28 & 0.95 & 1.04 & 0.99 \\
59478 & 410487 & 2005 & 89.50 & 0.09 & 10291.00 & 97797.10 & 0.87 & 1.09 & 0.95 \\
48498 & 240090 & 2005 & 11.60 & 0.05 & 1268.00 & 11650.42 & 0.91 & 1.00 & 0.92 \\
46673 & 200276 & 2005 & 65.80 & 0.13 & 6604.00 & 62905.24 & 1.00 & 0.96 & 0.95 \\
9774 & 101192 & 2005 & 481.30 & 0.13 & 46662.00 & 469087.09 & 1.03 & 0.97 & 1.01 \\
21168 & 102835 & 2005 & 743.90 & 0.08 & 68130.00 & 553513.11 & 1.09 & 0.74 & 0.81 \\
34843 & 106282 & 2005 & 892.40 & 0.05 & 89196.00 & 855452.85 & 1.00 & 0.96 & 0.96 \\
47089 & 200332 & 2005 & 114.70 & -0.08 & 11692.00 & 116480.00 & 0.98 & 1.02 & 1.00 \\
4068 & 100544 & 2005 & 1030.00 & 0.27 & 101804.00 & 1060181.90 & 1.01 & 1.03 & 1.04 \\
30715 & 105788 & 2005 & 40.00 & -0.01 & 4015.00 & 38930.08 & 1.00 & 0.97 & 0.97 \\
52107 & 301571 & 2005 & 303.50 & 0.09 & 30096.00 & 288851.89 & 1.01 & 0.95 & 0.96 \\
59257 & 410448 & 2005 & 581.70 & 0.10 & 57123.00 & 564029.40 & 1.02 & 0.97 & 0.99 \\
56273 & 400186 & 2005 & 109.10 & 0.03 & 10948.00 & 109200.13 & 1.00 & 1.00 & 1.00 \\
52155 & 302206 & 2005 & 794.80 & 0.08 & 82446.00 & 820701.93 & 0.96 & 1.03 & 1.00 \\
52833 & 330728 & 2005 & 913.50 & 0.07 & 86660.00 & 866572.20 & 1.05 & 0.95 & 1.00 \\
15919 & 102059 & 2005 & 982.80 & 0.06 & 102799.00 & 980608.60 & 0.96 & 1.00 & 0.95 \\
14917 & 101922 & 2005 & 2668.00 & 0.05 & 244137.00 & 2485939.23 & 1.09 & 0.93 & 1.02 \\
21146 & 102833 & 2005 & 12.00 & -0.10 & 1159.00 & 11335.34 & 1.04 & 0.94 & 0.98 \\
31858 & 105948 & 2005 & 36.90 & 0.15 & 3721.00 & 35730.74 & 0.99 & 0.97 & 0.96 \\
59578 & 410498 & 2005 & 65.90 & 0.08 & 6768.00 & 67081.70 & 0.97 & 1.02 & 0.99 \\
31819 & 105943 & 2005 & 24.40 & 0.06 & 2432.00 & 24316.57 & 1.00 & 1.00 & 1.00 \\
48261 & 240057 & 2005 & 141.10 & 0.06 & 14122.00 & 122850.40 & 1.00 & 0.87 & 0.87 \\
31836 & 105946 & 2005 & 132.40 & 0.08 & 15672.00 & 153453.74 & 0.84 & 1.16 & 0.98 \\
46683 & 200277 & 2005 & 54.30 & 0.14 & 5506.00 & 54148.78 & 0.99 & 1.00 & 0.98 \\
10298 & 101278 & 2005 & 397.50 & 0.12 & 34503.00 & 295912.15 & 1.15 & 0.74 & 0.86 \\
43170 & 109072 & 2005 & 89.30 & 0.03 & 10628.00 & 104970.35 & 0.84 & 1.18 & 0.99 \\
43164 & 109071 & 2005 & 16.00 & 0.07 & 1606.00 & 15899.22 & 1.00 & 0.99 & 0.99 \\
43155 & 109069 & 2005 & 208.40 & 0.10 & 22407.00 & 214119.42 & 0.93 & 1.03 & 0.96 \\
9804 & 101193 & 2005 & 700.70 & 0.06 & 67905.00 & 681641.96 & 1.03 & 0.97 & 1.00 \\
34770 & 106276 & 2005 & 257.70 & -0.04 & 22029.00 & 218165.60 & 1.17 & 0.85 & 0.99 \\
56201 & 400181 & 2005 & 19.70 & 0.05 & 2233.00 & 22127.51 & 0.88 & 1.12 & 0.99 \\
42682 & 108993 & 2005 & 200.50 & 0.07 & 12909.00 & 115134.25 & 1.55 & 0.57 & 0.89 \\
59537 & 410495 & 2005 & 21.00 & 0.10 & 2097.00 & 19188.85 & 1.00 & 0.91 & 0.92 \\
42697 & 108994 & 2005 & 82.10 & 0.04 & 11047.00 & 106257.50 & 0.74 & 1.29 & 0.96 \\
46697 & 200279 & 2005 & 44.40 & 0.05 & 4443.00 & 44077.39 & 1.00 & 0.99 & 0.99 \\
54819 & 400015 & 2005 & 6.30 & 0.12 & 318.00 & 3044.59 & 1.98 & 0.48 & 0.96 \\
30688 & 105783 & 2005 & 7197.10 & 0.12 & 705155.00 & 7039875.34 & 1.02 & 0.98 & 1.00 \\
52811 & 330079 & 2005 & 44.20 & 0.07 & 4452.00 & 42574.13 & 0.99 & 0.96 & 0.96 \\
56578 & 400228 & 2005 & 63.00 & 0.08 & 5990.00 & 55973.39 & 1.05 & 0.89 & 0.93 \\
54793 & 400014 & 2005 & 144.70 & 0.01 & 14257.00 & 127341.47 & 1.01 & 0.88 & 0.89 \\
52143 & 302067 & 2005 & 19.80 & -0.00 & 1948.00 & 19476.97 & 1.02 & 0.98 & 1.00 \\
13519 & 101742 & 2005 & 4589.60 & 0.07 & 473573.00 & 4115342.57 & 0.97 & 0.90 & 0.87 \\
52135 & 302060 & 2005 & 79.50 & 0.15 & 8027.00 & 79686.66 & 0.99 & 1.00 & 0.99 \\
34743 & 106275 & 2005 & 132.10 & 0.14 & 10041.00 & 102022.86 & 1.32 & 0.77 & 1.02 \\
34797 & 106277 & 2005 & 730.90 & 0.14 & 68572.00 & 665097.17 & 1.07 & 0.91 & 0.97 \\
20392 & 102733 & 2005 & 3862.40 & 0.30 & 373089.00 & 3061058.85 & 1.04 & 0.79 & 0.82 \\
51619 & 240535 & 2005 & 8.40 & 0.15 & 808.00 & 7670.45 & 1.04 & 0.91 & 0.95 \\
61939 & 500310 & 2005 & 7.00 & 0.07 & 743.00 & 7385.66 & 0.94 & 1.06 & 0.99 \\
21114 & 102832 & 2005 & 49.00 & -0.03 & 4950.00 & 48531.13 & 0.99 & 0.99 & 0.98 \\
30657 & 105781 & 2005 & 799.00 & 0.11 & 78754.00 & 768966.38 & 1.01 & 0.96 & 0.98 \\
31847 & 105947 & 2005 & 45.30 & 0.06 & 4512.00 & 43942.02 & 1.00 & 0.97 & 0.97 \\
43127 & 109065 & 2005 & 58.70 & 0.03 & 5873.00 & 58573.98 & 1.00 & 1.00 & 1.00 \\
30597 & 105775 & 2005 & 938.70 & 0.10 & 93915.00 & 914566.03 & 1.00 & 0.97 & 0.97 \\
47094 & 200333 & 2005 & 2565.20 & 0.10 & 257187.00 & 2545442.40 & 1.00 & 0.99 & 0.99 \\
51682 & 240540 & 2005 & 81.20 & 0.11 & 7293.00 & 76449.28 & 1.11 & 0.94 & 1.05 \\
56590 & 400229 & 2005 & 19.60 & 0.04 & 1944.00 & 19877.98 & 1.01 & 1.01 & 1.02 \\
51583 & 240531 & 2005 & 40.10 & 0.08 & 6902.00 & 62073.31 & 0.58 & 1.55 & 0.90 \\
9756 & 101186 & 2005 & 476.90 & 0.03 & 47206.00 & 478751.89 & 1.01 & 1.00 & 1.01 \\
52166 & 302545 & 2005 & 104.00 & 0.02 & 11155.00 & 104788.01 & 0.93 & 1.01 & 0.94 \\
9835 & 101194 & 2005 & 305.20 & 0.05 & 29124.00 & 267266.95 & 1.05 & 0.88 & 0.92 \\
51680 & 240539 & 2005 & 23.00 & 0.08 & 2340.00 & 22147.50 & 0.98 & 0.96 & 0.95 \\
21200 & 102837 & 2005 & 912.50 & 0.08 & 85971.00 & 878541.89 & 1.06 & 0.96 & 1.02 \\
43768 & 109218 & 2005 & 38.30 & 0.12 & 2839.00 & 28992.53 & 1.35 & 0.76 & 1.02 \\
59451 & 410484 & 2005 & 4.40 & 0.11 & 436.00 & 4176.44 & 1.01 & 0.95 & 0.96 \\
13779 & 101763 & 2005 & 1112.90 & 0.16 & 111533.00 & 1092392.27 & 1.00 & 0.98 & 0.98 \\
7468 & 101040 & 2005 & 6103.60 & 0.10 & 588117.00 & 5970279.30 & 1.04 & 0.98 & 1.02 \\
30558 & 105769 & 2005 & 16.20 & -0.04 & 1466.00 & 12652.05 & 1.11 & 0.78 & 0.86 \\
31942 & 105963 & 2005 & 669.10 & 0.07 & 63126.00 & 623535.58 & 1.06 & 0.93 & 0.99 \\
15430 & 101990 & 2005 & 664.10 & 0.07 & 66524.00 & 645507.02 & 1.00 & 0.97 & 0.97 \\
51596 & 240533 & 2005 & 7.40 & 0.13 & 781.00 & 7092.50 & 0.95 & 0.96 & 0.91 \\
20341 & 102716 & 2005 & 1012.80 & 0.16 & 101993.00 & 952738.65 & 0.99 & 0.94 & 0.93 \\
46654 & 200273 & 2005 & 454.60 & 0.13 & 40377.00 & 378220.88 & 1.13 & 0.83 & 0.94 \\
15878 & 102050 & 2005 & 335.80 & 0.07 & 43975.00 & 371312.94 & 0.76 & 1.11 & 0.84 \\
34905 & 106292 & 2005 & 297.20 & 0.02 & 29695.00 & 293463.71 & 1.00 & 0.99 & 0.99 \\
43202 & 109083 & 2005 & 133.30 & 0.04 & 13844.00 & 139630.32 & 0.96 & 1.05 & 1.01 \\
56293 & 400187 & 2005 & 62.20 & 0.04 & 6242.00 & 61915.57 & 1.00 & 1.00 & 0.99 \\
33796 & 106170 & 2005 & 458.90 & 0.09 & 45872.00 & 447165.85 & 1.00 & 0.97 & 0.97 \\
42627 & 108987 & 2005 & 38.30 & 0.01 & 4011.00 & 39612.25 & 0.95 & 1.03 & 0.99 \\
43199 & 109080 & 2005 & 15.40 & 0.03 & 1485.00 & 15143.26 & 1.04 & 0.98 & 1.02 \\
47139 & 200336 & 2005 & 9.30 & 0.00 & 937.00 & 8882.64 & 0.99 & 0.96 & 0.95 \\
51557 & 240529 & 2005 & 241.00 & 0.08 & 24059.00 & 237543.72 & 1.00 & 0.99 & 0.99 \\
47046 & 200330 & 2005 & 37.50 & 0.02 & 3751.00 & 37275.25 & 1.00 & 0.99 & 0.99 \\
51563 & 240530 & 2005 & 55.30 & 0.16 & 4723.00 & 46140.44 & 1.17 & 0.83 & 0.98 \\
30757 & 105793 & 2005 & 2350.20 & 0.14 & 239506.00 & 2043380.71 & 0.98 & 0.87 & 0.85 \\
43196 & 109079 & 2005 & 26.90 & -0.01 & 2529.00 & 24902.01 & 1.06 & 0.93 & 0.98 \\
33673 & 106160 & 2005 & 183.40 & 0.02 & 18942.00 & 184074.70 & 0.97 & 1.00 & 0.97 \\
3686 & 100468 & 2005 & 3334.50 & 0.03 & 330746.00 & 3287924.75 & 1.01 & 0.99 & 0.99 \\
20454 & 102744 & 2005 & 685.70 & 0.07 & 68044.00 & 680389.58 & 1.01 & 0.99 & 1.00 \\
47481 & 212351 & 2005 & 61.20 & 0.08 & 5993.00 & 59544.92 & 1.02 & 0.97 & 0.99 \\
6089 & 100822 & 2005 & 15.10 & 0.04 & 1490.00 & 14897.13 & 1.01 & 0.99 & 1.00 \\
34681 & 106270 & 2005 & 23.30 & 0.04 & 2323.00 & 23233.81 & 1.00 & 1.00 & 1.00 \\
43190 & 109077 & 2005 & 6.00 & 0.08 & 493.00 & 4519.75 & 1.22 & 0.75 & 0.92 \\
47116 & 200334 & 2005 & 177.80 & -0.01 & 22691.00 & 218180.18 & 0.78 & 1.23 & 0.96 \\
52549 & 303123 & 2005 & 92.30 & 0.02 & 13005.00 & 95219.33 & 0.71 & 1.03 & 0.73 \\
42642 & 108990 & 2005 & 15.10 & 0.07 & 1736.00 & 14253.18 & 0.87 & 0.94 & 0.82 \\
6675 & 100908 & 2005 & 399.90 & 0.06 & 37696.00 & 355382.50 & 1.06 & 0.89 & 0.94 \\
34870 & 106283 & 2005 & 414.70 & 0.05 & 41965.00 & 378628.37 & 0.99 & 0.91 & 0.90 \\
9121 & 101112 & 2005 & 791.50 & 0.09 & 77260.00 & 786580.12 & 1.02 & 0.99 & 1.02 \\
33769 & 106169 & 2005 & 4112.80 & 0.10 & 410988.00 & 3906172.20 & 1.00 & 0.95 & 0.95 \\
19259 & 102575 & 2005 & 126.50 & 0.02 & 12661.00 & 117939.33 & 1.00 & 0.93 & 0.93 \\
33757 & 106167 & 2005 & 709.10 & 0.11 & 69650.00 & 658913.14 & 1.02 & 0.93 & 0.95 \\
33706 & 106163 & 2005 & 1137.50 & 0.12 & 113445.00 & 1082432.44 & 1.00 & 0.95 & 0.95 \\
31906 & 105957 & 2005 & 7.60 & 0.04 & 761.00 & 7534.66 & 1.00 & 0.99 & 0.99 \\
42650 & 108991 & 2005 & 200.40 & 0.05 & 20148.00 & 190810.79 & 0.99 & 0.95 & 0.95 \\
46666 & 200274 & 2005 & 30.20 & -0.02 & 2589.00 & 23973.00 & 1.17 & 0.79 & 0.93 \\
31757 & 105933 & 2005 & 1624.70 & 0.01 & 157673.00 & 1576490.81 & 1.03 & 0.97 & 1.00 \\
43183 & 109076 & 2005 & 133.40 & 0.07 & 14602.00 & 132578.42 & 0.91 & 0.99 & 0.91 \\
31768 & 105935 & 2005 & 455.90 & 0.07 & 48411.00 & 433658.24 & 0.94 & 0.95 & 0.90 \\
30736 & 105791 & 2005 & 6.60 & 0.06 & 654.00 & 5988.84 & 1.01 & 0.91 & 0.92 \\
31931 & 105961 & 2005 & 54.50 & 0.01 & 6009.00 & 61345.46 & 0.91 & 1.13 & 1.02 \\
30568 & 105770 & 2005 & 46.60 & 0.05 & 4658.00 & 43403.51 & 1.00 & 0.93 & 0.93 \\
15503 & 101999 & 2005 & 1477.40 & 0.05 & 160559.00 & 1412941.44 & 0.92 & 0.96 & 0.88 \\
19560 & 102624 & 2005 & 1292.70 & 0.03 & 130286.00 & 1286041.51 & 0.99 & 0.99 & 0.99 \\
42636 & 108988 & 2005 & 155.30 & 0.03 & 20199.00 & 203098.48 & 0.77 & 1.31 & 1.01 \\
47136 & 200335 & 2005 & 2.20 & 0.12 & 212.00 & 2111.69 & 1.04 & 0.96 & 1.00 \\
51608 & 240534 & 2005 & 307.10 & 0.19 & 23323.00 & 225098.52 & 1.32 & 0.73 & 0.97 \\
10327 & 101279 & 2005 & 489.60 & 0.13 & 37184.00 & 371841.04 & 1.32 & 0.76 & 1.00 \\
4181 & 100567 & 2005 & 594.00 & 0.03 & 59975.00 & 614250.78 & 0.99 & 1.03 & 1.02 \\
43104 & 109064 & 2005 & 161.00 & 0.07 & 14469.00 & 138718.51 & 1.11 & 0.86 & 0.96 \\
34890 & 106284 & 2005 & 436.00 & 0.15 & 41497.00 & 388108.80 & 1.05 & 0.89 & 0.94 \\
96742 & 611010 & 2005 & 255.80 & 0.06 & 25482.00 & 236957.90 & 1.00 & 0.93 & 0.93 \\
31921 & 105960 & 2005 & 118.50 & -0.02 & 12284.00 & 126269.93 & 0.96 & 1.07 & 1.03 \\
59465 & 410486 & 2005 & 153.10 & 0.08 & 16274.00 & 153024.62 & 0.94 & 1.00 & 0.94 \\
42397 & 108963 & 2005 & 145.90 & 0.18 & 12114.00 & 111875.15 & 1.20 & 0.77 & 0.92 \\
6384 & 100856 & 2005 & 122.60 & 0.23 & 11973.00 & 112524.39 & 1.02 & 0.92 & 0.94 \\
52900 & 333181 & 2005 & 63.50 & 0.00 & 6347.00 & 61576.73 & 1.00 & 0.97 & 0.97 \\
43318 & 109094 & 2005 & 185.20 & 0.04 & 18616.00 & 183343.01 & 0.99 & 0.99 & 0.98 \\
43975 & 109249 & 2005 & 145.30 & 0.07 & 16363.00 & 132362.21 & 0.89 & 0.91 & 0.81 \\
46956 & 200322 & 2005 & 38.10 & 0.08 & 3666.00 & 36796.33 & 1.04 & 0.97 & 1.00 \\
47305 & 200505 & 2005 & 680.10 & 0.04 & 76650.00 & 754092.50 & 0.89 & 1.11 & 0.98 \\
51726 & 240546 & 2005 & 28.80 & 0.04 & 2723.00 & 28131.45 & 1.06 & 0.98 & 1.03 \\
3583 & 100457 & 2005 & 101.10 & 0.08 & 10117.00 & 100478.83 & 1.00 & 0.99 & 0.99 \\
15593 & 102007 & 2005 & 2936.90 & 0.07 & 292342.00 & 2823896.81 & 1.00 & 0.96 & 0.97 \\
30183 & 105705 & 2005 & 90.30 & 0.02 & 9318.00 & 90317.24 & 0.97 & 1.00 & 0.97 \\
54988 & 400040 & 2005 & 147.50 & 0.07 & 14758.00 & 145434.88 & 1.00 & 0.99 & 0.99 \\
60762 & 410723 & 2005 & 5.50 & 0.06 & 543.00 & 4383.42 & 1.01 & 0.80 & 0.81 \\
6027 & 100820 & 2005 & 626.80 & 0.09 & 62191.00 & 617726.16 & 1.01 & 0.99 & 0.99 \\
43307 & 109092 & 2005 & 76.90 & 0.05 & 7458.00 & 75758.81 & 1.03 & 0.99 & 1.02 \\
51339 & 240502 & 2005 & 100.40 & 0.13 & 9828.00 & 95295.53 & 1.02 & 0.95 & 0.97 \\
35190 & 106333 & 2005 & 183.00 & 0.02 & 19203.00 & 178396.85 & 0.95 & 0.97 & 0.93 \\
52719 & 307603 & 2005 & 73.30 & 0.15 & 7453.00 & 69114.90 & 0.98 & 0.94 & 0.93 \\
20082 & 102665 & 2005 & 17.00 & 0.01 & 1700.00 & 13901.52 & 1.00 & 0.82 & 0.82 \\
51335 & 240501 & 2005 & 19.10 & 0.07 & 1928.00 & 18962.62 & 0.99 & 0.99 & 0.98 \\
20953 & 102813 & 2005 & 1502.20 & 0.05 & 150188.00 & 1461426.75 & 1.00 & 0.97 & 0.97 \\
10172 & 101264 & 2005 & 424.60 & -0.01 & 52715.00 & 502386.52 & 0.81 & 1.18 & 0.95 \\
43330 & 109095 & 2005 & 25.20 & 0.08 & 2494.00 & 23943.22 & 1.01 & 0.95 & 0.96 \\
52984 & 336508 & 2005 & 132.80 & 0.11 & 20178.00 & 135615.05 & 0.66 & 1.02 & 0.67 \\
52714 & 307384 & 2005 & 26.50 & 0.08 & 3871.00 & 38967.51 & 0.68 & 1.47 & 1.01 \\
59366 & 410470 & 2005 & 46.20 & 0.03 & 3818.00 & 37093.70 & 1.21 & 0.80 & 0.97 \\
35225 & 106335 & 2005 & 443.30 & 0.10 & 44435.00 & 438137.70 & 1.00 & 0.99 & 0.99 \\
33315 & 106114 & 2005 & 185.20 & 0.11 & 25306.00 & 189187.97 & 0.73 & 1.02 & 0.75 \\
47571 & 212809 & 2005 & 28.50 & 0.00 & 2844.00 & 27536.68 & 1.00 & 0.97 & 0.97 \\
59235 & 410446 & 2005 & 34.50 & 0.04 & 3551.00 & 33246.76 & 0.97 & 0.96 & 0.94 \\
52451 & 302942 & 2005 & 2004.30 & 0.09 & 200673.00 & 1997852.05 & 1.00 & 1.00 & 1.00 \\
3218 & 100415 & 2005 & 469.20 & 0.08 & 44794.00 & 452041.82 & 1.05 & 0.96 & 1.01 \\
33342 & 106116 & 2005 & 10.90 & 0.12 & 1084.00 & 8733.77 & 1.01 & 0.80 & 0.81 \\
15257 & 101972 & 2005 & 2947.00 & 0.04 & 345852.00 & 3216978.05 & 0.85 & 1.09 & 0.93 \\
59384 & 410472 & 2005 & 1308.90 & 0.07 & 129555.00 & 1270094.03 & 1.01 & 0.97 & 0.98 \\
32337 & 106011 & 2005 & 369.00 & 0.22 & 34779.00 & 381866.05 & 1.06 & 1.03 & 1.10 \\
51320 & 240500 & 2005 & 3008.80 & 0.09 & 304558.00 & 2966316.38 & 0.99 & 0.99 & 0.97 \\
9928 & 101212 & 2005 & 337.60 & 0.15 & 36788.00 & 326844.74 & 0.92 & 0.97 & 0.89 \\
21399 & 102861 & 2005 & 39.90 & 0.10 & 3897.00 & 38639.30 & 1.02 & 0.97 & 0.99 \\
56738 & 400249 & 2005 & 41.00 & 0.10 & 5313.00 & 55706.34 & 0.77 & 1.36 & 1.05 \\
32299 & 106009 & 2005 & 1300.40 & 0.09 & 129844.00 & 1157047.64 & 1.00 & 0.89 & 0.89 \\
30216 & 105716 & 2005 & 40.20 & -0.22 & 3944.00 & 39355.12 & 1.02 & 0.98 & 1.00 \\
56704 & 400241 & 2005 & 34.40 & 0.18 & 3222.00 & 27347.75 & 1.07 & 0.79 & 0.85 \\
34030 & 106198 & 2005 & 750.40 & 0.12 & 74964.00 & 746298.36 & 1.00 & 0.99 & 1.00 \\
19497 & 102607 & 2005 & 545.50 & 0.02 & 54008.00 & 540067.82 & 1.01 & 0.99 & 1.00 \\
42752 & 109015 & 2005 & 206.40 & 0.04 & 19289.00 & 185033.34 & 1.07 & 0.90 & 0.96 \\
35174 & 106330 & 2005 & 193.50 & -0.09 & 20464.00 & 202224.64 & 0.95 & 1.05 & 0.99 \\
46745 & 200294 & 2005 & 152.50 & 0.07 & 15269.00 & 151865.51 & 1.00 & 1.00 & 0.99 \\
30921 & 105836 & 2005 & 188.00 & 0.08 & 18617.00 & 191956.55 & 1.01 & 1.02 & 1.03 \\
52272 & 302731 & 2005 & 3000.60 & 0.14 & 300792.00 & 2631323.57 & 1.00 & 0.88 & 0.87 \\
52573 & 303130 & 2005 & 70.90 & 0.06 & 8030.00 & 74877.37 & 0.88 & 1.06 & 0.93 \\
10467 & 101286 & 2005 & 1579.10 & 0.01 & 157776.00 & 1515044.44 & 1.00 & 0.96 & 0.96 \\
20115 & 102667 & 2005 & 28543.00 & 0.09 & 2756000.00 & 27567339.16 & 1.04 & 0.97 & 1.00 \\
34003 & 106197 & 2005 & 70.80 & 0.00 & 10406.00 & 87768.75 & 0.68 & 1.24 & 0.84 \\
33386 & 106127 & 2005 & 593.80 & 0.05 & 67235.00 & 576264.55 & 0.88 & 0.97 & 0.86 \\
52963 & 336226 & 2005 & 213.40 & 0.10 & 18966.00 & 195619.15 & 1.13 & 0.92 & 1.03 \\
19662 & 102641 & 2005 & 470.00 & 0.14 & 46672.00 & 464199.27 & 1.01 & 0.99 & 0.99 \\
7283 & 101018 & 2005 & 22311.70 & 0.04 & 2186112.00 & 22306537.20 & 1.02 & 1.00 & 1.02 \\
51346 & 240504 & 2005 & 73.20 & -0.05 & 4207.00 & 41829.22 & 1.74 & 0.57 & 0.99 \\
8973 & 101107 & 2005 & 873.70 & 0.07 & 86962.00 & 864232.13 & 1.00 & 0.99 & 0.99 \\
33360 & 106124 & 2005 & 1864.00 & 0.05 & 207251.00 & 1755414.86 & 0.90 & 0.94 & 0.85 \\
51352 & 240505 & 2005 & 122.00 & 0.06 & 12210.00 & 120425.66 & 1.00 & 0.99 & 0.99 \\
7319 & 101020 & 2005 & 7315.60 & 0.06 & 619784.00 & 5454378.02 & 1.18 & 0.75 & 0.88 \\
32312 & 106010 & 2005 & 2933.60 & 0.05 & 293429.00 & 2817947.52 & 1.00 & 0.96 & 0.96 \\
54928 & 400028 & 2005 & 4.30 & 0.13 & 381.00 & 3779.49 & 1.13 & 0.88 & 0.99 \\
10483 & 101287 & 2005 & 917.90 & 0.03 & 94507.00 & 816587.98 & 0.97 & 0.89 & 0.86 \\
56712 & 400242 & 2005 & 173.50 & 0.13 & 18010.00 & 140358.34 & 0.96 & 0.81 & 0.78 \\
60743 & 410722 & 2005 & 253.60 & 0.08 & 24355.00 & 238649.41 & 1.04 & 0.94 & 0.98 \\
31534 & 105905 & 2005 & 17.70 & 0.10 & 1780.00 & 17752.47 & 0.99 & 1.00 & 1.00 \\
42767 & 109016 & 2005 & 1956.90 & 0.04 & 183239.00 & 1789357.83 & 1.07 & 0.91 & 0.98 \\
56716 & 400244 & 2005 & 3.10 & 0.11 & 271.00 & 2730.37 & 1.14 & 0.88 & 1.01 \\
30151 & 105703 & 2005 & 446.80 & 0.12 & 45197.00 & 365687.88 & 0.99 & 0.82 & 0.81 \\
31471 & 105895 & 2005 & 399.90 & -0.02 & 49314.00 & 467846.05 & 0.81 & 1.17 & 0.95 \\
62199 & 500364 & 2005 & 121.30 & 0.02 & 9390.00 & 94059.13 & 1.29 & 0.78 & 1.00 \\
44114 & 109266 & 2005 & 3521.40 & 0.17 & 331526.00 & 3426311.92 & 1.06 & 0.97 & 1.03 \\
9550 & 101149 & 2005 & 1209.10 & 0.02 & 124112.00 & 1229480.64 & 0.97 & 1.02 & 0.99 \\
4135 & 100559 & 2005 & 19.20 & 0.05 & 1988.00 & 20214.21 & 0.97 & 1.05 & 1.02 \\
43375 & 109104 & 2005 & 24.80 & 0.02 & 2475.00 & 24745.06 & 1.00 & 1.00 & 1.00 \\
32409 & 106023 & 2005 & 203.50 & 0.12 & 19992.00 & 199912.98 & 1.02 & 0.98 & 1.00 \\
56486 & 400217 & 2005 & 4.10 & 0.04 & 395.00 & 3953.45 & 1.04 & 0.96 & 1.00 \\
46923 & 200317 & 2005 & 28.00 & -0.07 & 3414.00 & 30305.67 & 0.82 & 1.08 & 0.89 \\
96784 & 611013 & 2005 & 31.60 & 0.05 & 3151.00 & 30583.69 & 1.00 & 0.97 & 0.97 \\
5988 & 100817 & 2005 & 86.90 & 0.06 & 8685.00 & 90021.11 & 1.00 & 1.04 & 1.04 \\
51734 & 240548 & 2005 & 9.40 & 0.14 & 780.00 & 7729.55 & 1.21 & 0.82 & 0.99 \\
56746 & 400250 & 2005 & 3.60 & 0.07 & 337.00 & 2958.66 & 1.07 & 0.82 & 0.88 \\
51755 & 240549 & 2005 & 110.70 & 0.11 & 9783.00 & 90769.25 & 1.13 & 0.82 & 0.93 \\
43380 & 109108 & 2005 & 65.60 & 0.12 & 6550.00 & 63689.83 & 1.00 & 0.97 & 0.97 \\
10151 & 101263 & 2005 & 220.80 & 0.04 & 21977.00 & 209689.97 & 1.00 & 0.95 & 0.95 \\
20020 & 102663 & 2005 & 3731.40 & 0.02 & 373140.00 & 3334735.55 & 1.00 & 0.89 & 0.89 \\
19675 & 102645 & 2005 & 329.20 & 0.01 & 35919.00 & 328963.81 & 0.92 & 1.00 & 0.92 \\
43007 & 109048 & 2005 & 190.40 & 0.06 & 18937.00 & 182374.74 & 1.01 & 0.96 & 0.96 \\
16082 & 102079 & 2005 & 353.50 & 0.04 & 39131.00 & 351039.59 & 0.90 & 0.99 & 0.90 \\
20624 & 102775 & 2005 & 3591.80 & 0.05 & 351291.00 & 3343764.57 & 1.02 & 0.93 & 0.95 \\
19150 & 102551 & 2005 & 3794.00 & 0.10 & 365577.00 & 3655801.82 & 1.04 & 0.96 & 1.00 \\
30992 & 105845 & 2005 & 1.90 & 0.01 & 327.00 & 3244.21 & 0.58 & 1.71 & 0.99 \\
42421 & 108964 & 2005 & 511.70 & 0.05 & 51528.00 & 506843.97 & 0.99 & 0.99 & 0.98 \\
52692 & 306690 & 2005 & 281.70 & 0.09 & 28149.00 & 271517.18 & 1.00 & 0.96 & 0.96 \\
33252 & 106108 & 2005 & 115.30 & 0.10 & 11267.00 & 103726.75 & 1.02 & 0.90 & 0.92 \\
51292 & 240498 & 2005 & 94.60 & 0.05 & 9149.00 & 85460.96 & 1.03 & 0.90 & 0.93 \\
15222 & 101968 & 2005 & 105.10 & 0.04 & 10509.00 & 102615.71 & 1.00 & 0.98 & 0.98 \\
19474 & 102606 & 2005 & 4429.00 & 0.05 & 436596.00 & 4348480.50 & 1.01 & 0.98 & 1.00 \\
62251 & 500369 & 2005 & 3.50 & 0.08 & 360.00 & 3354.91 & 0.97 & 0.96 & 0.93 \\
56371 & 400203 & 2005 & 31.30 & 0.10 & 3554.00 & 34338.98 & 0.88 & 1.10 & 0.97 \\
34099 & 106207 & 2005 & 18.40 & 0.12 & 1613.00 & 17024.17 & 1.14 & 0.93 & 1.06 \\
33273 & 106109 & 2005 & 30.20 & -0.02 & 3369.00 & 31087.73 & 0.90 & 1.03 & 0.92 \\
42438 & 108966 & 2005 & 136.40 & 0.02 & 13578.00 & 133737.97 & 1.00 & 0.98 & 0.98 \\
30965 & 105842 & 2005 & 437.50 & 0.08 & 43842.00 & 437159.93 & 1.00 & 1.00 & 1.00 \\
42456 & 108968 & 2005 & 51.00 & 0.01 & 4969.00 & 51056.31 & 1.03 & 1.00 & 1.03 \\
15241 & 101970 & 2005 & 124.90 & 0.06 & 12493.00 & 122439.01 & 1.00 & 0.98 & 0.98 \\
47441 & 211051 & 2005 & 285.30 & 0.21 & 27521.00 & 270887.90 & 1.04 & 0.95 & 0.98 \\
21427 & 102871 & 2005 & 674.40 & 0.11 & 66726.00 & 562331.79 & 1.01 & 0.83 & 0.84 \\
14778 & 101913 & 2005 & 293.70 & 0.10 & 22343.00 & 214067.10 & 1.31 & 0.73 & 0.96 \\
9089 & 101111 & 2005 & 372.80 & -0.00 & 39410.00 & 343275.41 & 0.95 & 0.92 & 0.87 \\
20604 & 102774 & 2005 & 2741.00 & 0.12 & 272307.00 & 2723006.00 & 1.01 & 0.99 & 1.00 \\
52596 & 303140 & 2005 & 852.90 & 0.04 & 83586.00 & 803890.09 & 1.02 & 0.94 & 0.96 \\
46934 & 200319 & 2005 & 268.90 & 0.14 & 22216.00 & 221786.23 & 1.21 & 0.82 & 1.00 \\
182 & 100017 & 2005 & 111.50 & 0.06 & 11264.00 & 109846.40 & 0.99 & 0.99 & 0.98 \\
43338 & 109099 & 2005 & 238.70 & 0.14 & 18518.00 & 197093.72 & 1.29 & 0.83 & 1.06 \\
6120 & 100823 & 2005 & 18.30 & 0.06 & 1811.00 & 18078.59 & 1.01 & 0.99 & 1.00 \\
54867 & 400019 & 2005 & 2539.20 & 0.13 & 231457.00 & 2310039.40 & 1.10 & 0.91 & 1.00 \\
32365 & 106014 & 2005 & 41.70 & 0.04 & 4205.00 & 40337.47 & 0.99 & 0.97 & 0.96 \\
44068 & 109264 & 2005 & 39.00 & 0.13 & 3657.00 & 36100.66 & 1.07 & 0.93 & 0.99 \\
43361 & 109100 & 2005 & 40.10 & 0.09 & 3253.00 & 29269.90 & 1.23 & 0.73 & 0.90 \\
52000 & 300695 & 2005 & 526.20 & 0.06 & 50379.00 & 472497.93 & 1.04 & 0.90 & 0.94 \\
52300 & 302760 & 2005 & 713.50 & 0.05 & 73408.00 & 652418.17 & 0.97 & 0.91 & 0.89 \\
3570 & 100456 & 2005 & 1.80 & 0.15 & 126.00 & 1273.53 & 1.43 & 0.71 & 1.01 \\
30120 & 105700 & 2005 & 165.90 & 0.01 & 18105.00 & 172502.95 & 0.92 & 1.04 & 0.95 \\
53001 & 336942 & 2005 & 320.10 & 0.17 & 25376.00 & 255393.61 & 1.26 & 0.80 & 1.01 \\
61189 & 410904 & 2005 & 332.60 & 0.07 & 32328.00 & 323281.53 & 1.03 & 0.97 & 1.00 \\
13622 & 101748 & 2005 & 2601.40 & -0.01 & 282644.00 & 2850063.93 & 0.92 & 1.10 & 1.01 \\
34073 & 106203 & 2005 & 33.90 & -0.00 & 3704.00 & 30765.93 & 0.92 & 0.91 & 0.83 \\
52997 & 336593 & 2005 & 54.50 & 0.05 & 4051.00 & 39718.17 & 1.35 & 0.73 & 0.98 \\
44091 & 109265 & 2005 & 58.60 & 0.07 & 5457.00 & 56507.99 & 1.07 & 0.96 & 1.04 \\
34538 & 106250 & 2005 & 152.80 & 0.06 & 14944.00 & 134150.57 & 1.02 & 0.88 & 0.90 \\
20054 & 102664 & 2005 & 5320.30 & 0.09 & 532030.00 & 4880675.72 & 1.00 & 0.92 & 0.92 \\
30130 & 105701 & 2005 & 364.60 & -0.02 & 44202.00 & 439188.54 & 0.82 & 1.20 & 0.99 \\
43822 & 109223 & 2005 & 27.40 & 0.10 & 3438.00 & 25243.22 & 0.80 & 0.92 & 0.73 \\
33288 & 106113 & 2005 & 468.90 & 0.05 & 46983.00 & 460694.75 & 1.00 & 0.98 & 0.98 \\
35252 & 106336 & 2005 & 83.70 & 0.12 & 7673.00 & 74465.49 & 1.09 & 0.89 & 0.97 \\
56059 & 400170 & 2005 & 366.00 & 0.05 & 27985.00 & 269262.77 & 1.31 & 0.74 & 0.96 \\
43290 & 109090 & 2005 & 98.20 & 0.11 & 9825.00 & 96623.53 & 1.00 & 0.98 & 0.98 \\
30240 & 105720 & 2005 & 744.40 & 0.08 & 74633.00 & 726617.05 & 1.00 & 0.98 & 0.97 \\
19338 & 102597 & 2005 & 147.10 & 0.07 & 14710.00 & 146856.14 & 1.00 & 1.00 & 1.00 \\
51417 & 240517 & 2005 & 62.70 & 0.12 & 5868.00 & 58986.45 & 1.07 & 0.94 & 1.01 \\
13590 & 101744 & 2005 & 1319.00 & 0.02 & 155281.00 & 1313316.79 & 0.85 & 1.00 & 0.85 \\
3246 & 100417 & 2005 & 13.60 & 0.04 & 1361.00 & 13527.61 & 1.00 & 0.99 & 0.99 \\
35071 & 106317 & 2005 & 87.90 & 0.02 & 8787.00 & 86846.84 & 1.00 & 0.99 & 0.99 \\
31616 & 105917 & 2005 & 223.80 & 0.02 & 21445.00 & 220480.48 & 1.04 & 0.99 & 1.03 \\
61572 & 500096 & 2005 & 149.50 & 0.14 & 15815.00 & 156186.17 & 0.95 & 1.04 & 0.99 \\
48649 & 240116 & 2005 & 304.70 & 0.16 & 26486.00 & 225135.60 & 1.15 & 0.74 & 0.85 \\
56092 & 400172 & 2005 & 734.80 & 0.01 & 62282.00 & 537434.68 & 1.18 & 0.73 & 0.86 \\
55688 & 400138 & 2005 & 82.70 & 0.07 & 8266.00 & 82227.90 & 1.00 & 0.99 & 0.99 \\
52194 & 302676 & 2005 & 758.40 & 0.04 & 67063.00 & 657313.08 & 1.13 & 0.87 & 0.98 \\
10423 & 101285 & 2005 & 765.50 & 0.07 & 76724.00 & 752346.97 & 1.00 & 0.98 & 0.98 \\
34561 & 106255 & 2005 & 606.00 & 0.05 & 52393.00 & 541466.59 & 1.16 & 0.89 & 1.03 \\
10216 & 101274 & 2005 & 385.70 & 0.11 & 54913.00 & 450794.26 & 0.70 & 1.17 & 0.82 \\
6236 & 100831 & 2005 & 134.60 & 0.08 & 12151.00 & 127229.03 & 1.11 & 0.95 & 1.05 \\
32187 & 105999 & 2005 & 4795.70 & 0.09 & 432538.00 & 4014628.13 & 1.11 & 0.84 & 0.93 \\
13809 & 101764 & 2005 & 561.20 & 0.07 & 63572.00 & 628927.30 & 0.88 & 1.12 & 0.99 \\
43057 & 109058 & 2005 & 95.30 & 0.08 & 9550.00 & 92825.87 & 1.00 & 0.97 & 0.97 \\
52220 & 302677 & 2005 & 48.00 & 0.01 & 5208.00 & 49966.23 & 0.92 & 1.04 & 0.96 \\
55011 & 400046 & 2005 & 107.10 & 0.06 & 10709.00 & 106000.25 & 1.00 & 0.99 & 0.99 \\
33484 & 106140 & 2005 & 868.70 & 0.04 & 90682.00 & 886015.63 & 0.96 & 1.02 & 0.98 \\
124 & 100009 & 2005 & 195.50 & 0.04 & 20306.00 & 202535.74 & 0.96 & 1.04 & 1.00 \\
9636 & 101160 & 2005 & 721.30 & 0.12 & 72320.00 & 698125.06 & 1.00 & 0.97 & 0.97 \\
51401 & 240512 & 2005 & 1.80 & 0.16 & 156.00 & 1577.25 & 1.15 & 0.88 & 1.01 \\
35080 & 106318 & 2005 & 97.60 & 0.11 & 9768.00 & 96073.79 & 1.00 & 0.98 & 0.98 \\
46612 & 200262 & 2005 & 2.20 & 0.02 & 160.00 & 1460.10 & 1.38 & 0.66 & 0.91 \\
164 & 100016 & 2005 & 135.60 & 0.06 & 13387.00 & 137473.52 & 1.01 & 1.01 & 1.03 \\
3789 & 100481 & 2005 & 172.20 & 0.12 & 16499.00 & 167635.14 & 1.04 & 0.97 & 1.02 \\
56507 & 400221 & 2005 & 70.00 & 0.11 & 5889.00 & 59268.13 & 1.19 & 0.85 & 1.01 \\
13387 & 101736 & 2005 & 42.10 & 0.07 & 4181.00 & 41722.67 & 1.01 & 0.99 & 1.00 \\
46607 & 200261 & 2005 & 2.80 & 0.01 & 266.00 & 2460.27 & 1.05 & 0.88 & 0.92 \\
48247 & 240056 & 2005 & 72.70 & 0.14 & 7422.00 & 67251.34 & 0.98 & 0.93 & 0.91 \\
33510 & 106143 & 2005 & 111.50 & 0.10 & 10959.00 & 104857.16 & 1.02 & 0.94 & 0.96 \\
56529 & 400222 & 2005 & 56.10 & 0.03 & 5423.00 & 54354.37 & 1.03 & 0.97 & 1.00 \\
30860 & 105805 & 2005 & 68.00 & 0.13 & 4789.00 & 49024.90 & 1.42 & 0.72 & 1.02 \\
47513 & 212408 & 2005 & 1492.30 & 0.04 & 151577.00 & 1493774.81 & 0.98 & 1.00 & 0.99 \\
9188 & 101116 & 2005 & 1014.10 & 0.10 & 104471.00 & 1031000.33 & 0.97 & 1.02 & 0.99 \\
48567 & 240107 & 2005 & 892.90 & 0.42 & 100896.00 & 926154.31 & 0.88 & 1.04 & 0.92 \\
52933 & 335811 & 2005 & 987.30 & 0.07 & 106826.00 & 952631.74 & 0.92 & 0.96 & 0.89 \\
15342 & 101987 & 2005 & 2003.00 & 0.18 & 169564.00 & 1778330.99 & 1.18 & 0.89 & 1.05 \\
34594 & 106257 & 2005 & 261.60 & 0.05 & 26142.00 & 255780.06 & 1.00 & 0.98 & 0.98 \\
33556 & 106148 & 2005 & 94.90 & 0.02 & 9493.00 & 94031.36 & 1.00 & 0.99 & 0.99 \\
46986 & 200325 & 2005 & 96.20 & 0.09 & 9602.00 & 86387.72 & 1.00 & 0.90 & 0.90 \\
61911 & 500306 & 2005 & 7.80 & 0.04 & 720.00 & 6713.58 & 1.08 & 0.86 & 0.93 \\
56118 & 400174 & 2005 & 270.30 & 0.14 & 27010.00 & 270096.87 & 1.00 & 1.00 & 1.00 \\
59246 & 410447 & 2005 & 126.80 & 0.09 & 13045.00 & 124066.91 & 0.97 & 0.98 & 0.95 \\
32136 & 105984 & 2005 & 151.10 & 0.03 & 17333.00 & 158304.39 & 0.87 & 1.05 & 0.91 \\
33913 & 106182 & 2005 & 654.70 & 0.08 & 95683.00 & 974762.78 & 0.68 & 1.49 & 1.02 \\
30378 & 105746 & 2005 & 277.30 & 0.02 & 27663.00 & 260794.03 & 1.00 & 0.94 & 0.94 \\
43745 & 109217 & 2005 & 498.50 & 0.20 & 39642.00 & 407673.09 & 1.26 & 0.82 & 1.03 \\
52921 & 335108 & 2005 & 143.40 & -0.01 & 18515.00 & 148618.09 & 0.77 & 1.04 & 0.80 \\
13736 & 101759 & 2005 & 16.60 & 0.04 & 1161.00 & 11574.14 & 1.43 & 0.70 & 1.00 \\
43235 & 109086 & 2005 & 1161.60 & 0.03 & 114213.00 & 1142126.88 & 1.02 & 0.98 & 1.00 \\
56098 & 400173 & 2005 & 34.00 & 0.04 & 3391.00 & 33472.31 & 1.00 & 0.98 & 0.99 \\
4218 & 100575 & 2005 & 8.10 & 0.04 & 791.00 & 8079.43 & 1.02 & 1.00 & 1.02 \\
33542 & 106147 & 2005 & 51.60 & 0.05 & 5125.00 & 49068.46 & 1.01 & 0.95 & 0.96 \\
56549 & 400224 & 2005 & 33.40 & 0.12 & 2975.00 & 30658.16 & 1.12 & 0.92 & 1.03 \\
56554 & 400225 & 2005 & 53.00 & 0.13 & 4752.00 & 47386.21 & 1.12 & 0.89 & 1.00 \\
31626 & 105918 & 2005 & 416.30 & -0.00 & 45773.00 & 444386.45 & 0.91 & 1.07 & 0.97 \\
20527 & 102761 & 2005 & 15676.50 & 0.05 & 1536723.00 & 15358807.19 & 1.02 & 0.98 & 1.00 \\
3262 & 100419 & 2005 & 373.00 & 0.05 & 37277.00 & 353249.69 & 1.00 & 0.95 & 0.95 \\
32156 & 105990 & 2005 & 207.30 & 0.12 & 26186.00 & 231395.41 & 0.79 & 1.12 & 0.88 \\
62142 & 500348 & 2005 & 691.00 & 0.13 & 69238.00 & 690833.41 & 1.00 & 1.00 & 1.00 \\
46978 & 200324 & 2005 & 318.40 & 0.12 & 33172.00 & 300962.57 & 0.96 & 0.95 & 0.91 \\
30361 & 105741 & 2005 & 218.80 & 0.03 & 21960.00 & 205551.83 & 1.00 & 0.94 & 0.94 \\
54829 & 400017 & 2005 & 133.40 & 0.02 & 13838.00 & 132181.51 & 0.96 & 0.99 & 0.96 \\
59417 & 410481 & 2005 & 23.10 & -0.00 & 2153.00 & 21206.98 & 1.07 & 0.92 & 0.98 \\
9893 & 101200 & 2005 & 19.90 & -0.00 & 1958.00 & 19087.21 & 1.02 & 0.96 & 0.97 \\
6642 & 100906 & 2005 & 1274.50 & -0.00 & 127200.00 & 1271480.50 & 1.00 & 1.00 & 1.00 \\
31566 & 105909 & 2005 & 60.80 & 0.09 & 6200.00 & 59655.29 & 0.98 & 0.98 & 0.96 \\
96762 & 611011 & 2005 & 38.40 & -0.02 & 3109.00 & 29696.63 & 1.24 & 0.77 & 0.96 \\
13287 & 101717 & 2005 & 54.50 & 0.03 & 5594.00 & 55285.47 & 0.97 & 1.01 & 0.99 \\
54926 & 400027 & 2005 & 16.40 & 0.05 & 1663.00 & 16209.22 & 0.99 & 0.99 & 0.97 \\
56660 & 400237 & 2005 & 31.20 & -0.04 & 3112.00 & 30193.37 & 1.00 & 0.97 & 0.97 \\
9910 & 101211 & 2005 & 386.60 & 0.04 & 36181.00 & 323938.42 & 1.07 & 0.84 & 0.90 \\
43982 & 109250 & 2005 & 159.50 & 0.08 & 18204.00 & 156323.57 & 0.88 & 0.98 & 0.86 \\
14958 & 101925 & 2005 & 11152.60 & 0.08 & 1103950.00 & 10509024.24 & 1.01 & 0.94 & 0.95 \\
54920 & 400025 & 2005 & 168.80 & 0.13 & 18095.00 & 157196.68 & 0.93 & 0.93 & 0.87 \\
10202 & 101268 & 2005 & 351.00 & 0.04 & 37279.00 & 356442.19 & 0.94 & 1.02 & 0.96 \\
56071 & 400171 & 2005 & 160.50 & 0.05 & 16829.00 & 157705.07 & 0.95 & 0.98 & 0.94 \\
4232 & 100590 & 2005 & 173.80 & 0.13 & 16476.00 & 174284.32 & 1.05 & 1.00 & 1.06 \\
61216 & 410909 & 2005 & 93.60 & 0.07 & 8555.00 & 76166.23 & 1.09 & 0.81 & 0.89 \\
13488 & 101741 & 2005 & 6241.20 & 0.12 & 613398.00 & 5149088.56 & 1.02 & 0.83 & 0.84 \\
33424 & 106135 & 2005 & 46.10 & 0.11 & 4610.00 & 45106.61 & 1.00 & 0.98 & 0.98 \\
15578 & 102005 & 2005 & 418.90 & 0.01 & 45465.00 & 433733.42 & 0.92 & 1.04 & 0.95 \\
21366 & 102854 & 2005 & 207.00 & 0.11 & 21320.00 & 213154.55 & 0.97 & 1.03 & 1.00 \\
52246 & 302698 & 2005 & 379.20 & 0.12 & 27838.00 & 283287.41 & 1.36 & 0.75 & 1.02 \\
42479 & 108970 & 2005 & 172.70 & 0.07 & 14681.00 & 149698.12 & 1.18 & 0.87 & 1.02 \\
16031 & 102073 & 2005 & 7913.40 & 0.03 & 838331.00 & 7974905.89 & 0.94 & 1.01 & 0.95 \\
33405 & 106129 & 2005 & 372.30 & 0.03 & 39102.00 & 402055.08 & 0.95 & 1.08 & 1.03 \\
43036 & 109056 & 2005 & 592.80 & 0.03 & 58985.00 & 506402.54 & 1.01 & 0.85 & 0.86 \\
35148 & 106329 & 2005 & 9.20 & 0.09 & 843.00 & 8438.05 & 1.09 & 0.92 & 1.00 \\
52732 & 307849 & 2005 & 108.90 & 0.09 & 11199.00 & 98204.63 & 0.97 & 0.90 & 0.88 \\
32271 & 106008 & 2005 & 324.60 & 0.20 & 32560.00 & 278121.72 & 1.00 & 0.86 & 0.85 \\
3401 & 100431 & 2005 & 445.30 & 0.07 & 52044.00 & 524504.61 & 0.86 & 1.18 & 1.01 \\
46722 & 200293 & 2005 & 88.10 & 0.07 & 8134.00 & 81183.20 & 1.08 & 0.92 & 1.00 \\
20564 & 102767 & 2005 & 3984.70 & 0.08 & 398275.00 & 3886007.19 & 1.00 & 0.98 & 0.98 \\
56349 & 400197 & 2005 & 10.90 & 0.07 & 1080.00 & 11207.27 & 1.01 & 1.03 & 1.04 \\
42491 & 108971 & 2005 & 179.20 & 0.08 & 20520.00 & 162339.00 & 0.87 & 0.91 & 0.79 \\
56361 & 400199 & 2005 & 131.70 & 0.03 & 13251.00 & 133151.43 & 0.99 & 1.01 & 1.00 \\
56559 & 400226 & 2005 & 468.70 & 0.07 & 43453.00 & 434496.97 & 1.08 & 0.93 & 1.00 \\
30273 & 105723 & 2005 & 919.50 & 0.12 & 88829.00 & 856464.59 & 1.04 & 0.93 & 0.96 \\
15306 & 101982 & 2005 & 241.20 & 0.05 & 24112.00 & 240048.40 & 1.00 & 1.00 & 1.00 \\
30888 & 105806 & 2005 & 159.30 & 0.07 & 15962.00 & 155498.72 & 1.00 & 0.98 & 0.97 \\
51377 & 240509 & 2005 & 43.90 & 0.04 & 4208.00 & 40625.07 & 1.04 & 0.93 & 0.97 \\
47257 & 200344 & 2005 & 16316.30 & 0.04 & 1789404.00 & 16300690.71 & 0.91 & 1.00 & 0.91 \\
56340 & 400192 & 2005 & 8.80 & 0.06 & 919.00 & 7800.99 & 0.96 & 0.89 & 0.85 \\
32215 & 106000 & 2005 & 210.10 & 0.13 & 20590.00 & 208996.03 & 1.02 & 0.99 & 1.02 \\
33960 & 106193 & 2005 & 23.50 & 0.07 & 2328.00 & 22729.54 & 1.01 & 0.97 & 0.98 \\
33933 & 106192 & 2005 & 1603.70 & 0.07 & 162477.00 & 1543953.03 & 0.99 & 0.96 & 0.95 \\
21335 & 102852 & 2005 & 934.60 & 0.01 & 98198.00 & 952963.38 & 0.95 & 1.02 & 0.97 \\
52559 & 303124 & 2005 & 75.70 & 0.23 & 7608.00 & 76523.67 & 1.00 & 1.01 & 1.01 \\
61264 & 500025 & 2005 & 4.40 & -0.01 & 387.00 & 3880.51 & 1.14 & 0.88 & 1.00 \\
6058 & 100821 & 2005 & 64.00 & 0.09 & 6301.00 & 62931.28 & 1.02 & 0.98 & 1.00 \\
51387 & 240510 & 2005 & 27.20 & 0.18 & 2646.00 & 26454.58 & 1.03 & 0.97 & 1.00 \\
35095 & 106320 & 2005 & 630.00 & 0.07 & 63833.00 & 642130.79 & 0.99 & 1.02 & 1.01 \\
30299 & 105731 & 2005 & 2041.10 & 0.09 & 197803.00 & 1793502.05 & 1.03 & 0.88 & 0.91 \\
59413 & 410479 & 2005 & 241.40 & 0.03 & 23983.00 & 239822.37 & 1.01 & 0.99 & 1.00 \\
51365 & 240507 & 2005 & 97.60 & -0.45 & 10551.00 & 107611.03 & 0.93 & 1.10 & 1.02 \\
3595 & 100460 & 2005 & 105.80 & 0.09 & 11547.00 & 112130.60 & 0.92 & 1.06 & 0.97 \\
20983 & 102814 & 2005 & 164.80 & 0.05 & 16541.00 & 162055.08 & 1.00 & 0.98 & 0.98 \\
47586 & 215413 & 2005 & 80.50 & 0.08 & 8090.00 & 77210.30 & 1.00 & 0.96 & 0.95 \\
8825 & 101100 & 2005 & 1466.80 & 0.54 & 132442.00 & 1355623.41 & 1.11 & 0.92 & 1.02 \\
52958 & 336065 & 2005 & 15.00 & -0.13 & 1561.00 & 15171.42 & 0.96 & 1.01 & 0.97 \\
33451 & 106136 & 2005 & 168.40 & 0.09 & 17078.00 & 159627.14 & 0.99 & 0.95 & 0.93 \\
61638 & 500109 & 2005 & 7431.90 & 1.09 & 809360.00 & 8314676.56 & 0.92 & 1.12 & 1.03 \\
35121 & 106321 & 2005 & 11.30 & 0.03 & 1179.00 & 11361.15 & 0.96 & 1.01 & 0.96 \\
20159 & 102671 & 2005 & 157.20 & -0.00 & 16343.00 & 166734.22 & 0.96 & 1.06 & 1.02 \\
42516 & 108973 & 2005 & 806.30 & 0.08 & 66087.00 & 624054.84 & 1.22 & 0.77 & 0.94 \\
34547 & 106251 & 2005 & 157.30 & 0.07 & 15421.00 & 147075.75 & 1.02 & 0.94 & 0.95 \\
14880 & 101919 & 2005 & 839.90 & 0.12 & 84094.00 & 865690.05 & 1.00 & 1.03 & 1.03 \\
43258 & 109087 & 2005 & 102.20 & 0.12 & 10952.00 & 98450.78 & 0.93 & 0.96 & 0.90 \\
52954 & 335933 & 2005 & 33.20 & 0.05 & 3337.00 & 32987.68 & 0.99 & 0.99 & 0.99 \\
9617 & 101158 & 2005 & 385.20 & 0.09 & 41476.00 & 413332.23 & 0.93 & 1.07 & 1.00 \\
58227 & 410130 & 2005 & 413.80 & 0.12 & 38383.00 & 372712.82 & 1.08 & 0.90 & 0.97 \\
7838 & 101062 & 2005 & 5834.80 & 0.15 & 558514.00 & 5711349.46 & 1.04 & 0.98 & 1.02 \\
74443 & 601001 & 2005 & 87.60 & 0.12 & 10575.00 & 90984.89 & 0.83 & 1.04 & 0.86 \\
23452 & 103177 & 2005 & 331.60 & 0.13 & 34117.00 & 323914.14 & 0.97 & 0.98 & 0.95 \\
26932 & 103628 & 2005 & 811.60 & 0.17 & 84110.00 & 830390.91 & 0.96 & 1.02 & 0.99 \\
37967 & 107175 & 2005 & 1606.80 & 0.06 & 151212.00 & 1586879.36 & 1.06 & 0.99 & 1.05 \\
45782 & 200133 & 2005 & 15.90 & 0.02 & 1594.00 & 16653.96 & 1.00 & 1.05 & 1.04 \\
18477 & 102462 & 2005 & 20.50 & 0.08 & 2126.00 & 21259.97 & 0.96 & 1.04 & 1.00 \\
5220 & 100736 & 2005 & 408.40 & 0.07 & 40928.00 & 389172.31 & 1.00 & 0.95 & 0.95 \\
37944 & 107173 & 2005 & 60.10 & 0.14 & 6742.00 & 67267.70 & 0.89 & 1.12 & 1.00 \\
44686 & 109359 & 2005 & 149.60 & 0.07 & 27075.00 & 261232.70 & 0.55 & 1.75 & 0.96 \\
748 & 100093 & 2005 & 197.20 & 0.16 & 17691.00 & 174828.37 & 1.11 & 0.89 & 0.99 \\
17257 & 102274 & 2005 & 11401.90 & 0.16 & 1128640.00 & 10832554.80 & 1.01 & 0.95 & 0.96 \\
44669 & 109358 & 2005 & 110.60 & 0.01 & 12523.00 & 119073.70 & 0.88 & 1.08 & 0.95 \\
47893 & 222658 & 2005 & 458.60 & 0.15 & 45784.00 & 451657.68 & 1.00 & 0.98 & 0.99 \\
53647 & 355536 & 2005 & 7.90 & 0.12 & 793.00 & 7689.25 & 1.00 & 0.97 & 0.97 \\
40961 & 108168 & 2005 & 395.60 & 0.01 & 39685.00 & 396513.63 & 1.00 & 1.00 & 1.00 \\
23418 & 103175 & 2005 & 851.90 & 0.03 & 84953.00 & 849535.20 & 1.00 & 1.00 & 1.00 \\
44661 & 109357 & 2005 & 302.50 & 0.05 & 31776.00 & 301802.06 & 0.95 & 1.00 & 0.95 \\
54296 & 367116 & 2005 & 3.50 & 0.00 & 289.00 & 2980.53 & 1.21 & 0.85 & 1.03 \\
26992 & 103643 & 2005 & 67.00 & 0.06 & 6720.00 & 68403.53 & 1.00 & 1.02 & 1.02 \\
40985 & 108170 & 2005 & 158.70 & 0.02 & 15946.00 & 155501.82 & 1.00 & 0.98 & 0.98 \\
40994 & 108172 & 2005 & 288.30 & 0.10 & 27741.00 & 227771.11 & 1.04 & 0.79 & 0.82 \\
45795 & 200140 & 2005 & 1187.30 & 0.03 & 121913.00 & 1186175.55 & 0.97 & 1.00 & 0.97 \\
11686 & 101456 & 2005 & 650.20 & 0.03 & 65903.00 & 629710.26 & 0.99 & 0.97 & 0.96 \\
41006 & 108175 & 2005 & 158.90 & 0.07 & 14912.00 & 150934.17 & 1.07 & 0.95 & 1.01 \\
17230 & 102271 & 2005 & 763.90 & 0.13 & 78990.00 & 781568.52 & 0.97 & 1.02 & 0.99 \\
23388 & 103174 & 2005 & 1496.80 & -0.02 & 149597.00 & 1487606.09 & 1.00 & 0.99 & 0.99 \\
55231 & 400074 & 2005 & 1786.40 & 0.06 & 177431.00 & 1721437.68 & 1.01 & 0.96 & 0.97 \\
18499 & 102465 & 2005 & 203.70 & 0.03 & 22413.00 & 231125.11 & 0.91 & 1.13 & 1.03 \\
54290 & 366837 & 2005 & 13.00 & 0.03 & 1299.00 & 11226.49 & 1.00 & 0.86 & 0.86 \\
49662 & 240330 & 2005 & 883.80 & 0.17 & 80760.00 & 755104.05 & 1.09 & 0.85 & 0.93 \\
23525 & 103183 & 2005 & 490.10 & 0.00 & 48744.00 & 487396.01 & 1.01 & 0.99 & 1.00 \\
26826 & 103608 & 2005 & 62.10 & 0.08 & 6296.00 & 66048.04 & 0.99 & 1.06 & 1.05 \\
38021 & 107192 & 2005 & 1066.80 & 0.10 & 104438.00 & 1044314.63 & 1.02 & 0.98 & 1.00 \\
7151 & 100998 & 2005 & 95.60 & 0.03 & 9554.00 & 94128.36 & 1.00 & 0.98 & 0.99 \\
14334 & 101850 & 2005 & 764.80 & 0.10 & 76427.00 & 737678.19 & 1.00 & 0.96 & 0.97 \\
53658 & 355965 & 2005 & 1675.30 & 0.12 & 168849.00 & 1692279.53 & 0.99 & 1.01 & 1.00 \\
17297 & 102278 & 2005 & 124.00 & 0.06 & 12817.00 & 123725.24 & 0.97 & 1.00 & 0.97 \\
2172 & 100293 & 2005 & 63.20 & 0.02 & 7075.00 & 71480.54 & 0.89 & 1.13 & 1.01 \\
26849 & 103609 & 2005 & 19.40 & 0.05 & 1909.00 & 18835.78 & 1.02 & 0.97 & 0.99 \\
64124 & 500588 & 2005 & 1787.30 & 0.13 & 178652.00 & 1785242.59 & 1.00 & 1.00 & 1.00 \\
48960 & 240174 & 2005 & 146.80 & 0.14 & 15499.00 & 128549.60 & 0.95 & 0.88 & 0.83 \\
49637 & 240327 & 2005 & 20.50 & 0.10 & 2523.00 & 20371.76 & 0.81 & 0.99 & 0.81 \\
54246 & 364993 & 2005 & 100.10 & 0.04 & 9858.00 & 100850.42 & 1.02 & 1.01 & 1.02 \\
26865 & 103614 & 2005 & 184.40 & 0.13 & 18029.00 & 175918.89 & 1.02 & 0.95 & 0.98 \\
38001 & 107181 & 2005 & 125.40 & 0.07 & 12599.00 & 116526.29 & 1.00 & 0.93 & 0.92 \\
18460 & 102461 & 2005 & 649.60 & 0.11 & 65077.00 & 641713.21 & 1.00 & 0.99 & 0.99 \\
5202 & 100731 & 2005 & 5484.30 & 0.05 & 580571.00 & 5862091.26 & 0.94 & 1.07 & 1.01 \\
37989 & 107179 & 2005 & 940.50 & 0.04 & 98301.00 & 967452.23 & 0.96 & 1.03 & 0.98 \\
57938 & 410003 & 2005 & 364.20 & 0.04 & 38615.00 & 383434.43 & 0.94 & 1.05 & 0.99 \\
2205 & 100295 & 2005 & 17.00 & 0.04 & 1734.00 & 16600.34 & 0.98 & 0.98 & 0.96 \\
778 & 100096 & 2005 & 92.90 & 0.11 & 8955.00 & 91822.09 & 1.04 & 0.99 & 1.03 \\
64101 & 500587 & 2005 & 1955.00 & 0.06 & 183311.00 & 1892158.96 & 1.07 & 0.97 & 1.03 \\
49653 & 240328 & 2005 & 19.50 & 0.15 & 1950.00 & 18664.54 & 1.00 & 0.96 & 0.96 \\
26897 & 103620 & 2005 & 340.30 & 0.08 & 35499.00 & 357954.99 & 0.96 & 1.05 & 1.01 \\
37979 & 107178 & 2005 & 72.70 & 0.04 & 7201.00 & 72001.76 & 1.01 & 0.99 & 1.00 \\
54268 & 365483 & 2005 & 2437.20 & 0.15 & 235875.00 & 2378155.06 & 1.03 & 0.98 & 1.01 \\
14359 & 101851 & 2005 & 3869.20 & 0.09 & 394046.00 & 3837795.69 & 0.98 & 0.99 & 0.97 \\
11719 & 101457 & 2005 & 252.70 & 0.10 & 25420.00 & 261431.26 & 0.99 & 1.03 & 1.03 \\
27018 & 103644 & 2005 & 47.60 & 0.11 & 4788.00 & 48158.67 & 0.99 & 1.01 & 1.01 \\
47866 & 222408 & 2005 & 1220.40 & 0.06 & 121885.00 & 1194141.09 & 1.00 & 0.98 & 0.98 \\
5253 & 100741 & 2005 & 261.50 & 0.46 & 25125.00 & 245409.24 & 1.04 & 0.94 & 0.98 \\
27137 & 105246 & 2005 & 6221.90 & 0.05 & 600697.00 & 5779910.36 & 1.04 & 0.93 & 0.96 \\
37805 & 107145 & 2005 & 215.30 & 0.10 & 41764.00 & 419392.61 & 0.52 & 1.95 & 1.00 \\
37782 & 107144 & 2005 & 55.60 & 0.15 & 5847.00 & 53563.75 & 0.95 & 0.96 & 0.92 \\
54367 & 367500 & 2005 & 251.10 & 0.14 & 25317.00 & 243594.63 & 0.99 & 0.97 & 0.96 \\
5296 & 100746 & 2005 & 2123.70 & 0.07 & 212264.00 & 2122893.92 & 1.00 & 1.00 & 1.00 \\
2339 & 100319 & 2005 & 358.10 & 0.03 & 41360.00 & 345915.03 & 0.87 & 0.97 & 0.84 \\
53611 & 354931 & 2005 & 75.90 & 0.10 & 7329.00 & 73290.13 & 1.04 & 0.97 & 1.00 \\
45849 & 200148 & 2005 & 404.90 & 0.11 & 37854.00 & 378987.44 & 1.07 & 0.94 & 1.00 \\
64055 & 500585 & 2005 & 2619.90 & 0.04 & 214779.00 & 2242764.25 & 1.22 & 0.86 & 1.04 \\
27167 & 105248 & 2005 & 347.90 & 0.33 & 30596.00 & 292395.44 & 1.14 & 0.84 & 0.96 \\
37754 & 107141 & 2005 & 1866.60 & 0.03 & 184577.00 & 1864135.85 & 1.01 & 1.00 & 1.01 \\
17167 & 102261 & 2005 & 3645.90 & 0.09 & 361601.00 & 3543983.13 & 1.01 & 0.97 & 0.98 \\
54389 & 367567 & 2005 & 427.10 & 0.04 & 42565.00 & 390030.56 & 1.00 & 0.91 & 0.92 \\
49721 & 240337 & 2005 & 11.30 & -0.01 & 1073.00 & 9933.39 & 1.05 & 0.88 & 0.93 \\
37723 & 107135 & 2005 & 636.20 & 0.10 & 63602.00 & 624460.10 & 1.00 & 0.98 & 0.98 \\
49728 & 240344 & 2005 & 147.20 & 0.10 & 16292.00 & 156842.37 & 0.90 & 1.07 & 0.96 \\
55450 & 400099 & 2005 & 57.40 & 0.13 & 9346.00 & 90050.20 & 0.61 & 1.57 & 0.96 \\
662 & 100087 & 2005 & 2353.70 & 0.07 & 242491.00 & 2485415.75 & 0.97 & 1.06 & 1.02 \\
41042 & 108183 & 2005 & 171.10 & 0.12 & 16981.00 & 159763.57 & 1.01 & 0.93 & 0.94 \\
58914 & 410232 & 2005 & 13.50 & 0.10 & 1357.00 & 13568.48 & 0.99 & 1.01 & 1.00 \\
14409 & 101854 & 2005 & 19628.80 & 0.06 & 1962898.00 & 19088539.72 & 1.00 & 0.97 & 0.97 \\
49734 & 240345 & 2005 & 114.20 & 0.10 & 10971.00 & 98616.64 & 1.04 & 0.86 & 0.90 \\
49743 & 240352 & 2005 & 3.10 & 0.11 & 307.00 & 2963.76 & 1.01 & 0.96 & 0.97 \\
23255 & 103152 & 2005 & 3052.20 & 0.03 & 303048.00 & 2924188.24 & 1.01 & 0.96 & 0.96 \\
27224 & 105256 & 2005 & 649.00 & 0.01 & 64666.00 & 646660.87 & 1.00 & 1.00 & 1.00 \\
47842 & 222351 & 2005 & 512.90 & 0.08 & 51353.00 & 494203.20 & 1.00 & 0.96 & 0.96 \\
23314 & 103160 & 2005 & 285.70 & 0.12 & 28607.00 & 278775.24 & 1.00 & 0.98 & 0.97 \\
5275 & 100745 & 2005 & 856.40 & 0.15 & 95753.00 & 877158.21 & 0.89 & 1.02 & 0.92 \\
41019 & 108180 & 2005 & 123.20 & 0.12 & 10723.00 & 105087.78 & 1.15 & 0.85 & 0.98 \\
37906 & 107160 & 2005 & 13200.00 & 0.14 & 1319142.00 & 13173828.21 & 1.00 & 1.00 & 1.00 \\
2265 & 100303 & 2005 & 423.40 & 0.07 & 42538.00 & 408262.56 & 1.00 & 0.96 & 0.96 \\
714 & 100092 & 2005 & 672.50 & 0.09 & 64343.00 & 653222.62 & 1.05 & 0.97 & 1.02 \\
54301 & 367166 & 2005 & 216.80 & 0.09 & 21798.00 & 209932.43 & 0.99 & 0.97 & 0.96 \\
54323 & 367168 & 2005 & 17.30 & 0.06 & 1735.00 & 16526.43 & 1.00 & 0.96 & 0.95 \\
8323 & 101082 & 2005 & 2332.90 & 0.17 & 231973.00 & 2298470.17 & 1.01 & 0.99 & 0.99 \\
2278 & 100305 & 2005 & 84.00 & 0.08 & 8107.00 & 84250.84 & 1.04 & 1.00 & 1.04 \\
37885 & 107156 & 2005 & 149.50 & -0.01 & 16104.00 & 135565.60 & 0.93 & 0.91 & 0.84 \\
37868 & 107152 & 2005 & 94.50 & 0.04 & 9473.00 & 76568.32 & 1.00 & 0.81 & 0.81 \\
23356 & 103166 & 2005 & 1.10 & -0.09 & 110.00 & 1110.55 & 1.00 & 1.01 & 1.01 \\
7170 & 101000 & 2005 & 837.60 & 0.05 & 83769.00 & 791253.19 & 1.00 & 0.94 & 0.94 \\
5180 & 100730 & 2005 & 497.90 & 0.18 & 55437.00 & 574151.82 & 0.90 & 1.15 & 1.04 \\
27074 & 103647 & 2005 & 31.40 & 0.07 & 3398.00 & 30008.11 & 0.92 & 0.96 & 0.88 \\
49695 & 240333 & 2005 & 99.10 & 0.14 & 10983.00 & 114154.67 & 0.90 & 1.15 & 1.04 \\
7806 & 101061 & 2005 & 3478.00 & 0.10 & 342112.00 & 3244461.03 & 1.02 & 0.93 & 0.95 \\
64078 & 500586 & 2005 & 6185.00 & 0.07 & 533953.00 & 5497999.64 & 1.16 & 0.89 & 1.03 \\
37830 & 107147 & 2005 & 18.50 & -0.04 & 2046.00 & 18801.63 & 0.90 & 1.02 & 0.92 \\
74594 & 601139 & 2005 & 6902.30 & 0.06 & 693030.00 & 6852922.64 & 1.00 & 0.99 & 0.99 \\
45832 & 200147 & 2005 & 57.30 & 0.02 & 6149.00 & 61132.16 & 0.93 & 1.07 & 0.99 \\
17196 & 102270 & 2005 & 618.40 & 0.05 & 63933.00 & 637885.63 & 0.97 & 1.03 & 1.00 \\
690 & 100090 & 2005 & 477.50 & 0.11 & 45340.00 & 468325.37 & 1.05 & 0.98 & 1.03 \\
54342 & 367206 & 2005 & 278.10 & 0.07 & 28689.00 & 262936.80 & 0.97 & 0.95 & 0.92 \\
45811 & 200146 & 2005 & 126.50 & 0.04 & 13395.00 & 109196.98 & 0.94 & 0.86 & 0.82 \\
8282 & 101081 & 2005 & 488.00 & 0.07 & 50048.00 & 463466.93 & 0.98 & 0.95 & 0.93 \\
26526 & 103590 & 2005 & 1360.60 & 0.05 & 136574.00 & 1212242.13 & 1.00 & 0.89 & 0.89 \\
23695 & 103208 & 2005 & 1183.20 & 0.12 & 118655.00 & 1047222.81 & 1.00 & 0.89 & 0.88 \\
38347 & 107244 & 2005 & 204.90 & 0.05 & 20398.00 & 200306.18 & 1.00 & 0.98 & 0.98 \\
64239 & 500593 & 2005 & 2513.30 & 0.11 & 251146.00 & 2508440.46 & 1.00 & 1.00 & 1.00 \\
5139 & 100726 & 2005 & 3290.10 & 0.14 & 328462.00 & 3356947.16 & 1.00 & 1.02 & 1.02 \\
838 & 100098 & 2005 & 1224.30 & 0.10 & 109370.00 & 1016374.30 & 1.12 & 0.83 & 0.93 \\
18404 & 102447 & 2005 & 7414.90 & 0.08 & 741489.00 & 6750072.68 & 1.00 & 0.91 & 0.91 \\
26558 & 103591 & 2005 & 1669.00 & 0.03 & 171525.00 & 1588432.09 & 0.97 & 0.95 & 0.93 \\
38322 & 107243 & 2005 & 1049.10 & 0.08 & 105632.00 & 1085613.74 & 0.99 & 1.03 & 1.03 \\
40826 & 108155 & 2005 & 70.70 & 0.02 & 7067.00 & 70917.97 & 1.00 & 1.00 & 1.00 \\
38297 & 107242 & 2005 & 519.80 & -0.01 & 52199.00 & 490778.18 & 1.00 & 0.94 & 0.94 \\
23662 & 103205 & 2005 & 356.40 & 0.11 & 29483.00 & 296080.09 & 1.21 & 0.83 & 1.00 \\
17389 & 102284 & 2005 & 342.00 & 0.08 & 40136.00 & 318465.97 & 0.85 & 0.93 & 0.79 \\
49570 & 240316 & 2005 & 31.60 & 0.13 & 3468.00 & 31208.02 & 0.91 & 0.99 & 0.90 \\
44737 & 109370 & 2005 & 1050.60 & 0.05 & 116125.00 & 1047523.93 & 0.90 & 1.00 & 0.90 \\
12645 & 101561 & 2005 & 169.50 & 0.08 & 16854.00 & 163782.20 & 1.01 & 0.97 & 0.97 \\
26589 & 103592 & 2005 & 374.40 & 0.04 & 39206.00 & 368361.62 & 0.95 & 0.98 & 0.94 \\
38285 & 107235 & 2005 & 60.70 & 0.05 & 6036.00 & 53772.91 & 1.01 & 0.89 & 0.89 \\
64216 & 500592 & 2005 & 3014.60 & 0.06 & 301262.00 & 3010769.87 & 1.00 & 1.00 & 1.00 \\
7022 & 100985 & 2005 & 2828.80 & 0.01 & 281415.00 & 2744055.42 & 1.01 & 0.97 & 0.98 \\
38277 & 107234 & 2005 & 4.80 & 0.07 & 471.00 & 4339.94 & 1.02 & 0.90 & 0.92 \\
40836 & 108156 & 2005 & 14.30 & 0.04 & 1391.00 & 13856.72 & 1.03 & 0.97 & 1.00 \\
57972 & 410010 & 2005 & 2016.70 & 0.09 & 188889.00 & 1694074.75 & 1.07 & 0.84 & 0.90 \\
49579 & 240318 & 2005 & 92.70 & 0.00 & 9755.00 & 96913.14 & 0.95 & 1.05 & 0.99 \\
44727 & 109368 & 2005 & 1081.90 & 0.09 & 107800.00 & 1065619.54 & 1.00 & 0.98 & 0.99 \\
11786 & 101461 & 2005 & 1889.10 & 0.02 & 198653.00 & 1950270.22 & 0.95 & 1.03 & 0.98 \\
14027 & 101800 & 2005 & 419.70 & 0.06 & 41952.00 & 399503.65 & 1.00 & 0.95 & 0.95 \\
38462 & 107260 & 2005 & 703.30 & -0.01 & 76577.00 & 721001.55 & 0.92 & 1.03 & 0.94 \\
44752 & 109371 & 2005 & 374.40 & 0.02 & 46031.00 & 465940.73 & 0.81 & 1.24 & 1.01 \\
38437 & 107259 & 2005 & 847.10 & -0.00 & 108193.00 & 934943.45 & 0.78 & 1.10 & 0.86 \\
66297 & 500806 & 2005 & 335.30 & 0.00 & 36318.00 & 355979.48 & 0.92 & 1.06 & 0.98 \\
49511 & 240308 & 2005 & 41.30 & 0.10 & 5163.00 & 44572.19 & 0.80 & 1.08 & 0.86 \\
58624 & 410180 & 2005 & 114.00 & 0.16 & 11349.00 & 110894.47 & 1.00 & 0.97 & 0.98 \\
26438 & 103580 & 2005 & 256.40 & 0.08 & 25945.00 & 245072.38 & 0.99 & 0.96 & 0.94 \\
38428 & 107258 & 2005 & 87.30 & 0.07 & 10182.00 & 87021.55 & 0.86 & 1.00 & 0.85 \\
40717 & 108146 & 2005 & 37.40 & 0.04 & 3615.00 & 36152.57 & 1.03 & 0.97 & 1.00 \\
58016 & 410018 & 2005 & 12.90 & 0.09 & 1287.00 & 12051.75 & 1.00 & 0.93 & 0.94 \\
49003 & 240198 & 2005 & 348.80 & 0.07 & 35489.00 & 351751.61 & 0.98 & 1.01 & 0.99 \\
38403 & 107257 & 2005 & 195.00 & 0.03 & 20837.00 & 211114.21 & 0.94 & 1.08 & 1.01 \\
868 & 100099 & 2005 & 81.70 & 0.16 & 7308.00 & 71095.80 & 1.12 & 0.87 & 0.97 \\
40846 & 108158 & 2005 & 24.20 & 0.06 & 2381.00 & 23811.89 & 1.02 & 0.98 & 1.00 \\
58644 & 410181 & 2005 & 34.30 & 0.65 & 2572.00 & 23333.64 & 1.33 & 0.68 & 0.91 \\
23729 & 103209 & 2005 & 208.30 & 0.07 & 21090.00 & 204121.30 & 0.99 & 0.98 & 0.97 \\
49536 & 240311 & 2005 & 18.50 & 0.01 & 1840.00 & 18231.01 & 1.01 & 0.99 & 0.99 \\
47925 & 222809 & 2005 & 527.20 & 0.00 & 52652.00 & 505735.09 & 1.00 & 0.96 & 0.96 \\
26489 & 103582 & 2005 & 17.50 & 0.08 & 1760.00 & 17188.21 & 0.99 & 0.98 & 0.98 \\
49559 & 240312 & 2005 & 310.90 & 0.09 & 31154.00 & 306953.66 & 1.00 & 0.99 & 0.99 \\
2104 & 100291 & 2005 & 790.80 & 0.08 & 78507.00 & 772901.46 & 1.01 & 0.98 & 0.98 \\
64262 & 500594 & 2005 & 3014.30 & 2.14 & 285267.00 & 2884595.89 & 1.06 & 0.96 & 1.01 \\
40746 & 108148 & 2005 & 80.50 & 0.10 & 8276.00 & 82942.25 & 0.97 & 1.03 & 1.00 \\
11817 & 101462 & 2005 & 1063.00 & 0.03 & 107775.00 & 974364.93 & 0.99 & 0.92 & 0.90 \\
40771 & 108149 & 2005 & 178.10 & 0.38 & 17481.00 & 173307.49 & 1.02 & 0.97 & 0.99 \\
73368 & 600006 & 2005 & 241.80 & 0.07 & 24317.00 & 217766.92 & 0.99 & 0.90 & 0.90 \\
4601 & 100642 & 2005 & 972.20 & 0.03 & 110276.00 & 1003674.21 & 0.88 & 1.03 & 0.91 \\
11605 & 101431 & 2005 & 290.80 & 0.13 & 37362.00 & 297183.56 & 0.78 & 1.02 & 0.80 \\
26623 & 103593 & 2005 & 76551.90 & 0.03 & 7649879.00 & 74790823.96 & 1.00 & 0.98 & 0.98 \\
38246 & 107226 & 2005 & 544.90 & 0.08 & 72180.00 & 687669.61 & 0.75 & 1.26 & 0.95 \\
58651 & 410185 & 2005 & 2.10 & 0.08 & 228.00 & 2099.08 & 0.92 & 1.00 & 0.92 \\
23577 & 103193 & 2005 & 173.20 & 0.14 & 14987.00 & 150344.57 & 1.16 & 0.87 & 1.00 \\
26728 & 103601 & 2005 & 2074.00 & -0.02 & 228785.00 & 1847763.73 & 0.91 & 0.89 & 0.81 \\
38119 & 107202 & 2005 & 25.30 & 10.09 & 2439.00 & 20130.94 & 1.04 & 0.80 & 0.83 \\
45705 & 200091 & 2005 & 8.70 & 0.03 & 824.00 & 8208.62 & 1.06 & 0.94 & 1.00 \\
38094 & 107201 & 2005 & 453.70 & 0.04 & 63083.00 & 469994.53 & 0.72 & 1.04 & 0.75 \\
58659 & 410199 & 2005 & 17.30 & 0.09 & 1497.00 & 15684.47 & 1.16 & 0.91 & 1.05 \\
808 & 100097 & 2005 & 154.10 & 0.14 & 13145.00 & 133216.79 & 1.17 & 0.86 & 1.01 \\
58664 & 410200 & 2005 & 44.40 & 0.05 & 4221.00 & 42619.97 & 1.05 & 0.96 & 1.01 \\
49613 & 240322 & 2005 & 1638.70 & 0.14 & 157535.00 & 1396779.51 & 1.04 & 0.85 & 0.89 \\
45709 & 200092 & 2005 & 44.50 & 0.06 & 4519.00 & 44574.21 & 0.98 & 1.00 & 0.99 \\
23558 & 103186 & 2005 & 761.40 & 0.06 & 70542.00 & 633151.81 & 1.08 & 0.83 & 0.90 \\
26767 & 103606 & 2005 & 240.50 & 0.06 & 22048.00 & 223493.01 & 1.09 & 0.93 & 1.01 \\
11753 & 101460 & 2005 & 1597.60 & 0.08 & 159636.00 & 1499806.73 & 1.00 & 0.94 & 0.94 \\
38085 & 107199 & 2005 & 24.00 & 0.02 & 2903.00 & 29024.30 & 0.83 & 1.21 & 1.00 \\
45731 & 200094 & 2005 & 186.50 & 0.09 & 17367.00 & 169828.60 & 1.07 & 0.91 & 0.98 \\
38069 & 107198 & 2005 & 356.70 & 0.10 & 36755.00 & 367524.27 & 0.97 & 1.03 & 1.00 \\
2141 & 100292 & 2005 & 4702.70 & 0.01 & 470701.00 & 4668305.61 & 1.00 & 0.99 & 0.99 \\
64193 & 500591 & 2005 & 1606.20 & 0.06 & 160941.00 & 1603680.37 & 1.00 & 1.00 & 1.00 \\
45748 & 200097 & 2005 & 20.40 & 0.06 & 2032.00 & 20115.31 & 1.00 & 0.99 & 0.99 \\
12676 & 101562 & 2005 & 457.80 & 0.11 & 45207.00 & 426262.05 & 1.01 & 0.93 & 0.94 \\
17315 & 102280 & 2005 & 2982.00 & 0.18 & 298780.00 & 2859024.93 & 1.00 & 0.96 & 0.96 \\
64170 & 500590 & 2005 & 1280.50 & 0.09 & 127955.00 & 1279030.68 & 1.00 & 1.00 & 1.00 \\
26799 & 103607 & 2005 & 195.20 & 0.05 & 19133.00 & 185582.06 & 1.02 & 0.95 & 0.97 \\
38044 & 107196 & 2005 & 19.90 & 0.02 & 1982.00 & 19200.52 & 1.00 & 0.96 & 0.97 \\
64147 & 500589 & 2005 & 3389.30 & 1.25 & 338756.00 & 3384209.07 & 1.00 & 1.00 & 1.00 \\
17345 & 102282 & 2005 & 539.80 & 0.08 & 53796.00 & 484587.32 & 1.00 & 0.90 & 0.90 \\
40907 & 108163 & 2005 & 167.70 & 0.02 & 17729.00 & 152127.59 & 0.95 & 0.91 & 0.86 \\
38135 & 107204 & 2005 & 405.90 & -0.04 & 36491.00 & 334018.38 & 1.11 & 0.82 & 0.92 \\
54229 & 364947 & 2005 & 32.00 & 0.02 & 3225.00 & 29345.57 & 0.99 & 0.92 & 0.91 \\
38236 & 107224 & 2005 & 62.10 & 0.05 & 6583.00 & 66305.22 & 0.94 & 1.07 & 1.01 \\
14316 & 101849 & 2005 & 15.20 & 0.12 & 1434.00 & 14097.60 & 1.06 & 0.93 & 0.98 \\
40855 & 108159 & 2005 & 37.00 & 0.06 & 3707.00 & 37549.99 & 1.00 & 1.01 & 1.01 \\
73478 & 600391 & 2005 & 631.70 & 0.07 & 58341.00 & 607370.81 & 1.08 & 0.96 & 1.04 \\
44703 & 109366 & 2005 & 60.50 & 0.10 & 5984.00 & 59625.96 & 1.01 & 0.99 & 1.00 \\
45649 & 200087 & 2005 & 18.50 & 0.10 & 2065.00 & 17168.82 & 0.90 & 0.93 & 0.83 \\
26655 & 103595 & 2005 & 167.50 & -0.02 & 16144.00 & 157386.15 & 1.04 & 0.94 & 0.97 \\
40879 & 108160 & 2005 & 73.90 & 0.07 & 6870.00 & 68331.69 & 1.08 & 0.92 & 0.99 \\
45671 & 200088 & 2005 & 68.30 & 0.06 & 6838.00 & 66064.04 & 1.00 & 0.97 & 0.97 \\
23631 & 103204 & 2005 & 220.20 & 0.09 & 22087.00 & 213775.76 & 1.00 & 0.97 & 0.97 \\
38215 & 107222 & 2005 & 1787.50 & 0.10 & 174307.00 & 1679613.00 & 1.03 & 0.94 & 0.96 \\
48989 & 240197 & 2005 & 53.00 & 0.05 & 5405.00 & 52302.69 & 0.98 & 0.99 & 0.97 \\
38190 & 107215 & 2005 & 695.60 & 0.13 & 71237.00 & 662820.13 & 0.98 & 0.95 & 0.93 \\
45680 & 200089 & 2005 & 99.60 & -0.08 & 10184.00 & 101590.98 & 0.98 & 1.02 & 1.00 \\
49587 & 240319 & 2005 & 474.70 & 0.17 & 48128.00 & 456542.73 & 0.99 & 0.96 & 0.95 \\
45683 & 200090 & 2005 & 23.50 & 0.03 & 2367.00 & 19449.92 & 0.99 & 0.83 & 0.82 \\
40896 & 108161 & 2005 & 210.50 & 0.05 & 21472.00 & 215769.70 & 0.98 & 1.03 & 1.00 \\
26699 & 103600 & 2005 & 335.00 & -0.00 & 32216.00 & 284365.25 & 1.04 & 0.85 & 0.88 \\
38161 & 107209 & 2005 & 318.00 & 0.04 & 30361.00 & 304509.06 & 1.05 & 0.96 & 1.00 \\
58650 & 410184 & 2005 & 10.60 & 0.04 & 855.00 & 7757.55 & 1.24 & 0.73 & 0.91 \\
26669 & 103597 & 2005 & 3244.00 & -0.03 & 352517.00 & 2692643.81 & 0.92 & 0.83 & 0.76 \\
636 & 100085 & 2005 & 5162.90 & 0.05 & 546612.00 & 5082297.89 & 0.94 & 0.98 & 0.93 \\
64016 & 500577 & 2005 & 904.70 & 0.03 & 98935.00 & 852120.54 & 0.91 & 0.94 & 0.86 \\
48146 & 240027 & 2005 & 472.90 & 0.06 & 70847.00 & 571861.04 & 0.67 & 1.21 & 0.81 \\
63808 & 500556 & 2005 & 216.50 & 0.06 & 20680.00 & 205633.05 & 1.05 & 0.95 & 0.99 \\
46031 & 200178 & 2005 & 35.30 & 0.07 & 3803.00 & 38199.89 & 0.93 & 1.08 & 1.00 \\
63780 & 500554 & 2005 & 440.40 & 0.05 & 41573.00 & 400737.03 & 1.06 & 0.91 & 0.96 \\
46037 & 200179 & 2005 & 75.60 & 0.06 & 6660.00 & 66056.31 & 1.14 & 0.87 & 0.99 \\
11405 & 101400 & 2005 & 246.90 & 0.05 & 29584.00 & 246135.95 & 0.83 & 1.00 & 0.83 \\
41366 & 108733 & 2005 & 27.40 & 0.10 & 2006.00 & 19648.56 & 1.37 & 0.72 & 0.98 \\
22866 & 103073 & 2005 & 374.90 & 0.05 & 37443.00 & 372499.17 & 1.00 & 0.99 & 0.99 \\
27800 & 105331 & 2005 & 21.70 & 0.05 & 1932.00 & 17794.36 & 1.12 & 0.82 & 0.92 \\
63775 & 500553 & 2005 & 146.80 & -0.00 & 14202.00 & 142412.28 & 1.03 & 0.97 & 1.00 \\
47772 & 221210 & 2005 & 131.80 & 0.05 & 12954.00 & 124354.16 & 1.02 & 0.94 & 0.96 \\
5451 & 100763 & 2005 & 1295.80 & 0.05 & 161719.00 & 1359657.07 & 0.80 & 1.05 & 0.84 \\
37208 & 106708 & 2005 & 463.80 & 0.05 & 48427.00 & 481263.78 & 0.96 & 1.04 & 0.99 \\
37182 & 106707 & 2005 & 336.50 & 0.10 & 35973.00 & 354412.10 & 0.94 & 1.05 & 0.99 \\
12887 & 101603 & 2005 & 1888.00 & 0.09 & 205074.00 & 2130120.14 & 0.92 & 1.13 & 1.04 \\
13941 & 101788 & 2005 & 417.10 & 0.04 & 47802.00 & 417717.13 & 0.87 & 1.00 & 0.87 \\
55507 & 400114 & 2005 & 585.80 & 0.13 & 58575.00 & 546280.76 & 1.00 & 0.93 & 0.93 \\
27829 & 105332 & 2005 & 66.80 & 0.10 & 6293.00 & 62417.90 & 1.06 & 0.93 & 0.99 \\
41387 & 108742 & 2005 & 26.50 & 0.07 & 4221.00 & 38484.36 & 0.63 & 1.45 & 0.91 \\
49930 & 240376 & 2005 & 60.70 & 0.15 & 6067.00 & 59913.27 & 1.00 & 0.99 & 0.99 \\
37168 & 106706 & 2005 & 128.50 & 0.07 & 15365.00 & 124888.73 & 0.84 & 0.97 & 0.81 \\
22834 & 103067 & 2005 & 82.40 & 0.10 & 8202.00 & 76511.85 & 1.00 & 0.93 & 0.93 \\
5473 & 100764 & 2005 & 690.70 & 0.08 & 74718.00 & 595501.77 & 0.92 & 0.86 & 0.80 \\
49953 & 240377 & 2005 & 31.00 & 0.02 & 3079.00 & 30132.06 & 1.01 & 0.97 & 0.98 \\
63735 & 500550 & 2005 & 58539.20 & 0.05 & 4699967.00 & 47587373.56 & 1.25 & 0.81 & 1.01 \\
46074 & 200183 & 2005 & 15.00 & 0.12 & 1441.00 & 13138.55 & 1.04 & 0.88 & 0.91 \\
2511 & 100336 & 2005 & 130.30 & -0.03 & 13034.00 & 123574.48 & 1.00 & 0.95 & 0.95 \\
37221 & 106710 & 2005 & 43.90 & 0.03 & 4411.00 & 44614.44 & 1.00 & 1.02 & 1.01 \\
27758 & 105321 & 2005 & 843.50 & 0.01 & 74449.00 & 722975.89 & 1.13 & 0.86 & 0.97 \\
54448 & 367985 & 2005 & 221.10 & 0.10 & 21238.00 & 206897.94 & 1.04 & 0.94 & 0.97 \\
45963 & 200174 & 2005 & 484.00 & 0.05 & 48308.00 & 468477.49 & 1.00 & 0.97 & 0.97 \\
27669 & 105309 & 2005 & 4172.70 & 0.15 & 384420.00 & 3537510.87 & 1.09 & 0.85 & 0.92 \\
37318 & 106729 & 2005 & 1683.40 & 0.07 & 169405.00 & 1673567.26 & 0.99 & 0.99 & 0.99 \\
63859 & 500561 & 2005 & 32.90 & 0.09 & 3301.00 & 32839.49 & 1.00 & 1.00 & 0.99 \\
16973 & 102224 & 2005 & 10638.20 & 0.07 & 1064977.00 & 10606921.35 & 1.00 & 1.00 & 1.00 \\
13960 & 101789 & 2005 & 187.10 & 0.05 & 21191.00 & 192568.75 & 0.88 & 1.03 & 0.91 \\
37291 & 106726 & 2005 & 1460.30 & 0.04 & 156390.00 & 1578791.26 & 0.93 & 1.08 & 1.01 \\
63848 & 500560 & 2005 & 23.30 & 0.02 & 2353.00 & 23255.28 & 0.99 & 1.00 & 0.99 \\
49859 & 240368 & 2005 & 402.90 & 0.01 & 45881.00 & 435079.39 & 0.88 & 1.08 & 0.95 \\
14474 & 101861 & 2005 & 1939.80 & 0.06 & 201100.00 & 1861596.84 & 0.96 & 0.96 & 0.93 \\
49869 & 240369 & 2005 & 10.70 & 0.08 & 931.00 & 9280.97 & 1.15 & 0.87 & 1.00 \\
54457 & 367992 & 2005 & 94.80 & 0.04 & 13829.00 & 137517.47 & 0.69 & 1.45 & 0.99 \\
11439 & 101402 & 2005 & 146.30 & 0.01 & 14627.00 & 140413.28 & 1.00 & 0.96 & 0.96 \\
41394 & 108745 & 2005 & 70.50 & 0.10 & 7022.00 & 69735.28 & 1.00 & 0.99 & 0.99 \\
63829 & 500559 & 2005 & 68.20 & 0.11 & 6274.00 & 61830.12 & 1.09 & 0.91 & 0.99 \\
27715 & 105317 & 2005 & 156.50 & 0.12 & 15973.00 & 143295.40 & 0.98 & 0.92 & 0.90 \\
37279 & 106725 & 2005 & 1.90 & 0.03 & 200.00 & 1959.95 & 0.95 & 1.03 & 0.98 \\
41353 & 108732 & 2005 & 147.80 & -0.02 & 14762.00 & 129972.74 & 1.00 & 0.88 & 0.88 \\
27732 & 105320 & 2005 & 371.70 & 0.06 & 37223.00 & 320174.99 & 1.00 & 0.86 & 0.86 \\
22897 & 103084 & 2005 & 519.30 & 0.11 & 53545.00 & 532630.73 & 0.97 & 1.03 & 0.99 \\
8434 & 101086 & 2005 & 438.20 & 0.04 & 45390.00 & 446004.07 & 0.97 & 1.02 & 0.98 \\
46004 & 200176 & 2005 & 150.60 & 0.33 & 13920.00 & 123955.98 & 1.08 & 0.82 & 0.89 \\
37253 & 106724 & 2005 & 2148.00 & 0.08 & 211597.00 & 1992041.18 & 1.02 & 0.93 & 0.94 \\
54464 & 368366 & 2005 & 108.60 & 0.03 & 10845.00 & 107878.35 & 1.00 & 0.99 & 0.99 \\
53541 & 351589 & 2005 & 5.50 & 0.04 & 583.00 & 5779.85 & 0.94 & 1.05 & 0.99 \\
529 & 100072 & 2005 & 5398.70 & 0.08 & 554813.00 & 5446194.36 & 0.97 & 1.01 & 0.98 \\
46025 & 200177 & 2005 & 36.80 & 0.06 & 3949.00 & 40518.22 & 0.93 & 1.10 & 1.03 \\
18660 & 102500 & 2005 & 31.10 & 0.06 & 3061.00 & 29873.69 & 1.02 & 0.96 & 0.98 \\
41328 & 108728 & 2005 & 288.00 & 0.05 & 17843.00 & 173835.85 & 1.61 & 0.60 & 0.97 \\
45957 & 200173 & 2005 & 34.70 & 0.06 & 3566.00 & 36278.99 & 0.97 & 1.05 & 1.02 \\
49975 & 240379 & 2005 & 6.20 & 0.04 & 621.00 & 6131.96 & 1.00 & 0.99 & 0.99 \\
53465 & 350572 & 2005 & 57.30 & 0.11 & 5329.00 & 54738.51 & 1.08 & 0.96 & 1.03 \\
49979 & 240380 & 2005 & 5.50 & -0.03 & 332.00 & 3178.35 & 1.66 & 0.58 & 0.96 \\
27988 & 105364 & 2005 & 139.30 & -0.01 & 14030.00 & 139616.95 & 0.99 & 1.00 & 1.00 \\
5511 & 100769 & 2005 & 7461.90 & 0.08 & 759078.00 & 7282256.45 & 0.98 & 0.98 & 0.96 \\
47750 & 221051 & 2005 & 4741.60 & 0.11 & 465639.00 & 4381437.25 & 1.02 & 0.92 & 0.94 \\
49983 & 240381 & 2005 & 47.20 & 0.13 & 4693.00 & 45646.74 & 1.01 & 0.97 & 0.97 \\
11321 & 101393 & 2005 & 628.20 & 0.02 & 69070.00 & 662892.48 & 0.91 & 1.06 & 0.96 \\
12920 & 101606 & 2005 & 3262.80 & 0.08 & 321426.00 & 3131026.93 & 1.02 & 0.96 & 0.97 \\
37065 & 106655 & 2005 & 22.30 & -0.05 & 2235.00 & 22476.43 & 1.00 & 1.01 & 1.01 \\
2617 & 100347 & 2005 & 777.20 & 0.04 & 77833.00 & 768909.99 & 1.00 & 0.99 & 0.99 \\
46101 & 200189 & 2005 & 341.10 & 0.07 & 33706.00 & 288953.89 & 1.01 & 0.85 & 0.86 \\
48908 & 240152 & 2005 & 188.00 & 0.07 & 18488.00 & 179913.54 & 1.02 & 0.96 & 0.97 \\
54487 & 372487 & 2005 & 18.10 & 0.09 & 1783.00 & 17825.34 & 1.02 & 0.98 & 1.00 \\
28020 & 105369 & 2005 & 157.40 & 0.06 & 15494.00 & 162988.68 & 1.02 & 1.04 & 1.05 \\
16823 & 102193 & 2005 & 148.10 & 0.07 & 14132.00 & 141306.90 & 1.05 & 0.95 & 1.00 \\
41489 & 108761 & 2005 & 404.10 & 0.08 & 41556.00 & 397726.56 & 0.97 & 0.98 & 0.96 \\
37039 & 106654 & 2005 & 501.80 & 0.04 & 50941.00 & 510441.28 & 0.99 & 1.02 & 1.00 \\
41511 & 108762 & 2005 & 124.40 & 0.17 & 6318.00 & 117270.10 & 1.97 & 0.94 & 1.86 \\
50005 & 240382 & 2005 & 40.60 & 0.09 & 4263.00 & 39226.38 & 0.95 & 0.97 & 0.92 \\
50016 & 240383 & 2005 & 171.22 & 0.09 & 16938.00 & 165019.18 & 1.01 & 0.96 & 0.97 \\
41536 & 108764 & 2005 & 368.50 & 0.04 & 37517.00 & 372276.35 & 0.98 & 1.01 & 0.99 \\
28049 & 105370 & 2005 & 80.10 & 0.14 & 7675.00 & 82489.43 & 1.04 & 1.03 & 1.07 \\
37020 & 106650 & 2005 & 725.20 & 0.03 & 69291.00 & 692729.39 & 1.05 & 0.96 & 1.00 \\
22686 & 103028 & 2005 & 5933.90 & 0.04 & 598049.00 & 5366505.60 & 0.99 & 0.90 & 0.90 \\
37091 & 106675 & 2005 & 131.10 & 0.13 & 13104.00 & 128267.43 & 1.00 & 0.98 & 0.98 \\
11331 & 101394 & 2005 & 26.30 & 0.07 & 2610.00 & 26529.18 & 1.01 & 1.01 & 1.02 \\
27956 & 105358 & 2005 & 5121.90 & 0.06 & 502888.00 & 4714414.15 & 1.02 & 0.92 & 0.94 \\
41403 & 108749 & 2005 & 498.00 & 0.04 & 68027.00 & 423383.37 & 0.73 & 0.85 & 0.62 \\
11377 & 101399 & 2005 & 82.20 & 0.05 & 8700.00 & 81299.83 & 0.94 & 0.99 & 0.93 \\
41428 & 108752 & 2005 & 69.40 & 0.02 & 7253.00 & 69814.38 & 0.96 & 1.01 & 0.96 \\
6916 & 100968 & 2005 & 551.00 & 0.02 & 55532.00 & 485238.95 & 0.99 & 0.88 & 0.87 \\
37132 & 106692 & 2005 & 2862.90 & 0.04 & 286231.00 & 2661658.07 & 1.00 & 0.93 & 0.93 \\
22799 & 103065 & 2005 & 218.00 & 0.15 & 21830.00 & 209052.99 & 1.00 & 0.96 & 0.96 \\
48185 & 240040 & 2005 & 651.60 & 0.45 & 60743.00 & 629326.44 & 1.07 & 0.97 & 1.04 \\
54482 & 372363 & 2005 & 12.10 & 0.06 & 1251.00 & 12605.67 & 0.97 & 1.04 & 1.01 \\
7708 & 101055 & 2005 & 27415.00 & 0.14 & 2681315.00 & 26891962.46 & 1.02 & 0.98 & 1.00 \\
16882 & 102213 & 2005 & 579.30 & 0.03 & 57952.00 & 579480.86 & 1.00 & 1.00 & 1.00 \\
53517 & 351459 & 2005 & 898.80 & 0.08 & 102236.00 & 781665.79 & 0.88 & 0.87 & 0.76 \\
2545 & 100343 & 2005 & 643.20 & 0.07 & 79248.00 & 652706.75 & 0.81 & 1.01 & 0.82 \\
18700 & 102503 & 2005 & 292.40 & 0.03 & 27647.00 & 290878.87 & 1.06 & 0.99 & 1.05 \\
27914 & 105346 & 2005 & 1937.90 & 0.05 & 192449.00 & 1896419.58 & 1.01 & 0.98 & 0.99 \\
27866 & 105335 & 2005 & 233.70 & 0.12 & 23367.00 & 217445.38 & 1.00 & 0.93 & 0.93 \\
44583 & 109343 & 2005 & 255.90 & 0.03 & 25737.00 & 227606.07 & 0.99 & 0.89 & 0.88 \\
8467 & 101087 & 2005 & 1020.30 & 0.30 & 101202.00 & 841800.92 & 1.01 & 0.83 & 0.83 \\
14518 & 101871 & 2005 & 365.20 & 0.10 & 36145.00 & 361448.63 & 1.01 & 0.99 & 1.00 \\
41440 & 108759 & 2005 & 168.10 & 0.07 & 16707.00 & 160588.03 & 1.01 & 0.96 & 0.96 \\
27937 & 105353 & 2005 & 125.70 & 0.05 & 12567.00 & 125389.32 & 1.00 & 1.00 & 1.00 \\
37101 & 106678 & 2005 & 43.90 & 0.07 & 4723.00 & 43800.54 & 0.93 & 1.00 & 0.93 \\
41464 & 108760 & 2005 & 166.80 & 0.06 & 16903.00 & 168987.41 & 0.99 & 1.01 & 1.00 \\
493 & 100071 & 2005 & 2432.60 & 0.05 & 253772.00 & 2189641.46 & 0.96 & 0.90 & 0.86 \\
46097 & 200184 & 2005 & 8.70 & 0.07 & 830.00 & 7992.60 & 1.05 & 0.92 & 0.96 \\
16852 & 102197 & 2005 & 1342.20 & 0.08 & 114074.00 & 1213547.86 & 1.18 & 0.90 & 1.06 \\
44574 & 109341 & 2005 & 199.50 & 0.05 & 21634.00 & 214966.98 & 0.92 & 1.08 & 0.99 \\
55530 & 400116 & 2005 & 69.50 & 0.04 & 6420.00 & 60445.43 & 1.08 & 0.87 & 0.94 \\
23765 & 103212 & 2005 & 2246.20 & 0.07 & 225513.00 & 1970836.47 & 1.00 & 0.88 & 0.87 \\
22953 & 103090 & 2005 & 874.90 & 0.06 & 89423.00 & 881439.45 & 0.98 & 1.01 & 0.99 \\
37364 & 106737 & 2005 & 145.40 & -0.90 & 14395.00 & 143925.73 & 1.01 & 0.99 & 1.00 \\
45886 & 200153 & 2005 & 43.30 & -0.03 & 4325.00 & 42986.22 & 1.00 & 0.99 & 0.99 \\
58989 & 410242 & 2005 & 24.00 & 0.19 & 2587.00 & 22689.52 & 0.93 & 0.95 & 0.88 \\
37604 & 106972 & 2005 & 69.00 & 0.04 & 7106.00 & 71058.33 & 0.97 & 1.03 & 1.00 \\
17088 & 102255 & 2005 & 17.30 & 0.10 & 1732.00 & 17130.58 & 1.00 & 0.99 & 0.99 \\
63939 & 500568 & 2005 & 56.10 & 0.12 & 6170.00 & 53396.53 & 0.91 & 0.95 & 0.87 \\
41115 & 108202 & 2005 & 110.60 & 0.04 & 10053.00 & 100169.37 & 1.10 & 0.91 & 1.00 \\
23147 & 103134 & 2005 & 384.10 & 0.04 & 38680.00 & 380009.23 & 0.99 & 0.99 & 0.98 \\
27376 & 105275 & 2005 & 177.30 & 0.05 & 17777.00 & 174536.03 & 1.00 & 0.98 & 0.98 \\
37579 & 106969 & 2005 & 35.20 & 0.03 & 4144.00 & 33525.09 & 0.85 & 0.95 & 0.81 \\
54407 & 367600 & 2005 & 113.70 & 0.10 & 11351.00 & 108816.68 & 1.00 & 0.96 & 0.96 \\
37554 & 106968 & 2005 & 43.20 & 0.02 & 4566.00 & 45441.73 & 0.95 & 1.05 & 1.00 \\
14453 & 101858 & 2005 & 1664.00 & 0.07 & 184759.00 & 1352219.76 & 0.90 & 0.81 & 0.73 \\
4494 & 100635 & 2005 & 730.80 & 0.07 & 70783.00 & 711456.99 & 1.03 & 0.97 & 1.01 \\
11548 & 101427 & 2005 & 107.10 & 0.32 & 10750.00 & 86097.87 & 1.00 & 0.80 & 0.80 \\
41138 & 108203 & 2005 & 171.90 & 0.04 & 15711.00 & 144064.17 & 1.09 & 0.84 & 0.92 \\
27406 & 105276 & 2005 & 3586.20 & 0.06 & 358622.00 & 3217708.96 & 1.00 & 0.90 & 0.90 \\
41161 & 108211 & 2005 & 739.90 & 0.07 & 73114.00 & 700015.07 & 1.01 & 0.95 & 0.96 \\
5360 & 100757 & 2005 & 19.70 & 0.10 & 1903.00 & 18338.72 & 1.04 & 0.93 & 0.96 \\
55490 & 400107 & 2005 & 26.20 & 0.05 & 2593.00 & 25932.53 & 1.01 & 0.99 & 1.00 \\
55208 & 400072 & 2005 & 402.90 & 0.08 & 40327.00 & 398399.51 & 1.00 & 0.99 & 0.99 \\
54429 & 367713 & 2005 & 6.20 & 0.07 & 407.00 & 3714.59 & 1.52 & 0.60 & 0.91 \\
47803 & 221485 & 2005 & 352.10 & 0.07 & 35381.00 & 352033.22 & 1.00 & 1.00 & 0.99 \\
53576 & 354018 & 2005 & 132.40 & 0.04 & 13221.00 & 128587.64 & 1.00 & 0.97 & 0.97 \\
74651 & 601147 & 2005 & 154.40 & 0.14 & 18633.00 & 169317.64 & 0.83 & 1.10 & 0.91 \\
55497 & 400113 & 2005 & 124.20 & 0.07 & 10180.00 & 99875.45 & 1.22 & 0.80 & 0.98 \\
23107 & 103122 & 2005 & 198.90 & 0.15 & 18846.00 & 195486.36 & 1.06 & 0.98 & 1.04 \\
27435 & 105278 & 2005 & 536.40 & 0.17 & 50824.00 & 484922.58 & 1.06 & 0.90 & 0.95 \\
37527 & 106948 & 2005 & 389.20 & 0.06 & 38931.00 & 371620.30 & 1.00 & 0.95 & 0.95 \\
27343 & 105269 & 2005 & 877.70 & -0.02 & 88407.00 & 864495.11 & 0.99 & 0.98 & 0.98 \\
11565 & 101430 & 2005 & 155.70 & 0.06 & 19610.00 & 159701.26 & 0.79 & 1.03 & 0.81 \\
49766 & 240358 & 2005 & 49.80 & 0.11 & 4517.00 & 45607.68 & 1.10 & 0.92 & 1.01 \\
27251 & 105259 & 2005 & 403.80 & 0.03 & 40170.00 & 401275.10 & 1.01 & 0.99 & 1.00 \\
37686 & 106995 & 2005 & 3232.90 & 0.07 & 316506.00 & 2971246.97 & 1.02 & 0.92 & 0.94 \\
47817 & 222027 & 2005 & 1727.30 & 0.02 & 172656.00 & 1714189.49 & 1.00 & 0.99 & 0.99 \\
37675 & 106993 & 2005 & 294.90 & 0.08 & 26279.00 & 268841.16 & 1.12 & 0.91 & 1.02 \\
23222 & 103145 & 2005 & 45.30 & 0.06 & 4537.00 & 43488.43 & 1.00 & 0.96 & 0.96 \\
12788 & 101595 & 2005 & 954.20 & 0.09 & 121156.00 & 1081068.50 & 0.79 & 1.13 & 0.89 \\
2371 & 100320 & 2005 & 68.70 & -0.03 & 8144.00 & 67168.58 & 0.84 & 0.98 & 0.82 \\
8364 & 101084 & 2005 & 1867.90 & 0.05 & 191902.00 & 1707216.40 & 0.97 & 0.91 & 0.89 \\
27280 & 105260 & 2005 & 378.30 & 0.14 & 38195.00 & 357455.84 & 0.99 & 0.94 & 0.94 \\
37664 & 106992 & 2005 & 151.70 & 0.06 & 17717.00 & 179954.59 & 0.86 & 1.19 & 1.02 \\
5329 & 100753 & 2005 & 1374.20 & 0.03 & 151973.00 & 1421307.76 & 0.90 & 1.03 & 0.94 \\
5374 & 100758 & 2005 & 9.80 & 0.05 & 962.00 & 9729.96 & 1.02 & 0.99 & 1.01 \\
37651 & 106984 & 2005 & 86.90 & -0.07 & 12650.00 & 123140.31 & 0.69 & 1.42 & 0.97 \\
55464 & 400100 & 2005 & 272.20 & 0.08 & 25317.00 & 242722.71 & 1.08 & 0.89 & 0.96 \\
53588 & 354930 & 2005 & 1343.40 & 0.11 & 128507.00 & 1285072.74 & 1.05 & 0.96 & 1.00 \\
49748 & 240353 & 2005 & 1.20 & 0.11 & 120.00 & 1160.71 & 1.00 & 0.97 & 0.97 \\
23192 & 103144 & 2005 & 21.70 & 0.09 & 1974.00 & 19135.29 & 1.10 & 0.88 & 0.97 \\
17111 & 102257 & 2005 & 2406.80 & 0.05 & 241270.00 & 2407595.68 & 1.00 & 1.00 & 1.00 \\
58925 & 410235 & 2005 & 2.70 & 0.17 & 276.00 & 2871.01 & 0.98 & 1.06 & 1.04 \\
27318 & 105268 & 2005 & 614.20 & 0.09 & 64418.00 & 640982.90 & 0.95 & 1.04 & 1.00 \\
41103 & 108200 & 2005 & 83.60 & 0.09 & 8414.00 & 72977.42 & 0.99 & 0.87 & 0.87 \\
13983 & 101794 & 2005 & 946.50 & 0.03 & 94708.00 & 907632.04 & 1.00 & 0.96 & 0.96 \\
6945 & 100973 & 2005 & 58.30 & 0.06 & 5846.00 & 57370.25 & 1.00 & 0.98 & 0.98 \\
41097 & 108197 & 2005 & 57.20 & 0.09 & 5466.00 & 54145.86 & 1.05 & 0.95 & 0.99 \\
55180 & 400066 & 2005 & 158.30 & 0.05 & 15856.00 & 158563.30 & 1.00 & 1.00 & 1.00 \\
41190 & 108670 & 2005 & 224.20 & 0.09 & 22438.00 & 216318.09 & 1.00 & 0.96 & 0.96 \\
8402 & 101085 & 2005 & 275.70 & 0.09 & 28377.00 & 279640.31 & 0.97 & 1.01 & 0.99 \\
41282 & 108719 & 2005 & 348.00 & 0.10 & 34942.00 & 344163.23 & 1.00 & 0.99 & 0.98 \\
45919 & 200167 & 2005 & 205.30 & 0.03 & 21338.00 & 198853.75 & 0.96 & 0.97 & 0.93 \\
17009 & 102230 & 2005 & 21.50 & 0.02 & 2254.00 & 22427.63 & 0.95 & 1.04 & 1.00 \\
44621 & 109348 & 2005 & 757.20 & 0.05 & 79599.00 & 772340.23 & 0.95 & 1.02 & 0.97 \\
23005 & 103101 & 2005 & 228.30 & 0.18 & 24788.00 & 201807.30 & 0.92 & 0.88 & 0.81 \\
44598 & 109347 & 2005 & 1000.60 & 0.08 & 100448.00 & 972269.53 & 1.00 & 0.97 & 0.97 \\
37418 & 106869 & 2005 & 139.10 & 0.01 & 12212.00 & 131362.60 & 1.14 & 0.94 & 1.08 \\
5407 & 100760 & 2005 & 596.70 & 0.10 & 66797.00 & 534703.76 & 0.89 & 0.90 & 0.80 \\
550 & 100075 & 2005 & 1568.00 & 0.05 & 160573.00 & 1558393.98 & 0.98 & 0.99 & 0.97 \\
41307 & 108723 & 2005 & 222.20 & 0.04 & 22730.00 & 229159.63 & 0.98 & 1.03 & 1.01 \\
22987 & 103100 & 2005 & 214.00 & 0.08 & 24533.00 & 206907.98 & 0.87 & 0.97 & 0.84 \\
49827 & 240365 & 2005 & 42.70 & 0.15 & 3779.00 & 35779.67 & 1.13 & 0.84 & 0.95 \\
45924 & 200168 & 2005 & 377.40 & 0.09 & 37561.00 & 357863.29 & 1.00 & 0.95 & 0.95 \\
2446 & 100330 & 2005 & 2900.80 & 0.12 & 268536.00 & 2671880.41 & 1.08 & 0.92 & 0.99 \\
27611 & 105303 & 2005 & 117.40 & 0.13 & 12664.00 & 118245.03 & 0.93 & 1.01 & 0.93 \\
37392 & 106747 & 2005 & 150.70 & 0.00 & 15203.00 & 147572.58 & 0.99 & 0.98 & 0.97 \\
37381 & 106742 & 2005 & 149.90 & 0.08 & 14987.00 & 149086.32 & 1.00 & 0.99 & 0.99 \\
41317 & 108726 & 2005 & 200.70 & 0.04 & 22784.00 & 179375.47 & 0.88 & 0.89 & 0.79 \\
45945 & 200171 & 2005 & 82.00 & 0.07 & 8708.00 & 88801.20 & 0.94 & 1.08 & 1.02 \\
22968 & 103099 & 2005 & 93.40 & 0.04 & 9131.00 & 90696.45 & 1.02 & 0.97 & 0.99 \\
45951 & 200172 & 2005 & 17.50 & 0.13 & 1947.00 & 19869.92 & 0.90 & 1.14 & 1.02 \\
27640 & 105306 & 2005 & 19.30 & 0.11 & 2444.00 & 18503.05 & 0.79 & 0.96 & 0.76 \\
63869 & 500562 & 2005 & 29.20 & 0.09 & 2942.00 & 29147.13 & 0.99 & 1.00 & 0.99 \\
11462 & 101414 & 2005 & 31.10 & 0.11 & 3389.00 & 29788.96 & 0.92 & 0.96 & 0.88 \\
27547 & 105287 & 2005 & 88.00 & -0.01 & 8794.00 & 83329.53 & 1.00 & 0.95 & 0.95 \\
41270 & 108710 & 2005 & 294.20 & -0.05 & 29475.00 & 275103.86 & 1.00 & 0.94 & 0.93 \\
45914 & 200164 & 2005 & 28.90 & -0.03 & 2889.00 & 27813.76 & 1.00 & 0.96 & 0.96 \\
53548 & 351713 & 2005 & 404.10 & 0.13 & 39346.00 & 389564.61 & 1.03 & 0.96 & 0.99 \\
54433 & 367841 & 2005 & 687.20 & 0.01 & 78007.00 & 711493.31 & 0.88 & 1.04 & 0.91 \\
601 & 100079 & 2005 & 2358.70 & 0.11 & 225504.00 & 2128367.07 & 1.05 & 0.90 & 0.94 \\
41215 & 108673 & 2005 & 169.20 & 0.03 & 17623.00 & 158579.42 & 0.96 & 0.94 & 0.90 \\
49777 & 240359 & 2005 & 27.70 & 0.08 & 3000.00 & 24976.03 & 0.92 & 0.90 & 0.83 \\
17045 & 102231 & 2005 & 448.50 & 0.05 & 45848.00 & 442632.76 & 0.98 & 0.99 & 0.97 \\
2403 & 100322 & 2005 & 354.20 & 0.01 & 41680.00 & 367900.38 & 0.85 & 1.04 & 0.88 \\
18595 & 102490 & 2005 & 57.50 & 0.05 & 5722.00 & 57211.14 & 1.00 & 0.99 & 1.00 \\
63880 & 500563 & 2005 & 35.70 & 0.12 & 3586.00 & 35427.64 & 1.00 & 0.99 & 0.99 \\
49786 & 240360 & 2005 & 1416.60 & 0.09 & 150314.00 & 1239057.61 & 0.94 & 0.87 & 0.82 \\
23079 & 103110 & 2005 & 2264.90 & 0.07 & 211631.00 & 1923079.64 & 1.07 & 0.85 & 0.91 \\
44653 & 109351 & 2005 & 56.00 & 0.08 & 8005.00 & 79880.10 & 0.70 & 1.43 & 1.00 \\
27480 & 105281 & 2005 & 634.00 & 0.12 & 63558.00 & 586197.00 & 1.00 & 0.92 & 0.92 \\
37492 & 106934 & 2005 & 1649.80 & 0.03 & 165179.00 & 1573310.91 & 1.00 & 0.95 & 0.95 \\
45894 & 200156 & 2005 & 65.30 & 0.08 & 5933.00 & 56441.24 & 1.10 & 0.86 & 0.95 \\
37483 & 106931 & 2005 & 364.50 & 0.00 & 36052.00 & 360520.89 & 1.01 & 0.99 & 1.00 \\
11511 & 101425 & 2005 & 20.50 & 0.12 & 2034.00 & 17502.21 & 1.01 & 0.85 & 0.86 \\
12830 & 101601 & 2005 & 155.80 & 0.08 & 15344.00 & 156018.56 & 1.02 & 1.00 & 1.02 \\
54441 & 367842 & 2005 & 199.50 & 0.08 & 19992.00 & 200812.71 & 1.00 & 1.01 & 1.00 \\
53567 & 351891 & 2005 & 22.70 & 0.06 & 2200.00 & 22001.10 & 1.03 & 0.97 & 1.00 \\
37458 & 106919 & 2005 & 28.00 & 0.08 & 2678.00 & 27534.31 & 1.05 & 0.98 & 1.03 \\
7742 & 101056 & 2005 & 31082.50 & 0.03 & 3070041.00 & 30676010.08 & 1.01 & 0.99 & 1.00 \\
2423 & 100323 & 2005 & 1866.20 & 0.03 & 193791.00 & 1946922.25 & 0.96 & 1.04 & 1.00 \\
18611 & 102491 & 2005 & 524.40 & 0.06 & 57640.00 & 570776.69 & 0.91 & 1.09 & 0.99 \\
576 & 100076 & 2005 & 389.50 & 0.05 & 39519.00 & 396157.58 & 0.99 & 1.02 & 1.00 \\
44645 & 109350 & 2005 & 43.40 & 0.05 & 6663.00 & 66902.73 & 0.65 & 1.54 & 1.00 \\
12843 & 101602 & 2005 & 5418.40 & 0.13 & 533952.00 & 5082581.21 & 1.01 & 0.94 & 0.95 \\
55187 & 400069 & 2005 & 48.40 & -0.11 & 4852.00 & 48469.02 & 1.00 & 1.00 & 1.00 \\
26406 & 103579 & 2005 & 607.10 & 0.12 & 58470.00 & 540489.28 & 1.04 & 0.89 & 0.92 \\
40692 & 108145 & 2005 & 214.50 & 0.48 & 21282.00 & 211723.69 & 1.01 & 0.99 & 0.99 \\
17442 & 102306 & 2005 & 33230.70 & 0.10 & 3125679.00 & 28090512.47 & 1.06 & 0.85 & 0.90 \\
49152 & 240234 & 2005 & 153.40 & 0.04 & 16343.00 & 160289.69 & 0.94 & 1.04 & 0.98 \\
64574 & 500610 & 2005 & 512.60 & 0.03 & 51256.00 & 511178.25 & 1.00 & 1.00 & 1.00 \\
39985 & 107967 & 2005 & 130.20 & 0.07 & 13546.00 & 128283.14 & 0.96 & 0.99 & 0.95 \\
25317 & 103466 & 2005 & 1376.40 & 0.07 & 138138.00 & 1317650.39 & 1.00 & 0.96 & 0.95 \\
49375 & 240286 & 2005 & 30.70 & 0.08 & 3064.00 & 31265.12 & 1.00 & 1.02 & 1.02 \\
39379 & 107677 & 2005 & 593.40 & 0.07 & 59868.00 & 574207.80 & 0.99 & 0.97 & 0.96 \\
4894 & 100691 & 2005 & 297.40 & 0.02 & 36772.00 & 347736.89 & 0.81 & 1.17 & 0.95 \\
39372 & 107673 & 2005 & 225.10 & 0.12 & 21819.00 & 206807.21 & 1.03 & 0.92 & 0.95 \\
44956 & 109404 & 2005 & 21.30 & 0.04 & 2091.00 & 19207.21 & 1.02 & 0.90 & 0.92 \\
24456 & 103327 & 2005 & 1390.30 & 0.07 & 134715.00 & 1325503.34 & 1.03 & 0.95 & 0.98 \\
12130 & 101511 & 2005 & 728.70 & 0.06 & 70310.00 & 739808.46 & 1.04 & 1.02 & 1.05 \\
39351 & 107672 & 2005 & 3.20 & 0.03 & 315.00 & 2853.00 & 1.02 & 0.89 & 0.91 \\
65067 & 500659 & 2005 & 787.10 & 0.49 & 78070.00 & 634213.51 & 1.01 & 0.81 & 0.81 \\
1373 & 100192 & 2005 & 66.00 & 0.07 & 6602.00 & 63402.49 & 1.00 & 0.96 & 0.96 \\
54017 & 363232 & 2005 & 775.60 & 0.09 & 72099.00 & 633706.93 & 1.08 & 0.82 & 0.88 \\
8015 & 101069 & 2005 & 7147.50 & 0.05 & 1295670.00 & 13014251.11 & 0.55 & 1.82 & 1.00 \\
53898 & 360021 & 2005 & 90.10 & 0.05 & 10736.00 & 107605.23 & 0.84 & 1.19 & 1.00 \\
55359 & 400090 & 2005 & 37.50 & 0.02 & 3721.00 & 36909.18 & 1.01 & 0.98 & 0.99 \\
39995 & 107968 & 2005 & 96.10 & 0.07 & 9954.00 & 93014.57 & 0.97 & 0.97 & 0.93 \\
25358 & 103478 & 2005 & 5505.60 & 0.10 & 548891.00 & 4878786.07 & 1.00 & 0.89 & 0.89 \\
44933 & 109402 & 2005 & 123.70 & 0.03 & 12382.00 & 123387.36 & 1.00 & 1.00 & 1.00 \\
39339 & 107670 & 2005 & 348.80 & 0.04 & 42370.00 & 417295.63 & 0.82 & 1.20 & 0.98 \\
24426 & 103326 & 2005 & 1628.00 & 0.09 & 151961.00 & 1555995.52 & 1.07 & 0.96 & 1.02 \\
55381 & 400092 & 2005 & 63.50 & 0.03 & 6695.00 & 53362.58 & 0.95 & 0.84 & 0.80 \\
58380 & 410151 & 2005 & 726.80 & 0.09 & 75208.00 & 723934.77 & 0.97 & 1.00 & 0.96 \\
25385 & 103481 & 2005 & 510.90 & 0.03 & 53471.00 & 539512.33 & 0.96 & 1.06 & 1.01 \\
12415 & 101539 & 2005 & 1177.10 & 0.04 & 120898.00 & 1156587.59 & 0.97 & 0.98 & 0.96 \\
17848 & 102365 & 2005 & 336.70 & 0.15 & 33670.00 & 328594.76 & 1.00 & 0.98 & 0.98 \\
39397 & 107691 & 2005 & 43.10 & 0.05 & 4318.00 & 40686.70 & 1.00 & 0.94 & 0.94 \\
24478 & 103328 & 2005 & 426.40 & 0.08 & 52345.00 & 417217.37 & 0.81 & 0.98 & 0.80 \\
12395 & 101538 & 2005 & 256.50 & 0.07 & 26215.00 & 272067.33 & 0.98 & 1.06 & 1.04 \\
25204 & 103460 & 2005 & 1145.20 & 0.07 & 114570.00 & 1065192.62 & 1.00 & 0.93 & 0.93 \\
12164 & 101513 & 2005 & 814.90 & 0.05 & 73142.00 & 693947.64 & 1.11 & 0.85 & 0.95 \\
39481 & 107702 & 2005 & 1245.60 & 0.06 & 124976.00 & 1248402.11 & 1.00 & 1.00 & 1.00 \\
1892 & 100247 & 2005 & 1024.10 & 0.15 & 102718.00 & 895314.49 & 1.00 & 0.87 & 0.87 \\
55351 & 400088 & 2005 & 112.90 & 0.40 & 11282.00 & 107617.28 & 1.00 & 0.95 & 0.95 \\
7088 & 100996 & 2005 & 1323.30 & 0.04 & 132205.00 & 1268622.09 & 1.00 & 0.96 & 0.96 \\
64636 & 500614 & 2005 & 177.30 & 0.03 & 17740.00 & 177220.66 & 1.00 & 1.00 & 1.00 \\
17876 & 102367 & 2005 & 137.10 & 0.04 & 13707.00 & 134601.37 & 1.00 & 0.98 & 0.98 \\
65026 & 500656 & 2005 & 1607.30 & 0.03 & 160782.00 & 1606999.13 & 1.00 & 1.00 & 1.00 \\
1454 & 100200 & 2005 & 700.20 & 0.16 & 69689.00 & 672969.01 & 1.00 & 0.96 & 0.97 \\
64597 & 500612 & 2005 & 381.70 & 0.04 & 38169.00 & 381247.37 & 1.00 & 1.00 & 1.00 \\
17817 & 102364 & 2005 & 1598.80 & 0.04 & 159880.00 & 1572913.59 & 1.00 & 0.98 & 0.98 \\
25242 & 103463 & 2005 & 394.10 & 0.39 & 30442.00 & 289187.75 & 1.29 & 0.73 & 0.95 \\
39460 & 107694 & 2005 & 13.00 & 0.10 & 1293.00 & 12266.21 & 1.01 & 0.94 & 0.95 \\
39936 & 107958 & 2005 & 46.40 & 0.10 & 4213.00 & 39754.83 & 1.10 & 0.86 & 0.94 \\
14137 & 101805 & 2005 & 660.10 & 0.01 & 66052.00 & 657104.59 & 1.00 & 1.00 & 0.99 \\
24508 & 103329 & 2005 & 374.20 & 0.09 & 47655.00 & 380671.66 & 0.79 & 1.02 & 0.80 \\
18135 & 102404 & 2005 & 1874.90 & 0.04 & 188375.00 & 1873410.34 & 1.00 & 1.00 & 0.99 \\
49343 & 240284 & 2005 & 100.30 & 0.05 & 9743.00 & 99366.92 & 1.03 & 0.99 & 1.02 \\
39961 & 107960 & 2005 & 393.30 & 0.10 & 39167.00 & 391666.86 & 1.00 & 1.00 & 1.00 \\
1424 & 100196 & 2005 & 8058.70 & 0.09 & 892351.00 & 8155276.62 & 0.90 & 1.01 & 0.91 \\
45300 & 200015 & 2005 & 61.90 & 0.12 & 8590.00 & 72605.80 & 0.72 & 1.17 & 0.85 \\
44961 & 109405 & 2005 & 50.80 & 0.01 & 5090.00 & 50591.49 & 1.00 & 1.00 & 0.99 \\
39977 & 107964 & 2005 & 131.80 & 0.05 & 13427.00 & 132954.79 & 0.98 & 1.01 & 0.99 \\
25286 & 103464 & 2005 & 967.10 & 0.10 & 93879.00 & 938782.67 & 1.03 & 0.97 & 1.00 \\
39422 & 107692 & 2005 & 9.60 & 0.08 & 940.00 & 9897.24 & 1.02 & 1.03 & 1.05 \\
53921 & 360123 & 2005 & 10.70 & 0.07 & 1065.00 & 10599.80 & 1.00 & 0.99 & 1.00 \\
24540 & 103339 & 2005 & 591.00 & -0.02 & 59176.00 & 567892.32 & 1.00 & 0.96 & 0.96 \\
58402 & 410153 & 2005 & 12.70 & 0.08 & 1286.00 & 13397.89 & 0.99 & 1.05 & 1.04 \\
40013 & 107994 & 2005 & 447.60 & 0.12 & 44435.00 & 432583.72 & 1.01 & 0.97 & 0.97 \\
49391 & 240288 & 2005 & 13.00 & 0.10 & 1304.00 & 13267.35 & 1.00 & 1.02 & 1.02 \\
44916 & 109401 & 2005 & 37.20 & 0.17 & 3594.00 & 35061.48 & 1.04 & 0.94 & 0.98 \\
25502 & 103494 & 2005 & 213.70 & 0.10 & 21009.00 & 213285.92 & 1.02 & 1.00 & 1.02 \\
45344 & 200047 & 2005 & 97.60 & 0.07 & 8672.00 & 79664.20 & 1.13 & 0.82 & 0.92 \\
39288 & 107648 & 2005 & 285.30 & 0.08 & 28598.00 & 284912.00 & 1.00 & 1.00 & 1.00 \\
40077 & 108021 & 2005 & 1488.30 & 0.14 & 142110.00 & 1363875.62 & 1.05 & 0.92 & 0.96 \\
40101 & 108029 & 2005 & 434.30 & 0.04 & 41585.00 & 408197.89 & 1.04 & 0.94 & 0.98 \\
45370 & 200050 & 2005 & 117.10 & 0.13 & 12228.00 & 120508.87 & 0.96 & 1.03 & 0.99 \\
53857 & 359285 & 2005 & 338.20 & 0.04 & 33775.00 & 279294.03 & 1.00 & 0.83 & 0.83 \\
24322 & 103308 & 2005 & 4086.60 & 0.08 & 414464.00 & 3790540.95 & 0.99 & 0.93 & 0.91 \\
17757 & 102356 & 2005 & 12.30 & 0.19 & 1081.00 & 10557.15 & 1.14 & 0.86 & 0.98 \\
44906 & 109399 & 2005 & 34.20 & 0.05 & 3234.00 & 30408.51 & 1.06 & 0.89 & 0.94 \\
65301 & 500684 & 2005 & 1666.30 & 1.26 & 166029.00 & 1659332.18 & 1.00 & 1.00 & 1.00 \\
12449 & 101541 & 2005 & 604.00 & 0.05 & 49129.00 & 510620.86 & 1.23 & 0.85 & 1.04 \\
65350 & 500689 & 2005 & 51.70 & 0.06 & 4854.00 & 48959.86 & 1.07 & 0.95 & 1.01 \\
64508 & 500606 & 2005 & 1300.20 & 0.08 & 130122.00 & 1299377.78 & 1.00 & 1.00 & 1.00 \\
65373 & 500692 & 2005 & 130.40 & 0.01 & 12666.00 & 124464.98 & 1.03 & 0.95 & 0.98 \\
54026 & 363941 & 2005 & 105.40 & 0.16 & 11522.00 & 101606.83 & 0.91 & 0.96 & 0.88 \\
44883 & 109397 & 2005 & 1089.10 & 0.12 & 118689.00 & 893296.06 & 0.92 & 0.82 & 0.75 \\
25559 & 103496 & 2005 & 300.40 & 0.10 & 30556.00 & 309535.75 & 0.98 & 1.03 & 1.01 \\
53845 & 357762 & 2005 & 42.00 & 0.07 & 4122.00 & 41318.12 & 1.02 & 0.98 & 1.00 \\
40133 & 108037 & 2005 & 113.10 & 0.06 & 11087.00 & 104472.91 & 1.02 & 0.92 & 0.94 \\
39259 & 107627 & 2005 & 1260.40 & 0.06 & 119221.00 & 1183846.98 & 1.06 & 0.94 & 0.99 \\
24288 & 103304 & 2005 & 64.50 & -0.01 & 6616.00 & 64129.64 & 0.97 & 0.99 & 0.97 \\
39242 & 107626 & 2005 & 1016.00 & 0.06 & 93680.00 & 974020.84 & 1.08 & 0.96 & 1.04 \\
1228 & 100166 & 2005 & 22535.70 & 0.07 & 2190753.00 & 18083950.01 & 1.03 & 0.80 & 0.83 \\
58409 & 410155 & 2005 & 6822.40 & 0.06 & 684781.00 & 6816029.16 & 1.00 & 1.00 & 1.00 \\
54049 & 364291 & 2005 & 97.40 & 0.10 & 11127.00 & 94211.45 & 0.88 & 0.97 & 0.85 \\
25590 & 103497 & 2005 & 534.40 & 0.08 & 46674.00 & 472079.25 & 1.14 & 0.88 & 1.01 \\
18185 & 102414 & 2005 & 8948.50 & 0.04 & 879185.00 & 8782388.74 & 1.02 & 0.98 & 1.00 \\
24348 & 103315 & 2005 & 196.30 & 0.13 & 19204.00 & 178582.34 & 1.02 & 0.91 & 0.93 \\
53869 & 359485 & 2005 & 31.50 & 0.03 & 3077.00 & 31162.56 & 1.02 & 0.99 & 1.01 \\
24410 & 103319 & 2005 & 247.70 & 0.06 & 24744.00 & 243841.61 & 1.00 & 0.98 & 0.99 \\
58136 & 410098 & 2005 & 10.60 & 0.06 & 1069.00 & 10063.53 & 0.99 & 0.95 & 0.94 \\
12107 & 101503 & 2005 & 139.00 & 0.12 & 13927.00 & 135050.90 & 1.00 & 0.97 & 0.97 \\
25415 & 103483 & 2005 & 451.60 & 0.04 & 47998.00 & 470244.79 & 0.94 & 1.04 & 0.98 \\
1940 & 100259 & 2005 & 132.60 & 0.08 & 13526.00 & 132426.57 & 0.98 & 1.00 & 0.98 \\
53874 & 359773 & 2005 & 3.20 & 0.08 & 344.00 & 3414.78 & 0.93 & 1.07 & 0.99 \\
4767 & 100671 & 2005 & 1503.30 & 0.06 & 135131.00 & 1275118.59 & 1.11 & 0.85 & 0.94 \\
14169 & 101819 & 2005 & 134.90 & 0.11 & 14857.00 & 136519.43 & 0.91 & 1.01 & 0.92 \\
49383 & 240287 & 2005 & 24.40 & 0.09 & 2436.00 & 24861.37 & 1.00 & 1.02 & 1.02 \\
4914 & 100692 & 2005 & 322.70 & 0.06 & 39247.00 & 338111.31 & 0.82 & 1.05 & 0.86 \\
40026 & 108009 & 2005 & 920.20 & 0.06 & 90163.00 & 872350.21 & 1.02 & 0.95 & 0.97 \\
1354 & 100190 & 2005 & 1451.90 & 0.07 & 148038.00 & 1457233.56 & 0.98 & 1.00 & 0.98 \\
25439 & 103487 & 2005 & 444.90 & 0.46 & 47514.00 & 442047.77 & 0.94 & 0.99 & 0.93 \\
39316 & 107653 & 2005 & 25.10 & 0.11 & 2587.00 & 24074.07 & 0.97 & 0.96 & 0.93 \\
48074 & 235413 & 2005 & 194.90 & 0.08 & 19461.00 & 184495.08 & 1.00 & 0.95 & 0.95 \\
24376 & 103318 & 2005 & 1331.30 & 0.02 & 136849.00 & 1335894.64 & 0.97 & 1.00 & 0.98 \\
65138 & 500664 & 2005 & 2069.20 & 0.06 & 209312.00 & 2048600.81 & 0.99 & 0.99 & 0.98 \\
12087 & 101497 & 2005 & 994.10 & 0.02 & 99493.00 & 967685.15 & 1.00 & 0.97 & 0.97 \\
49111 & 240222 & 2005 & 416.50 & 0.03 & 41589.00 & 399240.32 & 1.00 & 0.96 & 0.96 \\
17788 & 102357 & 2005 & 807.90 & 0.05 & 80815.00 & 804229.94 & 1.00 & 1.00 & 1.00 \\
65210 & 500670 & 2005 & 277.60 & 0.36 & 28440.00 & 271594.53 & 0.98 & 0.98 & 0.95 \\
1277 & 100171 & 2005 & 1927.10 & 0.10 & 168084.00 & 1433191.21 & 1.15 & 0.74 & 0.85 \\
58407 & 410154 & 2005 & 15.90 & 0.07 & 1698.00 & 17505.25 & 0.94 & 1.10 & 1.03 \\
48034 & 226946 & 2005 & 265.10 & 0.12 & 25374.00 & 206853.69 & 1.04 & 0.78 & 0.82 \\
40049 & 108018 & 2005 & 393.80 & 0.05 & 39379.00 & 327955.04 & 1.00 & 0.83 & 0.83 \\
64551 & 500609 & 2005 & 389.50 & 0.06 & 38961.00 & 388662.09 & 1.00 & 1.00 & 1.00 \\
4745 & 100670 & 2005 & 65.70 & 0.03 & 6692.00 & 65862.98 & 0.98 & 1.00 & 0.98 \\
17889 & 102371 & 2005 & 238.30 & -0.01 & 24559.00 & 212482.63 & 0.97 & 0.89 & 0.87 \\
49333 & 240270 & 2005 & 160.80 & 0.03 & 14198.00 & 134170.31 & 1.13 & 0.83 & 0.94 \\
24908 & 103394 & 2005 & 432.10 & 0.06 & 43652.00 & 402859.09 & 0.99 & 0.93 & 0.92 \\
4845 & 100685 & 2005 & 19.80 & -0.02 & 2022.00 & 18836.90 & 0.98 & 0.95 & 0.93 \\
49287 & 240264 & 2005 & 267.00 & 0.14 & 29516.00 & 273018.07 & 0.90 & 1.02 & 0.92 \\
39668 & 107833 & 2005 & 135.70 & 0.05 & 13563.00 & 133404.36 & 1.00 & 0.98 & 0.98 \\
64744 & 500625 & 2005 & 94.80 & 0.07 & 11383.00 & 109438.78 & 0.83 & 1.15 & 0.96 \\
45130 & 109431 & 2005 & 22.30 & 0.02 & 2134.00 & 19410.49 & 1.04 & 0.87 & 0.91 \\
12250 & 101530 & 2005 & 1551.80 & 0.02 & 164075.00 & 1452223.15 & 0.95 & 0.94 & 0.89 \\
64846 & 500636 & 2005 & 98.20 & 1.11 & 10410.00 & 104331.60 & 0.94 & 1.06 & 1.00 \\
1689 & 100223 & 2005 & 2389.40 & 0.04 & 238984.00 & 2279092.47 & 1.00 & 0.95 & 0.95 \\
8073 & 101073 & 2005 & 8167.60 & 0.09 & 772605.00 & 8005733.30 & 1.06 & 0.98 & 1.04 \\
45006 & 109413 & 2005 & 515.80 & 0.11 & 40181.00 & 415855.04 & 1.28 & 0.81 & 1.03 \\
24946 & 103395 & 2005 & 113.40 & 0.10 & 11327.00 & 109445.25 & 1.00 & 0.97 & 0.97 \\
39641 & 107832 & 2005 & 267.90 & 0.10 & 26889.00 & 262329.58 & 1.00 & 0.98 & 0.98 \\
39829 & 107881 & 2005 & 19.90 & 0.05 & 1982.00 & 17541.47 & 1.00 & 0.88 & 0.89 \\
17973 & 102377 & 2005 & 183.90 & 0.03 & 18047.00 & 175624.71 & 1.02 & 0.96 & 0.97 \\
24690 & 103375 & 2005 & 983.80 & 0.22 & 104298.00 & 906924.23 & 0.94 & 0.92 & 0.87 \\
58251 & 410133 & 2005 & 204.60 & 0.08 & 20476.00 & 198512.55 & 1.00 & 0.97 & 0.97 \\
12235 & 101528 & 2005 & 82.90 & 0.00 & 8885.00 & 79886.96 & 0.93 & 0.96 & 0.90 \\
12331 & 101536 & 2005 & 1636.90 & 0.10 & 165249.00 & 1606936.22 & 0.99 & 0.98 & 0.97 \\
4813 & 100682 & 2005 & 72.50 & 0.06 & 7280.00 & 70384.07 & 1.00 & 0.97 & 0.97 \\
58218 & 410121 & 2005 & 46.80 & 0.05 & 4673.00 & 47042.55 & 1.00 & 1.01 & 1.01 \\
64869 & 500638 & 2005 & 103.50 & 0.07 & 10612.00 & 106332.97 & 0.98 & 1.03 & 1.00 \\
1829 & 100244 & 2005 & 342.60 & 0.09 & 34437.00 & 331469.00 & 0.99 & 0.97 & 0.96 \\
39851 & 107882 & 2005 & 247.90 & 0.08 & 43876.00 & 394978.20 & 0.57 & 1.59 & 0.90 \\
1791 & 100237 & 2005 & 16.60 & 0.09 & 1761.00 & 18275.51 & 0.94 & 1.10 & 1.04 \\
45028 & 109414 & 2005 & 315.40 & 0.14 & 24032.00 & 248335.86 & 1.31 & 0.79 & 1.03 \\
55313 & 400084 & 2005 & 52.40 & 0.17 & 5242.00 & 48971.45 & 1.00 & 0.93 & 0.93 \\
49272 & 240261 & 2005 & 100.00 & 0.04 & 10663.00 & 98848.70 & 0.94 & 0.99 & 0.93 \\
1752 & 100227 & 2005 & 109.20 & 0.02 & 10915.00 & 106252.06 & 1.00 & 0.97 & 0.97 \\
53963 & 362424 & 2005 & 246.20 & 0.06 & 24751.00 & 230186.60 & 0.99 & 0.93 & 0.93 \\
24806 & 103380 & 2005 & 3647.60 & 0.00 & 379844.00 & 3410589.49 & 0.96 & 0.94 & 0.90 \\
39777 & 107870 & 2005 & 248.60 & 0.07 & 23792.00 & 242941.79 & 1.04 & 0.98 & 1.02 \\
39734 & 107858 & 2005 & 29.90 & 0.07 & 3026.00 & 27744.19 & 0.99 & 0.93 & 0.92 \\
24767 & 103377 & 2005 & 958.20 & 0.22 & 86562.00 & 842656.10 & 1.11 & 0.88 & 0.97 \\
64777 & 500633 & 2005 & 69.70 & 0.09 & 5595.00 & 56401.56 & 1.25 & 0.81 & 1.01 \\
12313 & 101534 & 2005 & 1158.00 & 0.09 & 116955.00 & 1152488.73 & 0.99 & 1.00 & 0.99 \\
64800 & 500634 & 2005 & 97.00 & 2.18 & 9350.00 & 92950.28 & 1.04 & 0.96 & 0.99 \\
45114 & 109429 & 2005 & 20.50 & 0.02 & 1924.00 & 18608.93 & 1.07 & 0.91 & 0.97 \\
49244 & 240254 & 2005 & 290.90 & 0.04 & 29083.00 & 285901.50 & 1.00 & 0.98 & 0.98 \\
12278 & 101531 & 2005 & 275.20 & 0.01 & 29539.00 & 280408.77 & 0.93 & 1.02 & 0.95 \\
18013 & 102386 & 2005 & 5327.20 & 0.06 & 531990.00 & 5020490.66 & 1.00 & 0.94 & 0.94 \\
1772 & 100228 & 2005 & 108.90 & 0.02 & 10869.00 & 105611.36 & 1.00 & 0.97 & 0.97 \\
64768 & 500628 & 2005 & 48.30 & 0.01 & 4916.00 & 47598.98 & 0.98 & 0.99 & 0.97 \\
39793 & 107872 & 2005 & 63.30 & 0.08 & 6670.00 & 66163.29 & 0.95 & 1.05 & 0.99 \\
53986 & 362981 & 2005 & 232.00 & 0.04 & 23864.00 & 245309.43 & 0.97 & 1.06 & 1.03 \\
7119 & 100997 & 2005 & 945.20 & 0.12 & 94511.00 & 794035.21 & 1.00 & 0.84 & 0.84 \\
39685 & 107835 & 2005 & 815.20 & 0.04 & 81412.00 & 793436.99 & 1.00 & 0.97 & 0.97 \\
18048 & 102387 & 2005 & 839.00 & 0.11 & 83815.00 & 828638.25 & 1.00 & 0.99 & 0.99 \\
64823 & 500635 & 2005 & 143.40 & 0.03 & 11945.00 & 122400.33 & 1.20 & 0.85 & 1.02 \\
1708 & 100226 & 2005 & 16905.30 & 0.04 & 1678073.00 & 13667573.39 & 1.01 & 0.81 & 0.81 \\
45061 & 109416 & 2005 & 226.50 & 0.14 & 16875.00 & 174802.44 & 1.34 & 0.77 & 1.04 \\
24730 & 103376 & 2005 & 4911.00 & 0.05 & 478124.00 & 4800006.94 & 1.03 & 0.98 & 1.00 \\
24887 & 103383 & 2005 & 2308.70 & 0.09 & 219589.00 & 2212901.66 & 1.05 & 0.96 & 1.01 \\
45050 & 109415 & 2005 & 137.80 & 0.13 & 10844.00 & 112850.82 & 1.27 & 0.82 & 1.04 \\
24846 & 103381 & 2005 & 21338.00 & 0.02 & 2154807.00 & 20888330.84 & 0.99 & 0.98 & 0.97 \\
45281 & 200011 & 2005 & 196.30 & 0.10 & 18683.00 & 192323.18 & 1.05 & 0.98 & 1.03 \\
58193 & 410115 & 2005 & 403.50 & 0.11 & 27748.00 & 374635.24 & 1.45 & 0.93 & 1.35 \\
49296 & 240266 & 2005 & 257.40 & 0.04 & 29799.00 & 287606.79 & 0.86 & 1.12 & 0.97 \\
12195 & 101518 & 2005 & 280.30 & 0.12 & 27987.00 & 274556.89 & 1.00 & 0.98 & 0.98 \\
45237 & 109439 & 2005 & 708.30 & 0.06 & 69234.00 & 689608.14 & 1.02 & 0.97 & 1.00 \\
44996 & 109410 & 2005 & 252.40 & 0.10 & 22826.00 & 234131.31 & 1.11 & 0.93 & 1.03 \\
25117 & 103432 & 2005 & 1871.90 & 0.15 & 179729.00 & 1891625.47 & 1.04 & 1.01 & 1.05 \\
8034 & 101071 & 2005 & 9948.90 & 0.16 & 905299.00 & 9545440.08 & 1.10 & 0.96 & 1.05 \\
39528 & 107719 & 2005 & 310.50 & 0.01 & 34164.00 & 346785.65 & 0.91 & 1.12 & 1.02 \\
44986 & 109407 & 2005 & 346.90 & 0.07 & 37876.00 & 349517.85 & 0.92 & 1.01 & 0.92 \\
45259 & 109444 & 2005 & 20.70 & 0.04 & 1837.00 & 17424.45 & 1.13 & 0.84 & 0.95 \\
1541 & 100213 & 2005 & 280.60 & 0.14 & 28352.00 & 278277.20 & 0.99 & 0.99 & 0.98 \\
39925 & 107938 & 2005 & 67.10 & 0.05 & 7414.00 & 74142.42 & 0.91 & 1.10 & 1.00 \\
24578 & 103369 & 2005 & 749.50 & 0.08 & 74076.00 & 735831.37 & 1.01 & 0.98 & 0.99 \\
8103 & 101074 & 2005 & 1808.80 & 0.46 & 178173.00 & 1829744.72 & 1.02 & 1.01 & 1.03 \\
17909 & 102372 & 2005 & 6135.00 & -0.01 & 624777.00 & 5921474.72 & 0.98 & 0.97 & 0.95 \\
44967 & 109406 & 2005 & 166.70 & 0.05 & 19168.00 & 179166.05 & 0.87 & 1.07 & 0.93 \\
25152 & 103439 & 2005 & 98.30 & 0.09 & 10084.00 & 93696.51 & 0.97 & 0.95 & 0.93 \\
12364 & 101537 & 2005 & 387.70 & 0.06 & 38819.00 & 371815.55 & 1.00 & 0.96 & 0.96 \\
55342 & 400087 & 2005 & 68.60 & 0.09 & 7597.00 & 68207.37 & 0.90 & 0.99 & 0.90 \\
1863 & 100245 & 2005 & 1221.60 & 0.04 & 121434.00 & 1165504.56 & 1.01 & 0.95 & 0.96 \\
58141 & 410100 & 2005 & 489.60 & 0.17 & 48360.00 & 457820.98 & 1.01 & 0.94 & 0.95 \\
55274 & 400076 & 2005 & 699.20 & 0.01 & 90066.00 & 825321.62 & 0.78 & 1.18 & 0.92 \\
1523 & 100209 & 2005 & 5187.70 & 0.07 & 523633.00 & 4993834.88 & 0.99 & 0.96 & 0.95 \\
64659 & 500617 & 2005 & 8941.90 & 0.12 & 653521.00 & 7866764.92 & 1.37 & 0.88 & 1.20 \\
49322 & 240269 & 2005 & 184.40 & 0.11 & 18358.00 & 179384.62 & 1.00 & 0.97 & 0.98 \\
48045 & 227155 & 2005 & 367.50 & 0.06 & 33489.00 & 318235.89 & 1.10 & 0.87 & 0.95 \\
1485 & 100207 & 2005 & 1563.80 & 0.15 & 156512.00 & 1559569.56 & 1.00 & 1.00 & 1.00 \\
64964 & 500652 & 2005 & 72.80 & 0.00 & 4818.00 & 47867.76 & 1.51 & 0.66 & 0.99 \\
24598 & 103370 & 2005 & 261.90 & 0.06 & 25275.00 & 256611.08 & 1.04 & 0.98 & 1.02 \\
39899 & 107928 & 2005 & 2535.90 & 0.05 & 244489.00 & 2444826.47 & 1.04 & 0.96 & 1.00 \\
64950 & 500648 & 2005 & 332.20 & 0.01 & 32333.00 & 320984.60 & 1.03 & 0.97 & 0.99 \\
24998 & 103406 & 2005 & 2034.80 & 0.10 & 204139.00 & 2037122.07 & 1.00 & 1.00 & 1.00 \\
49217 & 240250 & 2005 & 733.50 & 0.06 & 73081.00 & 670363.11 & 1.00 & 0.91 & 0.92 \\
45154 & 109433 & 2005 & 45.60 & 0.06 & 4433.00 & 38038.64 & 1.03 & 0.83 & 0.86 \\
55305 & 400081 & 2005 & 2684.10 & 0.13 & 280529.00 & 2822480.70 & 0.96 & 1.05 & 1.01 \\
39604 & 107786 & 2005 & 1197.90 & 0.04 & 119839.00 & 1197595.31 & 1.00 & 1.00 & 1.00 \\
45176 & 109435 & 2005 & 16.60 & 0.10 & 1528.00 & 14784.41 & 1.09 & 0.89 & 0.97 \\
24650 & 103373 & 2005 & 49.90 & -0.02 & 4707.00 & 46024.08 & 1.06 & 0.92 & 0.98 \\
64938 & 500646 & 2005 & 2.70 & 0.04 & 261.00 & 2527.72 & 1.03 & 0.94 & 0.97 \\
1601 & 100217 & 2005 & 31.20 & 0.08 & 3055.00 & 30553.57 & 1.02 & 0.98 & 1.00 \\
45183 & 109436 & 2005 & 71.70 & 0.14 & 7176.00 & 70902.91 & 1.00 & 0.99 & 0.99 \\
53940 & 361995 & 2005 & 2.10 & -0.14 & 384.00 & 3798.88 & 0.55 & 1.81 & 0.99 \\
64720 & 500621 & 2005 & 1045.80 & 1.18 & 104104.00 & 1040442.35 & 1.00 & 0.99 & 1.00 \\
25041 & 103426 & 2005 & 1350.30 & 0.11 & 131823.00 & 1320384.00 & 1.02 & 0.98 & 1.00 \\
39595 & 107781 & 2005 & 153.00 & 0.05 & 15254.00 & 152050.21 & 1.00 & 0.99 & 1.00 \\
53945 & 362337 & 2005 & 21.70 & 0.14 & 1807.00 & 17583.18 & 1.20 & 0.81 & 0.97 \\
64697 & 500620 & 2005 & 703.60 & 0.15 & 70359.00 & 702666.79 & 1.00 & 1.00 & 1.00 \\
18086 & 102396 & 2005 & 2549.80 & 0.09 & 284081.00 & 2146749.18 & 0.90 & 0.84 & 0.76 \\
39870 & 107892 & 2005 & 1035.10 & 0.18 & 103507.00 & 849418.17 & 1.00 & 0.82 & 0.82 \\
64944 & 500647 & 2005 & 125.00 & 0.09 & 12063.00 & 119449.02 & 1.04 & 0.96 & 0.99 \\
1572 & 100214 & 2005 & 193.00 & 0.15 & 19319.00 & 187157.48 & 1.00 & 0.97 & 0.97 \\
17939 & 102376 & 2005 & 229.80 & 0.09 & 22900.00 & 192991.92 & 1.00 & 0.84 & 0.84 \\
64670 & 500618 & 2005 & 497.40 & -0.02 & 49703.00 & 496274.89 & 1.00 & 1.00 & 1.00 \\
24621 & 103372 & 2005 & 582.70 & 0.02 & 57399.00 & 496424.81 & 1.02 & 0.85 & 0.86 \\
55336 & 400085 & 2005 & 26.30 & 0.01 & 2710.00 & 23980.64 & 0.97 & 0.91 & 0.88 \\
49193 & 240243 & 2005 & 781.90 & 0.29 & 70389.00 & 590260.05 & 1.11 & 0.75 & 0.84 \\
25079 & 103429 & 2005 & 2502.40 & 0.17 & 238200.00 & 2075677.44 & 1.05 & 0.83 & 0.87 \\
39567 & 107722 & 2005 & 487.50 & 0.09 & 42676.00 & 434860.03 & 1.14 & 0.89 & 1.02 \\
45215 & 109438 & 2005 & 210.60 & 0.08 & 21441.00 & 212382.20 & 0.98 & 1.01 & 0.99 \\
39553 & 107720 & 2005 & 242.30 & 0.05 & 23986.00 & 232474.39 & 1.01 & 0.96 & 0.97 \\
53995 & 363121 & 2005 & 3716.30 & 0.06 & 372708.00 & 3684274.88 & 1.00 & 0.99 & 0.99 \\
39583 & 107726 & 2005 & 1394.00 & 0.09 & 140443.00 & 1365561.70 & 0.99 & 0.98 & 0.97 \\
46108 & 200190 & 2005 & 272.80 & 0.05 & 27227.00 & 230261.28 & 1.00 & 0.84 & 0.85 \\
17736 & 102350 & 2005 & 257.30 & 0.04 & 25251.00 & 252504.37 & 1.02 & 0.98 & 1.00 \\
39233 & 107623 & 2005 & 3.60 & 0.03 & 356.00 & 3566.26 & 1.01 & 0.99 & 1.00 \\
26111 & 103539 & 2005 & 983.60 & 0.04 & 98419.00 & 981091.52 & 1.00 & 1.00 & 1.00 \\
47980 & 225484 & 2005 & 34.30 & 0.02 & 3502.00 & 35199.64 & 0.98 & 1.03 & 1.01 \\
11910 & 101465 & 2005 & 138.10 & 0.05 & 13831.00 & 136788.89 & 1.00 & 0.99 & 0.99 \\
40511 & 108134 & 2005 & 271.40 & 0.10 & 27135.00 & 256512.56 & 1.00 & 0.95 & 0.95 \\
58551 & 410165 & 2005 & 76.80 & 0.12 & 7092.00 & 71705.96 & 1.08 & 0.93 & 1.01 \\
23936 & 103242 & 2005 & 984.90 & -0.02 & 98590.00 & 983979.60 & 1.00 & 1.00 & 1.00 \\
38728 & 107316 & 2005 & 1502.10 & 0.07 & 150181.00 & 1459451.35 & 1.00 & 0.97 & 0.97 \\
12576 & 101554 & 2005 & 217.10 & 0.09 & 22440.00 & 211233.69 & 0.97 & 0.97 & 0.94 \\
65674 & 500714 & 2005 & 102.50 & 0.13 & 10235.00 & 94838.35 & 1.00 & 0.93 & 0.93 \\
18326 & 102425 & 2005 & 1404.10 & 0.08 & 135681.00 & 1424386.77 & 1.03 & 1.01 & 1.05 \\
26145 & 103544 & 2005 & 65419.60 & 0.06 & 6471459.00 & 64421515.04 & 1.01 & 0.98 & 1.00 \\
23916 & 103232 & 2005 & 135.20 & 0.00 & 14591.00 & 144316.83 & 0.93 & 1.07 & 0.99 \\
45556 & 200075 & 2005 & 405.70 & 0.06 & 40539.00 & 386378.96 & 1.00 & 0.95 & 0.95 \\
38716 & 107309 & 2005 & 41.90 & 0.04 & 4207.00 & 41747.44 & 1.00 & 1.00 & 0.99 \\
54179 & 364809 & 2005 & 41.20 & 0.09 & 4113.00 & 39726.02 & 1.00 & 0.96 & 0.97 \\
49451 & 240301 & 2005 & 16.80 & 0.04 & 1536.00 & 15722.64 & 1.09 & 0.94 & 1.02 \\
38691 & 107308 & 2005 & 2966.70 & 0.16 & 297888.00 & 2797678.32 & 1.00 & 0.94 & 0.94 \\
984 & 100113 & 2005 & 841.50 & 0.06 & 85436.00 & 843564.51 & 0.98 & 1.00 & 0.99 \\
64354 & 500598 & 2005 & 3861.50 & 2.23 & 386176.00 & 3854730.25 & 1.00 & 1.00 & 1.00 \\
17511 & 102317 & 2005 & 27.80 & -0.00 & 2761.00 & 26048.71 & 1.01 & 0.94 & 0.94 \\
40543 & 108137 & 2005 & 1310.10 & -0.01 & 129821.00 & 1298206.49 & 1.01 & 0.99 & 1.00 \\
23900 & 103228 & 2005 & 79.70 & 0.06 & 8465.00 & 75261.96 & 0.94 & 0.94 & 0.89 \\
58556 & 410166 & 2005 & 80.90 & 0.04 & 7756.00 & 77420.07 & 1.04 & 0.96 & 1.00 \\
38668 & 107306 & 2005 & 299.50 & 0.03 & 30317.00 & 311022.35 & 0.99 & 1.04 & 1.03 \\
7913 & 101065 & 2005 & 1679.00 & 0.09 & 167018.00 & 1686530.71 & 1.01 & 1.00 & 1.01 \\
38660 & 107303 & 2005 & 42.10 & 0.02 & 4338.00 & 43606.11 & 0.97 & 1.04 & 1.01 \\
49475 & 240302 & 2005 & 7.70 & 0.12 & 642.00 & 6249.08 & 1.20 & 0.81 & 0.97 \\
58561 & 410167 & 2005 & 16.50 & 0.09 & 1603.00 & 16701.47 & 1.03 & 1.01 & 1.04 \\
58531 & 410163 & 2005 & 21.60 & 0.11 & 1655.00 & 16988.85 & 1.31 & 0.79 & 1.03 \\
40485 & 108122 & 2005 & 943.30 & 0.15 & 96107.00 & 924707.23 & 0.98 & 0.98 & 0.96 \\
65669 & 500713 & 2005 & 213.10 & 0.04 & 21267.00 & 211341.66 & 1.00 & 0.99 & 0.99 \\
64377 & 500600 & 2005 & 3242.40 & 0.03 & 323894.00 & 3237406.55 & 1.00 & 1.00 & 1.00 \\
1028 & 100127 & 2005 & 2124.90 & 0.05 & 211050.00 & 1732822.95 & 1.01 & 0.82 & 0.82 \\
40425 & 108118 & 2005 & 73.70 & 0.02 & 7387.00 & 74022.05 & 1.00 & 1.00 & 1.00 \\
44791 & 109380 & 2005 & 40.60 & -0.00 & 3999.00 & 36321.17 & 1.02 & 0.89 & 0.91 \\
24008 & 103253 & 2005 & 259.50 & 0.10 & 26802.00 & 252699.29 & 0.97 & 0.97 & 0.94 \\
64400 & 500601 & 2005 & 213.60 & 0.05 & 17974.00 & 186277.95 & 1.19 & 0.87 & 1.04 \\
45514 & 200072 & 2005 & 9.50 & 0.01 & 598.00 & 4798.31 & 1.59 & 0.51 & 0.80 \\
40433 & 108119 & 2005 & 382.50 & 0.06 & 38247.00 & 389167.16 & 1.00 & 1.02 & 1.02 \\
26034 & 103535 & 2005 & 3420.20 & 0.08 & 342128.00 & 3419407.37 & 1.00 & 1.00 & 1.00 \\
38849 & 107336 & 2005 & 229.80 & 0.05 & 37997.00 & 344994.81 & 0.60 & 1.50 & 0.91 \\
54137 & 364519 & 2005 & 85.00 & 0.06 & 8656.00 & 83183.46 & 0.98 & 0.98 & 0.96 \\
14248 & 101835 & 2005 & 1462.60 & 0.05 & 147743.00 & 1464983.06 & 0.99 & 1.00 & 0.99 \\
44777 & 109375 & 2005 & 98.10 & 0.05 & 9795.00 & 94662.35 & 1.00 & 0.96 & 0.97 \\
38828 & 107331 & 2005 & 48.40 & 0.16 & 3785.00 & 37570.14 & 1.28 & 0.78 & 0.99 \\
65690 & 500719 & 2005 & 93.00 & 0.07 & 11319.00 & 113192.17 & 0.82 & 1.22 & 1.00 \\
23987 & 103252 & 2005 & 362.00 & 0.06 & 37229.00 & 366690.59 & 0.97 & 1.01 & 0.98 \\
45540 & 200073 & 2005 & 241.40 & 0.04 & 32474.00 & 318189.27 & 0.74 & 1.32 & 0.98 \\
47993 & 225687 & 2005 & 226.50 & 0.09 & 22698.00 & 225004.34 & 1.00 & 0.99 & 0.99 \\
26064 & 103536 & 2005 & 2684.00 & 0.05 & 268752.00 & 2687331.68 & 1.00 & 1.00 & 1.00 \\
17554 & 102318 & 2005 & 2436.60 & 0.06 & 243927.00 & 2097313.57 & 1.00 & 0.86 & 0.86 \\
38812 & 107328 & 2005 & 35.80 & 0.07 & 3747.00 & 29372.60 & 0.96 & 0.82 & 0.78 \\
40459 & 108121 & 2005 & 3881.10 & 0.06 & 388223.00 & 3873669.59 & 1.00 & 1.00 & 1.00 \\
54159 & 364633 & 2005 & 56.50 & 0.10 & 3980.00 & 40164.64 & 1.42 & 0.71 & 1.01 \\
38787 & 107323 & 2005 & 100.00 & 0.09 & 11665.00 & 98952.33 & 0.86 & 0.99 & 0.85 \\
23966 & 103251 & 2005 & 308.50 & 0.07 & 31424.00 & 315357.03 & 0.98 & 1.02 & 1.00 \\
38764 & 107322 & 2005 & 20.90 & 0.09 & 1970.00 & 20141.25 & 1.06 & 0.96 & 1.02 \\
48097 & 235952 & 2005 & 54.40 & 0.16 & 5105.00 & 46395.87 & 1.07 & 0.85 & 0.91 \\
58495 & 410161 & 2005 & 558.50 & 0.06 & 56143.00 & 561019.65 & 0.99 & 1.00 & 1.00 \\
58512 & 410162 & 2005 & 61.70 & 0.08 & 5410.00 & 51369.90 & 1.14 & 0.83 & 0.95 \\
49437 & 240297 & 2005 & 986.90 & 0.06 & 111656.00 & 936250.95 & 0.88 & 0.95 & 0.84 \\
65641 & 500710 & 2005 & 1022.30 & 0.06 & 93118.00 & 903502.62 & 1.10 & 0.88 & 0.97 \\
58582 & 410170 & 2005 & 245.30 & 0.10 & 20323.00 & 208404.93 & 1.21 & 0.85 & 1.03 \\
26315 & 103567 & 2005 & 591.30 & 0.04 & 59369.00 & 583828.82 & 1.00 & 0.99 & 0.98 \\
40592 & 108141 & 2005 & 196.70 & 0.02 & 19472.00 & 195830.91 & 1.01 & 1.00 & 1.01 \\
5088 & 100723 & 2005 & 40.10 & 0.03 & 4013.00 & 39721.00 & 1.00 & 0.99 & 0.99 \\
58038 & 410055 & 2005 & 82.80 & 0.02 & 6459.00 & 65619.91 & 1.28 & 0.79 & 1.02 \\
38545 & 107281 & 2005 & 18.50 & 0.07 & 1805.00 & 18684.35 & 1.02 & 1.01 & 1.04 \\
40617 & 108142 & 2005 & 44.00 & 0.03 & 3495.00 & 34952.23 & 1.26 & 0.79 & 1.00 \\
18360 & 102446 & 2005 & 286.60 & 0.06 & 27377.00 & 273765.43 & 1.05 & 0.96 & 1.00 \\
65879 & 500750 & 2005 & 2.10 & -0.01 & 194.00 & 1751.69 & 1.08 & 0.83 & 0.90 \\
49498 & 240305 & 2005 & 55.00 & 0.10 & 5458.00 & 47481.49 & 1.01 & 0.86 & 0.87 \\
45581 & 200079 & 2005 & 11.50 & 0.11 & 1148.00 & 11203.88 & 1.00 & 0.97 & 0.98 \\
23801 & 103213 & 2005 & 453.90 & 0.08 & 45536.00 & 447737.27 & 1.00 & 0.99 & 0.98 \\
26347 & 103570 & 2005 & 24.40 & 0.02 & 2216.00 & 22063.22 & 1.10 & 0.90 & 1.00 \\
44771 & 109374 & 2005 & 90.70 & 0.13 & 9179.00 & 91992.94 & 0.99 & 1.01 & 1.00 \\
38534 & 107274 & 2005 & 7.00 & -0.04 & 885.00 & 8310.46 & 0.79 & 1.19 & 0.94 \\
64285 & 500595 & 2005 & 2412.90 & 0.06 & 231891.00 & 2395203.90 & 1.04 & 0.99 & 1.03 \\
38511 & 107266 & 2005 & 730.50 & 0.08 & 61753.00 & 605225.41 & 1.18 & 0.83 & 0.98 \\
898 & 100101 & 2005 & 369.30 & 0.14 & 36185.00 & 369679.79 & 1.02 & 1.00 & 1.02 \\
40642 & 108143 & 2005 & 10.30 & -0.04 & 811.00 & 8103.99 & 1.27 & 0.79 & 1.00 \\
11849 & 101463 & 2005 & 931.90 & 0.06 & 91833.00 & 928162.92 & 1.01 & 1.00 & 1.01 \\
40667 & 108144 & 2005 & 141.90 & 0.11 & 14066.00 & 140635.09 & 1.01 & 0.99 & 1.00 \\
12615 & 101560 & 2005 & 264.50 & 0.17 & 26736.00 & 246247.30 & 0.99 & 0.93 & 0.92 \\
26387 & 103573 & 2005 & 156.00 & 0.08 & 16240.00 & 152863.26 & 0.96 & 0.98 & 0.94 \\
38486 & 107263 & 2005 & 1846.10 & 0.05 & 184712.00 & 1825933.84 & 1.00 & 0.99 & 0.99 \\
44759 & 109373 & 2005 & 1242.40 & 0.13 & 104817.00 & 1054528.49 & 1.19 & 0.85 & 1.01 \\
53682 & 355987 & 2005 & 613.40 & 0.10 & 61458.00 & 521050.67 & 1.00 & 0.85 & 0.85 \\
54184 & 364818 & 2005 & 2021.30 & 0.10 & 218261.00 & 1821226.99 & 0.93 & 0.90 & 0.83 \\
49016 & 240199 & 2005 & 597.80 & 0.06 & 59478.00 & 579968.87 & 1.01 & 0.97 & 0.98 \\
58604 & 410178 & 2005 & 4.30 & 0.15 & 432.00 & 4361.34 & 1.00 & 1.01 & 1.01 \\
26219 & 103546 & 2005 & 21277.30 & 0.05 & 4052223.00 & 40475818.10 & 0.53 & 1.90 & 1.00 \\
5056 & 100710 & 2005 & 1280.10 & 0.01 & 125527.00 & 1255145.65 & 1.02 & 0.98 & 1.00 \\
4647 & 100659 & 2005 & 969.60 & 0.07 & 169765.00 & 1634556.47 & 0.57 & 1.69 & 0.96 \\
49483 & 240303 & 2005 & 8.40 & 0.09 & 837.00 & 8361.48 & 1.00 & 1.00 & 1.00 \\
38635 & 107302 & 2005 & 296.70 & 0.04 & 30286.00 & 297770.73 & 0.98 & 1.00 & 0.98 \\
55419 & 400094 & 2005 & 1230.40 & 0.13 & 124053.00 & 1085045.13 & 0.99 & 0.88 & 0.87 \\
2066 & 100287 & 2005 & 75.10 & 0.09 & 7529.00 & 69627.70 & 1.00 & 0.93 & 0.92 \\
38625 & 107300 & 2005 & 56.80 & 0.04 & 5791.00 & 58160.35 & 0.98 & 1.02 & 1.00 \\
64331 & 500597 & 2005 & 14229.30 & 0.05 & 1421603.00 & 14208547.65 & 1.00 & 1.00 & 1.00 \\
23864 & 103224 & 2005 & 95.30 & 0.07 & 9880.00 & 93957.52 & 0.96 & 0.99 & 0.95 \\
11880 & 101464 & 2005 & 1914.50 & 0.03 & 224388.00 & 1937715.98 & 0.85 & 1.01 & 0.86 \\
26250 & 103547 & 2005 & 6254.60 & 0.11 & 632001.00 & 6030589.12 & 0.99 & 0.96 & 0.95 \\
23882 & 103226 & 2005 & 83.30 & 0.01 & 8663.00 & 87372.60 & 0.96 & 1.05 & 1.01 \\
48109 & 240010 & 2005 & 261.10 & 0.11 & 26527.00 & 256206.15 & 0.98 & 0.98 & 0.97 \\
40554 & 108138 & 2005 & 9.60 & -0.01 & 1127.00 & 11112.75 & 0.85 & 1.16 & 0.99 \\
14281 & 101842 & 2005 & 976.90 & 0.06 & 107864.00 & 1115217.51 & 0.91 & 1.14 & 1.03 \\
65744 & 500729 & 2005 & 1093.70 & 1.72 & 109355.00 & 1092582.90 & 1.00 & 1.00 & 1.00 \\
940 & 100112 & 2005 & 5327.20 & 0.04 & 533100.00 & 5330934.43 & 1.00 & 1.00 & 1.00 \\
47952 & 225413 & 2005 & 149.70 & 0.03 & 15000.00 & 143061.56 & 1.00 & 0.96 & 0.95 \\
40583 & 108140 & 2005 & 96.60 & 0.09 & 9540.00 & 95399.04 & 1.01 & 0.99 & 1.00 \\
12601 & 101557 & 2005 & 15.60 & 0.01 & 1562.00 & 14754.17 & 1.00 & 0.95 & 0.94 \\
23831 & 103214 & 2005 & 1417.60 & 0.09 & 141688.00 & 1406874.03 & 1.00 & 0.99 & 0.99 \\
26297 & 103564 & 2005 & 780.20 & 0.10 & 77859.00 & 754182.57 & 1.00 & 0.97 & 0.97 \\
38578 & 107290 & 2005 & 789.30 & 0.05 & 79144.00 & 790536.95 & 1.00 & 1.00 & 1.00 \\
64308 & 500596 & 2005 & 5399.30 & 0.05 & 440789.00 & 4567312.82 & 1.22 & 0.85 & 1.04 \\
53836 & 357756 & 2005 & 81.20 & 0.10 & 7928.00 & 72627.94 & 1.02 & 0.89 & 0.92 \\
53717 & 356500 & 2005 & 285.00 & 0.06 & 28475.00 & 283538.71 & 1.00 & 0.99 & 1.00 \\
24204 & 103296 & 2005 & 580.70 & 0.05 & 57431.00 & 574273.96 & 1.01 & 0.99 & 1.00 \\
58104 & 410093 & 2005 & 110.00 & 0.10 & 9807.00 & 96947.06 & 1.12 & 0.88 & 0.99 \\
18232 & 102417 & 2005 & 828.30 & 0.01 & 86619.00 & 843952.03 & 0.96 & 1.02 & 0.97 \\
49079 & 240218 & 2005 & 1318.10 & 0.08 & 133653.00 & 1208065.66 & 0.99 & 0.92 & 0.90 \\
39134 & 107608 & 2005 & 52.10 & -0.01 & 5185.00 & 51139.40 & 1.00 & 0.98 & 0.99 \\
45411 & 200057 & 2005 & 1218.50 & 0.11 & 113756.00 & 1184501.91 & 1.07 & 0.97 & 1.04 \\
14202 & 101820 & 2005 & 165.40 & -0.03 & 17390.00 & 154434.10 & 0.95 & 0.93 & 0.89 \\
25732 & 103520 & 2005 & 10609.10 & 0.05 & 1060782.00 & 10585868.12 & 1.00 & 1.00 & 1.00 \\
49401 & 240291 & 2005 & 6.70 & 0.04 & 998.00 & 9977.62 & 0.67 & 1.49 & 1.00 \\
39124 & 107607 & 2005 & 110.90 & 0.06 & 11156.00 & 107181.36 & 0.99 & 0.97 & 0.96 \\
40227 & 108074 & 2005 & 99.00 & 0.08 & 9351.00 & 95972.63 & 1.06 & 0.97 & 1.03 \\
45422 & 200058 & 2005 & 4066.40 & 0.10 & 414437.00 & 3772149.58 & 0.98 & 0.93 & 0.91 \\
39099 & 107605 & 2005 & 1844.50 & 0.07 & 190717.00 & 1856824.39 & 0.97 & 1.01 & 0.97 \\
17682 & 102342 & 2005 & 134.70 & 0.11 & 13574.00 & 133938.33 & 0.99 & 0.99 & 0.99 \\
49068 & 240212 & 2005 & 2717.60 & 0.13 & 268662.00 & 2373551.41 & 1.01 & 0.87 & 0.88 \\
65489 & 500700 & 2005 & 228.10 & 0.02 & 22813.00 & 228011.51 & 1.00 & 1.00 & 1.00 \\
44837 & 109394 & 2005 & 377.20 & 0.03 & 37848.00 & 313377.69 & 1.00 & 0.83 & 0.83 \\
11995 & 101476 & 2005 & 2659.00 & 0.08 & 265924.00 & 2617294.86 & 1.00 & 0.98 & 0.98 \\
40238 & 108082 & 2005 & 80.90 & 0.11 & 12135.00 & 114685.90 & 0.67 & 1.42 & 0.95 \\
24169 & 103294 & 2005 & 3683.40 & -0.04 & 367606.00 & 3533562.33 & 1.00 & 0.96 & 0.96 \\
25764 & 103521 & 2005 & 7079.30 & 0.07 & 708559.00 & 7079382.03 & 1.00 & 1.00 & 1.00 \\
39087 & 107604 & 2005 & 843.90 & 0.07 & 81217.00 & 822109.49 & 1.04 & 0.97 & 1.01 \\
48015 & 226438 & 2005 & 552.70 & 0.05 & 54810.00 & 492106.55 & 1.01 & 0.89 & 0.90 \\
39062 & 107598 & 2005 & 142.00 & 0.02 & 14170.00 & 140966.70 & 1.00 & 0.99 & 0.99 \\
44830 & 109393 & 2005 & 19.70 & 0.13 & 1734.00 & 17518.94 & 1.14 & 0.89 & 1.01 \\
1108 & 100153 & 2005 & 111.20 & 0.08 & 11097.00 & 109807.13 & 1.00 & 0.99 & 0.99 \\
4970 & 100697 & 2005 & 995.70 & 0.03 & 90642.00 & 906416.87 & 1.10 & 0.91 & 1.00 \\
40213 & 108073 & 2005 & 310.10 & 0.16 & 58604.00 & 569268.56 & 0.53 & 1.84 & 0.97 \\
25700 & 103514 & 2005 & 4289.80 & 0.04 & 435174.00 & 4281476.84 & 0.99 & 1.00 & 0.98 \\
4944 & 100695 & 2005 & 211.60 & 0.05 & 20862.00 & 177343.04 & 1.01 & 0.84 & 0.85 \\
7051 & 100992 & 2005 & 717.20 & 0.12 & 71529.00 & 708719.97 & 1.00 & 0.99 & 0.99 \\
40157 & 108051 & 2005 & 208.20 & 0.06 & 21689.00 & 215103.88 & 0.96 & 1.03 & 0.99 \\
24270 & 103301 & 2005 & 741.70 & 0.05 & 75002.00 & 744231.82 & 0.99 & 1.00 & 0.99 \\
39214 & 107619 & 2005 & 349.80 & 0.05 & 37938.00 & 352261.95 & 0.92 & 1.01 & 0.93 \\
53813 & 357133 & 2005 & 329.10 & 0.10 & 33039.00 & 324053.65 & 1.00 & 0.98 & 0.98 \\
54060 & 364292 & 2005 & 383.30 & 0.03 & 43175.00 & 389031.73 & 0.89 & 1.01 & 0.90 \\
7983 & 101068 & 2005 & 51181.00 & 0.06 & 5243768.00 & 52679601.66 & 0.98 & 1.03 & 1.00 \\
53793 & 357122 & 2005 & 1013.20 & 0.06 & 101306.00 & 984862.48 & 1.00 & 0.97 & 0.97 \\
58429 & 410157 & 2005 & 57.40 & 0.09 & 7777.00 & 66060.54 & 0.74 & 1.15 & 0.85 \\
25634 & 103498 & 2005 & 441.10 & 0.11 & 40269.00 & 411976.26 & 1.10 & 0.93 & 1.02 \\
45385 & 200055 & 2005 & 145.10 & 0.05 & 14546.00 & 142428.09 & 1.00 & 0.98 & 0.98 \\
39200 & 107618 & 2005 & 2816.80 & 0.05 & 276540.00 & 2710058.64 & 1.02 & 0.96 & 0.98 \\
40167 & 108070 & 2005 & 57.00 & 0.10 & 5635.00 & 55187.22 & 1.01 & 0.97 & 0.98 \\
55398 & 400093 & 2005 & 153.60 & 0.08 & 23278.00 & 215880.83 & 0.66 & 1.41 & 0.93 \\
39175 & 107616 & 2005 & 1301.80 & 0.11 & 130651.00 & 1296802.90 & 1.00 & 1.00 & 0.99 \\
4729 & 100669 & 2005 & 260.80 & 0.07 & 26412.00 & 256275.39 & 0.99 & 0.98 & 0.97 \\
65402 & 500694 & 2005 & 807.40 & 2.41 & 79527.00 & 798893.33 & 1.02 & 0.99 & 1.00 \\
12483 & 101542 & 2005 & 119.80 & 0.09 & 7922.00 & 67967.81 & 1.51 & 0.57 & 0.86 \\
65441 & 500697 & 2005 & 18.10 & 0.10 & 2089.00 & 16341.93 & 0.87 & 0.90 & 0.78 \\
53780 & 357075 & 2005 & 476.40 & 0.07 & 47907.00 & 479060.05 & 0.99 & 1.01 & 1.00 \\
17713 & 102349 & 2005 & 732.30 & 0.06 & 73048.00 & 730464.67 & 1.00 & 1.00 & 1.00 \\
1988 & 100278 & 2005 & 31.90 & 0.06 & 3452.00 & 30537.45 & 0.92 & 0.96 & 0.88 \\
1158 & 100157 & 2005 & 886.00 & 0.04 & 85365.00 & 852652.01 & 1.04 & 0.96 & 1.00 \\
64485 & 500605 & 2005 & 812.40 & 0.34 & 81243.00 & 811739.73 & 1.00 & 1.00 & 1.00 \\
39150 & 107611 & 2005 & 5614.10 & 0.10 & 547187.00 & 5449934.49 & 1.03 & 0.97 & 1.00 \\
65466 & 500699 & 2005 & 279.60 & 0.02 & 27960.00 & 278572.63 & 1.00 & 1.00 & 1.00 \\
1143 & 100155 & 2005 & 1358.60 & 0.05 & 148687.00 & 1457638.42 & 0.91 & 1.07 & 0.98 \\
40187 & 108071 & 2005 & 118.50 & 0.09 & 10849.00 & 112624.54 & 1.09 & 0.95 & 1.04 \\
24239 & 103299 & 2005 & 245.50 & 0.05 & 24316.00 & 243175.02 & 1.01 & 0.99 & 1.00 \\
58077 & 410075 & 2005 & 1144.70 & 0.54 & 114742.00 & 1110930.77 & 1.00 & 0.97 & 0.97 \\
54091 & 364393 & 2005 & 207.70 & 0.08 & 19083.00 & 191699.91 & 1.09 & 0.92 & 1.00 \\
5003 & 100698 & 2005 & 87.90 & 0.05 & 8768.00 & 87680.44 & 1.00 & 1.00 & 1.00 \\
58048 & 410063 & 2005 & 2839.70 & 0.09 & 283967.00 & 2839305.56 & 1.00 & 1.00 & 1.00 \\
53774 & 357053 & 2005 & 59.20 & 0.01 & 5898.00 & 58976.07 & 1.00 & 1.00 & 1.00 \\
40347 & 108112 & 2005 & 24.90 & 0.13 & 2488.00 & 24913.56 & 1.00 & 1.00 & 1.00 \\
25928 & 103529 & 2005 & 7356.70 & 0.06 & 732915.00 & 7240059.40 & 1.00 & 0.98 & 0.99 \\
24060 & 103259 & 2005 & 2924.80 & 0.05 & 309512.00 & 2856340.60 & 0.94 & 0.98 & 0.92 \\
38886 & 107350 & 2005 & 3696.00 & 0.03 & 382466.00 & 3588904.99 & 0.97 & 0.97 & 0.94 \\
45480 & 200065 & 2005 & 14.50 & 0.04 & 1454.00 & 13571.05 & 1.00 & 0.94 & 0.93 \\
18283 & 102424 & 2005 & 534.00 & 0.31 & 43590.00 & 425230.77 & 1.23 & 0.80 & 0.98 \\
54114 & 364518 & 2005 & 131.90 & 0.01 & 12858.00 & 126628.55 & 1.03 & 0.96 & 0.98 \\
44807 & 109392 & 2005 & 92.20 & 0.14 & 8209.00 & 67824.34 & 1.12 & 0.74 & 0.83 \\
65574 & 500706 & 2005 & 1695.50 & 0.13 & 147983.00 & 1546684.77 & 1.15 & 0.91 & 1.05 \\
65597 & 500707 & 2005 & 1312.20 & 0.12 & 107011.00 & 1136226.41 & 1.23 & 0.87 & 1.06 \\
40373 & 108115 & 2005 & 2232.80 & 0.07 & 225429.00 & 2198349.20 & 0.99 & 0.98 & 0.98 \\
14233 & 101834 & 2005 & 176.30 & 0.08 & 17717.00 & 185995.87 & 1.00 & 1.05 & 1.05 \\
44801 & 109389 & 2005 & 102.60 & 0.08 & 11998.00 & 119979.51 & 0.86 & 1.17 & 1.00 \\
25966 & 103531 & 2005 & 8244.10 & 0.45 & 828161.00 & 7363594.14 & 1.00 & 0.89 & 0.89 \\
24038 & 103255 & 2005 & 143.50 & 0.05 & 14776.00 & 138427.79 & 0.97 & 0.96 & 0.94 \\
65620 & 500708 & 2005 & 2012.10 & 0.13 & 172141.00 & 1773596.06 & 1.17 & 0.88 & 1.03 \\
1043 & 100128 & 2005 & 245.50 & 0.16 & 24548.00 & 198520.93 & 1.00 & 0.81 & 0.81 \\
2040 & 100286 & 2005 & 42.90 & 0.04 & 4303.00 & 41589.34 & 1.00 & 0.97 & 0.97 \\
40399 & 108117 & 2005 & 536.00 & 0.06 & 53758.00 & 482197.53 & 1.00 & 0.90 & 0.90 \\
17588 & 102319 & 2005 & 554.60 & 0.06 & 53398.00 & 517336.64 & 1.04 & 0.93 & 0.97 \\
12552 & 101553 & 2005 & 152.80 & 0.11 & 15389.00 & 148640.33 & 0.99 & 0.97 & 0.97 \\
40321 & 108109 & 2005 & 108.90 & 0.05 & 10255.00 & 99764.80 & 1.06 & 0.92 & 0.97 \\
38910 & 107352 & 2005 & 338.50 & 0.18 & 34238.00 & 335221.00 & 0.99 & 0.99 & 0.98 \\
11961 & 101473 & 2005 & 882.90 & 0.03 & 91629.00 & 917339.48 & 0.96 & 1.04 & 1.00 \\
38927 & 107354 & 2005 & 157.10 & 0.12 & 15887.00 & 144346.43 & 0.99 & 0.92 & 0.91 \\
39033 & 107573 & 2005 & 180.80 & 0.04 & 18399.00 & 177102.96 & 0.98 & 0.98 & 0.96 \\
64446 & 500603 & 2005 & 472.10 & 0.06 & 47163.00 & 470412.47 & 1.00 & 1.00 & 1.00 \\
8214 & 101079 & 2005 & 2281.00 & 0.52 & 220740.00 & 2200958.34 & 1.03 & 0.96 & 1.00 \\
40265 & 108083 & 2005 & 263.90 & 0.02 & 21120.00 & 222892.13 & 1.25 & 0.84 & 1.06 \\
49411 & 240293 & 2005 & 2909.30 & 0.07 & 285840.00 & 2623058.80 & 1.02 & 0.90 & 0.92 \\
48006 & 225696 & 2005 & 70.00 & -0.04 & 10262.00 & 69620.43 & 0.68 & 0.99 & 0.68 \\
58436 & 410158 & 2005 & 187.20 & 0.04 & 18720.00 & 170311.34 & 1.00 & 0.91 & 0.91 \\
25832 & 103524 & 2005 & 107403.30 & 0.05 & 10738002.00 & 106591007.55 & 1.00 & 0.99 & 0.99 \\
38996 & 107563 & 2005 & 1950.60 & 0.04 & 203894.00 & 1895938.98 & 0.96 & 0.97 & 0.93 \\
40295 & 108087 & 2005 & 23.00 & 0.06 & 3099.00 & 26141.58 & 0.74 & 1.14 & 0.84 \\
24119 & 103267 & 2005 & 4390.90 & 0.03 & 434949.00 & 4349104.81 & 1.01 & 0.99 & 1.00 \\
58457 & 410159 & 2005 & 39.50 & 0.05 & 4283.00 & 42589.67 & 0.92 & 1.08 & 0.99 \\
65512 & 500701 & 2005 & 414.90 & 0.13 & 41494.00 & 414766.57 & 1.00 & 1.00 & 1.00 \\
12523 & 101545 & 2005 & 344.40 & 0.01 & 32610.00 & 327089.17 & 1.06 & 0.95 & 1.00 \\
25798 & 103523 & 2005 & 10549.20 & 0.10 & 1053614.00 & 10425586.46 & 1.00 & 0.99 & 0.99 \\
49419 & 240295 & 2005 & 3012.50 & 0.06 & 290140.00 & 2790000.63 & 1.04 & 0.93 & 0.96 \\
25865 & 103525 & 2005 & 56249.30 & 0.05 & 5619619.00 & 56041319.35 & 1.00 & 1.00 & 1.00 \\
65535 & 500702 & 2005 & 247.70 & -0.05 & 24778.00 & 242040.46 & 1.00 & 0.98 & 0.98 \\
4708 & 100667 & 2005 & 16.50 & 0.06 & 1652.00 & 15675.37 & 1.00 & 0.95 & 0.95 \\
24103 & 103266 & 2005 & 134.90 & 0.12 & 17960.00 & 186443.79 & 0.75 & 1.38 & 1.04 \\
17625 & 102321 & 2005 & 170.20 & 0.08 & 16851.00 & 156742.59 & 1.01 & 0.92 & 0.93 \\
38952 & 107358 & 2005 & 671.40 & 0.11 & 67245.00 & 610788.00 & 1.00 & 0.91 & 0.91 \\
55251 & 400075 & 2005 & 1534.80 & 0.15 & 150911.00 & 1505839.25 & 1.02 & 0.98 & 1.00 \\
54082 & 364391 & 2005 & 56.50 & 0.09 & 5762.00 & 55912.04 & 0.98 & 0.99 & 0.97 \\
1078 & 100150 & 2005 & 53.70 & 0.12 & 4933.00 & 49308.69 & 1.09 & 0.92 & 1.00 \\
64423 & 500602 & 2005 & 433.00 & 0.07 & 42001.00 & 431733.99 & 1.03 & 1.00 & 1.03 \\
49427 & 240296 & 2005 & 1593.90 & 0.12 & 167219.00 & 1569646.22 & 0.95 & 0.98 & 0.94 \\
25899 & 103526 & 2005 & 6098.40 & 0.09 & 627244.00 & 5649572.89 & 0.97 & 0.93 & 0.90 \\
45448 & 200060 & 2005 & 2278.30 & 0.07 & 229556.00 & 2185776.23 & 0.99 & 0.96 & 0.95 \\
18727 & 102504 & 2005 & 906.80 & 0.03 & 96322.00 & 830946.27 & 0.94 & 0.92 & 0.86 \\
4533 & 100637 & 2005 & 868.40 & 0.16 & 83430.00 & 833843.82 & 1.04 & 0.96 & 1.00 \\
53704 & 355988 & 2005 & 61.80 & 0.01 & 7071.00 & 62218.43 & 0.87 & 1.01 & 0.88 \\
44450 & 109325 & 2005 & 685.20 & 0.01 & 68417.00 & 683854.01 & 1.00 & 1.00 & 1.00 \\
28544 & 105437 & 2005 & 834.20 & 0.04 & 82504.00 & 819072.37 & 1.01 & 0.98 & 0.99 \\
29282 & 105581 & 2005 & 105.70 & -0.02 & 11896.00 & 110773.81 & 0.89 & 1.05 & 0.93 \\
63295 & 500493 & 2005 & 1540.80 & -0.00 & 130713.00 & 1313013.41 & 1.18 & 0.85 & 1.00 \\
50218 & 240405 & 2005 & 37.20 & 0.03 & 2445.00 & 24766.42 & 1.52 & 0.67 & 1.01 \\
55611 & 400130 & 2005 & 357.00 & 0.03 & 34428.00 & 343665.93 & 1.04 & 0.96 & 1.00 \\
41761 & 108853 & 2005 & 186.10 & 0.17 & 18666.00 & 151600.76 & 1.00 & 0.81 & 0.81 \\
74780 & 601168 & 2005 & 35.90 & 0.01 & 4458.00 & 36245.46 & 0.81 & 1.01 & 0.81 \\
54647 & 377385 & 2005 & 378.50 & 0.04 & 37805.00 & 370067.65 & 1.00 & 0.98 & 0.98 \\
395 & 100048 & 2005 & 91.60 & 0.04 & 9236.00 & 89500.86 & 0.99 & 0.98 & 0.97 \\
41769 & 108855 & 2005 & 597.70 & 0.04 & 82409.00 & 735922.33 & 0.73 & 1.23 & 0.89 \\
28574 & 105444 & 2005 & 68.30 & 0.04 & 6548.00 & 65473.50 & 1.04 & 0.96 & 1.00 \\
55144 & 400063 & 2005 & 11.60 & 0.06 & 1128.00 & 10681.07 & 1.03 & 0.92 & 0.95 \\
22338 & 103007 & 2005 & 1125.60 & 0.07 & 112448.00 & 1128876.86 & 1.00 & 1.00 & 1.00 \\
7245 & 101015 & 2005 & 1844.80 & 0.02 & 188017.00 & 1805252.72 & 0.98 & 0.98 & 0.96 \\
2757 & 100357 & 2005 & 299.30 & 0.14 & 36482.00 & 317309.16 & 0.82 & 1.06 & 0.87 \\
42010 & 108907 & 2005 & 981.40 & 0.12 & 96853.00 & 889022.39 & 1.01 & 0.91 & 0.92 \\
44477 & 109327 & 2005 & 992.80 & 0.06 & 92088.00 & 882082.00 & 1.08 & 0.89 & 0.96 \\
35928 & 106434 & 2005 & 517.50 & 0.05 & 52495.00 & 514480.11 & 0.99 & 0.99 & 0.98 \\
36559 & 106561 & 2005 & 14.50 & 0.06 & 1571.00 & 13581.88 & 0.92 & 0.94 & 0.86 \\
50695 & 240440 & 2005 & 765.00 & 0.04 & 84161.00 & 841000.24 & 0.91 & 1.10 & 1.00 \\
57274 & 400389 & 2005 & 5.10 & 0.45 & 510.00 & 4127.27 & 1.00 & 0.81 & 0.81 \\
63318 & 500494 & 2005 & 1752.30 & 0.17 & 206848.00 & 1920921.53 & 0.85 & 1.10 & 0.93 \\
41730 & 108849 & 2005 & 1551.20 & 0.08 & 155245.00 & 1551375.34 & 1.00 & 1.00 & 1.00 \\
35902 & 106424 & 2005 & 421.50 & 0.06 & 42193.00 & 415870.73 & 1.00 & 0.99 & 0.99 \\
5829 & 100804 & 2005 & 4472.10 & 0.04 & 475253.00 & 4424524.15 & 0.94 & 0.99 & 0.93 \\
29308 & 105585 & 2005 & 148.60 & 0.08 & 11663.00 & 122677.46 & 1.27 & 0.83 & 1.05 \\
57325 & 400396 & 2005 & 3.50 & 0.17 & 384.00 & 2845.54 & 0.91 & 0.81 & 0.74 \\
55101 & 400061 & 2005 & 2964.70 & 0.06 & 350151.00 & 2860721.80 & 0.85 & 0.96 & 0.82 \\
10835 & 101334 & 2005 & 404.60 & -0.03 & 40455.00 & 363471.32 & 1.00 & 0.90 & 0.90 \\
36540 & 106560 & 2005 & 103.50 & 0.05 & 10246.00 & 101493.10 & 1.01 & 0.98 & 0.99 \\
54580 & 376139 & 2005 & 16.00 & 0.14 & 1599.00 & 15256.72 & 1.00 & 0.95 & 0.95 \\
22382 & 103008 & 2005 & 196.20 & 0.03 & 19025.00 & 192098.58 & 1.03 & 0.98 & 1.01 \\
28522 & 105432 & 2005 & 34.50 & 0.00 & 3454.00 & 33853.08 & 1.00 & 0.98 & 0.98 \\
63009 & 500466 & 2005 & 1023.00 & 0.06 & 101935.00 & 947990.03 & 1.00 & 0.93 & 0.93 \\
6840 & 100962 & 2005 & 11032.50 & 0.09 & 1107178.00 & 10415530.79 & 1.00 & 0.94 & 0.94 \\
6774 & 100953 & 2005 & 540.10 & 0.09 & 59097.00 & 590965.08 & 0.91 & 1.09 & 1.00 \\
42143 & 108929 & 2005 & 282.90 & 0.04 & 27730.00 & 278966.69 & 1.02 & 0.99 & 1.01 \\
29269 & 105574 & 2005 & 257.70 & 0.03 & 25643.00 & 252614.19 & 1.00 & 0.98 & 0.99 \\
36474 & 106535 & 2005 & 606.30 & 0.07 & 65015.00 & 552683.56 & 0.93 & 0.91 & 0.85 \\
8572 & 101090 & 2005 & 544.30 & 0.03 & 55671.00 & 483888.21 & 0.98 & 0.89 & 0.87 \\
22302 & 103005 & 2005 & 1385.30 & 0.04 & 132588.00 & 1328642.31 & 1.04 & 0.96 & 1.00 \\
53243 & 342448 & 2005 & 146.00 & 0.04 & 14776.00 & 132793.96 & 0.99 & 0.91 & 0.90 \\
44261 & 109279 & 2005 & 88.90 & 0.12 & 8474.00 & 80213.29 & 1.05 & 0.90 & 0.95 \\
46306 & 200210 & 2005 & 7.70 & 0.03 & 783.00 & 7128.41 & 0.98 & 0.93 & 0.91 \\
16367 & 102130 & 2005 & 240.90 & 0.06 & 25697.00 & 250754.74 & 0.94 & 1.04 & 0.98 \\
28640 & 105457 & 2005 & 2582.20 & 0.05 & 270764.00 & 2680955.78 & 0.95 & 1.04 & 0.99 \\
7615 & 101048 & 2005 & 7499.80 & 0.05 & 765222.00 & 7310051.87 & 0.98 & 0.97 & 0.96 \\
46426 & 200243 & 2005 & 2.30 & 0.15 & 228.00 & 2117.27 & 1.01 & 0.92 & 0.93 \\
63233 & 500490 & 2005 & 2896.30 & 0.40 & 343744.00 & 3211369.09 & 0.84 & 1.11 & 0.93 \\
21914 & 102979 & 2005 & 50.60 & 0.09 & 5046.00 & 48966.98 & 1.00 & 0.97 & 0.97 \\
50308 & 240410 & 2005 & 33.00 & 0.07 & 1908.00 & 19645.24 & 1.73 & 0.60 & 1.03 \\
10867 & 101340 & 2005 & 28132.70 & 0.11 & 2722635.00 & 22381230.28 & 1.03 & 0.80 & 0.82 \\
42132 & 108925 & 2005 & 320.20 & 0.22 & 30801.00 & 331386.65 & 1.04 & 1.03 & 1.08 \\
16566 & 102156 & 2005 & 88.60 & 0.26 & 9225.00 & 83086.10 & 0.96 & 0.94 & 0.90 \\
48731 & 240134 & 2005 & 313.70 & 0.10 & 33981.00 & 342957.11 & 0.92 & 1.09 & 1.01 \\
18854 & 102524 & 2005 & 3391.60 & 0.04 & 344593.00 & 3382521.68 & 0.98 & 1.00 & 0.98 \\
29242 & 105561 & 2005 & 874.60 & 0.14 & 86925.00 & 853338.33 & 1.01 & 0.98 & 0.98 \\
41829 & 108858 & 2005 & 29.10 & 0.08 & 2908.00 & 29270.06 & 1.00 & 1.01 & 1.01 \\
28590 & 105448 & 2005 & 1388.30 & 0.14 & 138313.00 & 1340427.86 & 1.00 & 0.97 & 0.97 \\
63256 & 500491 & 2005 & 1871.70 & 0.06 & 157448.00 & 1684567.62 & 1.19 & 0.90 & 1.07 \\
36514 & 106545 & 2005 & 19.10 & 0.04 & 1758.00 & 15966.17 & 1.09 & 0.84 & 0.91 \\
41788 & 108856 & 2005 & 304.10 & 0.07 & 31537.00 & 301708.34 & 0.96 & 0.99 & 0.96 \\
5715 & 100789 & 2005 & 6.60 & 0.12 & 771.00 & 7206.18 & 0.86 & 1.09 & 0.93 \\
50661 & 240437 & 2005 & 1626.90 & 0.04 & 153497.00 & 1516070.74 & 1.06 & 0.93 & 0.99 \\
36502 & 106541 & 2005 & 368.30 & 0.03 & 36438.00 & 363611.84 & 1.01 & 0.99 & 1.00 \\
13019 & 101621 & 2005 & 1780.10 & 0.05 & 188454.00 & 1884477.15 & 0.94 & 1.06 & 1.00 \\
2999 & 100395 & 2005 & 694.10 & 0.05 & 82235.00 & 782310.28 & 0.84 & 1.13 & 0.95 \\
35938 & 106441 & 2005 & 1636.00 & 0.25 & 167087.00 & 1632326.20 & 0.98 & 1.00 & 0.98 \\
55616 & 400131 & 2005 & 277.60 & 0.07 & 27903.00 & 278190.57 & 0.99 & 1.00 & 1.00 \\
11093 & 101367 & 2005 & 739.40 & 0.06 & 73655.00 & 677441.80 & 1.00 & 0.92 & 0.92 \\
35962 & 106442 & 2005 & 6066.70 & 0.05 & 553276.00 & 5497908.90 & 1.10 & 0.91 & 0.99 \\
46281 & 200207 & 2005 & 37.10 & 0.03 & 3614.00 & 36645.43 & 1.03 & 0.99 & 1.01 \\
28618 & 105450 & 2005 & 22.60 & 0.17 & 2497.00 & 22159.29 & 0.91 & 0.98 & 0.89 \\
53238 & 342127 & 2005 & 48.00 & 0.07 & 5166.00 & 50639.88 & 0.93 & 1.05 & 0.98 \\
41810 & 108857 & 2005 & 41.60 & 0.11 & 4313.00 & 41994.51 & 0.96 & 1.01 & 0.97 \\
57272 & 400388 & 2005 & 3.60 & 0.15 & 317.00 & 2953.03 & 1.14 & 0.82 & 0.93 \\
19052 & 102545 & 2005 & 2222.60 & 0.03 & 222134.00 & 2142542.85 & 1.00 & 0.96 & 0.96 \\
50131 & 240395 & 2005 & 185.20 & 0.16 & 17365.00 & 179405.79 & 1.07 & 0.97 & 1.03 \\
57246 & 400323 & 2005 & 291.70 & -0.02 & 29174.00 & 270583.28 & 1.00 & 0.93 & 0.93 \\
7563 & 101045 & 2005 & 12121.70 & 0.05 & 1239653.00 & 11989265.55 & 0.98 & 0.99 & 0.97 \\
36693 & 106577 & 2005 & 3701.40 & 0.09 & 346690.00 & 3369551.06 & 1.07 & 0.91 & 0.97 \\
54674 & 378072 & 2005 & 85.60 & 0.12 & 8869.00 & 83509.70 & 0.97 & 0.98 & 0.94 \\
41681 & 108839 & 2005 & 159.60 & -0.01 & 15980.00 & 156843.64 & 1.00 & 0.98 & 0.98 \\
16313 & 102124 & 2005 & 1299.70 & 0.07 & 136478.00 & 1299670.21 & 0.95 & 1.00 & 0.95 \\
3033 & 100399 & 2005 & 389.20 & 0.06 & 38999.00 & 377386.20 & 1.00 & 0.97 & 0.97 \\
35835 & 106415 & 2005 & 206.10 & 0.07 & 20412.00 & 204112.03 & 1.01 & 0.99 & 1.00 \\
59054 & 410418 & 2005 & 1102.80 & 0.07 & 142650.00 & 1416866.86 & 0.77 & 1.28 & 0.99 \\
46465 & 200246 & 2005 & 191.00 & 0.17 & 19145.00 & 177405.18 & 1.00 & 0.93 & 0.93 \\
46272 & 200205 & 2005 & 1200.20 & 0.54 & 128251.00 & 1086481.29 & 0.94 & 0.91 & 0.85 \\
41706 & 108840 & 2005 & 495.30 & 0.04 & 51364.00 & 502339.15 & 0.96 & 1.01 & 0.98 \\
10805 & 101331 & 2005 & 74.40 & 0.09 & 7443.00 & 74042.35 & 1.00 & 1.00 & 0.99 \\
28406 & 105421 & 2005 & 17.50 & 0.04 & 2369.00 & 17615.20 & 0.74 & 1.01 & 0.74 \\
55581 & 400127 & 2005 & 143.50 & 0.05 & 14191.00 & 141915.84 & 1.01 & 0.99 & 1.00 \\
28380 & 105420 & 2005 & 44.80 & 0.19 & 4576.00 & 45014.61 & 0.98 & 1.00 & 0.98 \\
42192 & 108933 & 2005 & 180.80 & 0.10 & 17283.00 & 166924.21 & 1.05 & 0.92 & 0.97 \\
47691 & 220681 & 2005 & 431.70 & 0.02 & 45412.00 & 394408.97 & 0.95 & 0.91 & 0.87 \\
50762 & 240448 & 2005 & 86.50 & -0.01 & 10084.00 & 100576.76 & 0.86 & 1.16 & 1.00 \\
55173 & 400065 & 2005 & 116.40 & 0.01 & 11642.00 & 116405.97 & 1.00 & 1.00 & 1.00 \\
28354 & 105419 & 2005 & 46.50 & 0.05 & 5188.00 & 48165.67 & 0.90 & 1.04 & 0.93 \\
42167 & 108932 & 2005 & 2121.70 & 0.04 & 213518.00 & 1871295.49 & 0.99 & 0.88 & 0.88 \\
35791 & 106402 & 2005 & 225.70 & 0.09 & 20051.00 & 194536.54 & 1.13 & 0.86 & 0.97 \\
36735 & 106583 & 2005 & 39.10 & 0.06 & 3755.00 & 37202.87 & 1.04 & 0.95 & 0.99 \\
50158 & 240397 & 2005 & 17.00 & 0.15 & 1709.00 & 16339.56 & 0.99 & 0.96 & 0.96 \\
7645 & 101050 & 2005 & 898.40 & 0.11 & 85353.00 & 841064.22 & 1.05 & 0.94 & 0.99 \\
53435 & 349198 & 2005 & 904.60 & 0.04 & 91039.00 & 867133.11 & 0.99 & 0.96 & 0.95 \\
3059 & 100401 & 2005 & 1372.70 & 0.04 & 137181.00 & 1351907.28 & 1.00 & 0.98 & 0.99 \\
35809 & 106413 & 2005 & 5054.50 & 0.09 & 576967.00 & 5769568.57 & 0.88 & 1.14 & 1.00 \\
50093 & 240392 & 2005 & 413.50 & 0.03 & 41533.00 & 392616.14 & 1.00 & 0.95 & 0.95 \\
36719 & 106580 & 2005 & 130.60 & 0.08 & 13163.00 & 130066.31 & 0.99 & 1.00 & 0.99 \\
63435 & 500508 & 2005 & 7633.60 & 0.02 & 740479.00 & 7601568.62 & 1.03 & 1.00 & 1.03 \\
4361 & 100611 & 2005 & 385.40 & 0.11 & 32904.00 & 324770.32 & 1.17 & 0.84 & 0.99 \\
22473 & 103014 & 2005 & 232.20 & 0.05 & 24180.00 & 237879.97 & 0.96 & 1.02 & 0.98 \\
55576 & 400126 & 2005 & 34.30 & 0.01 & 3397.00 & 33423.03 & 1.01 & 0.97 & 0.98 \\
36658 & 106573 & 2005 & 8.20 & 0.04 & 816.00 & 7918.30 & 1.00 & 0.97 & 0.97 \\
21835 & 102954 & 2005 & 568.60 & 0.20 & 48092.00 & 452371.78 & 1.18 & 0.80 & 0.94 \\
46432 & 200244 & 2005 & 6.20 & 0.09 & 644.00 & 6137.61 & 0.96 & 0.99 & 0.95 \\
19028 & 102544 & 2005 & 1288.40 & 0.09 & 127098.00 & 1270916.54 & 1.01 & 0.99 & 1.00 \\
28464 & 105426 & 2005 & 1064.80 & 0.07 & 108390.00 & 1043543.73 & 0.98 & 0.98 & 0.96 \\
35870 & 106420 & 2005 & 125.90 & 0.02 & 12937.00 & 118477.40 & 0.97 & 0.94 & 0.92 \\
50709 & 240441 & 2005 & 1763.00 & 0.05 & 179790.00 & 1790720.11 & 0.98 & 1.02 & 1.00 \\
18823 & 102523 & 2005 & 2754.50 & 0.12 & 266137.00 & 2310378.03 & 1.03 & 0.84 & 0.87 \\
63405 & 500506 & 2005 & 12.60 & 0.02 & 1190.00 & 9547.38 & 1.06 & 0.76 & 0.80 \\
5683 & 100785 & 2005 & 1493.40 & 0.02 & 160580.00 & 1618634.59 & 0.93 & 1.08 & 1.01 \\
21859 & 102957 & 2005 & 412.70 & 0.08 & 40895.00 & 416596.84 & 1.01 & 1.01 & 1.02 \\
42154 & 108930 & 2005 & 145.80 & 0.02 & 15816.00 & 150563.95 & 0.92 & 1.03 & 0.95 \\
55607 & 400129 & 2005 & 593.90 & 0.07 & 63970.00 & 615054.14 & 0.93 & 1.04 & 0.96 \\
29334 & 105587 & 2005 & 264.20 & 0.11 & 25034.00 & 250786.40 & 1.06 & 0.95 & 1.00 \\
57253 & 400387 & 2005 & 176.60 & 0.04 & 17638.00 & 154402.96 & 1.00 & 0.87 & 0.88 \\
35878 & 106421 & 2005 & 56.40 & 0.13 & 8917.00 & 89146.90 & 0.63 & 1.58 & 1.00 \\
63354 & 500500 & 2005 & 547.30 & 0.11 & 52540.00 & 531183.45 & 1.04 & 0.97 & 1.01 \\
28493 & 105427 & 2005 & 240.80 & 0.19 & 24048.00 & 228555.06 & 1.00 & 0.95 & 0.95 \\
50164 & 240398 & 2005 & 1012.10 & 0.15 & 104643.00 & 1028151.90 & 0.97 & 1.02 & 0.98 \\
28655 & 105458 & 2005 & 547.50 & 0.04 & 54871.00 & 534302.28 & 1.00 & 0.98 & 0.97 \\
53411 & 348766 & 2005 & 945.00 & 0.07 & 97818.00 & 978052.09 & 0.97 & 1.03 & 1.00 \\
36632 & 106571 & 2005 & 544.80 & 0.11 & 48468.00 & 473466.45 & 1.12 & 0.87 & 0.98 \\
50753 & 240447 & 2005 & 61.40 & 0.07 & 6336.00 & 60274.32 & 0.97 & 0.98 & 0.95 \\
35851 & 106418 & 2005 & 1678.40 & 0.07 & 167745.00 & 1632551.32 & 1.00 & 0.97 & 0.97 \\
16663 & 102175 & 2005 & 480.30 & 0.11 & 47834.00 & 445320.74 & 1.00 & 0.93 & 0.93 \\
11163 & 101369 & 2005 & 1142.90 & 0.10 & 115246.00 & 1175096.06 & 0.99 & 1.03 & 1.02 \\
29366 & 105589 & 2005 & 242.00 & 0.03 & 24195.00 & 241437.08 & 1.00 & 1.00 & 1.00 \\
54576 & 375967 & 2005 & 55.80 & -0.02 & 9944.00 & 103604.94 & 0.56 & 1.86 & 1.04 \\
422 & 100055 & 2005 & 19205.50 & 0.03 & 1878310.00 & 19270483.75 & 1.02 & 1.00 & 1.03 \\
28435 & 105424 & 2005 & 3686.00 & 0.07 & 375205.00 & 3581405.04 & 0.98 & 0.97 & 0.95 \\
44238 & 109278 & 2005 & 34.30 & 0.05 & 3433.00 & 33721.23 & 1.00 & 0.98 & 0.98 \\
55602 & 400128 & 2005 & 83.20 & 0.06 & 8370.00 & 79085.47 & 0.99 & 0.95 & 0.94 \\
54668 & 377933 & 2005 & 31.00 & 0.13 & 3894.00 & 38939.16 & 0.80 & 1.26 & 1.00 \\
48217 & 240051 & 2005 & 768.70 & 0.06 & 78053.00 & 765284.32 & 0.98 & 1.00 & 0.98 \\
36597 & 106568 & 2005 & 7037.50 & 0.09 & 719187.00 & 6757620.46 & 0.98 & 0.96 & 0.94 \\
2726 & 100355 & 2005 & 4880.90 & 0.04 & 530919.00 & 4908320.00 & 0.92 & 1.01 & 0.92 \\
13161 & 101681 & 2005 & 378.40 & 0.05 & 41074.00 & 381096.13 & 0.92 & 1.01 & 0.93 \\
46455 & 200245 & 2005 & 27.50 & 0.10 & 2557.00 & 26174.48 & 1.08 & 0.95 & 1.02 \\
29444 & 105595 & 2005 & 19.30 & 0.03 & 1589.00 & 16088.06 & 1.21 & 0.83 & 1.01 \\
35990 & 106444 & 2005 & 240.60 & 0.06 & 24133.00 & 240074.82 & 1.00 & 1.00 & 0.99 \\
22281 & 102999 & 2005 & 77.60 & -0.04 & 8148.00 & 81204.49 & 0.95 & 1.05 & 1.00 \\
28859 & 105487 & 2005 & 50.40 & 0.07 & 7785.00 & 51552.89 & 0.65 & 1.02 & 0.66 \\
10989 & 101358 & 2005 & 413.30 & 0.10 & 42372.00 & 423924.84 & 0.98 & 1.03 & 1.00 \\
18916 & 102527 & 2005 & 81.20 & 0.07 & 8924.00 & 92208.28 & 0.91 & 1.14 & 1.03 \\
36134 & 106467 & 2005 & 247.60 & 0.11 & 23319.00 & 235985.27 & 1.06 & 0.95 & 1.01 \\
13077 & 101626 & 2005 & 1219.60 & 0.07 & 121103.00 & 1039897.94 & 1.01 & 0.85 & 0.86 \\
53291 & 342993 & 2005 & 65.20 & 0.06 & 6567.00 & 65191.49 & 0.99 & 1.00 & 0.99 \\
48858 & 240148 & 2005 & 592.70 & 0.11 & 71560.00 & 530846.00 & 0.83 & 0.90 & 0.74 \\
46346 & 200224 & 2005 & 24.40 & 0.11 & 2475.00 & 23602.94 & 0.99 & 0.97 & 0.95 \\
327 & 100036 & 2005 & 20.60 & 0.05 & 2074.00 & 20779.80 & 0.99 & 1.01 & 1.00 \\
28887 & 105502 & 2005 & 1368.60 & 0.06 & 137090.00 & 1147733.66 & 1.00 & 0.84 & 0.84 \\
36143 & 106470 & 2005 & 169.40 & -0.00 & 17848.00 & 169181.63 & 0.95 & 1.00 & 0.95 \\
41951 & 108874 & 2005 & 49.20 & 0.03 & 4832.00 & 46554.47 & 1.02 & 0.95 & 0.96 \\
46354 & 200225 & 2005 & 2.90 & 0.06 & 308.00 & 3158.09 & 0.94 & 1.09 & 1.03 \\
29043 & 105522 & 2005 & 420.30 & 0.12 & 44303.00 & 413914.95 & 0.95 & 0.98 & 0.93 \\
22021 & 102987 & 2005 & 1087.00 & 0.03 & 113810.00 & 1032678.80 & 0.96 & 0.95 & 0.91 \\
22147 & 102993 & 2005 & 9647.90 & 0.10 & 886509.00 & 7840712.03 & 1.09 & 0.81 & 0.88 \\
44330 & 109284 & 2005 & 20.60 & 0.03 & 2063.00 & 20133.83 & 1.00 & 0.98 & 0.98 \\
50454 & 240421 & 2005 & 722.60 & 0.09 & 72817.00 & 684820.36 & 0.99 & 0.95 & 0.94 \\
53337 & 344277 & 2005 & 598.30 & 0.06 & 66277.00 & 590458.85 & 0.90 & 0.99 & 0.89 \\
4289 & 100600 & 2005 & 117.60 & 0.12 & 11777.00 & 117696.93 & 1.00 & 1.00 & 1.00 \\
36339 & 106485 & 2005 & 146.40 & 0.05 & 14633.00 & 142138.98 & 1.00 & 0.97 & 0.97 \\
41894 & 108868 & 2005 & 782.70 & 0.06 & 78038.00 & 778514.81 & 1.00 & 0.99 & 1.00 \\
54643 & 377379 & 2005 & 27.20 & -0.06 & 3127.00 & 29794.06 & 0.87 & 1.10 & 0.95 \\
36108 & 106464 & 2005 & 350.80 & 0.11 & 33629.00 & 305196.13 & 1.04 & 0.87 & 0.91 \\
16500 & 102151 & 2005 & 6.90 & 0.12 & 694.00 & 6763.64 & 0.99 & 0.98 & 0.97 \\
55125 & 400062 & 2005 & 212.60 & 0.07 & 25227.00 & 236068.91 & 0.84 & 1.11 & 0.94 \\
46401 & 200236 & 2005 & 151.00 & 0.11 & 13044.00 & 119135.88 & 1.16 & 0.79 & 0.91 \\
53349 & 344278 & 2005 & 163.80 & 0.10 & 19111.00 & 155899.34 & 0.86 & 0.95 & 0.82 \\
41924 & 108870 & 2005 & 47.20 & 0.01 & 4559.00 & 46720.76 & 1.04 & 0.99 & 1.02 \\
44307 & 109283 & 2005 & 3092.60 & 0.07 & 287879.00 & 2953922.02 & 1.07 & 0.96 & 1.03 \\
36303 & 106482 & 2005 & 148.60 & 0.16 & 14916.00 & 139489.28 & 1.00 & 0.94 & 0.94 \\
16419 & 102134 & 2005 & 54.10 & 0.04 & 7693.00 & 76185.73 & 0.70 & 1.41 & 0.99 \\
57505 & 400428 & 2005 & 29.10 & 0.08 & 2512.00 & 24783.45 & 1.16 & 0.85 & 0.99 \\
29082 & 105525 & 2005 & 5392.00 & 0.08 & 544585.00 & 4418065.41 & 0.99 & 0.82 & 0.81 \\
41979 & 108886 & 2005 & 59.70 & 0.04 & 7784.00 & 59764.44 & 0.77 & 1.00 & 0.77 \\
44338 & 109286 & 2005 & 418.70 & 0.04 & 42293.00 & 418332.95 & 0.99 & 1.00 & 0.99 \\
48808 & 240143 & 2005 & 182.50 & 0.11 & 18757.00 & 175348.09 & 0.97 & 0.96 & 0.93 \\
22052 & 102988 & 2005 & 98.90 & 0.15 & 10045.00 & 96771.07 & 0.98 & 0.98 & 0.96 \\
36221 & 106478 & 2005 & 728.70 & 0.15 & 84053.00 & 833050.52 & 0.87 & 1.14 & 0.99 \\
22080 & 102989 & 2005 & 2179.60 & 0.04 & 221888.00 & 2135898.49 & 0.98 & 0.98 & 0.96 \\
59145 & 410439 & 2005 & 17.00 & 0.04 & 1693.00 & 17376.51 & 1.00 & 1.02 & 1.03 \\
42030 & 108910 & 2005 & 7.40 & 0.08 & 696.00 & 6912.55 & 1.06 & 0.93 & 0.99 \\
8635 & 101092 & 2005 & 305.70 & -0.07 & 31961.00 & 308030.71 & 0.96 & 1.01 & 0.96 \\
28990 & 105510 & 2005 & 9.90 & -0.00 & 1130.00 & 10778.89 & 0.88 & 1.09 & 0.95 \\
28964 & 105508 & 2005 & 56.40 & -0.05 & 5552.00 & 55524.07 & 1.02 & 0.98 & 1.00 \\
16447 & 102145 & 2005 & 312.40 & 0.03 & 30677.00 & 305843.77 & 1.02 & 0.98 & 1.00 \\
36198 & 106477 & 2005 & 1955.40 & 0.06 & 357615.00 & 3365973.45 & 0.55 & 1.72 & 0.94 \\
50475 & 240422 & 2005 & 37.90 & 0.06 & 3782.00 & 38496.95 & 1.00 & 1.02 & 1.02 \\
53311 & 343540 & 2005 & 75.60 & 0.13 & 6264.00 & 61820.12 & 1.21 & 0.82 & 0.99 \\
28945 & 105507 & 2005 & 1264.80 & 0.05 & 141352.00 & 1382944.16 & 0.89 & 1.09 & 0.98 \\
42001 & 108901 & 2005 & 280.60 & 0.08 & 28004.00 & 277575.00 & 1.00 & 0.99 & 0.99 \\
29008 & 105512 & 2005 & 14.90 & 0.12 & 1338.00 & 12024.75 & 1.11 & 0.81 & 0.90 \\
36270 & 106480 & 2005 & 1371.80 & 0.10 & 194513.00 & 1763705.73 & 0.71 & 1.29 & 0.91 \\
46361 & 200227 & 2005 & 16.80 & 0.05 & 1696.00 & 17260.06 & 0.99 & 1.03 & 1.02 \\
10931 & 101354 & 2005 & 1490.20 & 0.05 & 150727.00 & 1466462.10 & 0.99 & 0.98 & 0.97 \\
22113 & 102990 & 2005 & 3767.80 & 0.07 & 386494.00 & 3633484.85 & 0.97 & 0.96 & 0.94 \\
28914 & 105503 & 2005 & 440.30 & 0.08 & 56679.00 & 398312.39 & 0.78 & 0.90 & 0.70 \\
42043 & 108914 & 2005 & 1308.70 & 0.11 & 161232.00 & 1596768.81 & 0.81 & 1.22 & 0.99 \\
42067 & 108915 & 2005 & 17.00 & -0.19 & 2047.00 & 20469.42 & 0.83 & 1.20 & 1.00 \\
16469 & 102150 & 2005 & 59.40 & 0.15 & 5949.00 & 57781.57 & 1.00 & 0.97 & 0.97 \\
18964 & 102531 & 2005 & 53.00 & 0.21 & 4870.00 & 48649.88 & 1.09 & 0.92 & 1.00 \\
28929 & 105506 & 2005 & 828.00 & 0.01 & 82565.00 & 820443.06 & 1.00 & 0.99 & 0.99 \\
14678 & 101908 & 2005 & 193.90 & 0.01 & 19226.00 & 191128.47 & 1.01 & 0.99 & 0.99 \\
36162 & 106474 & 2005 & 233.90 & 0.04 & 22847.00 & 223431.34 & 1.02 & 0.96 & 0.98 \\
36244 & 106479 & 2005 & 167.60 & 0.09 & 29498.00 & 291867.54 & 0.57 & 1.74 & 0.99 \\
18932 & 102528 & 2005 & 101.50 & 0.13 & 9756.00 & 93344.46 & 1.04 & 0.92 & 0.96 \\
5757 & 100791 & 2005 & 7190.80 & 0.07 & 696506.00 & 5896349.30 & 1.03 & 0.82 & 0.85 \\
10963 & 101356 & 2005 & 218.30 & 0.01 & 22602.00 & 224198.88 & 0.97 & 1.03 & 0.99 \\
50449 & 240419 & 2005 & 31.60 & 0.03 & 3311.00 & 32654.44 & 0.95 & 1.03 & 0.99 \\
44371 & 109295 & 2005 & 441.70 & 0.08 & 44713.00 & 401690.42 & 0.99 & 0.91 & 0.90 \\
50438 & 240417 & 2005 & 29.30 & 0.04 & 5033.00 & 49080.22 & 0.58 & 1.68 & 0.98 \\
13889 & 101785 & 2005 & 1327.40 & 0.03 & 139312.00 & 1305729.50 & 0.95 & 0.98 & 0.94 \\
28709 & 105469 & 2005 & 55.70 & 0.02 & 6321.00 & 58213.18 & 0.88 & 1.05 & 0.92 \\
50373 & 240413 & 2005 & 30.40 & -0.04 & 3267.00 & 28973.14 & 0.93 & 0.95 & 0.89 \\
2959 & 100389 & 2005 & 372.60 & 0.11 & 68902.00 & 664622.74 & 0.54 & 1.78 & 0.96 \\
6796 & 100954 & 2005 & 1506.80 & 0.07 & 150805.00 & 1482085.26 & 1.00 & 0.98 & 0.98 \\
74844 & 601186 & 2005 & 15.70 & 0.07 & 1459.00 & 14575.80 & 1.08 & 0.93 & 1.00 \\
50613 & 240431 & 2005 & 2.40 & 0.03 & 212.00 & 1978.49 & 1.13 & 0.82 & 0.93 \\
74800 & 601172 & 2005 & 785.80 & 0.11 & 75847.00 & 772618.20 & 1.04 & 0.98 & 1.02 \\
50604 & 240430 & 2005 & 60.80 & 0.21 & 5555.00 & 54607.31 & 1.09 & 0.90 & 0.98 \\
50395 & 240414 & 2005 & 1385.70 & 0.06 & 144642.00 & 1402204.05 & 0.96 & 1.01 & 0.97 \\
13054 & 101623 & 2005 & 1116.30 & 0.11 & 112512.00 & 1143102.25 & 0.99 & 1.02 & 1.02 \\
5789 & 100792 & 2005 & 1010.60 & 0.07 & 95760.00 & 892528.50 & 1.06 & 0.88 & 0.93 \\
5733 & 100790 & 2005 & 1967.40 & 0.02 & 202013.00 & 1938173.73 & 0.97 & 0.99 & 0.96 \\
18885 & 102525 & 2005 & 2200.40 & 0.12 & 220276.00 & 2139013.73 & 1.00 & 0.97 & 0.97 \\
28730 & 105472 & 2005 & 57.10 & 0.09 & 5920.00 & 59197.76 & 0.96 & 1.04 & 1.00 \\
36039 & 106449 & 2005 & 121.20 & 0.13 & 12286.00 & 117301.17 & 0.99 & 0.97 & 0.95 \\
36443 & 106528 & 2005 & 235.50 & -0.04 & 22905.00 & 228942.84 & 1.03 & 0.97 & 1.00 \\
29187 & 105535 & 2005 & 122.20 & 0.04 & 12250.00 & 120126.02 & 1.00 & 0.98 & 0.98 \\
16545 & 102154 & 2005 & 208.40 & 0.03 & 22543.00 & 218321.81 & 0.92 & 1.05 & 0.97 \\
36013 & 106447 & 2005 & 22.30 & 0.15 & 1824.00 & 21046.77 & 1.22 & 0.94 & 1.15 \\
50653 & 240434 & 2005 & 20.00 & 0.01 & 1945.00 & 16075.23 & 1.03 & 0.80 & 0.83 \\
42120 & 108924 & 2005 & 142.20 & 0.16 & 20945.00 & 208346.20 & 0.68 & 1.47 & 0.99 \\
36453 & 106529 & 2005 & 245.80 & 0.06 & 25007.00 & 244127.85 & 0.98 & 0.99 & 0.98 \\
44421 & 109307 & 2005 & 213.30 & 0.06 & 21594.00 & 191886.50 & 0.99 & 0.90 & 0.89 \\
29203 & 105536 & 2005 & 154.30 & 0.04 & 15454.00 & 148350.07 & 1.00 & 0.96 & 0.96 \\
22222 & 102996 & 2005 & 507.40 & 0.11 & 52290.00 & 511271.88 & 0.97 & 1.01 & 0.98 \\
46421 & 200241 & 2005 & 2.30 & 0.14 & 229.00 & 2225.96 & 1.00 & 0.97 & 0.97 \\
44272 & 109281 & 2005 & 160.50 & 0.11 & 16636.00 & 157025.30 & 0.96 & 0.98 & 0.94 \\
41863 & 108866 & 2005 & 428.30 & 0.04 & 42907.00 & 423013.73 & 1.00 & 0.99 & 0.99 \\
22250 & 102997 & 2005 & 3431.60 & 0.07 & 344975.00 & 3187999.25 & 0.99 & 0.93 & 0.92 \\
59113 & 410433 & 2005 & 2572.30 & 0.08 & 260803.00 & 2524312.31 & 0.99 & 0.98 & 0.97 \\
295 & 100033 & 2005 & 171.10 & 0.08 & 18383.00 & 178196.16 & 0.93 & 1.04 & 0.97 \\
13040 & 101622 & 2005 & 528.50 & 0.09 & 56016.00 & 546282.37 & 0.94 & 1.03 & 0.98 \\
41843 & 108860 & 2005 & 33.10 & 0.13 & 3312.00 & 32991.68 & 1.00 & 1.00 & 1.00 \\
53248 & 342547 & 2005 & 69.70 & 0.05 & 5979.00 & 60030.31 & 1.17 & 0.86 & 1.00 \\
48883 & 240149 & 2005 & 199.50 & 0.02 & 20048.00 & 200318.06 & 1.00 & 1.00 & 1.00 \\
4309 & 100603 & 2005 & 2465.60 & 0.15 & 243099.00 & 2449742.87 & 1.01 & 0.99 & 1.01 \\
36369 & 106519 & 2005 & 641.60 & 0.01 & 63973.00 & 638089.01 & 1.00 & 0.99 & 1.00 \\
54622 & 377074 & 2005 & 619.50 & 0.09 & 58217.00 & 549921.58 & 1.06 & 0.89 & 0.94 \\
42100 & 108919 & 2005 & 87.40 & -0.01 & 8786.00 & 83648.40 & 0.99 & 0.96 & 0.95 \\
7595 & 101047 & 2005 & 234.50 & 0.05 & 23531.00 & 230839.12 & 1.00 & 0.98 & 0.98 \\
47680 & 217585 & 2005 & 1357.30 & 0.05 & 124309.00 & 1268352.66 & 1.09 & 0.93 & 1.02 \\
10902 & 101345 & 2005 & 3346.00 & 0.07 & 352392.00 & 2892800.94 & 0.95 & 0.86 & 0.82 \\
63210 & 500489 & 2005 & 1787.40 & 0.06 & 162152.00 & 1635310.56 & 1.10 & 0.91 & 1.01 \\
53375 & 344889 & 2005 & 924.70 & 0.12 & 89886.00 & 848099.46 & 1.03 & 0.92 & 0.94 \\
42075 & 108918 & 2005 & 153.10 & 0.14 & 15378.00 & 152674.63 & 1.00 & 1.00 & 0.99 \\
29110 & 105527 & 2005 & 8.90 & 0.11 & 824.00 & 7652.80 & 1.08 & 0.86 & 0.93 \\
63190 & 500488 & 2005 & 952.80 & 0.04 & 75965.00 & 803445.62 & 1.25 & 0.84 & 1.06 \\
53361 & 344311 & 2005 & 2.60 & 0.04 & 462.00 & 4719.19 & 0.56 & 1.82 & 1.02 \\
28812 & 105478 & 2005 & 57.60 & 0.02 & 5703.00 & 57034.17 & 1.01 & 0.99 & 1.00 \\
36089 & 106461 & 2005 & 348.10 & 0.03 & 33704.00 & 302334.36 & 1.03 & 0.87 & 0.90 \\
53363 & 344625 & 2005 & 85.60 & 0.10 & 4942.00 & 50786.63 & 1.73 & 0.59 & 1.03 \\
41836 & 108859 & 2005 & 13.20 & 0.06 & 1326.00 & 13202.56 & 1.00 & 1.00 & 1.00 \\
22191 & 102994 & 2005 & 123.50 & 0.08 & 11950.00 & 102011.82 & 1.03 & 0.83 & 0.85 \\
16522 & 102152 & 2005 & 252.60 & 0.01 & 26798.00 & 259533.96 & 0.94 & 1.03 & 0.97 \\
50403 & 240415 & 2005 & 246.10 & 0.21 & 38706.00 & 238451.82 & 0.64 & 0.97 & 0.62 \\
29153 & 105533 & 2005 & 123.20 & -0.01 & 12616.00 & 128440.91 & 0.98 & 1.04 & 1.02 \\
44403 & 109300 & 2005 & 464.60 & 0.07 & 46404.00 & 458868.22 & 1.00 & 0.99 & 0.99 \\
28752 & 105475 & 2005 & 469.40 & 0.12 & 48927.00 & 464354.83 & 0.96 & 0.99 & 0.95 \\
47665 & 216749 & 2005 & 98.00 & 0.03 & 10221.00 & 103699.63 & 0.96 & 1.06 & 1.01 \\
36418 & 106524 & 2005 & 19.60 & -0.00 & 2003.00 & 17330.49 & 0.98 & 0.88 & 0.87 \\
53268 & 342548 & 2005 & 48.70 & 0.03 & 4819.00 & 47641.79 & 1.01 & 0.98 & 0.99 \\
28781 & 105476 & 2005 & 659.30 & 0.15 & 65556.00 & 641112.16 & 1.01 & 0.97 & 0.98 \\
2834 & 100362 & 2005 & 185.70 & 0.12 & 18506.00 & 152821.52 & 1.00 & 0.82 & 0.83 \\
13130 & 101668 & 2005 & 83.90 & -0.03 & 8445.00 & 79426.33 & 0.99 & 0.95 & 0.94 \\
36398 & 106523 & 2005 & 137.70 & -0.01 & 19871.00 & 165123.21 & 0.69 & 1.20 & 0.83 \\
29133 & 105531 & 2005 & 142.00 & 0.02 & 13237.00 & 132210.03 & 1.07 & 0.93 & 1.00 \\
344 & 100040 & 2005 & 828.90 & -0.05 & 82974.00 & 828668.03 & 1.00 & 1.00 & 1.00 \\
21962 & 102981 & 2005 & 218.60 & 0.07 & 21848.00 & 217276.30 & 1.00 & 0.99 & 0.99 \\
11029 & 101360 & 2005 & 2258.40 & 0.04 & 229703.00 & 2024613.97 & 0.98 & 0.90 & 0.88 \\
22491 & 103015 & 2005 & 143.90 & 0.06 & 14966.00 & 146330.99 & 0.96 & 1.02 & 0.98 \\
11127 & 101368 & 2005 & 1097.10 & -0.01 & 117883.00 & 1007592.50 & 0.93 & 0.92 & 0.85 \\
36856 & 106605 & 2005 & 361.60 & 0.09 & 34168.00 & 367309.89 & 1.06 & 1.02 & 1.08 \\
4405 & 100622 & 2005 & 747.60 & 0.05 & 75682.00 & 748071.91 & 0.99 & 1.00 & 0.99 \\
62918 & 500447 & 2005 & 58.60 & 0.11 & 5889.00 & 58888.21 & 1.00 & 1.00 & 1.00 \\
10737 & 101320 & 2005 & 375.30 & 0.38 & 37995.00 & 345923.89 & 0.99 & 0.92 & 0.91 \\
46170 & 200196 & 2005 & 12.70 & 0.06 & 1250.00 & 12486.74 & 1.02 & 0.98 & 1.00 \\
36828 & 106602 & 2005 & 11.40 & -0.04 & 1423.00 & 14230.53 & 0.80 & 1.25 & 1.00 \\
21712 & 102940 & 2005 & 1634.80 & 0.05 & 172478.00 & 1660146.59 & 0.95 & 1.02 & 0.96 \\
28175 & 105390 & 2005 & 396.30 & 0.07 & 39043.00 & 387249.45 & 1.02 & 0.98 & 0.99 \\
53135 & 340902 & 2005 & 4.50 & 0.02 & 434.00 & 4408.36 & 1.04 & 0.98 & 1.02 \\
11224 & 101376 & 2005 & 2707.60 & 0.05 & 264017.00 & 2632609.23 & 1.03 & 0.97 & 1.00 \\
57215 & 400320 & 2005 & 328.00 & 0.35 & 33019.00 & 284233.34 & 0.99 & 0.87 & 0.86 \\
36927 & 106640 & 2005 & 336.70 & 0.15 & 33811.00 & 329328.10 & 1.00 & 0.98 & 0.97 \\
19084 & 102548 & 2005 & 1311.80 & 0.13 & 128295.00 & 1282936.21 & 1.02 & 0.98 & 1.00 \\
54509 & 373584 & 2005 & 19.00 & 0.02 & 2591.00 & 21749.92 & 0.73 & 1.14 & 0.84 \\
16794 & 102192 & 2005 & 3082.50 & 0.08 & 305813.00 & 2955010.32 & 1.01 & 0.96 & 0.97 \\
28253 & 105399 & 2005 & 114.90 & 0.03 & 11952.00 & 114757.70 & 0.96 & 1.00 & 0.96 \\
29492 & 105598 & 2005 & 372.90 & 0.06 & 37315.00 & 370438.31 & 1.00 & 0.99 & 0.99 \\
16730 & 102182 & 2005 & 205.20 & 0.10 & 19746.00 & 192483.83 & 1.04 & 0.94 & 0.97 \\
44520 & 109334 & 2005 & 115.30 & 0.06 & 11437.00 & 113520.57 & 1.01 & 0.98 & 0.99 \\
8499 & 101088 & 2005 & 2686.70 & 0.16 & 266529.00 & 2257372.32 & 1.01 & 0.84 & 0.85 \\
2674 & 100351 & 2005 & 37.40 & 0.06 & 3689.00 & 36885.52 & 1.01 & 0.99 & 1.00 \\
54530 & 373714 & 2005 & 292.40 & 0.13 & 24146.00 & 223774.38 & 1.21 & 0.77 & 0.93 \\
41561 & 108766 & 2005 & 126.20 & 0.10 & 15510.00 & 156627.89 & 0.81 & 1.24 & 1.01 \\
12955 & 101616 & 2005 & 8727.10 & 0.01 & 947544.00 & 9351510.72 & 0.92 & 1.07 & 0.99 \\
28240 & 105397 & 2005 & 87.60 & 0.09 & 8724.00 & 85483.25 & 1.00 & 0.98 & 0.98 \\
62899 & 500446 & 2005 & 41.20 & 0.10 & 3997.00 & 39605.65 & 1.03 & 0.96 & 0.99 \\
28129 & 105383 & 2005 & 117.50 & 0.07 & 11752.00 & 114040.53 & 1.00 & 0.97 & 0.97 \\
62937 & 500448 & 2005 & 40.30 & 0.06 & 4022.00 & 40127.13 & 1.00 & 1.00 & 1.00 \\
41625 & 108782 & 2005 & 780.00 & 0.04 & 85294.00 & 828273.33 & 0.91 & 1.06 & 0.97 \\
18780 & 102508 & 2005 & 68.30 & 0.03 & 7034.00 & 71337.36 & 0.97 & 1.04 & 1.01 \\
36777 & 106590 & 2005 & 123.10 & 0.10 & 11170.00 & 114833.32 & 1.10 & 0.93 & 1.03 \\
63546 & 500521 & 2005 & 5.00 & 0.11 & 283.00 & 2829.20 & 1.77 & 0.57 & 1.00 \\
55545 & 400118 & 2005 & 303.60 & 0.08 & 27458.00 & 283374.18 & 1.11 & 0.93 & 1.03 \\
21774 & 102951 & 2005 & 4858.20 & 0.05 & 485456.00 & 4892903.30 & 1.00 & 1.01 & 1.01 \\
47719 & 220770 & 2005 & 2799.20 & 0.05 & 410274.00 & 4138358.66 & 0.68 & 1.48 & 1.01 \\
46129 & 200192 & 2005 & 2.40 & 0.04 & 256.00 & 2142.93 & 0.94 & 0.89 & 0.84 \\
46135 & 200193 & 2005 & 133.20 & 0.05 & 13313.00 & 132885.90 & 1.00 & 1.00 & 1.00 \\
55063 & 400050 & 2005 & 3933.50 & 0.11 & 361450.00 & 3607017.62 & 1.09 & 0.92 & 1.00 \\
36939 & 106642 & 2005 & 1728.30 & 0.11 & 164375.00 & 1643785.12 & 1.05 & 0.95 & 1.00 \\
50824 & 240453 & 2005 & 73.70 & 0.14 & 7057.00 & 66903.22 & 1.04 & 0.91 & 0.95 \\
28282 & 105400 & 2005 & 403.20 & 0.15 & 41419.00 & 401266.17 & 0.97 & 1.00 & 0.97 \\
7206 & 101013 & 2005 & 3963.50 & 0.08 & 406749.00 & 3993131.70 & 0.97 & 1.01 & 0.98 \\
29581 & 105616 & 2005 & 138.00 & 0.20 & 13148.00 & 131482.30 & 1.05 & 0.95 & 1.00 \\
35740 & 106394 & 2005 & 47.30 & 0.02 & 4764.00 & 47932.49 & 0.99 & 1.01 & 1.01 \\
5609 & 100773 & 2005 & 3912.20 & 0.06 & 407467.00 & 3780337.68 & 0.96 & 0.97 & 0.93 \\
54554 & 375941 & 2005 & 39.40 & 0.06 & 3650.00 & 35069.40 & 1.08 & 0.89 & 0.96 \\
14572 & 101885 & 2005 & 1067.50 & 0.03 & 106791.00 & 1016101.76 & 1.00 & 0.95 & 0.95 \\
46155 & 200194 & 2005 & 16.90 & 0.15 & 1337.00 & 12202.29 & 1.26 & 0.72 & 0.91 \\
36797 & 106595 & 2005 & 21.50 & 0.06 & 2141.00 & 17319.88 & 1.00 & 0.81 & 0.81 \\
10773 & 101330 & 2005 & 2113.60 & 0.01 & 211136.00 & 2070076.93 & 1.00 & 0.98 & 0.98 \\
62880 & 500445 & 2005 & 68.20 & 0.18 & 6669.00 & 66690.28 & 1.02 & 0.98 & 1.00 \\
22574 & 103021 & 2005 & 239.70 & 0.07 & 23390.00 & 228120.25 & 1.02 & 0.95 & 0.98 \\
29552 & 105611 & 2005 & 5309.00 & 0.10 & 517595.00 & 4361139.89 & 1.03 & 0.82 & 0.84 \\
62861 & 500444 & 2005 & 49.90 & 0.10 & 5015.00 & 49800.82 & 1.00 & 1.00 & 0.99 \\
47657 & 216504 & 2005 & 16.50 & -0.01 & 1139.00 & 11569.59 & 1.45 & 0.70 & 1.02 \\
5876 & 100809 & 2005 & 1507.50 & 0.05 & 144988.00 & 1425266.67 & 1.04 & 0.95 & 0.98 \\
53449 & 350408 & 2005 & 57.60 & 0.10 & 5591.00 & 49756.47 & 1.03 & 0.86 & 0.89 \\
46200 & 200198 & 2005 & 31.90 & 0.09 & 3186.00 & 31472.69 & 1.00 & 0.99 & 0.99 \\
59194 & 410444 & 2005 & 114.70 & 0.12 & 11621.00 & 114181.10 & 0.99 & 1.00 & 0.98 \\
53157 & 341114 & 2005 & 42.00 & 0.12 & 3131.00 & 31536.59 & 1.34 & 0.75 & 1.01 \\
36883 & 106620 & 2005 & 249.00 & 0.08 & 25313.00 & 211582.26 & 0.98 & 0.85 & 0.84 \\
53177 & 341116 & 2005 & 123.20 & 0.18 & 7027.00 & 71407.25 & 1.75 & 0.58 & 1.02 \\
53214 & 341126 & 2005 & 4236.70 & 0.19 & 303336.00 & 2542418.46 & 1.40 & 0.60 & 0.84 \\
22605 & 103024 & 2005 & 1468.20 & 0.06 & 147519.00 & 1473347.33 & 1.00 & 1.00 & 1.00 \\
3089 & 100408 & 2005 & 165.30 & 0.05 & 16482.00 & 161156.70 & 1.00 & 0.97 & 0.98 \\
2655 & 100350 & 2005 & 97.00 & 0.03 & 9435.00 & 94353.05 & 1.03 & 0.97 & 1.00 \\
55562 & 400125 & 2005 & 123.40 & 0.12 & 12515.00 & 122128.79 & 0.99 & 0.99 & 0.98 \\
62842 & 500443 & 2005 & 96.70 & 0.06 & 9680.00 & 96718.69 & 1.00 & 1.00 & 1.00 \\
41576 & 108776 & 2005 & 739.20 & 0.06 & 78684.00 & 733336.19 & 0.94 & 0.99 & 0.93 \\
29540 & 105610 & 2005 & 305.70 & 0.06 & 29983.00 & 299826.04 & 1.02 & 0.98 & 1.00 \\
5570 & 100772 & 2005 & 2621.00 & 0.06 & 305446.00 & 2873435.01 & 0.86 & 1.10 & 0.94 \\
62827 & 500442 & 2005 & 187.60 & 0.20 & 18707.00 & 181306.76 & 1.00 & 0.97 & 0.97 \\
53194 & 341117 & 2005 & 217.50 & 0.19 & 15055.00 & 147018.11 & 1.44 & 0.68 & 0.98 \\
6884 & 100967 & 2005 & 1097.40 & 0.06 & 107895.00 & 911740.39 & 1.02 & 0.83 & 0.85 \\
41569 & 108773 & 2005 & 221.60 & 0.08 & 23445.00 & 226986.10 & 0.95 & 1.02 & 0.97 \\
63503 & 500512 & 2005 & 517.30 & 0.06 & 76107.00 & 641119.29 & 0.68 & 1.24 & 0.84 \\
44509 & 109333 & 2005 & 151.60 & 0.12 & 12171.00 & 114304.61 & 1.25 & 0.75 & 0.94 \\
8684 & 101094 & 2005 & 629.30 & 0.00 & 65549.00 & 603780.27 & 0.96 & 0.96 & 0.92 \\
74728 & 601157 & 2005 & 25.20 & 0.05 & 2541.00 & 23385.05 & 0.99 & 0.93 & 0.92 \\
59038 & 410401 & 2005 & 372.70 & 0.51 & 36650.00 & 332421.66 & 1.02 & 0.89 & 0.91 \\
44504 & 109332 & 2005 & 18.10 & 0.02 & 1980.00 & 18967.99 & 0.91 & 1.05 & 0.96 \\
18762 & 102507 & 2005 & 156.70 & 0.07 & 15928.00 & 159980.59 & 0.98 & 1.02 & 1.00 \\
467 & 100068 & 2005 & 98.20 & 0.07 & 9660.00 & 90698.23 & 1.02 & 0.92 & 0.94 \\
53137 & 341107 & 2005 & 57.50 & 0.09 & 4692.00 & 45762.35 & 1.23 & 0.80 & 0.98 \\
55552 & 400120 & 2005 & 1208.50 & 0.07 & 113959.00 & 1169059.10 & 1.06 & 0.97 & 1.03 \\
22642 & 103027 & 2005 & 9908.10 & 0.12 & 935652.00 & 9790765.63 & 1.06 & 0.99 & 1.05 \\
50068 & 240391 & 2005 & 514.50 & 0.02 & 50588.00 & 450791.28 & 1.02 & 0.88 & 0.89 \\
28143 & 105384 & 2005 & 47.80 & 0.05 & 4744.00 & 47441.86 & 1.01 & 0.99 & 1.00 \\
41611 & 108780 & 2005 & 61.00 & 0.14 & 5786.00 & 58502.63 & 1.05 & 0.96 & 1.01 \\
16747 & 102183 & 2005 & 1721.90 & 0.13 & 157575.00 & 1588327.99 & 1.09 & 0.92 & 1.01 \\
35708 & 106391 & 2005 & 87.20 & 0.10 & 8716.00 & 85128.81 & 1.00 & 0.98 & 0.98 \\
28213 & 105393 & 2005 & 3932.10 & 0.10 & 402170.00 & 3673131.40 & 0.98 & 0.93 & 0.91 \\
36909 & 106627 & 2005 & 3680.00 & 0.06 & 291811.00 & 2776329.84 & 1.26 & 0.75 & 0.95 \\
42214 & 108943 & 2005 & 3349.30 & 0.10 & 386448.00 & 3114602.43 & 0.87 & 0.93 & 0.81 \\
48720 & 240130 & 2005 & 300.50 & 0.03 & 37105.00 & 308556.01 & 0.81 & 1.03 & 0.83 \\
57220 & 400322 & 2005 & 53.60 & 0.19 & 3760.00 & 33478.05 & 1.43 & 0.62 & 0.89 \\
7670 & 101054 & 2005 & 9601.60 & 0.08 & 991145.00 & 9406815.00 & 0.97 & 0.98 & 0.95 \\
41601 & 108777 & 2005 & 67.30 & 0.22 & 4336.00 & 35122.75 & 1.55 & 0.52 & 0.81 \\
46190 & 200197 & 2005 & 29.30 & 0.04 & 2929.00 & 29273.97 & 1.00 & 1.00 & 1.00 \\
50062 & 240388 & 2005 & 14.90 & 0.07 & 1608.00 & 16468.76 & 0.93 & 1.11 & 1.02 \\
35678 & 106386 & 2005 & 436.10 & 0.04 & 52581.00 & 524424.14 & 0.83 & 1.20 & 1.00 \\
16780 & 102191 & 2005 & 73.20 & -0.03 & 7922.00 & 70017.84 & 0.92 & 0.96 & 0.88 \\
29466 & 105597 & 2005 & 172.00 & 0.06 & 17204.00 & 171507.51 & 1.00 & 1.00 & 1.00 \\
57213 & 400319 & 2005 & 7.00 & 0.01 & 676.00 & 6989.56 & 1.04 & 1.00 & 1.03 \\
63565 & 500522 & 2005 & 20310.50 & 0.02 & 2031127.00 & 20238053.00 & 1.00 & 1.00 & 1.00 \\
36749 & 106584 & 2005 & 2127.10 & 0.08 & 205209.00 & 1941358.55 & 1.04 & 0.91 & 0.95 \\
16288 & 102121 & 2005 & 890.30 & 0.16 & 86304.00 & 863004.73 & 1.03 & 0.97 & 1.00 \\
36957 & 106643 & 2005 & 95.90 & 0.14 & 9614.00 & 92588.86 & 1.00 & 0.97 & 0.96 \\
42198 & 108934 & 2005 & 85.50 & 0.08 & 8149.00 & 84397.40 & 1.05 & 0.99 & 1.04 \\
28081 & 105379 & 2005 & 363.00 & 0.02 & 37132.00 & 348898.50 & 0.98 & 0.96 & 0.94 \\
28342 & 105416 & 2005 & 2470.50 & 0.04 & 247470.00 & 2405339.76 & 1.00 & 0.97 & 0.97 \\
63479 & 500511 & 2005 & 919.40 & 0.08 & 125137.00 & 1173469.82 & 0.73 & 1.28 & 0.94 \\
28328 & 105412 & 2005 & 171.40 & 0.04 & 14442.00 & 146137.40 & 1.19 & 0.85 & 1.01 \\
35644 & 106381 & 2005 & 159.70 & 0.04 & 20423.00 & 196177.79 & 0.78 & 1.23 & 0.96 \\
54496 & 372855 & 2005 & 13.90 & 0.09 & 2737.00 & 26345.61 & 0.51 & 1.90 & 0.96 \\
35765 & 106401 & 2005 & 2929.60 & 0.08 & 273130.00 & 2612754.25 & 1.07 & 0.89 & 0.96 \\
22506 & 103016 & 2005 & 788.40 & 0.05 & 102043.00 & 795369.19 & 0.77 & 1.01 & 0.78 \\
42239 & 108944 & 2005 & 127.70 & 0.02 & 7824.00 & 73150.56 & 1.63 & 0.57 & 0.93 \\
50802 & 240451 & 2005 & 368.20 & 0.03 & 35833.00 & 356352.58 & 1.03 & 0.97 & 0.99 \\
2694 & 100352 & 2005 & 2414.60 & 0.07 & 245931.00 & 2226980.98 & 0.98 & 0.92 & 0.91 \\
59171 & 410443 & 2005 & 22.50 & 0.13 & 2234.00 & 21465.17 & 1.01 & 0.95 & 0.96 \\
2636 & 100348 & 2005 & 66.70 & 0.04 & 6645.00 & 64944.96 & 1.00 & 0.97 & 0.98 \\
12974 & 101617 & 2005 & 89.70 & 0.25 & 8832.00 & 92498.66 & 1.02 & 1.03 & 1.05 \\
41673 & 108827 & 2005 & 645.90 & 0.06 & 77502.00 & 638705.54 & 0.83 & 0.99 & 0.82 \\
41648 & 108826 & 2005 & 411.30 & 0.11 & 35093.00 & 332678.72 & 1.17 & 0.81 & 0.95 \\
36983 & 106644 & 2005 & 1.70 & 0.07 & 185.00 & 1775.27 & 0.92 & 1.04 & 0.96 \\
5653 & 100784 & 2005 & 52184.20 & 0.09 & 5206195.00 & 48447527.43 & 1.00 & 0.93 & 0.93 \\
50782 & 240450 & 2005 & 525.90 & 0.06 & 54710.00 & 531356.61 & 0.96 & 1.01 & 0.97 \\
74770 & 601165 & 2005 & 24.00 & 0.12 & 2392.00 & 22723.52 & 1.00 & 0.95 & 0.95 \\
8535 & 101089 & 2005 & 208.00 & 0.14 & 24229.00 & 216860.66 & 0.86 & 1.04 & 0.90 \\
11289 & 101390 & 2005 & 4073.50 & 0.03 & 409861.00 & 4033343.76 & 0.99 & 0.99 & 0.98 \\
28311 & 105401 & 2005 & 705.60 & 0.11 & 70451.00 & 675824.91 & 1.00 & 0.96 & 0.96 \\
5542 & 100771 & 2005 & 1820.90 & 0.13 & 178745.00 & 1724063.12 & 1.02 & 0.95 & 0.96 \\
46488 & 200248 & 2005 & 206.20 & 0.09 & 18891.00 & 186594.55 & 1.09 & 0.90 & 0.99 \\
49420 & 240295 & 2006 & 4138.00 & 0.13 & 418537.00 & 4020573.55 & 0.99 & 0.97 & 0.96 \\
36055 & 106451 & 2006 & 115.50 & 0.15 & 15919.00 & 160368.93 & 0.73 & 1.39 & 1.01 \\
45007 & 109413 & 2006 & 1306.60 & 0.17 & 123312.00 & 1223009.18 & 1.06 & 0.94 & 0.99 \\
42031 & 108910 & 2006 & 2.70 & 0.17 & 258.00 & 2494.87 & 1.05 & 0.92 & 0.97 \\
62365 & 500388 & 2006 & 431.30 & 0.25 & 43105.00 & 436200.12 & 1.00 & 1.01 & 1.01 \\
65536 & 500702 & 2006 & 263.10 & 0.11 & 28121.00 & 276196.15 & 0.94 & 1.05 & 0.98 \\
45131 & 109431 & 2006 & 37.50 & 0.06 & 3399.00 & 31212.44 & 1.10 & 0.83 & 0.92 \\
42215 & 108943 & 2006 & 4951.60 & 0.14 & 544892.00 & 4592943.15 & 0.91 & 0.93 & 0.84 \\
56119 & 400174 & 2006 & 253.10 & 0.09 & 25846.00 & 258466.70 & 0.98 & 1.02 & 1.00 \\
58458 & 410159 & 2006 & 43.60 & 0.09 & 4718.00 & 46944.44 & 0.92 & 1.08 & 1.00 \\
40296 & 108087 & 2006 & 17.70 & 0.03 & 2038.00 & 19719.48 & 0.87 & 1.11 & 0.97 \\
45449 & 200060 & 2006 & 2713.60 & 0.07 & 255064.00 & 2598570.74 & 1.06 & 0.96 & 1.02 \\
53269 & 342548 & 2006 & 42.50 & 0.18 & 4058.00 & 40181.51 & 1.05 & 0.95 & 0.99 \\
65513 & 500701 & 2006 & 305.20 & 0.00 & 29440.00 & 292721.51 & 1.04 & 0.96 & 0.99 \\
43184 & 109076 & 2006 & 108.00 & 0.08 & 11687.00 & 107819.34 & 0.92 & 1.00 & 0.92 \\
57396 & 400407 & 2006 & 7.40 & 0.15 & 693.00 & 6933.77 & 1.07 & 0.94 & 1.00 \\
47666 & 216749 & 2006 & 101.60 & 0.16 & 10341.00 & 103598.91 & 0.98 & 1.02 & 1.00 \\
57187 & 400303 & 2006 & 71.50 & 0.19 & 6759.00 & 62319.78 & 1.06 & 0.87 & 0.92 \\
40544 & 108137 & 2006 & 870.00 & 0.14 & 86180.00 & 859195.43 & 1.01 & 0.99 & 1.00 \\
33557 & 106148 & 2006 & 237.40 & 0.22 & 23768.00 & 229626.46 & 1.00 & 0.97 & 0.97 \\
51135 & 240486 & 2006 & 7.70 & 0.05 & 768.00 & 7452.16 & 1.00 & 0.97 & 0.97 \\
45115 & 109429 & 2006 & 27.80 & -0.05 & 2471.00 & 24499.29 & 1.13 & 0.88 & 0.99 \\
38953 & 107358 & 2006 & 872.30 & 0.29 & 86576.00 & 812968.03 & 1.01 & 0.93 & 0.94 \\
64745 & 500625 & 2006 & 73.30 & 0.07 & 8581.00 & 72710.90 & 0.85 & 0.99 & 0.85 \\
38928 & 107354 & 2006 & 173.70 & 0.11 & 17372.00 & 166019.51 & 1.00 & 0.96 & 0.96 \\
49245 & 240254 & 2006 & 232.20 & 0.17 & 23224.00 & 222732.36 & 1.00 & 0.96 & 0.96 \\
47417 & 210770 & 2006 & 8788.60 & 0.20 & 932950.00 & 8282815.69 & 0.94 & 0.94 & 0.89 \\
39669 & 107833 & 2006 & 46.80 & 0.06 & 4656.00 & 46371.86 & 1.01 & 0.99 & 1.00 \\
38911 & 107352 & 2006 & 337.70 & 0.06 & 36517.00 & 327376.92 & 0.92 & 0.97 & 0.90 \\
51163 & 240489 & 2006 & 23.40 & 0.05 & 2667.00 & 26502.03 & 0.88 & 1.13 & 0.99 \\
35284 & 106344 & 2006 & 592.10 & 0.16 & 58817.00 & 533302.89 & 1.01 & 0.90 & 0.91 \\
40322 & 108109 & 2006 & 101.80 & 0.04 & 10131.00 & 102881.60 & 1.00 & 1.01 & 1.02 \\
53023 & 337150 & 2006 & 185.30 & 0.06 & 18513.00 & 185126.32 & 1.00 & 1.00 & 1.00 \\
54989 & 400040 & 2006 & 164.90 & 0.12 & 16466.00 & 162425.49 & 1.00 & 0.98 & 0.99 \\
48332 & 240062 & 2006 & 1412.50 & 0.04 & 145080.00 & 1287752.91 & 0.97 & 0.91 & 0.89 \\
38692 & 107308 & 2006 & 3038.00 & 0.11 & 303593.00 & 2829271.50 & 1.00 & 0.93 & 0.93 \\
47193 & 200342 & 2006 & 18254.20 & 0.15 & 1731257.00 & 16886333.69 & 1.05 & 0.93 & 0.98 \\
49428 & 240296 & 2006 & 2139.40 & 0.35 & 230651.00 & 2064028.92 & 0.93 & 0.96 & 0.89 \\
58477 & 410160 & 2006 & 53.40 & 0.10 & 5775.00 & 58287.38 & 0.92 & 1.09 & 1.01 \\
57409 & 400409 & 2006 & 30.90 & 0.11 & 3067.00 & 28698.76 & 1.01 & 0.93 & 0.94 \\
54083 & 364391 & 2006 & 88.90 & 0.06 & 6074.00 & 59015.97 & 1.46 & 0.66 & 0.97 \\
61294 & 500027 & 2006 & 268.00 & 0.09 & 26411.00 & 265511.19 & 1.01 & 0.99 & 1.01 \\
57398 & 400408 & 2006 & 31.70 & 0.18 & 3113.00 & 30159.73 & 1.02 & 0.95 & 0.97 \\
49452 & 240301 & 2006 & 20.20 & 0.06 & 1912.00 & 19309.31 & 1.06 & 0.96 & 1.01 \\
33568 & 106149 & 2006 & 119.30 & 0.07 & 6763.00 & 67141.89 & 1.76 & 0.56 & 0.99 \\
39642 & 107832 & 2006 & 299.10 & 0.14 & 30041.00 & 291185.35 & 1.00 & 0.97 & 0.97 \\
42101 & 108919 & 2006 & 65.50 & 0.04 & 6474.00 & 62808.42 & 1.01 & 0.96 & 0.97 \\
54735 & 378592 & 2006 & 7.70 & 0.16 & 747.00 & 6034.10 & 1.03 & 0.78 & 0.81 \\
43536 & 109142 & 2006 & 59.80 & -0.02 & 6002.00 & 60017.30 & 1.00 & 1.00 & 1.00 \\
50907 & 240467 & 2006 & 24.10 & 0.12 & 2411.00 & 22943.85 & 1.00 & 0.95 & 0.95 \\
47137 & 200335 & 2006 & 3.10 & 0.14 & 290.00 & 3026.06 & 1.07 & 0.98 & 1.04 \\
32842 & 106067 & 2006 & 5634.70 & 0.13 & 556320.00 & 5601774.03 & 1.01 & 0.99 & 1.01 \\
53138 & 341107 & 2006 & 107.60 & 0.05 & 10090.00 & 101591.89 & 1.07 & 0.94 & 1.01 \\
51157 & 240487 & 2006 & 12.70 & 0.11 & 1085.00 & 10416.73 & 1.17 & 0.82 & 0.96 \\
49017 & 240199 & 2006 & 719.40 & 0.18 & 64900.00 & 681081.52 & 1.11 & 0.95 & 1.05 \\
62385 & 500389 & 2006 & 197.70 & 0.09 & 19828.00 & 198107.64 & 1.00 & 1.00 & 1.00 \\
58049 & 410063 & 2006 & 3514.80 & 0.16 & 351423.00 & 3147818.93 & 1.00 & 0.90 & 0.90 \\
38997 & 107563 & 2006 & 2256.90 & 0.15 & 217120.00 & 2131208.10 & 1.04 & 0.94 & 0.98 \\
48016 & 226438 & 2006 & 518.70 & 0.05 & 51568.00 & 474196.08 & 1.01 & 0.91 & 0.92 \\
32920 & 106082 & 2006 & 408.90 & 0.15 & 45173.00 & 395363.18 & 0.91 & 0.97 & 0.88 \\
58562 & 410167 & 2006 & 17.60 & 0.18 & 1842.00 & 18679.24 & 0.96 & 1.06 & 1.01 \\
42011 & 108907 & 2006 & 923.10 & 0.06 & 87641.00 & 845009.07 & 1.05 & 0.92 & 0.96 \\
39063 & 107598 & 2006 & 124.20 & 0.12 & 12554.00 & 129279.15 & 0.99 & 1.04 & 1.03 \\
56072 & 400171 & 2006 & 224.60 & 0.21 & 20046.00 & 212089.39 & 1.12 & 0.94 & 1.06 \\
44194 & 109273 & 2006 & 277.30 & 0.28 & 48881.00 & 396952.42 & 0.57 & 1.43 & 0.81 \\
33511 & 106143 & 2006 & 136.10 & 0.18 & 15768.00 & 127885.20 & 0.86 & 0.94 & 0.81 \\
56895 & 400264 & 2006 & 68.80 & 0.14 & 6878.00 & 66203.04 & 1.00 & 0.96 & 0.96 \\
49476 & 240302 & 2006 & 9.60 & 0.08 & 920.00 & 9149.80 & 1.04 & 0.95 & 0.99 \\
53178 & 341116 & 2006 & 175.70 & 0.05 & 20037.00 & 209603.34 & 0.88 & 1.19 & 1.05 \\
42389 & 108960 & 2006 & 508.10 & 0.06 & 48929.00 & 489263.06 & 1.04 & 0.96 & 1.00 \\
62445 & 500392 & 2006 & 62.90 & 0.10 & 6248.00 & 62900.27 & 1.01 & 1.00 & 1.01 \\
47095 & 200333 & 2006 & 2511.90 & 0.21 & 250847.00 & 2376135.36 & 1.00 & 0.95 & 0.95 \\
40239 & 108082 & 2006 & 108.60 & 0.24 & 11038.00 & 107423.21 & 0.98 & 0.99 & 0.97 \\
53158 & 341114 & 2006 & 88.00 & 0.19 & 7323.00 & 78306.33 & 1.20 & 0.89 & 1.07 \\
44831 & 109393 & 2006 & 31.90 & 0.12 & 3145.00 & 31138.86 & 1.01 & 0.98 & 0.99 \\
58252 & 410133 & 2006 & 554.40 & 0.06 & 41678.00 & 428823.50 & 1.33 & 0.77 & 1.03 \\
39088 & 107604 & 2006 & 2291.90 & 0.08 & 233438.00 & 2144780.66 & 0.98 & 0.94 & 0.92 \\
52380 & 302825 & 2006 & 323.80 & 0.01 & 28525.00 & 283486.39 & 1.14 & 0.88 & 0.99 \\
49069 & 240212 & 2006 & 2267.50 & 0.03 & 226981.00 & 1831359.54 & 1.00 & 0.81 & 0.81 \\
45423 & 200058 & 2006 & 4944.40 & 0.09 & 461645.00 & 4550676.59 & 1.07 & 0.92 & 0.99 \\
35253 & 106336 & 2006 & 77.00 & 0.11 & 8084.00 & 80398.59 & 0.95 & 1.04 & 0.99 \\
63130 & 500483 & 2006 & 1367.00 & 0.13 & 174128.00 & 1284270.35 & 0.79 & 0.94 & 0.74 \\
51223 & 240492 & 2006 & 41.40 & 0.16 & 3331.00 & 33057.27 & 1.24 & 0.80 & 0.99 \\
50614 & 240431 & 2006 & 3.80 & 0.03 & 456.00 & 3328.08 & 0.83 & 0.88 & 0.73 \\
33485 & 106140 & 2006 & 794.60 & 0.12 & 79220.00 & 769625.74 & 1.00 & 0.97 & 0.97 \\
65691 & 500719 & 2006 & 85.70 & 0.21 & 7813.00 & 77075.77 & 1.10 & 0.90 & 0.99 \\
65490 & 500700 & 2006 & 234.70 & 0.04 & 34304.00 & 234411.67 & 0.68 & 1.00 & 0.68 \\
53946 & 362337 & 2006 & 131.80 & 0.13 & 14327.00 & 119323.03 & 0.92 & 0.91 & 0.83 \\
58271 & 410136 & 2006 & 3050.00 & 0.15 & 255870.00 & 2389006.40 & 1.19 & 0.78 & 0.93 \\
50605 & 240430 & 2006 & 218.80 & 0.12 & 25300.00 & 209635.95 & 0.86 & 0.96 & 0.83 \\
58228 & 410130 & 2006 & 568.90 & 0.19 & 48073.00 & 477655.30 & 1.18 & 0.84 & 0.99 \\
43605 & 109153 & 2006 & 474.20 & 0.17 & 45458.00 & 477949.57 & 1.04 & 1.01 & 1.05 \\
36040 & 106449 & 2006 & 74.30 & 0.07 & 7769.00 & 72168.33 & 0.96 & 0.97 & 0.93 \\
39830 & 107881 & 2006 & 23.00 & 0.16 & 2276.00 & 18951.88 & 1.01 & 0.82 & 0.83 \\
55689 & 400138 & 2006 & 82.80 & 0.15 & 8261.00 & 74184.54 & 1.00 & 0.90 & 0.90 \\
52550 & 303123 & 2006 & 97.00 & 0.19 & 10973.00 & 101379.82 & 0.88 & 1.05 & 0.92 \\
53964 & 362424 & 2006 & 257.90 & 0.14 & 23721.00 & 250807.75 & 1.09 & 0.97 & 1.06 \\
35679 & 106386 & 2006 & 315.70 & 0.15 & 56722.00 & 564522.75 & 0.56 & 1.79 & 1.00 \\
40266 & 108083 & 2006 & 395.50 & 0.19 & 39374.00 & 388425.73 & 1.00 & 0.98 & 0.99 \\
38669 & 107306 & 2006 & 264.20 & 0.18 & 29450.00 & 272635.01 & 0.90 & 1.03 & 0.93 \\
46558 & 200253 & 2006 & 3.60 & 0.00 & 335.00 & 3426.47 & 1.07 & 0.95 & 1.02 \\
47117 & 200334 & 2006 & 455.00 & 0.19 & 42581.00 & 357651.47 & 1.07 & 0.79 & 0.84 \\
54180 & 364809 & 2006 & 44.40 & 0.04 & 4463.00 & 45092.94 & 0.99 & 1.02 & 1.01 \\
50854 & 240459 & 2006 & 183.30 & 0.18 & 18380.00 & 174195.21 & 1.00 & 0.95 & 0.95 \\
58437 & 410158 & 2006 & 196.10 & 0.21 & 26776.00 & 193967.62 & 0.73 & 0.99 & 0.72 \\
52184 & 302627 & 2006 & 120.00 & 0.14 & 13987.00 & 117519.41 & 0.86 & 0.98 & 0.84 \\
56877 & 400263 & 2006 & 101.60 & 0.14 & 10172.00 & 95984.15 & 1.00 & 0.94 & 0.94 \\
43236 & 109086 & 2006 & 1183.50 & 0.19 & 119008.00 & 1153667.37 & 0.99 & 0.97 & 0.97 \\
49412 & 240293 & 2006 & 4026.90 & 0.13 & 422438.00 & 3642681.22 & 0.95 & 0.90 & 0.86 \\
58557 & 410166 & 2006 & 107.60 & 0.13 & 9805.00 & 103758.88 & 1.10 & 0.96 & 1.06 \\
62405 & 500390 & 2006 & 141.70 & 0.16 & 13245.00 & 132862.10 & 1.07 & 0.94 & 1.00 \\
53775 & 357053 & 2006 & 56.80 & 0.16 & 5691.00 & 56870.98 & 1.00 & 1.00 & 1.00 \\
55669 & 400136 & 2006 & 1757.90 & 0.16 & 214027.00 & 2157189.83 & 0.82 & 1.23 & 1.01 \\
53244 & 342448 & 2006 & 143.90 & 0.19 & 14213.00 & 130283.81 & 1.01 & 0.91 & 0.92 \\
58078 & 410075 & 2006 & 932.60 & 0.05 & 96432.00 & 993973.66 & 0.97 & 1.07 & 1.03 \\
52195 & 302676 & 2006 & 1176.20 & 0.11 & 110114.00 & 1145128.57 & 1.07 & 0.97 & 1.04 \\
35618 & 106380 & 2006 & 921.80 & 0.15 & 104169.00 & 887833.80 & 0.88 & 0.96 & 0.85 \\
56093 & 400172 & 2006 & 1440.20 & 0.13 & 128553.00 & 1071266.06 & 1.12 & 0.74 & 0.83 \\
62425 & 500391 & 2006 & 138.00 & 0.20 & 13632.00 & 137383.20 & 1.01 & 1.00 & 1.01 \\
38661 & 107303 & 2006 & 22.60 & 0.02 & 3630.00 & 33398.88 & 0.62 & 1.48 & 0.92 \\
39034 & 107573 & 2006 & 170.10 & 0.16 & 17166.00 & 168779.89 & 0.99 & 0.99 & 0.98 \\
53075 & 338387 & 2006 & 7769.90 & 0.19 & 725160.00 & 7441307.74 & 1.07 & 0.96 & 1.03 \\
44808 & 109392 & 2006 & 49.40 & 0.11 & 4473.00 & 40192.60 & 1.10 & 0.81 & 0.90 \\
55252 & 400075 & 2006 & 1561.60 & 0.10 & 166544.00 & 1624152.33 & 0.94 & 1.04 & 0.98 \\
43721 & 109208 & 2006 & 183.90 & 0.12 & 17758.00 & 167345.87 & 1.04 & 0.91 & 0.94 \\
64447 & 500603 & 2006 & 539.50 & 0.15 & 53706.00 & 537308.18 & 1.00 & 1.00 & 1.00 \\
35339 & 106348 & 2006 & 132.10 & 0.04 & 12736.00 & 133979.01 & 1.04 & 1.01 & 1.05 \\
33543 & 106147 & 2006 & 41.20 & 0.11 & 4130.00 & 40584.17 & 1.00 & 0.99 & 0.98 \\
56099 & 400173 & 2006 & 46.80 & 0.09 & 3381.00 & 34738.21 & 1.38 & 0.74 & 1.03 \\
53249 & 342547 & 2006 & 21.10 & 0.07 & 2286.00 & 22705.52 & 0.92 & 1.08 & 0.99 \\
40348 & 108112 & 2006 & 28.50 & 0.34 & 2849.00 & 27260.90 & 1.00 & 0.96 & 0.96 \\
96743 & 611010 & 2006 & 169.90 & 0.08 & 17002.00 & 162576.27 & 1.00 & 0.96 & 0.96 \\
58223 & 410122 & 2006 & 108.90 & 0.14 & 5893.00 & 56071.43 & 1.85 & 0.51 & 0.95 \\
64778 & 500633 & 2006 & 83.60 & 0.13 & 7934.00 & 74090.93 & 1.05 & 0.89 & 0.93 \\
33674 & 106160 & 2006 & 253.40 & 0.20 & 24957.00 & 249568.63 & 1.02 & 0.98 & 1.00 \\
62795 & 500435 & 2006 & 947.40 & 0.07 & 100752.00 & 870895.89 & 0.94 & 0.92 & 0.86 \\
38829 & 107331 & 2006 & 87.50 & 0.12 & 8505.00 & 82813.66 & 1.03 & 0.95 & 0.97 \\
33630 & 106157 & 2006 & 1130.60 & 0.06 & 113243.00 & 1135594.06 & 1.00 & 1.00 & 1.00 \\
39794 & 107872 & 2006 & 69.90 & 0.14 & 7371.00 & 67151.32 & 0.95 & 0.96 & 0.91 \\
39686 & 107835 & 2006 & 1064.30 & 0.15 & 106211.00 & 1038471.04 & 1.00 & 0.98 & 0.98 \\
56792 & 400255 & 2006 & 56.70 & 0.16 & 5718.00 & 52601.97 & 0.99 & 0.93 & 0.92 \\
55306 & 400081 & 2006 & 3838.80 & 0.12 & 398132.00 & 3365519.01 & 0.96 & 0.88 & 0.85 \\
62777 & 500434 & 2006 & 365.30 & 0.15 & 41308.00 & 373148.03 & 0.88 & 1.02 & 0.90 \\
49438 & 240297 & 2006 & 1237.60 & 0.15 & 157582.00 & 1116927.62 & 0.79 & 0.90 & 0.71 \\
62325 & 500382 & 2006 & 15.30 & 0.20 & 1458.00 & 12771.13 & 1.05 & 0.83 & 0.88 \\
45541 & 200073 & 2006 & 172.40 & 0.09 & 21853.00 & 214226.42 & 0.79 & 1.24 & 0.98 \\
56772 & 400254 & 2006 & 32.90 & 0.06 & 3204.00 & 30876.72 & 1.03 & 0.94 & 0.96 \\
52374 & 302819 & 2006 & 1.80 & 0.04 & 178.00 & 1760.22 & 1.01 & 0.98 & 0.99 \\
47994 & 225687 & 2006 & 176.30 & 0.04 & 17678.00 & 176265.41 & 1.00 & 1.00 & 1.00 \\
56811 & 400256 & 2006 & 63.70 & 0.14 & 6222.00 & 58842.24 & 1.02 & 0.92 & 0.95 \\
40512 & 108134 & 2006 & 345.50 & 0.16 & 31097.00 & 306907.52 & 1.11 & 0.89 & 0.99 \\
44778 & 109375 & 2006 & 135.70 & 0.21 & 13029.00 & 124012.05 & 1.04 & 0.91 & 0.95 \\
43571 & 109145 & 2006 & 59.30 & 0.05 & 7088.00 & 60186.30 & 0.84 & 1.01 & 0.85 \\
64401 & 500601 & 2006 & 294.10 & 0.09 & 23187.00 & 248483.03 & 1.27 & 0.84 & 1.07 \\
57506 & 400428 & 2006 & 28.30 & 0.10 & 3440.00 & 27516.84 & 0.82 & 0.97 & 0.80 \\
57126 & 400290 & 2006 & 50.60 & 0.15 & 5052.00 & 48189.93 & 1.00 & 0.95 & 0.95 \\
45515 & 200072 & 2006 & 12.40 & 0.20 & 1221.00 & 11984.03 & 1.02 & 0.97 & 0.98 \\
62345 & 500387 & 2006 & 243.60 & 0.05 & 24555.00 & 243388.83 & 0.99 & 1.00 & 0.99 \\
61404 & 500048 & 2006 & 66.40 & 0.16 & 5667.00 & 56564.56 & 1.17 & 0.85 & 1.00 \\
57507 & 400429 & 2006 & 174.90 & 0.13 & 16677.00 & 166768.49 & 1.05 & 0.95 & 1.00 \\
40434 & 108119 & 2006 & 465.90 & 0.15 & 42879.00 & 433950.36 & 1.09 & 0.93 & 1.01 \\
38850 & 107336 & 2006 & 207.00 & 0.17 & 21368.00 & 213102.23 & 0.97 & 1.03 & 1.00 \\
44205 & 109274 & 2006 & 43.50 & 0.09 & 4519.00 & 47098.32 & 0.96 & 1.08 & 1.04 \\
44331 & 109284 & 2006 & 27.40 & 0.10 & 2642.00 & 27633.73 & 1.04 & 1.01 & 1.05 \\
42265 & 108946 & 2006 & 16.20 & 0.15 & 1620.00 & 13560.80 & 1.00 & 0.84 & 0.84 \\
54138 & 364519 & 2006 & 101.20 & 0.22 & 10875.00 & 107425.43 & 0.93 & 1.06 & 0.99 \\
57211 & 400307 & 2006 & 4.30 & 0.11 & 428.00 & 4280.35 & 1.00 & 1.00 & 1.00 \\
33618 & 106156 & 2006 & 8.80 & 0.06 & 837.00 & 8779.98 & 1.05 & 1.00 & 1.05 \\
52525 & 302997 & 2006 & 589.80 & 0.15 & 56456.00 & 505751.03 & 1.04 & 0.86 & 0.90 \\
59583 & 410499 & 2006 & 581.70 & 0.08 & 57268.00 & 572685.64 & 1.02 & 0.98 & 1.00 \\
47649 & 216438 & 2006 & 549.20 & 0.17 & 51820.00 & 521276.57 & 1.06 & 0.95 & 1.01 \\
58532 & 410163 & 2006 & 38.80 & 0.11 & 3701.00 & 36407.67 & 1.05 & 0.94 & 0.98 \\
65642 & 500710 & 2006 & 1527.10 & 0.14 & 146867.00 & 1429118.96 & 1.04 & 0.94 & 0.97 \\
47981 & 225484 & 2006 & 31.50 & 0.13 & 3269.00 & 30564.39 & 0.96 & 0.97 & 0.93 \\
35645 & 106381 & 2006 & 541.30 & 0.26 & 33990.00 & 354959.44 & 1.59 & 0.66 & 1.04 \\
47145 & 200338 & 2006 & 1040.30 & 0.16 & 93894.00 & 909815.08 & 1.11 & 0.87 & 0.97 \\
52367 & 302813 & 2006 & 23.50 & 0.10 & 2259.00 & 22418.70 & 1.04 & 0.95 & 0.99 \\
53061 & 337871 & 2006 & 257.70 & 0.17 & 30645.00 & 314762.36 & 0.84 & 1.22 & 1.03 \\
43208 & 109084 & 2006 & 355.00 & 0.05 & 44629.00 & 357267.47 & 0.80 & 1.01 & 0.80 \\
58513 & 410162 & 2006 & 114.30 & 0.15 & 11545.00 & 106347.55 & 0.99 & 0.93 & 0.92 \\
64378 & 500600 & 2006 & 4360.30 & 0.10 & 431156.00 & 4319097.49 & 1.01 & 0.99 & 1.00 \\
43203 & 109083 & 2006 & 102.30 & 0.05 & 13924.00 & 111473.53 & 0.73 & 1.09 & 0.80 \\
47140 & 200336 & 2006 & 9.30 & 0.12 & 898.00 & 9115.44 & 1.04 & 0.98 & 1.02 \\
44339 & 109286 & 2006 & 570.30 & 0.13 & 56987.00 & 561960.12 & 1.00 & 0.99 & 0.99 \\
40486 & 108122 & 2006 & 2385.60 & 0.37 & 238889.00 & 2207968.09 & 1.00 & 0.93 & 0.92 \\
50868 & 240462 & 2006 & 74.70 & 0.15 & 6174.00 & 59803.29 & 1.21 & 0.80 & 0.97 \\
65670 & 500713 & 2006 & 200.40 & 0.10 & 20029.00 & 195320.85 & 1.00 & 0.97 & 0.98 \\
42044 & 108914 & 2006 & 2106.60 & 0.37 & 185195.00 & 1982867.62 & 1.14 & 0.94 & 1.07 \\
53292 & 342993 & 2006 & 219.90 & 0.19 & 22043.00 & 219939.55 & 1.00 & 1.00 & 1.00 \\
55003 & 400041 & 2006 & 260.10 & 0.14 & 26023.00 & 257616.35 & 1.00 & 0.99 & 0.99 \\
33650 & 106158 & 2006 & 329.80 & 0.08 & 35679.00 & 364396.27 & 0.92 & 1.10 & 1.02 \\
57513 & 400430 & 2006 & 4.10 & -0.02 & 353.00 & 4110.01 & 1.16 & 1.00 & 1.16 \\
43578 & 109147 & 2006 & 93.40 & 0.16 & 9527.00 & 85604.30 & 0.98 & 0.92 & 0.90 \\
46402 & 200236 & 2006 & 125.00 & -0.01 & 13728.00 & 129882.78 & 0.91 & 1.04 & 0.95 \\
43635 & 109175 & 2006 & 101.60 & 0.17 & 10188.00 & 98975.62 & 1.00 & 0.97 & 0.97 \\
38813 & 107328 & 2006 & 33.30 & 0.21 & 3418.00 & 31925.81 & 0.97 & 0.96 & 0.93 \\
56180 & 400180 & 2006 & 28.70 & 0.08 & 5166.00 & 49101.22 & 0.56 & 1.71 & 0.95 \\
64769 & 500628 & 2006 & 49.30 & 0.07 & 4959.00 & 44463.78 & 0.99 & 0.90 & 0.90 \\
36144 & 106470 & 2006 & 127.30 & 0.09 & 13153.00 & 133600.39 & 0.97 & 1.05 & 1.02 \\
48098 & 235952 & 2006 & 73.70 & 0.10 & 7310.00 & 68227.26 & 1.01 & 0.93 & 0.93 \\
40460 & 108121 & 2006 & 5134.30 & 0.19 & 450184.00 & 4728538.95 & 1.14 & 0.92 & 1.05 \\
62296 & 500377 & 2006 & 4.40 & 0.07 & 407.00 & 4032.84 & 1.08 & 0.92 & 0.99 \\
57163 & 400293 & 2006 & 9.20 & 0.04 & 909.00 & 8932.15 & 1.01 & 0.97 & 0.98 \\
38788 & 107323 & 2006 & 88.50 & 0.13 & 11603.00 & 93963.78 & 0.76 & 1.06 & 0.81 \\
54160 & 364633 & 2006 & 93.00 & 0.13 & 9306.00 & 91639.64 & 1.00 & 0.99 & 0.98 \\
57214 & 400319 & 2006 & 11.20 & 0.03 & 1122.00 & 11189.98 & 1.00 & 1.00 & 1.00 \\
38765 & 107322 & 2006 & 24.70 & 0.10 & 2379.00 & 23388.63 & 1.04 & 0.95 & 0.98 \\
64801 & 500634 & 2006 & 95.70 & 0.20 & 9389.00 & 95398.10 & 1.02 & 1.00 & 1.02 \\
57145 & 400291 & 2006 & 20.70 & 0.09 & 2078.00 & 19657.35 & 1.00 & 0.95 & 0.95 \\
58219 & 410121 & 2006 & 49.20 & 0.20 & 4846.00 & 48921.03 & 1.02 & 0.99 & 1.01 \\
64824 & 500635 & 2006 & 213.70 & 0.09 & 14560.00 & 147849.53 & 1.47 & 0.69 & 1.02 \\
43191 & 109077 & 2006 & 4.90 & -0.01 & 526.00 & 5042.45 & 0.93 & 1.03 & 0.96 \\
45090 & 109427 & 2006 & 1997.10 & 0.16 & 196581.00 & 1874056.53 & 1.02 & 0.94 & 0.95 \\
65575 & 500706 & 2006 & 2763.20 & 0.15 & 229226.00 & 2417414.68 & 1.21 & 0.87 & 1.05 \\
38717 & 107309 & 2006 & 37.60 & 0.13 & 4082.00 & 38050.93 & 0.92 & 1.01 & 0.93 \\
56827 & 400257 & 2006 & 474.30 & 0.19 & 47501.00 & 473563.75 & 1.00 & 1.00 & 1.00 \\
45029 & 109414 & 2006 & 633.60 & 0.14 & 59276.00 & 587049.40 & 1.07 & 0.93 & 0.99 \\
47465 & 211485 & 2006 & 10.00 & 0.15 & 972.00 & 9850.52 & 1.03 & 0.99 & 1.01 \\
65598 & 500707 & 2006 & 1593.20 & 0.10 & 143355.00 & 1491106.55 & 1.11 & 0.94 & 1.04 \\
48478 & 240087 & 2006 & 210.20 & 0.15 & 20819.00 & 208595.70 & 1.01 & 0.99 & 1.00 \\
45557 & 200075 & 2006 & 534.10 & 0.18 & 50605.00 & 503849.89 & 1.06 & 0.94 & 1.00 \\
55314 & 400084 & 2006 & 21.50 & -0.09 & 2143.00 & 21240.63 & 1.00 & 0.99 & 0.99 \\
40374 & 108115 & 2006 & 2584.70 & 0.14 & 239708.00 & 2500625.14 & 1.08 & 0.97 & 1.04 \\
51203 & 240491 & 2006 & 38.40 & 0.03 & 3789.00 & 37702.70 & 1.01 & 0.98 & 1.00 \\
44802 & 109389 & 2006 & 115.50 & 0.19 & 14557.00 & 140615.63 & 0.79 & 1.22 & 0.97 \\
39778 & 107870 & 2006 & 244.70 & 0.11 & 25546.00 & 240824.39 & 0.96 & 0.98 & 0.94 \\
36090 & 106461 & 2006 & 380.20 & 0.17 & 34998.00 & 341985.27 & 1.09 & 0.90 & 0.98 \\
36163 & 106474 & 2006 & 231.20 & 0.15 & 23307.00 & 232336.69 & 0.99 & 1.00 & 1.00 \\
51183 & 240490 & 2006 & 227.60 & 0.17 & 22752.00 & 226476.02 & 1.00 & 1.00 & 1.00 \\
61322 & 500028 & 2006 & 1614.00 & 0.08 & 159439.00 & 1552641.99 & 1.01 & 0.96 & 0.97 \\
47681 & 217585 & 2006 & 2262.80 & 0.16 & 207298.00 & 2110718.01 & 1.09 & 0.93 & 1.02 \\
42076 & 108918 & 2006 & 226.40 & 0.03 & 22766.00 & 227181.73 & 0.99 & 1.00 & 1.00 \\
57216 & 400320 & 2006 & 891.10 & 0.13 & 76934.00 & 784911.52 & 1.16 & 0.88 & 1.02 \\
62252 & 500369 & 2006 & 17.90 & 0.31 & 1810.00 & 16189.44 & 0.99 & 0.90 & 0.89 \\
43564 & 109144 & 2006 & 144.40 & -0.03 & 13281.00 & 139667.92 & 1.09 & 0.97 & 1.05 \\
50904 & 240464 & 2006 & 56.20 & 0.13 & 8444.00 & 56837.01 & 0.67 & 1.01 & 0.67 \\
35311 & 106345 & 2006 & 379.00 & 0.05 & 37812.00 & 373270.55 & 1.00 & 0.98 & 0.99 \\
57415 & 400410 & 2006 & 449.30 & 0.17 & 46634.00 & 376880.41 & 0.96 & 0.84 & 0.81 \\
45481 & 200065 & 2006 & 26.60 & 0.22 & 2660.00 & 25064.69 & 1.00 & 0.94 & 0.94 \\
33591 & 106151 & 2006 & 1079.20 & 0.14 & 112488.00 & 1094429.62 & 0.96 & 1.01 & 0.97 \\
54092 & 364393 & 2006 & 229.20 & 0.05 & 21730.00 & 223709.70 & 1.05 & 0.98 & 1.03 \\
57108 & 400286 & 2006 & 49.10 & 0.19 & 5035.00 & 49858.42 & 0.98 & 1.02 & 0.99 \\
57122 & 400288 & 2006 & 255.60 & 0.19 & 25594.00 & 253253.90 & 1.00 & 0.99 & 0.99 \\
38887 & 107350 & 2006 & 3608.00 & 0.14 & 357096.00 & 3524407.20 & 1.01 & 0.98 & 0.99 \\
56859 & 400262 & 2006 & 133.70 & 0.13 & 13384.00 & 123726.32 & 1.00 & 0.93 & 0.92 \\
58552 & 410165 & 2006 & 128.80 & 0.14 & 10827.00 & 113657.27 & 1.19 & 0.88 & 1.05 \\
56845 & 400261 & 2006 & 162.20 & 0.11 & 16212.00 & 146668.40 & 1.00 & 0.90 & 0.90 \\
49035 & 240207 & 2006 & 4.00 & 0.17 & 416.00 & 3610.57 & 0.96 & 0.90 & 0.87 \\
47482 & 212351 & 2006 & 52.50 & 0.18 & 5253.00 & 52166.74 & 1.00 & 0.99 & 0.99 \\
45062 & 109416 & 2006 & 439.70 & 0.08 & 47304.00 & 438310.80 & 0.93 & 1.00 & 0.93 \\
44792 & 109380 & 2006 & 31.20 & 0.13 & 3063.00 & 30630.82 & 1.02 & 0.98 & 1.00 \\
56144 & 400176 & 2006 & 1496.20 & 0.19 & 135556.00 & 1361198.95 & 1.10 & 0.91 & 1.00 \\
53737 & 356752 & 2006 & 26.60 & 0.07 & 2150.00 & 22270.56 & 1.24 & 0.84 & 1.04 \\
53987 & 362981 & 2006 & 156.00 & 0.11 & 19699.00 & 201951.06 & 0.79 & 1.29 & 1.03 \\
53718 & 356500 & 2006 & 299.70 & 0.15 & 29369.00 & 311151.39 & 1.02 & 1.04 & 1.06 \\
53040 & 337653 & 2006 & 212.40 & 0.09 & 22450.00 & 213401.21 & 0.95 & 1.00 & 0.95 \\
59588 & 410501 & 2006 & 15.30 & 0.07 & 1591.00 & 16310.28 & 0.96 & 1.07 & 1.03 \\
43746 & 109217 & 2006 & 595.20 & 0.02 & 60991.00 & 604335.48 & 0.98 & 1.02 & 0.99 \\
50888 & 240463 & 2006 & 27.60 & 0.02 & 2772.00 & 27120.04 & 1.00 & 0.98 & 0.98 \\
56165 & 400178 & 2006 & 70.20 & 0.02 & 7071.00 & 68764.89 & 0.99 & 0.98 & 0.97 \\
57492 & 400425 & 2006 & 59.50 & 0.09 & 5921.00 & 55485.09 & 1.00 & 0.93 & 0.94 \\
54764 & 378620 & 2006 & 455.30 & 0.15 & 43693.00 & 373705.92 & 1.04 & 0.82 & 0.86 \\
54115 & 364518 & 2006 & 125.60 & 0.03 & 12571.00 & 124159.56 & 1.00 & 0.99 & 0.99 \\
61378 & 500047 & 2006 & 4.10 & 0.09 & 395.00 & 3614.76 & 1.04 & 0.88 & 0.92 \\
56767 & 400252 & 2006 & 78.80 & 0.04 & 7931.00 & 78783.92 & 0.99 & 1.00 & 0.99 \\
40400 & 108117 & 2006 & 460.20 & 0.05 & 46053.00 & 429684.35 & 1.00 & 0.93 & 0.93 \\
54756 & 378596 & 2006 & 672.90 & 0.12 & 67280.00 & 586552.95 & 1.00 & 0.87 & 0.87 \\
64355 & 500598 & 2006 & 4660.40 & 0.15 & 460473.00 & 4610841.30 & 1.01 & 0.99 & 1.00 \\
65675 & 500714 & 2006 & 104.40 & 0.17 & 10467.00 & 96688.39 & 1.00 & 0.93 & 0.92 \\
54644 & 377379 & 2006 & 24.60 & 0.10 & 2603.00 & 25434.51 & 0.95 & 1.03 & 0.98 \\
61336 & 500037 & 2006 & 1445.40 & -0.02 & 144769.00 & 1446742.86 & 1.00 & 1.00 & 1.00 \\
52167 & 302545 & 2006 & 112.60 & 0.20 & 14578.00 & 107462.22 & 0.77 & 0.95 & 0.74 \\
49273 & 240261 & 2006 & 110.00 & 0.18 & 10660.00 & 104851.70 & 1.03 & 0.95 & 0.98 \\
44308 & 109283 & 2006 & 4351.40 & 0.14 & 413709.00 & 4138801.01 & 1.05 & 0.95 & 1.00 \\
39735 & 107858 & 2006 & 34.60 & 0.14 & 3430.00 & 31325.13 & 1.01 & 0.91 & 0.91 \\
50833 & 240458 & 2006 & 439.60 & 0.13 & 46514.00 & 465692.05 & 0.95 & 1.06 & 1.00 \\
35326 & 106347 & 2006 & 87.20 & 0.06 & 8414.00 & 90185.25 & 1.04 & 1.03 & 1.07 \\
53115 & 339977 & 2006 & 182.40 & 0.10 & 19889.00 & 174021.37 & 0.92 & 0.95 & 0.87 \\
52485 & 302964 & 2006 & 2626.80 & 0.11 & 373117.00 & 2696249.75 & 0.70 & 1.03 & 0.72 \\
36109 & 106464 & 2006 & 364.60 & 0.10 & 38107.00 & 310927.08 & 0.96 & 0.85 & 0.82 \\
56123 & 400175 & 2006 & 787.30 & 0.16 & 80163.00 & 801629.41 & 0.98 & 1.02 & 1.00 \\
57209 & 400306 & 2006 & 50.80 & 0.13 & 5079.00 & 50679.48 & 1.00 & 1.00 & 1.00 \\
35608 & 106379 & 2006 & 201.40 & 0.14 & 20509.00 & 194190.05 & 0.98 & 0.96 & 0.95 \\
48353 & 240065 & 2006 & 452.30 & 0.14 & 44322.00 & 441100.23 & 1.02 & 0.98 & 1.00 \\
45051 & 109415 & 2006 & 170.90 & 0.06 & 18177.00 & 173936.55 & 0.94 & 1.02 & 0.96 \\
50974 & 240474 & 2006 & 179.50 & 0.09 & 18204.00 & 163875.74 & 0.99 & 0.91 & 0.90 \\
39100 & 107605 & 2006 & 1665.90 & 0.06 & 192072.00 & 1778703.58 & 0.87 & 1.07 & 0.93 \\
54714 & 378591 & 2006 & 8.60 & 0.10 & 871.00 & 8120.38 & 0.99 & 0.94 & 0.93 \\
46527 & 200251 & 2006 & 216.50 & 0.08 & 21497.00 & 206696.03 & 1.01 & 0.95 & 0.96 \\
45238 & 109439 & 2006 & 889.30 & 0.25 & 81248.00 & 801802.82 & 1.09 & 0.90 & 0.99 \\
62485 & 500394 & 2006 & 136.00 & 0.14 & 13593.00 & 135697.06 & 1.00 & 1.00 & 1.00 \\
56019 & 400166 & 2006 & 685.90 & 0.20 & 67722.00 & 667447.65 & 1.01 & 0.97 & 0.99 \\
60940 & 410756 & 2006 & 286.60 & 0.20 & 20266.00 & 204035.06 & 1.41 & 0.71 & 1.01 \\
58381 & 410151 & 2006 & 1459.90 & 0.09 & 149662.00 & 1474896.66 & 0.98 & 1.01 & 0.99 \\
42298 & 108950 & 2006 & 1255.70 & 0.14 & 115848.00 & 1186391.43 & 1.08 & 0.94 & 1.02 \\
60951 & 410758 & 2006 & 182.90 & 0.21 & 19524.00 & 185750.06 & 0.94 & 1.02 & 0.95 \\
53882 & 360020 & 2006 & 28.50 & 0.03 & 2855.00 & 27805.70 & 1.00 & 0.98 & 0.97 \\
60953 & 410762 & 2006 & 213.20 & 0.14 & 20494.00 & 203803.76 & 1.04 & 0.96 & 0.99 \\
46433 & 200244 & 2006 & 6.50 & 0.27 & 630.00 & 6800.63 & 1.03 & 1.05 & 1.08 \\
50825 & 240453 & 2006 & 99.70 & 0.36 & 9906.00 & 96864.37 & 1.01 & 0.97 & 0.98 \\
55815 & 400153 & 2006 & 142.00 & 0.14 & 15770.00 & 122388.07 & 0.90 & 0.86 & 0.78 \\
33025 & 106086 & 2006 & 45.90 & 0.16 & 4586.00 & 44069.37 & 1.00 & 0.96 & 0.96 \\
58403 & 410153 & 2006 & 15.80 & 0.09 & 1563.00 & 15724.65 & 1.01 & 1.00 & 1.01 \\
35741 & 106394 & 2006 & 46.30 & 0.20 & 4605.00 & 46652.26 & 1.01 & 1.01 & 1.01 \\
55382 & 400092 & 2006 & 78.00 & 0.15 & 6906.00 & 67116.64 & 1.13 & 0.86 & 0.97 \\
50710 & 240441 & 2006 & 2604.40 & 0.21 & 261269.00 & 2510225.72 & 1.00 & 0.96 & 0.96 \\
51037 & 240477 & 2006 & 7.50 & 0.18 & 750.00 & 6870.07 & 1.00 & 0.92 & 0.92 \\
55064 & 400050 & 2006 & 4583.50 & 0.19 & 459794.00 & 4267517.88 & 1.00 & 0.93 & 0.93 \\
50803 & 240451 & 2006 & 445.10 & 0.18 & 42231.00 & 433772.88 & 1.05 & 0.97 & 1.03 \\
51016 & 240476 & 2006 & 2403.10 & 0.22 & 210900.00 & 2193504.80 & 1.14 & 0.91 & 1.04 \\
44987 & 109407 & 2006 & 443.10 & 0.18 & 46947.00 & 410353.79 & 0.94 & 0.93 & 0.87 \\
46456 & 200245 & 2006 & 28.50 & 0.10 & 2760.00 & 27317.50 & 1.03 & 0.96 & 0.99 \\
43389 & 109110 & 2006 & 48.70 & 0.14 & 4809.00 & 47724.37 & 1.01 & 0.98 & 0.99 \\
54018 & 363232 & 2006 & 932.80 & 0.13 & 97555.00 & 963282.72 & 0.96 & 1.03 & 0.99 \\
55836 & 400155 & 2006 & 452.40 & 0.17 & 41064.00 & 410676.11 & 1.10 & 0.91 & 1.00 \\
42155 & 108930 & 2006 & 118.20 & 0.12 & 14157.00 & 126863.87 & 0.83 & 1.07 & 0.90 \\
44239 & 109278 & 2006 & 37.10 & 0.08 & 3715.00 & 36802.04 & 1.00 & 0.99 & 0.99 \\
54669 & 377933 & 2006 & 100.20 & 0.13 & 10628.00 & 108907.55 & 0.94 & 1.09 & 1.02 \\
33222 & 106103 & 2006 & 9.00 & 0.07 & 1209.00 & 12520.00 & 0.74 & 1.39 & 1.04 \\
52327 & 302763 & 2006 & 1515.80 & 0.21 & 132580.00 & 1401968.10 & 1.14 & 0.92 & 1.06 \\
39996 & 107968 & 2006 & 110.30 & 10.82 & 11006.00 & 109560.43 & 1.00 & 0.99 & 1.00 \\
44934 & 109402 & 2006 & 118.00 & 0.21 & 11788.00 & 116111.92 & 1.00 & 0.98 & 0.99 \\
39529 & 107719 & 2006 & 323.00 & 0.08 & 33342.00 & 331269.66 & 0.97 & 1.03 & 0.99 \\
55360 & 400090 & 2006 & 65.60 & 0.22 & 6814.00 & 60524.83 & 0.96 & 0.92 & 0.89 \\
39340 & 107670 & 2006 & 312.10 & 0.16 & 30512.00 & 311131.31 & 1.02 & 1.00 & 1.02 \\
53996 & 363121 & 2006 & 5553.40 & 0.16 & 483902.00 & 5084066.95 & 1.15 & 0.92 & 1.05 \\
33243 & 106107 & 2006 & 50.70 & 0.06 & 5165.00 & 50298.75 & 0.98 & 0.99 & 0.97 \\
39900 & 107928 & 2006 & 2051.40 & 0.08 & 206591.00 & 2111538.89 & 0.99 & 1.03 & 1.02 \\
49384 & 240287 & 2006 & 30.30 & 0.15 & 3035.00 & 31111.37 & 1.00 & 1.03 & 1.03 \\
62938 & 500448 & 2006 & 83.20 & 0.17 & 8333.00 & 83303.57 & 1.00 & 1.00 & 1.00 \\
49112 & 240222 & 2006 & 414.60 & 0.12 & 41481.00 & 395038.46 & 1.00 & 0.95 & 0.95 \\
39317 & 107653 & 2006 & 60.80 & 0.06 & 6097.00 & 60091.81 & 1.00 & 0.99 & 0.99 \\
39554 & 107720 & 2006 & 296.00 & 0.16 & 29363.00 & 290575.64 & 1.01 & 0.98 & 0.99 \\
54980 & 400037 & 2006 & 433.00 & 0.04 & 61446.00 & 612454.99 & 0.70 & 1.41 & 1.00 \\
62919 & 500447 & 2006 & 65.40 & 0.06 & 6542.00 & 65415.04 & 1.00 & 1.00 & 1.00 \\
48075 & 235413 & 2006 & 204.10 & 0.14 & 20424.00 & 194169.34 & 1.00 & 0.95 & 0.95 \\
57275 & 400389 & 2006 & 5.90 & 0.17 & 1090.00 & 9223.71 & 0.54 & 1.56 & 0.85 \\
43362 & 109100 & 2006 & 57.80 & 0.10 & 4959.00 & 46249.71 & 1.17 & 0.80 & 0.93 \\
58173 & 410114 & 2006 & 717.40 & 0.21 & 70261.00 & 705993.86 & 1.02 & 0.98 & 1.00 \\
65139 & 500664 & 2006 & 2362.50 & 0.17 & 238590.00 & 2359254.28 & 0.99 & 1.00 & 0.99 \\
35903 & 106424 & 2006 & 288.50 & 0.13 & 28859.00 & 278866.94 & 1.00 & 0.97 & 0.97 \\
39568 & 107722 & 2006 & 676.50 & 0.16 & 59878.00 & 611171.14 & 1.13 & 0.90 & 1.02 \\
33274 & 106109 & 2006 & 25.00 & 0.12 & 2610.00 & 26765.39 & 0.96 & 1.07 & 1.03 \\
62900 & 500446 & 2006 & 72.40 & 0.14 & 7292.00 & 72623.44 & 0.99 & 1.00 & 1.00 \\
52301 & 302760 & 2006 & 681.10 & 0.15 & 66180.00 & 661504.66 & 1.03 & 0.97 & 1.00 \\
32998 & 106085 & 2006 & 236.30 & 0.31 & 23281.00 & 232178.03 & 1.01 & 0.98 & 1.00 \\
61190 & 410904 & 2006 & 281.00 & 0.14 & 27578.00 & 275761.73 & 1.02 & 0.98 & 1.00 \\
96785 & 611013 & 2006 & 34.20 & 0.15 & 3262.00 & 32219.39 & 1.05 & 0.94 & 0.99 \\
39352 & 107672 & 2006 & 4.00 & 0.07 & 344.00 & 3412.60 & 1.16 & 0.85 & 0.99 \\
57273 & 400388 & 2006 & 4.30 & 0.19 & 412.00 & 3655.72 & 1.04 & 0.85 & 0.89 \\
45216 & 109438 & 2006 & 436.10 & 0.16 & 44539.00 & 422214.27 & 0.98 & 0.97 & 0.95 \\
55740 & 400144 & 2006 & 16.90 & 0.12 & 2757.00 & 27905.74 & 0.61 & 1.65 & 1.01 \\
51057 & 240478 & 2006 & 2.20 & 0.19 & 216.00 & 2207.68 & 1.02 & 1.00 & 1.02 \\
48682 & 240118 & 2006 & 326.20 & 0.35 & 26137.00 & 262538.11 & 1.25 & 0.80 & 1.00 \\
50945 & 240469 & 2006 & 105.50 & 0.17 & 9698.00 & 103620.09 & 1.09 & 0.98 & 1.07 \\
33253 & 106108 & 2006 & 143.10 & 0.20 & 13419.00 & 133256.71 & 1.07 & 0.93 & 0.99 \\
64951 & 500648 & 2006 & 242.50 & 0.08 & 24296.00 & 233878.84 & 1.00 & 0.96 & 0.96 \\
42144 & 108929 & 2006 & 256.60 & 0.15 & 24972.00 & 250145.05 & 1.03 & 0.97 & 1.00 \\
43455 & 109124 & 2006 & 254.30 & 0.13 & 21881.00 & 221463.74 & 1.16 & 0.87 & 1.01 \\
64552 & 500609 & 2006 & 547.80 & 0.19 & 54471.00 & 546318.89 & 1.01 & 1.00 & 1.00 \\
43678 & 109189 & 2006 & 1133.00 & 0.18 & 125380.00 & 1218323.89 & 0.90 & 1.08 & 0.97 \\
40014 & 107994 & 2006 & 372.70 & 0.09 & 40475.00 & 340006.66 & 0.92 & 0.91 & 0.84 \\
48721 & 240130 & 2006 & 223.80 & 0.01 & 27655.00 & 232652.75 & 0.81 & 1.04 & 0.84 \\
35879 & 106421 & 2006 & 122.30 & 0.36 & 11609.00 & 117844.74 & 1.05 & 0.96 & 1.02 \\
50696 & 240440 & 2006 & 967.00 & 0.15 & 97150.00 & 963330.68 & 1.00 & 1.00 & 0.99 \\
43381 & 109108 & 2006 & 61.60 & 0.09 & 6143.00 & 60444.53 & 1.00 & 0.98 & 0.98 \\
44997 & 109410 & 2006 & 236.40 & 0.08 & 22568.00 & 231711.65 & 1.05 & 0.98 & 1.03 \\
51062 & 240480 & 2006 & 48.50 & 0.19 & 4324.00 & 37646.32 & 1.12 & 0.78 & 0.87 \\
35457 & 106361 & 2006 & 323.00 & 0.08 & 32274.00 & 310678.23 & 1.00 & 0.96 & 0.96 \\
40027 & 108009 & 2006 & 823.70 & 0.09 & 82482.00 & 771136.25 & 1.00 & 0.94 & 0.93 \\
58408 & 410154 & 2006 & 19.30 & 0.14 & 1959.00 & 19657.79 & 0.99 & 1.02 & 1.00 \\
43397 & 109111 & 2006 & 383.60 & 0.06 & 35191.00 & 345592.40 & 1.09 & 0.90 & 0.98 \\
35852 & 106418 & 2006 & 1308.60 & 0.05 & 128992.00 & 1288528.57 & 1.01 & 0.98 & 1.00 \\
33129 & 106092 & 2006 & 552.30 & 0.19 & 55544.00 & 565354.13 & 0.99 & 1.02 & 1.02 \\
45282 & 200011 & 2006 & 210.00 & 0.09 & 28279.00 & 217059.07 & 0.74 & 1.03 & 0.77 \\
57045 & 400282 & 2006 & 44.00 & 0.01 & 4391.00 & 40716.22 & 1.00 & 0.93 & 0.93 \\
35810 & 106413 & 2006 & 4653.00 & 0.10 & 536313.00 & 5363131.15 & 0.87 & 1.15 & 1.00 \\
62619 & 500412 & 2006 & 169.20 & 0.37 & 16921.00 & 153848.19 & 1.00 & 0.91 & 0.91 \\
65027 & 500656 & 2006 & 1604.70 & 0.01 & 159592.00 & 1576375.83 & 1.01 & 0.98 & 0.99 \\
54929 & 400028 & 2006 & 6.40 & 0.11 & 671.00 & 6711.68 & 0.95 & 1.05 & 1.00 \\
57027 & 400281 & 2006 & 41.20 & 0.13 & 3873.00 & 39239.16 & 1.06 & 0.95 & 1.01 \\
42323 & 108951 & 2006 & 114.50 & 0.21 & 11350.00 & 92865.97 & 1.01 & 0.81 & 0.82 \\
35505 & 106367 & 2006 & 110.40 & 0.15 & 10730.00 & 110080.35 & 1.03 & 1.00 & 1.03 \\
50783 & 240450 & 2006 & 713.80 & 0.33 & 59999.00 & 650441.64 & 1.19 & 0.91 & 1.08 \\
64598 & 500612 & 2006 & 505.20 & 0.17 & 49365.00 & 508723.88 & 1.02 & 1.01 & 1.03 \\
49297 & 240266 & 2006 & 265.20 & 0.04 & 32164.00 & 266368.08 & 0.82 & 1.00 & 0.83 \\
53922 & 360123 & 2006 & 10.60 & 0.09 & 1048.00 & 10476.38 & 1.01 & 0.99 & 1.00 \\
48391 & 240074 & 2006 & 1232.80 & 0.19 & 113557.00 & 1018990.66 & 1.09 & 0.83 & 0.90 \\
74879 & 601188 & 2006 & 298.40 & 0.13 & 28735.00 & 250928.12 & 1.04 & 0.84 & 0.87 \\
39461 & 107694 & 2006 & 16.00 & 0.21 & 1603.00 & 15398.41 & 1.00 & 0.96 & 0.96 \\
46512 & 200249 & 2006 & 294.30 & 0.09 & 29394.00 & 291562.69 & 1.00 & 0.99 & 0.99 \\
39937 & 107958 & 2006 & 60.10 & 0.17 & 5728.00 & 54419.31 & 1.05 & 0.91 & 0.95 \\
55343 & 400087 & 2006 & 60.50 & 0.16 & 7860.00 & 65806.66 & 0.77 & 1.09 & 0.84 \\
57053 & 400283 & 2006 & 12.30 & 0.29 & 1173.00 & 11248.61 & 1.05 & 0.91 & 0.96 \\
62657 & 500416 & 2006 & 443.20 & 0.14 & 44651.00 & 446506.21 & 0.99 & 1.01 & 1.00 \\
54973 & 400034 & 2006 & 9.80 & 0.14 & 958.00 & 9727.26 & 1.02 & 0.99 & 1.02 \\
48046 & 227155 & 2006 & 445.20 & 0.08 & 43438.00 & 446314.53 & 1.02 & 1.00 & 1.03 \\
50763 & 240448 & 2006 & 19.20 & 0.07 & 3673.00 & 35993.87 & 0.52 & 1.87 & 0.98 \\
35766 & 106401 & 2006 & 3109.50 & 0.12 & 288232.00 & 2968577.81 & 1.08 & 0.95 & 1.03 \\
49323 & 240269 & 2006 & 37.00 & 0.35 & 4964.00 & 37720.06 & 0.75 & 1.02 & 0.76 \\
55878 & 400157 & 2006 & 731.20 & 0.18 & 64406.00 & 654974.54 & 1.14 & 0.90 & 1.02 \\
33102 & 106091 & 2006 & 521.70 & 0.06 & 51518.00 & 504324.96 & 1.01 & 0.97 & 0.98 \\
49334 & 240270 & 2006 & 149.80 & 0.10 & 14198.00 & 147975.22 & 1.06 & 0.99 & 1.04 \\
57071 & 400284 & 2006 & 25.40 & 0.05 & 2391.00 & 22990.95 & 1.06 & 0.91 & 0.96 \\
33088 & 106090 & 2006 & 1142.30 & 0.10 & 113846.00 & 1114997.51 & 1.00 & 0.98 & 0.98 \\
74859 & 601187 & 2006 & 108.30 & 0.16 & 12860.00 & 132501.05 & 0.84 & 1.22 & 1.03 \\
55899 & 400158 & 2006 & 130.00 & 0.08 & 11800.00 & 118690.83 & 1.10 & 0.91 & 1.01 \\
42168 & 108932 & 2006 & 1891.20 & 0.06 & 205324.00 & 1898950.69 & 0.92 & 1.00 & 0.92 \\
58329 & 410141 & 2006 & 219.40 & 0.13 & 17289.00 & 177128.61 & 1.27 & 0.81 & 1.02 \\
64660 & 500617 & 2006 & 8557.50 & 0.10 & 832423.00 & 8821483.63 & 1.03 & 1.03 & 1.06 \\
39482 & 107702 & 2006 & 1417.40 & 0.15 & 126630.00 & 1375666.09 & 1.12 & 0.97 & 1.09 \\
35792 & 106402 & 2006 & 310.30 & 0.15 & 24377.00 & 253758.31 & 1.27 & 0.82 & 1.04 \\
59172 & 410443 & 2006 & 30.00 & 0.02 & 3022.00 & 29443.10 & 0.99 & 0.98 & 0.97 \\
52400 & 302879 & 2006 & 1312.80 & 0.05 & 130360.00 & 1281209.97 & 1.01 & 0.98 & 0.98 \\
58358 & 410144 & 2006 & 1994.00 & 0.13 & 162087.00 & 1417066.87 & 1.23 & 0.71 & 0.87 \\
33152 & 106097 & 2006 & 548.20 & 0.07 & 54378.00 & 543133.28 & 1.01 & 0.99 & 1.00 \\
43701 & 109190 & 2006 & 28.90 & 0.18 & 3023.00 & 31500.47 & 0.96 & 1.09 & 1.04 \\
39442 & 107693 & 2006 & 212.80 & 0.23 & 21129.00 & 176657.33 & 1.01 & 0.83 & 0.84 \\
35525 & 106370 & 2006 & 91.30 & 0.16 & 9134.00 & 85910.13 & 1.00 & 0.94 & 0.94 \\
49153 & 240234 & 2006 & 165.50 & 0.28 & 16295.00 & 154747.62 & 1.02 & 0.94 & 0.95 \\
44214 & 109275 & 2006 & 1014.40 & 0.04 & 93912.00 & 915628.15 & 1.08 & 0.90 & 0.97 \\
42199 & 108934 & 2006 & 65.30 & 0.02 & 6332.00 & 67018.88 & 1.03 & 1.03 & 1.06 \\
55941 & 400160 & 2006 & 52.50 & 0.12 & 5139.00 & 44681.70 & 1.02 & 0.85 & 0.87 \\
55352 & 400088 & 2006 & 97.10 & 0.14 & 9711.00 & 95277.65 & 1.00 & 0.98 & 0.98 \\
62505 & 500395 & 2006 & 123.30 & 0.13 & 12269.00 & 123689.68 & 1.00 & 1.00 & 1.01 \\
46521 & 200250 & 2006 & 395.50 & 0.15 & 37445.00 & 393934.73 & 1.06 & 1.00 & 1.05 \\
45260 & 109444 & 2006 & 21.90 & 0.13 & 2015.00 & 20710.96 & 1.09 & 0.95 & 1.03 \\
39380 & 107677 & 2006 & 703.30 & 0.13 & 44934.00 & 457210.54 & 1.57 & 0.65 & 1.02 \\
43420 & 109112 & 2006 & 43.60 & 0.02 & 4468.00 & 43820.38 & 0.98 & 1.01 & 0.98 \\
47342 & 210203 & 2006 & 8749.60 & 0.02 & 864548.00 & 8645489.99 & 1.01 & 0.99 & 1.00 \\
50995 & 240475 & 2006 & 69.00 & 0.17 & 5237.00 & 51501.09 & 1.32 & 0.75 & 0.98 \\
39986 & 107967 & 2006 & 136.60 & 10.01 & 13491.00 & 132272.53 & 1.01 & 0.97 & 0.98 \\
50754 & 240447 & 2006 & 87.10 & 0.13 & 8343.00 & 88270.36 & 1.04 & 1.01 & 1.06 \\
49376 & 240286 & 2006 & 39.50 & 0.15 & 3941.00 & 40399.17 & 1.00 & 1.02 & 1.03 \\
52420 & 302907 & 2006 & 968.60 & 0.13 & 98746.00 & 1009288.45 & 0.98 & 1.04 & 1.02 \\
48691 & 240121 & 2006 & 239.70 & 0.09 & 27341.00 & 257094.08 & 0.88 & 1.07 & 0.94 \\
55857 & 400156 & 2006 & 731.20 & 0.09 & 64462.00 & 634155.76 & 1.13 & 0.87 & 0.98 \\
33052 & 106088 & 2006 & 112.50 & 0.11 & 11610.00 & 121473.95 & 0.97 & 1.08 & 1.05 \\
39398 & 107691 & 2006 & 40.40 & 0.19 & 3932.00 & 39527.85 & 1.03 & 0.98 & 1.01 \\
57247 & 400323 & 2006 & 257.20 & 0.12 & 25718.00 & 237117.69 & 1.00 & 0.92 & 0.92 \\
53899 & 360021 & 2006 & 37.10 & 0.04 & 3720.00 & 37203.19 & 1.00 & 1.00 & 1.00 \\
64575 & 500610 & 2006 & 677.60 & 0.09 & 69284.00 & 676732.35 & 0.98 & 1.00 & 0.98 \\
54675 & 378072 & 2006 & 299.20 & 0.13 & 30095.00 & 295063.32 & 0.99 & 0.99 & 0.98 \\
46489 & 200248 & 2006 & 244.60 & 0.09 & 23875.00 & 240337.34 & 1.02 & 0.98 & 1.01 \\
44968 & 109406 & 2006 & 182.50 & 0.04 & 20989.00 & 174569.79 & 0.87 & 0.96 & 0.83 \\
33169 & 106101 & 2006 & 41.30 & 0.17 & 7803.00 & 81606.53 & 0.53 & 1.98 & 1.05 \\
59150 & 410442 & 2006 & 833.60 & 0.14 & 83726.00 & 832865.59 & 1.00 & 1.00 & 0.99 \\
39962 & 107960 & 2006 & 498.90 & 0.11 & 49589.00 & 487810.58 & 1.01 & 0.98 & 0.98 \\
50953 & 240473 & 2006 & 89.30 & 0.09 & 8477.00 & 89808.70 & 1.05 & 1.01 & 1.06 \\
49344 & 240284 & 2006 & 95.00 & 0.07 & 9421.00 & 96994.87 & 1.01 & 1.02 & 1.03 \\
43431 & 109118 & 2006 & 166.30 & 0.10 & 17006.00 & 157365.20 & 0.98 & 0.95 & 0.93 \\
45301 & 200015 & 2006 & 48.50 & 0.13 & 5058.00 & 51219.10 & 0.96 & 1.06 & 1.01 \\
44962 & 109405 & 2006 & 42.20 & 0.06 & 4366.00 & 41336.03 & 0.97 & 0.98 & 0.95 \\
35836 & 106415 & 2006 & 138.30 & -0.01 & 13655.00 & 136553.06 & 1.01 & 0.99 & 1.00 \\
39423 & 107692 & 2006 & 22.90 & 0.19 & 2225.00 & 20434.12 & 1.03 & 0.89 & 0.92 \\
58142 & 410100 & 2006 & 953.40 & 0.08 & 89800.00 & 807227.31 & 1.06 & 0.85 & 0.90 \\
55920 & 400159 & 2006 & 118.60 & 0.09 & 10822.00 & 106457.26 & 1.10 & 0.90 & 0.98 \\
48035 & 226946 & 2006 & 292.30 & 0.07 & 31732.00 & 237926.44 & 0.92 & 0.81 & 0.75 \\
55102 & 400061 & 2006 & 3224.10 & 0.15 & 391277.00 & 3079369.04 & 0.82 & 0.96 & 0.79 \\
49194 & 240243 & 2006 & 1077.40 & 0.27 & 95923.00 & 811632.97 & 1.12 & 0.75 & 0.85 \\
40168 & 108070 & 2006 & 36.50 & 0.17 & 3537.00 & 35410.09 & 1.03 & 0.97 & 1.00 \\
39176 & 107616 & 2006 & 2964.40 & 0.22 & 295812.00 & 2922334.17 & 1.00 & 0.99 & 0.99 \\
56913 & 400265 & 2006 & 44.70 & 0.15 & 7893.00 & 77386.22 & 0.57 & 1.73 & 0.98 \\
33406 & 106129 & 2006 & 266.10 & 0.10 & 27120.00 & 277066.65 & 0.98 & 1.04 & 1.02 \\
55399 & 400093 & 2006 & 224.00 & 0.06 & 21713.00 & 181624.66 & 1.03 & 0.81 & 0.84 \\
46427 & 200243 & 2006 & 3.80 & 0.18 & 386.00 & 3746.02 & 0.98 & 0.99 & 0.97 \\
35389 & 106356 & 2006 & 21.10 & 0.11 & 2120.00 & 19916.31 & 1.00 & 0.94 & 0.94 \\
65403 & 500694 & 2006 & 1323.50 & 0.35 & 102570.00 & 1108301.00 & 1.29 & 0.84 & 1.08 \\
45155 & 109433 & 2006 & 56.80 & 0.09 & 5061.00 & 44706.55 & 1.12 & 0.79 & 0.88 \\
65442 & 500697 & 2006 & 18.10 & 0.12 & 1819.00 & 16230.94 & 1.00 & 0.90 & 0.89 \\
39631 & 107830 & 2006 & 262.30 & 0.21 & 26312.00 & 247385.11 & 1.00 & 0.94 & 0.94 \\
53781 & 357075 & 2006 & 635.60 & 0.08 & 61767.00 & 616702.19 & 1.03 & 0.97 & 1.00 \\
49218 & 240250 & 2006 & 812.20 & 0.22 & 79092.00 & 742960.01 & 1.03 & 0.91 & 0.94 \\
51097 & 240482 & 2006 & 4.30 & 0.16 & 413.00 & 3966.90 & 1.04 & 0.92 & 0.96 \\
62843 & 500443 & 2006 & 205.50 & 0.05 & 20578.00 & 205602.49 & 1.00 & 1.00 & 1.00 \\
39852 & 107882 & 2006 & 241.50 & 0.06 & 39269.00 & 368052.72 & 0.61 & 1.52 & 0.94 \\
54921 & 400025 & 2006 & 152.50 & 0.06 & 23350.00 & 200541.45 & 0.65 & 1.32 & 0.86 \\
50654 & 240434 & 2006 & 12.40 & 0.11 & 943.00 & 8531.00 & 1.31 & 0.69 & 0.90 \\
64486 & 500605 & 2006 & 901.30 & 0.20 & 84946.00 & 900205.04 & 1.06 & 1.00 & 1.06 \\
43259 & 109087 & 2006 & 101.20 & 0.13 & 11509.00 & 97989.69 & 0.88 & 0.97 & 0.85 \\
44262 & 109279 & 2006 & 110.50 & 0.09 & 11972.00 & 103830.37 & 0.92 & 0.94 & 0.87 \\
43507 & 109129 & 2006 & 137.30 & 0.20 & 13770.00 & 136368.78 & 1.00 & 0.99 & 0.99 \\
45386 & 200055 & 2006 & 151.70 & 0.13 & 14746.00 & 154920.86 & 1.03 & 1.02 & 1.05 \\
35963 & 106442 & 2006 & 5904.20 & 0.04 & 550811.00 & 5795262.00 & 1.07 & 0.98 & 1.05 \\
53814 & 357133 & 2006 & 545.40 & -0.06 & 54051.00 & 540510.99 & 1.01 & 0.99 & 1.00 \\
44138 & 109268 & 2006 & 2825.50 & 0.04 & 313019.00 & 3049129.36 & 0.90 & 1.08 & 0.97 \\
62465 & 500393 & 2006 & 115.80 & 0.17 & 11356.00 & 115168.83 & 1.02 & 0.99 & 1.01 \\
52346 & 302780 & 2006 & 82.60 & 0.10 & 8940.00 & 82705.72 & 0.92 & 1.00 & 0.93 \\
54061 & 364292 & 2006 & 439.00 & 0.09 & 45171.00 & 437802.79 & 0.97 & 1.00 & 0.97 \\
32947 & 106083 & 2006 & 719.50 & 0.06 & 78018.00 & 730192.22 & 0.92 & 1.01 & 0.94 \\
53794 & 357122 & 2006 & 1511.50 & 0.14 & 146920.00 & 1442110.46 & 1.03 & 0.95 & 0.98 \\
47386 & 210681 & 2006 & 68565.20 & 0.12 & 6706539.00 & 56973359.39 & 1.02 & 0.83 & 0.85 \\
45177 & 109435 & 2006 & 17.40 & 0.15 & 1225.00 & 12200.19 & 1.42 & 0.70 & 1.00 \\
96763 & 611011 & 2006 & 54.30 & 0.17 & 3905.00 & 38187.61 & 1.39 & 0.70 & 0.98 \\
53239 & 342127 & 2006 & 39.30 & 0.12 & 4146.00 & 40334.54 & 0.95 & 1.03 & 0.97 \\
33387 & 106127 & 2006 & 650.00 & 0.20 & 65420.00 & 631688.46 & 0.99 & 0.97 & 0.97 \\
48732 & 240134 & 2006 & 426.60 & 0.33 & 39638.00 & 417226.58 & 1.08 & 0.98 & 1.05 \\
43291 & 109090 & 2006 & 131.80 & 0.18 & 13161.00 & 126946.88 & 1.00 & 0.96 & 0.96 \\
62711 & 500420 & 2006 & 29.00 & 0.36 & 2669.00 & 26668.89 & 1.09 & 0.92 & 1.00 \\
47658 & 216504 & 2006 & 18.90 & 0.06 & 2605.00 & 24752.76 & 0.73 & 1.31 & 0.95 \\
39201 & 107618 & 2006 & 2784.40 & 0.10 & 301697.00 & 2749144.17 & 0.92 & 0.99 & 0.91 \\
56060 & 400170 & 2006 & 328.20 & 0.13 & 32606.00 & 334093.59 & 1.01 & 1.02 & 1.02 \\
52247 & 302698 & 2006 & 663.80 & 0.08 & 60958.00 & 600652.65 & 1.09 & 0.90 & 0.99 \\
39605 & 107786 & 2006 & 1087.00 & 0.09 & 108740.00 & 1063588.26 & 1.00 & 0.98 & 0.98 \\
47636 & 215952 & 2006 & 1041.50 & 0.10 & 101327.00 & 989972.18 & 1.03 & 0.95 & 0.98 \\
49392 & 240288 & 2006 & 18.70 & 0.16 & 1856.00 & 19021.12 & 1.01 & 1.02 & 1.02 \\
43529 & 109133 & 2006 & 48.40 & 0.08 & 4838.00 & 46713.26 & 1.00 & 0.97 & 0.97 \\
39151 & 107611 & 2006 & 7132.50 & 0.13 & 712450.00 & 7122514.61 & 1.00 & 1.00 & 1.00 \\
35355 & 106353 & 2006 & 1375.60 & 0.13 & 134013.00 & 1307639.53 & 1.03 & 0.95 & 0.98 \\
45412 & 200057 & 2006 & 1115.40 & 0.02 & 114734.00 & 1142283.41 & 0.97 & 1.02 & 1.00 \\
65467 & 500699 & 2006 & 351.70 & 0.08 & 34304.00 & 335395.30 & 1.03 & 0.95 & 0.98 \\
33452 & 106136 & 2006 & 220.30 & 0.15 & 20507.00 & 211887.09 & 1.07 & 0.96 & 1.03 \\
62828 & 500442 & 2006 & 700.70 & 0.08 & 68890.00 & 688509.80 & 1.02 & 0.98 & 1.00 \\
52221 & 302677 & 2006 & 75.20 & 0.15 & 5624.00 & 58630.60 & 1.34 & 0.78 & 1.04 \\
44273 & 109281 & 2006 & 164.30 & 0.08 & 19494.00 & 163028.40 & 0.84 & 0.99 & 0.84 \\
47258 & 200344 & 2006 & 19859.20 & 0.16 & 1895970.00 & 18212347.12 & 1.05 & 0.92 & 0.96 \\
40214 & 108073 & 2006 & 925.90 & 0.36 & 95489.00 & 933129.47 & 0.97 & 1.01 & 0.98 \\
54954 & 400030 & 2006 & 90.70 & 0.18 & 8837.00 & 89873.40 & 1.03 & 0.99 & 1.02 \\
44838 & 109394 & 2006 & 371.40 & 0.24 & 27299.00 & 246256.12 & 1.36 & 0.66 & 0.90 \\
59598 & 410502 & 2006 & 301.90 & 0.10 & 30435.00 & 291140.21 & 0.99 & 0.96 & 0.96 \\
39125 & 107607 & 2006 & 146.30 & 0.20 & 14847.00 & 141268.92 & 0.99 & 0.97 & 0.95 \\
54696 & 378134 & 2006 & 767.50 & 0.05 & 77003.00 & 780239.36 & 1.00 & 1.02 & 1.01 \\
51114 & 240485 & 2006 & 1040.20 & 0.16 & 80236.00 & 849023.13 & 1.30 & 0.82 & 1.06 \\
64870 & 500638 & 2006 & 112.00 & 0.15 & 10806.00 & 114160.40 & 1.04 & 1.02 & 1.06 \\
64939 & 500646 & 2006 & 2.90 & 0.09 & 284.00 & 2756.25 & 1.02 & 0.95 & 0.97 \\
39135 & 107608 & 2006 & 40.50 & 0.23 & 4064.00 & 39526.10 & 1.00 & 0.98 & 0.97 \\
48441 & 240083 & 2006 & 38.40 & 0.16 & 4019.00 & 41284.67 & 0.96 & 1.08 & 1.03 \\
58194 & 410115 & 2006 & 1240.80 & 0.09 & 116072.00 & 1156979.01 & 1.07 & 0.93 & 1.00 \\
35991 & 106444 & 2006 & 241.30 & 0.09 & 27486.00 & 251213.64 & 0.88 & 1.04 & 0.91 \\
48418 & 240080 & 2006 & 587.20 & 0.12 & 55625.00 & 456923.98 & 1.06 & 0.78 & 0.82 \\
59195 & 410444 & 2006 & 170.80 & 0.10 & 17463.00 & 174300.25 & 0.98 & 1.02 & 1.00 \\
42121 & 108924 & 2006 & 161.00 & 0.38 & 14778.00 & 164487.61 & 1.09 & 1.02 & 1.11 \\
53215 & 341126 & 2006 & 7577.20 & 0.11 & 730168.00 & 6231781.28 & 1.04 & 0.82 & 0.85 \\
40188 & 108071 & 2006 & 62.40 & -0.01 & 6258.00 & 61418.50 & 1.00 & 0.98 & 0.98 \\
33425 & 106135 & 2006 & 64.70 & 0.19 & 6285.00 & 62713.27 & 1.03 & 0.97 & 1.00 \\
61217 & 410909 & 2006 & 117.70 & 0.21 & 14940.00 & 121139.87 & 0.79 & 1.03 & 0.81 \\
61242 & 500005 & 2006 & 497.40 & 0.22 & 45432.00 & 393168.30 & 1.09 & 0.79 & 0.87 \\
42376 & 108955 & 2006 & 45.40 & 0.08 & 4426.00 & 44856.58 & 1.03 & 0.99 & 1.01 \\
46422 & 200241 & 2006 & 3.00 & 0.17 & 299.00 & 2939.21 & 1.00 & 0.98 & 0.98 \\
58105 & 410093 & 2006 & 81.10 & -0.03 & 8411.00 & 82781.47 & 0.96 & 1.02 & 0.98 \\
58210 & 410118 & 2006 & 349.40 & 0.12 & 34300.00 & 287596.02 & 1.02 & 0.82 & 0.84 \\
52355 & 302811 & 2006 & 25.90 & 0.11 & 2853.00 & 29409.08 & 0.91 & 1.14 & 1.03 \\
51104 & 240483 & 2006 & 3.20 & 0.14 & 322.00 & 2878.70 & 0.99 & 0.90 & 0.89 \\
49080 & 240218 & 2006 & 1249.60 & 0.04 & 119794.00 & 1125056.86 & 1.04 & 0.90 & 0.94 \\
59649 & 410506 & 2006 & 70.60 & 0.13 & 5907.00 & 58086.76 & 1.20 & 0.82 & 0.98 \\
54927 & 400027 & 2006 & 13.90 & 0.09 & 1372.00 & 13845.04 & 1.01 & 1.00 & 1.01 \\
39215 & 107619 & 2006 & 395.20 & 0.15 & 41831.00 & 381069.79 & 0.94 & 0.96 & 0.91 \\
64671 & 500618 & 2006 & 948.00 & 0.32 & 92321.00 & 946255.27 & 1.03 & 1.00 & 1.02 \\
56969 & 400269 & 2006 & 1260.00 & 0.15 & 109484.00 & 1138647.96 & 1.15 & 0.90 & 1.04 \\
39871 & 107892 & 2006 & 1457.00 & 0.29 & 145682.00 & 1168345.41 & 1.00 & 0.80 & 0.80 \\
40078 & 108021 & 2006 & 1714.20 & 0.10 & 174278.00 & 1569495.27 & 0.98 & 0.92 & 0.90 \\
74901 & 601190 & 2006 & 478.20 & 0.07 & 58032.00 & 538808.10 & 0.82 & 1.13 & 0.93 \\
47442 & 211051 & 2006 & 622.70 & 0.13 & 58071.00 & 533525.03 & 1.07 & 0.86 & 0.92 \\
60895 & 410739 & 2006 & 39.40 & 0.07 & 4064.00 & 41581.56 & 0.97 & 1.06 & 1.02 \\
42274 & 108947 & 2006 & 270.00 & 0.12 & 27680.00 & 286675.88 & 0.98 & 1.06 & 1.04 \\
43331 & 109095 & 2006 & 33.00 & 0.18 & 3200.00 & 31297.56 & 1.03 & 0.95 & 0.98 \\
45371 & 200050 & 2006 & 149.80 & 0.32 & 13956.00 & 138848.40 & 1.07 & 0.93 & 0.99 \\
53858 & 359285 & 2006 & 367.10 & 0.17 & 36685.00 & 312508.83 & 1.00 & 0.85 & 0.85 \\
40102 & 108029 & 2006 & 512.60 & 0.11 & 51472.00 & 477120.29 & 1.00 & 0.93 & 0.93 \\
63010 & 500466 & 2006 & 859.20 & 0.04 & 85782.00 & 845360.09 & 1.00 & 0.98 & 0.99 \\
43319 & 109094 & 2006 & 182.70 & 0.15 & 18304.00 & 190932.19 & 1.00 & 1.05 & 1.04 \\
57293 & 400393 & 2006 & 15.90 & 0.04 & 1588.00 & 14421.26 & 1.00 & 0.91 & 0.91 \\
57221 & 400322 & 2006 & 120.40 & 0.16 & 11203.00 & 110559.24 & 1.07 & 0.92 & 0.99 \\
44907 & 109399 & 2006 & 55.60 & 0.16 & 5037.00 & 46574.55 & 1.10 & 0.84 & 0.92 \\
57306 & 400394 & 2006 & 7.00 & 0.16 & 640.00 & 5584.38 & 1.09 & 0.80 & 0.87 \\
52452 & 302942 & 2006 & 2982.10 & 0.12 & 297888.00 & 2953743.32 & 1.00 & 0.99 & 0.99 \\
56950 & 400268 & 2006 & 620.50 & 0.13 & 54488.00 & 507738.24 & 1.14 & 0.82 & 0.93 \\
39289 & 107648 & 2006 & 227.00 & -0.01 & 23803.00 & 233554.74 & 0.95 & 1.03 & 0.98 \\
43339 & 109099 & 2006 & 253.70 & 0.03 & 24931.00 & 257458.87 & 1.02 & 1.01 & 1.03 \\
55617 & 400131 & 2006 & 389.70 & 0.14 & 39309.00 & 392640.81 & 0.99 & 1.01 & 1.00 \\
58115 & 410094 & 2006 & 5249.10 & 0.13 & 453816.00 & 4292870.44 & 1.16 & 0.82 & 0.95 \\
54648 & 377385 & 2006 & 395.80 & 0.16 & 39547.00 & 382267.14 & 1.00 & 0.97 & 0.97 \\
57277 & 400390 & 2006 & 9.00 & 0.10 & 858.00 & 9030.95 & 1.05 & 1.00 & 1.05 \\
40050 & 108018 & 2006 & 445.00 & 0.16 & 44410.00 & 355691.52 & 1.00 & 0.80 & 0.80 \\
57284 & 400391 & 2006 & 65.00 & 0.14 & 6510.00 & 63457.84 & 1.00 & 0.98 & 0.97 \\
33289 & 106113 & 2006 & 590.70 & 0.16 & 60318.00 & 570816.98 & 0.98 & 0.97 & 0.95 \\
42339 & 108952 & 2006 & 870.30 & 0.08 & 88127.00 & 807085.95 & 0.99 & 0.93 & 0.92 \\
65211 & 500670 & 2006 & 426.00 & 0.45 & 41322.00 & 390890.10 & 1.03 & 0.92 & 0.95 \\
55337 & 400085 & 2006 & 20.40 & -0.00 & 2037.00 & 17411.03 & 1.00 & 0.85 & 0.85 \\
44160 & 109269 & 2006 & 1513.40 & 0.18 & 156631.00 & 1227822.85 & 0.97 & 0.81 & 0.78 \\
35430 & 106360 & 2006 & 2214.40 & 0.34 & 220973.00 & 2119477.63 & 1.00 & 0.96 & 0.96 \\
44917 & 109401 & 2006 & 60.20 & 0.08 & 5492.00 & 55250.07 & 1.10 & 0.92 & 1.01 \\
48380 & 240067 & 2006 & 244.00 & 0.06 & 31251.00 & 202721.82 & 0.78 & 0.83 & 0.65 \\
56039 & 400167 & 2006 & 281.70 & 0.09 & 26819.00 & 288099.74 & 1.05 & 1.02 & 1.07 \\
57009 & 400277 & 2006 & 2556.30 & 0.12 & 256781.00 & 2361973.86 & 1.00 & 0.92 & 0.92 \\
62881 & 500445 & 2006 & 125.60 & 0.09 & 12601.00 & 125990.91 & 1.00 & 1.00 & 1.00 \\
35560 & 106375 & 2006 & 20.60 & 0.09 & 2063.00 & 19593.08 & 1.00 & 0.95 & 0.95 \\
45345 & 200047 & 2006 & 105.60 & -0.02 & 10469.00 & 86337.13 & 1.01 & 0.82 & 0.82 \\
65302 & 500684 & 2006 & 2598.10 & 0.39 & 275750.00 & 2755585.90 & 0.94 & 1.06 & 1.00 \\
64509 & 500606 & 2006 & 1218.80 & 0.06 & 123303.00 & 1224066.40 & 0.99 & 1.00 & 0.99 \\
50662 & 240437 & 2006 & 2102.70 & 0.07 & 398366.00 & 3929335.23 & 0.53 & 1.87 & 0.99 \\
64721 & 500621 & 2006 & 1714.30 & 0.35 & 151490.00 & 1516335.49 & 1.13 & 0.88 & 1.00 \\
58410 & 410155 & 2006 & 11773.70 & 0.19 & 790258.00 & 7867802.04 & 1.49 & 0.67 & 1.00 \\
53837 & 357756 & 2006 & 112.00 & 0.10 & 11155.00 & 102655.52 & 1.00 & 0.92 & 0.92 \\
57345 & 400402 & 2006 & 198.80 & 0.13 & 19813.00 & 193013.89 & 1.00 & 0.97 & 0.97 \\
33361 & 106124 & 2006 & 2351.70 & 0.19 & 208118.00 & 2081375.76 & 1.13 & 0.89 & 1.00 \\
54050 & 364291 & 2006 & 97.70 & 0.13 & 10282.00 & 104952.17 & 0.95 & 1.07 & 1.02 \\
53108 & 339611 & 2006 & 30.80 & 0.22 & 3199.00 & 32087.89 & 0.96 & 1.04 & 1.00 \\
60744 & 410722 & 2006 & 332.40 & 0.19 & 31013.00 & 326815.76 & 1.07 & 0.98 & 1.05 \\
43484 & 109127 & 2006 & 38.50 & 0.09 & 3844.00 & 32364.97 & 1.00 & 0.84 & 0.84 \\
43644 & 109176 & 2006 & 78.10 & 0.13 & 7809.00 & 77309.50 & 1.00 & 0.99 & 0.99 \\
39234 & 107623 & 2006 & 2.60 & 0.02 & 277.00 & 2488.89 & 0.94 & 0.96 & 0.90 \\
45184 & 109436 & 2006 & 84.20 & 0.04 & 8410.00 & 80543.90 & 1.00 & 0.96 & 0.96 \\
74845 & 601186 & 2006 & 8.00 & 0.01 & 803.00 & 8034.91 & 1.00 & 1.00 & 1.00 \\
40158 & 108051 & 2006 & 169.10 & 0.18 & 20166.00 & 172526.30 & 0.84 & 1.02 & 0.86 \\
51073 & 240481 & 2006 & 104.60 & 0.10 & 11059.00 & 92467.42 & 0.95 & 0.88 & 0.84 \\
52273 & 302731 & 2006 & 3583.90 & 0.13 & 351273.00 & 3214487.06 & 1.02 & 0.90 & 0.92 \\
65374 & 500692 & 2006 & 127.40 & 0.06 & 12392.00 & 130032.66 & 1.03 & 1.02 & 1.05 \\
62862 & 500444 & 2006 & 383.20 & 0.24 & 38420.00 & 383635.24 & 1.00 & 1.00 & 1.00 \\
43308 & 109092 & 2006 & 84.60 & 0.18 & 8152.00 & 83545.33 & 1.04 & 0.99 & 1.02 \\
39243 & 107626 & 2006 & 1176.30 & 0.13 & 126146.00 & 1129377.15 & 0.93 & 0.96 & 0.90 \\
42364 & 108953 & 2006 & 124.90 & 0.04 & 12543.00 & 125313.17 & 1.00 & 1.00 & 1.00 \\
39584 & 107726 & 2006 & 1679.40 & 0.16 & 172816.00 & 1690490.81 & 0.97 & 1.01 & 0.98 \\
33343 & 106116 & 2006 & 12.50 & 0.08 & 1222.00 & 10056.36 & 1.02 & 0.80 & 0.82 \\
57324 & 400395 & 2006 & 4.30 & 0.23 & 393.00 & 3381.79 & 1.09 & 0.79 & 0.86 \\
57089 & 400285 & 2006 & 24.00 & 0.19 & 2246.00 & 22946.93 & 1.07 & 0.96 & 1.02 \\
44884 & 109397 & 2006 & 1344.20 & 0.09 & 153264.00 & 1135213.45 & 0.88 & 0.84 & 0.74 \\
54027 & 363941 & 2006 & 416.10 & 0.15 & 34927.00 & 317488.41 & 1.19 & 0.76 & 0.91 \\
39596 & 107781 & 2006 & 142.50 & 0.15 & 14284.00 & 135784.83 & 1.00 & 0.95 & 0.95 \\
53846 & 357762 & 2006 & 43.80 & 0.08 & 4483.00 & 40499.13 & 0.98 & 0.92 & 0.90 \\
64698 & 500620 & 2006 & 785.90 & 0.21 & 77251.00 & 751551.59 & 1.02 & 0.96 & 0.97 \\
35709 & 106391 & 2006 & 89.00 & 0.03 & 8877.00 & 87603.04 & 1.00 & 0.98 & 0.99 \\
56931 & 400266 & 2006 & 34.60 & 0.16 & 3085.00 & 29665.99 & 1.12 & 0.86 & 0.96 \\
47306 & 200505 & 2006 & 976.50 & 0.15 & 99434.00 & 958767.89 & 0.98 & 0.98 & 0.96 \\
39260 & 107627 & 2006 & 1465.20 & 0.19 & 161903.00 & 1384575.49 & 0.90 & 0.94 & 0.86 \\
40134 & 108037 & 2006 & 126.10 & 0.21 & 12443.00 & 113586.01 & 1.01 & 0.90 & 0.91 \\
32974 & 106084 & 2006 & 792.80 & 0.06 & 82381.00 & 754626.76 & 0.96 & 0.95 & 0.92 \\
46550 & 200252 & 2006 & 10.00 & 0.14 & 934.00 & 9163.36 & 1.07 & 0.92 & 0.98 \\
58583 & 410170 & 2006 & 441.60 & 0.16 & 38223.00 & 400873.61 & 1.16 & 0.91 & 1.05 \\
34771 & 106276 & 2006 & 261.00 & 0.07 & 26034.00 & 260415.11 & 1.00 & 1.00 & 1.00 \\
58605 & 410178 & 2006 & 55.80 & 0.21 & 5581.00 & 53738.79 & 1.00 & 0.96 & 0.96 \\
51469 & 240522 & 2006 & 4.50 & 0.05 & 425.00 & 4365.85 & 1.06 & 0.97 & 1.03 \\
49767 & 240358 & 2006 & 46.70 & 0.06 & 4910.00 & 47335.99 & 0.95 & 1.01 & 0.96 \\
37580 & 106969 & 2006 & 59.30 & 0.15 & 5888.00 & 58444.16 & 1.01 & 0.99 & 0.99 \\
54555 & 375941 & 2006 & 70.20 & 0.12 & 4375.00 & 42496.86 & 1.60 & 0.61 & 0.97 \\
34434 & 106239 & 2006 & 44.70 & 0.07 & 5552.00 & 43804.26 & 0.81 & 0.98 & 0.79 \\
51801 & 240557 & 2006 & 42.90 & 0.20 & 3560.00 & 36767.15 & 1.21 & 0.86 & 1.03 \\
54408 & 367600 & 2006 & 246.00 & 0.13 & 24347.00 & 234968.43 & 1.01 & 0.96 & 0.97 \\
37555 & 106968 & 2006 & 40.60 & 0.06 & 4258.00 & 42458.13 & 0.95 & 1.05 & 1.00 \\
41682 & 108839 & 2006 & 117.40 & 0.05 & 11726.00 & 116810.82 & 1.00 & 0.99 & 1.00 \\
56591 & 400229 & 2006 & 20.20 & 0.05 & 2151.00 & 21205.51 & 0.94 & 1.05 & 0.99 \\
50132 & 240395 & 2006 & 337.00 & 0.36 & 31360.00 & 301018.47 & 1.07 & 0.89 & 0.96 \\
55582 & 400127 & 2006 & 131.10 & 0.10 & 12954.00 & 129539.61 & 1.01 & 0.99 & 1.00 \\
41139 & 108203 & 2006 & 101.40 & 0.21 & 9494.00 & 90721.13 & 1.07 & 0.89 & 0.96 \\
51487 & 240524 & 2006 & 12.20 & 0.05 & 1225.00 & 12201.81 & 1.00 & 1.00 & 1.00 \\
34999 & 106305 & 2006 & 931.40 & 0.14 & 93666.00 & 868050.94 & 0.99 & 0.93 & 0.93 \\
41162 & 108211 & 2006 & 838.20 & 0.10 & 83370.00 & 828968.46 & 1.01 & 0.99 & 0.99 \\
51796 & 240556 & 2006 & 42.50 & 0.14 & 3770.00 & 37515.92 & 1.13 & 0.88 & 1.00 \\
36694 & 106577 & 2006 & 4520.10 & 0.08 & 454696.00 & 4021301.34 & 0.99 & 0.89 & 0.88 \\
51790 & 240554 & 2006 & 232.60 & 0.09 & 23235.00 & 231425.90 & 1.00 & 0.99 & 1.00 \\
43944 & 109237 & 2006 & 294.20 & 0.28 & 23189.00 & 243844.56 & 1.27 & 0.83 & 1.05 \\
34450 & 106240 & 2006 & 177.50 & 0.16 & 16519.00 & 175518.99 & 1.07 & 0.99 & 1.06 \\
51786 & 240553 & 2006 & 30.00 & 0.09 & 2997.00 & 29376.92 & 1.00 & 0.98 & 0.98 \\
46809 & 200300 & 2006 & 8.40 & 0.08 & 814.00 & 8186.53 & 1.03 & 0.97 & 1.01 \\
55209 & 400072 & 2006 & 452.50 & 0.16 & 45638.00 & 445959.60 & 0.99 & 0.99 & 0.98 \\
47804 & 221485 & 2006 & 316.10 & 0.13 & 31213.00 & 329025.10 & 1.01 & 1.04 & 1.05 \\
51819 & 240558 & 2006 & 71.20 & 0.16 & 7118.00 & 69126.36 & 1.00 & 0.97 & 0.97 \\
51848 & 240560 & 2006 & 159.20 & 0.22 & 15955.00 & 142856.59 & 1.00 & 0.90 & 0.90 \\
41104 & 108200 & 2006 & 68.00 & 0.09 & 6761.00 & 65041.73 & 1.01 & 0.96 & 0.96 \\
51914 & 300102 & 2006 & 28.30 & 0.13 & 2977.00 & 28080.41 & 0.95 & 0.99 & 0.94 \\
55465 & 400100 & 2006 & 415.60 & 0.17 & 42024.00 & 392033.74 & 0.99 & 0.94 & 0.93 \\
55483 & 400101 & 2006 & 5.70 & 0.10 & 576.00 & 5478.59 & 0.99 & 0.96 & 0.95 \\
34374 & 106230 & 2006 & 262.30 & 0.13 & 26274.00 & 262739.59 & 1.00 & 1.00 & 1.00 \\
59328 & 410465 & 2006 & 141.20 & 0.11 & 12141.00 & 120583.43 & 1.16 & 0.85 & 0.99 \\
42593 & 108984 & 2006 & 96.80 & 0.19 & 9240.00 & 95531.48 & 1.05 & 0.99 & 1.03 \\
36633 & 106571 & 2006 & 590.20 & 0.01 & 59010.00 & 587331.94 & 1.00 & 1.00 & 1.00 \\
50159 & 240397 & 2006 & 15.00 & 0.04 & 1730.00 & 14668.99 & 0.87 & 0.98 & 0.85 \\
56446 & 400210 & 2006 & 185.00 & 0.11 & 18344.00 & 180302.25 & 1.01 & 0.97 & 0.98 \\
59055 & 410418 & 2006 & 1277.80 & 0.14 & 177475.00 & 1728101.49 & 0.72 & 1.35 & 0.97 \\
58926 & 410235 & 2006 & 4.40 & 0.05 & 444.00 & 3965.74 & 0.99 & 0.90 & 0.89 \\
58952 & 410239 & 2006 & 89.00 & 0.03 & 8944.00 & 86313.08 & 1.00 & 0.97 & 0.97 \\
51453 & 240521 & 2006 & 14.30 & 0.12 & 1324.00 & 13411.26 & 1.08 & 0.94 & 1.01 \\
34401 & 106231 & 2006 & 312.50 & 0.21 & 31323.00 & 313048.61 & 1.00 & 1.00 & 1.00 \\
41707 & 108840 & 2006 & 512.70 & 0.20 & 51404.00 & 507948.32 & 1.00 & 0.99 & 0.99 \\
37605 & 106972 & 2006 & 79.90 & 0.12 & 8343.00 & 86855.14 & 0.96 & 1.09 & 1.04 \\
36659 & 106573 & 2006 & 8.90 & 0.14 & 893.00 & 8516.52 & 1.00 & 0.96 & 0.95 \\
56449 & 400211 & 2006 & 336.90 & 0.15 & 33537.00 & 330049.72 & 1.00 & 0.98 & 0.98 \\
51884 & 240562 & 2006 & 98.30 & 0.14 & 9831.00 & 89665.68 & 1.00 & 0.91 & 0.91 \\
58990 & 410242 & 2006 & 56.20 & 0.19 & 4704.00 & 45380.85 & 1.19 & 0.81 & 0.96 \\
63940 & 500568 & 2006 & 73.30 & 0.03 & 6931.00 & 66636.34 & 1.06 & 0.91 & 0.96 \\
46273 & 200205 & 2006 & 1829.10 & 0.21 & 182840.00 & 1612374.67 & 1.00 & 0.88 & 0.88 \\
41116 & 108202 & 2006 & 119.20 & 0.06 & 12185.00 & 118514.54 & 0.98 & 0.99 & 0.97 \\
51866 & 240561 & 2006 & 254.00 & 0.15 & 24292.00 & 221071.98 & 1.05 & 0.87 & 0.91 \\
51837 & 240559 & 2006 & 54.20 & 0.13 & 3689.00 & 37534.37 & 1.47 & 0.69 & 1.02 \\
52865 & 332404 & 2006 & 198.90 & 0.20 & 17093.00 & 176008.13 & 1.16 & 0.88 & 1.03 \\
54782 & 400012 & 2006 & 31.30 & 0.13 & 3011.00 & 29586.70 & 1.04 & 0.95 & 0.98 \\
62081 & 500327 & 2006 & 3.40 & 0.18 & 241.00 & 2440.52 & 1.41 & 0.72 & 1.01 \\
51756 & 240549 & 2006 & 125.40 & 0.11 & 12236.00 & 129254.78 & 1.02 & 1.03 & 1.06 \\
44041 & 109260 & 2006 & 224.10 & 0.07 & 27400.00 & 219805.36 & 0.82 & 0.98 & 0.80 \\
49787 & 240360 & 2006 & 1633.50 & 0.14 & 164175.00 & 1439248.02 & 0.99 & 0.88 & 0.88 \\
37484 & 106931 & 2006 & 226.40 & -0.00 & 25557.00 & 249855.90 & 0.89 & 1.10 & 0.98 \\
56452 & 400212 & 2006 & 272.40 & 0.21 & 28841.00 & 268681.38 & 0.94 & 0.99 & 0.93 \\
55188 & 400069 & 2006 & 38.00 & 0.06 & 3740.00 & 36930.91 & 1.02 & 0.97 & 0.99 \\
53436 & 349198 & 2006 & 769.60 & 0.02 & 87764.00 & 845742.05 & 0.88 & 1.10 & 0.96 \\
52685 & 306482 & 2006 & 243.00 & 0.13 & 22456.00 & 222194.60 & 1.08 & 0.91 & 0.99 \\
96669 & 611002 & 2006 & 2709.80 & 0.17 & 268734.00 & 2821262.41 & 1.01 & 1.04 & 1.05 \\
74652 & 601147 & 2006 & 175.50 & 0.18 & 16919.00 & 163085.92 & 1.04 & 0.93 & 0.96 \\
56471 & 400213 & 2006 & 4.40 & 0.17 & 417.00 & 4169.62 & 1.06 & 0.95 & 1.00 \\
34977 & 106298 & 2006 & 93.10 & 0.11 & 9439.00 & 93697.36 & 0.99 & 1.01 & 0.99 \\
48272 & 240058 & 2006 & 801.00 & 0.04 & 88614.00 & 802535.45 & 0.90 & 1.00 & 0.91 \\
41243 & 108690 & 2006 & 138.70 & 0.43 & 14899.00 & 123877.66 & 0.93 & 0.89 & 0.83 \\
47692 & 220681 & 2006 & 447.80 & 0.14 & 45402.00 & 389827.54 & 0.99 & 0.87 & 0.86 \\
41674 & 108827 & 2006 & 512.00 & 0.07 & 60255.00 & 525270.02 & 0.85 & 1.03 & 0.87 \\
53568 & 351891 & 2006 & 6.70 & 0.06 & 651.00 & 6292.61 & 1.03 & 0.94 & 0.97 \\
54442 & 367842 & 2006 & 329.70 & 0.10 & 34796.00 & 326439.40 & 0.95 & 0.99 & 0.94 \\
55174 & 400065 & 2006 & 119.20 & 0.09 & 11921.00 & 119168.60 & 1.00 & 1.00 & 1.00 \\
74771 & 601165 & 2006 & 21.60 & 0.10 & 2261.00 & 22468.33 & 0.96 & 1.04 & 0.99 \\
48615 & 240114 & 2006 & 248.40 & 0.03 & 25023.00 & 260265.69 & 0.99 & 1.05 & 1.04 \\
36750 & 106584 & 2006 & 3425.90 & 0.17 & 311330.00 & 3173430.27 & 1.10 & 0.93 & 1.02 \\
44646 & 109350 & 2006 & 45.20 & 0.17 & 6993.00 & 69926.90 & 0.65 & 1.55 & 1.00 \\
61984 & 500315 & 2006 & 117.70 & 0.21 & 11652.00 & 108647.68 & 1.01 & 0.92 & 0.93 \\
42819 & 109020 & 2006 & 226.40 & -0.00 & 22124.00 & 226105.21 & 1.02 & 1.00 & 1.02 \\
46759 & 200295 & 2006 & 438.40 & 0.17 & 44016.00 & 429595.96 & 1.00 & 0.98 & 0.98 \\
51516 & 240527 & 2006 & 955.70 & 0.16 & 93482.00 & 983462.37 & 1.02 & 1.03 & 1.05 \\
53577 & 354018 & 2006 & 142.50 & 0.15 & 14240.00 & 138238.34 & 1.00 & 0.97 & 0.97 \\
51774 & 240551 & 2006 & 30.20 & 0.15 & 3088.00 & 29679.52 & 0.98 & 0.98 & 0.96 \\
37528 & 106948 & 2006 & 371.50 & 0.15 & 44072.00 & 361760.32 & 0.84 & 0.97 & 0.82 \\
46626 & 200266 & 2006 & 9.70 & 0.15 & 909.00 & 8213.12 & 1.07 & 0.85 & 0.90 \\
50094 & 240392 & 2006 & 653.60 & 0.16 & 59691.00 & 604327.76 & 1.09 & 0.92 & 1.01 \\
55498 & 400113 & 2006 & 1165.80 & 0.07 & 100121.00 & 977698.55 & 1.16 & 0.84 & 0.98 \\
41191 & 108670 & 2006 & 249.20 & 0.12 & 23072.00 & 224596.90 & 1.08 & 0.90 & 0.97 \\
63436 & 500508 & 2006 & 6733.10 & 0.18 & 655311.00 & 6754790.14 & 1.03 & 1.00 & 1.03 \\
45895 & 200156 & 2006 & 93.00 & 0.13 & 8078.00 & 74097.45 & 1.15 & 0.80 & 0.92 \\
62061 & 500326 & 2006 & 13.50 & 0.21 & 1339.00 & 12640.99 & 1.01 & 0.94 & 0.94 \\
37493 & 106934 & 2006 & 488.70 & 0.03 & 58787.00 & 487517.94 & 0.83 & 1.00 & 0.83 \\
36720 & 106580 & 2006 & 154.10 & 0.13 & 15558.00 & 152137.09 & 0.99 & 0.99 & 0.98 \\
41216 & 108673 & 2006 & 146.80 & 0.17 & 14835.00 & 144481.65 & 0.99 & 0.98 & 0.97 \\
52679 & 305766 & 2006 & 81.20 & 0.07 & 8232.00 & 82287.75 & 0.99 & 1.01 & 1.00 \\
51511 & 240526 & 2006 & 137.10 & 0.23 & 10718.00 & 109792.06 & 1.28 & 0.80 & 1.02 \\
49778 & 240359 & 2006 & 29.20 & 0.02 & 3120.00 & 27517.39 & 0.94 & 0.94 & 0.88 \\
43962 & 109238 & 2006 & 16.30 & 0.13 & 1534.00 & 15750.94 & 1.06 & 0.97 & 1.03 \\
42803 & 109019 & 2006 & 101.30 & 0.23 & 10132.00 & 97412.59 & 1.00 & 0.96 & 0.96 \\
46782 & 200297 & 2006 & 2819.10 & 0.14 & 271830.00 & 2683674.21 & 1.04 & 0.95 & 0.99 \\
63881 & 500563 & 2006 & 53.40 & 0.13 & 5589.00 & 53354.28 & 0.96 & 1.00 & 0.95 \\
42795 & 109018 & 2006 & 16.70 & 0.07 & 1650.00 & 16383.53 & 1.01 & 0.98 & 0.99 \\
54434 & 367841 & 2006 & 567.00 & 0.18 & 60918.00 & 599651.42 & 0.93 & 1.06 & 0.98 \\
36736 & 106583 & 2006 & 59.70 & 0.04 & 5913.00 & 55244.36 & 1.01 & 0.93 & 0.93 \\
63953 & 500571 & 2006 & 58.80 & 0.04 & 5887.00 & 58871.65 & 1.00 & 1.00 & 1.00 \\
56598 & 400230 & 2006 & 23.40 & -0.01 & 2333.00 & 21481.35 & 1.00 & 0.92 & 0.92 \\
36541 & 106560 & 2006 & 134.10 & 0.16 & 11024.00 & 114235.30 & 1.22 & 0.85 & 1.04 \\
58825 & 410223 & 2006 & 3.70 & 0.13 & 294.00 & 2956.90 & 1.26 & 0.80 & 1.01 \\
45833 & 200147 & 2006 & 64.60 & 0.05 & 6799.00 & 65473.13 & 0.95 & 1.01 & 0.96 \\
64056 & 500585 & 2006 & 2954.00 & 0.13 & 270815.00 & 2563319.35 & 1.09 & 0.87 & 0.95 \\
53612 & 354931 & 2006 & 274.90 & 0.20 & 26365.00 & 263637.32 & 1.04 & 0.96 & 1.00 \\
43887 & 109228 & 2006 & 83.20 & 0.09 & 7303.00 & 63681.43 & 1.14 & 0.77 & 0.87 \\
54848 & 400018 & 2006 & 180.10 & 0.20 & 18273.00 & 187142.01 & 0.99 & 1.04 & 1.02 \\
52901 & 333181 & 2006 & 85.40 & 0.16 & 7561.00 & 78203.94 & 1.13 & 0.92 & 1.03 \\
35039 & 106309 & 2006 & 1541.40 & 0.30 & 153669.00 & 1457286.62 & 1.00 & 0.95 & 0.95 \\
37755 & 107141 & 2006 & 2018.60 & 0.14 & 190298.00 & 1964516.34 & 1.06 & 0.97 & 1.03 \\
34299 & 106221 & 2006 & 122.90 & 0.20 & 12021.00 & 122745.99 & 1.02 & 1.00 & 1.02 \\
58848 & 410226 & 2006 & 246.50 & 0.12 & 18162.00 & 175966.98 & 1.36 & 0.71 & 0.97 \\
45850 & 200148 & 2006 & 522.90 & 0.11 & 50774.00 & 486514.13 & 1.03 & 0.93 & 0.96 \\
42905 & 109031 & 2006 & 38.40 & 0.09 & 3683.00 & 36212.35 & 1.04 & 0.94 & 0.98 \\
58866 & 410227 & 2006 & 2.30 & 0.09 & 227.00 & 2127.71 & 1.01 & 0.93 & 0.94 \\
52881 & 333058 & 2006 & 1871.20 & 3.60 & 175910.00 & 1801451.53 & 1.06 & 0.96 & 1.02 \\
49722 & 240337 & 2006 & 16.50 & 0.11 & 1360.00 & 12116.90 & 1.21 & 0.73 & 0.89 \\
37724 & 107135 & 2006 & 676.80 & 0.13 & 67781.00 & 671629.54 & 1.00 & 0.99 & 0.99 \\
58884 & 410230 & 2006 & 400.70 & 0.12 & 38277.00 & 320428.70 & 1.05 & 0.80 & 0.84 \\
42887 & 109030 & 2006 & 77.90 & 0.06 & 7279.00 & 72721.06 & 1.07 & 0.93 & 1.00 \\
64017 & 500577 & 2006 & 1016.50 & 0.20 & 102517.00 & 975099.36 & 0.99 & 0.96 & 0.95 \\
58902 & 410231 & 2006 & 34.20 & 0.08 & 2581.00 & 26789.34 & 1.33 & 0.78 & 1.04 \\
52633 & 305184 & 2006 & 50.30 & 0.14 & 6978.00 & 57073.13 & 0.72 & 1.13 & 0.82 \\
47545 & 212658 & 2006 & 15337.40 & 0.07 & 1531297.00 & 14815065.25 & 1.00 & 0.97 & 0.97 \\
54368 & 367500 & 2006 & 323.10 & 0.11 & 32325.00 & 317985.25 & 1.00 & 0.98 & 0.98 \\
49696 & 240333 & 2006 & 164.10 & 0.36 & 18286.00 & 170609.14 & 0.90 & 1.04 & 0.93 \\
74595 & 601139 & 2006 & 7461.60 & 0.17 & 736484.00 & 7287419.93 & 1.01 & 0.98 & 0.99 \\
44451 & 109325 & 2006 & 3181.80 & 0.18 & 317667.00 & 3185166.32 & 1.00 & 1.00 & 1.00 \\
34266 & 106216 & 2006 & 650.60 & 0.09 & 67013.00 & 640238.24 & 0.97 & 0.98 & 0.96 \\
54581 & 376139 & 2006 & 75.60 & 0.09 & 7950.00 & 78850.37 & 0.95 & 1.04 & 0.99 \\
42913 & 109033 & 2006 & 24.30 & 0.08 & 2481.00 & 24654.72 & 0.98 & 1.01 & 0.99 \\
52934 & 335811 & 2006 & 1295.50 & 0.21 & 135775.00 & 1318491.07 & 0.95 & 1.02 & 0.97 \\
56428 & 400208 & 2006 & 259.60 & 0.13 & 25955.00 & 215432.70 & 1.00 & 0.83 & 0.83 \\
58787 & 410217 & 2006 & 56.10 & 0.17 & 5168.00 & 49782.84 & 1.09 & 0.89 & 0.96 \\
51924 & 300653 & 2006 & 499.30 & 0.05 & 50086.00 & 491730.37 & 1.00 & 0.98 & 0.98 \\
54343 & 367206 & 2006 & 351.00 & 0.12 & 33126.00 & 317114.71 & 1.06 & 0.90 & 0.96 \\
62130 & 500340 & 2006 & 56.20 & 0.06 & 5609.00 & 50845.29 & 1.00 & 0.90 & 0.91 \\
52922 & 335108 & 2006 & 163.00 & 0.15 & 20194.00 & 165228.49 & 0.81 & 1.01 & 0.82 \\
41020 & 108180 & 2006 & 136.00 & 0.08 & 12894.00 & 131607.42 & 1.05 & 0.97 & 1.02 \\
55608 & 400129 & 2006 & 1048.10 & 0.13 & 100068.00 & 968534.21 & 1.05 & 0.92 & 0.97 \\
58805 & 410218 & 2006 & 18.60 & 0.07 & 2340.00 & 24267.45 & 0.79 & 1.30 & 1.04 \\
37806 & 107145 & 2006 & 359.90 & 0.30 & 57953.00 & 599776.50 & 0.62 & 1.67 & 1.03 \\
44478 & 109327 & 2006 & 1594.90 & 0.18 & 155450.00 & 1497930.53 & 1.03 & 0.94 & 0.96 \\
58809 & 410219 & 2006 & 26.70 & 0.04 & 2675.00 & 26183.63 & 1.00 & 0.98 & 0.98 \\
37783 & 107144 & 2006 & 52.60 & 0.08 & 5302.00 & 51665.05 & 0.99 & 0.98 & 0.97 \\
41731 & 108849 & 2006 & 1883.30 & 0.14 & 188580.00 & 1876673.24 & 1.00 & 1.00 & 1.00 \\
49729 & 240344 & 2006 & 140.00 & 0.10 & 16802.00 & 146526.26 & 0.83 & 1.05 & 0.87 \\
42871 & 109028 & 2006 & 362.20 & 0.09 & 37488.00 & 348617.63 & 0.97 & 0.96 & 0.93 \\
52656 & 305590 & 2006 & 359.30 & 0.15 & 35509.00 & 352843.13 & 1.01 & 0.98 & 0.99 \\
37665 & 106992 & 2006 & 176.10 & 0.20 & 17551.00 & 175776.07 & 1.00 & 1.00 & 1.00 \\
46832 & 200304 & 2006 & 102.90 & 0.09 & 8628.00 & 83102.96 & 1.19 & 0.81 & 0.96 \\
36598 & 106568 & 2006 & 8019.00 & 0.17 & 804511.00 & 7695947.34 & 1.00 & 0.96 & 0.96 \\
48218 & 240051 & 2006 & 770.30 & 0.12 & 77032.00 & 722414.95 & 1.00 & 0.94 & 0.94 \\
41089 & 108192 & 2006 & 37.00 & 0.17 & 6054.00 & 31230.04 & 0.61 & 0.84 & 0.52 \\
58915 & 410232 & 2006 & 21.20 & 0.13 & 1980.00 & 20297.11 & 1.07 & 0.96 & 1.03 \\
42567 & 108979 & 2006 & 97.80 & 0.18 & 9204.00 & 93090.16 & 1.06 & 0.95 & 1.01 \\
55451 & 400099 & 2006 & 243.70 & 0.14 & 22323.00 & 218170.40 & 1.09 & 0.90 & 0.98 \\
58918 & 410233 & 2006 & 55.40 & 0.18 & 5277.00 & 51342.57 & 1.05 & 0.93 & 0.97 \\
53589 & 354930 & 2006 & 5354.70 & 0.24 & 489808.00 & 4897979.73 & 1.09 & 0.91 & 1.00 \\
55603 & 400128 & 2006 & 76.60 & 0.12 & 7722.00 & 76403.00 & 0.99 & 1.00 & 0.99 \\
46826 & 200303 & 2006 & 15.90 & 0.13 & 1583.00 & 15834.03 & 1.00 & 1.00 & 1.00 \\
41098 & 108197 & 2006 & 66.40 & 0.16 & 7604.00 & 64997.45 & 0.87 & 0.98 & 0.85 \\
63413 & 500507 & 2006 & 109.40 & 0.12 & 10526.00 & 84998.36 & 1.04 & 0.78 & 0.81 \\
44051 & 109263 & 2006 & 10.90 & 0.04 & 1107.00 & 10101.35 & 0.98 & 0.93 & 0.91 \\
96681 & 611003 & 2006 & 353.90 & 0.13 & 35157.00 & 322540.91 & 1.01 & 0.91 & 0.92 \\
49749 & 240353 & 2006 & 1.40 & 0.11 & 142.00 & 1356.38 & 0.99 & 0.97 & 0.96 \\
61793 & 500131 & 2006 & 615.00 & 0.35 & 56476.00 & 561014.37 & 1.09 & 0.91 & 0.99 \\
51445 & 240520 & 2006 & 183.60 & 0.35 & 22078.00 & 230313.88 & 0.83 & 1.25 & 1.04 \\
63406 & 500506 & 2006 & 12.70 & 0.13 & 1270.00 & 11742.06 & 1.00 & 0.92 & 0.92 \\
37676 & 106993 & 2006 & 485.10 & 0.17 & 44931.00 & 441489.37 & 1.08 & 0.91 & 0.98 \\
41043 & 108183 & 2006 & 293.90 & 0.12 & 27409.00 & 287540.75 & 1.07 & 0.98 & 1.05 \\
57191 & 400305 & 2006 & 168.80 & 0.09 & 17074.00 & 164067.70 & 0.99 & 0.97 & 0.96 \\
50188 & 240401 & 2006 & 21.00 & 0.09 & 1766.00 & 17616.04 & 1.19 & 0.84 & 1.00 \\
51425 & 240519 & 2006 & 24.40 & 0.07 & 2509.00 & 25711.69 & 0.97 & 1.05 & 1.02 \\
49735 & 240345 & 2006 & 208.80 & 0.12 & 25142.00 & 221164.33 & 0.83 & 1.06 & 0.88 \\
63355 & 500500 & 2006 & 824.00 & 0.06 & 88350.00 & 796133.53 & 0.93 & 0.97 & 0.90 \\
34326 & 106222 & 2006 & 98.70 & -0.06 & 10970.00 & 106645.16 & 0.90 & 1.08 & 0.97 \\
49744 & 240352 & 2006 & 3.10 & 0.04 & 333.00 & 3216.50 & 0.93 & 1.04 & 0.97 \\
43913 & 109230 & 2006 & 887.40 & 0.09 & 83520.00 & 835173.42 & 1.06 & 0.94 & 1.00 \\
50165 & 240398 & 2006 & 1342.20 & 0.37 & 123310.00 & 1303465.30 & 1.09 & 0.97 & 1.06 \\
41068 & 108188 & 2006 & 3342.20 & 0.19 & 305320.00 & 2447008.99 & 1.09 & 0.73 & 0.80 \\
48599 & 240112 & 2006 & 12.50 & 0.02 & 1259.00 & 12499.36 & 0.99 & 1.00 & 0.99 \\
56482 & 400216 & 2006 & 230.50 & 0.17 & 21213.00 & 208975.82 & 1.09 & 0.91 & 0.99 \\
54390 & 367567 & 2006 & 500.50 & 0.19 & 50074.00 & 472472.11 & 1.00 & 0.94 & 0.94 \\
34336 & 106223 & 2006 & 108.80 & 0.16 & 10748.00 & 108437.86 & 1.01 & 1.00 & 1.01 \\
52646 & 305586 & 2006 & 110.40 & 0.06 & 12583.00 & 122155.01 & 0.88 & 1.11 & 0.97 \\
64007 & 500576 & 2006 & 233.10 & 0.13 & 20548.00 & 191906.32 & 1.13 & 0.82 & 0.93 \\
48147 & 240027 & 2006 & 420.70 & 0.32 & 48206.00 & 430073.36 & 0.87 & 1.02 & 0.89 \\
46854 & 200309 & 2006 & 133.70 & 0.16 & 13361.00 & 132939.14 & 1.00 & 0.99 & 0.99 \\
47818 & 222027 & 2006 & 1387.20 & 0.02 & 138303.00 & 1378822.88 & 1.00 & 0.99 & 1.00 \\
53412 & 348766 & 2006 & 1282.90 & 0.05 & 124367.00 & 1243667.96 & 1.03 & 0.97 & 1.00 \\
42845 & 109025 & 2006 & 1506.10 & 0.28 & 213388.00 & 1537177.74 & 0.71 & 1.02 & 0.72 \\
47843 & 222351 & 2006 & 688.60 & 0.20 & 73244.00 & 659940.22 & 0.94 & 0.96 & 0.90 \\
53549 & 351713 & 2006 & 465.40 & 0.10 & 47331.00 & 452477.21 & 0.98 & 0.97 & 0.96 \\
41649 & 108826 & 2006 & 335.50 & 0.10 & 35093.00 & 334173.36 & 0.96 & 1.00 & 0.95 \\
46674 & 200276 & 2006 & 49.30 & 0.00 & 4942.00 & 48121.41 & 1.00 & 0.98 & 0.97 \\
61912 & 500306 & 2006 & 8.00 & -0.02 & 800.00 & 7864.82 & 1.00 & 0.98 & 0.98 \\
37133 & 106692 & 2006 & 3625.50 & 0.15 & 361538.00 & 3463333.92 & 1.00 & 0.96 & 0.96 \\
52774 & 320640 & 2006 & 3.20 & 0.12 & 321.00 & 3133.88 & 1.00 & 0.98 & 0.98 \\
59306 & 410463 & 2006 & 636.20 & 0.17 & 56273.00 & 481563.25 & 1.13 & 0.76 & 0.86 \\
36910 & 106627 & 2006 & 5236.30 & 0.07 & 506581.00 & 4464651.05 & 1.03 & 0.85 & 0.88 \\
46713 & 200291 & 2006 & 6.40 & 0.06 & 632.00 & 6239.33 & 1.01 & 0.97 & 0.99 \\
41429 & 108752 & 2006 & 59.90 & 0.10 & 6392.00 & 62680.63 & 0.94 & 1.05 & 0.98 \\
59258 & 410448 & 2006 & 710.60 & 0.13 & 69741.00 & 697424.45 & 1.02 & 0.98 & 1.00 \\
52834 & 330728 & 2006 & 977.50 & 0.02 & 93337.00 & 938341.41 & 1.05 & 0.96 & 1.01 \\
55553 & 400120 & 2006 & 1469.20 & 0.13 & 144326.00 & 1400119.27 & 1.02 & 0.95 & 0.97 \\
48186 & 240040 & 2006 & 769.30 & 0.16 & 74663.00 & 764903.53 & 1.03 & 0.99 & 1.02 \\
53518 & 351459 & 2006 & 972.40 & 0.13 & 105431.00 & 851005.26 & 0.92 & 0.88 & 0.81 \\
55546 & 400118 & 2006 & 470.00 & 0.15 & 47010.00 & 470101.31 & 1.00 & 1.00 & 1.00 \\
54510 & 373584 & 2006 & 15.30 & 0.09 & 1917.00 & 16178.30 & 0.80 & 1.06 & 0.84 \\
44584 & 109343 & 2006 & 283.50 & 0.15 & 28645.00 & 266320.81 & 0.99 & 0.94 & 0.93 \\
63657 & 500541 & 2006 & 323.10 & 0.13 & 32061.00 & 295494.68 & 1.01 & 0.91 & 0.92 \\
41562 & 108766 & 2006 & 86.90 & 0.11 & 10508.00 & 107998.42 & 0.83 & 1.24 & 1.03 \\
34833 & 106281 & 2006 & 7.10 & 0.15 & 1138.00 & 9157.09 & 0.62 & 1.29 & 0.80 \\
53494 & 351048 & 2006 & 2000.70 & 0.16 & 156215.00 & 1561622.21 & 1.28 & 0.78 & 1.00 \\
44521 & 109334 & 2006 & 124.80 & 0.13 & 12478.00 & 120031.54 & 1.00 & 0.96 & 0.96 \\
37102 & 106678 & 2006 & 37.50 & 0.19 & 3772.00 & 37031.64 & 0.99 & 0.99 & 0.98 \\
41441 & 108759 & 2006 & 165.50 & 0.16 & 16520.00 & 161426.46 & 1.00 & 0.98 & 0.98 \\
44017 & 109258 & 2006 & 579.30 & 0.16 & 53462.00 & 560156.38 & 1.08 & 0.97 & 1.05 \\
52812 & 330079 & 2006 & 50.00 & 0.16 & 5010.00 & 49325.64 & 1.00 & 0.99 & 0.98 \\
34844 & 106282 & 2006 & 863.50 & 0.09 & 86318.00 & 816828.71 & 1.00 & 0.95 & 0.95 \\
37209 & 106708 & 2006 & 494.30 & 0.14 & 51042.00 & 477502.77 & 0.97 & 0.97 & 0.94 \\
41570 & 108773 & 2006 & 240.10 & 0.20 & 24533.00 & 233522.03 & 0.98 & 0.97 & 0.95 \\
37183 & 106707 & 2006 & 341.80 & 0.09 & 41695.00 & 329901.38 & 0.82 & 0.97 & 0.79 \\
42643 & 108990 & 2006 & 16.70 & 0.17 & 1789.00 & 15765.98 & 0.93 & 0.94 & 0.88 \\
41377 & 108736 & 2006 & 54.60 & 0.07 & 6966.00 & 63838.28 & 0.78 & 1.17 & 0.92 \\
46667 & 200274 & 2006 & 25.00 & 0.07 & 2589.00 & 23950.96 & 0.97 & 0.96 & 0.93 \\
34626 & 106261 & 2006 & 139.70 & 0.08 & 14069.00 & 136436.17 & 0.99 & 0.98 & 0.97 \\
55508 & 400114 & 2006 & 640.20 & 0.16 & 64146.00 & 593025.23 & 1.00 & 0.93 & 0.92 \\
51689 & 240541 & 2006 & 65.10 & 0.10 & 7300.00 & 63503.75 & 0.89 & 0.98 & 0.87 \\
49931 & 240376 & 2006 & 84.60 & 0.15 & 8684.00 & 85885.23 & 0.97 & 1.02 & 0.99 \\
42720 & 109009 & 2006 & 436.40 & 0.15 & 45939.00 & 407348.63 & 0.95 & 0.93 & 0.89 \\
37169 & 106706 & 2006 & 167.30 & 0.13 & 20716.00 & 163189.42 & 0.81 & 0.98 & 0.79 \\
34637 & 106262 & 2006 & 149.20 & 0.06 & 14844.00 & 147563.60 & 1.01 & 0.99 & 0.99 \\
44510 & 109333 & 2006 & 194.30 & 0.13 & 19133.00 & 192042.49 & 1.02 & 0.99 & 1.00 \\
49954 & 240377 & 2006 & 37.00 & 0.15 & 3495.00 & 36611.45 & 1.06 & 0.99 & 1.05 \\
63736 & 500550 & 2006 & 63997.10 & 0.16 & 5753968.00 & 59767695.60 & 1.11 & 0.93 & 1.04 \\
36884 & 106620 & 2006 & 222.80 & 0.05 & 23922.00 & 193480.21 & 0.93 & 0.87 & 0.81 \\
46075 & 200183 & 2006 & 7.30 & 0.08 & 666.00 & 6661.66 & 1.10 & 0.91 & 1.00 \\
51683 & 240540 & 2006 & 61.00 & 0.08 & 5792.00 & 58889.60 & 1.05 & 0.97 & 1.02 \\
52748 & 320323 & 2006 & 7.50 & 0.09 & 702.00 & 6941.58 & 1.07 & 0.93 & 0.99 \\
41395 & 108745 & 2006 & 92.00 & 0.36 & 8727.00 & 71802.51 & 1.05 & 0.78 & 0.82 \\
52753 & 320584 & 2006 & 10.10 & 0.10 & 968.00 & 9645.81 & 1.04 & 0.96 & 1.00 \\
46171 & 200196 & 2006 & 72.20 & 0.29 & 7371.00 & 73616.61 & 0.98 & 1.02 & 1.00 \\
41465 & 108760 & 2006 & 211.20 & 0.15 & 16701.00 & 171299.32 & 1.26 & 0.81 & 1.03 \\
63615 & 500531 & 2006 & 30.60 & 0.16 & 2661.00 & 26581.82 & 1.15 & 0.87 & 1.00 \\
34715 & 106272 & 2006 & 2680.90 & 0.15 & 301431.00 & 2747723.92 & 0.89 & 1.02 & 0.91 \\
42659 & 108992 & 2006 & 12.10 & 0.15 & 2346.00 & 24278.53 & 0.52 & 2.01 & 1.03 \\
52781 & 322114 & 2006 & 18.40 & 0.18 & 1788.00 & 17010.13 & 1.03 & 0.92 & 0.95 \\
49984 & 240381 & 2006 & 116.00 & 0.38 & 11465.00 & 110968.17 & 1.01 & 0.96 & 0.97 \\
36958 & 106643 & 2006 & 253.30 & 0.18 & 19328.00 & 196691.55 & 1.31 & 0.78 & 1.02 \\
51620 & 240535 & 2006 & 4.10 & 0.04 & 371.00 & 3709.78 & 1.11 & 0.90 & 1.00 \\
34744 & 106275 & 2006 & 115.30 & 0.19 & 11403.00 & 112640.03 & 1.01 & 0.98 & 0.99 \\
37040 & 106654 & 2006 & 510.70 & 0.15 & 48438.00 & 502733.09 & 1.05 & 0.98 & 1.04 \\
34798 & 106277 & 2006 & 523.50 & 0.01 & 52083.00 & 516273.62 & 1.01 & 0.99 & 0.99 \\
41490 & 108761 & 2006 & 343.90 & 0.16 & 33449.00 & 332579.81 & 1.03 & 0.97 & 0.99 \\
41512 & 108762 & 2006 & 115.20 & 0.16 & 12602.00 & 115745.40 & 0.91 & 1.00 & 0.92 \\
54488 & 372487 & 2006 & 34.90 & 0.05 & 3415.00 & 34755.18 & 1.02 & 1.00 & 1.02 \\
42698 & 108994 & 2006 & 60.60 & 0.15 & 7665.00 & 77748.84 & 0.79 & 1.28 & 1.01 \\
50006 & 240382 & 2006 & 44.00 & 0.20 & 4280.00 & 41716.77 & 1.03 & 0.95 & 0.97 \\
50017 & 240383 & 2006 & 214.20 & 0.20 & 20057.00 & 202296.31 & 1.07 & 0.94 & 1.01 \\
46109 & 200190 & 2006 & 268.80 & 0.10 & 26811.00 & 249195.07 & 1.00 & 0.93 & 0.93 \\
37021 & 106650 & 2006 & 694.40 & 0.13 & 83657.00 & 763656.23 & 0.83 & 1.10 & 0.91 \\
46684 & 200277 & 2006 & 113.40 & 0.09 & 13892.00 & 136964.55 & 0.82 & 1.21 & 0.99 \\
42683 & 108993 & 2006 & 242.30 & 0.17 & 40671.00 & 344705.14 & 0.60 & 1.42 & 0.85 \\
36984 & 106644 & 2006 & 1.50 & 0.03 & 153.00 & 1384.41 & 0.98 & 0.92 & 0.90 \\
63547 & 500521 & 2006 & 62.20 & 0.20 & 4882.00 & 46855.41 & 1.27 & 0.75 & 0.96 \\
51624 & 240536 & 2006 & 324.20 & 0.05 & 30001.00 & 315198.79 & 1.08 & 0.97 & 1.05 \\
55541 & 400117 & 2006 & 469.80 & 0.13 & 45176.00 & 450033.12 & 1.04 & 0.96 & 1.00 \\
48909 & 240152 & 2006 & 196.60 & 0.13 & 19677.00 & 180054.10 & 1.00 & 0.92 & 0.92 \\
34821 & 106278 & 2006 & 116.60 & 0.15 & 13752.00 & 138007.70 & 0.85 & 1.18 & 1.00 \\
48262 & 240057 & 2006 & 141.10 & 0.08 & 14159.00 & 125354.22 & 1.00 & 0.89 & 0.89 \\
46098 & 200184 & 2006 & 4.20 & 0.10 & 352.00 & 3515.87 & 1.19 & 0.84 & 1.00 \\
46156 & 200194 & 2006 & 34.70 & 0.14 & 3196.00 & 31964.22 & 1.09 & 0.92 & 1.00 \\
41537 & 108764 & 2006 & 194.60 & 0.15 & 20516.00 & 195205.01 & 0.95 & 1.00 & 0.95 \\
34682 & 106270 & 2006 & 22.40 & 0.10 & 2028.00 & 21527.96 & 1.10 & 0.96 & 1.06 \\
54483 & 372363 & 2006 & 12.50 & 0.13 & 1274.00 & 12912.26 & 0.98 & 1.03 & 1.01 \\
36940 & 106642 & 2006 & 2172.90 & 0.19 & 207923.00 & 2079246.14 & 1.05 & 0.96 & 1.00 \\
37092 & 106675 & 2006 & 133.80 & 0.10 & 14006.00 & 138188.87 & 0.96 & 1.03 & 0.99 \\
46136 & 200193 & 2006 & 188.80 & 0.15 & 18778.00 & 188050.98 & 1.01 & 1.00 & 1.00 \\
46698 & 200279 & 2006 & 22.60 & 0.05 & 2274.00 & 22414.36 & 0.99 & 0.99 & 0.99 \\
34871 & 106283 & 2006 & 434.50 & 0.06 & 41209.00 & 412087.42 & 1.05 & 0.95 & 1.00 \\
51659 & 240538 & 2006 & 107.30 & 0.29 & 9569.00 & 78237.56 & 1.12 & 0.73 & 0.82 \\
47751 & 221051 & 2006 & 4707.30 & 0.12 & 476911.00 & 4546401.57 & 0.99 & 0.97 & 0.95 \\
51645 & 240537 & 2006 & 45.80 & 0.15 & 3881.00 & 31775.26 & 1.18 & 0.69 & 0.82 \\
54820 & 400015 & 2006 & 11.70 & 0.09 & 1120.00 & 10348.83 & 1.04 & 0.88 & 0.92 \\
54794 & 400014 & 2006 & 100.20 & 0.00 & 10324.00 & 93091.71 & 0.97 & 0.93 & 0.90 \\
46130 & 200192 & 2006 & 3.80 & 0.16 & 376.00 & 3697.34 & 1.01 & 0.97 & 0.98 \\
61940 & 500310 & 2006 & 11.20 & 0.16 & 1231.00 & 12256.67 & 0.91 & 1.09 & 1.00 \\
51609 & 240534 & 2006 & 588.40 & 0.05 & 57223.00 & 598687.85 & 1.03 & 1.02 & 1.05 \\
56423 & 400207 & 2006 & 7.20 & 0.02 & 716.00 & 6713.07 & 1.01 & 0.93 & 0.94 \\
47773 & 221210 & 2006 & 141.40 & 0.08 & 14558.00 & 140975.94 & 0.97 & 1.00 & 0.97 \\
46191 & 200197 & 2006 & 344.60 & 0.17 & 34091.00 & 344852.98 & 1.01 & 1.00 & 1.01 \\
55563 & 400125 & 2006 & 142.90 & 0.10 & 14262.00 & 141849.34 & 1.00 & 0.99 & 0.99 \\
51727 & 240546 & 2006 & 48.10 & 0.06 & 4584.00 & 45684.76 & 1.05 & 0.95 & 1.00 \\
49828 & 240365 & 2006 & 40.80 & 0.00 & 3964.00 & 38430.67 & 1.03 & 0.94 & 0.97 \\
45946 & 200171 & 2006 & 121.10 & 0.18 & 12508.00 & 117368.05 & 0.97 & 0.97 & 0.94 \\
43976 & 109249 & 2006 & 123.30 & 0.04 & 12337.00 & 114169.57 & 1.00 & 0.93 & 0.93 \\
36798 & 106595 & 2006 & 25.80 & 0.12 & 2582.00 & 21202.57 & 1.00 & 0.82 & 0.82 \\
55181 & 400066 & 2006 & 171.00 & 0.18 & 17147.00 & 171456.89 & 1.00 & 1.00 & 1.00 \\
63870 & 500562 & 2006 & 55.00 & 0.19 & 5483.00 & 54132.76 & 1.00 & 0.98 & 0.99 \\
45952 & 200172 & 2006 & 15.80 & 0.17 & 1792.00 & 18326.02 & 0.88 & 1.16 & 1.02 \\
42768 & 109016 & 2006 & 2666.20 & 0.19 & 266893.00 & 2507331.60 & 1.00 & 0.94 & 0.94 \\
41626 & 108782 & 2006 & 639.40 & 0.11 & 66797.00 & 696722.84 & 0.96 & 1.09 & 1.04 \\
42628 & 108987 & 2006 & 58.20 & 0.15 & 5693.00 & 60206.93 & 1.02 & 1.03 & 1.06 \\
42753 & 109015 & 2006 & 217.50 & 0.16 & 22145.00 & 212276.73 & 0.98 & 0.98 & 0.96 \\
46746 & 200294 & 2006 & 352.20 & 0.21 & 35187.00 & 343764.70 & 1.00 & 0.98 & 0.98 \\
63480 & 500511 & 2006 & 1189.00 & 0.14 & 176909.00 & 1570165.09 & 0.67 & 1.32 & 0.89 \\
51558 & 240529 & 2006 & 338.50 & 0.08 & 35623.00 & 367957.50 & 0.95 & 1.09 & 1.03 \\
52720 & 307603 & 2006 & 209.70 & 0.20 & 20830.00 & 178050.11 & 1.01 & 0.85 & 0.85 \\
45958 & 200173 & 2006 & 29.60 & 0.20 & 2986.00 & 30932.46 & 0.99 & 1.05 & 1.04 \\
54531 & 373714 & 2006 & 1513.60 & 0.19 & 154612.00 & 1217111.72 & 0.98 & 0.80 & 0.79 \\
59247 & 410447 & 2006 & 249.20 & 0.18 & 25098.00 & 242812.85 & 0.99 & 0.97 & 0.97 \\
45964 & 200174 & 2006 & 812.70 & 0.20 & 81233.00 & 778358.71 & 1.00 & 0.96 & 0.96 \\
37319 & 106729 & 2006 & 1946.10 & 0.17 & 195441.00 & 1914681.30 & 1.00 & 0.98 & 0.98 \\
41318 & 108726 & 2006 & 262.50 & 0.07 & 30599.00 & 259671.66 & 0.86 & 0.99 & 0.85 \\
52733 & 307849 & 2006 & 111.80 & 0.16 & 10585.00 & 92180.53 & 1.06 & 0.82 & 0.87 \\
42603 & 108985 & 2006 & 5372.10 & 0.14 & 721872.00 & 6037263.23 & 0.74 & 1.12 & 0.84 \\
37393 & 106747 & 2006 & 154.80 & 0.16 & 15525.00 & 146996.14 & 1.00 & 0.95 & 0.95 \\
45915 & 200164 & 2006 & 26.50 & 0.07 & 2659.00 & 25107.24 & 1.00 & 0.95 & 0.94 \\
44622 & 109348 & 2006 & 1161.20 & 0.21 & 117566.00 & 1124066.85 & 0.99 & 0.97 & 0.96 \\
56487 & 400217 & 2006 & 13.80 & 0.15 & 1355.00 & 13545.69 & 1.02 & 0.98 & 1.00 \\
45920 & 200167 & 2006 & 180.60 & 0.15 & 18617.00 & 175866.88 & 0.97 & 0.97 & 0.94 \\
41283 & 108719 & 2006 & 288.40 & 0.07 & 37350.00 & 287338.86 & 0.77 & 1.00 & 0.77 \\
34966 & 106297 & 2006 & 225.40 & 0.32 & 18990.00 & 200684.70 & 1.19 & 0.89 & 1.06 \\
52693 & 306690 & 2006 & 389.20 & 0.13 & 37454.00 & 386993.67 & 1.04 & 0.99 & 1.03 \\
44033 & 109259 & 2006 & 270.20 & 0.17 & 34247.00 & 252179.73 & 0.79 & 0.93 & 0.74 \\
44599 & 109347 & 2006 & 1187.10 & 0.14 & 119024.00 & 1168655.80 & 1.00 & 0.98 & 0.98 \\
37419 & 106869 & 2006 & 107.70 & 0.09 & 10568.00 & 104020.33 & 1.02 & 0.97 & 0.98 \\
36778 & 106590 & 2006 & 167.20 & 0.17 & 17132.00 & 161447.46 & 0.98 & 0.97 & 0.94 \\
52858 & 330794 & 2006 & 121.90 & 0.10 & 11867.00 & 121486.27 & 1.03 & 1.00 & 1.02 \\
41308 & 108723 & 2006 & 220.50 & 0.15 & 22592.00 & 213596.79 & 0.98 & 0.97 & 0.95 \\
34939 & 106294 & 2006 & 398.20 & 0.36 & 39884.00 & 386145.50 & 1.00 & 0.97 & 0.97 \\
47720 & 220770 & 2006 & 3821.00 & 0.16 & 553582.00 & 4845040.90 & 0.69 & 1.27 & 0.88 \\
52715 & 307384 & 2006 & 67.30 & 0.08 & 10814.00 & 108139.01 & 0.62 & 1.61 & 1.00 \\
45925 & 200168 & 2006 & 486.10 & 0.11 & 48587.00 & 477784.87 & 1.00 & 0.98 & 0.98 \\
47572 & 212809 & 2006 & 33.20 & 0.22 & 2981.00 & 28520.78 & 1.11 & 0.86 & 0.96 \\
37382 & 106742 & 2006 & 129.10 & 0.12 & 12839.00 & 129010.25 & 1.01 & 1.00 & 1.00 \\
63860 & 500561 & 2006 & 46.70 & 0.03 & 4693.00 & 46706.31 & 1.00 & 1.00 & 1.00 \\
46723 & 200293 & 2006 & 133.80 & 0.12 & 11574.00 & 94014.83 & 1.16 & 0.70 & 0.81 \\
51564 & 240530 & 2006 & 57.20 & 0.13 & 5484.00 & 52987.83 & 1.04 & 0.93 & 0.97 \\
34562 & 106255 & 2006 & 1349.70 & 0.07 & 139191.00 & 1338557.71 & 0.97 & 0.99 & 0.96 \\
46026 & 200177 & 2006 & 40.10 & 0.10 & 3773.00 & 37819.09 & 1.06 & 0.94 & 1.00 \\
51723 & 240545 & 2006 & 71.50 & 0.08 & 7158.00 & 67160.00 & 1.00 & 0.94 & 0.94 \\
56530 & 400222 & 2006 & 67.70 & 0.04 & 6589.00 & 67427.27 & 1.03 & 1.00 & 1.02 \\
59039 & 410401 & 2006 & 381.70 & 0.07 & 39140.00 & 369104.69 & 0.98 & 0.97 & 0.94 \\
37222 & 106710 & 2006 & 42.00 & 0.16 & 4276.00 & 44714.71 & 0.98 & 1.06 & 1.05 \\
46201 & 200198 & 2006 & 104.60 & 0.20 & 10314.00 & 103771.76 & 1.01 & 0.99 & 1.01 \\
54830 & 400017 & 2006 & 107.40 & 0.07 & 12649.00 & 119371.56 & 0.85 & 1.11 & 0.94 \\
34891 & 106284 & 2006 & 399.00 & 0.05 & 40737.00 & 405943.68 & 0.98 & 1.02 & 1.00 \\
46032 & 200178 & 2006 & 38.70 & 0.18 & 3575.00 & 36532.69 & 1.08 & 0.94 & 1.02 \\
51693 & 240542 & 2006 & 246.00 & 0.06 & 24680.00 & 227132.69 & 1.00 & 0.92 & 0.92 \\
56550 & 400224 & 2006 & 51.20 & 0.10 & 4640.00 & 48841.33 & 1.10 & 0.95 & 1.05 \\
63781 & 500554 & 2006 & 578.30 & 0.14 & 53125.00 & 535417.68 & 1.09 & 0.93 & 1.01 \\
46038 & 200179 & 2006 & 224.20 & 0.11 & 22154.00 & 216177.64 & 1.01 & 0.96 & 0.98 \\
42637 & 108988 & 2006 & 125.70 & 0.12 & 20199.00 & 200535.83 & 0.62 & 1.60 & 0.99 \\
34595 & 106257 & 2006 & 270.30 & 0.09 & 26918.00 & 254511.59 & 1.00 & 0.94 & 0.95 \\
41577 & 108776 & 2006 & 693.30 & 0.11 & 66367.00 & 669220.52 & 1.04 & 0.97 & 1.01 \\
56555 & 400225 & 2006 & 139.60 & 0.20 & 8046.00 & 83053.01 & 1.74 & 0.59 & 1.03 \\
36857 & 106605 & 2006 & 362.70 & 0.09 & 35047.00 & 335036.26 & 1.03 & 0.92 & 0.96 \\
53450 & 350408 & 2006 & 47.40 & 0.05 & 4761.00 & 49588.90 & 1.00 & 1.05 & 1.04 \\
41602 & 108777 & 2006 & 756.50 & 0.29 & 57868.00 & 555658.90 & 1.31 & 0.73 & 0.96 \\
54465 & 368366 & 2006 & 101.80 & 0.15 & 10189.00 & 98881.52 & 1.00 & 0.97 & 0.97 \\
61965 & 500312 & 2006 & 19.50 & 0.33 & 2531.00 & 24663.11 & 0.77 & 1.26 & 0.97 \\
46655 & 200273 & 2006 & 2428.00 & 0.16 & 202781.00 & 2035092.84 & 1.20 & 0.84 & 1.00 \\
34906 & 106292 & 2006 & 276.30 & 0.10 & 27751.00 & 276913.98 & 1.00 & 1.00 & 1.00 \\
37292 & 106726 & 2006 & 1231.60 & 0.15 & 122099.00 & 1218118.41 & 1.01 & 0.99 & 1.00 \\
63849 & 500560 & 2006 & 44.10 & 0.10 & 4426.00 & 43852.59 & 1.00 & 0.99 & 0.99 \\
43983 & 109250 & 2006 & 113.30 & -0.02 & 12688.00 & 115852.32 & 0.89 & 1.02 & 0.91 \\
54449 & 367985 & 2006 & 207.10 & 0.17 & 20676.00 & 211247.34 & 1.00 & 1.02 & 1.02 \\
49860 & 240368 & 2006 & 205.80 & 0.11 & 20583.00 & 194596.73 & 1.00 & 0.95 & 0.95 \\
63830 & 500559 & 2006 & 80.80 & 0.08 & 7908.00 & 77447.96 & 1.02 & 0.96 & 0.98 \\
51584 & 240531 & 2006 & 191.80 & 0.18 & 18505.00 & 178868.58 & 1.04 & 0.93 & 0.97 \\
49870 & 240369 & 2006 & 8.00 & 0.02 & 974.00 & 9739.92 & 0.82 & 1.22 & 1.00 \\
54458 & 367992 & 2006 & 86.00 & 0.19 & 9929.00 & 97712.41 & 0.87 & 1.14 & 0.98 \\
63809 & 500556 & 2006 & 279.40 & 0.15 & 25393.00 & 268468.44 & 1.10 & 0.96 & 1.06 \\
63776 & 500553 & 2006 & 141.60 & 0.07 & 14034.00 & 140537.01 & 1.01 & 0.99 & 1.00 \\
56579 & 400228 & 2006 & 217.70 & 0.23 & 20291.00 & 186528.90 & 1.07 & 0.86 & 0.92 \\
41612 & 108780 & 2006 & 70.50 & 0.12 & 6878.00 & 69781.63 & 1.03 & 0.99 & 1.01 \\
50069 & 240391 & 2006 & 614.60 & 0.09 & 60286.00 & 602273.01 & 1.02 & 0.98 & 1.00 \\
43992 & 109255 & 2006 & 548.10 & 0.18 & 52691.00 & 519587.30 & 1.04 & 0.95 & 0.99 \\
41354 & 108732 & 2006 & 137.40 & 0.16 & 13290.00 & 111378.58 & 1.03 & 0.81 & 0.84 \\
45985 & 200175 & 2006 & 417.80 & 0.22 & 41625.00 & 358353.63 & 1.00 & 0.86 & 0.86 \\
56508 & 400221 & 2006 & 92.70 & 0.10 & 8435.00 & 85982.21 & 1.10 & 0.93 & 1.02 \\
51597 & 240533 & 2006 & 13.80 & 0.12 & 1388.00 & 13944.08 & 0.99 & 1.01 & 1.00 \\
37254 & 106724 & 2006 & 2179.00 & 0.15 & 218651.00 & 2088498.26 & 1.00 & 0.96 & 0.96 \\
36837 & 106604 & 2006 & 105.90 & 0.08 & 10707.00 & 111596.54 & 0.99 & 1.05 & 1.04 \\
46005 & 200176 & 2006 & 208.80 & 0.12 & 20474.00 & 204743.84 & 1.02 & 0.98 & 1.00 \\
41329 & 108728 & 2006 & 538.60 & 0.23 & 52119.00 & 517481.81 & 1.03 & 0.96 & 0.99 \\
46385 & 200228 & 2006 & 3.60 & 0.11 & 346.00 & 3115.74 & 1.04 & 0.87 & 0.90 \\
42937 & 109037 & 2006 & 22.60 & 0.04 & 2291.00 & 22653.08 & 0.99 & 1.00 & 0.99 \\
63296 & 500493 & 2006 & 2018.10 & 0.12 & 181813.00 & 1909886.07 & 1.11 & 0.95 & 1.05 \\
72113 & 501665 & 2006 & 418.20 & 0.26 & 42057.00 & 419748.97 & 0.99 & 1.00 & 1.00 \\
36304 & 106482 & 2006 & 165.60 & 0.34 & 17171.00 & 162558.41 & 0.96 & 0.98 & 0.95 \\
43784 & 109219 & 2006 & 71.80 & 0.05 & 7055.00 & 71452.52 & 1.02 & 1.00 & 1.01 \\
53338 & 344277 & 2006 & 645.60 & 0.16 & 69565.00 & 625124.51 & 0.93 & 0.97 & 0.90 \\
59218 & 410445 & 2006 & 145.40 & 0.16 & 14516.00 & 145416.97 & 1.00 & 1.00 & 1.00 \\
72115 & 501666 & 2006 & 16.90 & 0.16 & 1689.00 & 16893.26 & 1.00 & 1.00 & 1.00 \\
72133 & 501667 & 2006 & 37.00 & 0.12 & 3750.00 & 37234.85 & 0.99 & 1.01 & 0.99 \\
49537 & 240311 & 2006 & 17.80 & 0.06 & 1809.00 & 17072.80 & 0.98 & 0.96 & 0.94 \\
47926 & 222809 & 2006 & 653.80 & 0.21 & 53211.00 & 557194.65 & 1.23 & 0.85 & 1.05 \\
35226 & 106335 & 2006 & 402.40 & 0.06 & 40121.00 & 414163.07 & 1.00 & 1.03 & 1.03 \\
33864 & 106176 & 2006 & 15.30 & -0.03 & 1834.00 & 19008.71 & 0.83 & 1.24 & 1.04 \\
49560 & 240312 & 2006 & 361.80 & 0.16 & 36290.00 & 358128.18 & 1.00 & 0.99 & 0.99 \\
48530 & 240103 & 2006 & 369.20 & 0.12 & 37620.00 & 356349.03 & 0.98 & 0.97 & 0.95 \\
41895 & 108868 & 2006 & 774.20 & 0.17 & 77456.00 & 755933.21 & 1.00 & 0.98 & 0.98 \\
40747 & 108148 & 2006 & 66.90 & 0.04 & 6930.00 & 68542.73 & 0.97 & 1.02 & 0.99 \\
36340 & 106485 & 2006 & 117.10 & 0.04 & 11735.00 & 116721.30 & 1.00 & 1.00 & 0.99 \\
51315 & 240499 & 2006 & 107.50 & 0.14 & 14855.00 & 112299.65 & 0.72 & 1.04 & 0.76 \\
61562 & 500094 & 2006 & 232.30 & 0.35 & 31133.00 & 231771.15 & 0.75 & 1.00 & 0.74 \\
50450 & 240419 & 2006 & 16.60 & -0.01 & 1632.00 & 16049.19 & 1.02 & 0.97 & 0.98 \\
73369 & 600006 & 2006 & 298.80 & 0.05 & 30174.00 & 286317.28 & 0.99 & 0.96 & 0.95 \\
47029 & 200329 & 2006 & 2042.00 & 0.16 & 186805.00 & 1913282.94 & 1.09 & 0.94 & 1.02 \\
33837 & 106173 & 2006 & 1547.30 & 0.15 & 154645.00 & 1529841.64 & 1.00 & 0.99 & 0.99 \\
40772 & 108149 & 2006 & 138.00 & 0.09 & 14028.00 & 139244.52 & 0.98 & 1.01 & 0.99 \\
72095 & 501664 & 2006 & 56.30 & 0.15 & 5646.00 & 56160.04 & 1.00 & 1.00 & 0.99 \\
72064 & 501662 & 2006 & 4047.10 & 0.15 & 455211.00 & 4019455.38 & 0.89 & 0.99 & 0.88 \\
58025 & 410034 & 2006 & 52.10 & 0.13 & 5539.00 & 49569.43 & 0.94 & 0.95 & 0.89 \\
33797 & 106170 & 2006 & 386.10 & 0.11 & 40324.00 & 402687.98 & 0.96 & 1.04 & 1.00 \\
52985 & 336508 & 2006 & 130.80 & 0.13 & 15668.00 & 131777.44 & 0.83 & 1.01 & 0.84 \\
56294 & 400187 & 2006 & 144.50 & 0.19 & 14444.00 & 144427.61 & 1.00 & 1.00 & 1.00 \\
49512 & 240308 & 2006 & 27.40 & -0.06 & 3080.00 & 29420.89 & 0.89 & 1.07 & 0.96 \\
38429 & 107258 & 2006 & 70.50 & 0.01 & 8544.00 & 77339.46 & 0.83 & 1.10 & 0.91 \\
40718 & 108146 & 2006 & 27.40 & 0.06 & 2851.00 & 28322.03 & 0.96 & 1.03 & 0.99 \\
58017 & 410018 & 2006 & 15.20 & 0.11 & 1510.00 & 14668.45 & 1.01 & 0.97 & 0.97 \\
44069 & 109264 & 2006 & 29.20 & 0.02 & 3056.00 & 26919.88 & 0.96 & 0.92 & 0.88 \\
38404 & 107257 & 2006 & 170.30 & 0.16 & 16628.00 & 172748.06 & 1.02 & 1.01 & 1.04 \\
61496 & 500083 & 2006 & 21.90 & 0.33 & 2027.00 & 21114.63 & 1.08 & 0.96 & 1.04 \\
43769 & 109218 & 2006 & 103.40 & 0.15 & 9565.00 & 97658.69 & 1.08 & 0.94 & 1.02 \\
56314 & 400188 & 2006 & 453.30 & 0.07 & 45324.00 & 453104.30 & 1.00 & 1.00 & 1.00 \\
56332 & 400190 & 2006 & 33.50 & 0.05 & 2826.00 & 28548.89 & 1.19 & 0.85 & 1.01 \\
52095 & 301560 & 2006 & 1127.60 & 0.11 & 121340.00 & 1194702.77 & 0.93 & 1.06 & 0.98 \\
41925 & 108870 & 2006 & 51.10 & 0.09 & 4685.00 & 48028.88 & 1.09 & 0.94 & 1.03 \\
58625 & 410180 & 2006 & 231.30 & 0.36 & 17733.00 & 178737.06 & 1.30 & 0.77 & 1.01 \\
33810 & 106172 & 2006 & 64.30 & 0.12 & 6431.00 & 63090.94 & 1.00 & 0.98 & 0.98 \\
43089 & 109062 & 2006 & 51.20 & 0.09 & 5121.00 & 49807.15 & 1.00 & 0.97 & 0.97 \\
66340 & 500815 & 2006 & 244.10 & 0.02 & 24421.00 & 244214.31 & 1.00 & 1.00 & 1.00 \\
66422 & 500820 & 2006 & 78.20 & 0.11 & 7858.00 & 78395.07 & 1.00 & 1.00 & 1.00 \\
66440 & 500821 & 2006 & 87.70 & 0.05 & 8791.00 & 87913.19 & 1.00 & 1.00 & 1.00 \\
66458 & 500822 & 2006 & 71.20 & 0.11 & 7119.00 & 71185.07 & 1.00 & 1.00 & 1.00 \\
72082 & 501663 & 2006 & 175.10 & 0.18 & 17532.00 & 175267.93 & 1.00 & 1.00 & 1.00 \\
50455 & 240421 & 2006 & 1220.00 & 0.13 & 122135.00 & 1091262.59 & 1.00 & 0.89 & 0.89 \\
43791 & 109221 & 2006 & 1210.20 & 0.14 & 121184.00 & 1117009.29 & 1.00 & 0.92 & 0.92 \\
50439 & 240417 & 2006 & 30.70 & 0.13 & 5098.00 & 52336.97 & 0.60 & 1.70 & 1.03 \\
35191 & 106333 & 2006 & 159.50 & 0.15 & 17221.00 & 169817.13 & 0.93 & 1.06 & 0.99 \\
64217 & 500592 & 2006 & 4530.50 & 0.21 & 450647.00 & 4517558.18 & 1.01 & 1.00 & 1.00 \\
57577 & 400440 & 2006 & 519.30 & 0.14 & 46123.00 & 445399.67 & 1.13 & 0.86 & 0.97 \\
46979 & 200324 & 2006 & 300.80 & 0.20 & 30566.00 & 285351.69 & 0.98 & 0.95 & 0.93 \\
54623 & 377074 & 2006 & 811.30 & 0.13 & 76196.00 & 755963.79 & 1.06 & 0.93 & 0.99 \\
38278 & 107234 & 2006 & 2.10 & 0.09 & 191.00 & 1910.14 & 1.10 & 0.91 & 1.00 \\
56735 & 400245 & 2006 & 217.20 & 0.10 & 21685.00 & 216847.14 & 1.00 & 1.00 & 1.00 \\
57973 & 410010 & 2006 & 4201.80 & 0.17 & 373544.00 & 3637262.00 & 1.12 & 0.87 & 0.97 \\
36370 & 106519 & 2006 & 827.30 & 0.07 & 82162.00 & 821621.06 & 1.01 & 0.99 & 1.00 \\
44728 & 109368 & 2006 & 1087.80 & 0.14 & 119028.00 & 1114309.02 & 0.91 & 1.02 & 0.94 \\
49580 & 240318 & 2006 & 103.00 & 0.05 & 12054.00 & 96097.36 & 0.85 & 0.93 & 0.80 \\
51347 & 240504 & 2006 & 59.40 & 0.09 & 5260.00 & 46795.25 & 1.13 & 0.79 & 0.89 \\
56717 & 400244 & 2006 & 4.30 & 0.03 & 409.00 & 4344.62 & 1.05 & 1.01 & 1.06 \\
40847 & 108158 & 2006 & 24.80 & 0.04 & 2405.00 & 24854.23 & 1.03 & 1.00 & 1.03 \\
59134 & 410435 & 2006 & 8.00 & 0.22 & 763.00 & 7544.55 & 1.05 & 0.94 & 0.99 \\
43058 & 109058 & 2006 & 113.50 & 0.20 & 11407.00 & 110425.61 & 1.00 & 0.97 & 0.97 \\
38247 & 107226 & 2006 & 1124.10 & 0.19 & 99416.00 & 797898.30 & 1.13 & 0.71 & 0.80 \\
54230 & 364947 & 2006 & 28.30 & 0.10 & 2736.00 & 28707.73 & 1.03 & 1.01 & 1.05 \\
38237 & 107224 & 2006 & 49.80 & 0.24 & 4575.00 & 48237.94 & 1.09 & 0.97 & 1.05 \\
56713 & 400242 & 2006 & 89.60 & 0.19 & 8505.00 & 89271.81 & 1.05 & 1.00 & 1.05 \\
44704 & 109366 & 2006 & 66.10 & 0.20 & 6655.00 & 66232.51 & 0.99 & 1.00 & 1.00 \\
61573 & 500096 & 2006 & 199.60 & 0.36 & 26666.00 & 219432.62 & 0.75 & 1.10 & 0.82 \\
47514 & 212408 & 2006 & 1148.10 & 0.06 & 114596.00 & 1142980.72 & 1.00 & 1.00 & 1.00 \\
48568 & 240107 & 2006 & 730.40 & 0.10 & 72690.00 & 723599.13 & 1.00 & 0.99 & 1.00 \\
51340 & 240502 & 2006 & 128.20 & 0.14 & 12881.00 & 123993.73 & 1.00 & 0.97 & 0.96 \\
49571 & 240316 & 2006 & 36.80 & 0.36 & 3850.00 & 36835.82 & 0.96 & 1.00 & 0.96 \\
33914 & 106182 & 2006 & 654.20 & 0.25 & 64774.00 & 665590.15 & 1.01 & 1.02 & 1.03 \\
46339 & 200223 & 2006 & 77.60 & 0.06 & 9650.00 & 64446.72 & 0.80 & 0.83 & 0.67 \\
51321 & 240500 & 2006 & 5270.20 & 0.16 & 529989.00 & 5238434.53 & 0.99 & 0.99 & 0.99 \\
96696 & 611006 & 2006 & 10.30 & 0.16 & 1347.00 & 10186.23 & 0.76 & 0.99 & 0.76 \\
52065 & 301438 & 2006 & 610.70 & 0.11 & 61371.00 & 613980.99 & 1.00 & 1.01 & 1.00 \\
64240 & 500593 & 2006 & 2570.60 & 0.14 & 256087.00 & 2571935.22 & 1.00 & 1.00 & 1.00 \\
38348 & 107244 & 2006 & 160.40 & 0.01 & 15908.00 & 156638.10 & 1.01 & 0.98 & 0.98 \\
47009 & 200327 & 2006 & 42.40 & 0.09 & 4402.00 & 40164.27 & 0.96 & 0.95 & 0.91 \\
44372 & 109295 & 2006 & 1276.00 & 0.19 & 87146.00 & 828735.42 & 1.46 & 0.65 & 0.95 \\
48542 & 240105 & 2006 & 173.60 & 0.19 & 17953.00 & 157043.61 & 0.97 & 0.90 & 0.87 \\
51336 & 240501 & 2006 & 27.90 & 0.33 & 2814.00 & 25654.11 & 0.99 & 0.92 & 0.91 \\
40797 & 108153 & 2006 & 69.70 & 0.14 & 6963.00 & 63343.39 & 1.00 & 0.91 & 0.91 \\
53350 & 344278 & 2006 & 185.50 & 0.15 & 22200.00 & 176729.17 & 0.84 & 0.95 & 0.80 \\
42457 & 108968 & 2006 & 51.30 & 0.10 & 5151.00 & 49456.25 & 1.00 & 0.96 & 0.96 \\
38323 & 107243 & 2006 & 1028.20 & 0.14 & 103756.00 & 977014.83 & 0.99 & 0.95 & 0.94 \\
40827 & 108155 & 2006 & 62.90 & 0.07 & 6291.00 & 62907.19 & 1.00 & 1.00 & 1.00 \\
44738 & 109370 & 2006 & 951.80 & 0.17 & 105628.00 & 925041.75 & 0.90 & 0.97 & 0.88 \\
46987 & 200325 & 2006 & 300.60 & 0.17 & 29842.00 & 271120.37 & 1.01 & 0.90 & 0.91 \\
52042 & 301299 & 2006 & 2818.10 & -0.01 & 281855.00 & 2800901.11 & 1.00 & 0.99 & 0.99 \\
56739 & 400249 & 2006 & 47.00 & 0.18 & 6740.00 & 64835.85 & 0.70 & 1.38 & 0.96 \\
38286 & 107235 & 2006 & 53.60 & 0.03 & 5443.00 & 50414.73 & 0.98 & 0.94 & 0.93 \\
40693 & 108145 & 2006 & 140.90 & 0.11 & 14240.00 & 141401.08 & 0.99 & 1.00 & 0.99 \\
66298 & 500806 & 2006 & 260.40 & 0.13 & 27564.00 & 281874.53 & 0.94 & 1.08 & 1.02 \\
57532 & 400434 & 2006 & 234.00 & 0.14 & 21123.00 & 206645.34 & 1.11 & 0.88 & 0.98 \\
40555 & 108138 & 2006 & 29.90 & 0.18 & 3187.00 & 30098.44 & 0.94 & 1.01 & 0.94 \\
59579 & 410498 & 2006 & 75.30 & 0.19 & 7506.00 & 74984.16 & 1.00 & 1.00 & 1.00 \\
65745 & 500729 & 2006 & 1505.00 & 0.16 & 135909.00 & 1439690.89 & 1.11 & 0.96 & 1.06 \\
42422 & 108964 & 2006 & 359.20 & 0.16 & 35697.00 & 360060.32 & 1.01 & 1.00 & 1.01 \\
36222 & 106478 & 2006 & 1404.60 & 0.36 & 140457.00 & 1399715.89 & 1.00 & 1.00 & 1.00 \\
47953 & 225413 & 2006 & 75.60 & 0.12 & 7566.00 & 73284.29 & 1.00 & 0.97 & 0.97 \\
57550 & 400437 & 2006 & 32.10 & 0.17 & 3210.00 & 31173.87 & 1.00 & 0.97 & 0.97 \\
48110 & 240010 & 2006 & 359.50 & 0.34 & 35557.00 & 350779.94 & 1.01 & 0.98 & 0.99 \\
53705 & 355988 & 2006 & 49.30 & 0.07 & 5773.00 & 53122.60 & 0.85 & 1.08 & 0.92 \\
42002 & 108901 & 2006 & 282.10 & 0.13 & 28261.00 & 280412.25 & 1.00 & 0.99 & 0.99 \\
40584 & 108140 & 2006 & 83.80 & 0.15 & 9020.00 & 93220.09 & 0.93 & 1.11 & 1.03 \\
51293 & 240498 & 2006 & 106.80 & 0.13 & 10700.00 & 103788.92 & 1.00 & 0.97 & 0.97 \\
44115 & 109266 & 2006 & 4908.60 & 0.18 & 471182.00 & 4521795.63 & 1.04 & 0.92 & 0.96 \\
38579 & 107290 & 2006 & 1071.40 & 0.15 & 107004.00 & 1045757.83 & 1.00 & 0.98 & 0.98 \\
59560 & 410497 & 2006 & 346.70 & 0.14 & 32075.00 & 283179.10 & 1.08 & 0.82 & 0.88 \\
57559 & 400438 & 2006 & 44.30 & 0.13 & 4146.00 & 40788.52 & 1.07 & 0.92 & 0.98 \\
43165 & 109071 & 2006 & 15.60 & 0.01 & 1609.00 & 15880.84 & 0.97 & 1.02 & 0.99 \\
57568 & 400439 & 2006 & 71.40 & 0.35 & 6428.00 & 61529.13 & 1.11 & 0.86 & 0.96 \\
49491 & 240304 & 2006 & 778.50 & 0.10 & 79288.00 & 630222.29 & 0.98 & 0.81 & 0.79 \\
59538 & 410495 & 2006 & 102.70 & 0.14 & 10345.00 & 93770.86 & 0.99 & 0.91 & 0.91 \\
36245 & 106479 & 2006 & 342.90 & 0.14 & 34591.00 & 333263.10 & 0.99 & 0.97 & 0.96 \\
43156 & 109069 & 2006 & 171.10 & 0.08 & 20610.00 & 169797.72 & 0.83 & 0.99 & 0.82 \\
52144 & 302067 & 2006 & 19.60 & 0.11 & 1857.00 & 18635.00 & 1.06 & 0.95 & 1.00 \\
47090 & 200332 & 2006 & 84.70 & 0.22 & 10864.00 & 85732.88 & 0.78 & 1.01 & 0.79 \\
48499 & 240090 & 2006 & 10.70 & 0.05 & 1139.00 & 11593.14 & 0.94 & 1.08 & 1.02 \\
51243 & 240493 & 2006 & 103.10 & 0.30 & 10361.00 & 100957.29 & 1.00 & 0.98 & 0.97 \\
36173 & 106476 & 2006 & 702.00 & 0.19 & 57724.00 & 591672.32 & 1.22 & 0.84 & 1.03 \\
38636 & 107302 & 2006 & 274.60 & 0.16 & 27408.00 & 275170.70 & 1.00 & 1.00 & 1.00 \\
33707 & 106163 & 2006 & 1214.80 & 0.10 & 121623.00 & 1184602.25 & 1.00 & 0.98 & 0.97 \\
52156 & 302206 & 2006 & 843.40 & 0.19 & 90540.00 & 853603.55 & 0.93 & 1.01 & 0.94 \\
48809 & 240143 & 2006 & 237.50 & 0.06 & 23932.00 & 226119.45 & 0.99 & 0.95 & 0.94 \\
55420 & 400094 & 2006 & 1262.10 & 0.07 & 151624.00 & 1356380.20 & 0.83 & 1.07 & 0.89 \\
38626 & 107300 & 2006 & 52.70 & 0.08 & 5202.00 & 54260.10 & 1.01 & 1.03 & 1.04 \\
64332 & 500597 & 2006 & 17981.60 & 0.16 & 1789566.00 & 17933147.94 & 1.00 & 1.00 & 1.00 \\
65715 & 500726 & 2006 & 55.30 & 0.13 & 4988.00 & 41666.81 & 1.11 & 0.75 & 0.84 \\
33734 & 106164 & 2006 & 41.20 & 0.04 & 4117.00 & 41168.63 & 1.00 & 1.00 & 1.00 \\
51268 & 240496 & 2006 & 7.40 & 0.07 & 732.00 & 7109.98 & 1.01 & 0.96 & 0.97 \\
51270 & 240497 & 2006 & 40.70 & 0.11 & 4115.00 & 37511.99 & 0.99 & 0.92 & 0.91 \\
57517 & 400431 & 2006 & 6.50 & 0.09 & 545.00 & 6461.81 & 1.19 & 0.99 & 1.19 \\
61486 & 500082 & 2006 & 590.10 & 0.17 & 58229.00 & 601650.01 & 1.01 & 1.02 & 1.03 \\
57521 & 400432 & 2006 & 22.70 & 0.09 & 2260.00 & 19997.02 & 1.00 & 0.88 & 0.88 \\
36199 & 106477 & 2006 & 6092.80 & 0.16 & 633404.00 & 5753739.05 & 0.96 & 0.94 & 0.91 \\
62200 & 500364 & 2006 & 142.50 & 0.10 & 17404.00 & 168672.50 & 0.82 & 1.18 & 0.97 \\
40593 & 108141 & 2006 & 192.10 & 0.13 & 19168.00 & 197391.93 & 1.00 & 1.03 & 1.03 \\
54185 & 364818 & 2006 & 2104.90 & 0.10 & 206403.00 & 2064013.66 & 1.02 & 0.98 & 1.00 \\
44092 & 109265 & 2006 & 59.60 & 0.03 & 6377.00 & 61624.31 & 0.93 & 1.03 & 0.97 \\
53312 & 343540 & 2006 & 143.90 & 0.14 & 13624.00 & 128947.35 & 1.06 & 0.90 & 0.95 \\
40668 & 108144 & 2006 & 81.10 & -0.04 & 8372.00 & 83231.17 & 0.97 & 1.03 & 0.99 \\
33758 & 106167 & 2006 & 998.40 & 0.14 & 93549.00 & 885365.77 & 1.07 & 0.89 & 0.95 \\
41952 & 108874 & 2006 & 60.70 & 0.13 & 6379.00 & 60057.91 & 0.95 & 0.99 & 0.94 \\
59466 & 410486 & 2006 & 249.30 & 0.12 & 25169.00 & 249850.33 & 0.99 & 1.00 & 0.99 \\
43105 & 109064 & 2006 & 155.60 & 0.12 & 14593.00 & 148235.56 & 1.07 & 0.95 & 1.02 \\
38487 & 107263 & 2006 & 1983.10 & 0.12 & 200135.00 & 1974287.81 & 0.99 & 1.00 & 0.99 \\
33770 & 106169 & 2006 & 4710.20 & 0.16 & 471221.00 & 4596705.77 & 1.00 & 0.98 & 0.98 \\
66009 & 500766 & 2006 & 257.70 & 0.15 & 21931.00 & 218741.43 & 1.18 & 0.85 & 1.00 \\
66103 & 500777 & 2006 & 13.30 & 0.15 & 897.00 & 8143.12 & 1.48 & 0.61 & 0.91 \\
66181 & 500790 & 2006 & 544.50 & 0.18 & 36449.00 & 376847.74 & 1.49 & 0.69 & 1.03 \\
53683 & 355987 & 2006 & 601.80 & -0.06 & 67461.00 & 650104.09 & 0.89 & 1.08 & 0.96 \\
53332 & 344017 & 2006 & 7.20 & 0.07 & 698.00 & 7484.44 & 1.03 & 1.04 & 1.07 \\
59452 & 410484 & 2006 & 10.90 & 0.16 & 1075.00 & 9528.94 & 1.01 & 0.87 & 0.89 \\
46347 & 200224 & 2006 & 27.10 & 0.09 & 2791.00 & 28172.17 & 0.97 & 1.04 & 1.01 \\
38463 & 107260 & 2006 & 604.50 & 0.14 & 60556.00 & 569163.01 & 1.00 & 0.94 & 0.94 \\
48859 & 240148 & 2006 & 623.60 & 0.09 & 84939.00 & 627936.05 & 0.73 & 1.01 & 0.74 \\
59418 & 410481 & 2006 & 13.00 & 0.05 & 1521.00 & 15485.64 & 0.85 & 1.19 & 1.02 \\
44753 & 109371 & 2006 & 322.90 & 0.17 & 41106.00 & 393475.86 & 0.79 & 1.22 & 0.96 \\
47047 & 200330 & 2006 & 44.10 & 0.10 & 4413.00 & 43861.31 & 1.00 & 0.99 & 0.99 \\
44760 & 109373 & 2006 & 1581.50 & 0.06 & 162017.00 & 1571784.20 & 0.98 & 0.99 & 0.97 \\
45597 & 200082 & 2006 & 465.40 & 0.07 & 50187.00 & 437600.30 & 0.93 & 0.94 & 0.87 \\
43128 & 109065 & 2006 & 81.10 & 0.11 & 7231.00 & 73998.59 & 1.12 & 0.91 & 1.02 \\
56274 & 400186 & 2006 & 253.10 & 0.09 & 25318.00 & 253166.95 & 1.00 & 1.00 & 1.00 \\
48833 & 240144 & 2006 & 567.70 & 0.23 & 44305.00 & 364528.93 & 1.28 & 0.64 & 0.82 \\
38546 & 107281 & 2006 & 20.50 & 0.10 & 2034.00 & 20384.42 & 1.01 & 0.99 & 1.00 \\
40618 & 108142 & 2006 & 43.40 & 0.22 & 3774.00 & 37459.52 & 1.15 & 0.86 & 0.99 \\
56747 & 400250 & 2006 & 6.80 & 0.20 & 658.00 & 6357.86 & 1.03 & 0.93 & 0.97 \\
53002 & 336942 & 2006 & 854.10 & 0.15 & 71494.00 & 715269.92 & 1.19 & 0.84 & 1.00 \\
45582 & 200079 & 2006 & 24.50 & 0.09 & 2452.00 & 24514.52 & 1.00 & 1.00 & 1.00 \\
44772 & 109374 & 2006 & 64.30 & -0.02 & 6978.00 & 67825.07 & 0.92 & 1.05 & 0.97 \\
49499 & 240305 & 2006 & 60.40 & 0.10 & 5993.00 & 52062.13 & 1.01 & 0.86 & 0.87 \\
46362 & 200227 & 2006 & 16.30 & 0.07 & 1593.00 & 15840.69 & 1.02 & 0.97 & 0.99 \\
47068 & 200331 & 2006 & 44.40 & 0.14 & 4444.00 & 39801.96 & 1.00 & 0.90 & 0.90 \\
38512 & 107266 & 2006 & 1603.30 & 0.18 & 164824.00 & 1579784.56 & 0.97 & 0.99 & 0.96 \\
36271 & 106480 & 2006 & 2723.50 & 0.15 & 331596.00 & 2491154.74 & 0.82 & 0.91 & 0.75 \\
56254 & 400185 & 2006 & 226.80 & 0.22 & 22676.00 & 226758.06 & 1.00 & 1.00 & 1.00 \\
52108 & 301571 & 2006 & 315.00 & 0.04 & 31298.00 & 307198.61 & 1.01 & 0.98 & 0.98 \\
64263 & 500594 & 2006 & 4374.90 & 0.23 & 368336.00 & 3831131.45 & 1.19 & 0.88 & 1.04 \\
65880 & 500750 & 2006 & 2.50 & 0.13 & 248.00 & 2219.99 & 1.01 & 0.89 & 0.90 \\
59479 & 410487 & 2006 & 156.30 & 0.14 & 16079.00 & 161906.92 & 0.97 & 1.04 & 1.01 \\
41980 & 108886 & 2006 & 61.70 & 10.09 & 6671.00 & 59463.24 & 0.92 & 0.96 & 0.89 \\
40643 & 108143 & 2006 & 26.10 & 0.10 & 2569.00 & 25681.28 & 1.02 & 0.98 & 1.00 \\
42439 & 108966 & 2006 & 110.20 & 0.07 & 9473.00 & 89812.60 & 1.16 & 0.81 & 0.95 \\
52998 & 336593 & 2006 & 62.20 & 0.17 & 6219.00 & 61282.52 & 1.00 & 0.99 & 0.99 \\
33934 & 106192 & 2006 & 1566.80 & 0.12 & 159547.00 & 1439550.39 & 0.98 & 0.92 & 0.90 \\
52560 & 303124 & 2006 & 70.50 & -0.02 & 7113.00 & 72315.59 & 0.99 & 1.03 & 1.02 \\
63211 & 500489 & 2006 & 1842.90 & 0.07 & 194814.00 & 1936320.71 & 0.95 & 1.05 & 0.99 \\
46307 & 200210 & 2006 & 11.70 & 0.10 & 1431.00 & 11127.30 & 0.82 & 0.95 & 0.78 \\
34143 & 106209 & 2006 & 321.80 & 0.12 & 34451.00 & 343380.09 & 0.93 & 1.07 & 1.00 \\
42970 & 109044 & 2006 & 48.60 & 0.03 & 4937.00 & 45084.82 & 0.98 & 0.93 & 0.91 \\
54269 & 365483 & 2006 & 3255.30 & 0.16 & 325586.00 & 3120777.81 & 1.00 & 0.96 & 0.96 \\
61678 & 500114 & 2006 & 186.40 & 0.09 & 26164.00 & 271723.92 & 0.71 & 1.46 & 1.04 \\
49654 & 240328 & 2006 & 35.40 & 0.16 & 3724.00 & 37208.53 & 0.95 & 1.05 & 1.00 \\
73739 & 600527 & 2006 & 136.00 & 0.13 & 12486.00 & 117180.92 & 1.09 & 0.86 & 0.94 \\
34170 & 106210 & 2006 & 1049.40 & 0.28 & 98434.00 & 960764.18 & 1.07 & 0.92 & 0.98 \\
49663 & 240330 & 2006 & 2060.40 & 0.16 & 205229.00 & 1865517.14 & 1.00 & 0.91 & 0.91 \\
51378 & 240509 & 2006 & 66.10 & 0.18 & 6161.00 & 61278.52 & 1.07 & 0.93 & 0.99 \\
36475 & 106535 & 2006 & 1152.40 & 0.12 & 107498.00 & 963833.11 & 1.07 & 0.84 & 0.90 \\
35096 & 106320 & 2006 & 608.30 & 0.13 & 67365.00 & 603481.68 & 0.90 & 0.99 & 0.90 \\
74444 & 601001 & 2006 & 83.00 & 0.10 & 9746.00 & 82134.48 & 0.85 & 0.99 & 0.84 \\
51958 & 300673 & 2006 & 106.90 & 0.11 & 12449.00 & 105139.77 & 0.86 & 0.98 & 0.84 \\
46282 & 200207 & 2006 & 41.10 & 0.12 & 4088.00 & 39844.07 & 1.01 & 0.97 & 0.97 \\
46900 & 200312 & 2006 & 732.40 & -0.02 & 73084.00 & 721155.87 & 1.00 & 0.98 & 0.99 \\
37968 & 107175 & 2006 & 1279.10 & -0.01 & 127701.00 & 1307160.11 & 1.00 & 1.02 & 1.02 \\
56620 & 400232 & 2006 & 120.00 & 0.10 & 9905.00 & 101349.52 & 1.21 & 0.84 & 1.02 \\
45783 & 200133 & 2006 & 17.00 & 0.09 & 1698.00 & 16904.62 & 1.00 & 0.99 & 1.00 \\
34197 & 106211 & 2006 & 54.80 & 0.21 & 5659.00 & 56825.82 & 0.97 & 1.04 & 1.00 \\
52597 & 303140 & 2006 & 823.90 & 0.07 & 76565.00 & 780318.59 & 1.08 & 0.95 & 1.02 \\
43846 & 109224 & 2006 & 13.00 & 0.12 & 1479.00 & 12460.30 & 0.88 & 0.96 & 0.84 \\
51388 & 240510 & 2006 & 34.00 & 0.04 & 3368.00 & 34685.87 & 1.01 & 1.02 & 1.03 \\
44687 & 109359 & 2006 & 162.80 & 0.16 & 30754.00 & 310179.17 & 0.53 & 1.91 & 1.01 \\
54247 & 364993 & 2006 & 139.60 & 0.13 & 13728.00 & 129661.35 & 1.02 & 0.93 & 0.94 \\
54240 & 364951 & 2006 & 79.80 & 0.14 & 9557.00 & 93046.01 & 0.83 & 1.17 & 0.97 \\
56372 & 400203 & 2006 & 199.80 & 0.23 & 20461.00 & 193124.14 & 0.98 & 0.97 & 0.94 \\
73635 & 600501 & 2006 & 520.80 & 0.13 & 44166.00 & 386531.34 & 1.18 & 0.74 & 0.88 \\
56392 & 400204 & 2006 & 29.10 & 0.18 & 3196.00 & 30976.05 & 0.91 & 1.06 & 0.97 \\
35122 & 106321 & 2006 & 10.90 & 0.10 & 1125.00 & 10743.23 & 0.97 & 0.99 & 0.95 \\
48961 & 240174 & 2006 & 156.80 & 0.25 & 15761.00 & 125571.81 & 0.99 & 0.80 & 0.80 \\
48589 & 240111 & 2006 & 892.00 & -0.02 & 115335.00 & 826322.21 & 0.77 & 0.93 & 0.72 \\
42517 & 108973 & 2006 & 3081.10 & 0.18 & 319926.00 & 2608121.07 & 0.96 & 0.85 & 0.82 \\
52955 & 335933 & 2006 & 26.80 & 0.00 & 2691.00 & 26865.94 & 1.00 & 1.00 & 1.00 \\
49638 & 240327 & 2006 & 35.80 & 0.13 & 4491.00 & 35428.21 & 0.80 & 0.99 & 0.79 \\
38002 & 107181 & 2006 & 141.10 & 0.18 & 13850.00 & 132699.12 & 1.02 & 0.94 & 0.96 \\
51984 & 300684 & 2006 & 137.10 & 0.18 & 13058.00 & 139181.56 & 1.05 & 1.02 & 1.07 \\
51366 & 240507 & 2006 & 142.00 & 0.23 & 13713.00 & 144649.38 & 1.04 & 1.02 & 1.05 \\
57939 & 410003 & 2006 & 375.30 & 0.11 & 37330.00 & 370706.62 & 1.01 & 0.99 & 0.99 \\
37990 & 107179 & 2006 & 960.50 & 0.15 & 105889.00 & 941516.56 & 0.91 & 0.98 & 0.89 \\
64102 & 500587 & 2006 & 2814.30 & 0.16 & 217714.00 & 2367380.28 & 1.29 & 0.84 & 1.09 \\
37980 & 107178 & 2006 & 57.60 & 0.06 & 5880.00 & 59125.78 & 0.98 & 1.03 & 1.01 \\
55637 & 400132 & 2006 & 460.30 & 0.18 & 45370.00 & 453121.01 & 1.01 & 0.98 & 1.00 \\
51969 & 300679 & 2006 & 1020.20 & 0.13 & 112531.00 & 1117604.09 & 0.91 & 1.10 & 0.99 \\
51399 & 240511 & 2006 & 107.90 & 0.29 & 10913.00 & 101655.71 & 0.99 & 0.94 & 0.93 \\
42961 & 109042 & 2006 & 94.40 & 0.15 & 6733.00 & 56957.04 & 1.40 & 0.60 & 0.85 \\
37945 & 107173 & 2006 & 40.60 & 0.34 & 3735.00 & 40933.43 & 1.09 & 1.01 & 1.10 \\
51418 & 240517 & 2006 & 90.70 & 0.35 & 8099.00 & 79349.20 & 1.12 & 0.87 & 0.98 \\
47867 & 222408 & 2006 & 1407.10 & 0.14 & 142075.00 & 1393039.35 & 0.99 & 0.99 & 0.98 \\
34239 & 106214 & 2006 & 56.70 & 0.20 & 6026.00 & 60374.32 & 0.94 & 1.06 & 1.00 \\
55126 & 400062 & 2006 & 675.10 & 0.22 & 75218.00 & 714874.09 & 0.90 & 1.06 & 0.95 \\
37907 & 107160 & 2006 & 17967.50 & 0.14 & 1793404.00 & 17641542.07 & 1.00 & 0.98 & 0.98 \\
62143 & 500348 & 2006 & 958.40 & 0.12 & 96038.00 & 959303.60 & 1.00 & 1.00 & 1.00 \\
56419 & 400206 & 2006 & 60.10 & 0.07 & 6124.00 & 55000.81 & 0.98 & 0.92 & 0.90 \\
55145 & 400063 & 2006 & 21.40 & 0.14 & 1960.00 & 18102.76 & 1.09 & 0.85 & 0.92 \\
54302 & 367166 & 2006 & 266.30 & 0.21 & 29376.00 & 263823.79 & 0.91 & 0.99 & 0.90 \\
44654 & 109351 & 2006 & 48.10 & 0.05 & 7122.00 & 71066.16 & 0.68 & 1.48 & 1.00 \\
54324 & 367168 & 2006 & 26.00 & 0.17 & 2572.00 & 25755.16 & 1.01 & 0.99 & 1.00 \\
41770 & 108855 & 2006 & 724.90 & 0.29 & 82416.00 & 742036.03 & 0.88 & 1.02 & 0.90 \\
37886 & 107156 & 2006 & 69.10 & -0.06 & 11135.00 & 104573.32 & 0.62 & 1.51 & 0.94 \\
42947 & 109038 & 2006 & 38.40 & 0.10 & 3841.00 & 37247.44 & 1.00 & 0.97 & 0.97 \\
74781 & 601168 & 2006 & 48.90 & 0.08 & 5460.00 & 48332.58 & 0.90 & 0.99 & 0.89 \\
55612 & 400130 & 2006 & 132.60 & 0.02 & 14249.00 & 142473.51 & 0.93 & 1.07 & 1.00 \\
45812 & 200146 & 2006 & 143.00 & 0.10 & 14329.00 & 134588.28 & 1.00 & 0.94 & 0.94 \\
56414 & 400205 & 2006 & 56.90 & 0.19 & 5367.00 & 54125.67 & 1.06 & 0.95 & 1.01 \\
63257 & 500491 & 2006 & 2300.60 & 0.19 & 195142.00 & 2062464.14 & 1.18 & 0.90 & 1.06 \\
41007 & 108175 & 2006 & 165.00 & 0.14 & 15958.00 & 166796.94 & 1.03 & 1.01 & 1.05 \\
34227 & 106213 & 2006 & 69.90 & 0.08 & 11240.00 & 101862.86 & 0.62 & 1.46 & 0.91 \\
35081 & 106318 & 2006 & 88.30 & 0.08 & 8821.00 & 87459.35 & 1.00 & 0.99 & 0.99 \\
34209 & 106212 & 2006 & 72.30 & 0.06 & 8006.00 & 78728.00 & 0.90 & 1.09 & 0.98 \\
41811 & 108857 & 2006 & 48.40 & 0.31 & 4590.00 & 47567.15 & 1.05 & 0.98 & 1.04 \\
51402 & 240512 & 2006 & 4.10 & 0.25 & 341.00 & 3294.06 & 1.20 & 0.80 & 0.97 \\
44670 & 109358 & 2006 & 102.40 & 0.14 & 13130.00 & 134598.67 & 0.78 & 1.31 & 1.03 \\
46885 & 200311 & 2006 & 688.70 & 0.21 & 68799.00 & 668682.82 & 1.00 & 0.97 & 0.97 \\
47894 & 222658 & 2006 & 429.20 & 0.07 & 43003.00 & 439912.84 & 1.00 & 1.02 & 1.02 \\
53648 & 355536 & 2006 & 4.50 & 0.04 & 451.00 & 4622.52 & 1.00 & 1.03 & 1.02 \\
36503 & 106541 & 2006 & 249.30 & 0.16 & 24964.00 & 247167.06 & 1.00 & 0.99 & 0.99 \\
40962 & 108168 & 2006 & 305.90 & 0.15 & 30632.00 & 275660.47 & 1.00 & 0.90 & 0.90 \\
42984 & 109046 & 2006 & 49.70 & 0.14 & 7353.00 & 71726.59 & 0.68 & 1.44 & 0.98 \\
41789 & 108856 & 2006 & 328.30 & 0.25 & 30118.00 & 321243.25 & 1.09 & 0.98 & 1.07 \\
44662 & 109357 & 2006 & 228.70 & 0.12 & 26093.00 & 237900.79 & 0.88 & 1.04 & 0.91 \\
40986 & 108170 & 2006 & 112.60 & 0.01 & 11268.00 & 109590.04 & 1.00 & 0.97 & 0.97 \\
36515 & 106545 & 2006 & 17.70 & 0.10 & 1680.00 & 16662.99 & 1.05 & 0.94 & 0.99 \\
46876 & 200310 & 2006 & 155.50 & 0.10 & 15550.00 & 152977.66 & 1.00 & 0.98 & 0.98 \\
61715 & 500116 & 2006 & 1678.80 & 0.15 & 165717.00 & 1477178.06 & 1.01 & 0.88 & 0.89 \\
43872 & 109226 & 2006 & 122.80 & 0.16 & 14530.00 & 115578.59 & 0.85 & 0.94 & 0.80 \\
45796 & 200140 & 2006 & 1166.00 & 0.16 & 116519.00 & 1087333.06 & 1.00 & 0.93 & 0.93 \\
48650 & 240116 & 2006 & 355.90 & 0.07 & 33174.00 & 314406.42 & 1.07 & 0.88 & 0.95 \\
52623 & 303175 & 2006 & 1224.90 & 0.14 & 126110.00 & 1175318.79 & 0.97 & 0.96 & 0.93 \\
52959 & 336065 & 2006 & 11.20 & 0.02 & 1213.00 & 11212.19 & 0.92 & 1.00 & 0.92 \\
36454 & 106529 & 2006 & 227.30 & 0.13 & 27802.00 & 229770.84 & 0.82 & 1.01 & 0.83 \\
57624 & 400447 & 2006 & 5.80 & 0.09 & 583.00 & 5608.06 & 0.99 & 0.97 & 0.96 \\
49588 & 240319 & 2006 & 296.90 & 0.09 & 32158.00 & 301909.77 & 0.92 & 1.02 & 0.94 \\
56350 & 400197 & 2006 & 10.90 & -0.01 & 1093.00 & 10823.57 & 1.00 & 0.99 & 0.99 \\
57642 & 400448 & 2006 & 22.10 & 0.11 & 2210.00 & 21591.98 & 1.00 & 0.98 & 0.98 \\
43037 & 109056 & 2006 & 645.50 & 0.24 & 63105.00 & 545230.48 & 1.02 & 0.84 & 0.86 \\
45684 & 200090 & 2006 & 25.50 & 0.17 & 2431.00 & 19807.73 & 1.05 & 0.78 & 0.81 \\
51353 & 240505 & 2006 & 153.80 & 0.11 & 16326.00 & 148566.87 & 0.94 & 0.97 & 0.91 \\
57660 & 400449 & 2006 & 191.40 & 0.13 & 20878.00 & 197236.84 & 0.92 & 1.03 & 0.94 \\
34004 & 106197 & 2006 & 145.90 & 0.22 & 10476.00 & 108487.36 & 1.39 & 0.74 & 1.04 \\
48884 & 240149 & 2006 & 255.70 & 0.27 & 24457.00 & 239761.86 & 1.05 & 0.94 & 0.98 \\
40897 & 108161 & 2006 & 179.00 & 0.15 & 17564.00 & 171303.79 & 1.02 & 0.96 & 0.98 \\
52574 & 303130 & 2006 & 69.70 & 0.09 & 8749.00 & 70864.03 & 0.80 & 1.02 & 0.81 \\
38162 & 107209 & 2006 & 399.40 & 0.13 & 39692.00 & 383311.87 & 1.01 & 0.96 & 0.97 \\
44404 & 109300 & 2006 & 512.40 & 0.20 & 51173.00 & 499330.22 & 1.00 & 0.97 & 0.98 \\
57678 & 400450 & 2006 & 59.30 & 0.11 & 6032.00 & 59592.97 & 0.98 & 1.00 & 0.99 \\
57705 & 400456 & 2006 & 317.10 & 0.22 & 31734.00 & 267544.47 & 1.00 & 0.84 & 0.84 \\
34031 & 106198 & 2006 & 707.30 & 0.16 & 85808.00 & 781886.75 & 0.82 & 1.11 & 0.91 \\
55012 & 400046 & 2006 & 97.50 & 0.13 & 9753.00 & 96542.23 & 1.00 & 0.99 & 0.99 \\
36444 & 106528 & 2006 & 151.60 & -0.02 & 14790.00 & 147700.36 & 1.03 & 0.97 & 1.00 \\
38179 & 107210 & 2006 & 65.50 & -0.00 & 8655.00 & 60408.84 & 0.76 & 0.92 & 0.70 \\
45681 & 200089 & 2006 & 56.20 & 0.09 & 5575.00 & 56638.40 & 1.01 & 1.01 & 1.02 \\
59236 & 410446 & 2006 & 47.80 & 0.11 & 4768.00 & 46761.88 & 1.00 & 0.98 & 0.98 \\
56341 & 400192 & 2006 & 9.50 & 0.05 & 989.00 & 8297.72 & 0.96 & 0.87 & 0.84 \\
59414 & 410479 & 2006 & 220.50 & 0.08 & 21730.00 & 217302.46 & 1.01 & 0.99 & 1.00 \\
40856 & 108159 & 2006 & 37.00 & 0.17 & 3495.00 & 37001.29 & 1.06 & 1.00 & 1.06 \\
58645 & 410181 & 2006 & 50.30 & 0.20 & 4954.00 & 46207.64 & 1.02 & 0.92 & 0.93 \\
33961 & 106193 & 2006 & 20.40 & 0.00 & 2313.00 & 19801.89 & 0.88 & 0.97 & 0.86 \\
45650 & 200087 & 2006 & 39.10 & 0.16 & 3875.00 & 38749.13 & 1.01 & 0.99 & 1.00 \\
36399 & 106523 & 2006 & 105.00 & 0.12 & 11569.00 & 114096.62 & 0.91 & 1.09 & 0.99 \\
61639 & 500109 & 2006 & 12392.00 & 0.18 & 1016807.00 & 10228710.64 & 1.22 & 0.83 & 1.01 \\
40880 & 108160 & 2006 & 178.90 & 0.15 & 19948.00 & 205354.82 & 0.90 & 1.15 & 1.03 \\
36419 & 106524 & 2006 & 43.20 & 0.16 & 2531.00 & 23454.05 & 1.71 & 0.54 & 0.93 \\
45672 & 200088 & 2006 & 78.20 & 0.12 & 7853.00 & 78116.03 & 1.00 & 1.00 & 0.99 \\
35175 & 106330 & 2006 & 146.60 & -0.01 & 17878.00 & 184180.50 & 0.82 & 1.26 & 1.03 \\
38216 & 107222 & 2006 & 2237.60 & 0.19 & 223363.00 & 2094574.05 & 1.00 & 0.94 & 0.94 \\
57596 & 400442 & 2006 & 4.50 & 0.16 & 453.00 & 4214.53 & 0.99 & 0.94 & 0.93 \\
57603 & 400445 & 2006 & 35.10 & 0.11 & 3449.00 & 34160.91 & 1.02 & 0.97 & 0.99 \\
38191 & 107215 & 2006 & 1058.00 & 0.34 & 95005.00 & 990741.32 & 1.11 & 0.94 & 1.04 \\
52964 & 336226 & 2006 & 397.90 & 0.15 & 32367.00 & 307876.40 & 1.23 & 0.77 & 0.95 \\
50404 & 240415 & 2006 & 1231.80 & 0.17 & 100809.00 & 893577.86 & 1.22 & 0.73 & 0.89 \\
58655 & 410186 & 2006 & 258.70 & 0.14 & 23263.00 & 232737.77 & 1.11 & 0.90 & 1.00 \\
38120 & 107202 & 2006 & 23.40 & 0.15 & 2039.00 & 19976.35 & 1.15 & 0.85 & 0.98 \\
46957 & 200322 & 2006 & 43.20 & 0.07 & 4358.00 & 42687.01 & 0.99 & 0.99 & 0.98 \\
58691 & 410209 & 2006 & 2.40 & 0.16 & 228.00 & 2129.79 & 1.05 & 0.89 & 0.93 \\
34074 & 106203 & 2006 & 80.50 & 0.22 & 8016.00 & 71925.74 & 1.00 & 0.89 & 0.90 \\
58709 & 410210 & 2006 & 5.30 & 0.19 & 521.00 & 4830.04 & 1.02 & 0.91 & 0.93 \\
64171 & 500590 & 2006 & 1248.50 & 0.10 & 131806.00 & 1313593.77 & 0.95 & 1.05 & 1.00 \\
59114 & 410433 & 2006 & 3156.90 & 0.19 & 319313.00 & 3053344.18 & 0.99 & 0.97 & 0.96 \\
46924 & 200317 & 2006 & 26.30 & 0.10 & 3110.00 & 26915.85 & 0.85 & 1.02 & 0.87 \\
38045 & 107196 & 2006 & 34.80 & 0.22 & 3478.00 & 32321.44 & 1.00 & 0.93 & 0.93 \\
34100 & 106207 & 2006 & 17.60 & 0.09 & 1697.00 & 17316.26 & 1.04 & 0.98 & 1.02 \\
58718 & 410211 & 2006 & 2.40 & 0.15 & 221.00 & 2053.97 & 1.09 & 0.86 & 0.93 \\
64148 & 500589 & 2006 & 4189.20 & 0.10 & 418681.00 & 4189422.15 & 1.00 & 1.00 & 1.00 \\
56641 & 400236 & 2006 & 383.70 & 0.21 & 39000.00 & 389854.28 & 0.98 & 1.02 & 1.00 \\
58762 & 410216 & 2006 & 149.60 & 0.18 & 9648.00 & 90823.00 & 1.55 & 0.61 & 0.94 \\
64125 & 500588 & 2006 & 1750.90 & 0.06 & 178306.00 & 1777578.78 & 0.98 & 1.02 & 1.00 \\
47587 & 215413 & 2006 & 327.70 & 0.12 & 23674.00 & 234478.47 & 1.38 & 0.72 & 0.99 \\
41844 & 108860 & 2006 & 38.20 & 0.06 & 3814.00 & 37951.35 & 1.00 & 0.99 & 1.00 \\
43008 & 109048 & 2006 & 162.50 & 0.10 & 16626.00 & 166937.78 & 0.98 & 1.03 & 1.00 \\
53659 & 355965 & 2006 & 1881.20 & 0.10 & 188285.00 & 1849327.64 & 1.00 & 0.98 & 0.98 \\
38022 & 107192 & 2006 & 1125.90 & 0.15 & 112859.00 & 1111805.44 & 1.00 & 0.99 & 0.99 \\
34116 & 106208 & 2006 & 26.50 & 0.14 & 2616.00 & 27076.46 & 1.01 & 1.02 & 1.04 \\
55232 & 400074 & 2006 & 1874.20 & 0.12 & 196863.00 & 1884720.24 & 0.95 & 1.01 & 0.96 \\
45749 & 200097 & 2006 & 24.40 & 0.16 & 2431.00 & 23595.52 & 1.00 & 0.97 & 0.97 \\
56661 & 400237 & 2006 & 38.80 & 0.16 & 3868.00 & 38183.31 & 1.00 & 0.98 & 0.99 \\
54602 & 377010 & 2006 & 2559.00 & 0.17 & 255576.00 & 2389831.26 & 1.00 & 0.93 & 0.94 \\
50396 & 240414 & 2006 & 7202.90 & 0.15 & 742580.00 & 6835499.75 & 0.97 & 0.95 & 0.92 \\
64194 & 500591 & 2006 & 1960.80 & 0.19 & 195362.00 & 1973530.31 & 1.00 & 1.01 & 1.01 \\
38095 & 107201 & 2006 & 507.70 & 0.12 & 58162.00 & 513663.18 & 0.87 & 1.01 & 0.88 \\
59385 & 410472 & 2006 & 1101.40 & 0.14 & 125918.00 & 1282727.94 & 0.87 & 1.16 & 1.02 \\
74801 & 601172 & 2006 & 1171.10 & 0.35 & 119102.00 & 1189670.95 & 0.98 & 1.02 & 1.00 \\
42480 & 108970 & 2006 & 232.70 & 0.04 & 23124.00 & 218672.25 & 1.01 & 0.94 & 0.95 \\
58660 & 410199 & 2006 & 42.10 & 0.28 & 3802.00 & 39272.88 & 1.11 & 0.93 & 1.03 \\
35149 & 106329 & 2006 & 9.20 & 0.22 & 948.00 & 9476.22 & 0.97 & 1.03 & 1.00 \\
59367 & 410470 & 2006 & 60.50 & 0.17 & 5634.00 & 58041.79 & 1.07 & 0.96 & 1.03 \\
58665 & 410200 & 2006 & 67.10 & 0.17 & 6536.00 & 63199.26 & 1.03 & 0.94 & 0.97 \\
41864 & 108866 & 2006 & 377.10 & 0.18 & 38226.00 & 372141.99 & 0.99 & 0.99 & 0.97 \\
49614 & 240322 & 2006 & 2206.60 & 0.10 & 211657.00 & 1983568.79 & 1.04 & 0.90 & 0.94 \\
44422 & 109307 & 2006 & 277.20 & 0.15 & 25310.00 & 239039.98 & 1.10 & 0.86 & 0.94 \\
54868 & 400019 & 2006 & 3622.60 & 0.19 & 352229.00 & 3398829.64 & 1.03 & 0.94 & 0.96 \\
49630 & 240326 & 2006 & 13.40 & 0.15 & 1286.00 & 11800.41 & 1.04 & 0.88 & 0.92 \\
45710 & 200092 & 2006 & 74.70 & 0.23 & 7553.00 & 70377.99 & 0.99 & 0.94 & 0.93 \\
46935 & 200319 & 2006 & 375.00 & 0.06 & 35478.00 & 326044.69 & 1.06 & 0.87 & 0.92 \\
42492 & 108971 & 2006 & 225.30 & 0.13 & 27488.00 & 198677.26 & 0.82 & 0.88 & 0.72 \\
38086 & 107199 & 2006 & 30.70 & 0.08 & 3464.00 & 35455.50 & 0.89 & 1.15 & 1.02 \\
58686 & 410208 & 2006 & 226.50 & 0.15 & 17584.00 & 181295.37 & 1.29 & 0.80 & 1.03 \\
53376 & 344889 & 2006 & 1839.10 & 0.15 & 183979.00 & 1816920.45 & 1.00 & 0.99 & 0.99 \\
45732 & 200094 & 2006 & 207.60 & 0.07 & 20695.00 & 204404.95 & 1.00 & 0.98 & 0.99 \\
38070 & 107198 & 2006 & 547.80 & 0.09 & 54797.00 & 553077.36 & 1.00 & 1.01 & 1.01 \\
46928 & 200318 & 2006 & 9.50 & 0.18 & 874.00 & 7406.94 & 1.09 & 0.78 & 0.85 \\
52001 & 300695 & 2006 & 1040.80 & 0.19 & 101996.00 & 942253.18 & 1.02 & 0.91 & 0.92 \\
38438 & 107259 & 2006 & 543.10 & 0.09 & 75840.00 & 649820.78 & 0.72 & 1.20 & 0.86 \\
48453 & 240085 & 2006 & 210.20 & 0.15 & 21494.00 & 206190.02 & 0.98 & 0.98 & 0.96 \\
4260 & 100598 & 2006 & 3432.30 & 0.15 & 325806.00 & 3304750.61 & 1.05 & 0.96 & 1.01 \\
2546 & 100343 & 2006 & 617.00 & 0.21 & 85113.00 & 737422.34 & 0.72 & 1.20 & 0.87 \\
27989 & 105364 & 2006 & 131.60 & 0.06 & 13316.00 & 133473.90 & 0.99 & 1.01 & 1.00 \\
17849 & 102365 & 2006 & 302.10 & 0.04 & 32413.00 & 297673.25 & 0.93 & 0.99 & 0.92 \\
31351 & 105879 & 2006 & 568.80 & 0.07 & 72868.00 & 582003.98 & 0.78 & 1.02 & 0.80 \\
27957 & 105358 & 2006 & 5898.40 & 0.15 & 534332.00 & 5518264.77 & 1.10 & 0.94 & 1.03 \\
14203 & 101820 & 2006 & 162.30 & 0.14 & 15997.00 & 145961.77 & 1.01 & 0.90 & 0.91 \\
17818 & 102364 & 2006 & 1691.90 & 0.15 & 177453.00 & 1623562.25 & 0.95 & 0.96 & 0.91 \\
31369 & 105880 & 2006 & 1255.60 & 0.10 & 143674.00 & 1182986.53 & 0.87 & 0.94 & 0.82 \\
11754 & 101460 & 2006 & 1634.30 & 0.13 & 162897.00 & 1534348.21 & 1.00 & 0.94 & 0.94 \\
22800 & 103065 & 2006 & 191.30 & 0.10 & 18945.00 & 181459.54 & 1.01 & 0.95 & 0.96 \\
27938 & 105353 & 2006 & 143.40 & 0.12 & 17255.00 & 142658.69 & 0.83 & 0.99 & 0.83 \\
27915 & 105346 & 2006 & 2074.20 & 0.16 & 207323.00 & 2065649.35 & 1.00 & 1.00 & 1.00 \\
6917 & 100968 & 2006 & 495.50 & 0.05 & 49444.00 & 475692.42 & 1.00 & 0.96 & 0.96 \\
31396 & 105881 & 2006 & 6779.60 & 0.15 & 691330.00 & 6627853.77 & 0.98 & 0.98 & 0.96 \\
28021 & 105369 & 2006 & 189.20 & 0.10 & 17394.00 & 175197.05 & 1.09 & 0.93 & 1.01 \\
14170 & 101819 & 2006 & 152.50 & 0.19 & 15375.00 & 142727.44 & 0.99 & 0.94 & 0.93 \\
17877 & 102367 & 2006 & 157.40 & 0.16 & 16671.00 & 151237.90 & 0.94 & 0.96 & 0.91 \\
17890 & 102371 & 2006 & 162.30 & 0.14 & 16418.00 & 149819.84 & 0.99 & 0.92 & 0.91 \\
17940 & 102376 & 2006 & 300.90 & 0.11 & 30070.00 & 254266.74 & 1.00 & 0.85 & 0.85 \\
11818 & 101462 & 2006 & 1400.40 & 0.22 & 142945.00 & 1372230.89 & 0.98 & 0.98 & 0.96 \\
10359 & 101283 & 2006 & 5201.00 & 0.16 & 464758.00 & 4398192.76 & 1.12 & 0.85 & 0.95 \\
2512 & 100336 & 2006 & 177.30 & 0.17 & 17686.00 & 160078.65 & 1.00 & 0.90 & 0.91 \\
22606 & 103024 & 2006 & 1347.90 & 0.16 & 134652.00 & 1238258.52 & 1.00 & 0.92 & 0.92 \\
31285 & 105874 & 2006 & 299.30 & 0.20 & 26439.00 & 280866.77 & 1.13 & 0.94 & 1.06 \\
4290 & 100600 & 2006 & 101.90 & 0.15 & 10247.00 & 101898.05 & 0.99 & 1.00 & 0.99 \\
4233 & 100590 & 2006 & 172.80 & 0.08 & 17015.00 & 162685.25 & 1.02 & 0.94 & 0.96 \\
28050 & 105370 & 2006 & 97.10 & 0.13 & 9219.00 & 92801.65 & 1.05 & 0.96 & 1.01 \\
22643 & 103027 & 2006 & 10812.10 & 0.08 & 1083639.00 & 10729284.45 & 1.00 & 0.99 & 0.99 \\
11787 & 101461 & 2006 & 1803.00 & 0.08 & 188136.00 & 1774850.03 & 0.96 & 0.98 & 0.94 \\
14138 & 101805 & 2006 & 639.70 & 0.13 & 63985.00 & 626847.07 & 1.00 & 0.98 & 0.98 \\
31323 & 105878 & 2006 & 1724.70 & 0.15 & 163897.00 & 1514497.72 & 1.05 & 0.88 & 0.92 \\
22687 & 103028 & 2006 & 6107.40 & 0.14 & 588205.00 & 5470242.98 & 1.04 & 0.90 & 0.93 \\
17910 & 102372 & 2006 & 5979.40 & 0.18 & 597371.00 & 5548387.52 & 1.00 & 0.93 & 0.93 \\
6885 & 100967 & 2006 & 1050.70 & 0.09 & 105336.00 & 898289.66 & 1.00 & 0.85 & 0.85 \\
8826 & 101100 & 2006 & 1383.70 & 0.09 & 158366.00 & 1404257.69 & 0.87 & 1.01 & 0.89 \\
17758 & 102356 & 2006 & 13.50 & 0.08 & 1323.00 & 12750.02 & 1.02 & 0.94 & 0.96 \\
22835 & 103067 & 2006 & 106.90 & 0.10 & 10564.00 & 100719.06 & 1.01 & 0.94 & 0.95 \\
23006 & 103101 & 2006 & 536.30 & 0.17 & 63565.00 & 454788.70 & 0.84 & 0.85 & 0.72 \\
11687 & 101456 & 2006 & 940.40 & 0.20 & 93940.00 & 912083.23 & 1.00 & 0.97 & 0.97 \\
27801 & 105331 & 2006 & 28.10 & 0.11 & 2806.00 & 24790.38 & 1.00 & 0.88 & 0.88 \\
31455 & 105890 & 2006 & 82.40 & 0.13 & 12101.00 & 86151.64 & 0.68 & 1.05 & 0.71 \\
31472 & 105895 & 2006 & 216.40 & 0.08 & 24183.00 & 246428.85 & 0.89 & 1.14 & 1.02 \\
17626 & 102321 & 2006 & 173.40 & 0.06 & 17878.00 & 165119.19 & 0.97 & 0.95 & 0.92 \\
2656 & 100350 & 2006 & 94.50 & 0.09 & 9453.00 & 93927.43 & 1.00 & 0.99 & 0.99 \\
2637 & 100348 & 2006 & 102.40 & 0.18 & 10396.00 & 100749.33 & 0.98 & 0.98 & 0.97 \\
7023 & 100985 & 2006 & 2152.60 & 0.15 & 214534.00 & 2109115.13 & 1.00 & 0.98 & 0.98 \\
2675 & 100351 & 2006 & 45.00 & 0.10 & 4473.00 & 42770.81 & 1.01 & 0.95 & 0.96 \\
10424 & 101285 & 2006 & 650.60 & 0.02 & 65225.00 & 640378.37 & 1.00 & 0.98 & 0.98 \\
23080 & 103110 & 2006 & 2880.40 & 0.17 & 272792.00 & 2654035.34 & 1.06 & 0.92 & 0.97 \\
17589 & 102319 & 2006 & 544.50 & 0.08 & 54098.00 & 517720.75 & 1.01 & 0.95 & 0.96 \\
11654 & 101455 & 2006 & 161911.90 & 0.11 & 15567415.00 & 125436405.02 & 1.04 & 0.77 & 0.81 \\
4021 & 100538 & 2006 & 1169.70 & 0.15 & 119852.00 & 1103026.53 & 0.98 & 0.94 & 0.92 \\
2695 & 100352 & 2006 & 3387.30 & 0.14 & 330991.00 & 3285859.76 & 1.02 & 0.97 & 0.99 \\
7052 & 100992 & 2006 & 609.70 & 0.08 & 60931.00 & 598106.94 & 1.00 & 0.98 & 0.98 \\
27759 & 105321 & 2006 & 1033.70 & 0.12 & 95137.00 & 972968.00 & 1.09 & 0.94 & 1.02 \\
22575 & 103021 & 2006 & 342.00 & 0.21 & 29477.00 & 294846.75 & 1.16 & 0.86 & 1.00 \\
14282 & 101842 & 2006 & 954.60 & 0.16 & 97008.00 & 976335.96 & 0.98 & 1.02 & 1.01 \\
22988 & 103100 & 2006 & 229.40 & 0.17 & 29514.00 & 227454.04 & 0.78 & 0.99 & 0.77 \\
17737 & 102350 & 2006 & 194.70 & 0.08 & 19321.00 & 192292.78 & 1.01 & 0.99 & 1.00 \\
29738 & 105643 & 2006 & 1000.80 & 0.20 & 92270.00 & 984839.20 & 1.08 & 0.98 & 1.07 \\
5877 & 100809 & 2006 & 1454.50 & 0.16 & 136621.00 & 1485604.38 & 1.06 & 1.02 & 1.09 \\
27867 & 105335 & 2006 & 233.50 & 0.10 & 25435.00 & 255143.51 & 0.92 & 1.09 & 1.00 \\
14249 & 101835 & 2006 & 1490.30 & 0.16 & 152408.00 & 1471000.95 & 0.98 & 0.99 & 0.97 \\
11720 & 101457 & 2006 & 264.70 & 0.13 & 24611.00 & 257260.28 & 1.08 & 0.97 & 1.05 \\
22867 & 103073 & 2006 & 365.10 & 0.18 & 35955.00 & 376877.00 & 1.02 & 1.03 & 1.05 \\
22898 & 103084 & 2006 & 589.00 & 0.19 & 55862.00 & 574984.40 & 1.05 & 0.98 & 1.03 \\
8765 & 101097 & 2006 & 513.90 & 0.27 & 62727.00 & 499074.72 & 0.82 & 0.97 & 0.80 \\
2618 & 100347 & 2006 & 725.00 & 0.15 & 72733.00 & 716754.87 & 1.00 & 0.99 & 0.99 \\
17714 & 102349 & 2006 & 594.80 & 0.10 & 59463.00 & 588010.53 & 1.00 & 0.99 & 0.99 \\
27830 & 105332 & 2006 & 61.20 & 0.06 & 6183.00 & 56141.77 & 0.99 & 0.92 & 0.91 \\
17683 & 102342 & 2006 & 140.00 & 0.09 & 14238.00 & 135218.25 & 0.98 & 0.97 & 0.95 \\
10389 & 101284 & 2006 & 2849.50 & 0.05 & 281845.00 & 2683050.53 & 1.01 & 0.94 & 0.95 \\
4310 & 100603 & 2006 & 3601.90 & 0.37 & 324094.00 & 3417831.89 & 1.11 & 0.95 & 1.05 \\
31423 & 105882 & 2006 & 570.80 & 0.08 & 45519.00 & 482970.60 & 1.25 & 0.85 & 1.06 \\
22954 & 103090 & 2006 & 850.60 & 0.18 & 84578.00 & 727127.72 & 1.01 & 0.85 & 0.86 \\
22969 & 103099 & 2006 & 77.70 & 0.16 & 8080.00 & 79703.24 & 0.96 & 1.03 & 0.99 \\
6946 & 100973 & 2006 & 61.90 & 0.18 & 5747.00 & 58491.10 & 1.08 & 0.94 & 1.02 \\
22534 & 103017 & 2006 & 2202.80 & 0.16 & 271216.00 & 2640704.53 & 0.81 & 1.20 & 0.97 \\
22507 & 103016 & 2006 & 131.00 & 0.11 & 17068.00 & 135542.65 & 0.77 & 1.03 & 0.79 \\
31234 & 105869 & 2006 & 1624.00 & 0.01 & 160805.00 & 1659889.14 & 1.01 & 1.02 & 1.03 \\
28343 & 105416 & 2006 & 1885.20 & 0.12 & 197474.00 & 2046433.30 & 0.95 & 1.09 & 1.04 \\
28329 & 105412 & 2006 & 169.20 & 0.10 & 17121.00 & 170321.22 & 0.99 & 1.01 & 0.99 \\
18136 & 102404 & 2006 & 2150.70 & 0.15 & 214820.00 & 2126136.59 & 1.00 & 0.99 & 0.99 \\
11996 & 101476 & 2006 & 2743.30 & 0.10 & 260707.00 & 2709061.75 & 1.05 & 0.99 & 1.04 \\
22148 & 102993 & 2006 & 11923.60 & 0.15 & 1107171.00 & 9898081.41 & 1.08 & 0.83 & 0.89 \\
28312 & 105401 & 2006 & 1324.80 & 0.04 & 124448.00 & 1257645.10 & 1.06 & 0.95 & 1.01 \\
2266 & 100303 & 2006 & 433.50 & 0.10 & 42911.00 & 424386.39 & 1.01 & 0.98 & 0.99 \\
13984 & 101794 & 2006 & 906.50 & 0.05 & 90787.00 & 904148.73 & 1.00 & 1.00 & 1.00 \\
11962 & 101473 & 2006 & 846.30 & 0.16 & 84049.00 & 881173.20 & 1.01 & 1.04 & 1.05 \\
2279 & 100305 & 2006 & 95.10 & 0.04 & 9752.00 & 97585.43 & 0.98 & 1.03 & 1.00 \\
2280 & 100310 & 2006 & 11.20 & 0.09 & 1129.00 & 10122.72 & 0.99 & 0.90 & 0.90 \\
2340 & 100319 & 2006 & 359.10 & 0.16 & 41574.00 & 352757.91 & 0.86 & 0.98 & 0.85 \\
31089 & 105857 & 2006 & 1194.50 & 0.16 & 119295.00 & 1064811.87 & 1.00 & 0.89 & 0.89 \\
22192 & 102994 & 2006 & 149.60 & 0.18 & 13838.00 & 124388.12 & 1.08 & 0.83 & 0.90 \\
10299 & 101278 & 2006 & 617.90 & 0.05 & 49797.00 & 489002.82 & 1.24 & 0.79 & 0.98 \\
22223 & 102996 & 2006 & 454.30 & 0.05 & 45849.00 & 440716.93 & 0.99 & 0.97 & 0.96 \\
18087 & 102396 & 2006 & 1956.40 & 0.12 & 247629.00 & 1895721.28 & 0.79 & 0.97 & 0.77 \\
14028 & 101800 & 2006 & 389.00 & 0.07 & 38930.00 & 386634.09 & 1.00 & 0.99 & 0.99 \\
22114 & 102990 & 2006 & 3814.80 & 0.21 & 383805.00 & 4015326.69 & 0.99 & 1.05 & 1.05 \\
2206 & 100295 & 2006 & 17.90 & 0.07 & 1689.00 & 16942.17 & 1.06 & 0.95 & 1.00 \\
2142 & 100292 & 2006 & 4150.80 & 0.10 & 422859.00 & 4104665.83 & 0.98 & 0.99 & 0.97 \\
12131 & 101511 & 2006 & 717.90 & 0.11 & 70722.00 & 707221.00 & 1.02 & 0.99 & 1.00 \\
30922 & 105836 & 2006 & 219.80 & 0.04 & 22119.00 & 218780.95 & 0.99 & 1.00 & 0.99 \\
30966 & 105842 & 2006 & 867.00 & 0.07 & 89959.00 & 881583.72 & 0.96 & 1.02 & 0.98 \\
12108 & 101503 & 2006 & 125.20 & 0.07 & 16157.00 & 125290.44 & 0.77 & 1.00 & 0.78 \\
22053 & 102988 & 2006 & 96.50 & 0.15 & 9263.00 & 93336.70 & 1.04 & 0.97 & 1.01 \\
31004 & 105846 & 2006 & 1112.90 & 0.11 & 118397.00 & 1117711.69 & 0.94 & 1.00 & 0.94 \\
6556 & 100890 & 2006 & 3938.80 & 0.13 & 466678.00 & 3870935.58 & 0.84 & 0.98 & 0.83 \\
18186 & 102414 & 2006 & 7910.50 & 0.07 & 801098.00 & 7409613.15 & 0.99 & 0.94 & 0.92 \\
28381 & 105420 & 2006 & 39.00 & -0.06 & 3974.00 & 38905.06 & 0.98 & 1.00 & 0.98 \\
13942 & 101788 & 2006 & 424.70 & 0.15 & 46221.00 & 418548.46 & 0.92 & 0.99 & 0.91 \\
12088 & 101497 & 2006 & 937.30 & 0.14 & 92976.00 & 888656.46 & 1.01 & 0.95 & 0.96 \\
22081 & 102989 & 2006 & 1875.00 & 0.15 & 191734.00 & 1918220.04 & 0.98 & 1.02 & 1.00 \\
13961 & 101789 & 2006 & 228.10 & 0.16 & 25573.00 & 231341.44 & 0.89 & 1.01 & 0.90 \\
28355 & 105419 & 2006 & 49.70 & 0.18 & 4865.00 & 49372.51 & 1.02 & 0.99 & 1.01 \\
2173 & 100293 & 2006 & 43.30 & 0.10 & 4471.00 & 44692.92 & 0.97 & 1.03 & 1.00 \\
31046 & 105852 & 2006 & 212.80 & 0.17 & 21333.00 & 198992.81 & 1.00 & 0.94 & 0.93 \\
22251 & 102997 & 2006 & 3752.50 & 0.21 & 381436.00 & 3378470.55 & 0.98 & 0.90 & 0.89 \\
28254 & 105399 & 2006 & 133.30 & 0.22 & 12133.00 & 128278.27 & 1.10 & 0.96 & 1.06 \\
28144 & 105384 & 2006 & 78.00 & 0.20 & 7798.00 & 74817.00 & 1.00 & 0.96 & 0.96 \\
11850 & 101463 & 2006 & 1673.10 & 0.13 & 166610.00 & 1633896.73 & 1.00 & 0.98 & 0.98 \\
28130 & 105383 & 2006 & 79.10 & 0.02 & 10022.00 & 96877.03 & 0.79 & 1.22 & 0.97 \\
6693 & 100910 & 2006 & 195.50 & 0.15 & 20818.00 & 189278.60 & 0.94 & 0.97 & 0.91 \\
6707 & 100913 & 2006 & 571.00 & 0.22 & 70363.00 & 583540.06 & 0.81 & 1.02 & 0.83 \\
31195 & 105866 & 2006 & 8133.40 & 0.16 & 809034.00 & 8155379.78 & 1.01 & 1.00 & 1.01 \\
31167 & 105865 & 2006 & 331.00 & 0.20 & 32335.00 & 323469.72 & 1.02 & 0.98 & 1.00 \\
4182 & 100567 & 2006 & 475.20 & 0.01 & 58272.00 & 485864.11 & 0.82 & 1.02 & 0.83 \\
2447 & 100330 & 2006 & 3579.10 & 0.15 & 337027.00 & 3294325.02 & 1.06 & 0.92 & 0.98 \\
22474 & 103014 & 2006 & 243.10 & 0.17 & 25365.00 & 237662.24 & 0.96 & 0.98 & 0.94 \\
28101 & 105382 & 2006 & 761.50 & 0.02 & 72995.00 & 748232.89 & 1.04 & 0.98 & 1.03 \\
22492 & 103015 & 2006 & 122.70 & 0.08 & 12329.00 & 121516.41 & 1.00 & 0.99 & 0.99 \\
6797 & 100954 & 2006 & 1229.80 & 0.12 & 122713.00 & 1215583.45 & 1.00 & 0.99 & 0.99 \\
28082 & 105379 & 2006 & 336.60 & 0.10 & 34322.00 & 346128.86 & 0.98 & 1.03 & 1.01 \\
6841 & 100962 & 2006 & 11935.10 & 0.10 & 1197265.00 & 11275823.15 & 1.00 & 0.94 & 0.94 \\
6775 & 100953 & 2006 & 262.10 & -0.00 & 27745.00 & 276258.40 & 0.94 & 1.05 & 1.00 \\
11911 & 101465 & 2006 & 91.10 & 0.09 & 8793.00 & 90478.56 & 1.04 & 0.99 & 1.03 \\
31144 & 105861 & 2006 & 331.10 & 0.19 & 35302.00 & 329234.82 & 0.94 & 0.99 & 0.93 \\
5909 & 100811 & 2006 & 776.30 & 0.14 & 82950.00 & 772677.39 & 0.94 & 1.00 & 0.93 \\
28241 & 105397 & 2006 & 84.70 & 0.33 & 8391.00 & 82963.15 & 1.01 & 0.98 & 0.99 \\
31116 & 105860 & 2006 & 659.20 & 0.09 & 65600.00 & 657263.62 & 1.00 & 1.00 & 1.00 \\
22282 & 102999 & 2006 & 79.50 & 0.20 & 8067.00 & 78077.66 & 0.99 & 0.98 & 0.97 \\
10328 & 101279 & 2006 & 558.60 & 0.04 & 48843.00 & 488431.50 & 1.14 & 0.87 & 1.00 \\
22303 & 103005 & 2006 & 1518.70 & 0.11 & 192459.00 & 1452051.03 & 0.79 & 0.96 & 0.75 \\
18049 & 102387 & 2006 & 659.80 & 0.02 & 66074.00 & 660754.09 & 1.00 & 1.00 & 1.00 \\
28214 & 105393 & 2006 & 4065.90 & 0.19 & 467391.00 & 3899358.48 & 0.87 & 0.96 & 0.83 \\
6643 & 100906 & 2006 & 1237.60 & 0.12 & 122744.00 & 1188662.28 & 1.01 & 0.96 & 0.97 \\
18014 & 102386 & 2006 & 6063.10 & 0.13 & 606456.00 & 6061627.51 & 1.00 & 1.00 & 1.00 \\
2372 & 100320 & 2006 & 69.30 & 0.20 & 7777.00 & 68523.75 & 0.89 & 0.99 & 0.88 \\
4136 & 100559 & 2006 & 17.30 & 0.12 & 1630.00 & 17130.23 & 1.06 & 0.99 & 1.05 \\
28176 & 105390 & 2006 & 421.40 & 0.17 & 42264.00 & 410177.76 & 1.00 & 0.97 & 0.97 \\
22339 & 103007 & 2006 & 1176.40 & 0.15 & 120348.00 & 1154684.17 & 0.98 & 0.98 & 0.96 \\
2404 & 100322 & 2006 & 294.40 & 0.15 & 31029.00 & 299223.38 & 0.95 & 1.02 & 0.96 \\
11881 & 101464 & 2006 & 2076.30 & 0.17 & 219494.00 & 1988019.00 & 0.95 & 0.96 & 0.91 \\
6676 & 100908 & 2006 & 402.30 & 0.06 & 43645.00 & 386745.61 & 0.92 & 0.96 & 0.89 \\
22383 & 103008 & 2006 & 164.90 & 0.04 & 20296.00 & 167205.13 & 0.81 & 1.01 & 0.82 \\
4117 & 100552 & 2006 & 6.30 & 0.22 & 629.00 & 6285.63 & 1.00 & 1.00 & 1.00 \\
27716 & 105317 & 2006 & 153.10 & 0.24 & 13391.00 & 128158.09 & 1.14 & 0.84 & 0.96 \\
17555 & 102318 & 2006 & 2338.90 & 0.09 & 227259.00 & 2153671.89 & 1.03 & 0.92 & 0.95 \\
23802 & 103213 & 2006 & 404.50 & 0.04 & 41102.00 & 408182.54 & 0.98 & 1.01 & 0.99 \\
3060 & 100401 & 2006 & 1305.20 & 0.19 & 130368.00 & 1278668.86 & 1.00 & 0.98 & 0.98 \\
16974 & 102224 & 2006 & 13663.50 & 0.16 & 1381613.00 & 13788411.11 & 0.99 & 1.01 & 1.00 \\
23832 & 103214 & 2006 & 1355.50 & 0.08 & 136585.00 & 1356562.83 & 0.99 & 1.00 & 0.99 \\
14573 & 101885 & 2006 & 1146.20 & 0.22 & 114045.00 & 959783.96 & 1.01 & 0.84 & 0.84 \\
31907 & 105957 & 2006 & 6.20 & 0.00 & 627.00 & 6262.73 & 0.99 & 1.01 & 1.00 \\
31932 & 105961 & 2006 & 34.60 & 0.06 & 3765.00 & 38699.70 & 0.92 & 1.12 & 1.03 \\
31943 & 105963 & 2006 & 839.40 & 0.13 & 83153.00 & 754273.83 & 1.01 & 0.90 & 0.91 \\
8573 & 101090 & 2006 & 471.40 & 0.09 & 53394.00 & 389475.07 & 0.88 & 0.83 & 0.73 \\
27225 & 105256 & 2006 & 721.10 & 0.21 & 65582.00 & 697717.93 & 1.10 & 0.97 & 1.06 \\
11332 & 101394 & 2006 & 23.30 & 0.11 & 2382.00 & 23569.55 & 0.98 & 1.01 & 0.99 \\
11322 & 101393 & 2006 & 445.00 & 0.15 & 44030.00 & 424047.29 & 1.01 & 0.95 & 0.96 \\
8536 & 101089 & 2006 & 210.30 & 0.12 & 23100.00 & 204456.99 & 0.91 & 0.97 & 0.89 \\
23865 & 103224 & 2006 & 88.30 & 0.12 & 8504.00 & 74784.60 & 1.04 & 0.85 & 0.88 \\
23883 & 103226 & 2006 & 70.80 & 0.05 & 7257.00 & 73615.93 & 0.98 & 1.04 & 1.01 \\
23901 & 103228 & 2006 & 70.50 & 0.15 & 6848.00 & 62146.91 & 1.03 & 0.88 & 0.91 \\
11378 & 101399 & 2006 & 85.60 & 0.13 & 10271.00 & 87696.21 & 0.83 & 1.02 & 0.85 \\
23917 & 103232 & 2006 & 114.80 & 0.14 & 11666.00 & 120320.03 & 0.98 & 1.05 & 1.03 \\
27252 & 105259 & 2006 & 649.30 & 0.16 & 64861.00 & 641659.68 & 1.00 & 0.99 & 0.99 \\
869 & 100099 & 2006 & 365.60 & 0.17 & 31327.00 & 299313.42 & 1.17 & 0.82 & 0.96 \\
31758 & 105933 & 2006 & 1597.10 & 0.06 & 160130.00 & 1466013.64 & 1.00 & 0.92 & 0.92 \\
17046 & 102231 & 2006 & 431.90 & 0.09 & 43293.00 & 421502.08 & 1.00 & 0.98 & 0.97 \\
4534 & 100637 & 2006 & 1512.60 & 0.11 & 153082.00 & 1456426.13 & 0.99 & 0.96 & 0.95 \\
17010 & 102230 & 2006 & 21.30 & 0.11 & 2107.00 & 21828.29 & 1.01 & 1.02 & 1.04 \\
23766 & 103212 & 2006 & 2052.10 & 0.12 & 205842.00 & 1826930.11 & 1.00 & 0.89 & 0.89 \\
27319 & 105268 & 2006 & 702.90 & 0.20 & 79930.00 & 695228.53 & 0.88 & 0.99 & 0.87 \\
31769 & 105935 & 2006 & 571.60 & 0.17 & 61593.00 & 548833.51 & 0.93 & 0.96 & 0.89 \\
11406 & 101400 & 2006 & 237.50 & 0.16 & 29214.00 & 239479.64 & 0.81 & 1.01 & 0.82 \\
5790 & 100792 & 2006 & 1913.90 & 0.23 & 185210.00 & 1678983.88 & 1.03 & 0.88 & 0.91 \\
14519 & 101871 & 2006 & 401.20 & 0.20 & 36816.00 & 402469.24 & 1.09 & 1.00 & 1.09 \\
27281 & 105260 & 2006 & 627.80 & 0.15 & 59093.00 & 578516.14 & 1.06 & 0.92 & 0.98 \\
7320 & 101020 & 2006 & 8698.60 & 0.14 & 874451.00 & 8224848.23 & 0.99 & 0.95 & 0.94 \\
3034 & 100399 & 2006 & 413.30 & 0.16 & 41202.00 & 406779.80 & 1.00 & 0.98 & 0.99 \\
31820 & 105943 & 2006 & 32.50 & 0.21 & 3243.00 & 31677.58 & 1.00 & 0.97 & 0.98 \\
31837 & 105946 & 2006 & 120.10 & 0.15 & 12373.00 & 121036.02 & 0.97 & 1.01 & 0.98 \\
31848 & 105947 & 2006 & 31.50 & 0.14 & 3294.00 & 31507.76 & 0.96 & 1.00 & 0.96 \\
31859 & 105948 & 2006 & 64.40 & 0.16 & 6252.00 & 57603.73 & 1.03 & 0.89 & 0.92 \\
31870 & 105949 & 2006 & 1576.50 & 0.08 & 157921.00 & 1499236.28 & 1.00 & 0.95 & 0.95 \\
7284 & 101018 & 2006 & 20649.90 & 0.06 & 2053535.00 & 20129205.93 & 1.01 & 0.97 & 0.98 \\
27344 & 105269 & 2006 & 687.00 & 0.13 & 87875.00 & 705683.61 & 0.78 & 1.03 & 0.80 \\
23937 & 103242 & 2006 & 761.30 & 0.15 & 76075.00 & 760746.98 & 1.00 & 1.00 & 1.00 \\
16795 & 102192 & 2006 & 2882.70 & 0.04 & 290397.00 & 2900242.75 & 0.99 & 1.01 & 1.00 \\
11225 & 101376 & 2006 & 3330.10 & 0.10 & 329329.00 & 3267630.86 & 1.01 & 0.98 & 0.99 \\
32110 & 105983 & 2006 & 1298.10 & 0.16 & 113279.00 & 1144834.44 & 1.15 & 0.88 & 1.01 \\
24120 & 103267 & 2006 & 3536.00 & -0.05 & 356772.00 & 3571026.95 & 0.99 & 1.01 & 1.00 \\
16781 & 102191 & 2006 & 70.30 & 0.14 & 6723.00 & 65891.93 & 1.05 & 0.94 & 0.98 \\
14713 & 101911 & 2006 & 3447.00 & 0.22 & 348767.00 & 2826752.62 & 0.99 & 0.82 & 0.81 \\
27075 & 103647 & 2006 & 36.40 & 0.11 & 3901.00 & 39380.96 & 0.93 & 1.08 & 1.01 \\
8435 & 101086 & 2006 & 348.40 & 0.02 & 38796.00 & 332854.93 & 0.90 & 0.96 & 0.86 \\
32137 & 105984 & 2006 & 128.10 & 0.01 & 14337.00 & 134155.68 & 0.89 & 1.05 & 0.94 \\
32157 & 105990 & 2006 & 204.50 & 0.14 & 24381.00 & 237318.21 & 0.84 & 1.16 & 0.97 \\
16748 & 102183 & 2006 & 1892.40 & 0.12 & 179721.00 & 1949095.13 & 1.05 & 1.03 & 1.08 \\
24170 & 103294 & 2006 & 2426.50 & 0.12 & 241715.00 & 2292493.58 & 1.00 & 0.94 & 0.95 \\
26993 & 103643 & 2006 & 63.30 & 0.12 & 6198.00 & 64948.76 & 1.02 & 1.03 & 1.05 \\
7436 & 101039 & 2006 & 5571.60 & 0.13 & 553018.00 & 4972361.31 & 1.01 & 0.89 & 0.90 \\
8500 & 101088 & 2006 & 2044.70 & 0.00 & 241510.00 & 1894830.51 & 0.85 & 0.93 & 0.78 \\
32078 & 105980 & 2006 & 553.00 & 0.12 & 56661.00 & 544236.09 & 0.98 & 0.98 & 0.96 \\
24061 & 103259 & 2006 & 3197.60 & 0.18 & 346019.00 & 3050464.48 & 0.92 & 0.95 & 0.88 \\
10573 & 101299 & 2006 & 2566.30 & 0.14 & 256477.00 & 2507196.85 & 1.00 & 0.98 & 0.98 \\
7354 & 101023 & 2006 & 30628.90 & 0.20 & 2914448.00 & 28324689.87 & 1.05 & 0.92 & 0.97 \\
23967 & 103251 & 2006 & 301.60 & 0.09 & 29054.00 & 303436.86 & 1.04 & 1.01 & 1.04 \\
3090 & 100408 & 2006 & 120.90 & 0.08 & 13105.00 & 124485.38 & 0.92 & 1.03 & 0.95 \\
16883 & 102213 & 2006 & 508.90 & 0.09 & 51214.00 & 515194.07 & 0.99 & 1.01 & 1.01 \\
23988 & 103252 & 2006 & 393.20 & 0.19 & 37528.00 & 391884.15 & 1.05 & 1.00 & 1.04 \\
16853 & 102197 & 2006 & 1568.20 & 0.16 & 164307.00 & 1494411.12 & 0.95 & 0.95 & 0.91 \\
8468 & 101087 & 2006 & 1242.00 & 0.21 & 139526.00 & 1031587.50 & 0.89 & 0.83 & 0.74 \\
16824 & 102193 & 2006 & 138.40 & 0.07 & 13514.00 & 135137.77 & 1.02 & 0.98 & 1.00 \\
32008 & 105973 & 2006 & 171.60 & 0.12 & 17145.00 & 170688.36 & 1.00 & 0.99 & 1.00 \\
27138 & 105246 & 2006 & 6112.60 & 0.13 & 610962.00 & 5809429.44 & 1.00 & 0.95 & 0.95 \\
11290 & 101390 & 2006 & 4537.90 & 0.17 & 434531.00 & 4448775.49 & 1.04 & 0.98 & 1.02 \\
14679 & 101908 & 2006 & 197.10 & 0.16 & 19145.00 & 191445.13 & 1.03 & 0.97 & 1.00 \\
24039 & 103255 & 2006 & 136.40 & 0.08 & 13350.00 & 134458.06 & 1.02 & 0.99 & 1.01 \\
4568 & 100639 & 2006 & 618.90 & 0.20 & 56529.00 & 551372.34 & 1.09 & 0.89 & 0.98 \\
7398 & 101038 & 2006 & 3198.10 & 0.09 & 370108.00 & 3632426.58 & 0.86 & 1.14 & 0.98 \\
32050 & 105977 & 2006 & 1585.80 & 0.16 & 157167.00 & 1464281.29 & 1.01 & 0.92 & 0.93 \\
24009 & 103253 & 2006 & 320.70 & 0.18 & 28801.00 & 296293.75 & 1.11 & 0.92 & 1.03 \\
30889 & 105806 & 2006 & 170.40 & 0.16 & 17123.00 & 167305.23 & 1.00 & 0.98 & 0.98 \\
17112 & 102257 & 2006 & 2576.00 & 0.15 & 257244.00 & 2530008.81 & 1.00 & 0.98 & 0.98 \\
11549 & 101427 & 2006 & 123.80 & 0.25 & 12376.00 & 115775.11 & 1.00 & 0.94 & 0.94 \\
23256 & 103152 & 2006 & 3464.60 & 0.15 & 343601.00 & 3362177.30 & 1.01 & 0.97 & 0.98 \\
17390 & 102284 & 2006 & 306.60 & 0.11 & 43778.00 & 353135.57 & 0.70 & 1.15 & 0.81 \\
3978 & 100535 & 2006 & 296.80 & 0.10 & 28410.00 & 271448.00 & 1.04 & 0.91 & 0.96 \\
27612 & 105303 & 2006 & 116.50 & -0.00 & 13839.00 & 119494.83 & 0.84 & 1.03 & 0.86 \\
2758 & 100357 & 2006 & 342.70 & 0.25 & 30159.00 & 303208.76 & 1.14 & 0.88 & 1.01 \\
10468 & 101286 & 2006 & 1508.40 & 0.08 & 149865.00 & 1339596.46 & 1.01 & 0.89 & 0.89 \\
27641 & 105306 & 2006 & 24.30 & 0.14 & 2961.00 & 28522.03 & 0.82 & 1.17 & 0.96 \\
27595 & 105295 & 2006 & 445.40 & 0.12 & 53027.00 & 468391.51 & 0.84 & 1.05 & 0.88 \\
23389 & 103174 & 2006 & 1310.10 & 0.10 & 160139.00 & 1379273.31 & 0.82 & 1.05 & 0.86 \\
10484 & 101287 & 2006 & 1003.20 & 0.16 & 99218.00 & 924704.95 & 1.01 & 0.92 & 0.93 \\
7152 & 100998 & 2006 & 108.90 & 0.17 & 10885.00 & 99353.49 & 1.00 & 0.91 & 0.91 \\
17346 & 102282 & 2006 & 845.30 & 0.14 & 84417.00 & 757607.56 & 1.00 & 0.90 & 0.90 \\
23315 & 103160 & 2006 & 278.40 & 0.09 & 27443.00 & 284323.08 & 1.01 & 1.02 & 1.04 \\
23419 & 103175 & 2006 & 961.10 & 0.22 & 102362.00 & 916157.88 & 0.94 & 0.95 & 0.90 \\
31567 & 105909 & 2006 & 65.00 & 0.12 & 8062.00 & 63896.30 & 0.81 & 0.98 & 0.79 \\
7120 & 100997 & 2006 & 1215.20 & 0.10 & 121577.00 & 1063426.01 & 1.00 & 0.88 & 0.87 \\
23108 & 103122 & 2006 & 199.00 & 0.21 & 19294.00 & 192377.53 & 1.03 & 0.97 & 1.00 \\
14317 & 101849 & 2006 & 16.60 & 0.05 & 1715.00 & 17373.92 & 0.97 & 1.05 & 1.01 \\
11606 & 101431 & 2006 & 289.60 & 0.15 & 38544.00 & 286535.74 & 0.75 & 0.99 & 0.74 \\
14335 & 101850 & 2006 & 832.90 & 0.11 & 83366.00 & 803990.62 & 1.00 & 0.97 & 0.96 \\
5830 & 100804 & 2006 & 4397.70 & 0.16 & 435040.00 & 4408835.11 & 1.01 & 1.00 & 1.01 \\
2727 & 100355 & 2006 & 4807.10 & 0.17 & 504031.00 & 4900885.48 & 0.95 & 1.02 & 0.97 \\
4406 & 100622 & 2006 & 902.80 & 0.19 & 90211.00 & 900980.79 & 1.00 & 1.00 & 1.00 \\
23223 & 103145 & 2006 & 124.30 & 0.16 & 12374.00 & 121742.86 & 1.00 & 0.98 & 0.98 \\
17512 & 102317 & 2006 & 35.30 & 0.25 & 3117.00 & 33023.61 & 1.13 & 0.94 & 1.06 \\
7089 & 100996 & 2006 & 1351.40 & 0.15 & 134775.00 & 1320702.89 & 1.00 & 0.98 & 0.98 \\
27670 & 105309 & 2006 & 4285.40 & 0.08 & 464825.00 & 3930473.09 & 0.92 & 0.92 & 0.85 \\
23148 & 103134 & 2006 & 383.30 & 0.15 & 38074.00 & 384693.76 & 1.01 & 1.00 & 1.01 \\
11566 & 101430 & 2006 & 181.20 & 0.22 & 20742.00 & 175837.80 & 0.87 & 0.97 & 0.85 \\
23193 & 103144 & 2006 & 20.40 & 0.14 & 2014.00 & 19359.19 & 1.01 & 0.95 & 0.96 \\
17443 & 102306 & 2006 & 43669.40 & 0.15 & 4119802.00 & 35881704.17 & 1.06 & 0.82 & 0.87 \\
14360 & 101851 & 2006 & 3694.40 & 0.08 & 369872.00 & 3647560.18 & 1.00 & 0.99 & 0.99 \\
23730 & 103209 & 2006 & 294.30 & 0.13 & 28274.00 & 275552.52 & 1.04 & 0.94 & 0.97 \\
2835 & 100362 & 2006 & 150.00 & 0.07 & 15067.00 & 144690.16 & 1.00 & 0.96 & 0.96 \\
4495 & 100635 & 2006 & 656.00 & 0.28 & 65316.00 & 652567.52 & 1.00 & 0.99 & 1.00 \\
27436 & 105278 & 2006 & 435.80 & 0.01 & 52853.00 & 451120.72 & 0.82 & 1.04 & 0.85 \\
3924 & 100514 & 2006 & 43.50 & 0.15 & 4361.00 & 43452.55 & 1.00 & 1.00 & 1.00 \\
23663 & 103205 & 2006 & 736.30 & 0.14 & 63580.00 & 599856.55 & 1.16 & 0.81 & 0.94 \\
31704 & 105931 & 2006 & 9431.00 & 0.20 & 943111.00 & 8891136.13 & 1.00 & 0.94 & 0.94 \\
17231 & 102271 & 2006 & 743.50 & 0.08 & 75325.00 & 781825.44 & 0.99 & 1.05 & 1.04 \\
11440 & 101402 & 2006 & 146.80 & 0.16 & 14712.00 & 138894.94 & 1.00 & 0.95 & 0.94 \\
27407 & 105276 & 2006 & 6427.30 & 0.20 & 649119.00 & 5862466.18 & 0.99 & 0.91 & 0.90 \\
8636 & 101092 & 2006 & 273.00 & 0.12 & 29776.00 & 266299.60 & 0.92 & 0.98 & 0.89 \\
23632 & 103204 & 2006 & 214.40 & 0.18 & 20916.00 & 213820.98 & 1.03 & 1.00 & 1.02 \\
17197 & 102270 & 2006 & 528.10 & 0.07 & 67523.00 & 549756.84 & 0.78 & 1.04 & 0.81 \\
23696 & 103208 & 2006 & 1116.40 & 0.15 & 110844.00 & 981461.95 & 1.01 & 0.88 & 0.89 \\
7246 & 101015 & 2006 & 1770.30 & 0.16 & 179707.00 & 1700935.42 & 0.99 & 0.96 & 0.95 \\
14475 & 101861 & 2006 & 1474.40 & 0.17 & 140648.00 & 1395732.15 & 1.05 & 0.95 & 0.99 \\
17168 & 102261 & 2006 & 4249.90 & 0.17 & 425489.00 & 4020878.14 & 1.00 & 0.95 & 0.95 \\
3000 & 100395 & 2006 & 679.10 & 0.21 & 65989.00 & 702883.97 & 1.03 & 1.04 & 1.07 \\
31731 & 105932 & 2006 & 263.40 & 0.13 & 26252.00 & 256824.43 & 1.00 & 0.98 & 0.98 \\
27377 & 105275 & 2006 & 227.90 & 0.15 & 19888.00 & 210729.71 & 1.15 & 0.92 & 1.06 \\
72 & 100004 & 2006 & 1796.80 & 0.08 & 178867.00 & 1816771.08 & 1.00 & 1.01 & 1.02 \\
14454 & 101858 & 2006 & 1477.50 & 0.10 & 146474.00 & 1409390.40 & 1.01 & 0.95 & 0.96 \\
11512 & 101425 & 2006 & 22.00 & 0.11 & 2179.00 & 18977.85 & 1.01 & 0.86 & 0.87 \\
17258 & 102274 & 2006 & 10397.00 & 0.07 & 1037276.00 & 9833887.64 & 1.00 & 0.95 & 0.95 \\
31685 & 105930 & 2006 & 165.40 & 0.06 & 16481.00 & 161498.62 & 1.00 & 0.98 & 0.98 \\
31617 & 105917 & 2006 & 221.00 & 0.20 & 22136.00 & 220845.03 & 1.00 & 1.00 & 1.00 \\
31627 & 105918 & 2006 & 282.40 & 0.00 & 33467.00 & 295604.84 & 0.84 & 1.05 & 0.88 \\
27548 & 105287 & 2006 & 115.30 & 0.14 & 14061.00 & 109771.65 & 0.82 & 0.95 & 0.78 \\
23453 & 103177 & 2006 & 358.00 & 0.10 & 45266.00 & 405751.25 & 0.79 & 1.13 & 0.90 \\
31651 & 105920 & 2006 & 2981.50 & 0.01 & 297116.00 & 2943880.48 & 1.00 & 0.99 & 0.99 \\
23578 & 103193 & 2006 & 99.60 & 0.01 & 10870.00 & 102193.02 & 0.92 & 1.03 & 0.94 \\
14410 & 101854 & 2006 & 20784.40 & 0.06 & 2078871.00 & 20551357.83 & 1.00 & 0.99 & 0.99 \\
27481 & 105281 & 2006 & 862.30 & 0.34 & 77561.00 & 689971.00 & 1.11 & 0.80 & 0.89 \\
2921 & 100379 & 2006 & 753.50 & 0.19 & 76415.00 & 750478.81 & 0.99 & 1.00 & 0.98 \\
17316 & 102280 & 2006 & 3967.50 & 0.13 & 410036.00 & 4008586.98 & 0.97 & 1.01 & 0.98 \\
17298 & 102278 & 2006 & 132.40 & 0.19 & 13726.00 & 127709.47 & 0.96 & 0.96 & 0.93 \\
7207 & 101013 & 2006 & 2894.70 & 0.05 & 306994.00 & 2876020.65 & 0.94 & 0.99 & 0.94 \\
11463 & 101414 & 2006 & 49.90 & 0.14 & 5028.00 & 44550.62 & 0.99 & 0.89 & 0.89 \\
2960 & 100389 & 2006 & 975.10 & 0.09 & 99356.00 & 962677.09 & 0.98 & 0.99 & 0.97 \\
7171 & 101000 & 2006 & 729.10 & 0.17 & 72701.00 & 718046.72 & 1.00 & 0.98 & 0.99 \\
28407 & 105421 & 2006 & 17.50 & 0.15 & 1755.00 & 17595.45 & 1.00 & 1.01 & 1.00 \\
22022 & 102987 & 2006 & 1127.90 & 0.11 & 114538.00 & 1153578.16 & 0.98 & 1.02 & 1.01 \\
8974 & 101107 & 2006 & 911.30 & 0.13 & 87392.00 & 841345.16 & 1.04 & 0.92 & 0.96 \\
13520 & 101742 & 2006 & 3887.90 & 0.09 & 449787.00 & 3382501.30 & 0.86 & 0.87 & 0.75 \\
577 & 100076 & 2006 & 366.40 & 0.13 & 35058.00 & 363362.03 & 1.05 & 0.99 & 1.04 \\
30184 & 105705 & 2006 & 84.50 & 0.08 & 8052.00 & 83400.30 & 1.05 & 0.99 & 1.04 \\
20425 & 102737 & 2006 & 1662.50 & 0.15 & 168659.00 & 1608253.97 & 0.99 & 0.97 & 0.95 \\
12888 & 101603 & 2006 & 1237.60 & -0.00 & 136341.00 & 1304247.28 & 0.91 & 1.05 & 0.96 \\
19228 & 102570 & 2006 & 134.00 & -0.01 & 16501.00 & 137262.50 & 0.81 & 1.02 & 0.83 \\
6059 & 100821 & 2006 & 84.00 & 0.19 & 7808.00 & 80189.81 & 1.08 & 0.95 & 1.03 \\
6237 & 100831 & 2006 & 131.60 & 0.19 & 12667.00 & 131712.54 & 1.04 & 1.00 & 1.04 \\
530 & 100072 & 2006 & 7649.00 & 0.06 & 781590.00 & 7415310.01 & 0.98 & 0.97 & 0.95 \\
20455 & 102744 & 2006 & 649.30 & 0.12 & 64948.00 & 642363.54 & 1.00 & 0.99 & 0.99 \\
19185 & 102559 & 2006 & 86.20 & 0.13 & 8598.00 & 77588.80 & 1.00 & 0.90 & 0.90 \\
20477 & 102757 & 2006 & 4697.20 & 0.08 & 470971.00 & 4603126.64 & 1.00 & 0.98 & 0.98 \\
29283 & 105581 & 2006 & 95.80 & 0.19 & 9601.00 & 92164.94 & 1.00 & 0.96 & 0.96 \\
30217 & 105716 & 2006 & 35.60 & 0.11 & 3802.00 & 33895.86 & 0.94 & 0.95 & 0.89 \\
29270 & 105574 & 2006 & 249.60 & 0.15 & 23777.00 & 248485.33 & 1.05 & 1.00 & 1.05 \\
12844 & 101602 & 2006 & 7410.30 & 0.19 & 703354.00 & 7260762.52 & 1.05 & 0.98 & 1.03 \\
29309 & 105585 & 2006 & 207.20 & 0.09 & 17941.00 & 189195.38 & 1.15 & 0.91 & 1.05 \\
30241 & 105720 & 2006 & 541.80 & 0.00 & 53946.00 & 523091.25 & 1.00 & 0.97 & 0.97 \\
9589 & 101151 & 2006 & 355.90 & 0.25 & 31947.00 & 310017.10 & 1.11 & 0.87 & 0.97 \\
20393 & 102733 & 2006 & 4744.50 & 0.27 & 473034.00 & 3820398.71 & 1.00 & 0.81 & 0.81 \\
691 & 100090 & 2006 & 420.60 & 0.04 & 41613.00 & 404860.19 & 1.01 & 0.96 & 0.97 \\
29419 & 105594 & 2006 & 13.20 & 0.15 & 1330.00 & 13292.56 & 0.99 & 1.01 & 1.00 \\
20342 & 102716 & 2006 & 1227.10 & 0.19 & 122819.00 & 1146917.17 & 1.00 & 0.93 & 0.93 \\
26035 & 103535 & 2006 & 3991.60 & 0.14 & 373808.00 & 3938013.38 & 1.07 & 0.99 & 1.05 \\
29379 & 105591 & 2006 & 34.60 & 0.20 & 3161.00 & 28059.32 & 1.09 & 0.81 & 0.89 \\
12975 & 101617 & 2006 & 117.90 & 0.35 & 10448.00 & 97012.91 & 1.13 & 0.82 & 0.93 \\
30152 & 105703 & 2006 & 492.30 & 0.20 & 50731.00 & 402009.18 & 0.97 & 0.82 & 0.79 \\
19339 & 102597 & 2006 & 87.60 & -0.04 & 8695.00 & 80698.64 & 1.01 & 0.92 & 0.93 \\
29335 & 105587 & 2006 & 279.00 & 0.10 & 28559.00 & 280071.22 & 0.98 & 1.00 & 0.98 \\
29367 & 105589 & 2006 & 215.30 & 0.06 & 27055.00 & 208786.44 & 0.80 & 0.97 & 0.77 \\
19288 & 102579 & 2006 & 157.30 & 0.07 & 15842.00 & 156208.97 & 0.99 & 0.99 & 0.99 \\
9894 & 101200 & 2006 & 10.40 & 0.07 & 1124.00 & 11716.87 & 0.93 & 1.13 & 1.04 \\
602 & 100079 & 2006 & 2386.90 & 0.08 & 246035.00 & 1971504.42 & 0.97 & 0.83 & 0.80 \\
12921 & 101606 & 2006 & 3192.60 & 0.11 & 352758.00 & 2978933.33 & 0.91 & 0.93 & 0.84 \\
1278 & 100171 & 2006 & 5188.10 & 0.19 & 484564.00 & 4384551.36 & 1.07 & 0.85 & 0.90 \\
9911 & 101211 & 2006 & 457.50 & 0.14 & 48022.00 & 432310.47 & 0.95 & 0.94 & 0.90 \\
19260 & 102575 & 2006 & 111.30 & 0.12 & 11145.00 & 109116.54 & 1.00 & 0.98 & 0.98 \\
9929 & 101212 & 2006 & 356.60 & 0.21 & 40047.00 & 341313.95 & 0.89 & 0.96 & 0.85 \\
19308 & 102588 & 2006 & 310.20 & 0.17 & 30761.00 & 308755.20 & 1.01 & 1.00 & 1.00 \\
20308 & 102715 & 2006 & 1809.30 & 0.11 & 181360.00 & 1746107.46 & 1.00 & 0.97 & 0.96 \\
20506 & 102760 & 2006 & 851.20 & 0.10 & 85223.00 & 817997.99 & 1.00 & 0.96 & 0.96 \\
19113 & 102549 & 2006 & 581.60 & 0.06 & 57264.00 & 572636.18 & 1.02 & 0.98 & 1.00 \\
1524 & 100209 & 2006 & 4142.50 & 0.02 & 430325.00 & 4071834.96 & 0.96 & 0.98 & 0.95 \\
20716 & 102784 & 2006 & 8241.70 & 0.05 & 828365.00 & 8122495.45 & 0.99 & 0.99 & 0.98 \\
1542 & 100213 & 2006 & 284.80 & 0.21 & 28006.00 & 278470.91 & 1.02 & 0.98 & 0.99 \\
396 & 100048 & 2006 & 93.30 & 0.20 & 9160.00 & 83603.68 & 1.02 & 0.90 & 0.91 \\
9467 & 101137 & 2006 & 36.90 & 0.06 & 3646.00 & 34111.47 & 1.01 & 0.92 & 0.94 \\
20735 & 102788 & 2006 & 527.30 & 0.13 & 53328.00 & 536215.30 & 0.99 & 1.02 & 1.01 \\
423 & 100055 & 2006 & 21200.90 & 0.17 & 1931775.00 & 20043652.53 & 1.10 & 0.95 & 1.04 \\
12616 & 101560 & 2006 & 579.00 & 0.18 & 57995.00 & 572665.27 & 1.00 & 0.99 & 0.99 \\
12602 & 101557 & 2006 & 14.40 & 0.14 & 1439.00 & 13355.58 & 1.00 & 0.93 & 0.93 \\
30329 & 105737 & 2006 & 645.60 & 0.07 & 67468.00 & 633148.55 & 0.96 & 0.98 & 0.94 \\
29154 & 105533 & 2006 & 183.50 & 0.02 & 20093.00 & 197690.21 & 0.91 & 1.08 & 0.98 \\
19053 & 102545 & 2006 & 2008.60 & 0.20 & 198825.00 & 1988253.81 & 1.01 & 0.99 & 1.00 \\
10011 & 101252 & 2006 & 209.30 & 0.08 & 21042.00 & 210420.52 & 0.99 & 1.01 & 1.00 \\
345 & 100040 & 2006 & 493.60 & -0.00 & 48768.00 & 477220.77 & 1.01 & 0.97 & 0.98 \\
20774 & 102789 & 2006 & 953.90 & 0.16 & 92647.00 & 906396.02 & 1.03 & 0.95 & 0.98 \\
328 & 100036 & 2006 & 18.90 & 0.10 & 1926.00 & 18914.31 & 0.98 & 1.00 & 0.98 \\
19085 & 102548 & 2006 & 1261.00 & 0.04 & 124880.00 & 1248801.59 & 1.01 & 0.99 & 1.00 \\
20528 & 102761 & 2006 & 15172.10 & 0.15 & 1558360.00 & 15573280.65 & 0.97 & 1.03 & 1.00 \\
29188 & 105535 & 2006 & 90.30 & 0.05 & 9030.00 & 90068.95 & 1.00 & 1.00 & 1.00 \\
12646 & 101561 & 2006 & 188.00 & 0.12 & 18546.00 & 184031.90 & 1.01 & 0.98 & 0.99 \\
1355 & 100190 & 2006 & 1312.70 & 0.06 & 136141.00 & 1364984.06 & 0.96 & 1.04 & 1.00 \\
468 & 100068 & 2006 & 107.60 & 0.15 & 10762.00 & 104112.89 & 1.00 & 0.97 & 0.97 \\
1374 & 100192 & 2006 & 73.30 & 0.15 & 6937.00 & 69125.84 & 1.06 & 0.94 & 1.00 \\
20565 & 102767 & 2006 & 4107.70 & 0.13 & 410093.00 & 4096409.37 & 1.00 & 1.00 & 1.00 \\
20605 & 102774 & 2006 & 2388.90 & 0.05 & 239565.00 & 2351913.07 & 1.00 & 0.98 & 0.98 \\
13559 & 101743 & 2006 & 9788.90 & 0.12 & 992381.00 & 9287308.77 & 0.99 & 0.95 & 0.94 \\
20625 & 102775 & 2006 & 3842.60 & 0.19 & 383846.00 & 3566708.53 & 1.00 & 0.93 & 0.93 \\
1425 & 100196 & 2006 & 9521.90 & 0.16 & 1023961.00 & 9737536.56 & 0.93 & 1.02 & 0.95 \\
20661 & 102777 & 2006 & 707.60 & 0.13 & 70970.00 & 651359.38 & 1.00 & 0.92 & 0.92 \\
29243 & 105561 & 2006 & 1463.10 & 0.36 & 146921.00 & 1394819.34 & 1.00 & 0.95 & 0.95 \\
12677 & 101562 & 2006 & 609.60 & 0.18 & 58236.00 & 525004.09 & 1.05 & 0.86 & 0.90 \\
19151 & 102551 & 2006 & 4311.90 & 0.18 & 404608.00 & 4046178.50 & 1.07 & 0.94 & 1.00 \\
9551 & 101149 & 2006 & 1039.10 & 0.10 & 106775.00 & 953948.82 & 0.97 & 0.92 & 0.89 \\
1455 & 100200 & 2006 & 1156.60 & 0.37 & 99440.00 & 1037731.98 & 1.16 & 0.90 & 1.04 \\
9963 & 101215 & 2006 & 20.00 & 0.17 & 1664.00 & 16796.56 & 1.20 & 0.84 & 1.01 \\
1486 & 100207 & 2006 & 1439.80 & 0.03 & 140712.00 & 1494509.28 & 1.02 & 1.04 & 1.06 \\
30300 & 105731 & 2006 & 3349.40 & 0.22 & 332686.00 & 2871965.40 & 1.01 & 0.86 & 0.86 \\
29204 & 105536 & 2006 & 127.10 & 0.07 & 12713.00 & 125864.64 & 1.00 & 0.99 & 0.99 \\
30274 & 105723 & 2006 & 973.70 & 0.15 & 97445.00 & 934771.70 & 1.00 & 0.96 & 0.96 \\
30362 & 105741 & 2006 & 219.20 & 0.11 & 22116.00 & 207567.97 & 0.99 & 0.95 & 0.94 \\
19373 & 102599 & 2006 & 2670.60 & 0.12 & 281001.00 & 2690408.12 & 0.95 & 1.01 & 0.96 \\
29828 & 105652 & 2006 & 339.10 & 0.03 & 30264.00 & 279695.99 & 1.12 & 0.82 & 0.92 \\
19805 & 102652 & 2006 & 1666.60 & 0.14 & 159774.00 & 1546038.53 & 1.04 & 0.93 & 0.97 \\
6168 & 100827 & 2006 & 138.30 & 0.05 & 17884.00 & 135589.48 & 0.77 & 0.98 & 0.76 \\
19610 & 102636 & 2006 & 1610.60 & 0.09 & 161120.00 & 1594477.83 & 1.00 & 0.99 & 0.99 \\
19846 & 102653 & 2006 & 19069.50 & 0.14 & 1661133.00 & 15265446.49 & 1.15 & 0.80 & 0.92 \\
29623 & 105627 & 2006 & 878.20 & 0.19 & 94697.00 & 919115.08 & 0.93 & 1.05 & 0.97 \\
29611 & 105623 & 2006 & 76.00 & 0.12 & 7320.00 & 74168.11 & 1.04 & 0.98 & 1.01 \\
16 & 100001 & 2006 & 3442.50 & 0.11 & 341523.00 & 3312811.26 & 1.01 & 0.96 & 0.97 \\
1109 & 100153 & 2006 & 123.90 & 0.18 & 12345.00 & 119179.96 & 1.00 & 0.96 & 0.97 \\
29871 & 105655 & 2006 & 1905.40 & 0.39 & 167540.00 & 1808923.93 & 1.14 & 0.95 & 1.08 \\
13388 & 101736 & 2006 & 37.60 & 0.10 & 4022.00 & 40794.12 & 0.93 & 1.08 & 1.01 \\
9805 & 101193 & 2006 & 797.00 & 0.14 & 74275.00 & 784491.13 & 1.07 & 0.98 & 1.06 \\
9637 & 101160 & 2006 & 1010.90 & 0.14 & 100953.00 & 978589.50 & 1.00 & 0.97 & 0.97 \\
6121 & 100823 & 2006 & 25.10 & 0.20 & 2195.00 & 22323.23 & 1.14 & 0.89 & 1.02 \\
19884 & 102654 & 2006 & 806.40 & 0.07 & 81040.00 & 804026.45 & 1.00 & 1.00 & 0.99 \\
941 & 100112 & 2006 & 4603.40 & 0.17 & 460873.00 & 4409514.16 & 1.00 & 0.96 & 0.96 \\
19515 & 102608 & 2006 & 272.00 & 0.35 & 26709.00 & 260513.22 & 1.02 & 0.96 & 0.98 \\
29652 & 105630 & 2006 & 3.00 & 0.10 & 383.00 & 3943.71 & 0.78 & 1.31 & 1.03 \\
1044 & 100128 & 2006 & 268.90 & 0.24 & 26869.00 & 216077.07 & 1.00 & 0.80 & 0.80 \\
13228 & 101708 & 2006 & 270.90 & 0.16 & 26781.00 & 274223.07 & 1.01 & 1.01 & 1.02 \\
19740 & 102650 & 2006 & 23514.60 & 0.19 & 2354239.00 & 21998288.95 & 1.00 & 0.94 & 0.93 \\
29770 & 105645 & 2006 & 9328.60 & 0.17 & 935830.00 & 9145711.42 & 1.00 & 0.98 & 0.98 \\
9687 & 101165 & 2006 & 2055.10 & 0.15 & 208371.00 & 2006901.91 & 0.99 & 0.98 & 0.96 \\
9668 & 101161 & 2006 & 948.50 & 0.18 & 94789.00 & 906264.33 & 1.00 & 0.96 & 0.96 \\
985 & 100113 & 2006 & 792.20 & 0.01 & 79500.00 & 785322.79 & 1.00 & 0.99 & 0.99 \\
29709 & 105640 & 2006 & 2524.50 & 0.11 & 202246.00 & 1979133.12 & 1.25 & 0.78 & 0.98 \\
9757 & 101186 & 2006 & 505.30 & 0.13 & 48221.00 & 501014.11 & 1.05 & 0.99 & 1.04 \\
19706 & 102649 & 2006 & 871.60 & 0.16 & 81443.00 & 827078.46 & 1.07 & 0.95 & 1.02 \\
1079 & 100150 & 2006 & 120.80 & 0.21 & 11144.00 & 102685.05 & 1.08 & 0.85 & 0.92 \\
19771 & 102651 & 2006 & 7762.00 & 0.15 & 687852.00 & 7057627.88 & 1.13 & 0.91 & 1.03 \\
19642 & 102639 & 2006 & 240.60 & 0.31 & 23876.00 & 238693.93 & 1.01 & 0.99 & 1.00 \\
13288 & 101717 & 2006 & 66.80 & 0.17 & 6663.00 & 67215.58 & 1.00 & 1.01 & 1.01 \\
9775 & 101192 & 2006 & 509.80 & 0.09 & 48420.00 & 507919.88 & 1.05 & 1.00 & 1.05 \\
29688 & 105635 & 2006 & 11.30 & 0.10 & 1138.00 & 11155.26 & 0.99 & 0.99 & 0.98 \\
29799 & 105647 & 2006 & 622.60 & 0.17 & 58589.00 & 534098.34 & 1.06 & 0.86 & 0.91 \\
13055 & 101623 & 2006 & 1271.80 & 0.20 & 119204.00 & 1242095.49 & 1.07 & 0.98 & 1.04 \\
29582 & 105616 & 2006 & 347.40 & 0.15 & 32921.00 & 313002.54 & 1.06 & 0.90 & 0.95 \\
1159 & 100157 & 2006 & 867.10 & 0.15 & 83608.00 & 830437.94 & 1.04 & 0.96 & 0.99 \\
9618 & 101158 & 2006 & 386.10 & 0.06 & 41658.00 & 402361.86 & 0.93 & 1.04 & 0.97 \\
6090 & 100822 & 2006 & 15.80 & 0.11 & 1529.00 & 15481.94 & 1.03 & 0.98 & 1.01 \\
29995 & 105665 & 2006 & 37.90 & 0.10 & 3894.00 & 40199.11 & 0.97 & 1.06 & 1.03 \\
9866 & 101198 & 2006 & 279.20 & 0.21 & 26738.00 & 274913.06 & 1.04 & 0.98 & 1.03 \\
30024 & 105678 & 2006 & 44.40 & 0.16 & 6240.00 & 50376.03 & 0.71 & 1.13 & 0.81 \\
13464 & 101740 & 2006 & 11422.10 & 0.11 & 1137266.00 & 10604346.08 & 1.00 & 0.93 & 0.93 \\
6203 & 100829 & 2006 & 1217.50 & 0.13 & 118007.00 & 1233461.26 & 1.03 & 1.01 & 1.05 \\
29467 & 105597 & 2006 & 227.00 & 0.15 & 24209.00 & 221402.09 & 0.94 & 0.98 & 0.91 \\
29493 & 105598 & 2006 & 343.90 & 0.12 & 34350.00 & 337344.86 & 1.00 & 0.98 & 0.98 \\
20083 & 102665 & 2006 & 18.30 & 0.16 & 1803.00 & 16266.14 & 1.01 & 0.89 & 0.90 \\
13489 & 101741 & 2006 & 6797.20 & 0.08 & 710231.00 & 5653366.40 & 0.96 & 0.83 & 0.80 \\
30052 & 105679 & 2006 & 370.70 & 0.04 & 39511.00 & 393524.89 & 0.94 & 1.06 & 1.00 \\
20116 & 102667 & 2006 & 31503.50 & 0.12 & 3158648.00 & 30327702.36 & 1.00 & 0.96 & 0.96 \\
19407 & 102600 & 2006 & 1237.00 & 0.23 & 119274.00 & 1208085.26 & 1.04 & 0.98 & 1.01 \\
29445 & 105595 & 2006 & 15.90 & 0.02 & 1606.00 & 15779.42 & 0.99 & 0.99 & 0.98 \\
13078 & 101626 & 2006 & 1468.40 & 0.23 & 144711.00 & 1273131.18 & 1.01 & 0.87 & 0.88 \\
715 & 100092 & 2006 & 604.10 & 0.09 & 60353.00 & 595105.00 & 1.00 & 0.99 & 0.99 \\
20160 & 102671 & 2006 & 149.00 & 0.17 & 14925.00 & 148857.85 & 1.00 & 1.00 & 1.00 \\
749 & 100093 & 2006 & 314.70 & 0.18 & 29561.00 & 286674.30 & 1.06 & 0.91 & 0.97 \\
29977 & 105664 & 2006 & 1000.90 & 0.01 & 100137.00 & 973444.30 & 1.00 & 0.97 & 0.97 \\
29949 & 105659 & 2006 & 194.80 & 0.16 & 20683.00 & 189366.72 & 0.94 & 0.97 & 0.92 \\
899 & 100101 & 2006 & 383.80 & 0.18 & 37397.00 & 370608.49 & 1.03 & 0.97 & 0.99 \\
29900 & 105656 & 2006 & 1259.90 & 0.07 & 121099.00 & 1210989.13 & 1.04 & 0.96 & 1.00 \\
19928 & 102655 & 2006 & 4438.50 & 0.17 & 447298.00 & 3992673.54 & 0.99 & 0.90 & 0.89 \\
19498 & 102607 & 2006 & 536.50 & 0.09 & 54084.00 & 532699.10 & 0.99 & 0.99 & 0.98 \\
19956 & 102659 & 2006 & 6992.90 & 0.07 & 698342.00 & 6871415.42 & 1.00 & 0.98 & 0.98 \\
29553 & 105611 & 2006 & 6479.00 & 0.15 & 679508.00 & 5920630.33 & 0.95 & 0.91 & 0.87 \\
13131 & 101668 & 2006 & 66.20 & 0.14 & 6605.00 & 63790.67 & 1.00 & 0.96 & 0.97 \\
19986 & 102660 & 2006 & 9722.10 & 0.10 & 971897.00 & 9547349.29 & 1.00 & 0.98 & 0.98 \\
29961 & 105662 & 2006 & 95.20 & 0.11 & 9440.00 & 98234.59 & 1.01 & 1.03 & 1.04 \\
19475 & 102606 & 2006 & 4458.40 & 0.15 & 428570.00 & 4329196.72 & 1.04 & 0.97 & 1.01 \\
29541 & 105610 & 2006 & 194.30 & 0.06 & 19359.00 & 189313.09 & 1.00 & 0.97 & 0.98 \\
13420 & 101738 & 2006 & 1776.50 & 0.13 & 197004.00 & 1742929.46 & 0.90 & 0.98 & 0.88 \\
20021 & 102663 & 2006 & 3190.40 & 0.07 & 327161.00 & 2632167.00 & 0.98 & 0.83 & 0.80 \\
1229 & 100166 & 2006 & 25097.10 & 0.08 & 2472488.00 & 20938513.93 & 1.02 & 0.83 & 0.85 \\
29934 & 105658 & 2006 & 152.80 & 0.13 & 15551.00 & 157779.64 & 0.98 & 1.03 & 1.01 \\
19441 & 102601 & 2006 & 7902.20 & 0.16 & 782503.00 & 7795119.59 & 1.01 & 0.99 & 1.00 \\
20055 & 102664 & 2006 & 6125.20 & 0.16 & 587545.00 & 5427651.40 & 1.04 & 0.89 & 0.92 \\
9836 & 101194 & 2006 & 400.10 & 0.19 & 36446.00 & 353562.40 & 1.10 & 0.88 & 0.97 \\
30379 & 105746 & 2006 & 315.30 & 0.19 & 29713.00 & 272269.30 & 1.06 & 0.86 & 0.92 \\
37 & 100003 & 2006 & 1135.30 & 0.13 & 111905.00 & 1076755.87 & 1.01 & 0.95 & 0.96 \\
12314 & 101534 & 2006 & 853.70 & 0.17 & 85595.00 & 833789.24 & 1.00 & 0.98 & 0.97 \\
18612 & 102491 & 2006 & 517.60 & 0.16 & 57415.00 & 567129.17 & 0.90 & 1.10 & 0.99 \\
18596 & 102490 & 2006 & 44.30 & 0.05 & 4572.00 & 45845.46 & 0.97 & 1.03 & 1.00 \\
28656 & 105458 & 2006 & 524.10 & 0.09 & 54291.00 & 556809.72 & 0.97 & 1.06 & 1.03 \\
18500 & 102465 & 2006 & 154.20 & 0.05 & 15463.00 & 150250.54 & 1.00 & 0.97 & 0.97 \\
21568 & 102894 & 2006 & 708.20 & 0.06 & 69873.00 & 692371.19 & 1.01 & 0.98 & 0.99 \\
9090 & 101111 & 2006 & 370.30 & 0.13 & 41485.00 & 356595.85 & 0.89 & 0.96 & 0.86 \\
21538 & 102893 & 2006 & 201.70 & 0.21 & 21252.00 & 211412.13 & 0.95 & 1.05 & 0.99 \\
10135 & 101262 & 2006 & 12.80 & 0.14 & 1280.00 & 13286.55 & 1.00 & 1.04 & 1.04 \\
10152 & 101263 & 2006 & 229.80 & 0.13 & 21778.00 & 221787.74 & 1.06 & 0.97 & 1.02 \\
28619 & 105450 & 2006 & 23.90 & 0.18 & 2260.00 & 24831.81 & 1.06 & 1.04 & 1.10 \\
18461 & 102461 & 2006 & 543.50 & 0.07 & 54334.00 & 538001.78 & 1.00 & 0.99 & 0.99 \\
21643 & 102937 & 2006 & 254.30 & 0.21 & 22721.00 & 205562.22 & 1.12 & 0.81 & 0.90 \\
12279 & 101531 & 2006 & 240.00 & 0.14 & 22534.00 & 231232.27 & 1.07 & 0.96 & 1.03 \\
2041 & 100286 & 2006 & 39.80 & 0.17 & 3994.00 & 39244.03 & 1.00 & 0.99 & 0.98 \\
30716 & 105788 & 2006 & 59.10 & 0.05 & 5906.00 & 58831.68 & 1.00 & 1.00 & 1.00 \\
30737 & 105791 & 2006 & 8.30 & 0.24 & 783.00 & 7161.23 & 1.06 & 0.86 & 0.91 \\
6462 & 100875 & 2006 & 1560.50 & 0.12 & 186887.00 & 1491047.93 & 0.83 & 0.96 & 0.80 \\
2013 & 100280 & 2006 & 781.50 & 0.08 & 115654.00 & 626752.04 & 0.68 & 0.80 & 0.54 \\
21367 & 102854 & 2006 & 192.90 & 0.11 & 18606.00 & 191116.47 & 1.04 & 0.99 & 1.03 \\
4069 & 100544 & 2006 & 1474.70 & 0.29 & 139020.00 & 1507611.42 & 1.06 & 1.02 & 1.08 \\
28782 & 105476 & 2006 & 1014.60 & 0.39 & 101356.00 & 996646.02 & 1.00 & 0.98 & 0.98 \\
21400 & 102861 & 2006 & 42.50 & 0.05 & 4406.00 & 44993.29 & 0.96 & 1.06 & 1.02 \\
18701 & 102503 & 2006 & 337.60 & 0.14 & 40680.00 & 337160.80 & 0.83 & 1.00 & 0.83 \\
9155 & 101115 & 2006 & 27264.70 & 0.16 & 2743880.00 & 26476489.82 & 0.99 & 0.97 & 0.96 \\
21428 & 102871 & 2006 & 911.40 & 0.22 & 91203.00 & 747471.76 & 1.00 & 0.82 & 0.82 \\
28710 & 105469 & 2006 & 44.50 & 0.13 & 5786.00 & 46314.66 & 0.77 & 1.04 & 0.80 \\
18661 & 102500 & 2006 & 34.30 & 0.07 & 3841.00 & 34330.85 & 0.89 & 1.00 & 0.89 \\
30689 & 105783 & 2006 & 11447.00 & 0.35 & 1143327.00 & 11211811.01 & 1.00 & 0.98 & 0.98 \\
21459 & 102872 & 2006 & 4528.00 & 0.09 & 405508.00 & 4055681.47 & 1.12 & 0.90 & 1.00 \\
9122 & 101112 & 2006 & 847.70 & 0.13 & 85600.00 & 803047.18 & 0.99 & 0.95 & 0.94 \\
28753 & 105475 & 2006 & 920.60 & 0.18 & 94110.00 & 898862.43 & 0.98 & 0.98 & 0.96 \\
12332 & 101536 & 2006 & 1294.60 & 0.11 & 131153.00 & 1273529.79 & 0.99 & 0.98 & 0.97 \\
28731 & 105472 & 2006 & 57.30 & 0.14 & 5476.00 & 55185.27 & 1.05 & 0.96 & 1.01 \\
5948 & 100812 & 2006 & 213.70 & 0.08 & 21517.00 & 205390.00 & 0.99 & 0.96 & 0.95 \\
28591 & 105448 & 2006 & 1872.90 & 0.36 & 183438.00 & 1607235.56 & 1.02 & 0.86 & 0.88 \\
28575 & 105444 & 2006 & 68.20 & 0.17 & 6345.00 & 61276.51 & 1.07 & 0.90 & 0.97 \\
21775 & 102951 & 2006 & 4606.50 & 0.13 & 450838.00 & 4769051.38 & 1.02 & 1.04 & 1.06 \\
9005 & 101108 & 2006 & 2774.80 & 0.15 & 291019.00 & 2412266.88 & 0.95 & 0.87 & 0.83 \\
18284 & 102424 & 2006 & 1044.60 & 0.43 & 83401.00 & 777130.68 & 1.25 & 0.74 & 0.93 \\
2077 & 100288 & 2006 & 11.00 & 0.13 & 1097.00 & 10488.46 & 1.00 & 0.95 & 0.96 \\
21860 & 102957 & 2006 & 324.80 & 0.07 & 32902.00 & 321665.28 & 0.99 & 0.99 & 0.98 \\
28436 & 105424 & 2006 & 3443.10 & 0.07 & 344339.00 & 3558654.94 & 1.00 & 1.03 & 1.03 \\
30833 & 105804 & 2006 & 905.60 & 0.12 & 89451.00 & 828421.14 & 1.01 & 0.91 & 0.93 \\
6511 & 100878 & 2006 & 2299.40 & 0.13 & 227443.00 & 2298513.21 & 1.01 & 1.00 & 1.01 \\
10236 & 101275 & 2006 & 724.30 & 0.04 & 75645.00 & 746423.67 & 0.96 & 1.03 & 0.99 \\
21915 & 102979 & 2006 & 87.80 & 0.20 & 8884.00 & 81192.25 & 0.99 & 0.92 & 0.91 \\
2105 & 100291 & 2006 & 1015.00 & 0.19 & 101149.00 & 1004212.48 & 1.00 & 0.99 & 0.99 \\
21963 & 102981 & 2006 & 247.00 & 0.04 & 24684.00 & 232750.17 & 1.00 & 0.94 & 0.94 \\
18233 & 102417 & 2006 & 730.70 & 0.05 & 73056.00 & 749531.85 & 1.00 & 1.03 & 1.03 \\
13923 & 101787 & 2006 & 150.20 & 0.11 & 21584.00 & 155498.27 & 0.70 & 1.04 & 0.72 \\
12165 & 101513 & 2006 & 912.60 & 0.16 & 89967.00 & 899669.19 & 1.01 & 0.99 & 1.00 \\
13780 & 101763 & 2006 & 3518.50 & 1.39 & 351992.00 & 3460786.14 & 1.00 & 0.98 & 0.98 \\
13890 & 101785 & 2006 & 1079.90 & 0.00 & 114756.00 & 979023.49 & 0.94 & 0.91 & 0.85 \\
12196 & 101518 & 2006 & 305.10 & 0.16 & 29138.00 & 312194.39 & 1.05 & 1.02 & 1.07 \\
28545 & 105437 & 2006 & 766.80 & 0.13 & 76804.00 & 757233.24 & 1.00 & 0.99 & 0.99 \\
28523 & 105432 & 2006 & 33.80 & 0.08 & 3857.00 & 37631.16 & 0.88 & 1.11 & 0.98 \\
30768 & 105794 & 2006 & 99.20 & 0.22 & 7283.00 & 61084.20 & 1.36 & 0.62 & 0.84 \\
10173 & 101264 & 2006 & 291.10 & 0.05 & 30915.00 & 321423.31 & 0.94 & 1.10 & 1.04 \\
21683 & 102939 & 2006 & 7006.80 & 0.15 & 643663.00 & 7010728.85 & 1.09 & 1.00 & 1.09 \\
18405 & 102447 & 2006 & 7679.90 & 0.13 & 771092.00 & 7206192.71 & 1.00 & 0.94 & 0.93 \\
9042 & 101109 & 2006 & 668.30 & -0.02 & 68326.00 & 636094.68 & 0.98 & 0.95 & 0.93 \\
28465 & 105426 & 2006 & 1187.70 & 0.20 & 120875.00 & 1219804.10 & 0.98 & 1.03 & 1.01 \\
12251 & 101530 & 2006 & 2019.00 & 0.28 & 177196.00 & 1590520.51 & 1.14 & 0.79 & 0.90 \\
10203 & 101268 & 2006 & 339.70 & 0.13 & 32732.00 & 330088.35 & 1.04 & 0.97 & 1.01 \\
10217 & 101274 & 2006 & 373.40 & 0.09 & 38336.00 & 388930.81 & 0.97 & 1.04 & 1.01 \\
13810 & 101764 & 2006 & 548.20 & 0.14 & 59317.00 & 603067.54 & 0.92 & 1.10 & 1.02 \\
30805 & 105803 & 2006 & 7777.30 & 0.13 & 791590.00 & 7160172.94 & 0.98 & 0.92 & 0.90 \\
28494 & 105427 & 2006 & 99.20 & 0.08 & 9796.00 & 97621.07 & 1.01 & 0.98 & 1.00 \\
21713 & 102940 & 2006 & 1681.40 & 0.17 & 165854.00 & 1676185.30 & 1.01 & 1.00 & 1.01 \\
18361 & 102446 & 2006 & 278.10 & 0.06 & 27652.00 & 273834.22 & 1.01 & 0.98 & 0.99 \\
18327 & 102425 & 2006 & 1278.00 & 0.04 & 130584.00 & 1275323.57 & 0.98 & 1.00 & 0.98 \\
2067 & 100287 & 2006 & 59.10 & -0.00 & 5929.00 & 59013.64 & 1.00 & 1.00 & 1.00 \\
1573 & 100214 & 2006 & 210.20 & 0.19 & 19637.00 & 208816.89 & 1.07 & 0.99 & 1.06 \\
4055 & 100543 & 2006 & 453.50 & 0.06 & 45722.00 & 474615.92 & 0.99 & 1.05 & 1.04 \\
30658 & 105781 & 2006 & 864.10 & 0.15 & 81752.00 & 845201.56 & 1.06 & 0.98 & 1.03 \\
20954 & 102813 & 2006 & 1553.70 & 0.19 & 153784.00 & 1465739.31 & 1.01 & 0.94 & 0.95 \\
30475 & 105761 & 2006 & 1587.30 & 0.18 & 168647.00 & 1503791.89 & 0.94 & 0.95 & 0.89 \\
1690 & 100223 & 2006 & 2239.40 & 0.16 & 224012.00 & 2191747.26 & 1.00 & 0.98 & 0.98 \\
1709 & 100226 & 2006 & 19577.00 & 0.16 & 1892460.00 & 16236497.94 & 1.03 & 0.83 & 0.86 \\
6028 & 100820 & 2006 & 604.10 & 0.05 & 60464.00 & 616208.82 & 1.00 & 1.02 & 1.02 \\
13623 & 101748 & 2006 & 2310.00 & 0.14 & 230841.00 & 2282777.47 & 1.00 & 0.99 & 0.99 \\
236 & 100019 & 2006 & 18970.20 & 0.14 & 1901083.00 & 16075550.66 & 1.00 & 0.85 & 0.85 \\
29009 & 105512 & 2006 & 14.50 & 0.07 & 1404.00 & 13060.06 & 1.03 & 0.90 & 0.93 \\
29044 & 105522 & 2006 & 444.30 & 0.13 & 51330.00 & 448853.71 & 0.87 & 1.01 & 0.87 \\
30503 & 105762 & 2006 & 490.50 & 0.08 & 56532.00 & 475304.19 & 0.87 & 0.97 & 0.84 \\
20984 & 102814 & 2006 & 222.90 & 0.21 & 21918.00 & 221164.39 & 1.02 & 0.99 & 1.01 \\
9329 & 101131 & 2006 & 1918.40 & 0.16 & 194981.00 & 1803843.17 & 0.98 & 0.94 & 0.93 \\
5989 & 100817 & 2006 & 78.50 & 0.15 & 7839.00 & 80970.46 & 1.00 & 1.03 & 1.03 \\
18965 & 102531 & 2006 & 31.50 & 0.09 & 3204.00 & 32827.92 & 0.98 & 1.04 & 1.02 \\
18933 & 102528 & 2006 & 64.70 & 0.02 & 7515.00 & 71619.07 & 0.86 & 1.11 & 0.95 \\
18917 & 102527 & 2006 & 70.20 & 0.09 & 7849.00 & 77528.99 & 0.89 & 1.10 & 0.99 \\
28965 & 105508 & 2006 & 50.80 & 0.07 & 3797.00 & 37968.98 & 1.34 & 0.75 & 1.00 \\
12450 & 101541 & 2006 & 625.40 & 0.11 & 61198.00 & 611980.62 & 1.02 & 0.98 & 1.00 \\
10083 & 101258 & 2006 & 4669.90 & 0.17 & 442063.00 & 4262049.06 & 1.06 & 0.91 & 0.96 \\
13591 & 101744 & 2006 & 1466.30 & 0.24 & 174929.00 & 1389138.12 & 0.84 & 0.95 & 0.79 \\
9419 & 101134 & 2006 & 253.90 & 0.28 & 23046.00 & 202078.00 & 1.10 & 0.80 & 0.88 \\
29134 & 105531 & 2006 & 168.20 & 0.21 & 16400.00 & 167374.16 & 1.03 & 1.00 & 1.02 \\
1602 & 100217 & 2006 & 32.00 & 0.10 & 4570.00 & 31249.00 & 0.70 & 0.98 & 0.68 \\
29111 & 105527 & 2006 & 12.40 & 0.18 & 1181.00 & 10391.70 & 1.05 & 0.84 & 0.88 \\
12577 & 101554 & 2006 & 358.80 & 0.19 & 35835.00 & 348738.74 & 1.00 & 0.97 & 0.97 \\
30401 & 105753 & 2006 & 475.70 & 0.15 & 50846.00 & 457620.83 & 0.94 & 0.96 & 0.90 \\
20813 & 102795 & 2006 & 3746.50 & 0.17 & 346110.00 & 3442467.20 & 1.08 & 0.92 & 0.99 \\
18997 & 102540 & 2006 & 491.40 & 0.14 & 49391.00 & 459692.28 & 0.99 & 0.94 & 0.93 \\
19029 & 102544 & 2006 & 1581.50 & 0.06 & 152260.00 & 1522601.38 & 1.04 & 0.96 & 1.00 \\
296 & 100033 & 2006 & 146.90 & 0.12 & 19096.00 & 141161.82 & 0.77 & 0.96 & 0.74 \\
30447 & 105760 & 2006 & 1211.00 & 0.09 & 131082.00 & 1162542.82 & 0.92 & 0.96 & 0.89 \\
29083 & 105525 & 2006 & 6003.10 & 0.14 & 568395.00 & 5382040.07 & 1.06 & 0.90 & 0.95 \\
9393 & 101133 & 2006 & 694.50 & 0.16 & 66322.00 & 681732.26 & 1.05 & 0.98 & 1.03 \\
20863 & 102797 & 2006 & 68.40 & 0.14 & 6635.00 & 66902.57 & 1.03 & 0.98 & 1.01 \\
20891 & 102798 & 2006 & 84.30 & 0.07 & 8320.00 & 82517.31 & 1.01 & 0.98 & 0.99 \\
779 & 100096 & 2006 & 100.20 & 0.04 & 11161.00 & 108862.74 & 0.90 & 1.09 & 0.98 \\
12524 & 101545 & 2006 & 390.00 & 0.17 & 37669.00 & 376683.59 & 1.04 & 0.97 & 1.00 \\
12553 & 101553 & 2006 & 136.90 & 0.05 & 13677.00 & 131592.37 & 1.00 & 0.96 & 0.96 \\
28813 & 105478 & 2006 & 51.20 & 0.10 & 5091.00 & 52365.06 & 1.01 & 1.02 & 1.03 \\
1753 & 100227 & 2006 & 108.60 & 0.14 & 10861.00 & 108380.46 & 1.00 & 1.00 & 1.00 \\
21072 & 102827 & 2006 & 365.60 & 0.21 & 34990.00 & 363143.17 & 1.04 & 0.99 & 1.04 \\
28888 & 105502 & 2006 & 1332.80 & 0.14 & 127487.00 & 1165396.17 & 1.05 & 0.87 & 0.91 \\
13679 & 101757 & 2006 & 1496.90 & 0.17 & 152127.00 & 1490757.07 & 0.98 & 1.00 & 0.98 \\
1893 & 100247 & 2006 & 1281.00 & 0.11 & 127244.00 & 1146372.42 & 1.01 & 0.89 & 0.90 \\
21259 & 102843 & 2006 & 3927.70 & 0.15 & 357981.00 & 3745497.92 & 1.10 & 0.95 & 1.05 \\
18824 & 102523 & 2006 & 4030.90 & 0.14 & 368491.00 & 3671126.11 & 1.09 & 0.91 & 1.00 \\
18781 & 102508 & 2006 & 39.40 & -0.02 & 4220.00 & 43398.98 & 0.93 & 1.10 & 1.03 \\
12396 & 101538 & 2006 & 225.30 & 0.05 & 22555.00 & 222143.78 & 1.00 & 0.99 & 0.98 \\
18763 & 102507 & 2006 & 114.20 & 0.01 & 12054.00 & 117737.48 & 0.95 & 1.03 & 0.98 \\
30598 & 105775 & 2006 & 901.10 & 0.17 & 100563.00 & 899379.98 & 0.90 & 1.00 & 0.89 \\
9189 & 101116 & 2006 & 1599.40 & 0.31 & 181066.00 & 1627823.40 & 0.88 & 1.02 & 0.90 \\
10104 & 101261 & 2006 & 18.40 & 0.11 & 1798.00 & 19121.10 & 1.02 & 1.04 & 1.06 \\
30622 & 105779 & 2006 & 825.50 & 0.11 & 83898.00 & 804697.26 & 0.98 & 0.97 & 0.96 \\
839 & 100098 & 2006 & 1423.30 & 0.09 & 140188.00 & 1397106.26 & 1.02 & 0.98 & 1.00 \\
6385 & 100856 & 2006 & 139.40 & 0.20 & 16317.00 & 132940.83 & 0.85 & 0.95 & 0.81 \\
21336 & 102852 & 2006 & 743.10 & 0.09 & 76151.00 & 778765.47 & 0.98 & 1.05 & 1.02 \\
13709 & 101758 & 2006 & 132.20 & 0.16 & 13211.00 & 129082.90 & 1.00 & 0.98 & 0.98 \\
12365 & 101537 & 2006 & 284.80 & 0.20 & 28555.00 & 268405.04 & 1.00 & 0.94 & 0.94 \\
18728 & 102504 & 2006 & 808.10 & 0.15 & 97610.00 & 825069.01 & 0.83 & 1.02 & 0.85 \\
21283 & 102844 & 2006 & 664.50 & 0.12 & 64001.00 & 661618.56 & 1.04 & 1.00 & 1.03 \\
30531 & 105763 & 2006 & 506.50 & 0.10 & 62535.00 & 471957.57 & 0.81 & 0.93 & 0.75 \\
12416 & 101539 & 2006 & 989.30 & 0.20 & 102102.00 & 997565.33 & 0.97 & 1.01 & 0.98 \\
21201 & 102837 & 2006 & 1020.90 & 0.18 & 92071.00 & 983676.35 & 1.11 & 0.96 & 1.07 \\
1773 & 100228 & 2006 & 111.80 & 0.17 & 11117.00 & 110814.02 & 1.01 & 0.99 & 1.00 \\
809 & 100097 & 2006 & 199.00 & 0.18 & 17326.00 & 166742.36 & 1.15 & 0.84 & 0.96 \\
1792 & 100237 & 2006 & 13.80 & 0.13 & 1381.00 & 13882.59 & 1.00 & 1.01 & 1.01 \\
9286 & 101127 & 2006 & 184.00 & 0.23 & 14999.00 & 150049.79 & 1.23 & 0.82 & 1.00 \\
13660 & 101754 & 2006 & 230.00 & 0.11 & 32103.00 & 224982.61 & 0.72 & 0.98 & 0.70 \\
18886 & 102525 & 2006 & 2285.40 & 0.10 & 262370.00 & 2293849.65 & 0.87 & 1.00 & 0.87 \\
1864 & 100245 & 2006 & 1176.60 & 0.04 & 119353.00 & 1167738.36 & 0.99 & 0.99 & 0.98 \\
28946 & 105507 & 2006 & 1933.10 & 0.15 & 168254.00 & 1803171.95 & 1.15 & 0.93 & 1.07 \\
21169 & 102835 & 2006 & 1089.20 & 0.17 & 97068.00 & 821639.13 & 1.12 & 0.75 & 0.85 \\
1830 & 100244 & 2006 & 453.50 & 0.05 & 45582.00 & 449066.20 & 0.99 & 0.99 & 0.99 \\
6337 & 100849 & 2006 & 44.90 & 0.16 & 4159.00 & 43688.82 & 1.08 & 0.97 & 1.05 \\
28930 & 105506 & 2006 & 379.80 & 0.16 & 37946.00 & 376200.97 & 1.00 & 0.99 & 0.99 \\
18855 & 102524 & 2006 & 3729.60 & 0.15 & 380183.00 & 3659736.50 & 0.98 & 0.98 & 0.96 \\
9224 & 101119 & 2006 & 485.60 & 0.17 & 48110.00 & 466064.31 & 1.01 & 0.96 & 0.97 \\
30569 & 105770 & 2006 & 50.20 & 0.13 & 4806.00 & 50347.06 & 1.04 & 1.00 & 1.05 \\
3112 & 100409 & 2006 & 2130.90 & 0.11 & 211715.00 & 2035048.93 & 1.01 & 0.96 & 0.96 \\
14234 & 101834 & 2006 & 148.70 & 0.07 & 15460.00 & 157175.66 & 0.96 & 1.06 & 1.02 \\
26527 & 103590 & 2006 & 1254.90 & 0.08 & 123413.00 & 1268171.87 & 1.02 & 1.01 & 1.03 \\
4814 & 100682 & 2006 & 86.10 & 0.20 & 8478.00 & 83505.20 & 1.02 & 0.97 & 0.98 \\
15920 & 102059 & 2006 & 962.60 & 0.10 & 96085.00 & 927295.16 & 1.00 & 0.96 & 0.97 \\
24999 & 103406 & 2006 & 2021.20 & 0.12 & 196142.00 & 2045235.49 & 1.03 & 1.01 & 1.04 \\
24947 & 103395 & 2006 & 111.40 & 0.11 & 11370.00 & 116003.84 & 0.98 & 1.04 & 1.02 \\
14959 & 101925 & 2006 & 10430.10 & 0.12 & 994029.00 & 10353499.47 & 1.05 & 0.99 & 1.04 \\
11062 & 101364 & 2006 & 906.30 & 0.19 & 82458.00 & 750801.54 & 1.10 & 0.83 & 0.91 \\
26559 & 103591 & 2006 & 1339.70 & 0.07 & 138135.00 & 1214706.09 & 0.97 & 0.91 & 0.88 \\
32479 & 106033 & 2006 & 3968.50 & 0.20 & 369953.00 & 3712131.52 & 1.07 & 0.94 & 1.00 \\
15950 & 102061 & 2006 & 1753.40 & 0.19 & 163671.00 & 1739918.37 & 1.07 & 0.99 & 1.06 \\
24909 & 103394 & 2006 & 648.40 & 0.12 & 61876.00 & 497060.97 & 1.05 & 0.77 & 0.80 \\
7671 & 101054 & 2006 & 7679.00 & 0.14 & 767162.00 & 7101139.07 & 1.00 & 0.92 & 0.93 \\
24888 & 103383 & 2006 & 2362.00 & 0.09 & 231713.00 & 2235693.56 & 1.02 & 0.95 & 0.96 \\
3760 & 100480 & 2006 & 114.40 & 0.15 & 11385.00 & 111101.30 & 1.00 & 0.97 & 0.98 \\
32451 & 106028 & 2006 & 922.40 & 0.13 & 92716.00 & 833732.40 & 0.99 & 0.90 & 0.90 \\
4768 & 100671 & 2006 & 2370.70 & 0.15 & 216880.00 & 2106968.74 & 1.09 & 0.89 & 0.97 \\
15988 & 102062 & 2006 & 1867.20 & 0.16 & 171890.00 & 1744637.88 & 1.09 & 0.93 & 1.01 \\
24847 & 103381 & 2006 & 14297.60 & -0.04 & 1547938.00 & 14597484.81 & 0.92 & 1.02 & 0.94 \\
14918 & 101922 & 2006 & 3192.00 & 0.13 & 300805.00 & 3062756.57 & 1.06 & 0.96 & 1.02 \\
16032 & 102073 & 2006 & 8289.70 & 0.22 & 888125.00 & 8242612.09 & 0.93 & 0.99 & 0.93 \\
26590 & 103592 & 2006 & 305.30 & 0.07 & 30540.00 & 297078.83 & 1.00 & 0.97 & 0.97 \\
4746 & 100670 & 2006 & 69.60 & 0.17 & 6854.00 & 67849.87 & 1.02 & 0.97 & 0.99 \\
16083 & 102079 & 2006 & 358.30 & 0.21 & 38092.00 & 358751.88 & 0.94 & 1.00 & 0.94 \\
24807 & 103380 & 2006 & 2713.80 & -0.02 & 296692.00 & 2653348.35 & 0.91 & 0.98 & 0.89 \\
15879 & 102050 & 2006 & 358.20 & 0.20 & 39323.00 & 366367.34 & 0.91 & 1.02 & 0.93 \\
25042 & 103426 & 2006 & 1363.20 & 0.08 & 131691.00 & 1334564.88 & 1.04 & 0.98 & 1.01 \\
7709 & 101055 & 2006 & 30069.70 & 0.10 & 3078991.00 & 28999657.13 & 0.98 & 0.96 & 0.94 \\
10709 & 101312 & 2006 & 7126.00 & 0.07 & 708666.00 & 6738805.22 & 1.01 & 0.95 & 0.95 \\
8104 & 101074 & 2006 & 1134.50 & 0.05 & 130867.00 & 1134648.37 & 0.87 & 1.00 & 0.87 \\
3263 & 100419 & 2006 & 347.40 & 0.16 & 34705.00 & 342710.20 & 1.00 & 0.99 & 0.99 \\
5684 & 100785 & 2006 & 1375.50 & 0.13 & 138097.00 & 1368418.32 & 1.00 & 0.99 & 0.99 \\
25205 & 103460 & 2006 & 1182.40 & 0.05 & 127643.00 & 1108685.78 & 0.93 & 0.94 & 0.87 \\
3247 & 100417 & 2006 & 13.30 & 0.05 & 1324.00 & 13155.89 & 1.00 & 0.99 & 0.99 \\
7743 & 101056 & 2006 & 30213.80 & 0.11 & 3017465.00 & 27815173.17 & 1.00 & 0.92 & 0.92 \\
25153 & 103439 & 2006 & 122.90 & 0.21 & 12298.00 & 116226.71 & 1.00 & 0.95 & 0.95 \\
26439 & 103580 & 2006 & 261.40 & 0.14 & 25235.00 & 248154.92 & 1.04 & 0.95 & 0.98 \\
32551 & 106039 & 2006 & 600.90 & 0.18 & 59658.00 & 628842.34 & 1.01 & 1.05 & 1.05 \\
32423 & 106025 & 2006 & 46.30 & 0.15 & 5420.00 & 42699.61 & 0.85 & 0.92 & 0.79 \\
... & ... & ... & ... & ... & ... & ... & ... & ... & ... \\
\bottomrule
\end{tabular}


% \end{table}

\section{Replicating Table 2}

Then, we moved on to replicating table 2. We began by using the same merged CRSP and S12 data 
that we computed in table 1. We found much more success in replicating this table; the main 
challenge of this project was re-creating the original dataset. To obtain the Fama French 
Factor returns, we pulled factor returns \textbf{df\_ff} from 
\href{http://mba.tuck.dartmouth.edu/pages/faculty/ken.french/index.html}{Kenneth 
R. French's Website}. We then merged the CRSP dataset with a Fama-French dataset based on
dates, calculated investment flow for each unique 'wficn' identifier as the 
percentage change in total net assets adjusted for returns, and merged this flow
information back into the original dataset.

\end{document}